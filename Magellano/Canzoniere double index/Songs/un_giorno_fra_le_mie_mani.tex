%titolo{Un giorno fra le mie mani}
%autore{Gen Rosso}
%album{Se siamo uniti}
%tonalita{Re}
%famiglia{Liturgica}
%gruppo{}
%momenti{}
%identificatore{un_giorno_fra_le_mie_mani}
%data_revisione{2014_10_01}
%trascrittore{Francesco Endrici}
\beginsong{Un giorno fra le mie mani}[by={Gen\ Rosso}]
\newchords{giorno}
\beginverse
Un \[D]giorno fra le mie \[A]mani,
un \[D]giorno qui davanti a \[F#-7]me,
che cosa \[G]mai fa\[D]rò perché alla \[E-7]fine \[D]tu
ne sia fe\[A4]lice? \[A]
Oh! come vor\[D]rei in ogni mo\[A]mento
strap\[D]pare questa oscuri\[F#-7]tà
che scende e \[G]non mi \[D]fa
guardare \[E-7]al di \[D]là dei passi \[A4]miei. \[A]
\endverse
\beginchorus\memorize[giorno]
Come vor{\[B-7]rei} a\[E-7]marti
in \[F#-7]chi cammina accanto a \[G]me,
in \[E-7]chi incrocia la mia \[A]vita,
in \[F#-7]chi mi sfiora ma non \[G]sa
che tu sei \[D]lì con \[E-7]lui.
È \[D]quello che più vor\[G]rei,
È \[E7]quello che più vor\[D]rei per \[A4]te.
\endchorus
\beginverse
%\chordsoff
La ^strada piena di ^gente,
ma l'^orizzonte è tutto ^lì,
la folla ^se ne ^va tra un ne^gozio e un ^bar \brk indiffe^rente. ^
Oh! come vor^rei parlare ad o^gnuno,
co^sì come faresti ^tu,
della fe^lici^tà, di quella ^pace ^che \brk tu solo ^dai. ^
\endverse
\beginchorus\replay[giorno]
%\chordsoff
Così vor^rò a^marti
negli ^ultimi della cit^tà,
nel ^buio di chi muore ^solo,
in ^chi dispera e non ^sa
che tu sei ^lì con ^lui.
Co^sì oggi ti ame^rò,
co^sì oggi ti ame^rò di ^più.
\endchorus
\endsong