%titolo{Fratelli d'Italia}
%autore{Goffredo Mameli, Michele Novaro}
%album{}
%tonalita{Sol}
%famiglia{Altre}
%gruppo{}
%momenti{}
%identificatore{fratelli_d_italia}
%data_revisione{2012_12_14}
%trascrittore{Antonio Badan}
\beginsong{Fratelli d'Italia}[by={Mameli, Novaro}]
\beginverse*
Fra\[G]telli d'Italia,
l'Italia \[D]s'è \[G]desta;
dell'\[G]elmo di Scipio
s'è cinta \[D]la \[G]testa.
Do\[B7]v'è la Vit\[E-]toria?
Le \[B7]porga la \[E-]chio\[D]ma;
che \[G]schiava di Roma
Iddio la \[D7]creò\[G]. 
\endverse
\beginverse
\[C]Fratelli d'Italia,
l'Italia s'è desta;
dell'elmo di Scipio
s'è cinta la \[G7]testa.
Dov'è la Vittoria?
Le porga la chioma;
che schiava di Roma
Iddio la cre\[C]ò.
\endverse
\beginchorus
\[A-]Stringiamci a coorte!
Siam pronti alla \[E7]morte;
siam \[A-]pronti alla morte;
l'Italia chiam\[E]ò.
Stringiamci a c\[C]oorte!
Siam pronti alla \[F]morte;
siam pronti alla \[C]morte;
L'I\[G7]talia chiam\[C]ò. (Si!)
\endchorus
\chordsoff
\beginverse
Noi siamo da secoli 
calpesti, derisi, 
perché non siam popolo, 
perché siam divisi. 
Raccolgaci un'unica 
Bandiera, una speme: 
di fonderci insieme 
già l'ora suonò. 
\endverse
\beginverse
Uniamoci, amiamoci, 
l'Unione, e l'amore 
rivelano ai Popoli 
le vie del Signore; 
giuriamo far libero 
il suolo natìo: 
uniti per Dio 
chi vincer ci può? 
\endverse
\beginverse
Dall'Alpi a Sicilia 
dovunque è Legnano, 
ogn'uom di Ferruccio 
ha il core, ha la mano. 
I bimbi d'Italia 
si chiaman Balilla, 
il suon d'ogni squilla 
i Vespri suonò. 
\endverse
\beginverse
Son giunchi che piegano 
le spade vendute: 
già l'Aquila d'Austria 
le penne ha perdute. 
Il sangue d'Italia, 
il sangue Polacco, 
bevé, col cosacco, 
ma il cor le bruciò. 
\endverse
\endsong
