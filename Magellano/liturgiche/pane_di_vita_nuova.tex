%titolo{Pane di vita nuova}
%autore{Frisina}
%album{Pane di vita nuova}
%tonalita{Re}
%famiglia{Liturgica}
%gruppo{}
%momenti{Comunione}
%identificatore{pane_di_vita_nuova}
%data_revisione{2014_09_30}
%trascrittore{Francesco Endrici}
\beginsong{Pane di vita nuova}[by={Frisina}]
\beginverse
\[D]Pane \[G]di vita \[A]nuo\[D]va, 
\[G]vero \[D]cibo dato agli \[E-]uomi\[A]ni,
\[G]nutri\[D]mento \[E-]che sostiene il \[A]mondo, 
\[B-]do\[G]no  \[D]splendi\[E-]do  di \[A]gra\[D]zia.
\endverse
\beginverse
%\chordsoff
^Tu sei ^sublime ^frut^to 
^di quell'^albero di ^vi^ta
^che A^damo ^non potè toc^care:
^Ora ^è in  ^Cristo a ^noi do^na^to.
\endverse
\beginchorus
\[G]Pane \[D]della \[G]vi\[A]ta, 
\[D]sangue \[F#-]di sal\[E-]vez\[A]za,
\[G]vero \[D]corpo, \[E-]vera be\[B-]vanda,
\[E-]cibo di \[D]grazia per il \[A]mon\[D]do.
\endchorus
\beginverse
%\chordsoff
^Sei l'A^gnello immo^la^to
^nel cui ^Sangue è la sal^vez^za,
^memo^riale ^della vera ^Pasqua
^del^la ^nuova ^Alle^an^za.
\endverse
\beginverse
%\chordsoff
^Manna ^che nel de^ser^to
^nutri il ^popolo in cam^mi^no,
^sei so^stegno e ^forza nella ^prova
^per ^la ^Chiesa in ^mezzo al ^mon^do.
\endverse
\beginchorus
%\chordsoff 
Rit. 
\endchorus
\beginverse
%\chordsoff
^Vino ^che ci dà ^gio^ia,
^che ri^scalda il nostro ^cuo^re,
^sei per ^noi ^il prezioso ^frutto
^del^la ^vigna ^del Si^gno^re.
\endverse
\beginverse
%\chordsoff
^Dalla ^vite ai ^tral^ci
^scorre ^la vitale ^lin^fa
^che ci ^dona ^la vita di^vina,
^scor^re il ^sangue ^dell'a^mo^re.
\endverse
\beginchorus
%\chordsoff 
Rit. 
\endchorus
\beginverse
%\chordsoff
^Al ban^chetto ci in^vi^ti
^che per ^noi hai prepa^ra^to,
^doni all'^uomo ^la tua Sa^pienza,
^do^ni il ^Verbo ^della ^vi^ta.
\endverse
\beginverse
%\chordsoff
^Segno ^d'amore e^ter^no
^pegno ^di sublimi ^noz^ze,
^comu^nione ^nell'unico ^corpo
^che ^in ^Cristo ^noi for^mia^mo.
\endverse
\beginchorus
%\chordsoff 
Rit. 
\endchorus
\beginverse
%\chordsoff
^Nel tuo ^Sangue è la ^vi^ta
^ed il ^fuoco dello ^Spiri^to,
^la sua ^fiamma in^cendia il nostro ^cuore 
^e ^pu^rifi^ca il ^mon^do.
\endverse
\beginverse
%\chordsoff
^Nel pro^digio dei ^pa^ni
^tu sfa^masti ogni ^uo^mo,
^nel tuo a^more il ^povero è nu^trito
^e ^ri^ceve ^la tua ^vi^ta.
\endverse
\beginchorus
%\chordsoff 
Rit. 
\endchorus
\beginverse
%\chordsoff
Sacerdote eterno
Tu sei vittima ed altare,
offri al Padre tutto l'universo,
sacrificio dell'amore.
\endverse
\beginverse
%\chordsoff
Il tuo Corpo è tempio
della lode della Chiesa,
dal costato tu l'hai generata,
nel tuo Sangue l'hai redenta.
\endverse
\beginchorus
%\chordsoff 
Rit. 
\endchorus
\beginverse
%\chordsoff
Vero Corpo di Cristo
tratto da Maria Vergine,
dal tuo fianco doni a noi la grazia, 
per mandarci tra le genti.
\endverse
\beginverse
%\chordsoff
Dai confini del mondo,
da ogni tempo e ogni luogo
il creato a te renda grazie,
per l'eternità ti adori.
\endverse
\beginchorus
%\chordsoff 
Rit. 
\endchorus
\beginverse
%\chordsoff
A te Padre la lode,
che donasti il Redentore,
e al Santo Spirito di vita 
sia per sempre onore e gloria. 
\endverse
\beginverse
\[D]\[A]A\[D]men.
\endverse
\endsong