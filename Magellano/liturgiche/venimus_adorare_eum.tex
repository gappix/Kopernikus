%titolo{Venimus adorare eum}
%autore{Linssen}
%album{Inno della XX GMG, Colonia 2005}
%tonalita{Re}
%famiglia{Liturgica}
%gruppo{}
%momenti{}
%identificatore{venimus_adorare_eum}
%data_revisione{2011_12_31}
%trascrittore{Francesco Endrici - Manuel Toniato}
\beginsong{Venimus adorare eum}[by={Linssen}]
\ifchorded
\beginverse*
\vspace*{-0.8\versesep}
{\nolyrics \[D] \[G] \[A4] \[A] }
\vspace*{-\versesep}
\endverse
\fi

\beginverse
\[D]Chiedi perché par\[G]tire \[E-]dal proprio \[A]regno
\[B-]solo per inse\[G]guire \[E-]una stella e per\[A]ché
\[G]per un Bimbo pi\[E7]egano quelle gi\[D]nocchia da \[B-]Re?
\[E-] Tu la risposta  sai che \[A]è:
\endverse

\beginchorus
Ve\[D]nimus adorare Eum, Em\[C]manuel \[G]Dio con noi
\[D]Venimus adorare \[C]Eum, Em\[G]manuel \rep{2}
\endchorus

\chordsoff
\beginverse
Chiedi perché lasciare sui monti il gregge
solo per ascoltare un canto e perché
per un Bimbo piegano quelle ginocchia, perché?
Tu la risposta sai che è:
\endverse

\beginverse
Ecco da lontano per adorarlo \brk siamo giunti anche noi,
noi, tutti figli suoi, profeti e sacerdoti ormai.
Nel pane e nel vino noi siamo in lui e lui è in noi:
e un canto qui si alza già.
\endverse
\endsong





