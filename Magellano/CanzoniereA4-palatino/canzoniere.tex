% !TEX encoding = UTF-8 Unicode
\documentclass[standard,Palatino, authorsindex, titleindex, chorded, cover]{canzoniereonline}%
%opzioni formato: singoli, standard (A4), a5vert, a5oriz, a6vert;
%opzioni accordi: lyric, chorded {quelli d Songs}
%opzioni font: palatino, libertine
%opzioni segno minore: "minorsign=quel che vuoi"
%opzioni indici: authorsindex, titleindex, tematicindex
%opzioi copertina: cover e nocover
\usepackage[italian]{babel}
\usepackage{hyperref}
%righe per la copertina
\renewcommand{\titolettocop}{Gino}
\renewcommand{\titolocop}{Trollolobrigido}
\renewcommand{\sottotitolocop}{Anchesì detto Guzzo}
\renewcommand{\piede}{\today}
\def\canzsongcolumsnumber{2}
%%%%%%%%%%%%%%%%%%%%%%%%%%%%%%%%%%%%%%%%%%%%%%%%%%%%%%%%%%%%
\begin{document}
\ifcover
	\firstpage
	\colophon
\else
	\relax
\fi
\begin{songs}{}
\songcolumns{\canzsongcolumsnumber}

%titolo{A Betlemme di Giudea}
%autore{Tradizionale Francese}
%album{}
%tonalita{Re}
%famiglia{Liturgica}
%gruppo{}
%momenti{Natale}
%identificatore{a_betlemme_di_giudea}
%data_revisione{2011_12_31}
%trascrittore{Francesco Endrici}
%video{https://www.youtube.com/watch?v=ynVbACsr0Hs;https://www.youtube.com/watch?v=AuEPfJaFGb8}
\beginsong{A Betlemme di Giudea}[by={}]
\beginverse
\[D]A Betlemme \[A]di Giu\[D]{dea} \[D]una gran luce \[A7]si le\[D]vò:
\[B-]nella notte, \[A]sui pa\[D]stori, \[B-]scese l'annuncio e \[A]si can\[D]tò.
\endverse



\beginchorus
\[D]\[B-]\[E-]\[A7]\[D]\[G]Glo\[A]{ria} \[D]in ex\[G]celsis \[D]{De}\[A]o
\[D]\[B-]\[E-]\[A7]\[D]\[G]Glo\[A]{ria} \[D]in ex\[G]celsis \[D]\[A]De\[D]o
\endchorus

\beginverse
\chordsoff
Cristo nasce sulla paglia, \brk figlio del Padre Dio con noi,
Verbo eterno, Re di pace, \brk pone la tenda in mezzo ai suoi.
\endverse

\beginverse
\chordsoff
Tornerà nella sua gloria, \brk quando quel giorno arriverà:
se lo accogli nel tuo cuore \brk tutto il suo Regno ti darà.
\endverse
\endsong

%titolo{Abbà misericordia}
%autore{de Luca, Aguila}
%album{Io scelgo te}
%tonalita{Do}
%famiglia{Liturgica}
%gruppo{}
%momenti{Ingresso;Quaresima;Conversione;Riconciliazione}
%identificatore{abba_misericordia}
%data_revisione{2011_12_31}
%trascrittore{Francesco Endrici}
\beginsong{Abbà misericordia}[by={De\ Luca, Aguila}]
\ifchorded
\beginverse*
\vspace*{-0.8\versesep}
{\nolyrics \[C] \[F/C] \[G/C] \[F/C] \[C]}
\vspace*{-\versesep}
\endverse
\fi%

\beginverse
\[C]Non sono degno di \[F/C]essere qui,
\[G/C]ho abbandonato \[C]la tua ca\[G/B]sa,
\[A-]ho dissipato i tuoi \[D-]beni,
\[G]Padre ho peccato contro il \[E7/G#]cielo e contro di \[A-]te. \[G4] \[G] 
\endverse

\beginchorus
Ab\[C]bà, \[F]miseri\[G]cordia Ab\[C]bà,
\[F]miseri\[E]cordia Ab\[A-]bà, \[D-7]{Ab}\[G4]bà. \[G] 
Ab\[C]bà, \[F]miseri\[G]cordia Ab\[C]bà,
\[F]miseri\[E]cordia Ab\[A-]bà, \[D-7]{Ab}\[G4]bà,  \[G]{Ab}\[C]bà
\endchorus

\beginverse
\chordsoff
Non sono degno di esser tuo figlio,
in Gesù Cristo rialzami
e in lui ridammi la vita,
Padre ho peccato contro il cielo e contro di te.
\endverse

\beginverse
\chordsoff
Non sono degno del tuo amore,
riempi il mio cuore del tuo Spirito,
insieme a te farò festa per sempre,
Padre ho peccato contro il cielo e contro di te.
\endverse

\beginchorus
Ab\[C]bà, \[F]miseri\[G]cordia Ab\[C]bà,
\[F]miseri\[E]cordia Ab\[A-]bà, \[D-7]Ab\[G4]bà. \[G]
Ab\[C]bà, \[F]miseri\[G]cordia Ab\[C]bà,
\[F]miseri\[E]cordia Ab\[A-]bà, \[D-7]Ab\[G4]bà,\[A7] 
Ab\[D]bà, \[G]miseri\[A]cordia Ab\[D]bà,
\[G]miseri\[F#]cordia Ab\[B-]{bà,} \[E-7]Ab\[A4]bà \[A] 
Ab\[D]bà, \[G]miseri\[A]cordia Ab\[D]bà,
\[G]miseri\[F#]cordia Ab\[B-]bà, \[E-7]Ab\[A4]bà, \[A]{Ab}\[D]bà
\endchorus
\endsong

%titolo{Acclamate al Signore}
%autore{Frisina}
%album{Benedici il Signore}
%tonalita{Do}
%famiglia{Liturgica}
%gruppo{}
%momenti{Salmi}
%identificatore{acclamate_al_signore_frisina}
%data_revisione{2011_12_31}
%trascrittore{Francesco Endrici}
\beginsong{Acclamate al Signore}[by={Frisina}]

\beginchorus
\[C]Accla\[A-]mate al Si\[F]gno\[C]re, \[A-]voi \[D-]tutti della \[G]terra
\[A-]e ser\[E-]vitelo con \[F]gio\[C]ia, an\[A-]date a \[F]lui con lieti \[G]canti.
Accla\[F]mate voi \[E-]{tut}\[A-]{ti}  \[D-]al Si\[G]gno\[C]re.
\endchorus

\beginverse
\[C]Ricono\[A-]scete che il Si\[F]gno\[C]re,
\[A-]che il Si\[D-]gnore è \[G]Dio,
\[A-]Egli ci ha \[E-]fatti siamo \[F]suo\[C]i,
suo \[A-]popolo e \[F]gregge del suo \[G]pascolo.
\endverse

\beginverse
\chordsoff
Entrate nelle sue porte
con degli inni di grazia,
i suoi atri nella lode,
Benedite, lodate il suo nome.
\endverse

\beginverse
\chordsoff
Poiché buono è il Signore,
eterna la sua misericordia,
la sua fedeltà si estende
sopra ogni generazione.
\endverse
\endsong

%titolo{Acclamate a Dio}
%autore{Attinà, Zunino}
%album{Ad una voce}
%tonalita{Re}
%famiglia{Liturgica}
%gruppo{}
%momenti{Ingresso;Salmi}
%identificatore{acclamate_a_dio}
%data_revisione{2011_12_31}
%trascrittore{Francesco Endrici}
\beginsong{Acclamate a Dio}[by={Attinà, Zunino}]
%\category{Ingresso}
%\category{Salmi}

\ifchorded
\beginverse*
\vspace*{-0.8\versesep}
{\nolyrics \[D] \[B-] \[G] \[D] \[B-] \[G] \[F#-] \[G6]}
\vspace*{-\versesep}
\endverse
\fi%
\beginchorus
Acclamate a \[A]Dio \[D]da tutta la \[F#-]{ter}\[B-]ra,
Can\[E-7]tate alla gloria \[G/A]del suo nome,
Date a \[A]Lui \[D]splendida \[F#-]{lo}\[B-]de
Stu\[E-]pende \[D]sono le \[G]sue ope\[A]{re} \[D] \[F#-] \[B-] 
Stu\[E-]pende \[D]sono le \[G]sue ope\[A]{re} \[D] \[G-] \[D] \[F#] 
\endchorus

\beginverse
\[B-]Per la grandezza della \[F#-]sua potenza
Da\[G]vanti a Lui si pie\[A]gano i nemici.
A \[F#-]Dio si prostri la \[B-]terra
A Lui \[G]canti inni, \[A]canti al suo nome
\endverse

\beginverse
\chordsoff
Il mare ha cambiato in terra ferma
Con la sua forza regnerà in eterno
Dio salva la nostra vita
Per questo in Lui esultiamo di gioia
\endverse

\beginverse
\chordsoff
Venite voi tutti che temete Dio
E narrerò quanto per me ha fatto
a Lui ho rivolto il mio grido
La mia lingua cantò la sua lode
\endverse
\endsong

%titolo{Acqua siamo noi}
%autore{Cento}
%album{È il giorno del Signore}
%tonalita{Re}
%famiglia{Liturgica}
%gruppo{}
%momenti{Ingresso}
%identificatore{acqua_siamo_noi}
%data_revisione{2013_12_30}
%trascrittore{Francesco Endrici}
\beginsong{Acqua siamo noi}[by={Cento}]
\ifchorded
\beginverse*
\vspace*{-0.8\versesep}
{\nolyrics | \[D] | \[D] | \[D] | \[G]\[A4]\[A] |}
\vspace*{-\versesep}
\endverse
\fi
\beginverse
\memorize
|\[D]Acqua \[A]siamo \[D]noi, \brk dall'an|\[G6]tica sor\[A]gente ve\[D]niamo,
\[D]fiumi \[A]siamo \[D]noi \brk se i ru\[G6]scelli si \[A]mettono in\[D]sieme,
\[D]mari \[G]siamo \[D]noi \brk se i tor\[G6]renti si \[A]danno la \[D]mano,
\[D]vita nuova \[B-7]c'è \brk se Ge\[G]sù è in \[A]mezzo a \[D]noi.
\endverse
\beginchorus
E allora \[F#-7]diamoci la \[B-7]mano
e tutti in\[F#-7]sieme cammi\[G7+]niamo
ed un o\[F#-7]ceano di \[B-7]pace nasce\[A4]rà. \[A]
E l'ego\[E-]ismo cancel\[C7+]liamo
e un cuore \[E-]limpido sen\[C]tiamo
è Dio che \[C7+]bagna del suo a\[A7]mor l'umani\[D]tà. \[A4]\[A]
\endchorus
\beginverse
^Su nel ^cielo ^c'è \brk Dio ^Padre che ^vive per ^l'uomo
^crea ^tutti ^noi \brk e ci ^ama di a^more infi^nito,
^figli ^siamo ^noi \brk e fra^telli di ^Cristo Si^gnore,
^vita nuova ^c'è \brk quando ^Lui è in ^mezzo a ^noi.
\endverse
\beginverse
%\chordsoff
^Nuova u^mani^tà \brk oggi ^nasce da ^chi crede in ^Lui,
^nuovi ^siamo ^noi \brk se l'a^more è la ^legge di ^vita,
^figli ^siamo ^noi \brk se non ^siamo di^visi da ^niente,
^vita eterna ^c'è \brk quando ^Lui è ^dentro ^noi.
\endverse
\endsong
%titolo{Acqua sole e verità}
%autore{Cento}
%album{Celebraimo la nostra speranza}
%tonalita{Sol}
%famiglia{Liturgica}
%gruppo{}
%momenti{Comunione}
%identificatore{acqua_sole_verità}
%data_revisione{2011_12_31}
%trascrittore{Francesco Endrici}
\beginsong{Acqua sole e verità}[by={Cento}]
\ifchorded
\beginverse*
\vspace*{-0.8\versesep}
{\nolyrics \[G]\[B-]\[C]\[D7]\[G]}
\vspace*{-\versesep}
\endverse
\fi
\beginverse
\memorize
Ho be\[G]vuto a una fon\[B-]tana un'acqua \[C]chiara
che è ve\[D7]nuta giù dal \[G]cielo.
Ho so\[E-]gnato nella \[C]notte di tu\ch{D}{f}{f}{ff}armi
nella luce del \[G]sole.
Ho cer\[C]cato dentro a \[A-]me la veri\[D]tà.
\endverse
\beginchorus
Ed ho ca\[G]pito, mio Si\[B-]gnore
che sei \[C]tu la vera \[D]acqua, \[D7]
sei tu il mio \[E-]sole
sei \[C]tu la veri\[D]tà. \rep{2}
\endchorus
\beginverse
\chordsoff
Tu ti ^siedi sul mio ^pozzo nel de^serto
e mi ^chiedi un po' da ^bere.
Per il ^sole che ri^splende a mezzo^giorno ti ri^spondo.
Ma tu ^sai già dentro ^me la veri^tà.
\endverse
\beginverse
\chordsoff
Un ^cervo che cer^cava un sorso d'^acqua
nel giorno ^corse e ti tro^vò.
Anch'^io vò cer^cando nell'ar^sura sotto il ^sole,
e ^trovo dentro ^me la veri^tà.
\endverse
\endsong

%titolo{Adoriamo il Sacramento}
%autore{}
%album{}
%tonalita{Fa}
%famiglia{Liturgica}
%gruppo{}
%momenti{Adorazione}
%identificatore{adoriamo_il_sacramento}
%data_revisione{2011_12_31}
%trascrittore{Francesco Endrici}
\beginsong{Adoriamo il Sacramento}
\beginverse
\[F]Ado\[C]riamo il \[B&]Sacra\[F]mento
\[F]che Dio \[B&]padre \[C]ci do\[F]nò.
\[F]Nuovo \[F]patto, \[B&]nuovo \[F]rito,
\[F]nella \[F]fede \[B&]si com\[C]pì.
\[F]Al mi\[C]stero è \[B&]fonda\[F]mento
\[F]la Pa\[B&]rola \[C]di Ge\[F]sù.
\endverse
\beginverse
^Gloria al ^Padre on^nipo^tente
^gloria al ^Figlio ^Reden^tor,
^lode ^grande, ^sommo o^nore
^all'e^terna ^Cari^tà.
^Gloria im^mensa, e^terno a^more
^alla ^Santa ^Trini^tà.
\endverse
\endsong



%\input{canzoni-coro}

\end{songs}
\ifcanzsingole
	\relax
\else
	\iftitleindex
		\ifxetex
		\printindex[alfabetico]
		\else
		\printindex
		\fi
	\else
	\fi
	\ifauthorsindex
	\printindex[autori]
	\else
	\fi
	\iftematicindex
	\printindex[tematico]
	\else
	\fi
	\ifcover
		\relax
	\else
		\colophon
	\fi
\fi
\end{document}
