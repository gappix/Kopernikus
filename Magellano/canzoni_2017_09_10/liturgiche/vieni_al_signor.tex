%titolo{Vieni al Signor}
%autore{Calisi}
%album{Benedici il Signore}
%tonalita{Re-}
%famiglia{Liturgica}
%gruppo{}
%momenti{Salmi;Quaresima}
%identificatore{vieni_al_signor}
%data_revisione{2011_12_31}
%trascrittore{Francesco Endrici - Manuel Toniato}
\beginsong{Vieni al Signor}[by={Calisi}]

\ifchorded
\beginverse*
\vspace*{-0.8\versesep}
{\nolyrics \[D-] \[D4] \[D-] \[D4] }
\vspace*{-\versesep}
\endverse
\fi

\beginverse
\[D-]Benedici \[C]il Signor \[D-]anima \[C]mia 
\[B&7+]quanto è in \[A-7]me lo bene\[D-]dica.\[D4] 
\[D-]Non dimenti\[C]care \[D-]i suoi bene\[C]fici, 
\[B&7+]quanto è in \[A-7]me lo bene\[D-]dica.
\endverse

\beginchorus
\[F]Egli per\[A-7]dona \[B&]tut\[G-]te le \[A]tue \[A7]colpe
\[D-]buono e pie\[C]toso è il Si\[B&7+]gnore \[A-7]lento all'\[A7]ira.
\[D-] Vieni \[D-7]al Signor,\[B&7+] ri\[A-7]cevi il suo a\[D-]mor.
\endchorus

\chordsoff
\beginverse
Salva dalla fossa la tua vita 
e ti incorona di grazia.
Come il cielo è alto sopra la terra 
così è la sua misericordia.
\endverse

\beginverse
Ma la grazia del Signor dura in eterno 
per quelli che lo temono.
Benedici il Signor, anima mia 
quanto è in me lo benedica.
\endverse
\endsong






