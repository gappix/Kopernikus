%titolo{Sol, soletto vola il pipistrello}
%autore{}
%album{}
%tonalita{Do}
%famiglia{Scout}
%gruppo{}
%momenti{}
%identificatore{sol_soletto_vola_il_pipistrello}
%data_revisione{2012_11_28}
%trascrittore{Antonio Badan}
\beginsong{Sol, soletto vola il pipistrello}
\beginverse
\[C7]Sol, so\[F]letto \[C]vola il pipi\[F]strello
e la luna fa \[G7]capo\[C]lin.
\[C7]Dentro al \[F]bosco, \[C]sopra un ramo\[F]scello,
messer gufo \[G-]canta \[C7]con ar\[F]dor.
\endverse
\beginchorus
\[C7]Po, \[F]po po po, \[G7]po po po po, \[C]po po po
\[B&]Po po po po, \[G7]po po po, \[F]po po \[C7]po po, \[F]po.
\endchorus
\chordsoff
\beginverse
Dallo stagno salta fuori in fretta
diguazzando un bel ranocchion
che a sintere quella musichetta
prende fiato e si unisce al cor.
\endverse
\beginchorus
Cra cra cra cra\dots{}
\endchorus
\beginverse
Il concerto aumentò di tono
fino a quando un calabrone udì
che, svegliato da quel gran frastuono,
irritato volò via di lì.
\endverse
\beginchorus
Zum zum zum zum\dots{}
\endchorus
\beginverse
Ma alla vista di quei cuor contenti,
tutti presi e intenti a solfeggiar,
dié un sospiro, borbottò fra i denti,
poi con loro prese a canticchiar.
\endverse
\beginchorus
{\scriptsize Una parte del branco:}
Po po po po\dots{}
{\scriptsize Una seconda parte del branco:}
Cra cra cra cra\dots{}
{\scriptsize Una terza parte del branco:}
Zum zum zum zum\dots{}
\endchorus
\endsong