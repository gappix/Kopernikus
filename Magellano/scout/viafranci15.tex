%titolo{viafranci15}
%autore{Andrea Brandalise, Sara Scaramuzza}
%album{}
%tonalita{Re-}
%famiglia{Scout}
%gruppo{}
%momenti{}
%identificatore{viafranci15}
%data_revisione{2015_08_19}
%trascrittore{Francesco Endrici}
\beginsong{\# viafranci15}[by={Brandalise, Scaramuzza}]
\transpose{-5}
\beginchorus
\memorize
{\[F#-]\#} Via Franci \[D]15 
\[A]\# a \[E]Roma da Canterbury 
\[F#-]\# \[D]Lucca è alle spalle ormai 
\[A]\# a\[E]desso siamo nei guai.
\endchorus
\beginverse
Il ^Daniele ha nostalgia della sua terra,
il Mat^tia ancora dorme dall'Inghilterra,
il ^Vale è un fan delle strisce pedonali, 
^Mauro e Fritz i nostri cuochi personali.
\endverse
\beginverse*
L'Eu^genio ha un look da mafioso,
e il Mi^chele è un tanto schizzinoso,
la ^Mery ha perso tutta la sua voce,
il Fran^cesco ha mangiato una noce, oh no.
\endverse
\beginverse
Il Mat^teo ha uno zaino monospalla, 
^Branda gira video a palla,
la ^Giulia ha un pigiama assai stiloso, 
^Maddy e Ila allergiche al lattosio.
\endverse
\beginverse*
La ^Marta ha l'ansia per l'orario,
per l'Al^berto il nostro passo è un calvario,
la ^Tere fa strada a quattro zampe,
la ^Gloria ne ha una che fa psss psss.
\endverse
\beginverse
Le cit^tà sulla via sono diverse,
la ^Cia e la Sara guide esperte,
^San Gimignano è la nostra meta, 
^ siam pellegrini allo sbaraglio \brk e senza cometa.
\endverse
\endsong