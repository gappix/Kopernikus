%titolo{Così sia}
%autore{Mattia Civico}
%album{}
%tonalita{Do}
%famiglia{Scout}
%gruppo{}
%momenti{}
%identificatore{cosi_sia}
%data_revisione{2017_05_15}
%trascrittore{Francesco Endrici}
\beginsong{Così sia}[by={Civico}]
\ifchorded
\beginverse*
\vspace*{-0.8\versesep}
{\nolyrics \[F]\[C]\[F]\[C]\[F]\[C]\[G] \rep{2}}
\vspace*{-\versesep}
\endverse
\fi
\beginverse
\memorize
Sei ve\[C9]nuto che era notte
era \[E-7]notte ed io dormivo. \[A-4] \[A-]
Ti ho sen\[C9]tito nel silenzio
e nel \[E-7]buio ti vedevo. \[A-4]\[A+ 4 5 ]
\endverse
\beginchorus
E mi hai \[D-]detto: "l'uomo giusto
non a\[C]scolta la paura";
la pa\[G]ura mi diceva di andar \[C]via. \[C7]
E mi hai \[F]detto: "serve un giusto
che si \[C]prenda sempre cura
di una \[G]donna che di nome fa Ma\[C]ria". 
\endchorus
\ifchorded
\beginverse*
\vspace*{-0.8\versesep}
{\nolyrics \[F]\[C]\[F]\[C]\[F]\[C]\[G]}
\vspace*{-\versesep}
\endverse
\fi
\beginverse
Non lo ^so come è successo: 
io non ^ero abituato.^^
Mi ri^trovo molto spesso
a pen^sarti e a esser pensato.^^
\endverse
\beginchorus
Ma ti ho \[D-]fatto una promessa
che avrei \[C]fatto "del mio meglio"
per ser\[G]vire e restare insieme a \[C]Te. \[C7]
E mi hai \[F]fatto la Promessa,
che era \[C]giusto ciò che voglio,
di re\[G]stare ogni giorno accanto a \[C]me. 
\endchorus
\ifchorded
\beginverse*
\vspace*{-0.8\versesep}
{\nolyrics \[F]\[C]\[F]\[C]\[F]\[C]\[G]}
\vspace*{-\versesep}
\endverse
\fi 
\beginverse
Non è ^mica cosa grande, 
non è ^mica complicata. ^^
È una ^storia elementare:
un bam^bino l'ha inventata.^^
\endverse
\beginchorus
La pa\[D-]rola che è nel legno
non la \[C]trovi col martello,
ma ti \[G]parla se quel troppo togli \[C]via. \[C7]
Non ci \[F]vuole grande ingegno
a indos\[C]sare quel mantello
ed in \[G]fondo dire: "Amen, così \[C]sia." 
\endchorus
\ifchorded
\beginverse*
\vspace*{-0.8\versesep}
{\nolyrics \[F]\[C]\[F]\[C]\[F]\[C]\[G] \rep{2}}
\vspace*{-\versesep}
\endverse
\fi
\endsong

