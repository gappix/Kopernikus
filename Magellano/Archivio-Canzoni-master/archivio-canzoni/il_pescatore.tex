%titolo{Il pescatore}
%autore{Fabrizio De André}
%album{L'ufficio delle cose perdute}
%tonalita{La}
%famiglia{Altre}
%gruppo{}
%momenti{}
%identificatore{il_pescatore}
%data_revisione{2012_04_03}
%trascrittore{Francesco Endrici}
\beginsong{Il pescatore}[by={De\ André}]

\beginverse*
All'ombra \[A]dell'ul\[E7]timo \[A]sole
s'era asso\[D]pito un pesca\[A]tore
e aveva un \[D]solco \[E7]lungo il \[A]viso
\[D]come una \[A]specie \[E7]di sor\[A]riso.
\endverse
\beginverse*
Venne alla ^spiaggia un ^assas^sino
 due occhi ^grandi da bam^bino
due occhi e^normi ^di pa^ura
^eran gli ^specchi di ^un'avven^tura.
La la la \[D]\[A E A D E7 A D A E7 A]{la\dots}
\endverse
\chordsoff
\beginverse*
E chiese al ^vecchio, ^dammi il ^pane
ho poco ^tempo e troppa ^fame
e chiese al ^vecchio, ^dammi il ^vino
^ho sete e ^sono un ^assas^sino.
\endverse
\beginverse*
Gli occhi di^schiuse il ^vecchio al ^giorno
non si guar^dò neppure in^torno
ma versò il ^vino e ^spezzò il ^pane
^per chi di^ceva ho ^sete e ho ^fame. La la la \[D]\[A E A D E7 A D A E7 A]{la\dots}
\endverse
\beginverse*
E fu il ca^lore ^di un mo^mento
poi via di ^nuovo ^verso il ^vento
davanti agli ^occhi ancora il ^sole
^dietro alle ^spalle un ^pesca^tore.
\endverse
\beginverse*
Dietro alle ^spalle un ^pesca^tore
e la me^moria è già do^lore
è già il rim^pianto ^di un a^prile
^giocato all'^ombra ^di un cor^tile. La la la \[D]\[A E A D E7 A D A E7 A]{la\dots}
\endverse
\beginverse*
Vennero in ^sella ^due gen^darmi
vennero in ^sella con le ^armi
chiesero al ^vecchio ^se lì vi^cino
^fosse pas^sato ^un assas^sino.
\endverse
\beginverse*
Ma all'ombra ^dell'ul^timo ^sole
s'era asso^pito il pesca^tore
e aveva un ^solco ^lungo il ^viso
^come una ^specie ^di sor^riso.
e aveva un \[D]solco \[E7]lungo il \[A]viso
\[D]come una \[A]specie \[E7]di sor\[A]riso.
\endverse
\endsong

