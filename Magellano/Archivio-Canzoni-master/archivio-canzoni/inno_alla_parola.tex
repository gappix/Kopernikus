%titolo{Inno alla Parola}
%autore{Valenti, Farruggio}
%album{}
%tonalita{Do}
%famiglia{Liturgica}
%gruppo{}
%momenti{}
%identificatore{inno_alla_parola}
%data_revisione{2011_12_31}
%trascrittore{Francesco Endrici}
\beginsong{Inno alla Parola}[by={Valenti, Farruggio}]
\beginverse
Nei \[C]giorni che non a\[F]vevano tempo,
viveva con Dio nel si\[C]lenzio.
Parola che era la \[F]gloria e l'amore
fiorita in segreto all'im\[C]menso.
In \[G]Lei è la forza del \[A-]mondo, la \[F]vita,
fu \[A7]fatto da Lei lo spazio e il \[D-]sole;
in\[G]fuse la mente alla \[A-]carne dell'\[F]uomo,
la \[A-]terra per \[B7]casa do\[E-]nò. \[G7]
\endverse
\beginverse
\chordsoff
Nascendo poi nella storia del mondo,
vedemmo tra noi la sua gloria;
nel buio la luce era apparsa in un volto,
l'amore ebbe il nome di un uomo.
Il mondo di tenebra fugge la luce
l'accoglie chi il cuore aprirà,
credendo che quella Parola è la vita
Iddio per Padre avrà.
\endverse
\beginverse
\chordsoff
Vivendo le nostre giornate, ai poveri
annuncia il perdono e il suo Regno;
e come un seme, per crescere grano,
dovrà nella terra morire,
così, per dar vita, fu uomo di croce,
vivente per Dio ritornò;
Signore del cielo, speranza del mondo,
la forza all'uomo donò.
\endverse
\beginverse
\chordsoff
Ci rese le ali per farci salire
gli spazi abitati da Dio.
Ci disse che il mondo,
crescendo nel tempo,
matura il suo corpo di gloria.
L'immenso rimane con noi incarnato,
la luce tramonto non avrà,
la fede proclama il nostro Signore,
il Dio che vive con noi.
\endverse
\endsong

