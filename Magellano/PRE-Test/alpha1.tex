% !TEX encoding = UTF-8 Unicode


%? STILE DOCUMENTO
%?????????????????????????????????????????????????????????
\documentclass[12pt, a4paper, twoside, openright]{book}
 %tipologia doc, dim. testo, apertura pagine, fronte/retro
 \usepackage{type1ec} 
 \usepackage[T1]{fontenc}		 %
 \usepackage[utf8]{inputenc}
 \usepackage[english,italian]{babel} %dizionario sillabazione
 \usepackage{afterpage} %per inserire pagine bianche
 \usepackage[pass]{geometry} %per la geometria di pagina
 
 
 %\usepackage{mathpazo} % FONT UTILIZZATA
 
 
\usepackage{ifxetex}



\usepackage{graphicx}
\usepackage{amssymb}
\usepackage{pict2e}
\usepackage{fancyhdr}
\usepackage{color}
\usepackage{tikz}
\usepackage{tabularx}
\usepackage{multirow}
\usepackage{ragged2e}
\usepackage{calc}
\usepackage{emptypage}
\usepackage{ifthen}
\definecolor{mygray}{gray}{0.0} 
\usepackage{titletoc}
\titlecontents{chapter}
[0em]
{\vspace{\baselineskip}}
{\bfseries}
{\bfseries}
{}
%************************************
%Creazione delle "unghie" dei capitoli*
%************************************
\setlength{\unitlength}{1pt} %unit� di misura per picture
\newcommand{\fgletterstyle}{\large} % scelgo come voglio appaiano le lettere 
\newlength{\outmargine} 
\setlength{\outmargine}{\paperwidth-1in-\oddsidemargin-\textwidth} %misura del margine esterno
\newlength{\fgwidth}
\setlength{\fgwidth}{0.8\outmargine} %larghezza della cella colorata
\newlength{\fgheight}
\settoheight{\fgheight}{\fgletterstyle H} %misuro l'altezza delle lettere maiuscole
\newcommand{\fgarg}{}
\newfont{\letterfont}{pplr9d scaled 6000} %font delle lettere nei capitoli
\newcounter{contatori}
\newcounter{letter} 
\makeatletter
\def\lettera{\@ifstar{\sblettera}{\nextcol\sblettera}}% la versione con asterisco non inizia una nuova colonna, serve all'inizio del documento.
\makeatother
\newcommand{\sblettera}{%
\begin{intersong}
\refstepcounter{letter}
\refstepcounter{contatori}
\addcontentsline{toc}{chapter}{\Alph{letter}}
\setlength{\parindent}{0pt}
\makebox[\hsize]{\letterfont\Alph{letter}}
\vspace*{2\baselineskip}
\end{intersong}%
\renewcommand{\fgarg}{\Alph{letter}}%
}
\makeatletter
\def\capitoletto{\@ifstar{\sbcapitoletto}{\nextcol\sbcapitoletto}}
\makeatother
\newcommand{\sbcapitoletto}[2]{%Versione estesa e contrazione per unghie
\begin{intersong}
\refstepcounter{contatori}
\addcontentsline{toc}{chapter}{#1}
\makebox[\hsize]{\bfseries\LARGE#1}
\vspace*{2\baselineskip}
\end{intersong}
\renewcommand{\fgarg}{#2}}

\newcommand{\blob}{\setlength{\fboxsep}{0pt}%
\colorbox{mygray}{%
\makebox[\fgwidth]{\textcolor{white}{\rule[-0.4\fgheight]{0pt}{1.8\fgheight}\fgletterstyle\fgarg}}}} %cella colorata, imposto fboxsep a zero per avere le misure che voglio io

\newcommand\rblob{%
\begin{picture}(0,0)(0,6)
\put(0,-\numexpr15*\value{contatori}\relax){\makebox[\dimexpr\outmargine-\fgwidth\relax]{}\blob}
\end{picture}}

%*********************
%Testatine
\pagestyle{fancy}%
\setlength{\headheight}{13pt}
\fancyhf{}
\renewcommand{\headrulewidth}{0pt}
\renewcommand{\footrulewidth}{0.4pt}
\fancyhead[RO]{\rblob}
\fancyfoot[RO]{\itshape\thepage}
\fancyfoot[LE]{\itshape\thepage}  
\fancypagestyle{plain}{%
\fancyhead{} % leva l'intestazione
%\fancyfoot[RO]{\itshape\thepage} %mette il numero di pagina in basso al centro
\renewcommand{\headrulewidth}{0pt} % e leva la linea dalle pagine di inizio capitolo
}
%
%%%%%%%%%%%%%%%%%%%%%%%%%%%%%%%%%%%%%%%%%%%%%%%%%%%%
%SONGS impostazioni del pacchetto songs
\usepackage[noindexes]{songs}
\MultiwordChords								% accordi sopra pi� sillabe
\notenamesin{A}{B}{C}{D}{E}{F}{G}  			% scriver� gli accordi secondo la notazione anglosassone
\notenamesout{La}{Si}{Do}{Re}{Mi}{Fa}{Sol} 	% e lui me li restituir� secondo la notazione italiana

\renewcommand{\printchord}[1]{\footnotesize\bf\sffamily#1}	% Accordi in neretto
\renewcommand{\sharpsymbol}{\ensuremath{\sharp}}  % diesis
\renewcommand{\flatsymbol}{$\flat$}				% bemolle
\setlength{\cbarwidth}{0pt}						% linea laterale ritornelli
%\setlength{\sbarheight}{0pt}						%linee sopra e sotto
\renewcommand{\chorusfont}{\bf}					%font ritornelli
\setlength{\columnsep}{10mm}
\setlength{\parindent}{0pt}

\ifchorded
\afterpreludeskip=-2pt plus 3pt minus 6pt
\setlength{\versesep}{0.3\baselineskip plus 4pt minus 2pt}
\else
\afterpreludeskip=6pt
\setlength{\versesep}{0.6\baselineskip plus 4pt minus 2pt}
\fi
\versenumwidth=5.2mm
\renewcommand{\everychorus}{\setlength{\leftskip}{\versenumwidth}}
\makeatletter
\renewcommand\textnote[1]{% note in senza grazie e senza sfondo
  \ifhmode\par\fi%
  \ifnum\lastpenalty=12345\else%
    \ifSB@inverse%
      \vskip2\p@\relax%
    \else\ifSB@inchorus%
      \vskip2\p@\relax%
    \else\ifSB@stanza%
      \nobreak\vskip\versesep%
    \fi\fi\fi%
  \fi%
  \begingroup%
    \everypar{}%
    \ifchorded\chordedfalse\SB@setbaselineskip\chordedtrue\fi%
    \placenote{\normalfont\sffamily{\SB@boxup{#1}}}%
  \endgroup%
  \nobreak%
  \ifSB@inverse%
    \vskip2\p@\relax%
  \else\ifSB@inchorus%
    \vskip2\p@\relax%
  \else\ifSB@stanza\else%
    \nobreak\vskip\versesep%
  \fi\fi\fi%
}
\renewcommand{\rep}[1]{($\times$#1)}		% cambia la x nelle ripetizioni
\makeatother

%%%%%%%%%%%%%%%%%%%%%%%%%%%%%%%%%%%%%%%
% titoli come canzoniere AC Padova
\newlength{\titlength}
\newlength{\numerwidth}
\newlength{\larcolonna}
\renewcommand\makeprelude{%
\resettitles%
\ifthenelse{\equal{\songauthors}{}}{%
\phantomsection
\addcontentsline{toc}{section}{\thesongnum.\ \songtitle}}%
{\phantomsection
\addcontentsline{toc}{section}{\thesongnum.\ \songtitle\ \footnotesize\itshape(\songauthors)}}
\settowidth{\numerwidth}{\bf\Huge \sffamily \thesongnum}
  \newcommand{\titolo}{\fontsize{14}{18}\selectfont\bfseries\sffamily \songtitle}
 \settowidth{\titlength}{\titolo}
 \setlength{\larcolonna}{\hsize-\numerwidth-30pt}
\begin{tabularx}{\hsize}{r|X}
\renewcommand{\arraystretch}{0} 
& \ifdim\titlength<\larcolonna \titolo \else \resizebox{\larcolonna}{!}{\titolo} \fi\\
\cline{2-2}  
\multirow{-2}{*}{{\bf\Huge \sffamily \thesongnum}} & \ifthenelse{\equal{\songauthors}{}}{}{\raggedleft\footnotesize\itshape(\songauthors)}
\end{tabularx}
\vspace*{-0.4\baselineskip}
       \nexttitle%
       \foreachtitle{(\songtitle)\par}%
      }
%%%%%%%%%%%%%%%%%%%%%%%%%%%%%%%%%%%%%%%%%%%%%%%%%%
% Modificare l'interlinea
\iflyric
\renewcommand{\colbotglue}{0pt plus .5\textheight minus 0pt}
\baselineadj=2pt plus 0pt minus 0pt  % nel canzoniere senza accordi regolo l'interline di TUTTE le strofe
\fi
\ifchorded  % nel canzoniere con accordi regolo l'interline delle:
\baselineadj=3pt plus 0pt minus 0pt  %strofe con accordi (interlinea generale)
\newcommand{\mychordsoff}{}
\let\mychordsoff\chordsoff
\renewcommand{\chordsoff}{\baselineadj=1.4pt\mychordsoff} %strofe senza accordi
\fi
\renewcommand{\clineparams}{ 
\baselineskip=8pt 
\lineskiplimit=0pt 
\lineskip=0pt
}
%%%%%%%%%%%%%%%%%%%%%%%%%%%%%%%%%%%%%%%
%Per modificare il font delle seconde voci-risposte
\makeatletter
\renewcommand\SB@echo[1]{%
	\endgroup% 
	\begingroup%
		\ifdim\fontdimen\@ne\font>\z@\upshape\else\itshape\fi% 
		\endlinechar\m@ne% 
		\scantokens{(#1)}%
	\endgroup% 
}

\newcommand{\ps@intro}{
   \renewcommand{\@oddhead}{}%     header blank
   \renewcommand{\@evenhead}{}%    header blank
   \renewcommand{\@oddfoot}{\hfill\itshape\thepage}%     
   \renewcommand{\@evenfoot}{\itshape\thepage\hfill}%     
}
\makeatother
\renewcommand{\makepostlude}{\vskip10pt plus 10pt minus 3pt}

%%%%%%%%%%%%%%%%%%%%%%%%%
%eliminare i bookmark di songs, utile per creare la tableofcontents
\makeatletter
\renewcommand\SB@endsong{%
  \ifSB@insong%
      \ifSB@inverse\SB@erreov\endverse\fi%
      \ifSB@inchorus\SB@erreoc\endchorus\fi%
      \global\SB@skip\versesep%
      \unskip%
      \ifrepchorus\ifvoid\SB@chorusbox\else%
        \ifSB@prevverse\ifvnumbered%
          \marks\SB@cmarkclass{\SB@lastcmark}%
        \fi\fi%
      \fi\fi%
    \endgroup\egroup%
    \setbox\SB@songbox\vbox{%
      \songmark%
      \unvbox\SB@songwrites%
      \ifnum\SB@numcols>\z@\hsize\SB@colwidth\fi%
      \leftskip\z@skip\rightskip\z@skip%
      \parfillskip\@flushglue\parskip\z@skip\parindent\z@%
      \ifdim\sbarheight>\z@%
        \hrule\@height\sbarheight\@width\hsize%
        \nobreak\vskip5\p@%
      \fi%

      %\ifpdfindex\begingroup%
        %\ifx\pdfbookmark\undefined\else%
        %\ifx\pdfbookmark\relax\else%
          %\resettitles%
          %\pdfbookmark[\ifnum\c@section=\z@1\else2\fi]%
           % {\thesongnum. \songtitle}%
           % {song\theSB@songsnum-\thesongnum}%
        %\fi\fi%
      %\endgroup\fi%
      \vbox{\makeprelude}%
      \nobreak\vskip\SB@skip%
      \vskip\afterpreludeskip\relax%
      \unvbox\SB@songbox%
      \nobreak\vskip\SB@skip%
      \vskip\beforepostludeskip\relax%
      \nointerlineskip%
      \vbox{\makepostlude}%
      \ifdim\sbarheight>\z@%
        \nobreak\vskip2\p@\@plus\p@%
        \nointerlineskip%
        \hbox{\vrule\@height\sbarheight\@width\hsize}%
      \fi%
    }%
    \SB@insongfalse%
    \edef\SB@sgroup{\thesongnum}%
    \global\SB@groupcnt\z@%
    \SB@submitsong%
    \ifnum\SB@grouplvl=\z@\let\SB@sgroup\@empty\fi%
    \stepcounter{songnum}%
  \else%
    \ifSB@intersong\SB@erreor\SB@closeall%
    \else\SB@erreot\fi%
  \fi%
}


\makeatother	


\songpos{0} % agire siu questo parametro per modificare il criterio di impaginazione di Songs
\makeatletter
\renewcommand{\tableofcontents}
{\chapter*{\contentsname}
\csname @twocolumntrue\endcsname
\csname @starttoc\endcsname{toc}}
\makeatother
\usepackage{hyperref}
%%%%%%%%%%%%%%%%%%%%%%%%%%%%%%%%%%%%%%%%%%%%%%%%%%%%%%%%%%%
%Inizio
%%%%%%%%%%%%%%%%%%%%%%%%%%%%%%%%%%%%%%%%%%%%%%%%%%%%%%%%%%%%
\begin{document}
\twocolumn
\renewcommand{\fgarg}{Ind}
\tableofcontents
\onecolumn
%INIZIO%canzoni
\cleardoublepage
\begin{songs}{}
\songcolumns{2}



\capitoletto{Agnello di Dio}{Agn}
%titolo{Agnello di Dio}
%autore{Buttazzo}
%album{È nato il salvatore}
%tonalita{Sol}
%gruppo{Agnello}
%momenti{Agnello}
%identificatore{agnello_di_dio_buttazzo_e_nato_il_salvatore}
%data_revisione{2011_12_31}
%trascrittore{Francesco Endrici}
\beginsong{Agnello di Dio}[by={Buttazzo}]
\ifchorded
\beginverse*
\vspace*{-0.8\versesep}
{\nolyrics \[G]\[D]\[C]\[D]\[G]\[Bm7]\[C]\[C]}
\vspace*{-\versesep}
\endverse
\fi
\beginverse*
A\[G]gnello di \[Bm7]Dio che \[C]togli i pec\[D]cati del \[G]mondo \[Em7]
\[C]abbi pie\[D]tà di \[G]noi,\[Bm7] \[C]abbi pie\[D]tà di \[G]noi.
A\[G]gnello di \[Bm7]Dio che \[C]togli i pec\[D]cati del \[G]mondo \[Em7]
\[C]abbi pie\[D]tà di \[G]noi,\[Bm7] \[C]abbi pie\[D]tà di \[G]noi. \[F]\[G]
A\[C]gnello di \[D]Dio che \[B7]togli i pec\[Em7]cati del \[D4]mondo\[D]
\[C]dona a \[D]noi la \[G]pa\[Bm7]ce, \[C]dona a \[D]noi la \[C]pa\[C]ce. \[G]
\endverse
\endsong

%titolo{Agnello di Dio}
%autore{Buttazzo}
%album{Vita nuova con Te}
%tonalita{Re}
%gruppo{Agnello di Dio}
%momenti{Agnello di Dio}
%identificatore{agnello_di_dio_buttazzo_vita_nuova}
%data_revisione{2011_12_31}
%trascrittore{Francesco Endrici - Manuel Toniato}
\beginsong{Agnello di Dio}[by={Buttazzo}]

\ifchorded
\beginverse*
\vspace*{-0.8\versesep}
{\nolyrics \[D] \[G7+] \[D] \[G7+] }
\vspace*{-\versesep}
\endverse
\fi

\beginverse*
A\[D]gnello Agnello di \[G7+]Dio 
che \[A4]togli i pec\[A/G]cati del \[F#m7]mon\[Bm7]do
\[G]abbi \[A4] pie\[A]tà di \[F#m7]no\[Bm7]i,
\[Em7]abbi  \[A7] pietà di \[G]no\[G/A]i.
\endverse

\beginverse*
A\[D]gnello Agnello di \[G7+]Dio
che \[A4]togli i pec\[A/G]cati del \[F#m7]mon\[Bm7]do
\[G]abbi \[A4] pie\[A]tà di \[F#m7]no\[Bm7]i,
\[Em7]abbi  \[A7] pietà di \[G]no\[G/A]i.
\endverse

\beginverse*
A\[D]gnello \[D/C]   Agnello di \[G/B]Dio \[G] 
che \[D]togli i pec\[Em7]cati, che \[D/F#]togli i pec\[G]cati
del \[A4]mon\[A]do
\[G]dona a \[G/A]noi la  \[D7+]Pace, \[Bm7] 
\[G]dona a \[G/A]noi la \[G/D]pa\[D]ce.
\endverse
\endsong

%titolo{Agnello di Dio}
%autore{Gen Rosso e Gen Verde}
%album{Come fuoco vivo}
%tonalita{Fa}
%gruppo{Agnello di Dio}
%momenti{Agnello di Dio}
%identificatore{agnello_di_dio_come_fuoco_vivo}
%data_revisione{2011_12_31}
%trascrittore{Francesco Endrici - Manuel Toniato}
\beginsong{Agnello di Dio}[by={Gen\ Rosso, Gen\ Verde}]

\ifchorded
\beginverse*
\vspace*{-0.8\versesep}
{\nolyrics \[F] \[C] \[Dm7] \[C] \[F] \[C] \[Am7] \[B&2]}
\vspace*{-\versesep}
\endverse
\fi

\beginverse*
A\[F]gnel\[C]lo di \[Dm7]Dio 
che \[B&2]togli i pec\[F]cati  \[C] del \[Am7]mon\[B&2]do \[F]  \[C]  \[Dm7]  \[C] 
\[F]abbi pie\[C]tà, \[Am7]abbi pie\[B&2]tà di \[F]noi. \[C]  \[Dm7]  \[C] 
\endverse

\beginverse*
A\[F]gnel\[C]lo di \[Dm7]Dio 
che \[B&2]togli i pec\[F]cati  \[C] del \[Am7]mon\[B&2]do \[F]  \[C]  \[Dm7]  \[C] 
\[F]abbi pie\[C]tà, \[Am7]abbi pie\[B&2]tà di \[F]noi. \[C]  \[Dm7] \[C] 
\endverse

\beginverse*
A\[F]gnel\[C]lo di \[Dm7]Dio che \[B&2]togli i pec\[F]cati  \[C] del \[Am7]mon\[B&2]do 
\[Dm]dona a noi la \[C]Pace, \[B&]dona a noi la \[Am7]Pa\[D]ce. 
\endverse
\ifchorded
\beginverse*
\vspace*{-\versesep}
{\nolyrics \[A/D] \[Bm7/D] \[A/D] \[D] \[A/D] \[F#m] \[G] \[D]}
\endverse
\fi
\endsong

%titolo{Agnello di Dio}
%autore{De Dominicis}
%album{Ad una voce}
%tonalita{Mi}
%gruppo{Agnello di Dio}
%momenti{Agnello di Dio}
%identificatore{agnello_di_dio_dominicis}
%data_revisione{2011_12_31}
%trascrittore{Francesco Endrici - Manuel Toniato}
\beginsong{Agnello di Dio}[by={De\ Dominicis}]

\ifchorded
\beginverse*
\vspace*{-0.8\versesep}
{\nolyrics \[E] \[B/D#] \[A/C#] \[E] \[B/D#] \[A/C#] 
\[E] \[B/D#] \[A/C#] \[A] \[B] \[A/C#] \[B/D#]}
\vspace*{-\versesep}
\endverse
\fi

\beginverse*
A\[E]gnello di Dio che \[C#m]togli i peccati del \[A]mondo
\[F#m]abbi pie\[A]tà, di \[B4]no\[B]i. 
A\[E]gnello di Dio che \[C#m]togli i peccati del \[A]mondo
\[F#m]abbi pie\[F#/A#]tà, di \[B4]no\[B/A]i. 
A\[G#m]gnello di Dio che \[C#m]togli i peccati del \[A]mondo
\[F#m]dona a \[B4]no\[B]i la \[Am/E]pa\[E]ce. \rep{2}
\endverse

\ifchorded
\beginverse*
\vspace*{-0.8\versesep}
{\nolyrics \[E] \[B/D#] \[A/C#] \[E] \[B/D#] \[A/C#] \[E] \[B/D#] \brk \[A/C#] \[A] \[B] \[C] \[D]}
\endverse
\fi

\beginverse*
A\[G]gnello di Dio che \[Em]togli i peccati del \[C]mondo 
\[Am]abbi pie\[A/C#]{tà,} di \[D4]{no}\[D]i 
A\[G]gnello di Dio che \[Em]togli i peccati del \[C]mondo
\[Am]abbi pie\[A/C#]{tà,} di \[D4]{no}\[D/C]i 
A\[Bm]gnello di Dio che \[Em]togli i peccati del \[C7+]mondo \[Bm] 
\[Am]{do}\[G/B]{na a} \[D4]{no}\[D]i la \[Cm/G]{pa}\[Cm6/G]\[G]{ce.}
\endverse
\endsong

%




\end{songs}
\end{document}