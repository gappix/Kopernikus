\lettera*
%titolo{A Betlemme di Giudea}
%autore{Tradizionale Francese}
%album{}
%tonalita{Re}
%gruppo{}
%momenti{Natale}
%identificatore{a_betlemme_di_giudea}
%data_revisione{2011_12_31}
%trascrittore{Francesco Endrici}
\beginsong{A Betlemme di Giudea}[by={}]
\beginverse
\[D]A Betlemme \[A]di Giu\[D]{dea} \[D]una gran luce \[A7]si le\[D]vò:
\[Bm]nella notte, \[A]sui pa\[D]stori, \[Bm]scese l'annuncio e \[A]si can\[D]tò.
\endverse

\beginchorus
\[D]\[Bm]\[Em]\[A7]\[D]\[G]Glo\[A]{ria} \[D]in ex\[G]celsis \[D]{De}\[A]o
\[D]\[Bm]\[Em]\[A7]\[D]\[G]Glo\[A]{ria} \[D]in ex\[G]celsis \[D]\[A]De\[D]o
\endchorus

\beginverse
\chordsoff
Cristo nasce sulla paglia, \brk figlio del Padre Dio con noi,
Verbo eterno, Re di pace, \brk pone la tenda in mezzo ai suoi.
\endverse

\beginverse
\chordsoff
Tornerà nella sua gloria, \brk quando quel giorno arriverà:
se lo accogli nel tuo cuore \brk tutto il suo Regno ti darà.
\endverse
\endsong

%titolo{Abbà misericordia}
%autore{de Luca, Aguila}
%album{Io scelgo te}
%tonalita{Do}
%gruppo{}
%momenti{Ingresso;Quaresima;Conversione;Riconciliazione}
%identificatore{abba_misericordia}
%data_revisione{2011_12_31}
%trascrittore{Francesco Endrici}
\beginsong{Abbà misericordia}[by={De\ Luca, Aguila}]
\ifchorded
\beginverse*
\vspace*{-0.8\versesep}
{\nolyrics \[C] \[F/C] \[G/C] \[F/C] \[C]}
\vspace*{-\versesep}
\endverse
\fi%

\beginverse
\[C]Non sono degno di \[F/C]essere qui,
\[G/C]ho abbandonato \[C]la tua ca\[G/B]sa,
\[Am]ho dissipato i tuoi \[Dm]beni,
\[G]Padre ho peccato contro il \[E7/G#]cielo e contro di \[Am]te. \[G4] \[G] 
\endverse

\beginchorus
Ab\[C]bà, \[F]miseri\[G]cordia Ab\[C]bà,
\[F]miseri\[E]cordia Ab\[Am]bà, \[Dm7]{Ab}\[G4]bà. \[G] 
Ab\[C]bà, \[F]miseri\[G]cordia Ab\[C]bà,
\[F]miseri\[E]cordia Ab\[Am]bà, \[Dm7]{Ab}\[G4]bà,  \[G]{Ab}\[C]bà
\endchorus

\beginverse
\chordsoff
Non sono degno di esser tuo figlio,
in Gesù Cristo rialzami
e in lui ridammi la vita,
Padre ho peccato contro il cielo e contro di te.
\endverse

\beginverse
\chordsoff
Non sono degno del tuo amore,
riempi il mio cuore del tuo Spirito,
insieme a te farò festa per sempre,
Padre ho peccato contro il cielo e contro di te.
\endverse

\beginchorus
Ab\[C]bà, \[F]miseri\[G]cordia Ab\[C]bà,
\[F]miseri\[E]cordia Ab\[Am]bà, \[Dm7]Ab\[G4]bà. \[G]
Ab\[C]bà, \[F]miseri\[G]cordia Ab\[C]bà,
\[F]miseri\[E]cordia Ab\[Am]bà, \[Dm7]Ab\[G4]bà,\[A7] 
Ab\[D]bà, \[G]miseri\[A]cordia Ab\[D]bà,
\[G]miseri\[F#]cordia Ab\[Bm]{bà,} \[Em7]Ab\[A4]bà \[A] 
Ab\[D]bà, \[G]miseri\[A]cordia Ab\[D]bà,
\[G]miseri\[F#]cordia Ab\[Bm]bà, \[Em7]Ab\[A4]bà, \[A]{Ab}\[D]bà
\endchorus
\endsong

%titolo{Acclamate a Dio}
%autore{Attinà, Zunino}
%album{Ad una voce}
%tonalita{Re}
%gruppo{}
%momenti{Ingresso;Salmi}
%identificatore{acclamate_a_dio}
%data_revisione{2011_12_31}
%trascrittore{Francesco Endrici}
\beginsong{Acclamate a Dio}[by={Attinà, Zunino}]
%\category{Ingresso}
%\category{Salmi}

\ifchorded
\beginverse*
\vspace*{-0.8\versesep}
{\nolyrics \[D] \[Bm] \[G] \[D] \[Bm] \[G] \[F#m] \[G6]}
\vspace*{-\versesep}
\endverse
\fi%
\beginchorus
Acclamate a \[A]Dio \[D]da tutta la \[F#m]{ter}\[Bm]ra,
Can\[Em7]tate alla gloria \[G/A]del suo nome,
Date a \[A]Lui \[D]splendida \[F#m]{lo}\[Bm]de
Stu\[Em]pende \[D]sono le \[G]sue ope\[A]{re} \[D] \[F#m] \[Bm] 
Stu\[Em]pende \[D]sono le \[G]sue ope\[A]{re} \[D] \[Gm] \[D] \[F#] 
\endchorus

\beginverse
\[Bm]Per la grandezza della \[F#m]sua potenza
Da\[G]vanti a Lui si pie\[A]gano i nemici.
A \[F#m]Dio si prostri la \[Bm]terra
A Lui \[G]canti inni, \[A]canti al suo nome
\endverse

\beginverse
\chordsoff
Il mare ha cambiato in terra ferma
Con la sua forza regnerà in eterno
Dio salva la nostra vita
Per questo in Lui esultiamo di gioia
\endverse

\beginverse
\chordsoff
Venite voi tutti che temete Dio
E narrerò quanto per me ha fatto
a Lui ho rivolto il mio grido
La mia lingua cantò la sua lode
\endverse
\endsong

%titolo{Acclamate al Signore}
%autore{Frisina}
%album{Benedici il Signore}
%tonalita{Do}
%gruppo{}
%momenti{Salmi}
%identificatore{acclamate_al_signore_frisina}
%data_revisione{2011_12_31}
%trascrittore{Francesco Endrici}
\beginsong{Acclamate al Signore}[by={Frisina}]

\beginchorus
\[C]Accla\[Am]mate al Si\[F]gno\[C]re, \[Am]voi \[Dm]tutti della \[G]terra
\[Am]e ser\[Em]vitelo con \[F]gio\[C]ia, an\[Am]date a \[F]lui con lieti \[G]canti.
Accla\[F]mate voi \[Em]{tut}\[Am]{ti}  \[Dm]al Si\[G]gno\[C]re.
\endchorus

\beginverse
\[C]Ricono\[Am]scete che il Si\[F]gno\[C]re,
\[Am]che il Si\[Dm]gnore è \[G]Dio,
\[Am]Egli ci ha \[Em]fatti siamo \[F]suo\[C]i,
suo \[Am]popolo e \[F]gregge del suo \[G]pascolo.
\endverse

\beginverse
\chordsoff
Entrate nelle sue porte
con degli inni di grazia,
i suoi atri nella lode,
Benedite, lodate il suo nome.
\endverse

\beginverse
\chordsoff
Poiché buono è il Signore,
eterna la sua misericordia,
la sua fedeltà si estende
sopra ogni generazione.
\endverse
\endsong

%titolo{Acclamate al Signore}
%autore{Rinaldi}
%album{Gerico, le tue mura crolleranno}
%tonalita{Re}
%gruppo{}
%momenti{Ingresso;Salmi}
%identificatore{acclamate_al_signore_rinaldi}
%data_revisione{2011_12_31}
%trascrittore{Francesco Endrici}
\beginsong{Acclamate al Signore}[by={Rinaldi}]
\ifchorded
\beginverse*
\vspace*{-0.8\versesep}
{\nolyrics \[D] \[A] \[D] \[A] \[Am7] \[G] \[D]}
\vspace*{-\versesep}
\endverse
\fi%

\beginchorus
\[D]Accla\[A]mate \[D]{al }Si\[A]gnore, voi \[D]tutti \[E]della \[A]terra,
ser\[D]vite il Signore \[E]nella gioia,
presen\[C#m7]tatevi a lui con \[E]{esul}\[F#7]tanza.
\[Bm7]Accla\[C#m7]mate, \[D]accla\[C#m7]mate, \[Bm7]accla\[E7]mate al Si\[A]gnor.
\endchorus

\beginverse
\[F]Riconoscete che il Si\[G]gnore è Dio;
\[C]egli ci ha fatto e noi \[E]siamo \[Am]suoi,
suo \[Dm]popolo  e \[F]gregge
del suo \[Dm7]{pa}\[E7]{sco}\[A]{lo.}
\endverse

\beginverse
\chordsoff
Varcate le sue porte con inni di grazie,
i suoi atri con canti di lode,
lodatelo, benedite il suo nome.
\endverse

\beginverse
\chordsoff
Poiché buono è il Signore,
eterna la sua misericordia,
la sua fedeltà per ogni generazione.
\endverse

\endsong


%titolo{Accogli i nostri doni}
%autore{Buttazzo}
%album{Vita nuova con Te}
%tonalita{Re}
%gruppo{}
%momenti{Offertorio}
%identificatore{accogli_i_nostri_doni_buttazzo}
%data_revisione{2011_12_31}
%trascrittore{Francesco Endrici}
\beginsong{Accogli i nostri doni}[by={Buttazzo}]
\ifchorded
\beginverse*
\vspace*{-0.8\versesep}
{\nolyrics \[G7+] \[A/G] \[F#m4] \[G7+] \[A/G] \[F#m4] 
\[Em7] \[A4] \[A] \[D] \[A/C#] \[Bm] \[Em7] \[A6] \[D4] \[D]}
\vspace*{-\versesep}
\endverse
\fi%

\beginverse
Ac\[D]cogli i nostri \[F#m7]doni, \[Em7]Dio dell'uni\[A4]verso \[A] 
in \[Bm]questo miste\[F#m7]rioso in\[G]contro col tuo \[A4]Figlio\[A].
Ti o\ch{Bm}{f}{f}{ff}riamo il \[F#m7]pane che \[G]{tu} \[G/B]{ci} \[D]{dai:} \[A/C#] 
tra\[Bm]sfor\[Em7]malo in \[G/A]{te,} \[Am]{Si}\[D]gnor.\[G/A] 
\endverse

\beginchorus
Bene\[D]detto nei \[Em7]secoli il Si\[A4]gnore \[A] 
in\[G]finita sor\[G/A]gente della \[F#m7]vita.     \[F#m5+] 
\[F#m]Bene\[Em7]detto nei \[A4]{se}\[A]{co}\[D]{li,}\[A/C#] \[Bm] 
bene\[Em7]detto nei \[A6]seco\[D]{li.}\[D4] \[D] 
\endchorus

\beginverse
\chordsoff
Accogli i nostri doni, Dio dell'universo
in questo misterioso incontro col tuo Figlio
Ti offriamo il vino che tu ci dai
trasformalo in te, Signor.
\endverse

\beginchorus
Bene\[D]detto nei \[Em7]secoli il Si\[A4]gnore \[A] 
in\[G]finita sor\[G/A]gente della \[F#m7]vita.     \[F#m5+] 
\[F#m]Bene\[Em7]detto nei \[A4]{se}\[A]{co}\[D]{li,}\[A/C#] \[Bm] 
Bene\[Em7]detto nei \[A6]seco\[G]{li.} \[D] 
\endchorus
\endsong


%titolo{Accogli i nostri doni}
%autore{Gen Verde}
%album{È bello lodarti}
%tonalita{Do}
%gruppo{}
%momenti{Offertorio}
%identificatore{accogli_i_nostri_doni_gen_verde}
%data_revisione{2011_12_31}
%trascrittore{Francesco Endrici}
\beginsong{Accogli i nostri doni}[by={Gen\ Verde}]
\beginverse*
Ac\[C]cogli Signore i nostri \[C7]do\[C]ni
in \[Am]questo misterioso in\[C]contro
\[F]tra la \[G]nostra \[C]po\[Em]ver\[Am]tà 
\[F]e la \[G]tua gran\[C]dezza. \[G]
\[Am]Noi ti offriamo le \[C]co\[G]se 
\[Am]che tu \[C]stesso ci hai \[G]da\[C]to
\[F]e tu in \[G]cambio \[C]do\[Em]na\[Am]ci 
\[Dm]dona\[G]ci te \[C]stes\[G]so.
\endverse
\endsong

%titolo{Acqua siamo noi}
%autore{Cento}
%album{È il giorno del Signore}
%tonalita{Re}
%gruppo{}
%momenti{Inizio}
%identificatore{acqua_siamo_noi}
%data_revisione{2011_12_31}
%trascrittore{Francesco Endrici}
\beginsong{Acqua siamo noi}[by={Cento}]

\ifchorded
\beginverse*
\vspace*{-0.8\versesep}
{\nolyrics \[D]\[G]\[A]\[G]\[D]\[G]\[A]\[G]
\[D]\[G]\[A]\[G]\[Em]\[G]\[A]}
\vspace*{-\versesep}
\endverse
\fi
\beginverse
\memorize
\[D]Acqua \[A]siamo \[D]noi, \brk dall'an\[G]tica sor\[A]gente ve\[D]niamo,
\[D]fiumi \[A]siamo \[D]noi \brk se i ru\[G]scelli si \[A]mettono in\[D]sieme,
\[D]mari \[G]siamo \[D]noi \brk se i tor\[G]renti si \[A]danno la \[D]mano,
\[D]vita nuova \[Bm]c'è \brk se Ge\[G]sù è in \[A]mezzo a \[D]noi.
\endverse
\beginchorus
E allora \[F#m]diamoci la \[Bm]mano
e tutti in\[F#m]sieme cammi\[G7+]niamo
ed un o\[F#m]ceano di \[Bm]pace nasce\[A4]rà. \[A]
E l'ego\[Em]ismo cancel\[C7+]liamo
e un cuore \[Em]limpido sen\[C]tiamo
è Dio che \[C7+]bagna del suo a\[A7]mor l'umani\[D]tà. \[A4]\[A]
\endchorus
\beginverse
^Su nel ^cielo ^c'è \brk Dio ^Padre che ^vive per ^l'uomo
^crea ^tutti ^noi \brk e ci ^ama di a^more infi^nito,
^figli ^siamo ^noi \brk e fra^telli di ^Cristo Si^gnore,
^vita nuova ^c'è \brk quando ^Lui è in ^mezzo a ^noi.
\endverse
\beginverse
%\chordsoff
^Nuova u^mani^tà \brk oggi ^nasce da ^chi crede in ^Lui,
^nuovi ^siamo ^noi \brk se l'a^more è la ^legge di ^vita,
^figli ^siamo ^noi \brk se non ^siamo di^visi da ^niente,
^vita eterna ^c'è \brk quando ^Lui è ^dentro ^noi.
\endverse
\endsong



%titolo{Acqua sole e verità}
%autore{Cento}
%album{Celebraimo la nostra speranza}
%tonalita{Sol}
%gruppo{}
%momenti{Comunione}
%identificatore{acqua_sole_verità}
%data_revisione{2011_12_31}
%trascrittore{Francesco Endrici}
\beginsong{Acqua sole e verità}[by={Cento}]
\ifchorded
\beginverse*
\vspace*{-0.8\versesep}
{\nolyrics \[G]\[Bm]\[C]\[D7]\[G]}
\vspace*{-\versesep}
\endverse
\fi
\beginverse
\memorize
Ho be\[G]vuto a una fon\[Bm]tana un'acqua \[C]chiara
che è ve\[D7]nuta giù dal \[G]cielo.
Ho so\[Em]gnato nella \[C]notte di tu\ch{D}{f}{f}{ff}armi
nella luce del \[G]sole.
Ho cer\[C]cato dentro a \[Am]me la veri\[D]tà.
\endverse
\beginchorus
Ed ho ca\[G]pito, mio Si\[Bm]gnore
che sei \[C]tu la vera \[D]acqua, \[D7]
sei tu il mio \[Em]sole
sei \[C]tu la veri\[D]tà. \rep{2}
\endchorus
\beginverse
\chordsoff
Tu ti ^siedi sul mio ^pozzo nel de^serto
e mi ^chiedi un po' da ^bere.
Per il ^sole che ri^splende a mezzo^giorno ti ri^spondo.
Ma tu ^sai già dentro ^me la veri^tà.
\endverse
\beginverse
\chordsoff
Un ^cervo che cer^cava un sorso d'^acqua
nel giorno ^corse e ti tro^vò.
Anch'^io vò cer^cando nell'ar^sura sotto il ^sole,
e ^trovo dentro ^me la veri^tà.
\endverse
\endsong

%titolo{Adesso è la pienezza}
%autore{Ricci}
%album{Sei venuto dal cielo}
%tonalita{Si}
%gruppo{}
%momenti{Natale}
%identificatore{adesso_e_la_pienezza}
%data_revisione{2011_12_31}
%trascrittore{Francesco Endrici}
\beginsong{Adesso è la pienezza}[by={Ricci}]
\ifchorded
\beginverse*
\vspace*{-0.8\versesep}
{\nolyrics \[B]\[E]\[B]\[E]}
\vspace*{-\versesep}
\endverse
\fi
\beginverse
\memorize
\[B]Dopo il tempo del deserto
adesso è il \[E]tempo di pianure fertili.
\[B]Dopo il tempo delle nebbie
adesso s'\[E]apre l'orizzonte limpido.
\[C#m]Dopo il tempo dell'attesa
adesso è il \[G#m]canto la pienezza \[F#]della gioia: \[E]
l'Immacolata donna ha \[F#]dato al mondo Dio.
\endverse
\beginverse
^La fanciulla più nascosta
adesso è ^madre del Signore Altissimo.
^La fanciulla più soave
adesso il^lumina la terra e i secoli.
^La fanciulla del silenzio
adesso è il ^canto, la pienezza ^della gioia: ^
l’immacolata Donna ha ^dato al mondo Dio.
\endverse
\beginchorus
È \[G#m]nato, nato!
\[B]È qualcosa di impen\[F#]sabile , eppure è \[G#m]nato, nato!
\[B]Noi non siamo soli il Si\[F#]gnore ci è a fianco! \[C#m]
È \[E]nato, \[B]questa valle torne\[F#]rà come un giardino
il \[C#m]cuore già lo \[E]sa.
È \[B]nata la speranza, è \[F#]nata la speranza.
\endchorus
\beginverse
^La potenza del creato
adesso è il ^pianto di un bambino fragile.
^La potenza della gloria
adesso ^sta in una capanna povera.
^La potenza dell'amore
adesso è il ^canto, la pienezza ^della gioia: ^
l'Immacolata Donna ha ^dato al mondo Dio.
\endverse
\beginchorus
È \[G#m]nato, nato!
\[B]È qualcosa di impen\[F#]sabile , eppure è \[G#m]nato, nato!
\[B]Noi non siamo soli il Si\[F#]gnore ci è a fianco! \[C#m]
È \[E]nato, \[B]questa valle torne\[F#]rà come un giardino
il \[C#m]cuore già lo \[E]sa.
È \[B]nata la speranza, è \[F#]nata la speranza.
\endchorus
\beginverse
\[B]Tu adesso \[E]sei bimbo
\[B]tu adesso \[E]hai una \[C#m]madre
tu l'hai cre\[G#m]a\[F#]ta bel\[E]lissima
e \[F#4/7]dormi nel suo grembo\dots
\endverse
\beginchorus
È \[G#m]nato! \[B]\[F#]\[G#m] \[B]\[F#]\[C#m]
È \[E]nato! \[B]Questa valle torne\[F#]rà come un giardino
il \[C#m]cuore già lo \[E]sa.
È \[B]nata la speranza, è \[F#]nata la speranza.
\[B]\[E]\[B]\[E]\[B]
\endchorus
\endsong

%titolo{Adoriamo il Sacramento}
%autore{}
%album{}
%tonalita{Fa}
%gruppo{}
%momenti{Adorazione}
%identificatore{adoriamo_il_sacramento}
%data_revisione{2011_12_31}
%trascrittore{Francesco Endrici}
\beginsong{Adoriamo il Sacramento}
\beginverse
\[F]Ado\[C]riamo il \[B&]Sacra\[F]mento
\[F]che Dio \[B&]padre \[C]ci do\[F]nò.
\[F]Nuovo \[F]patto, \[B&]nuovo \[F]rito,
\[F]nella \[F]fede \[B&]si com\[C]pì.
\[F]Al mi\[C]stero è \[B&]fonda\[F]mento
\[F]la Pa\[B&]rola \[C]di Ge\[F]sù.
\endverse
\beginverse
^Gloria al ^Padre on^nipo^tente
^gloria al ^Figlio ^Reden^tor,
^lode ^grande, ^sommo o^nore
^all'e^terna ^Cari^tà.
^Gloria im^mensa, e^terno a^more
^alla ^Santa ^Trini^tà.
\endverse
\endsong

%titolo{Adoro te}
%autore{Branca}
%album{Parola d'amore}
%tonalita{Re}
%gruppo{}
%momenti{Adorazione}
%identificatore{adoro_te}
%data_revisione{2011_12_31}
%trascrittore{Francesco Endrici}
\beginsong{Adoro te}[by={Branca}]
%\category{Adorazione}

\beginverse
\[D] Sei qui davanti a me, \[A] o mio Signore
\[Bm] sei in questa brezza \[F#]che ristora il cuore.
\[G] Roveto che mai si \[D]consumerà,
\[C] presenza \[G] che riempie \[A4]l'ani\[A]ma
\endverse

\beginchorus
\[Dm] Adoro \[B&7+]{te,} \[C]fonte della \[F4]{vi}\[F]ta
\[Dm] Adoro \[B&7+]{te,} \[C]Trinità infi\[F4]{ni}\[F]ta.
\[Am] I miei cal\[Dm]zari leve\[Dm7]rò su \[B&7+]questo \[Am]santo \[Gm7]suolo, \[Gm6] 
alla pre\[Dm]senza tua mi \[E]{pro}\[A]{stre}\[B&]{rò.} \[(C]  \[D)]  
\endchorus

\beginverse
\chordsoff
Sei qui davanti a me, o mio Signore,
nella tua grazia trovo la mia gioia.
Io lodo, ringrazio e prego perché
il mondo ritorni a vivere in te.
\endverse

\beginchorus
\[B&]Mio Signor. 
O Si\[E&m]gnor, \[B7+] oh \[D&] \[G&4] \[G&] \[E&m] \[B7+] \[D&] \[G&4] \[G&] 
\[B&m] I miei cal\[E&m]zari leve\[E&m7]{rò su} \[B7+]questo \[E&7]santo \[A&m7]suolo, \[A&m6] 
alla pre\[E&m]senza tua mi \[F]{pro}\[B&]{stre}\[B]{rò,} \[D&4]{mio} \[D&]{Si}\[E&m]gnor.
\endchorus
\endsong


%titolo{Al di là di noi}
%autore{Meregalli}
%album{}
%tonalita{Do}
%gruppo{}
%momenti{Congedo;Missione}
%identificatore{al_di_la_di_noi}
%data_revisione{2011_12_31}
%trascrittore{Francesco Endrici}
\beginsong{Al di là di noi}[by={Meregalli}]

\beginverse
\[C]Al di \[G]là di \[C]noi c'è un \[G]mondo \brk che \[C]sogna \[G]tutti i \[C]mondi che \[C]può
\[C]al di \[G]là di \[C]noi c'è un \[G]mondo \brk che \[C]ancora \[G]sognare non \[C]può \[G] 
\[C]oriz\[F]zonti che \[C]presto sva\[G]niscono \brk \[C]cuori \[F]che non \[C]battono \[G]più
\[C]desi\[F]deri che \[C]piano si \[G]spengono \brk \[C]mani \[F]che non si \[G]aprono \[C]più.
\endverse

\beginchorus
\[C]Ecco per questo io vi \[F]mando andate per il \[G]mondo
entrate in ogni \[C]casa che vi apri\[C7]rà
\[F]portate a tutti il lieto an\[Dm]nuncio lo Spirito \brk del \[E]Padre vi \[Am]guide\[G]rà
\[C]ecco vi mando come a\[F]gnelli \brk a voi non manche\[G]ranno
parole di spe\[C]ranza e di veri\[C7]tà
\[F]in mezzo agli altri da fra\[Dm]telli \brk vi riconosce\[E]ranno \[Am]dall'\[E]{a}\[F]mo\[A7]re
\[D]io sarò \[G]sempre con \[C]voi, sarò \[F]sempre con \[C]voi, 
sarò \[D]sempre con \[G]voi
\endchorus

\beginverse
\chordsoff
Al di là di noi c'è un mondo \brk che sogna tutti i mondi che può
al di là di noi c'è un mondo \brk che mondo chiamarsi non può
strade dove le vite si perdono \brk case che non sorridono più
vite che con le armi si affrontano \brk volti che non ritornano più
\endverse

\beginverse
\chordsoff
Al di là di noi c'è un mondo \brk che sogna tutti i mondi che può
al di là di noi c'è un mondo \brk che ancora sognare non può.
\endverse
\endsong


%titolo{Altissimo}
%autore{Spoladore}
%album{Unanima}
%tonalita{Re}
%gruppo{}
%momenti{Comunione;San Francesco}
%identificatore{altissimo}
%data_revisione{2011_12_31}
%trascrittore{Francesco Endrici - Manuel Toniato}
\beginsong{Altissimo}[by={Spoladore}]
\ifchorded
\beginverse*
\vspace*{-0.8\versesep}
{\nolyrics \[D4/9] \[D] \[A] \[D4/9]}
\vspace*{-\versesep}
\endverse
\fi

\beginverse
\[D9]Altissimo, Onni\[Bm4/7]potente, Buon Si\[D9]gnore, \[Bm4/7] 
tue son le \[D9]lodi, la gloria, l'o\[Bm4/7]nore \brk e ogni benedi\[D9]zione, \[Bm4/7] 
che a Te \[D9]solo e al tuo Nome Al\[Bm4/7]tissimo \brk possiamo ele\[D9]vare,\[Bm4/7]  
e nessun \[D9]uomo può credersi \[Bm4/7]degno \brk di poterti no\[F#m4/7]minare.   \[Em7] 
\endverse

\beginverse
\chordsoff
Laudato sii, mi Signore con tutte le tue creature,
specialmente per frate sole così bello e radioso,
con la sua luce illumini il giorno ed illumini noi
e con grande splendore ci parla di Te Signore.
\endverse

\beginchorus
Lo\[D]date, 
bene\[A]dite il \[Bm7]Signore, ringra\[G]ziate e ser\[D]vite 
con \[A]grande umil\[Bm7]tà. 
Lo\[G]date, lo\[D]date, 
bene\[A]dite il Si\[Bm7]gnore, con \[G]grande umil\[D]tà,
ringra\[A]ziate e ser\[Bm4/7]vite, \brk con \[G]grande umil\[F#m4/7]{tà.} \[Em7] 
\endchorus

\beginverse
\chordsoff
Laudato sii, mi Signore, per sora luna e le stelle,
le hai formate nel cielo così chiare preziose e belle.
Per frate vento, per l'aria e il sereno \brk ed ogni tempo.
Così la Vita Tu cresci e sostieni in ogni tua creatura.
\endverse

\beginverse
\chordsoff
Laudato sii, mi Signore, per sora acqua così preziosa,
per frate fuoco giocoso e potente \brk che ci illumina la notte.
Laudato sii, mi Signore, per sora nostra \brk madre la terra,
ci sostiene, governa e ci dona fiori, frutti ed erba.
\endverse

\beginverse
\chordsoff
Laudato sii, mi Signore per quelli che \brk per il tuo Amore
perdonano e sopportano in pace \brk ogni persecuzione,
che sora morte ha trovato viventi \brk nella tua volontà,
da Te Altissimo un giorno saranno \brk da Te incoronati.
\endverse
\endsong

%titolo{Alto e glorioso Dio}
%autore{Frisina}
%album{Tu sei bellezza}
%tonalita{Do}
%gruppo{}
%momenti{Comunione;San Francesco}
%identificatore{alto_e_glorioso_dio}
%data_revisione{2011_12_31}
%trascrittore{Francesco Endrici - Manuel Toniato}
\beginsong{Alto e glorioso Dio}[by={Frisina}]

\beginchorus
\[C]Alto e \[Dm]glorioso \[C]Dio, \[F]{il}\[G]lumina \[Am]{il} cuore \[G]mio, \[G7]  
dammi \[C]fede \[Dm]retta, spe\[C]ranza \[F]certa, \brk ca\[G]rità per\[Am]fet\[G]ta.
\[C]Dammi \[Dm]umiltà pro\[C]fonda, \brk \[F]dammi \[G]senno \[Am]e cognosci\[G]mento, \[G7]  
che io \[C]possa \[Dm]sempre ser\[C]vire con \[F]gioia \brk i \[G]tuoi \[Am]comanda\[G]{men}\[C]ti.
\endchorus

\beginverse*
Ra\[Am]pisca ti \[Em]prego Si\[F]gno\[C]re, l'ar\[Dm]dente \brk e dolce \[C]forza del tuo \[G]amore,
la \[Am]mente mia da \[Em]tutte le \[F]{co}\[C]se, \brk per\[Dm]ché io muoia \[C]per amor \[G]tuo,
come \[Am]tu mo\[Em]risti \[F]per amor \[C]dell'amor \[Dm]mio. \[G] 
\endverse
\endsong


%titolo{Alza le braccia, apri il tuo cuore}
%autore{Branca}
%album{Parola d'amore}
%tonalita{Do}
%gruppo{}
%momenti{Congedo}
%identificatore{alza_braccia_apri_cuore}
%data_revisione{2011_12_31}
%trascrittore{Francesco Endrici - Manuel Toniato}
\beginsong{Alza le braccia, apri il tuo cuore}[by={Branca}]

\ifchorded
\beginverse*
\vspace*{-0.8\versesep}
{\nolyrics \[C4] \[C] \[C2] \[C] \[C4] \[C] \[C2] \[C] }
\vspace*{-\versesep}
\endverse
\fi

\beginverse
\[C4]Gri\[C]da la tua \[C2]voglia di pace, \brk \[C4]gri\[C]{da} la giu\[F6]stizia che \[G]vuoi.
\[C4]Scopri\[C]rai che da \[C2]sempre una voce \brk \[Am]grida più \[Dm7]forte di \[G4]{te.} \[G] 
\[C4]{Sen}\[C]ti, questa \[C2]voce ti \[G4]cerca, \brk \[C4]sen\[C]ti, ha bi\[F6]sogno di \[G]te.
\[Am7]Credi che nel pro\[Dm7]fondo del \[E7]cuore \brk \[Am]chi sta chia\[Dm7]mando è Ge\[G4]sù. \[G]
\endverse

\beginchorus
\[C]Alza le \[C4]braccia, \[C]apri \[C4]il tuo cuo\[C]re,
\[C/A]dona al Si\[C4/A]gnore \[C/A]splendi\[C4/A]{da} lo\[C/A]de.
\[F7+]Non dare \[F]spazio al\[F7+]la tri\[F]stez\[F7+]za, \brk ma \[G4]{can}\[G]{ta} \[G2]{glo}\[G]ria.
\[C]In ogni \[C4]cosa \[C]rendi \[C4]il tuo gra\[C]zie,
\[C/A]conti\[C4/A]nuamente \[C/A]{in}\[C4/A]voca il suo \[C/A]Nome.
\[F7+]Apri il tuo \[F]cuore, \[F7+]la forza \[F]del \[F7+]suo a\[G4]mo\[G]re \[G2] 
\[G]è gia in \[C4]{te.} \[C]  \[C2]  \[C]  \[C4]  \[C]  \[C2]  \[C] 
\endchorus

\beginverse
\chordsoff
Canta la tua voglia di gioia, canta la speranza che è in te.
Scoprirai che la voce di Cristo canta più forte che mai.
Credi, è parola di vita, credi, egli è via e verità.
Lascia che nel suo amore infinito trovi un amico anche in te.
\endverse
%da completare 
%\beginchorus
%Alza le braccia, apri il tuo cuore,
%dona al Signore splendida lode.
%non dare spazio alla tristezza, ma canta gloria.
%in ogni cosa rendi il tuo grazie,
%continuamente invoca il suo Nome.
%apri il tuo cuore, la forza del suo amore
%è gia in te. 
%\endchorus

%\beginverse* \itshape
%Alza le braccia, apri il tuo cuore,
%dona al Signore splendida lode.
%non dare spazio alla tristezza,
%e canta, gloria, gloria, gloria,
%canta gloria.
%\endverse
%\beginchorus
%In ogni cosa rendi il tuo grazie,
%continuamente invoca il suo nome.
%apri il tuo cuore, il suo amore è già in te.
%in ogni cosa rendi il tuo grazie,
%continuamente invoca il suo nome.
%canta con gioia, gloria. 
%\endchorus

\endsong


%titolo{Alza i tuoi occhi al cielo}
%autore{Branca, Ciancio}
%album{Sentieri di speranza}
%tonalita{Sol}
%gruppo{}
%momenti{Congedo;Pasqua}
%identificatore{alza_i_tuoi_occhi_al_cielo}
%data_revisione{2011_12_31}
%trascrittore{Francesco Endrici - Manuel Toniato}
\beginsong{Alza i tuoi occhi al cielo}[by={Branca, Ciancio}]
\beginverse
\[G]Cristo Gesù, \[C]Figlio di Dio, \brk \[G]ad ogni uomo il suo a\[D]more mostrò.
\[G]Egli guarì \[C]e liberò, \[G]tutto \[D]sé stesso do\[G]nò.
\[G]Sai che Gesù, \[C]se tu lo vuoi, \brk \[G]i suoi prodigi rin\[D]nova per te.
\[G]Senza timor \[C]apriti a lui, \brk \[G]ti ama co\[D]sì come sei, come \[C]sei.
\endverse

\beginchorus
\[G]Alza i tuoi occhi al \[C]cielo e vedrai, \brk \[G]nuovi orizzonti \[D]tu scoprirai.
\[Em]lascia ogni cosa \[C]e insieme a noi \brk \[E&]canta al \[F]Dio dell'a\[G]mor.
\[G]Apri il tuo cuore a \[C]Cristo e vedrai, \brk \[G]nella tristezza \[D]più non vivrai.
\[Em]lascia ogni cosa \[C]e insieme a noi \brk \[E&]canta al \[F]Dio dell'a\[G]mor.
\endchorus

\beginverse
\chordsoff
Cristo Gesù, Figlio di Dio, \brk ad ogni uomo il suo amore mostrò.
Egli morì, risuscitò, la croce sua ci salvò.
Sai che Gesù, se tu lo vuoi, \brk i suoi prodigi rinnova per te.
Senza timor apriti a lui, \brk ti ama così come sei, come sei.
\endverse

\beginverse
\chordsoff
Cristo Gesù, Figlio di Dio, \brk ai suoi amici la pace portò.
Al ciel salì ma vive in noi \brk con il suo soffio d'amor.
Sai che Gesù, se tu lo vuoi, \brk i suoi prodigi rinnova per te.
Senza timor apriti a lui, \brk ti ama così come sei, come sei.
\endverse
\endsong


%titolo{Alzati e risplendi}
%autore{Gocam}
%album{Su ali d'aquila}
%tonalita{Mi-}
%gruppo{}
%momenti{Salmi}
%identificatore{alzati_e_risplendi}
%data_revisione{2011_12_31}
%trascrittore{Francesco Endrici}
\beginsong{Alzati e risplendi}[by={Gocam}]
\beginverse
\[Em]Alzati e risplendi ecco \[B]la tua luce
\[B7]è su te la gloria del Si\[Em]gnor. \rep{2}
\[E]Volgi i tuoi \[E7]occhi e \[Am]guarda lontano
\[D]che il tuo cuore \[D7]palpiti di \[G]alle\[B7]gria.
\[Em]Ecco i tuoi figli che \[B]vengono a te,
\[B7]le tue figlie danzano di \[Em]gioia.
\endverse
\beginchorus
\[E7]Gerusalem, \[Am]Gerusalem, \[G]spogliati della tua tri\[B4]stez\[B7]za.
\[E7]Gerusalem, \[Am]Gerusalem, \[Em]canta e \[B7]danza al tuo Si\[Em]gnor.
\endchorus
\beginverse
^Marceranno i popoli al^la tua luce 
^ed i re vedranno il tuo splen^dor. \rep{2}
^Stuoli di cam^melli ti in^vaderanno
te^sori dal ^mare afflui^ranno a ^te.
^Verranno da Efa, da ^Saba e Kedar
^per lodare il nome del Si^gnor.
\endverse
\beginverse
^Figli di stranieri costrui^ranno le tue mura
^ed i loro re verranno a ^te. \rep{2}
^Io farò di ^te una ^fonte di gioia
^tu sarai chia^mata cit^tà del Si^gnore.
^Il dolore, il lutto ^finiranno,
^sarai la mia gioia fra le ^genti.
\endverse
\endsong

%titolo{Amare questa vita}
%autore{Meregalli}
%album{Mi hai tenuto per mano}
%tonalita{Fa}
%gruppo{}
%momenti{}
%identificatore{amare_questa_vita}
%data_revisione{2011_12_31}
%trascrittore{Francesco Endrici}
\beginsong{Amare questa vita}[by={Meregalli}]
\beginverse
\[F]Erano \[B&]uomini \[C]senza pa\[7]ura,
di \[Dm]solcare il \[7]mare pen\[Am]sando alla \[7]riva,
\[B&]barche sotto il \[C]cielo, \[F] tra montagne e si\[C]lenzio,
\[B&]davano le \[C]reti al \[F]ma\[Dm]re, \brk \[B&]vita dalle \[Gm]mani di \[C4]Dio \[C]
\endverse
\beginverse
\chordsoff
^Venne nell'^ora più ^lenta del ^giorno,
^quando le ^reti si ^sdraiano a ^riva,
^l'aria senza ^vento, ^ si riempì di una ^voce,
^mani cari^che di ^sa^le, ^sale nelle ^mani di ^Dio. ^
\endverse
\beginchorus
Lo se\[F]guimmo fi\[Gm7]dandoci degli \[F]occhi \[7]
gli cre\[B&]demmo a\[A7]mando le pa\[Dm]role. \[D7]
Fu il \[Gm]sole caldo a \[C7]riva
o fu il \[F]vento sulla \[C]vela
o il \[B&]gusto e la fa\[F]tica di ri\[Gm7]schiare
e accettare quella \[C]sfida.
\endchorus
\beginverse
^Prima che un ^sole più ^alto vi ^insidi, 
^prima che il ^giorno vi ^lasci de^lusi,
^riprendete il ^largo ^ e gettate le ^reti, 
^barche cari^che di ^pe^sci, \brk ^vita dalle ^mani di ^Dio. ^
\endverse
\beginchorus
Lo se\[F]guimmo fi\[Gm7]dandoci degli \[F]occhi \[7]
gli cre\[B&]demmo a\[A7]mando le pa\[Dm]role. \[D7]
Lui \[Gm]voce lui no\[C7]tizia
lui \[F]strada e lui sua \[C]meta
lui \[B&]gioia impreve\[F]dibile e sin\[Gm7]cera
di amare questa \[C]vita.
\endchorus
\beginverse
^Erano ^uomini ^senza pa^ura
di s^olcare il ^mare pen^sando alla ^riva,
^anche quella ^sera, ^ senza dire pa^role,
^misero le ^barche in ^ma^re, ^vita dalle ^mani di ^Dio, ^
\[B&]misero le \[C]barche in \[F]ma\[Dm]re, \brk \[B&]vita dalle \[C]mani di \[F]Dio.
\endverse
\endsong

%titolo{Amatevi fratelli}
%autore{Machetta}
%album{Grazie, Signore}
%tonalita{Fa}
%gruppo{}
%momenti{}
%identificatore{amatevi_fratelli}
%data_revisione{2011_12_31}
%trascrittore{Francesco Endrici}
\beginsong{Amatevi fratelli}[by={Machetta}]
\beginverse
A\[F]mate\[C]vi, fra\[B&]tel\[F]li, come \[Dm]Io ho a\[G]mato \[C7]voi!
A\[Am]vrete \[F] la mia \[Dm]gioia, \[B&] \brk che nes\[Gm]suno vi \[C7]toglie\[F]rà.
A\[Am]vremo \[F] la sua \[Dm]gioia, \[B&] \brk che nes\[Gm]suno ci \[C7]toglie\[F]rà.
\endverse
\beginverse
\chordsoff
Vi^vrete in^sieme u^ni^ti come il ^Padre è u^nito a ^Me!
A^vrete ^ la mia ^vita, ^ se l'a^more sa^rà con ^voi!
A^vremo ^ la sua ^vita, ^ se l'a^more sa^rà con ^noi!
\endverse
\beginverse
\chordsoff
Vi ^dico ^queste pa^ro^le, perché ab^biate in ^voi la ^gioia, 
sa^rete ^ miei a^mici ^ se l'a^more sa^rà con ^voi!
Sa^remo ^ suoi a^mici ^ se l'a^more sa^rà con ^noi!
\endverse
\endsong

%titolo{Amo}
%autore{Fioravanti}
%album{Sempre canterò}
%tonalita{}
%gruppo{}
%momenti{}
%identificatore{amo_fioravanti}
%data_revisione{2011_12_31}
%trascrittore{Francesco Endrici}
\beginsong{Amo}[by={Fioravanti}]
\beginchorus
\[D]Amo il Signore per\[A]ché ascolta
il \[G]grido \[D]della mia pre\[A]ghiera.
\[D]Su di me ha steso la \[A]mano
nel \[G]giorno \[D]che lo cer\[A]cavo. \[G]
\endchorus
\beginverse
Ho invocato il \[D]nome del Signore \[A]
ed egli mi ha ri\[D7]sposto. \[G]
Buono e giusto \[D]è il nostro Dio: \[F#]
protegge gli umili e gli op\[Bm]pres\[A]si.
\endverse
\beginverse
%\chordsoff
Anima mia ^torna alla tua pace: ^
il Signore ti ha ascol^tato. ^
Ha liberato i tuoi ^occhi dalle lacrime ^
e non sono più ca^du^to.
\endverse
\beginverse
%\chordsoff
Ho creduto anche ^quando dicevo: ^
sono troppo infe^lice. ^
Che cosa posso ^rendere al Signore ^
per quello che mi ha ^da^to?
\endverse
\beginverse
%\chordsoff
Il calice della sal^vezza innalzerò, ^
invocherò il nome ^tuo, Signore. ^
Tu lo sai, ^sono il tuo servo: ^
a te offrirò sacri^fi^ci.
\endverse
\endsong



%titolo{Andate per le strade}
%autore{Roncari, Capello}
%album{Tener viva la speranza}
%tonalita{Si-}
%gruppo{}
%momenti{Cangedo;Missione}
%identificatore{andate_per_le_strade}
%data_revisione{2011_12_31}
%trascrittore{Francesco Endrici}
\beginsong{Andate per le strade}[by={Roncari, Capello}, ititle={Invitati alla festa di Dio}]
\beginchorus
\[Bm] An\[Bm]date per le \[D]strade in \[G]tutto il \[A]mondo,
chia\[F#]mate i miei a\[Bm]mici \[A]per far \[D]festa,
c'è un \[Bm]posto per cia\[G]scuno \[A]alla mia \[Bm]mensa.
\endchorus
\beginverse 
Nel \[D]vostro cam\[G]mino annun\[A]ciate il Van\[D]gelo 
di\[Bm]cendo “È vi\[Em]cino il \[F#]Regno dei \[Bm]cieli!”.
Gua\[D7]rite i ma\[G]lati, mon\[A]date i leb\[D]brosi,
ren\[Bm]dete la \[F#m]vita a \[C#]chi l'ha per\[F#]duta.
\endverse
\beginverse
\chordsoff
^Vi è stato do^nato con a^more gra^tuito: 
u^gualmente do^nate con ^gioia e per a^more. 
Con ^voi non pren^dete ^né oro né ar^gento, 
per^ché l'ope^raio ha di^ritto al suo ^cibo.
\endverse
\beginverse
\chordsoff
En^trando in una ^casa do^natele la ^pace: 
se ^c'è chi vi ri^fiuta e ^non accoglie il ^dono, 
la ^pace torni a ^voi e u^scite dalla ^casa, 
^scuotendo la ^polvere dai ^vostri cal^zari.
\endverse
\beginverse
\chordsoff
^Ecco, io vi ^mando a^gnelli in mezzo ai ^lupi: 
siate ^dunque avve^duti come ^sono i ser^penti, 
ma ^liberi e ^chiari co^me le co^lombe; 
do^vrete soppor^tare pri^gioni e tribu^nali.
\endverse
\beginverse
\chordsoff
Nes^suno è più ^grande del ^proprio ma^estro, 
né il ^servo è più im^portante ^del suo pa^drone. 
Se ^hanno odiato ^me, odie^ranno anche ^voi; 
ma ^voi non te^mete, io non ^vi lascio ^soli!
\endverse
\endsong

%titolo{Annunceremo che Tu}
%autore{Auricchio}
%album{Cammina con noi Maria}
%tonalita{Sol}
%gruppo{}
%momenti{Congedo;Missione}
%identificatore{annunceremo_che_tu}
%data_revisione{2011_12_31}
%trascrittore{Francesco Endrici}
\beginsong{Annunceremo che Tu}[by={Auricchio}]
\transpose{0}
\beginchorus
Annunceremo che \[G]Tu sei Veri\[D]tà, 
lo grideremo dai \[G]tetti della nostra cit\[D]tà, 
senza paura \[Am]anche tu \[C] lo puoi can\[D]tare.
\endchorus
\beginverse
\[G] E non te\[C]mere dai, \[G] \brk che non ci \[D]vuole poi \[C]tanto, \[G]
quello che \[C]non si sa \[G] \brk non reste\[D]rà na\[C]scosto. \[Bm]
Se ti parlo nel \[C]buio, lo dirai nella \[D]luce,
ogni giorno è il mo\[C]mento di credere in \[D]me.
\endverse
\beginverse
\chordsoff
^ Con il co^raggio tu ^ porterai la Pa^rola che ^salva,  ^
anche se ^ci sarà ^ chi non vuole ac^cogliere il ^dono,  ^
tu non devi fer^marti, ma continua a lot^tare, 
il mio Spirito ^sempre ti accompagne^rà.
\endverse
\beginverse
\chordsoff
^ Non ti abban^dono mai, ^ io sono il ^Dio fe^dele, ^
conosco il ^cuore tuo. ^ \brk Ogni tuo pen^siero mi è ^noto, ^ 
la tua vita è pre^ziosa, vale più di ogni ^cosa,
il segno più ^grande del mio amore per ^te.
\endverse
\endsong

%titolo{Antica, eterna danza}
%autore{Gen Verde}
%album{Cerco il tuo volto}
%tonalita{Sol}
%gruppo{}
%momenti{Offertorio}
%identificatore{antica_eterna_danza}
%data_revisione{2011_12_31}
%trascrittore{Francesco Endrici - Manuel Toniato}
\beginsong{Antica, eterna danza}[by={Gen\ Verde}]

\beginverse
\[G]Spighe \[D]d'oro al \[Em]vento, antica, e\[D]terna \[C]danza
per \[Am]fare un \[D]solo \[Em]pa\[D]ne spez\[Am]zato \[A]sulla \[B]mensa.
\[G]Grappoli \[D]dei \[Em]colli, profumo \[D]di le\[C]tizia
per \[Am]fare un \[D]solo \[Em]vi\[D]no be\[Am]vanda \[Bm]della \[Em]grazia.
\endverse

\beginverse
\chordsoff
Con il pane e il vino Signore ti doniamo
le nostre gioie pure, le attese e le paure.
Frutti del lavoro e fede nel futuro,
la voglia di cambiare e di ricominciare.
\endverse

\beginverse
\chordsoff
Dio della speranza, sorgente d'ogni dono
accogli questa offerta che insieme Ti portiamo.
Dio dell'universo raccogli chi è disperso
e facci tutti Chiesa, una cosa in Te.
\endverse
\endsong

%titolo{Apostoli di gioia}
%autore{}
%album{Inno Del Giubileo Dei Ragazzi}
%tonalita{Sol}
%gruppo{}
%momenti{Congedo}
%identificatore{apostoli_di_gioia}
%data_revisione{2011_12_31}
%trascrittore{Francesco Endrici - Manuel Toniato}
\beginsong{Apostoli di gioia}[by={}]


\beginchorus
\[G] Apostoli di gioia \[C] aposto\[D]li d'amore
\[G]cantiamo insieme allelu\[D]{ia} \[D7] 
\[G] apriamo i nostri cuori \[C] ad una \[D]vita nuova
\[G]tutti fratelli dell'u\[D]{ma}\[D7]ni\[G]tà
\endchorus

\beginverse
\[G] Noi siamo il sorriso \[D/F#] \brk che porta la speranza
\[Em] a chi la pace più non \[D]ha
\[C] e se il futuro è incerto \[Bm] \brk tu ci ter\[Em]rai per mano
\[C] cammine\[Am]remo insieme a \[D]te. \[D7] 
\endverse

\beginverse
\chordsoff
Tu sei la vera luce che illumina la strada
ai nostri passi stanchi ormai
nel buio della notte nel sogno di ogni uomo
per sempre in noi risplenderà.
\endverse

\beginverse
\chordsoff
La festa dell'incontro è festa dell'amore
trionfo della tua bontà
resta con noi Signore e non ci abbandonare
vogliamo vivere con te. 
\endverse
\endsong

%titolo{Apri la tua mano}
%autore{Ricci}
%album{La Tua dimora}
%tonalita{Re-}
%gruppo{}
%momenti{Salmi}
%identificatore{apri_la_tua_mano}
%data_revisione{2011_12_31}
%trascrittore{Francesco Endrici}
\beginsong{Apri la tua mano}[by={Ricci}]
\ifchorded
\beginverse*
\vspace*{-0.8\versesep}
{\nolyrics \[Dm]\[C]\[G]}
\vspace*{-\versesep}
\endverse
\fi
\beginchorus
\[Dm]Apri la tua mano, Si\[C]gnore, sazia \[G]ogni vi\[Dm]vente.
\[Dm]Apri la tua mano, Si\[C]gnore, sazia \[G]ogni vi\[Dm]vente. \[Dm]
\endchorus
\beginverse
\memorize
Pa\[Dm]ziente e misericor\[Dm]dioso è il Si\[Dm]gnore, e lento all'\[Dm]ira,
\[Dm]ricco di grazia, \[F]buono è il Si\[C]gnore verso \[Dm]tutti.
\[C]La sua tenerezza si e\[C]spande su \[G]tutte le crea\[G]ture.
\endverse
\beginverse
\chordsoff
Gli ^occhi di tutti sono rivolti a te in attesa,
tu provvedi loro il cibo a suo tempo. 
Tu apri la tua mano e sazi la fame d'ogni vivente.
\endverse
\beginverse
\chordsoff
Giusto è il Signore in tutte le sue vie, santo in
tutte le sue opere. Il signore è vicino a quanti
lo invocano, a quanti lo cercano con cuore sincero.
\endverse
\endsong


%titolo{Apri le tue braccia}
%autore{Machetta}
%album{Una voce che ti cerca}
%tonalita{Re-}
%gruppo{}
%momenti{Ingresso;Quaresima;Riconciliazione;Conversione}
%identificatore{apri_le_tue_braccia}
%data_revisione{2011_12_31}
%trascrittore{Francesco Endrici}
\beginsong{Apri le tue braccia}[by={Machetta}]
\beginverse
Hai cer\[Dm]cato la \[C]libertà lon\[Dm]tano,
hai tro\[Am]vato la \[E&]noia e le ca\[B&]tene;
hai va\[Gm]ga\[Dm]to \[Gm]senza \[Dm]via, \[Gm]solo, \[E&] con la tua \[C]fame.
\endverse
\beginchorus
\[F]A\[C]pri le tue \[Dm]brac\[Am]cia, \[B&]corri in\[Gm]contro al \[C]Padre;
\[F]oggi \[D] la sua \[Gm]ca\[B&]sa sarà in \[F]fe\[C7]sta per \[F]te.
\endchorus
\beginverse
\chordsoff
Se vor^rai spez^zare le ca^tene
trove^rai la ^strada dell'a^more;
la tua ^gio^ia ^cante^rai: ^questa ^ è liber^tà.
\endverse
\beginverse
\chordsoff
I tuoi ^occhi ri^cercano l'az^zurro;
c'è una ^casa che a^spetta il tuo ri^torno,
e la ^pa^ce ^torne^rà: ^questa ^ è liber^tà.
\endverse
\endsong

%titolo{Arderanno sempre i nostri cuori}
%autore{Auricchio}
%album{Alleluia è risorto}
%tonalita{Do}
%gruppo{}
%momenti{Pasqua;Comunione;Congedo}
%identificatore{arderanno_sempre_i_nostri_cuori}
%data_revisione{2011_12_31}
%trascrittore{Francesco Endrici}
\beginsong{Arderanno sempre i nostri cuori}[by={Auricchio}]
\ifchorded
\beginverse*
\vspace*{-0.8\versesep}
\[\vline]\[C]\[\vline]\[G]\[\vline]\[Dm]\[\vline]\[C]\[G]\[\vline]
\[\vline]\[C]\[\vline]\[G]\[\vline]\[Dm]\[\vline]\[C]\[G]\[\vline]
\vspace*{-\versesep}
\endverse
\fi
\beginverse
\memorize
\[C]Quando scende su di noi la \[G]sera e scopri che 
nel cuore \[Am]resta nostal\[F]gia
di un giorno che \[C]non avrà tramonto
ed a\[G]vrà il colore della sua \[\vline]\[F]pace. \[Am]\[G]\[\vline]
\endverse
\beginverse
^Quando scende su di noi il ^buio e senti che
nel cuore ^manca l'alle^gria
del tempo che ^non avrà mai fine
ed al^lora cercherai parole \[\vline]\[F]nuove. \[G]\[\vline] \[Am]
\endverse
\beginverse
E all'improvviso la \[Em7]strada si illumina
e \[F]scopri che non sei più \[\vline]\[C]solo; \[G]\[\vline] \[Am]
sarà il Signore ri\[Em7]sorto a tracciare il cam\[F]mino 
e a ridare la \[\vline]\[Dm]vi\[G]\[\vline]ta. \[C]
\endverse
\beginchorus
Arderanno sempre i nostri \[G]cuori \[Dm]
se la tua Parola in noi di\[\vline]\[C]more\[G]\[\vline]rà \[C]
spezza tu, Signore, questo \[G]pane: \[Dm]
porteremo al mondo la tua \[\vline]\[C]veri\[G]\[\vline]tà.
\endchorus

\ifchorded
\beginverse*
\vspace*{-0.8\versesep}
\[\vline]\[C]\[\vline]\[G]\[\vline]\[Dm]\[\vline]\[C]\[G]\[\vline]
\vspace*{-\versesep}
\endverse
\fi
\beginverse
^Quando all'alba sentirai la sua ^voce capirai
che non po^trà fermarsi ^mai
l'annuncio che ^non avrà confini
che ri^porterà nel mondo la spe\[\vline]\[F]ranza. \[G]\[\vline] \[Am]
\endverse
\beginverse
Gesù è il Signore ri\[Em7]sorto che vive nel \[F]tempo,
è presente tra gli \[\vline]\[C]uomi\[G]\[\vline]ni; \[Am]
è lui la vita del \[Em7]mondo,
il pane che \[F]nutre la Chiesa in cam\[\vline]\[Dm]mi\[G]\[\vline]no. \[C]
\endverse
\beginchorus
Arderanno sempre i nostri \[G]cuori \[Dm]
se la tua Parola in noi di\[\vline]\[C]more\[G]\[\vline]rà \[C]
spezza tu, Signore, questo \[G]pane: \[Dm]
porteremo al mondo la tua \[\vline]\[C]veri\[G]\[\vline]tà. \iflyric \rep{2}\fi
\endchorus
\ifchorded
\beginverse*
\vspace*{-\versesep}
\[\vline]\[Dm]\[F]\[\vline]\[C]\[G]\[\vline]\[Dm]\[F]\[\vline]\[C]\[G]\[F]\[G]\[\vline]\[C]\[\vline]
\endverse

\beginchorus
\chordsoff 
Rit. 
\endchorus
\beginverse*
\vspace*{-\versesep}
\[\vline]\[C]\[\vline]\[G]\[\vline]\[Dm]\[F]\[\vline]\[G4]\[C]\[\vline]
\endverse
\fi
\endsong


%titolo{Astro del ciel}
%autore{Gruber}
%album{}
%tonalita{La}
%gruppo{}
%momenti{Natale}
%identificatore{astro_del_ciel}
%data_revisione{2011_12_31}
%trascrittore{Francesco Endrici - Manuel Toniato}
\beginsong{Astro del ciel}[by={Gruber}]
\beginverse
\[A]Astro del ciel, Pargol divin, \[E]mite agnello \[A]redentor!
\[D]Tu che i Vati da \[A]lungi sognar, \brk \[D]tu che angeliche \[A]voci annunziar,
\endverse

\beginchorus
\[E]Luce dona alle \[A]men\[F#m]ti, \brk \[A]pace in\[E]fondi nei \[A]cuor. \rep{2}
\endchorus

\beginverse
\chordsoff
Astro del ciel, Pargol divin, mite agnello redentor,
tu di stirpe regale decor, \brk tu virgineo mistico fior.
\endverse

\beginverse
\chordsoff
Astro del ciel, Pargol divin, mite agnello redentor
tu disceso a scontare l'error, \brk tu sol nato a parlare d'amor.
\endverse
\endsong



%titolo{Ave Maria}
%autore{Casucci, Balduzzi}
%album{Verbum panis}
%tonalita{Re}
%gruppo{}
%momenti{Maria}
%identificatore{ave_maria_casucci}
%data_revisione{2011_12_31}
%trascrittore{Francesco Endrici}
\beginsong{Ave Maria}[by={Casucci, Balduzzi}]
\ifchorded
\beginverse*
\vspace*{-0.8\versesep}
\[D]\[A]\[Bm]\[G]\[D]\[A]\[Em] \[\vline]\[G]\[A]\[\vline]
\vspace*{-\versesep}
\endverse
\fi
\beginchorus
\[D]A\[A]ve Ma\[Bm]ria, \[G]\[D]\[A]a\[Em]ve, \[G]
\[D]a\[A]ve Ma\[Bm]ria, \[G]\[D]\[A]a\[D4]ve. \[D]
\endchorus
\beginverse
\memorize
\[D]Donna dell'at\[D]tesa e \[Bm]madre di spe\[Bm]ranza
\[A]ora pro no\[G]bis.
\[D]Donna del sor\[D]riso e \[Bm]madre del si\[Bm]lenzio
\[A]ora pro no\[G]bis.
\[D]Donna di fron\[D]tiera e \[A]madre dell'ar\[A]dore
\[Bm]ora pro no\[G]bis.
\[D]Donna del ri\[D]poso e \[A]madre del sen\[A]tiero
\[G]ora pro no\[A]bis.
\endverse
\beginchorus
\chordsoff 
Rit. 
\endchorus
\beginverse
^Donna del de^serto e ^madre del re^spiro
^ora pro no^bis.
^Donna della ^sera e ^madre del ri^cordo
^ora pro no^bis.
^Donna del pre^sente e ^madre del ri^torno
^ora pro no^bis.
^Donna della ^terra e ^madre dell'a^more
^ora pro no^bis.
\endverse
\beginchorus
\chordsoff 
Rit. 
\endchorus
\endsong




%titolo{Ave Maria}
%autore{Sequeri}
%album{Eppure Tu sei qui}
%tonalita{Sol}
%gruppo{}
%momenti{Maria}
%identificatore{ave_maria_sequeri}
%data_revisione{2011_12_31}
%trascrittore{Francesco Endrici - Manuel Toniato}
\beginsong{Ave Maria}[by={Sequeri}]

\beginverse*
\[G]Ave Ma\[C]ria \[Am]piena di \[Em]grazia 
\[G]il Si\[C]gnore \[D]è con \[G]te
\[E]tu sei bene\[Am]detta fra \[D]tutte le \[G]donne
\[B]e bene\[Em]detto è il \[Am]Figlio tuo Ge\[D]sù.
\[G]Santa Ma\[C]ria \[Am]madre di \[Em]Dio
\[G]prega per \[C]noi \[D]pecca\[G]tori
a\[E]desso e nel\[Am]l'ora \[D]della nostra \[G]morte
a\[B]desso e nel\[Em]l'ora \[Am]della nostra \[D]morte.
\[C]{A}\[G]men.
\endverse
\endsong



\lettera
%titolo{Beati quelli che ascoltano}
%autore{Bonfitto}
%album{Sei grande nell'amore}
%tonalita{Mi-}
%gruppo{}
%momenti{Parola}
%identificatore{beati_quelli_che_ascoltano}
%data_revisione{2011_12_31}
%trascrittore{Francesco Endrici - Manuel Toniato}
\beginsong{Beati quelli che ascoltano}[by={Bonfitto}]

\ifchorded
\beginverse*
\vspace*{-0.8\versesep}
{\nolyrics \[Em] \[B7] \[Em]}
\vspace*{-\versesep}
\endverse
\fi

\beginchorus
\[Em] Beati \[B7]quelli che a\[Em]scoltano \brk \[Am] la pa\[D7]rola di \[G]Dio \[Em] 
e la \[B]vivono \[B7] ogni \[Em]gior\[Am]no. \[Em] 
\endchorus

\beginverse
\[Em]La tua parola ha creato l'uni\[Am]verso,
\[B7]tutta la terra ci parla di Te, o Si\[Em]gnore.
\endverse

\beginverse
\chordsoff
La tua parola si è fatta uno di noi,
mostraci il tuo volto Signore.
\endverse
\endsong



%titolo{Beati quelli}
%autore{Sequeri}
%album{Qui dove tu ci chiami}
%tonalita{Fa}
%gruppo{}
%momenti{}
%identificatore{beati_quelli_sequeri}
%data_revisione{2011_12_31}
%trascrittore{Francesco Endrici}
\beginsong{Beati quelli}[by={Sequeri}]
\ifchorded
\beginverse*
\vspace*{-0.8\versesep}
{\nolyrics \[F]\[B&]\[C7]}
\vspace*{-\versesep}
\endverse
\fi
\beginverse
\memorize
Beati \[F]quelli che \[B&]poveri \[F]sono, \[A7]
beati \[Dm]quanti son \[C7]puri di \[F]cuore. \[C7]
Beati \[F]quelli che \[B&]vivono in \[F]pena \[A7]
nell'at\[Dm]tesa d'un \[Gm]nuovo mat\[C7]tino.
\endverse
\beginchorus
Saran be\[F]ati, vi \[B&]dico, be\[F]\[A7]a\[Dm]ti
perché di \[B&]essi è il \[B&m]regno dei \[C7]cieli.
Saran be\[F]ati, vi \[B&]dico, be\[F]\[A7]a\[Dm]ti
perché di \[B&]essi è il \[G7]regno dei \[F]\[C9]cie\[F]li.
\endchorus
\beginverse
\chordsoff
Beati ^quelli che ^fanno la ^pace, ^
beati ^quelli che in^segnano l'a^more. ^
Beati ^quelli che ^hanno la ^fame ^
e la ^sete di ^vera giu^stizia.
\endverse
\beginverse
\chordsoff
Beati ^quelli che un ^giorno sa^ranno ^
persegui^tati per ^causa di ^Cristo, ^
perché nel ^cuore non ^hanno vio^lenza, ^
ma la ^forza di ^questo Van^gelo.
\endverse
\endsong

%titolo{Beati voi}
%autore{Cento}
%album{Guarda laggiù l'orizzonte}
%tonalita{Mi}
%gruppo{}
%momenti{}
%identificatore{beati_voi_cento}
%data_revisione{2011_12_31}
%trascrittore{Francesco Endrici}
\beginsong{Beati voi}[by={Cento}, ititle={Beati}]
\beginchorus
\[E]Beati \[A]voi, \[F#m] \[B]beati \[E]voi, \[C#m]beati \[F#m]voi, be\[A]\[Am]a\[E]ti!
\endchorus
\beginverse
Se un \[E]uomo vive \[G#m]oggi nella \[C#m]vera pover\[7]tà,
il \[A]regno del Si\[F#m]gnore dentro \[B]Lui presente è \[7]già.
Per \[E]voi che siete \[G#m]tristi e senza \[C#m]senso nella \[7]vita
c'è un \[A]Dio che può do\[F#m]narvi \brk una spe\[B]ranza nel do\[B7]lor.
\endverse
\beginverse
\chordsoff
Voi ^che lottate ^senza vio^lenza e per a^more
pos^siederete un ^giorno questa ^terra, dice ^Dio.
Voi ^che deside^rate ciò che ^Dio vuole per ^noi
un ^infinito all'^alba piove^rà dentro di ^voi.
\endverse
\beginverse
\chordsoff
E ^quando nel tuo ^cuore nasce ^tanta compas^sione
è ^Dio che si com^muove come un ^bimbo dentro ^te.
Be^ati quelli ^che nel loro ^cuore sono ^puri
già ^vedono il Si^gnore total^mente anche quag^giù.
\endverse
\beginverse
\chordsoff
Be^ato chi dif^fonde pace ^vera intorno a ^sé
il ^Padre che è nei ^cieli già lo ^chiama figlio ^mio.
Chi ^soffre per a^more e sa mo^rire oggi per ^lui
ri^ceve il regno ^subito e la ^vita eterna ha in ^sé.
\endverse
\beginverse
\chordsoff
Se ^poi diranno ^male perché ^siete amici ^suoi
sap^piate che l'han ^fatto già con ^Lui prima di ^voi.
Se ^poi diranno ^male perché ^siete amici ^suoi
sap^piate che l'han ^fatto già con ^Lui prima di ^voi.
\endverse
\beginchorus
\chordsoff
\[E]Siate fe\[A]lici, \[F#m] \[B]siate fe\[E]lici, \[C#m]siate fe\[F#m]lici. \[A]\[Am]A\[E]men!
\endchorus
\endsong

%titolo{Beati voi}
%autore{Stacchiotti}
%album{Cristo è risorto}
%tonalita{Re}
%gruppo{}
%momenti{}
%identificatore{beati_voi_stacchiotti}
%data_revisione{2011_12_31}
%trascrittore{Francesco Endrici}
\beginsong{Beati voi}[by={Stacchiotti}]
\ifchorded
\beginverse*
\vspace*{-0.8\versesep}
{\nolyrics \[Bm]\[A]\[D]\[G]\[D]\[Em4/7]\[A7]\[A7]}
\vspace*{-\versesep}
\endverse
\fi
\beginverse*
Be\[D]ati i \[A]poveri in \[G]spirito, \[G]
per\[G]ché di essi è il \[A]regno dei \[D]cieli. \[D] 
Be\[D]a\[A]ti gli a\ch{G}{f}{fl}{ffl}itti, \[G]
per\[G]ché saranno \[A]conso\[D]lati \[D]
Bea\[A]ti i \[G]miti, \[G]
\[G]eredite\[A]ranno la \[D]terra \[D]
Be\[Bm]ati quelli che hanno \[A]fame e sete di giu\[G]stizia \[G]
per\[G]ché sa\[A]ranno sa\[D]ziati. \[D]
B\[D]eati i \[A]misericor\[G]diosi
\[G]troveranno \[A]miseri\[D]cordia. \[D]
Be\[D]ati i \[A]puri di \[G]cuore, \[G]
per\[G]ché ve\[A]dranno \[D]Dio. \[D]
Be\[D]ati gli o\[A]peratori di \[G]pace, \[G]
sa\[G]ran chiamati \[A]figli di \[D]Dio. \[D]
Be\[Bm]ati i persegui\[A]tati per la giu\[G]stizia \[G]
perché di \[C]essi è il \[Am]regno dei \[D]cie\[D]li.

Beati \[G]voi \[D]quando vi insulte\[Em]ranno
e men\[C]tendo di\[G]ranno ogni \[Em]sorta di \[D]male contro \[C]voi \[C]
per causa \[G]mia, \[G] per causa \[D]mia. \[D]
Ma \[C]voi \[C] ralle\[G]gratevi ed esul\[D]tate,
perché \[Em]grande è la \[C]vostra ricom\[G]pensa nei \[D]cie\[C]li \rep{2}
\[G]\[D]\[C]\[D]\[G]
\endverse
\endsong

%titolo{Beatitudini}
%autore{Castiglia}
%album{}
%tonalita{Do}
%gruppo{}
%momenti{}
%identificatore{beatitudini_castiglia}
%data_revisione{2011_12_31}
%trascrittore{Francesco Endrici}
\beginsong{Beatitudini}[by={Castiglia}, ititle={La preghiera di Gesù è la nostra}]
\beginverse
\[C]Dove due o \[Em]tre sono riu\[Am]niti nel mio \[F]nome, \[G]
io sarò con \[C]loro, \[F] pregherò con \[G]loro, \[E]
amerò con \[Am]loro perchè il mondo venga a \[F]Te,
o \[G]Padre, co\[C]noscere il tuo a\[Am]more,
avere \[F]vita \[G] in \[C]Te. \[G]
\endverse
\beginverse
%\chordsoff
^Voi che siete ^luce della ^terra miei a^mici ^
risplendete s^empre ^ della vera ^luce ^
perché il mondo ^creda nel amore che c'è in ^voi
o ^Padre con^sacrali per ^sempre
diano ^gloria ^ a ^te. ^
\endverse
\beginverse
%\chordsoff
^Ogni beati^tudine vi at^tende nel mio ^giorno, ^
se sarete u^niti, ^ se sarete ^pace, ^
se sarete ^puri perché voi vedrete ^Dio
che è ^Padre in ^Lui la vostra ^vita
gioia ^piena ^ sa^rà! ^
\endverse
\beginverse
^Voi che ora ^siete miei di^scepoli nel ^mondo, ^
siate testi^moni ^ di un amore im^menso, ^
date prova ^di quella speranza che c'è in ^voi
Co^raggio! Vi ^guiderò per ^sempre,
io ri^mango ^ con ^voi. \[A]
\endverse
\beginverse
\transpose{2}
^Spirito che ^animi la ^Chiesa e la rin^novi, ^
donale for^tezza, ^ fa' che sia fe^dele ^
come Cristo ^che muore e risorge, perchè il ^Regno \brk del ^Padre si ^compia in mezzo a ^noi:
abbiamo ^vita ^ in ^Lui. \[Am]
Si \[C]compia in mezzo a \[Am]noi abbiamo \[F]vita \[G] in \[F]lui. \[C]
\endverse
\endsong

%titolo{Benedetta sei tu}
%autore{Spoladore}
%album{Unanima}
%tonalita{Re}
%gruppo{}
%momenti{Maria}
%identificatore{benedetta_sei_tu}
%data_revisione{2011_12_31}
%trascrittore{Francesco Endrici - Manuel Toniato}
\beginsong{Benedetta sei tu}[by={Spoladore}]

\beginverse
\[D9] Benedetta sei \[A4/7]Tu o Maria, 
\[Em7] benedetta tu \[D9]Madre di Dio
\[D9] tra le braccia hai \[A4/7]stretto Colui
\[Em7] che ha creato l'im\[D9]menso universo

\[D9] Ti preghiamo dol\[A4/7]cissima Madre
\[Em7] ora stringi al \[D9]petto i tuoi figli
\[D9] dona pace e nel \[A4/7]buio conforto
\[Em7] Tu raccoglici \[D9]stretti al tuo cuore
\endverse

\beginchorus
\[B]Ave Ma\[A]ria \[E]Ave Ma\[B]ria, \[B]Ave Ma\[A]ria \[E]Ave Ma\[B]ria
Re\[A]gina del cielo, re\[E]gina del mondo,
re\[Em]gina di pace sei \[F#4]{tu} \[F#] 
\endchorus

\beginverse
\chordsoff
Benedetta sei tu o Maria, 
tu che hai dato alla luce il Signore
fai rinascere nuovo il mondo
come una una sola famiglia.

Tu sostegno di Dio contro il male
intercedi il perdono per noi
nella calma profonda del cuore
fai rinascere il nostro cammino.
\endverse
\endsong

%titolo{Benedetto sei Tu}
%autore{Gen Verde}
%album{Gen Verde}
%tonalita{Re}
%gruppo{}
%momenti{Offertorio}
%identificatore{benedetto_sei_tu_gen}
%data_revisione{2011_12_31}
%trascrittore{Francesco Endrici}
\beginsong{Benedetto sei Tu}[by={Gen\ Verde}]
\beginverse
\[A] Bene\[D]det\[Em]to sei \[F#m]Tu, \[G]Dio dell'\[D]uni\[Bm]verso
\[G]dalla \[A]tua bon\[D]tà \[Bm]
ab\[G]biamo \[A]ricevuto \[D]questo \[Bm]pane
\[G]frutto della \[A]terra e del \[D]nostro la\[Bm]voro;
lo \[G]presentiamo a \[F#m]Te,
per\[Em]ché diventi per \[D]noi \[Em]cibo di \[G]vita e\[A]terna.
\endverse
\beginverse
^ Bene^det^to sei ^Tu, ^Dio dell'^uni^verso
^dalla ^tua bon^tà ^ ab^biamo ^ricevuto ^questo ^vino
^frutto della ^vite e del ^nostro ^lavoro;
lo ^presentiamo a ^Te
per^ché diventi per ^noi be^vanda di ^vita e^terna.
Bene\[D]detto sei \[Em]Tu, Si\[F#m]gnor.
Bene\[G]detto sei \[A]Tu, Si\[D]gnor.
\endverse
\endsong


%titolo{Benedetto Tu, Signore}
%autore{Ricci}
%album{Venne nel mondo}
%tonalita{Do}
%gruppo{}
%momenti{Offertorio}
%identificatore{benedetto_tu_signore_ricci}
%data_revisione{2011_12_31}
%trascrittore{Francesco Endrici}
\beginsong{Benedetto Tu, Signore}[by={Ricci}]
\ifchorded
\beginverse*
\vspace*{-0.8\versesep}
{\nolyrics \[Dm7]\[C]\[F]\[F]\[Dm7]\[C]\[F]\[F]\[C]\[C]\[C]\[C]\[C]}
\vspace*{-\versesep}
\endverse
\fi
\beginchorus
Benedetto \[F]tu, Signore, \[C]
benedetto \[F]tu nei secoli \[C]
benedetto \[F]tu, Signore. \[C]\[F]
\endchorus
\beginverse
\memorize
\[C]Prendi da queste \[F]mani il pane
\[C]che offriamo a \[F]te,
fanne \[Am]cibo che porte\[G]rà la vita \[F]tua. \[\vline]\[G4]\[G]\[\vline]\[C]
\endverse
\beginchorus
Benedetto \[F]tu, Signore, \[C] 
benedetto \[F]tu nei secoli \[C]
benedetto \[F]tu, Signore. \[C]\[F]
\endchorus
\beginverse
^Prendi da queste ^mani il vino
^che offriamo a ^te,
fanne ^linfa che porte^rà l' eterni^tà. \[F]\[G]
\endverse
\beginverse
Queste nostre o\ch{F}{f}{f}{ff}erte, accogli\[C]le, Si\[C]gnore, \[F]\[G]
e saranno o\ch{F}{f}{f}{ff}erte pure. \[C] \[F]\[G]
Questo nostro \[F]mondo accogli, \[C]o Si\[C]gnore, \[F]\[G]
e saranno \[G]cieli e terre \[F]
che tu fa\[F]rai, nuovi. \[C]
\endverse
\beginchorus
Benedetto \[F]tu, Signore, \[C] 
benedetto \[F]tu nei secoli \[C]
benedetto \[F]tu, Signore. \[C]\[F]
\endchorus
\beginverse
^Prendi da queste ^mani il cuore
^che offriamo a ^te,
fanne ^dono che porte^rà la vita ^tua. \[G]
{\nolyrics \[C]\[F]\[C]\[F]\[C]\[F]\[C]}
\endverse
\endsong

%titolo{Benedici il Signore}
%autore{Frisina}
%album{Benedici il Signore}
%tonalita{}
%gruppo{}
%momenti{Salmi}
%identificatore{benedici_il_signore_frisina}
%data_revisione{2011_12_31}
%trascrittore{Francesco Endrici}
\beginsong{Benedici il Signore anima mia}[by={Frisina}]
\beginchorus
Bene\[Am]dici il Si\[G]gnore, anima \[C]mia
Quant'è in \[F]me bene\[G]dica il suo \[C]nome
Non di\[Dm]mentiche\[G]rò tutti i \[E]suoi bene\[Am]fici.
Bene\[F]dici il Si\[Em]gnore, anima \[Am]mia.
\endchorus
\beginverse
Lui per\[F]dona \[G]tutte le tue \[C]colpe
e ti \[Dm]salva dalla \[G]mor\[Am]te.
Ti co\[F]rona di \[G]grazia e ti \[E]sazia di \[Am]beni
nella \[F]tua giovi\[G]nez\[E]za.
\endverse
\beginverse
\chordsoff
Il Si^gnore a^gisce con giu^stizia
con a^more verso i ^pove^ri
rive^lò a Mo^sè le sue ^vie, ad Isra^ele
le sue ^grandi ^ope^re.
\endverse
\beginverse
\chordsoff
Il Signore è buono e pietoso
lento all'ira e grande nell'amor.
Non conserva in eterno il suo sdegno e la sua ira
verso i nostri peccati.
\endverse
\beginverse
\chordsoff
Come dista Oriente da Occidente
allontana le tue colpe.
Perché sa che di polvere siam tutti noi plasmati,
come l'erba i nostri giorni.
Benedite il Signore voi angeli.
\endverse
\beginverse
\chordsoff
voi tutti suoi ministri.
Beneditelo voi tutte sue opere e domíni.
Benedicilo tu, anima mia.
\endverse
\endsong

%titolo{Benedici, o Signore}
%autore{Gen Rosso}
%album{Se siamo uniti}
%tonalita{Si-}
%gruppo{}
%momenti{Offertorio}
%identificatore{benedici_o_signore}
%data_revisione{2011_12_31}
%trascrittore{Francesco Endrici}
\beginsong{Benedici, o Signore}[by={Gen\ Rosso}]
\beginverse
\[Bm]Nebbia e freddo, giorni lunghi e a\[A]mari
mentre il seme \[Bm]muore.
\[D]Poi prodigio, antico e sempre \[A]nuovo,
del primo filo d'\[G]erba.
E nel \[D]vento dell'e\[A]state on\[Bm]deggiano le \[D]spighe
a\[A]vremo ancora \[F#]pa\[B]ne.
\endverse
\beginchorus
\[E]Bene\[B]dici, \[E]o Si\[B]gnore,
\[A]questa o\ch{E}{f}{f}{ff}erta che por\[F#4]tiamo a \[F#]te
\[E]Facci \[B]uno \[G#m]come il \[E&m]pane
\[C#]che anche \[E]oggi hai \[B]dato a noi.
\endchorus
\beginverse
\chordsoff
^Nei filari, dopo il lungo in^verno \brk fremono le ^viti.
^La rugiada avvolge nel si^lenzio \brk i primi tralci ^verdi.
Poi co^lori dell'au^tunno, coi ^grappoli ma^turi
a^vremo ancora ^vi^no.
\endverse
\endsong

%titolo{Benedici}
%autore{Spoladore}
%album{Così}
%tonalita{Re}
%gruppo{}
%momenti{Offertorio;Comunione}
%identificatore{benedici_spoladore}
%data_revisione{2011_12_31}
%trascrittore{Francesco Endrici}
\beginsong{Benedici}[by={Spoladore}]
\beginverse
\[D]Padre buono che sei in cielo \[G]\[A]\[D]
il tuo nome è ogni \[A]uomo, \[Em] \[D]\[A]
ogni cuore è il tuo \[Em]cielo, \[G]\[D]\[A]
la tua casa siamo \[D]noi. \[G]\[A]\[D]
Benedici questa terra \[G]\[A]\[D]
che Tu abiti e fai \[A]bella, \[Em] \[D]\[A]
benedici questo \[Em]giorno \[G]\[D]\[A]
che ci cresce fra le \[D]mani. \[G]\[A]\[D]
\endverse
\beginchorus
Bene\[D7]dici questa \[G]vita \[B7]
che per tutti sia fe\[Em]lice, \[G7]
e il coraggio di do\[C]nare
sia lo \[A]Spirito del \[G]mondo,
benedici o mio Si\[D]gnore,
benedici o mio Si\[Cm]gnore. \[G] \[A7]
\endchorus
\beginverse
\chordsoff
^Benedici questo tempo ^^^ (\textsc{ii} voce: o o o)
ogni uomo che la^vora, ^^^
ogni mamma e il suo bam^bino ^^^
che di Te sono il sor^riso. ^^^
Benedici chi Ti cerca ^^^ \brk nel silenzio del de^serto ^^^
e chi invece ti ha con^fuso ^^^ \brk con la fretta ed il ru^more. ^^^
\endverse
\beginverse
\chordsoff
^Benedici chi Ti prega ^^^ (\textsc{ii} voce: o mio Signor)
e non sa il nome ^Tuo ^^^
ogni figlio che ha pa^ura ^^^
di esser solo e del fu^turo. ^^^
Benedici i Tuoi poveri ^^^ \brk ed il grido della ^fame ^^^
prendi questo nostro ^pane  ^^^ \brk per la mensa dell'a^more. ^^^
\endverse
\beginverse
\chordsoff
^Il dolore e la fatica  ^^^
benedici o Si^gnore ^^^
chi ha il fucile tra le ^mani ^^^
e il perché lui non lo ^sa. ^^^
Benedici la speranza ^^^ \brk che sa nascere dal ^buio ^^^
benedici questa ^gioia ^^^ \brk che cantiamo assieme a ^Te. ^^^
\endverse
\ifchorded
\beginverse*
\vspace*{-0.8\versesep}
{\nolyrics \[C]\[G]\[C7dim]\[Am] \[G]}
\vspace*{-\versesep}
\endverse
\fi
\endsong


%titolo{Benedirò}
%autore{Ricci}
%album{La Tua dimora}
%tonalita{Mi-}
%gruppo{}
%momenti{Offertorio;Salmi}
%identificatore{benediro}
%data_revisione{2011_12_31}
%trascrittore{Francesco Endrici - Manuel Toniato}
\beginsong{Benedirò}[by={Ricci}]

\ifchorded
\beginverse*
\vspace*{-0.8\versesep}
{\nolyrics \[Em] \[D] \[Em] \[D] \[E] }
\vspace*{-\versesep}
\endverse
\fi
\beginverse
Benedi\[Em]rò il Signore in \[D]ogni \[Em]tempo. \brk \[G]Io benedi\[D]rò!
Benedi\[Em]rò il Signore in \[D]ogni \[Em]tempo. \brk \[G]Oh, \[D]oh, \[Em]oh.
Sulla mia \[Em]bocca sempre \[D]la sua \[Em]lode. \brk \[G]Sempre lode\[D]rò!
Chi cerca \[Em]lui non manca \[D]mai di \[Em]nulla. \brk \[G]Oh, \[D]oh, \[Em]oh.
\endverse

\beginchorus
Bene\[G]dici questo \[C]nostro \[D]pane: \brk \[G]diver\[C]rà il \[D]corpo tuo.
Bene\[G]dici questo \[C]nostro \[D]vino: \brk \[G]diver\[C]rà il \[D]sangue tuo.
\endchorus

\beginverse
\chordsoff
Benedirò il Signore in ogni tempo. \brk Io benedirò!
Benedirò il Signore in ogni tempo. \brk Oh, oh, oh.
Come briciola del mondo nuovo. \brk Vieni a noi!
Come goccia pura del tuo cielo. \brk Oh, oh, oh.
\endverse

\beginverse
\chordsoff
Benedirò il Signore in ogni tempo. \brk Io benedirò!
Benedirò il Signore in ogni tempo. \brk Oh, oh, oh.
Che amore immenso il tuo, Gesù Signore. \brk Che ci fai te!
Che amore immenso il tuo, Gesù Signore. \brk Oh, oh, oh.
\endverse
\endsong


%titolo{Bonse aba}
%autore{}
%album{}
%tonalita{Reb}
%gruppo{}
%momenti{}
%identificatore{bonse_aba}
%data_revisione{2011_12_31}
%trascrittore{Francesco Endrici}
\beginsong{Bonse aba}
\textnote{Canto tradizionale dello Zambia che significa:
‘‘Tutti quelli che cantano vanno chiamati figli di Dio''.}
\chordsoff
\beginverse*
Bonse aba mu pokelela Ba li pele maka akuba bana	
		\echo{Bonse aba mu pokelela   Ba li pele maka akuba bana}

Bonse aba mu pokelela Ba li pele maka akuba bana	
		\echo{Bonse aba mu pokelela   Ba li pele maka akuba bana}

Kuba bana		\echo{Kuba bana} 

Kuba bana 		\echo{bakwa lesa} 

Kuba bana		\echo{Kuba bana}  

Kuba bana 		\echo{bakwa lesa} 

Kuba bana	\echo{Kuba bana   Kuba bana bakwa lesa  }  
    
Kuba bana	\echo{Kuba bana   Kuba bana bakwa lesa}
\endverse
\beginverse*
Bonse aba mu pokelela Ba li pele maka akuba bana	
		\echo{Bonse aba mu pokelela   Ba li pele maka akuba bana}

Bonse aba mu pokelela Ba li pele maka akuba bana	
		\echo{Bonse aba mu pokelela   Ba li pele maka akuba bana}

Kuba bana		\echo{Kuba bana} 

Kuba bana 		\echo{bakwa lesa} 

Kuba bana		\echo{Kuba bana}  

Kuba bana 		\echo{bakwa lesa} 


Muya ya ya	\echo{Muya ya ya  Muya ya ya  bakwa lesa}

Muya ya ya	\echo{Muya ya ya  Muya ya ya  bakwa lesa}
\endverse
\endsong

%titolo{Buon Natale}
%autore{Spoladore}
%album{Come in cielo così in terra}
%tonalita{Do}
%gruppo{}
%momenti{Natale}
%identificatore{buon_natale_spoladore}
%data_revisione{2011_12_31}
%trascrittore{Francesco Endrici}
\beginsong{Buon Natale}[by={Spoladore}]
\beginverse
Buon Na\[C]tale a questa \[G]terra \brk che si \[F]sveglia con il \[C]sole.
Buon Na\[F]tale alla buona \[C]gente di do\[Dm]mani. \[G]
Buon Na\[C]tale a tutti \[G]bimbi \brk nelle \[F]loro vesti \[C]bianche
e a \[F]tutti quei bam\[C]bini \brk che non \[Dm]li vedremo \[G]mai.
Buon Na\[C]tale al \[G]mondo degli a\[F]mici \[C]
nel nostro cuore non ci \[Em]lasceremo \[F]mai.
Buon Natale a \[G]chi si ama \[C] \brk ed ha fiducia an\[C7]cora \[F]
buon Natale a \[G]chi ha finito \[C] \brk l'ultimo sorso \[C7]della vita. \[F]
Buon Natale a \[G]chi si ama \[C] \brk ed ha fiducia an\[C7]cora \[F]
buon Natale a \[G]chi ha finito. \[C]\[G]
\endverse
\beginverse
%\chordsoff
E se ^scenderà la ^neve \brk e ci ^coprirà i pen^sieri
quei cat^tivi pen^sieri \brk che non dovrem^mo fare ^mai. 
Sotto il ^filo della ^luna \brk spende^remo con le ^mani
tanti ab^bracci quanto il ^tempo ci ri^mane. ^
Buon Na^tale al ^mondo che cam^mina ^
dietro ai suoi sogni e in ^fretta se ne ^va.
Con il freddo che ^fa restare ^so^li ^
buon Natale a ^chi si incontre^rà. ^ ^
Con il freddo che ^fa restare ^so^li ^
buon Natale a ^chi si incontre^rà.
\endverse
\beginverse
\transpose{2}
\ifchorded
{\nolyrics \[G4/7]}\fi
Buon Na^tale a chi ci ha ^dato \brk questa ^vita e questo ^nome
nel si^lenzio qual^cuno chiama ^già. ^
Buon Na^tale a tutti ^quelli \brk che si ^danno una ^mano
che il ^dolore il suo ^volto sanno ^già. ^
Buon Na^tale alla ^piccola mia ^gente ^
per quel Gesù che non si ^stanca mai di ^noi
che tutti i giorni ^sceglie la sua ^ca^sa ^
in qualche parte ^qui in mezzo a ^noi. ^^
Che tutti i giorni ^sceglie la sua ^ca^sa ^
in qualche parte ^qui in mezzo a ^noi.
\endverse
\endsong



\lettera
%titolo{Camminerò}
%autore{Marani, Gen Rosso}
%album{Cantinfesta}
%tonalita{Re}
%gruppo{}
%momenti{Congedo}
%identificatore{camminero}
%data_revisione{2011_12_31}
%trascrittore{Francesco Endrici - Manuel Toniato}
\beginsong{Camminerò}[by={Marani, Gen\ Rosso}]

\beginchorus
\[D]Camminerò, \[Bm]camminerò, \[G]sulla tua strada Si\[A]gnor
\[D]dammi la mano, \[Bm]voglio restar \brk per \[G]sempre insi\[A]eme a \[D]te.
\endchorus

\beginverse
\[D]Quando ero solo, solo e \[Bm]stanco nel mondo,
\[Em]quando non c'era l'\[A]amor,
\[D]tante persone vidi in\[Bm]torno a me,
sen\[G]tivo can\[A]tare co\[D]sì.
\endverse

\beginverse
\chordsoff
Io non capivo, ma rimasi a sentire
quando anch'io vidi il Signor.
Lui mi chiamava, chiamava anche me,
ed io gli risposi così.
\endverse

\beginverse
\chordsoff
Or non m'importa se uno ride di me
lui certamente non sa
del gran tesoro che trovai quel dì,
e dissi al Signore così.
\endverse

\beginverse
\chordsoff
A volte son triste, ma mi guardo intorno,
scopro il mondo e l'amor
sono questi i doni che Lui dona a me
felice ritorno a cantar.
\endverse
\endsong

%titolo{Camminiamo sulla strada}
%autore{Varnavà, Spiritual}
%album{I negro spirituals in italiano}
%tonalita{Mi}
%gruppo{}
%momenti{}
%identificatore{camminiamo_sulla_strada}
%data_revisione{2011_12_31}
%trascrittore{Francesco Endrici - Manuel Toniato}
\beginsong{Camminiamo sulla strada}[by={Varnavà, Spiritual}]

\beginverse
Cammi\[E]niamo \[E7]sulla \[A]strada
che han per\[E]corso i \[F#7]santi \[B7]tuoi
tutti \[E]ci ri\[E7]trove\[A]remo
dove e\[E]terno \[B7]splende il \[E]sol.
\endverse

\beginchorus
E \[E]quando in ciel dei santi tuoi
la grande schiera \[F#7]arrive\[B7]rà
o Si\[E]gnor co\[E7]me vor\[A]rei
che ci \[E]fosse un \[B7]posto per \[E]me.

\chordsoff
E quando il sol si spegnerà
e quando il sol si spegnerà
o Signor come vorrei
che ci fosse un posto per me.
\endchorus

\beginverse
\chordsoff
C'è chi dice che la vita
sia tristezza sia dolor
ma io so che viene il giorno
in cui tutto cambierà.
\endverse

\beginchorus
\chordsoff
E quando in ciel risuonerà
la tromba che tutti chiamerà
o Signor come vorrei
che ci fosse un posto per me.

Il giorno che la terra e il ciel
a nuova vita risorgeran
o Signor come vorrei
che ci fosse un posto per me.
\endchorus
\endsong

%titolo{Cantate al Signore un canto nuovo}
%autore{Fallormi}
%album{Giovani Verso Assisi 2004}
%tonalita{Sol}
%gruppo{}
%momenti{Ingresso;Salmi}
%identificatore{cantate_al_signore_un_canto_nuovo}
%data_revisione{2011_12_31}
%trascrittore{Francesco Endrici}
\beginsong{Cantate al Signore}[by={Fallormi}]
\beginchorus
Can\[G]tate al Si\[D]gnore un \[C]canto \[G]nuovo,
\[C]perché ha com\[G]piuto pro\[Am7]di|\[D4]gi.
\[D]Ha |\[G]manife\[D]stato la \[C]sua sal\[G]vezza,
\[C]su tutti i |\[Bm7]popo\[Em7]li la |\[Am7]sua bon\[D7]tà.
{\nolyrics \[G]\[D]\[C]\[D]\[G]}
\endchorus
\beginverse
Egli \[C]si è ricor\[G]dato \brk della \[Em7]sua |\[C]fe\[A7]del|\[D4]\[D]\[C]tà.
I con\[D]fini \[Bm7]della \[Em7]terra \[C]hanno ve\[G]duto \brk la sal\[Am7]vezza \[D7]del Si\[G]gnor.
\endverse
\beginverse
\chordsoff
Esultiamo di gioia acclamiamo al Signor. 
Con un suono melodioso: cantiamo insieme
lode e gloria al nostro Re.
\endverse
\beginverse
\chordsoff
Frema il mare e la terra, il Signore verrà! 
Con giudizio di giustizia, con rettitudine nel mondo porterà.
\endverse
\endsong



%titolo{Cantate inni a Dio}
%autore{Morrone, Altieri}
%album{Vittoria}
%tonalita{Fa}
%gruppo{}
%momenti{Ingresso;Salmi}
%identificatore{cantate_inni_a_dio}
%data_revisione{2011_12_31}
%trascrittore{Francesco Endrici - Manuel Toniato}
\beginsong{Cantate inni a Dio}[by={Morrone, Altieri}]

\ifchorded
\beginverse*
\vspace*{-0.8\versesep}
{\nolyrics \[F2] \[F2/A] \[B&2] \[B&7+/C] \rep{2}}
\vspace*{-\versesep}
\endverse
\fi
\beginchorus
Can\[F]tate inni a \[C2/E]Dio, cantate \[C/D]in\[Dm7]ni. \[Am7/4] 
Can\[B&]tate al nostro \[G7]Re, cantate \[B&7+]in\[C7]ni.
Can\[F]tate inni \[C2/E]a Dio, cantate \[C/D]in\[Dm7]ni. \[Am7/4] 
Can\[Gm7]tate al nostro \[Am7]Re, can\[Dm7]tate \[Gm7/C]in\[F2]ni.
\endchorus

\beginverse
\[B&7+]Cantate al Signore un can\[F2/A]to nuovo,
la sua \[Gm]lode su tutta la \[C/E]ter\[Dm7/C]ra,
\[C/B&]perché \[Am7]forte è il suo amore verso \[C/D]tut\[Dm7]ti,
in  e\[B&2]terno la sua \[G]fedel\[Gm7]tà! \[C] \[C/E] 
\endverse

\beginverse
\chordsoff
Dal sorgere del sole al suo tramonto,
lodato sia il nome del Signore.
Su tutti i popoli è l'eccelso,
più alta dei cieli è la sua gloria!
\endverse

\ifchorded
\beginverse*
\vspace*{-0.8\versesep}
{\nolyrics \[F2] \[F2/A] \[B&2] \[B&7+/C] 
\[F2] \[F2/A] \[B&2] \[C/D] 
\[G2] \[G2/B] \[C2] \[C7+/D] }
\endverse
\fi

\beginchorus
\[G]Cantate in\[D2/F#]ni a Dio, can\[D/E]tate in\[Em7]ni. \[Bm7/4] 
Can\[C]tate al nostro \[A2/C#]Re, cantate \[C7+/D]in\[D7]ni.
Can\[G]tate inni a \[D2/F#]Dio, cantate \[D/E]in\[Em7]ni.\[Bm7/4] 
Can\[Am7]tate al nostro \[Bm7]Re, can\[Em7]tate\[Am7/D]in\[G2]ni.
\endchorus

\beginverse
\[C7+]Ti lodino Signor tutte le \[G2/B]genti,
pro\[Am]clamino le \[Am/G]tue mera\[D/F#]\[Em7]vi\[D]glie,
e si ral\[D/E]legri\[Em7]no,  \[D] \[G6/D] 
an\[C2]nunzino la tua \[A]mae\[Am7]stà!  \[D] \[D/F#] 
\endverse

\beginchorus
Can\[C]tate al \[D/C]{no}\[C]stro \[G/D]Dio,
Can\[A9]{ta}\[C]{te al} \[D/C]{no}\[C]stro \[G/D]Re,
can\[Em7]{tate} \[C/D]{in}\[G]ni.
\endchorus
\endsong



%titolo{Canterò per Te}
%autore{Fanelli}
%album{Non spegnere l'amore}
%tonalita{Do}
%gruppo{}
%momenti{}
%identificatore{cantero_per_te}
%data_revisione{2011_12_31}
%trascrittore{Francesco Endrici}
\beginsong{Canterò per Te}[by={Fanelli}]
\beginverse
\[C]Tu che nel silenzio \[G]parli
solo \[Am]tu, che vivi nei miei \[F]giorni,
puoi col\[G]mare la sete che c'è in \[Em]me
e ri\[C]dare senso a \[Em]questa vita \[Am]mia \[C]
che rin\[Dm7]corre la sua veri\[G]tà.
\endverse
\beginverse
\chordsoff
^Tu di chiara luce ^splendi
solo ^tu, i miei pensieri ac^cendi
e di^segni la tua pace dentro ^me
scrivi ^note di una ^dolce melo^dia ^
che poi ^sale nel cielo in liber^tà.
\endverse
\beginchorus
E cante\[C]rò solo per \[Am7]te
la mia più \[F]bella melo\[C]dia
che vole\[Dm7]rà nel cielo im\[G]menso.
E gride\[C]rò al mondo \[Am7]che
un nuovo \[F]sole nasce\[C]rà
ed una \[Dm7]musica di \[G]pace cante\[C]rò. \[G]\[Dm]\[G]
\endchorus
\beginverse
^Tu che la tua mano ^tendi
solo ^tu che la tua vita ^doni
puoi ri^darmi la mia felici^tà
la spe^ranza di una ^vita nuova in ^te ^
la cer^tezza che io rinasce^rò.
\endverse
\beginchorus
\transpose{2}
\[G]
E cante\[C]rò solo per \[Am7]te
la mia più \[F]bella melo\[C]dia
che vole\[Dm7]rà nel cielo im\[G]menso.
E gride\[C]rò al mondo \[Am7]che
un nuovo \[F]sole nasce\[C]rà
ed una \[Dm7]musica di \[G]pace cante\[C]rò.
\endchorus
\endsong

%titolo{Cantiamo, cantiamo al Signore}
%autore{Cioffi}
%album{Una terra buona}
%tonalita{Fa}
%gruppo{}
%momenti{Congedo}
%identificatore{cantiamo_cantiamo_al_signore}
%data_revisione{2011_12_31}
%trascrittore{Francesco Endrici}
\beginsong{Cantiamo, cantiamo al Signore}[by={Cioffi}]
\beginchorus
\[F]\[Gm] \[F]\[B&]\[C]\[F]\[Gm]\[F]\[B&]\[C]
Can|\[F]tiamo, can\[Gm]tiamo, can|\[F]tiamo al Si\[B&]gnore 
un canto |\[F]nuovo. |\[F]\[C]| 
Su |\[F]tutta la \[Gm]terra si |\[F]canti, si \[B&]danzi senza |\[B&]fi|\[C]ne.
E|\[B&]sultino i cori degli |\[F]angeli,
si |\[Gm]vesta la terra di |\[Dm7]cielo. 
Can|\[F]tiamo, can\[Gm]tiamo, can|\[F]tiamo al Si\[B&]gnore 
un canto |\[C]nuo|\[C]vo.
\endchorus
\beginverse
Co|\[C]sì! Con tutte le |\[Dm]forze, con tutto l'a|\[B&]more, 
con tutta la |\[F]voce, con tutto il |\[C]cuore. 
Con tutta la |\[Dm]vita cantiamo al Si|\[Gm]gno|\[C]re !
\endverse
\beginchorus
Can|\[F]tiamo, can\[Gm]tiamo, can|\[F]tiamo al Si\[B&]gnore 
un canto |\[F]nuovo. |\[F]\[C]| 
Su |\[F]tutta la \[Gm]terra si |\[F]canti, si \[B&]danzi senza |\[B&]fi|\[C]ne.
E|\[B&]sultino i cori degli |\[F]angeli,
si |\[Gm]vesta la terra di |\[Dm7]cielo. 
Can|\[F]tiamo, can\[Gm]tiamo, can|\[F]tiamo al Si\[B&]gnore 
un canto |\[C]nuo|\[C]vo.
\endchorus
\beginverse
Co|^sì! Con l'anima in |^festa, al suono dell'|^arpa, 
cantiamo i suoi |^inni con squilli di |^tromba. 
Con tutta la |^vita cantiamo al Si|^gno|^re!
\endverse

\beginchorus
Can|\[F]tiamo, can\[Gm]tiamo, can|\[F]tiamo al Si\[B&]gnore 
un canto |\[F]nuovo. |\[F]\[C]| 
Su |\[F]tutta la \[Gm]terra si |\[F]canti, si \[B&]danzi senza |\[B&]fi|\[C]ne.
E|\[B&]sultino i cori degli |\[F]angeli,
si |\[Gm]vesta la terra di |\[Dm7]cielo. 
Can|\[F]tiamo, can\[Gm]tiamo, can|\[F]tiamo al Si\[B&]gnore 
un canto |\[C]nuo|\[C]vo.
\endchorus

\beginchorus
\transpose{2}
Can|\[F]tiamo, can\[Gm]tiamo, can|\[F]tiamo al Si\[B&]gnore 
un canto |\[F]nuovo. |\[F]\[C]| 
Su |\[F]tutta la \[Gm]terra si |\[F]canti, si \[B&]danzi senza |\[B&]fi|\[C]ne.
E|\[B&]sultino i cori degli |\[F]angeli,
si |\[Gm]vesta la terra di |\[Dm7]cielo. 
Can|\[F]tiamo, can\[Gm]tiamo, can|\[F]tiamo al Si\[B&]gnore 
un canto |\[C]nuo|\[C]vo.
\endchorus
\endsong

%titolo{Cantiamo te}
%autore{Gragnani}
%album{Alleluia canterò}
%tonalita{Fa}
%gruppo{}
%momenti{Ingresso;Lode}
%identificatore{cantiamo_te}
%data_revisione{2011_12_31}
%trascrittore{Francesco Endrici - Manuel Toniato}
\beginsong{Cantiamo te}[by={Gragnani}]

\beginverse
\[F]Can\[Gm7]tiamo \[F]Te Si\[Gm]gnore della \[C4]vita: \[C7] 
\[F]il \[Gm7]nome  \[F]tuo è \[Gm]grande sulla \[C4]ter\[C]ra
tutto \[F]parla di \[Am]te e \[Gm]canta la tua \[C]gloria.
\[F]Grande tu \[Am]sei e \[Gm]compi mera\[C]viglie:
tu sei \[F]Dio. \[Dm] \[Gm] \[C7] 
\endverse

\beginverse
\chordsoff
Cantiamo Te Signore Gesù Cristo:
Figlio di Dio venuto sulla terra,
fatto uomo per noi nel grembo di Maria.
Dolce Gesù risorto dalla morte,
sei con noi.
\endverse

\beginverse
\chordsoff
Cantiamo te, amore senza fine:
tu sei Dio lo Spirito del Padre
vivi dentro di noi e guida i nostri passi.
Accendi in noi il fuoco
dell'eterna carità.
\endverse
\endsong


%titolo{Cantico dei cantici}
%autore{Arguello}
%album{}
%tonalita{Mi-}
%gruppo{}
%momenti{Matrimonio}
%identificatore{cantico_dei_cantici}
%data_revisione{2011_12_31}
%trascrittore{Francesco Endrici - Manuel Toniato}
\beginsong{Cantico dei cantici}[by={Arguello}]

\beginverse
\[Em]Vieni dal Liba\[D]no mia sposa, \brk \[C]vieni dal \[D]Libano, \[Em]vieni.
\[Em]Avrai per corona le \[D]vette dei monti, \brk \[C]le alte \[D]cime dell'\[Em]Ermon.
\[Em]Tu m'hai ferito, fe\[D]rito nel cuore, \brk \[C]o so\[D]rella mia \[Em]sposa.
\[Em]Vieni dal Liba\[D]no mia sposa, \brk \[C]vieni dal \[D]Libano, \[Em]vieni.
\endverse

\beginchorus
Cercai l'a\[G]more dell'anima \[D]mia, 
lo cer\[Am]cai senza tro\[Em]varlo.
Trovai l'a\[G]more dell'anima \[D]mia,
l'ho abbrac\[C]ciato, non lo \[D]lascerò \[Em]mai.
\endchorus

\beginverse
\chordsoff
Io appartengo al mio diletto \brk ed egli è tutto per me.
Vieni usciamo alla campagna \brk dimoriamo nei villaggi.
Andremo all'alba nelle vigne, \brk vi raccoglieremo i frutti.
Io appartengo al mio diletto \brk ed egli è tutto per me.
\endverse

\beginverse
\chordsoff
Alzati in fretta o mia diletta, \brk vieni colomba vieni.
L'estate ormai è già passata, \brk il tempo dell'uva è venuto.
I fiori se ne vanno dalla terra, \brk il grande sole è cessato,
Alzati in fretta o mia diletta, \brk vieni colomba vieni.
\endverse

\beginverse
\chordsoff
Come sigillo sul tuo cuore, \brk come sigillo sul tuo braccio.
Che l'Amore è forte come la morte \brk e le acque non lo spegneranno.
Dare per esso tutti i beni della casa, \brk sarebbe disprezzarlo.
Come sigillo sul tuo cuore, \brk come sigillo sul braccio.
\endverse
\endsong

%titolo{Cantico dei redenti}
%autore{Marani}
%album{}
%tonalita{Mi-}
%gruppo{}
%momenti{Ingresso;Congedo;Comunione}
%identificatore{cantico_dei_redenti}
%data_revisione{2011_12_31}
%trascrittore{Francesco Endrici - Manuel Toniato}
\beginsong{Cantico dei redenti}[by={Marani}]

\beginchorus
Il Si\[Em]gnore è la \[D]mia sal\[Em]vezza
e con \[C]lui non \[D]temo \[G]più \[Em] 
perché ho nel \[Am]cuore \[C]la cer\[Em]tezza \[C] 
la sal\[G]vezza è \[D]qui con \[Em]me.
\endchorus

\beginverse
Ti \[G]lodo Si\[D]gnore per\[Em]ché \[C] 
un giorno \[G]eri lon\[D]tano da \[G]me, \[D]
\[G] ora in\[D]vece sei tor\[Em]nato
\[C]e mi hai \[D]preso con \[Em]te.
\endverse

\beginverse
\chordsoff
Berrete con gioia alle fonti
alle fonti della salvezza
e quel giorno voi direte:
lodate il Signore,
invocate il suo nome.
\endverse

\beginverse
\chordsoff
Fate conoscere ai popoli
tutto quello che lui ha compiuto
e ricordino per sempre
ricordino sempre
che il suo nome è grande.
\endverse

\beginverse
\chordsoff
Cantate a chi ha fatto grandezze
e sia fatto sapere nel mondo;
grida forte la tua gioia, abitante di Sion,
perché grande con te è il Signore.
\endverse
\endsong

%titolo{Cantico delle creature}
%autore{Varnavà, Mancinoni}
%album{}
%tonalita{Re-}
%gruppo{}
%momenti{San Francesco}
%identificatore{cantico_delle_creature}
%data_revisione{2011_12_31}
%trascrittore{Francesco Endrici - Manuel Toniato}
\beginsong{Cantico delle creature}[by={Varnavà, Mancinoni}]

\beginverse
\[Dm]Laudato \[Gm]sii mi Si\[Dm]gnore
\[C]per frate \[Dm]sole \[A7]sora \[Dm]luna
\[Dm]frate vento, il \[Gm]cielo e le \[Dm]stelle,
\[C]per sora \[Dm]acqua e \[A7]frate \[Dm]focu.
\endverse

\beginchorus
\[B&]Lau\[C]dato \[A]sii mi Si\[Dm]gnore
\[Gm]per la \[Dm]terra e \[A]le tue crea\[Dm]ture. \rep{2}
\endchorus

\beginverse
\chordsoff
Laudato sii mi Signore
quello che porta la tua pace
e saprà perdonare
per il tuo amore saprà amare.
\endverse

\beginverse
\chordsoff
Laudato sii mi Signore
per sora morte corporale
dalla quale homo vivente
non potrà mai, mai scappare.
\endverse

\beginverse
\chordsoff
Laudate e benedite,
ringraziate e servite
il Signore con umiltà
ringraziate e servite.
\endverse
\endsong


%titolo{Canto a Te, Maria}
%autore{Gen Verde}
%album{Sulla via della speranza}
%tonalita{Mi}
%gruppo{}
%momenti{Maria}
%identificatore{canto_a_te_maria}
%data_revisione{2011_12_31}
%trascrittore{Francesco Endrici}
\beginsong{Canto a Te, Maria}[by={Gen\ Verde}]
\beginchorus
\[E]Nella casa Tua io canto a Te, Ma\[A]ria,
\[E]prendi fra le mani Tu la vita \[D]mia,
\[B]accompagna il mio cammino \[E]verso Lui,
\[A]sulla strada che hai per\[B]corso Tu, Ma\[E]ria.
\endchorus
\beginverse
\[A]Tu che hai vissuto \[B7]nella veri\[E]tà,
\[F#m]Tu, vera donna \[B]della liber\[E]tà,
\[Bm]dal cuore Tuo l'a\[C#7]more im\[F#m]parerò
\[E]e nel mondo \[F#m]io lo porte\[B]rò.
\endverse
\beginverse
\chordsoff
^Resta vicina a ^me, Madre di ^Dio,
^del Tuo coraggio ^riempi il cuore ^mio,
^solo l'amore, al^lora, mi ^guiderà,
^sarò luce ^per l'umani^tà.
\endverse
\endsong




%titolo{Canto dei tre giovani}
%autore{Marranzino, de Luca}
%album{Cantiamo con gioia}
%tonalita{Do}
%gruppo{}
%momenti{Lode}
%identificatore{canto_dei_tre_giovani}
%data_revisione{2011_12_31}
%trascrittore{Francesco Endrici}
\beginsong{Canto dei tre giovani}[by={Marranzino, De\ Luca}]
\beginverse
\[C]Noi ti lo\[Em7]diamo Si\[F]gnore,
\[C]a Te la \[Em7]lode e la \[F]gloria per \[Am]sem\[G]pre
\[Dm]noi lo\[F]diamo il tuo \[G]nome,
\[Dm]a Te la \[F]lode e la \[G]gloria per \[E4]sem\[E]pre.
\endverse
\beginchorus
\[Am]Noi lode\[Em7]remo il Si\[F]gnor,
cante\[C]remo il suo a\[Dm7]mor
che dure\[C]rà \[G]per \[E4]sem\[E]pre.
\[Am]Noi lode\[Em7]remo il Si\[F]gnor,
cante\[C]remo il suo a\[Dm7]mor
che dure\[C]rà \[G]per \[F]sem\[G]pre.
\endchorus
\beginverse
\chordsoff
Astri del cielo lodate il Signore,
a Lui l'onore e la gloria per sempre.
Acque del cielo lodate il Signore,
a Lui l'onore e la gloria per sempre.
\endverse
\beginverse
\chordsoff
Sole e luna lodate il Signore,
a Lui l'onore e la gloria per sempre.
Piogge e rugiade lodate il Signore,
a Lui l'onore e la gloria per sempre.
\endverse
\beginverse
\chordsoff
O venti tutti lodate il Signore,
a Lui l'onore e la gloria per sempre.
Fuoco e calore lodate il Signore,
a Lui l'onore e la gloria per sempre.
\endverse
\beginverse
\chordsoff
Notte e giorno lodate il Signore,
a Lui l'onore e la gloria per sempre.
Nuvole e lampi lodate il Signore,
a Lui l'onore e la gloria per sempre.
\endverse
\beginverse
\chordsoff
Uomini tutti lodate il Signore,
a Lui l'onore e la gloria per sempre.
Tutta la terra dia loda al Signore,
a Lui l'onore e la gloria per sempre.
\endverse
\endsong


%titolo{Canto della pace}
%autore{Cori}
%album{Noi tuo popolo}
%tonalita{Re}
%gruppo{}
%momenti{Pace}
%identificatore{canto_della_pace_cori}
%data_revisione{2011_12_31}
%trascrittore{Francesco Endrici}
\beginsong{Canto della pace}[by={Cori}]
\beginchorus
La \[D]pace del Signore Ge\[A]sù sia con Te,
La \[D]pace del Signore Ge\[A]sù sia con Te,
La \[G]pace del Signore Ge\[A]sù sia con Te,
la \[G]pa\[D]ce. \rep{2}
\endchorus
\beginverse
\[G]Noi cantiamo a \[D]Te che hai vinto la \[Em7]morte, \[A]
\[G]noi cantiamo a \[D]Te che doni \[G]pace
al nostro cuore inquieto, \[Em7] Signore Ge\[A]sù.
\endverse
\endsong

%titolo{Canto di nozze}
%autore{Ricci}
%album{È l'amore che conta}
%tonalita{Sol}
%gruppo{}
%momenti{Matrimoni}
%identificatore{canto_di_nozze}
%data_revisione{2011_12_31}
%trascrittore{Francesco Endrici}
\beginsong{Canto di nozze}[by={Ricci}]
\ifchorded
\beginverse*
\vspace*{-0.8\versesep}
{\nolyrics \[G]\[D]\[C]\[Cm]\[G]\[D]\[C]\[Cm]\[G]\[D]}
\vspace*{-\versesep}
\endverse
\fi
\beginverse
\memorize
\[Em]Tu che inventi per ognuno di noi
un cam\[G]mino che porta verso \[D]Te.
\[Am]Tu che chiami ognuno \[C]per il suo nome, Tu
oggi ci \[D]chiami in\[Em]sieme.
\endverse
\beginverse
^Tu che hai in mente per ognuno di noi
una ^strada bellissima con ^Te,
^Tu stavolta parli ^di un'avventura che,
che si per^corre in^sieme.
\endverse
\beginchorus
Rendi \[G]agile il vascello dell'a\[D]more nostro \[Am]
al soffio lieve del Tuo \[Em]spirito.
E per \[G]gli inimmaginabili \[D]mari conducilo, \[Am7]
quelli che Tu \[C]sai.
Rendi \[G]agile il vascello dell'a\[D]more nostro \[Am]
che tocchi i lidi che Tu solo \[Em]sai.
E la \[G]scia della sua rotta che di\[D]segni un'onda, \[Am7]
quella che Tu \[C]vuoi.
\endchorus
\beginverse
^Dacci forza perché ognuno di noi
con la ^vita dispieghi la Tua i^dea.
^Dacci forza per sve^lare il pensiero Tuo
su questa ^vita in^sieme. \rep{2}
\endverse
\beginchorus
Rendi \[G]agile il vascello dell'a\[D]more nostro \[Am]
al soffio lieve del Tuo \[Em]spirito.
E per \[G]gli inimmaginabili \[D]mari conducilo, \[Am7]
quelli che Tu \[C]sai.
Rendi \[G]agile il vascello dell'a\[D]more nostro \[Am]
che tocchi i lidi che Tu solo \[Em]sai.
E la \[G]scia della sua rotta che di\[D]segni un'onda, \[Am7]
quella che Tu \[C]quella che, \[Am7]
quella che Tu \[C]quella che, \[Am7] quella che tu \[E]vuoi.
\endchorus
\endsong

%titolo{Canto per Jahvè}
%autore{Cento}
%album{Attendi in linea... ti passo Dio}
%tonalita{Re}
%gruppo{}
%momenti{Ingresso}
%identificatore{canto_per_jahve}
%data_revisione{2011_12_31}
%trascrittore{Francesco Endrici - Manuel Toniato}
\beginsong{Canto per Jahvè}[by={Cento}]

\beginchorus
\[D]Canto per Jahvè un \[G]canto di spe\[D]ranza,
un \[G]canto di spe\[D]ranza \[A]per Jah\[D]vè. \rep{2}
\endchorus

\beginverse
\[D]Signore sei grande, \[G]io ti e\[D]salto,
\[G]io ti \[D]canto: alle\[A]lui\[D]a! \rep{2}
\endverse

\beginverse
\chordsoff
Opere grandi tu fai, Signore, 
tutto tu crei con sapienza. \rep{2}
\endverse

\beginverse
\chordsoff
Tu mandi lo Spirito: siamo creati,
nuova è la faccia della terra. \rep{2}
\endverse

\beginverse
\chordsoff
Finché sono vivo canterò al Signore,
a Lui inneggerò finché esisto. \rep{2}
\endverse
\endsong


%titolo{Canzone di San Damiano}
%autore{Ortolani}
%album{Fratello sole sorella luna}
%tonalita{Re-}
%gruppo{}
%momenti{}
%identificatore{canzone_di_san_damiano}
%data_revisione{2011_12_31}
%trascrittore{Francesco Endrici}
\beginsong{Canzone di San Damiano}[by={Ortolani}]
\beginverse
\[Dm]Ogni \[C]uomo \[Dm]sempli\[C]ce \[Dm]porta in \[C]cuore un \[Dm]so\[C]gno,
\[Dm]con a\[C]more ed \[Dm]umil\[C]tà \[Dm]potrà \[C]costru\[Dm]ir\[C]lo;
\[F]se con \[C]fede \[F]tu sa\[C]prai \[F]vive\[C]re umil\[F]men\[C]te,
\[Dm]più fe\[C]lice \[Dm]tu sa\[C]rai \[Dm]anche \[C]senza \[Dm]nien\[C]te.
\[G]Se vorrai, \[B&]ogni giorno, \[F]con il tuo su\[C]dore,
\[G]una pietra \[B&]dopo l'altra \[F]alto arrive\[C]rai.
\endverse
\beginverse
\chordsoff
^Nella ^vita ^sempli^ce ^trove^rai la ^stra^da
^che la ^pace ^done^rà ^al tuo ^cuore ^pu^ro.
^E le ^gioie ^sempli^ci ^sono ^le più ^bel^le,
^sono ^quelle ^che, alla ^fine, ^sono ^le più ^gran^di.
^Dai e dai, ^ogni giorno, ^con il tuo su^dore,
^una pietra ^dopo l'altra, ^alto arrive^rai.
\endverse
\endsong

%titolo{Celebrate}
%autore{Marranzino, de Luca}
%album{Celebrate}
%tonalita{Re}
%gruppo{}
%momenti{Lode}
%identificatore{celebrate_marranzino}
%data_revisione{2011_12_31}
%trascrittore{Francesco Endrici}
\beginsong{Celebrate}[by={Marranzino, De\ Luca}]
\beginverse
\memorize
Cele\[D]brate il Si\[G]gnore e in\[D]vocate il Suo \[A]Nom,
si ral\[D]legri il vostro \[G]cuore alla pre\[D]senza \[A]del Si\[D]gnor.
\chordsoff
Ricer\[D]cate il Si\[G]gnore e la \[D]forza \[G]sua,
can\[D]tate e dan\[G]zate alla pre\[D]senza \[A]del Si\[D]gnor.
\endverse
\beginverse
\chordsoff
Ogni ^giorno ricer^cate il suo ^vol^to,
pro^stratevi e ado^rate alla pre^senza ^del Si^gnor.
\endverse
\beginverse
\chordsoff
Cele\[D]brate il Si\[G]gnore e in\[D]vocate il Suo \[A]Nom,
si ral\[D]legri il vostro \[G]cuore alla pre\[D]senza \[A]del Si\[D]gnor.
\chordsoff
Ricer\[D]cate il Si\[G]gnore e la \[D]forza \[G]sua,
can\[D]tate e dan\[G]zate alla pre\[D]senza \[A]del Si\[D]gnor.
\endverse
\beginchorus
Ricono\[G]scete i suoi pro\[D]digi e le \[F#]sue mera\[Bm]viglie
discen\[G]denti di A\[D]bramo, \[F#]figli del Si\[Bm]gnor,
di\[A]cendo: “\[D]vedo, \[7] \[G]sento, \[A]credo \[7]nel suo a\[D]mor!”
Mettendo \[D]mente e \[G]cuore \[A]alla sua pa\[D4]rola. \[D]
\endchorus
\beginverse
\chordsoff
Della ^Sua Pa^rola Egli ^si ricorde^rà, 
siate ^saldi e fe^deli alla pre^senza ^del Si^gnor.
Egli ^ora ci ^chiama a lo^dare il Suo ^Nom, 
non sa^remo più op^pressi alla pre^senza ^del Si^gnor.
\endverse
\beginchorus
\chordsoff
Ricono\[G]scete i suoi pro\[D]digi e le \[F#]sue mera\[Bm]viglie
discen\[G]denti di A\[D]bramo, \[F#]figli del Si\[Bm]gnor,
di\[A]cendo: “\[D]vedo, \[7] \[G]sento, \[A]credo \[7]nel suo a\[D]mor!”
Mettendo \[D]mente e \[G]cuore \[A]alla sua pa\[D4]rola. \[D]
\[G]\[D]\[F#]\[Bm]\[G]\[D]\[F#]\[Bm] \[A]
“\[D]vedo, \[7] \[G]sento, \[A]credo \[7]nel suo a\[D]mor!”
Mettendo \[D]mente e \[G]cuore \[A]alla sua pa\[D4]rola.
\endchorus
\endsong


%titolo{Cerco solo te}
%autore{Montuori}
%album{Il tuo amore è grande}
%tonalita{Mi}
%gruppo{}
%momenti{}
%identificatore{cerco_solo_te}
%data_revisione{2011_12_31}
%trascrittore{Francesco Endrici - Manuel Toniato}
\beginsong{Cerco solo te}[by={Montuori}]

\beginchorus
Cerco solo {\[A7+]te,} Si\[B]gnor, mio libera\[G#m7]{tore} \[C#m]sei,
fonte dell'a\[A7+]more, \[B]tu, riempi la mia \[A7+]vita.
\[G#m7]Cerco solo {\[F#m7]te,} Si\[B7]gnor, mio libera\[G#m7]{tore} \[C#m]sei,
fonte dell'a\[A7+]more, \[B]tu, riempi la mia \[C#m7]vita.
\endchorus

\beginverse*
Volgi il tuo \[A7+]sguardo su di \[B2/D#]me, 
Dio di im\[G#m7]mensa bon\[D/E]tà
e rendi \[A7+]forte la mia \[B]fede; 
più non vacille\[E7+]{rò.} \[F#m7] 
\[G#m7]Crea un \[A7+]cuore puro in \[B/A]me, 
rin\[G#7/C]nova il mio \[C#m7]spirito, \[B6] 
la mia bocca pro\[Am6/C]clami la tua  \[A6/B]lode.
\endverse
\endsong


%titolo{Che gioia ci dà}
%autore{Gen Rosso}
%album{Se siamo uniti}
%tonalita{Re}
%gruppo{}
%momenti{Ingresso;Congedo}
%identificatore{che_gioia_ci_da}
%data_revisione{2011_12_31}
%trascrittore{Francesco Endrici - Manuel Toniato}
\beginsong{Che gioia ci dà}[by={Gen\ Rosso}]

\ifchorded
\beginverse*
\vspace*{-0.8\versesep}
{\nolyrics \[D] \[A] \[Bm7] \[A] \[D] \[A] \[Bm7] \[A] 
\[D] \[Em7] \[F#m] \[G] \[Bm] \[A] }
\vspace*{-\versesep}
\endverse
\fi

\beginchorus
\[D]Che \[A]gioia ci {\[Bm7]dà}  a\[A]verti in mezzo a \[D]noi, 
e\[A]splode la \[Bm7]vi\[A]ta, \brk splende di \[D]lu\[Em7]ce \[F#m]la \[G]cit\[Bm]tà. \[A] 
\[D]Vo\[A]gliamo gri\[Bm7]dare a \[A]tutto il modo \[D]che 
non \[A]siamo mai \[Bm7]so\[A]li: \[D]sei \[Em7]sem\[F#m]pre  \[G]con \[Bm]noi. \[A] 
\endchorus

\beginverse
Ci hai cer\[D]cato tu e ci hai gui\[Em7]dato nel cam\[A]mino,
ci hai rial\[D]zato tu quando non \[Em7]speravamo \[A]più;
ed o\[Bm]gnuno ormai ti sente \[E7]sempre più vi\[G]cino
perché sap\[D]piamo che \[F#m7]tu cam\[G]mini in mezzo a \[A]noi.
\endverse

\beginchorus
\chordsoff
Che gioia ci dà averti in mezzo a noi,
esplode la vita: cantiamo di felicità.
Sei un fiume che avanza e porti via con te
le nostre paure: chi ti fermerà?
\endchorus

\beginverse
\chordsoff
Strappi gli argini e corri verso la pianura,
steppe aride, terre deserte inonderai:
dove arriverai germoglierà una vita nuova
che non appassirà mai perché tu sei con noi.
\endverse

\beginchorus
\chordsoff
Che gioia ci dà  averti in mezzo a noi,
esplode la vita, splende di luce la città.
Che gioia ci dà averti in mezzo a noi,
esplode la vita: cantiamo di felicità.
\endchorus
\endsong



%titolo{Chi ci separerà}
%autore{Frisina}
%album{Chi ci separerà dall'amore di Cristo}
%tonalita{Re}
%gruppo{}
%momenti{Comunione}
%identificatore{chi_ci_separera}
%data_revisione{2011_12_31}
%trascrittore{Francesco Endrici}
\beginsong{Chi ci separerà}[by={Frisina}]
\beginverse
\[D]Chi \[D]ci se\[A]pare\[Bm]rà \[G]dal \[Em]suo a\[F#m]more 
\[Em]la tribola\[Bm]zione, \[G]forse la \[A]spada? 
\[Bm]Né morte o \[F#m]vita \[G]ci separe\[D]rà, 
\[Em]dall’amore in \[Bm]Cri\[Em]sto Si\[D]\[A]gno\[D]re. 
\endverse
\beginverse
\[D]Chi ci se\[A]pare\[Bm]rà \[G]dal\[Em]la sua \[F#m]pace 
\[Em]la persecu\[Bm]zione, \[G]forse il do\[A]lore? 
\[Bm]Nessun po\[F#m]tere \[G]ci separe\[D]rà 
\[Em]da Colui che è \[Bm]mor\[Em]to per \[D]\[A]no\[D]i. 
\endverse
\beginverse
\transpose{1}
^Chi ^ci se^pare^rà ^dal^la sua ^gioia 
^chi potrà strap^parci ^il suo per^dono? 
^Nessuno al ^mondo ^ci allontane^rà 
^dalla vita in ^Cri^sto Si^^gno^re. 
\endverse
\endsong

%titolo{Chi}
%autore{Gen Rosso}
%album{In concerto per la pace}
%tonalita{Do}
%gruppo{}
%momenti{chi}
%identificatore{chi_gen_rosso}
%data_revisione{2011_12_31}
%trascrittore{Francesco Endrici}
\beginsong{Chi}[by={Gen\ Rosso}]
\beginverse
\[C]Filtra un raggio di sole \[F7+]tra le nubi del cielo,
\[Dm7]strappa la terra al gelo \[C]e nasce un \[G]fiore!
\[C]E poi mille corolle \[F7+]rivestite di poe\[Dm7]sia,
in un gioco d'armo\[C]nia e di co\[G]lori.
\[Am]Ma chi veste i fiori dei \[Em]campi?
Chi ad ognuno \[F]dà colore? \[G]
\endverse
\beginverse
^Va col vento leggera ^una rondine in volo,
^il suo canto sa solo ^di prima^vera!
^E poi intreccio di ali ^come giostra d'alle^gria,
mille voli in fanta^sia fra terre e ^mari.
^Ma chi nutre gli uccelli del ^cielo?
Chi ad ognuno ^dà un nido? \[D]Chi?
\endverse
\beginchorus
\[A]Tu, Creatore del mondo.
\[C#m]Tu, che pos\[F#m]siedi la vita.
\[D]Tu, sole \[E]infinito, \[A]Dio A\[E]more.
\[A]Tu, degli uomini Padre.
\[C#m]Tu, che \[F#m]abiti il cielo.
\[D]Tu, immenso \[E] mistero, \[A]Dio A\[E]more. \[D]Dio A\[E]more

\endchorus
\ifchorded
\beginverse*
\vspace*{-\versesep}
{\nolyrics \[C]\[F7+]\[Dm7]\[C]\[G]}
\endverse
\fi
\beginverse
^Un'immagine viva ^del Creatore del mondo
^un riflesso profondo ^della sua ^vita.
^L'uomo, centro del cosmo, ^ha un cuore per amare
e un ^mondo da plasmare ^con le sue ^mani.
^Ma chi ha dato all'uomo la ^vita?
Chi a lui ha ^dato un cuore? \[D]Chi?
\endverse
\beginchorus
\chordsoff 
Rit. 
\endchorus 
\ifchorded
\beginverse*
\vspace*{-\versesep}
{\nolyrics \[C]\[F7+]\[Dm7]\[C]\[G]\[C]}
\endverse
\fi
\endsong

%titolo{Chi potrà varcare}
%autore{Turoldo, Passoni}
%album{Salmi e cantici}
%tonalita{Re}
%gruppo{}
%momenti{Salmi;Comunione}
%identificatore{chi_potra_varcare}
%data_revisione{2011_12_31}
%trascrittore{Francesco Endrici - Manuel Toniato}
\beginsong{Chi potrà varcare}[by={Turoldo, Passoni}]

\beginverse
\[D]Chi potrà var\[A]{ca}\[D]re Si\[G]gnor la tua \[D]{so}\[A]glia
\[D]Chi fer\[F#m]mare il \[Bm]piede sul \[Em]tuo \[A]monte \[D]santo.
\endverse

\beginverse
\chordsoff
Uno che per vie diritte cammini
uno che in opere giuste s'adopri.
\endverse

\beginverse
\chordsoff
Uno che conservi un cuore sincero
uno che abbia monde le labbra da inganni.
\endverse

\beginverse
\chordsoff
Uno che al prossimo male non faccia
uno che al fratello non rechi offesa.
\endverse
\endsong



%titolo{Chi rimane in me}
%autore{Spoladore}
%album{Unanima}
%tonalita{Re-}
%gruppo{}
%momenti{Comunione}
%identificatore{chi_rimane_in_me}
%data_revisione{2011_12_31}
%trascrittore{Francesco Endrici - Manuel Toniato}
\beginsong{Chi rimane in me}[by={Spoladore}]

\beginverse
Chi rimane in \[Dm]me e io in \[E7]lui, \brk fa molto \[B&dim]frut\[A7]to per\[Dm]ché,
senza di \[Dm]me, senza di \[E7]me, \brk voi non po\[B&dim]te\[A7]te far \[Dm]nulla.
\endverse

\beginchorus
\[C7] Restate in \[F]me, restate in \[C]me, \brk io son la \[Dm]vite e voi i \[B&]tralci.
Restate in \[Am]me, restate in \[Dm]me, \brk amici \[G]miei la mia \[B&]forza \[C]vi da\[Dm]rò.
\endchorus

\beginverse
\chordsoff
Io vi chiamerò amici miei e do la vita per voi
Amatevi così nel nome mio, la vostra gioia sarà.
\endverse

\beginverse
\chordsoff
Scenderà tra voi la Verità, il mio Spirito in voi.
Resterà tra voi l'Amore mio, \brk farà di voi una cosa sola.
\endverse
\endsong


%titolo{Chiesa di Dio}
%autore{Costa, Villeneuve}
%album{}
%tonalita{Re}
%gruppo{}
%momenti{Ingresso}
%identificatore{chiesa_di_dio}
%data_revisione{2011_12_31}
%trascrittore{Francesco Endrici - Manuel Toniato}
\beginsong{Chiesa di Dio}[by={Costa, Villeneuve}]

\beginchorus
\[D]Chiesa di \[G]Dio, \[A]popolo in \[D]festa
\[G]allelu\[A]ia, \[Bm]al\[G]le\[A]lu\[D]ia.
\[Bm]Chiesa di \[G]Dio, \[A]popolo in \[D]festa,
\[G]canta di \[A]gio\[Bm]ia, il Si\[Em]gno\[A7]re è con \[D]te.
\endchorus

\beginverse
\[Bm]Dio ti ha \[G]scelto, \[Em]Dio ti \[F#m]chiama
\[A]nel suo a\[Bm]more ti \[Em]vuole con \[F#m7]sé
\[G]spargi nel \[Em]mondo il \[A]suo Van\[Bm]gelo,
\[Bm]seme di \[Em]pace e \[D]di bon\[A]tà.
\endverse

\beginverse
\chordsoff
Dio ti guida come un padre:
tu ritrovi la vita con lui.
Rendigli grazie, sii fedele,
finché il suo Regno ti aprirà.
\endverse

\beginverse
\chordsoff
Dio ti nutre col suo cibo,
nel deserto rimane con te.
Ora non chiudere il tuo cuore:
spezza il tuo pane a chi non ha.
\endverse

\beginverse
\chordsoff
Dio mantiene la promessa:
in Gesù Cristo ti trasformerà.
Porta ogni giorno la preghiera
di chi speranza non ha più.
\endverse


\beginverse
\chordsoff
Chiesa che vivi nella storia,
sei testimone di Cristo quaggiù:
apri le porte ad ogni uomo,
salva la vera libertà.
\endverse

\beginverse
\chordsoff
Chiesa, chiamata al sacrificio
dove nel pane si offre Gesù,
offri gioiosa la tua vita
per una nuova umanità.
\endverse
\endsong

%titolo{Chiesa di mattoni}
%autore{{Cento}}
%album{Attendi in linea... ti passo Dio}
%tonalita{Do}
%gruppo{}
%momenti{Ingresso}
%identificatore{chiesa_di_mattoni}
%data_revisione{2011_12_31}
%trascrittore{Francesco Endrici - Manuel Toniato}
\beginsong{Chiesa di mattoni}[by={Cento}]

\beginchorus
\[C]Chiesa di mattoni, \[G]no. \[C]Chiesa di persone, \[G]sì
\[C]siamo \[G]noi, \[C]sia\[G]mo \[C]noi.
\[C]Nasce la comuni\[G]tà \[C]vive nella liber\[G]tà,
\[C]siamo \[G]noi, \[C]sia\[G]mo \[C]noi.
\endchorus

\beginverse
\[C]Quando ci incontriamo \[G]nasce la speranza
\[C]che nel mondo c'è l'a\[G]more.
\[C]Grideremo insieme \[G]tutta questa forza
\[C]nata dalla \[G]liber\[C]tà.
\endverse

\beginverse
\chordsoff
Sopra quella pietra che si chiama Pietro
ieri la fondò il Signore.
Oggi siamo noi quelle pietre vive
che la costruiamo ancor.
\endverse

\beginverse
\chordsoff
Noi spezziamo il pane
noi preghiamo insieme sempre in fraternità.
La Parola è un dono che ci fa felici
oggi e per l'eternità.
\endverse

\beginverse
\chordsoff
Dividiamo i beni nelle nostre case
con tanta semplicità e se c'è una legge
è quella dell'amore è l'amore del Signor.
\endverse
\endsong

%titolo{Cieli nuovi e terra nuova}
%autore{Ricci}
%album{La Tua dimora}
%tonalita{Sol}
%gruppo{}
%momenti{Congedo}
%identificatore{cieli_nuovi_e_terra_nuova}
%data_revisione{2011_12_31}
%trascrittore{Francesco Endrici}
\beginsong{Cieli nuovi e terra nuova}[by={Ricci}]
\ifchorded
\beginverse*
\vspace*{-0.8\versesep}
{\nolyrics \[G]\[D]\[G]\[D]}
\vspace*{-\versesep}
\endverse
\fi
\beginverse
\memorize
Cieli \[G]nuovi \[D] e terra \[Em]nuova: \[C]
è il de\[G]stino dell'u\[D]mani\[Am]tà! \[C]
Viene il \[Em]tempo, \[G] arriva il \[D]tempo, \[Bm]
che ogni real\[G]tà si trasfi\[Bm]gure\[D]rà. \[D]
E in cieli \[G]nuovi \[D]e terra \[Em]nuova \[C]
il nostro a\[G]nelito si \[D]plache\[Am]rà. \[C]
La tua \[Em]casa, \[G] la tua di\[D]mora, \[Bm]
su tutti i \[G]popoli si e\[Bm]stende\[D]rà. \[D]\[Am]
\endverse
\beginchorus
È il pane \[C] del cielo \[Em] che ci fa \[G]vivere: \[Bm]
che chiama a \[C]vivere e an\[Em]dare nel \[D]mondo. \[Am]
È il pane \[C] del cielo \[Em] che ci fa \[G]vivere: \[Bm]
che chiama a \[C]vivere e an\[Em]dare a por\[Em]tare il tuo \[D]dono. \[D]
\endchorus
\beginverse
%\chordsoff
Cieli ^nuovi ^ e terra ^nuova: ^
la spe^ranza non in^ganna ^mai! ^
E tu ri^sorto, ^ ci fai ri^sorti, ^
tutto il cre^ato un canto ^diver^rà. ^
E in cieli ^nuovi ^ e terra ^nuova ^
c'è il di^segno che hai af^fidato a ^noi, ^
Gerusa^lemme, ^ dal cielo ^scende, ^
Gerusa^lemme in terra ^trove^rà. ^^
\endverse
\beginchorus\chordsoff 
Rit.
\endchorus
\beginverse
%\chordsoff
Cieli ^nuovi ^ e terra ^nuova:  ^
è il de^stino dell'u^mani^tà! ^
Viene il ^tempo, ^ arriva il ^tempo, ^
che ogni real^tà si trasfi^gure^rà. ^
\endverse
\ifchorded
\beginverse*
\vspace*{-\versesep}
{\nolyrics \[G]\[D]\[Em]\[C]\[G]}
\endverse
\fi
\endsong


%titolo{Come è bello, come dà gioia}
%autore{Marciani}
%album{Dio della mia lode}
%tonalita{La-}
%gruppo{}
%momenti{Salmi;Ingresso;Congedo}
%identificatore{come_e_bello_come_da_gioia}
%data_revisione{2011_12_31}
%trascrittore{Francesco Endrici - Manuel Toniato}
\beginsong{Come è bello, come dà gioia}[by={Marciani}]

\beginchorus
\[Am]Come è \[Dm]bello, \[Am]come dà \[E]gioia
\[Am]che i fra\[Dm]telli \[Am]stia\[E]no in\[Am]sieme.
\endchorus

\beginverse
\[Am]È come un\[G]guento che dal \[F]capo
di\[E]scende \[Am]giù sulla \[Dm]barba di A\[E7]ronne. \rep{2}
\endverse

\beginverse
\chordsoff
È come unguento che dal capo
discende giù sugli orli del manto. \rep{2}
\endverse

\beginverse
\chordsoff
Come rugiada che dall'Ermon discende
giù sui monti di Sion. \rep{2}
\endverse

\beginverse
\chordsoff
Ci benedica il Signore dall'alto:
la vita ci dona in eterno. \rep{2}
\endverse
\endsong

%titolo{Come è grande}
%autore{Grotti}
%album{Canti per l'assemblea cristiana}
%tonalita{Re-}
%gruppo{}
%momenti{Salmi;Quaresima}
%identificatore{come_e_grande}
%data_revisione{2011_12_31}
%trascrittore{Francesco Endrici - Manuel Toniato}
\beginsong{Come è grande}[by={Grotti}]

\beginverse
\[Dm]Come è \[Gm]grande la \[Dm]tua bon\[A]tà,
\[F]che con\[Gm]servi per \[C]chi ti \[F]teme.
\[Gm]E fai grandi \[Dm]cose per chi \[C]ha rifugio in \[F]te,
\[Gm]e fai grandi \[Dm]cose per chi a\[A]ma solo \[Dm]te. \[Dm4] 
\endverse

\beginverse
\chordsoff
Come un vento silenzioso,
ci hai raccolto dai monti e dal mare;
come un'alba nuova tu sei venuto a me,
la forza del tuo braccio mi ha voluto qui con te.
\endverse

\beginverse
\chordsoff
Com'è chiara l'acqua alla tua fonte
per chi ha sete ed è stanco di cercare:
sicuro di trovare i segni del tuo amore
che si erano perduti nell'ora del dolore.
\endverse

\beginverse
\chordsoff
Come un fiore nato tra le pietre
va a cercare il cielo su di lui,
così la tua grazia, il tuo Spirito per noi
nasce per vedere il mondo che tu vuoi.
\endverse
\endsong



%titolo{Come fuoco per il mondo}
%autore{}
%album{}
%tonalita{Do}
%gruppo{}
%momenti{Comunione}
%identificatore{come_fuoco_per_il_mondo}
%data_revisione{2011_12_31}
%trascrittore{Francesco Endrici}
\beginsong{Come fuoco per il mondo}
\beginverse
Il \[C]mondo che nasce ogni \[G]giorno alla vita
il mio \[F]sguardo non lo conter\[C]rà.
Il \[C]mondo che cresce e come un \[G]seme germoglia
il mio \[F]cuore non lo porte\[C]rà.
Il \[F]mondo che ha fame, lotta e \[C]cerca giustizia,
la mia \[G]vita non lo salve\[C]rà.
Ma se \[C]alzo lo sguardo ed al\[G]largo il mio cuore
questo \[F]mondo lui mi affide\[G]rà.
\endverse
\beginchorus
\[C]Dio cerca \[G]te come \[F]fuoco per il \[C]mondo,
\[C]santo come \[G]lui per cam\[F]biare il male in \[G]bene.
\[C]Dio vuole \[G]te appassio\[F]nato fino in \[C]fondo
per la \[G]vita di ogni \[F]uomo che con \[G]te lui salve\[C]rà.
\endchorus
\beginverse
%\chordsoff
Pre^gare col cuore è un gran ^gesto d'amore,
sono un ^uomo e ho bisogno di ^Te.
Fa^tica e dolore è co^struire la storia,
il mio ^posto nel mondo qual ^è.
^Dare la vita con ^tutti i miei doni,
fe^dele nella cari^tà.
^Gioco la libertà obbe^diente alla vita,
il suo ^spirito mi accende^rà.
\endverse
\endsong

%titolo{Come fuoco vivo}
%autore{Gen Verde, Gen Rosso}
%album{Come fuoco vivo}
%tonalita{Do}
%gruppo{}
%momenti{Comunione}
%identificatore{come_fuoco_vivo}
%data_revisione{2011_12_31}
%trascrittore{Francesco Endrici}
\beginsong{Come fuoco vivo}[by={Gen\ Verde, Gen\ Rosso}]
\ifchorded
\beginverse*
\vspace*{-0.8\versesep}
{\nolyrics \[C] \[Dm/C] \[C] \[G/B] \[Am7] \[Em7] \[F/C] \[C]}
\vspace*{-\versesep}
\endverse
\fi
\beginchorus
Come \[C]fuoco \[G]vivo si ac\[Am]cende in \[Am]noi
un'im\[Dm7]mensa \[G]felici\[C]tà \[C]
che mai \[F]più nes\[G]suno ci \[C]toglie\[F]rà
\[Dm7]perché tu \[Dm7]sei ritor\[G4]nato. \[G]
Chi po\[C]trà ta\[G]cere, da \[Am]ora in \[Am]poi,
che sei \[Dm]tu in cam\[G]mino con \[C]noi, \[C]
che la \[F]morte è \[G]vinta per \[C]sempre,
\[F]che \[Dm7]ci hai rido\[Dm7]nato la \[G4]vita? \[G]\[Am]
\endchorus
\beginverse
\memorize
Spezzi il \[Am]pane da\[F]vanti a \[C]noi \[C]
mentre il \[C]sole è al tra\[G4]monto: \[G]
\[Gm]o\[Gm]ra gli \[A]occhi ti \[Dm]vedono, \[F] sei \[F]tu! 
\[G4]Resta con \[G]noi.
\endverse
\beginverse
%\chordsoff
E per ^sempre ti ^mostre^rai ^
in quel ^gesto d'a^more: ^
^ma^ni che ^ancora ^spezzano ^ pa^ne d'^eterni^tà.
\endverse
\ifchorded
\beginverse*
\vspace*{-\versesep}
{\nolyrics \[C] \[Dm/C] \[C] \[G/B] \[Am7] \[Em7] \[F/C] \[C]}
\endverse
\fi
\endsong

%titolo{Come il cervo va}
%autore{Hurd, Kingsbury, Deflorian}
%album{}
%tonalita{Re-}
%gruppo{}
%momenti{Salmi}
%identificatore{come_il_cervo_va}
%data_revisione{2011_12_31}
%trascrittore{Francesco Endrici}
\beginsong{Come il cervo va}[by={Hurd, Kingsbury, Deflorian}]
\ifchorded
\beginverse*
\vspace*{-0.8\versesep}
{\nolyrics \[Dm]\[C]\[B&7+]\[Am] \[Dm]\[C]\[B&7+]\[Am]}
\vspace*{-\versesep}
\endverse
\fi
\beginchorus
\[Dm]Come il cervo va all'\[B&7+]acqua \[Am7]viva \[F]io cerco \[Am]Te
\[B&7+]ardente\[Dm]mente, \[Gm7]io cerco \[Am7]Te mio \[Dm]Dio. \[C] \[B&]\[Am]
\endchorus
\beginverse
\memorize
Di \[Dm]Te mio Dio ha sete l'\[B&7+]anima \[Am7]mia!
\[F]Il Tuo vol\[Am7]to, \[B&7+]il Tuo vol\[Dm]to, Si\[Gm7]gnore
\[Am7]quando ve\[Dm]drò? \[B&]\[Am]
\endverse
\beginverse
\chordsoff
Mi chiedono e mi tormentano: “dov'è,
dov'è il tuo Dio?” Ma io spero in Te, sei Tu la
mia salvezza.
\endverse
\beginverse
\chordsoff
Il cuore mio si strugge quando si ricorda
della Tua casa: io cantavo con gioia la tue lodi.
\endverse
\beginverse
\chordsoff
A Te io penso e rivedo quello che hai fatto Tu
per me: grandi cose, Signore mio Dio.
\endverse
\beginverse
\chordsoff
Ti loderò Signore e Ti canterò il mio grazie.
Tu sei fresca fonte, l'acqua della mia vita.
\endverse
\endsong

%titolo{Come in cielo così in terra}
%autore{Spoladore}
%album{Come in cielo così in terra}
%tonalita{Re}
%gruppo{}
%momenti{Comunione}
%identificatore{come_in_cielo_cosi_in_terra}
%data_revisione{2011_12_31}
%trascrittore{Francesco Endrici - Manuel Toniato}
\beginsong{Come in cielo così in terra}[by={Spoladore}]

\beginverse
Ti pre\[D]ghiamo, Signore del cielo,
Tu sei \[Em]guida al nostro cammino
sei sor\[F#m]gente di vita e di amore
sei l'a\[G]mico in terra straniera
sei la \[Em]strada di chi s'è perduto
la \[F#m]festa per chi è ritor\[A4]nato. \[A] 
\endverse

\beginverse
\chordsoff
Tu consoli chi piange tristezza
dai la pace a chi cerca la pace
compassione per chi sa soffrire
con amore abbraccia la croce
sei il tesoro di chi sa cercare
il Risorto che vogliamo incontrare.
\endverse

\beginchorus
Come in \[B&7]cielo così in \[C]terra 
\[F]guidaci tu Si\[B&]gnore.
\[Gm]Resta per sempre con \[A7]noi!
\endchorus

\beginverse
\chordsoff
Sei speranza e dolce certezza
per chi pensa col cuore di Dio;
sei letizia e gioia profonda
per chi soffre tradito dal mondo;
dai coraggio e forza infinita
ai tuoi piccoli e umili figli.
\endverse

\beginverse
\chordsoff
Il tuo regno è un cammino di fede:
cielo e terra ci fai abbracciare.
È il contagio del nuovo messaggio
che tu hai vinto il male del mondo;
è il tuo Spirito Santo d'amore
che ogni giorno ci cresce nel cuore.
\endverse
\endsong


%titolo{Come l'aurora verrai}
%autore{Gen Verde}
%album{Cerco il tuo volto}
%tonalita{Re-}
%gruppo{}
%momenti{Avvento}
%identificatore{come_l_aurora_verrai}
%data_revisione{2011_12_31}
%trascrittore{Francesco Endrici - Manuel Toniato}
\beginsong{Come l'aurora verrai}[by={Gen\ Verde}]
\beginverse
\[Dm]Come l'au\[Am7/C#]rora ver\[Dm]rai
le tenebre in \[Am]luce cambie\[B&]rai
tu per \[F/A]noi, Si\[B&]gno\[C]re.
\[Dm]Come la pi\[Am7/C#]oggia ca\[Dm]drai
sui nostri de\[Am]serti scende\[B&]rai
scorre\[F/A]{rà l'a}\[B&]{mo}\[C]re
\endverse

\beginchorus
\[B&]Tutti i nostri senti\[C]eri percorrer\[Dm]{ai,} \[Am] \[Dm] 
\[B&]tutti i figli dis\[C]persi raccoglier\[Dm]{ai,} \[Am] \[Dm] 
\[B&]chiamerai da ogni \[C]terra
il tuo \[A]popo\[B&]lo,
\[Gm]in eterno ti a\[Gm7]vremo con \[Gm/C]{noi.} \[C] 
\endchorus

\beginverse
\chordsoff
Re di giustizia sarai,
le spade in aratri forgerai:
ci darai la pace.
Lupo ed agnello vedrai
insieme sui prati dove mai
tornerà la notte.
\endverse

\beginverse
\chordsoff
Dio di salvezza tu sei
e come una stella sorgerai
su di noi per sempre.
E chi non vede, vedrà,
chi ha chiusi gli orecchi sentirà,
canterà di gioia.
\endverse
\endsong


%titolo{Come lampade fragili}
%autore{Cavallin}
%album{}
%tonalita{Re}
%gruppo{}
%momenti{Comunione}
%identificatore{come_lampade_fragili}
%data_revisione{2011_12_31}
%trascrittore{Francesco Endrici - Manuel Toniato}
\beginsong{Come lampade fragili}[by={Cavallin}]

\beginverse
\[D]Tu da sempre ci at\[F#m]tendi 
e abbracci i tuoi \[G]figli che vengono a \[A]te,
\[D]noi di te abbiam bi\[F#m]sogno 
più di ogni ric\[G]chezza, Signore Ge\[A]sù.
Tu riempi di \[F#m]vita noi poveri \[Bm]cuori,
sei tu la spe\[Em]ranza che mai fini\[A]rà
e noi corre\[F#m]remo senza stan\[Bm]carci 
fino alla \[Em]meta, Signore Ge\[A]sù.
\endverse

\beginchorus
\[D]Come lampade \[A/C#]fragili brilla in \[G]noi la tua luce,
siamo \[D]terre di cielo;
Tu per \[Bm]noi ti sei \[F#m]fatto peccato
nella \[G]croce hai sve\[C#]lato il tuo \[F#]amore per \[Bm]noi,
o Si\[G]gnore Ge\[Gm]sù, pane \[D7+]che dai la \[Bm]vita
dono im\[Em]menso del \[A]Padre per \[D]noi.
\endchorus

\beginverse
\chordsoff
Tu che abbassi l'orgoglio 
di voler far da soli, senza di te.
Tu che doni il perdono 
a noi che scappiamo lontani da te.
Tu non hai voluto difenderti mai,
ci ami da sempre, Signore Gesù,
e in ogni battaglia non ci lasci da soli,
ma a fianco combatti insieme con noi.
\endverse

\beginverse
\chordsoff
Tu l'Amore infinito 
che spezza ogni morte, che salva dal male.
Tu mistero nascosto 
che svela ad ogni uomo il volto di Dio.
Tu salvi gli oppressi da tutte le angosce
e fai maturare chi vuol camminare.
Tu liberi il cuore da ogni potere,
ci rendi fratelli, Signore Gesù.
\endverse

\beginchorus
\[D]Come lampade \[A/C#]fragili brilla in \[G]noi la tua luce,
siamo \[D]terre di cielo;
Tu per \[Bm]noi ti sei \[F#m]fatto peccato
nella \[G]croce hai sve\[C#]lato il tuo \[F#]amore per \[Bm]noi,
o Si\[G]gnore Ge\[Gm]sù, pane \[D7+]che dai la \[Bm]vita
dono im\[Em]menso del \[A]Padre per \[D]noi
o Si\[Gm]gnore Gesù, pane \[D7+]che dai la \[Bm]vita
dono im\[Em]menso del \[A]Padre per \[D]noi.
\endchorus
\endsong


%titolo{Come Maria}
%autore{Gen Rosso}
%album{Dove Tu sei}
%tonalita{La-}
%gruppo{}
%momenti{Maria;Offertorio}
%identificatore{come_maria_gen}
%data_revisione{2011_12_31}
%trascrittore{Francesco Endrici}
\beginsong{Come Maria}[by={Gen\ Rosso}]
\beginverse
\[Am] Vogliamo vivere Si\[C]gnore, \[Am]
offrendo a te la nostra \[Em]vita, \[F]
Con questo pane e questo \[D]vino \[Am]
accetta quello che noi \[E]siamo. \[Am]
Vogliamo vivere Si\[C]gnore, \[Am]
abbandonati alla tua \[Em]voce, \[F]
staccati dalle cose \[D]vane, \[Am]
fissati nella vita \[E4]vera. \[E]
\endverse
\beginchorus
Vo\[A]gliamo \[D]vivere \[E] come Ma\[F#m]ria \[D]
l'irraggiun\[E]gibile, \[D] la madre a\[E]mata, \[D]
che vince il \[E]mondo con l'a\[C#m]more \[D]
e offrire s\[Bm]empre la Tua \[C#]vita che \[D]viene dal \[A]cielo.
\endchorus
\beginverse
%\chordsoff
^ Accetta dalle nostre ^mani ^
come un'offerta a te gra^dita ^
i desideri di ogni ^cuore, ^
le ansie della nostra ^vita. ^
Vogliamo vivere Si^gnore, ^
accesi dalle Tue Pa^role, ^
per riportare in ogni ^uomo ^
la fiamma viva del Tuo a^more. ^
\endverse
\beginchorus
Finale:
e offrire \[Bm]sempre la Tua \[C#]vita che \[D]viene dal \[F#m]cielo.
\endchorus
\endsong

%titolo{Come tu mi vuoi}
%autore{Branca}
%album{Io scelgo te}
%tonalita{Sol}
%gruppo{}
%momenti{Comunione;Salmi}
%identificatore{come_tu_mi_vuoi}
%data_revisione{2011_12_31}
%trascrittore{Francesco Endrici - Manuel Toniato}
\beginsong{Come tu mi vuoi}[by={Branca}]
\ifchorded
\beginverse*
\vspace*{-0.8\versesep}
{\nolyrics \[G] \[C/G] \[D7/G] \[G]  \[Em] \[Am7] \[D4] \[G]}
\vspace*{-\versesep}
\endverse
\fi

\beginverse
\[G]Eccomi Signor, \[Am7/G]vengo a te mio Re
\[Em]che si compia in me la tua \[G]volon\[D]tà
\[G]Eccomi Signor, \[Am7/G]vengo a te mio Dio
\[Em]plasma il cuore mio \[Bm7]e di te vivrò
\[G/C]Se tu lo \[C]vuoi Si\[D/C]gnore manda \[Em/C]me
\[Am/D]e il tuo nome an\[Bm]{nun}\[G]{ce}\[D]rò.
\endverse

\beginchorus
Come tu mi \[G]vuoi io sa\[D/F#]rò,
dove tu mi \[Em]vuoi io an\[Bm7]drò.
Questa \[C]vita io voglio do\[Bm7]narla a \[Em]te
per dar \[F]gloria al tuo \[C]nome mio \[D4]Re.
\[D]Come tu mi \[C7+]vuoi io sa\[D/F#]rò,
\[B7/D#]dove tu mi \[Em]vuoi io an\[Bm7]drò.
Se mi \[C7+]guida il tuo a\[D4]more pa\[B7]ura non \[Em]ho,
per \[AmSim]sempre \[C]{io} sa\[D4]rò
\[D]come tu mi \[G]vuoi.
\endchorus

\ifchorded
\beginverse*
\vspace*{-\versesep}
{\nolyrics \[G] \[C/G] \[D7/G] \[G] }
\endverse
\fi

\beginverse
\chordsoff
Eccomi Signor, vengo a te mio Re
che si compia in me la tua volontà
Eccomi Signor, vengo a te mio Dio
plasma il cuore mio e di te vivrò
Tra le tue mani mai più vacillerò
e strumento tuo sarò.
\endverse

\ifchorded
\beginverse*
\vspace*{-\versesep}
{\nolyrics \[C2] \[C] \[D4] \[D] (ad libitum) \[C] \[G] }
\endverse
\fi
\endsong

%titolo{Come un fiume}
%autore{Paci, Preti}
%album{}
%tonalita{Do}
%gruppo{}
%momenti{}
%identificatore{come_un_fiume}
%data_revisione{2011_12_31}
%trascrittore{Francesco Endrici}
\beginsong{Come un fiume}[by={Paci, Preti}]
\beginchorus
Come un \[C]fiume in piena che \brk la sabbia \[G]non può arrestare
come l’\[C7]onda che dal mare si di\[F]stende sulla riva
ti pre\[Fm]ghiamo Padre \brk che così si \[C]sciolga il nostro amore
e l’a\[D7]more dove arriva sciolga il \[F]dubbio e la paura. \[(G7)]
\endchorus
\beginverse
Come un \[C]pesce che risale a nuoto \[G]fino alla sorgente
va a sco\[C7]prire dove nasce e si di\ch{F}{f}{f}{ff}onde la sua vita
ti pre\[Fm]ghiamo Padre che noi risa\[C]liamo la corrente
fino ad \[G]arrivare alla vita \[F]nell’a\[C]more.  \[(G7)]
\endverse
\beginchorus
Come un \[C]fiume in piena che \brk la sabbia \[G]non può arrestare
come l’\[C7]onda che dal mare si di\[F]stende sulla riva
ti pre\[Fm]ghiamo Padre \brk che così si \[C]sciolga il nostro amore
e l’a\[D7]more dove arriva sciolga il \[F]dubbio e la paura. \[(G7)]
\endchorus
\beginverse
Come l’^erba che germoglia cresce ^senza far rumore
ama il ^giorno della pioggia si addor^menta sotto il sole
ti pre^ghiamo Padre che così in un ^giorno di silenzio
anche in ^noi germogli questa vita ^nell’a^more.  \[A7]
\endverse
\transpose{2}
\beginchorus
Come un \[C]fiume in piena che la sabbia \[G]non può arrestare
come l’\[C7]onda che dal mare si di\[F]stende sulla riva
ti pre\[Fm]ghiamo Padre \brk che così si \[C]sciolga il nostro amore
e l’a\[D7]more dove arriva sciolga il \[F]dubbio e la paura. \[(G7)]
\endchorus
\beginverse
Come un ^albero che affonda le ra^dici nella terra
e su ^quella terra un uomo costru^isce la sua casa
ti pre^ghiamo Padre buono di por^tarci alla tua casa
dove ^vivere una vita piena ^nell’a^more. 
\endverse
\endsong

%titolo{Con Francesco profeti fra la gente}
%autore{di Fatta}
%album{}
%tonalita{Sol}
%gruppo{}
%momenti{Congedo}
%identificatore{con_francesco_profeti_fra_la_gente}
%data_revisione{2011_12_31}
%trascrittore{Francesco Endrici}
\beginsong{Con Francesco profeti fra la gente}[by={di\ Fatta}]
\beginverse
Quando an\[G]date per le strade del mondo, lodatelo,
quando \[Am]siete ancora lungo la via, amatelo,
quando \[C]stanchi vi buttate per \[D]terra, pre\[C]gatelo. \[G]
\chordsoff
Quando an\[G]cora tu sei pieno di forze, ringrazialo,
quando il \[Am]cuore dice “non ce la fai”, invocalo,
quando ar\[C]rivi e ti senti fe\[D]lice, a\[C]doralo. \[G]
\endverse
\beginchorus
Il \[G]Signore Dio \textsf{Loda, loda, lodatelo}
Onnipo\[Am]tente \textsf{Loda, loda, lodatelo}
\[C]Noi \[D] vi annun\[G]ciamo \[D] \textsf{Alleluia, Alleluia}
Con semplici\[G]tà \textsf{Loda, loda, lodatelo}
E fedel\[Am]tà \textsf{Loda, loda, lodatelo}
Lo\[C]date \[D] il Si\[G]gnor. \textsf{Alleluia, Alleluia}
\endchorus
\beginverse
\chordsoff
Quando ^vedi un fratello un po' triste, sorridigli
se ti ac^corgi ch'è rimasto un po' indietro,aspettalo,
se il suo ^zaino è troppo pe^sante, a^iutalo. ^
Alla ^gente che per strada incontri, annuncialo,
solo ^Lui è la gioia e l'amore, proclamalo,
con il ^canto e la pace nel ^cuore, di^mostralo. ^
\endverse
\beginverse
\chordsoff
Se il fra^tello ha sbagliato in qualcosa, perdonalo,
quando in^vece è sotto il tuo peccato, convertiti,
perché ^Cristo rinnovati nel ^cuore ci as^petta già. ^
Con Fran^cesco il giullare di Dio, Alleluia,
noi pro^feti tra la gente saremo, Alleluia,
per lo^dare e annunciare il Si^gnore, Al^leluia. ^
\endverse
\endsong

%titolo{Con il mio canto}
%autore{Sequeri}
%album{E mi sorprende}
%tonalita{Re}
%gruppo{}
%momenti{Comunione}
%identificatore{con_il_mio_canto}
%data_revisione{2011_12_31}
%trascrittore{Francesco Endrici - Manuel Toniato}
\beginsong{Con il mio canto}[by={Sequeri}]

\beginverse
\[D]Con il mio canto, \[Em]dolce Signore,
\[F#m]voglio danzare \[Bm]questa mia \[F#m]gioia,
\[G]voglio destare \[Em]tutte le cose,
\[A]un mondo nuovo \[A7]voglio cantare.

\chordsoff
Con il mio canto, dolce Signore,
voglio riempire lunghi silenzi,
voglio abitare sguardi di pace,
il tuo perdono voglio cantare.
\endverse

\beginchorus
\[D]Tu \[Em]sei per \[F#m]me, \[Bm]co\[F#m]me un \[G]can\[Em]to d'a\[A7]more
\[D]re\[Em]sta con \[F#m]noi \[Bm]fi\[F#m]no al \[G]nuo\[Em]vo mat\[A7]tino\[(Re)].
\endchorus

\beginverse
\chordsoff
Con il mio canto, dolce Signore,
voglio plasmare gesti d'amore
voglio arrivare oltre la morte,
la tua speranza voglio cantare.

Con il mio canto, dolce Signore
voglio gettare semi di luce,
voglio sognare cose mai viste,
la tua bellezza voglio cantare.
\endverse

\beginverse
\chordsoff
Se tu mi ascolti - dolce Signore-
questo mio canto - sarà una vita,
e sarà bello - vivere insieme,
finché la vita - un canto sarà.
\endverse
\endsong

%titolo{Con Te faremo cose grandi}
%autore{Meregalli}
%album{}
%tonalita{Fa}
%gruppo{}
%momenti{Congedo}
%identificatore{con_te_faremo_cose_grandi}
%data_revisione{2011_12_31}
%trascrittore{Francesco Endrici}
\beginsong{Con Te faremo cose grandi}[by={Meregalli}]
\beginchorus
Con \[F]Te fa\[C]remo cose \[Dm]gran\[G]di
il cam\[B&]mino che per\[C]correremo in\[Am]sie\[C]me
\[C7]di \[F]Te si \[C]riempiranno \[Dm]sguar\[G]di
la spe\[B&]ranza che ri\[C]splenderà nei \[Am]vol\[C]ti.
\endchorus
\beginverse
Tu la \[B&]luce che ri\[C]schiara,
Tu la \[B&]voce che ci \[C]chiama,
Tu la \[B&]gioia che dà \[C]vita ai nostri \[Am]so\[C]gni.

\[F]Parlaci Si\[C]gnore come \[B&]sai,
sei pre\[C]sente nel mi\[C7]stero in mezzo a \[F]noi. \[C]
\[F]Chiamaci col \[C]nome che vor\[B&]rai
e sia \[C]fatto il tuo di\[C7]segno su di \[F]noi. \[C]

Tu la \[B&]luce che ri\[C]schiara\dots

\endverse
\beginchorus
\chordsoff
Rit. 
\endchorus
\beginverse
\chordsoff
Tu l'a^more che dà ^vita,
Tu il sor^riso che ci al^lieta,
Tu la ^forza che ra^duna i nostri ^gior^ni.

^Guidaci Si^gnore dove ^sai
da chi ^soffre chi è più ^piccolo di ^noi ^
stru^menti di quel ^regno che Tu ^fai,
di quel ^regno che ora ^vive in mezzo a ^noi. ^

Tu l'a^more che dà ^vita\dots
\endverse
\beginchorus
\chordsoff 
Rit. 
\endchorus
\beginverse
Tu l'am\[B&]ore che dà \[C]vita,
Tu il sor\[B&]riso che ci al\[C]lieta,
Tu la \[B&]forza che ra\[C]duna i nostri \[B&]gior\[F]ni.
\endverse
\endsong



%titolo{Con un cuore solo}
%autore{Ferrante}
%album{Sentieri di speranza}
%tonalita{Sol}
%gruppo{}
%momenti{Ingresso;Congedo}
%identificatore{con_un_cuore_solo}
%data_revisione{2011_12_31}
%trascrittore{Francesco Endrici - Manuel Toniato}
\beginsong{Con un cuore solo}[by={Ferrante}]

\ifchorded
\beginverse*
\vspace*{-0.8\versesep}
{\nolyrics \[C/G]   \[G]   \[C]   \[C/D]    \[C/G]   \[G/B]   \[A7]   \[D] 
\[F/G]   \[G7/B]   \[C]   \[A7/C#]  \[G/D]   \[Em7]   \[Am7]   \[G] }
\vspace*{-\versesep}
\endverse
\fi

\beginchorus
Con un cuore \[C/G]so\[G]lo \brk cantiamo alla \[C]gloria di \[C/D]Dio,
con un'anima \[C/G]so\[G/B]la \brk noi diamo \[A7]lode al Si\[D4]gnor.
Come incenso \[F/G]salga al \[G7/B]cielo oggi \brk \[C]questa nostra \[A7/C#]lode,
for\[G/D]miamo un sol \[Em7]corpo in Cristo \[Am7/D]il Si\[G]gnor.

Con un cuore \[C/G]so\[G]lo \brk cantiamo alla \[C]gloria di \[C/D]Dio,
con un'anima \[C/G]so\[G/B]la \brk noi diamo \[A7]lode al Si\[D4]gnor.
Come incenso \[F/G]salga al \[G7/B]cielo  oggi \brk \[C]questa nostra \[A7/C#]lode,
for\[G/D]miamo un sol \[Em7]corpo uniti in \[Am7]Cri\[G/B]sto \[C/D]il \[B7/D#]Si\[Em]gnor.
\endchorus

\beginverse
\[Em] Signore, il \[D/F#]popolo \[G]tuo è riunito
\[C] per fare un \[D]solo corpo \[Em]che dia \[D]lode a te.
\[G] Ogni di\[Am]stanza in Te \[C]non ha misura,
noi \[Am7]siamo \[Bm]la tua \[C]Chiesa, Ge\[Am7/D]sù.
\endverse

\beginverse
\chordsoff
Sui tuoi sentieri di pace e amore
Noi camminiamo uniti verso te Signore.
Nella Parola tua saremo salvi,
rafforza in noi la fede, Gesù.
\endverse

\beginverse
\chordsoff
Che il mondo ci riconosca, Signore,
dal modo in cui ci ameremo gli uni e gli altri.
Nel volto di ogni uomo vediamo
La tua presenza viva, Gesù.
\endverse

%\beginchorus
%\[Bm7/E] Con un cuore \[D/A]{so}\[A]lo, 
%\[D] con un'anima \[D/A]{so}\[A7/C#]{la} \[B7] 
%\[Bm7/E] cantiamo in\[A]{sie}\[A7/C#]{me,} in\[D]{sie}\[B/D#]{me.}
%for\[C#m7]miamo un \[F#m7]sol corpo in Cristo \[Bm7/E]il Signore,     

%Can\[D/A]{tia}\[A]{mo} \[A/C#]  \[D] 
%\[Bm7] con un'anima \[D/A]{so}\[A7/C#]{la} \[B7] 
%\[B7/E] cantiamo in\[G/A]{sie}\[A7/C#]{me,} in\[D]{sie}\[B7/D]me.
%for\[C#m7]miamo un \[F#m7]sol corpo in Cristo \[Bm7/E]il Signore,     

%In\[D/A]{sie}\[A]{me} \[A/C#]cantiamo alla \[D]gloria di \[Bm7/E]Dio,
%con un'anima \[D/A]{so}\[A7/C#]la noi diamo \[B7]lode al Si\[Bm7/E]gnor.
%come incenso \[G/A]salga al \[A7/C#]cielo oggi \[D]questa nostra \[B7/D#]lode,
%for\[C#m7]miamo un \[F#m7]sol corpo uniti in \[Bm7/E]Cristo il Si\[A/C#]{gno}\[D]re,     
%il Si\[E/D]{gno}\[Dm7]re Cristo Si\[G/D]{gno}\[A]re.
%\endchorus

\endsong


%titolo{Con voce di giubilo}
%autore{Cento}
%album{Eucaristia sul mondo}
%tonalita{Fa}
%gruppo{}
%momenti{Ingresso;Congedo}
%identificatore{con_voce_di_giubilo}
%data_revisione{2011_12_31}
%trascrittore{Francesco Endrici - Manuel Toniato}
\beginsong{Con voce di giubilo}[by={Cento}]
\ifchorded
\beginverse*
\vspace*{-0.8\versesep}
{\nolyrics \[F4] \[F] \[F4] \[F] \[C4] \[C] \[C4] \[C] \[B&] \[F/A] \[G4] \[C4] }
\vspace*{-\versesep}
\endverse
\fi
\beginchorus
Con \[F]vo\[Gm]ce di \[F]giubilo \[B&]date il grande an\[C4]nun\[C]cio,
\[B&]fatelo \[F]giungere ai con\[Gm7]fini del \[C4]mon\[C]do.
Con \[F]vo\[Gm]ce di \[F/A]giubilo \[B&]date il grande an\[C4]nun\[C]cio,
\[B&]il Signore ha libe\[F/A]rato il suo \[Gm]po\[C]po\[F]lo.
\endchorus

\beginverse
Lo\[F]date \[Gm]il Si\[F]gnore egli è \[Gm]buono.
Egli ha \[C]fatto \[B&]mera\[F]viglie, alle\[Gm7]luia.  \[Gm7/C] 
\endverse

\beginverse
\chordsoff
Eterna è la sua misericordia
nel suo nome siamo salvi, alleluia.
\endverse

\beginverse
\chordsoff
La sua gloria riempie i cieli e la terra,
è il Signore della vita, alleluia.
\endverse

\beginverse*
\itshape\[F]Alle\[B&]luia, \[F]alle\[B&]luia\[F],
il Si\[B&]gnore ha libe\[F]rato il suo \[Gm7]popolo.\[Gm7/C] 
\[F]Alle\[B&]luia, \[F]alle\[B&]luia\[F],
il Si\[B&]gnore ha libe\[F]rato il suo \[Gm7]popolo.\[Gm7/C] 
\[F]Alle\[B&]luia, \[F]alle\[B&]luia\[F],
\endverse
\endsong

%titolo{Cosa offrirti}
%autore{Branca, Ciancio}
%album{Voglio vedere il tuo volto}
%tonalita{Do}
%gruppo{}
%momenti{Offertorio}
%identificatore{cosa_offrirti}
%data_revisione{2011_12_31}
%trascrittore{Francesco Endrici - Manuel Toniato}
\beginsong{Cosa offrirti}[by={Branca, Ciancio}]
\ifchorded
\beginverse*
\vspace*{-0.8\versesep}
{\nolyrics %
\[F7+] \[C/E] \[Dm] \[A/C#] \[F7+] \[C/E] \[G4] \[G] }
\vspace*{-\versesep}
\endverse
\fi

\beginverse
\[Cm]Cosa offrirti, o \[G/B]Dio, \[Gm/B&]cosa posso \[F/A]darti,
\[Fm/Ab]eccomi son \[Cm/G]qui davanti a \[D7/F#]te. \[G4]  \[G] 
\[Cm]Le gioie ed i do\[G/B]lori, gli a\ch{Gm/B&}{f}{f}{ff}anni di ogni \[F/A]giorno,
\[Fm/Ab]tutto voglio \[C/G]vivere in \[G4]te. \[C/E] 
\endverse

\beginchorus
Ac\[F7+]cetta, mio \[C/E]Re, questo \[Dm]poco \[A/C#]che ho,
\[F7+]offro a \[C/E]te la mia \[B&]vi\[G4]ta, \[C/E] 
\[F7+]gioia è per \[C/E]me far la \[E/G#]tua volon\[Am]tà,
il mio \[F]unico \[C/E]bene \[Dm]sei solo \[G4]tu, 
\[G]solo \[F7+]tu. \[C/E] \[Dm] \[A/C#] \[F7+] \[C/E] \[G4] \[G] 
\endchorus

\beginverse
\chordsoff
Vengo a te mio Dio, apro le mie braccia
che la tua letizia riempirà.
Rinnova questo cuore perché ti sappia amare
e nella tua pace io vivrò.
\endverse

\beginchorus
Ac\[F7+]cetta, mio \[C/E]Re, questo \[Dm]poco \[A/C#]che ho,
\[F7+]offro a \[C/E]te la mia \[B&]vi\[G4]ta, \[C/E] 
\[F7+]gioia è per \[C/E]me far la \[E/G#]tua volon\[Am]tà,
il mio \[F]unico \[C/E]bene \[Dm]sei solo \[G4]tu, 
\[G]solo \[F7+]tu. \brk \[C/E] \[Dm] \[A/C#] \[F7+] \[C/E] \[G4] \[C/E] 
\[F7+]gioia è per \[C/E]me far la \[E/G#]tua volon\[Am]tà,
il mio \[F]unico \[C/E]{bene} \[Dm]sei solo \[G4]tu. \[G] 
Ac\[G]cet\[D]ta, \[G]mio \[D]Re, questo \[Em]poco che \[B/D#]ho,
\[G7+]offro a \[D/F#]te la mia \[C]vi\[A4]ta, \[D/F#] 
\[G]gioia è per \[D/F#]me far la \[F#7/A#]tua volon\[Bm]tà,
il mio \[G7+]unico \[D/F#]bene \[Em]sei solo \[A4]tu, \brk \[A]solo \[G]tu. \[D/F#]  \[Em7]  \[D] 
\endchorus
\endsong


%titolo{Cristo è morto ed è risorto}
%autore{Comi}
%album{Per ogni uomo}
%tonalita{Do}
%gruppo{}
%momenti{Pasqua}
%identificatore{cristo_e_morto}
%data_revisione{2011_12_31}
%trascrittore{Francesco Endrici - Manuel Toniato}
\beginsong{Cristo è morto ed è risorto}[by={Comi}]

\beginverse
\[C]Cristo è \[Am]morto ed \[C]è ri\[Am]sorto
\[C]per ogni \[Am]uomo nei \[F]se\[G7]co\[C]li. 
\[F]Alle\[G7]luia, \[C]alle\[Am]luia: 
\[F]per ogni \[C]uomo nei \[G7]seco\[C]li. \[A7]
\endverse

\beginverse
\[D]Cristo è \[Bm]morto \[D]ed è ri\[Bm]sorto
\[D]per ogni \[Bm]uomo nei \[G]se\[A7]co\[D]li.
\[G]Alle\[A7]luia, \[D]alle\[Bm]luia: 
\[G]per ogni \[D]uomo nei \[A7]seco\[D]li. \[B7]
\endverse

\beginverse
\[E]Cristo è \[C#m]morto \[E]ed è ri\[C#m]sorto
\[E]per ogni \[C#m]uomo nei \[A]se\[B7]co\[E]li.
\[A]Alle\[B7]luia, \[E]alle\[C#m]luia: 
\[A]per ogni \[E]uomo nei \[B7]seco\[E]li.  
\endverse
\endsong

%titolo{Cristo è risorto veramente}
%autore{Giottoli}
%album{Risorto per amore}
%tonalita{Re}
%gruppo{}
%momenti{Pasqua;Ingresso;Congedo}
%identificatore{cristo_e_risorto_veramente}
%data_revisione{2011_12_31}
%trascrittore{Francesco Endrici}
\beginsong{Cristo è risorto veramente}[by={Giottoli}]
\transpose{3}
\beginchorus
\[D]Cristo è ri\[A]sorto vera\[D]mente alle\[A]luia
\[D]Gesù il vi\[A]vente qui con \[Bm]noi reste\[A4]rà \[A]
\[Bm]Cristo Ge\[G]sù, \[D]Cristo Ge\[G]sù
\[D]è il Si\[A]gnore della \[G]vi\[D]ta.
\endchorus
\beginverse
\[D]Morte dov'\[G]è la tua vit\[D]toria?
Pa\[F#m]ura \[Bm]non mi puoi far \[A4]più. \[A]
\[G]Se sulla \[A]croce io mori\[D]rò in\[A]sieme a \[Bm]Lui
\[G]poi insieme a \[D]lui risorge\[A4]rò. \[A]
\endverse
\beginverse
^Tu Signore a^mante della ^vita,
^mi hai creato ^per l'eterni^tà. ^
^La vita ^mia Tu dal se^polcro ^strappe^rai,
^con questo mio ^corpo Ti ve^drò. ^
\endverse
\beginverse
^Tu mi hai do^nato la Tua ^vita,
io ^voglio do^nar la mia a ^Te. ^
^Fa' che possa ^dire ``Cristo ^vive ^anche in ^me''
^e quel giorno ^io risorge\[A4]\[A]\[B4]\[B]rò.
\endverse
\beginchorus
\transpose{2}
\[D]Cristo è ri\[A]sorto vera\[D]mente alle\[A]luia
\[D]Gesù il vi\[A]vente qui con \[Bm]noi reste\[A4]rà \[A]
\[Bm]Cristo Ge\[G]sù, \[D]Cristo Ge\[G]sù
\[D]è il Si\[A]gnore della \[G]vi\[D]ta. \[Em]\[Em]\[D]
\endchorus
\endsong

%titolo{Cristo pellegrino}
%autore{Lacchin}
%album{}
%tonalita{Re}
%gruppo{}
%momenti{}
%identificatore{cristo_pellegrino}
%data_revisione{2011_12_31}
%trascrittore{Francesco Endrici}
\beginsong{Cristo pellegrino}[by={Lacchin}]
\beginverse
\[D]Voglio raccontarti \[A]come in un giorno qua\[G]lunque 
i miei passi si \[D]sono, si sono incro\[A]ciati 
con quelli di \[G]Dio, con quelli di \[D]Dio. \[A]
\endverse
\beginverse
\chordsoff
^Ero sempre stato ^io che ero andato a tro^varlo 
in quella chie^setta fredda cima al pa^ese, 
a portargli i miei ^dubbi e le mie pre^ghiere. ^ 
\endverse
\beginverse
Co\[C]sì non mi sarei \[G]mai aspettato \[D]che
sa\[C]rebbe stato \[G]Dio un giorno a \[D]scendere da me, 
la Sua \[C]croce ha portato \[G]fino in casa \[D]mia, 
a una ra\[C]gazza ho aperto la \[G]porta
e Dio è ri\[D]masto qui con me.
\endverse
\beginchorus
\[D]Cristo pellegrino del \[G]mondo mio fra\[D]tello, 
quanti sogni e spe\[G]ranze nel Tuo \[Bm]Nome
ogni giorno si a\ch{G}{f}{f}{ff}ollano e ti a\ch{A}{f}{f}{ff}ido con il \[D]cuore, 
sapendo che \[G]Tu a braccia a\[A]perte sulla \[D]croce
soffrendo sor\[G]ridi per offrirmi la Tua \[Bm]pace,
che sola sa \[G]darmi forza per \[A]continuare. \[D]
\endchorus
\beginverse
\chordsoff
^Tra i miei libri e le mie ^carte, \brk scritti e note a ma^tita 
ti sei fatto pre^sente come un bravo pa^pà metti ordine ai ^conti e ritorna la ^sereni^tà.
\endverse 
\beginverse
\chordsoff
^ Ho pregato in queste ^ore \brk che Ti ho avuto tra le ^mani, 
mi hai marchiato col tuo ^fuoco \brk che ora non se ne an^drà,
memoriale pe^renne del Tuo pas^saggio di ^qua. 
\endverse 
\beginverse
\chordsoff
E a\[C]desso ti porto \[G]io da un'altra \[D]anima, \brk che \[C]cerca senza sa\[G]perlo quel che a\[D]vrà. 
Apri\[C]rà la sua porta e \[G]rimarrà sor\[D]presa \brk e un sor\[C]riso mi spunte\[G]rà in pieno \[D]viso.
\endverse
\beginchorus
\chordsoff
\[D]Cristo pellegrino del \[G]mondo mio fra\[D]tello, 
quanti sogni e spe\[G]ranze nel Tuo \[Bm]Nome
ogni giorno si af\[G]follano e ti af\[A]fido con il \[D]cuore, 
sapendo che \[G]Tu a braccia a\[A]perte sulla \[D]croce
soffrendo sor\[G]ridi per offrirmi la Tua \[Bm]pace,
che sola sa \[G]darmi forza per \[A]continuare. \[D]
\endchorus
\endsong

%titolo{Cristo Re}
%autore{Montuori}
%album{Mi affido a Te}
%tonalita{La}
%gruppo{}
%momenti{Adorazione}
%identificatore{cristo_re}
%data_revisione{2011_12_31}
%trascrittore{Francesco Endrici - Manuel Toniato}
\beginsong{Cristo Re}[by={Montuori}]

\ifchorded
\beginverse*
\vspace*{-0.8\versesep}
{\nolyrics \[A]  \[D]  \[Bm]  \[D]  \[E4]  \[E]}
\vspace*{-\versesep}
\endverse
\fi

\beginverse
Bene\[A2]detto sei Si\[A]gnore, Re di \[D]pace e di vittoria,
sei ve\[Bm]nuto in mezzo a noi, mite ed \[E4]umile di \[E7]cuore.
Buon ma\[A2]estro ci hai inse\[A/C#]gnato ad a\[D]mare,
\[F#7/C#]questo \[Bm]popolo ti eleva la sua \[E4]lo\[E7]de. \[D/E]  
\endverse

\beginchorus
\[E]A te can\[A]tiamo Cristo Re dell'uni\[E]ver\[D]so,
\[C#m7]Tu di\[Bm7]vino Reden\[D]tore vieni a \[E4]dimo\[E]rare in noi.
\[C#7]Ti procla\[F#m7]miamo Cristo \[F#m/E]Re dell'uni\[D7+]verso,
\[A7+]la tua \[Bm7]luce ri\[Bm7/E]splende su di \[A]noi.
\endchorus

\ifchorded
\beginverse*
\vspace*{-\versesep}
{\nolyrics \[D] \[C#m7] \[F#7] \[Bm7] \[D] \[E4] \[E] }
\endverse
\fi

\beginverse
\chordsoff
La speranza è solo in te, perché sei un Dio fedele,
con la tua misericordia, hai guarito i nostri cuori.
Tu ci guidi e ci consoli con amore,
questo popolo ti eleva la sua lode. 
\endverse

\beginchorus
\chordsoff
A te cantiamo Cristo Re dell'universo,
Tu divino Redentore vieni a dimorare in noi.
Ti proclamiamo Cristo Re dell'universo,
la tua luce risplende su di noi.
\endchorus

\ifchorded
\beginverse*
\vspace*{-\versesep}
{\nolyrics \[F] \[E&/F] \[F/E&] \[B&] \[E&] \[Dm7] \[Cm7] \[E&] \[F4] \[F] \[D7] 
\[G7] \[Gm/F] \[E&7+] \[Dm7] \[Gm7] \[Cm7] \[Cm7/F]}
\endverse
\fi

\beginchorus
A te can\[B&]tiamo Cristo Re dell'uni\[F/E&]ver\[E&]so,
\[Dm7]Tu di\[Cm7]vino Reden\[E&]tore vieni a \[F4]dimo\[F]rare in \[D7/F#]noi.
Ti procla\[Gm7]miamo Cristo \[Gm/F]Re dell'uni\[E&7+]verso,
\[Dm7]la tua \[Cm7]luce ri\[Cm7/F]splende su di \[B&]noi, \[E&7+] 
la tua \[Cm7]luce risplende su di \[B&]noi.
\endchorus
\endsong

%titolo{Custodiscili nel Tuo Amore}
%autore{Cento}
%album{Tu hai inventato l'amore}
%tonalita{Mi}
%gruppo{}
%momenti{Matrimoni}
%identificatore{custodiscili_nel_tuo_amore}
%data_revisione{2011_12_31}
%trascrittore{Francesco Endrici}
\beginsong{Custodiscili nel Tuo Amore}[by={Cento}]
\beginchorus
\[E]Padre santo, custodisci nel tuo a\[A]more
\[Am]questi nostri a\[E]mici, \[B]questi nostri a\[E]mici. \[G#]
\endchorus
\beginverse
\[C#m]Tu li hai creati per a\[F#m]more, \brk \[B]tu li hai redenti per a\[E]more,
\[C#m]oggi li unisci nel tuo a\[F#m]more, Si\[G#7]gnore.
\endverse
\ifchorded
\beginverse*
\vspace*{-\versesep}
{\nolyrics \[C#m]\[F#m]\[G#7]}
\endverse
\fi
\beginverse
\chordsoff
^Quando la notte scende^rà
^vicino ad essi vieni ^tu,
^sul loro amore veglie^rai, Si^gnore.
\endverse
\ifchorded
\beginverse*
\vspace*{-\versesep}
{\nolyrics \[C#m]\[F#m]\[G#7]}
\endverse
\fi
\beginverse
\chordsoff
^Dona un sorriso ogni mat^tino,
^la voglia grande di lot^tare
^perché non manchi mai l'a^more, Si^gnore.
\endverse
\beginchorus
\chordsoff 
Rit. 
\endchorus
\endsong



%titolo{Custodiscimi}
%autore{Chemin Neuf}
%album{Davati al re}
%tonalita{Sol-}
%gruppo{}
%momenti{Quaresima;Salmi}
%identificatore{custodiscimi}
%data_revisione{2011_12_31}
%trascrittore{Francesco Endrici - Manuel Toniato}
\beginsong{Custodiscimi}[by={Chemin\ Neuf}]

\beginverse
\[Em]Ho detto a Dio, \[Am]senza \[B7]di \[Em]te,
alcun \[D]bene non \[G]ho, custo\[Am]disci\[B7]mi.
\[Em]Magnifica è la \[Am]mia e\[B7]redi\[Em]tà,
bene\[D]detto sei \[G]Tu, sempre \[Am]sei \[Bm7]con \[Em]me.
\endverse

\beginchorus
Custo\[Em]di\[Am]sci\[D]mi, mia \[G]forza sei \[Am]Tu,
Custo\[Em]di\[Am]sci\[D]mi, mia \[C]gioia Ge\[Em]sù. \rep{2}
\endchorus

\beginverse
\chordsoff
Ti pongo sempre innanzi a me
al sicuro sarò, mai vacillerò!
Via, verità e vita sei;
Mio Dio credo che Tu mi guiderai.
\endverse
\endsong


\lettera
%titolo{Dall'aurora al tramonto}
%autore{Casucci, Balduzzi}
%album{Verbum panis}
%tonalita{Do#-}
%gruppo{}
%momenti{Ingresso;Salmi}
%identificatore{dall_aurora_al_tramonto}
%data_revisione{2011_12_31}
%trascrittore{Francesco Endrici}
\beginsong{Dall'aurora al tramonto}[by={Casucci, Balduzzi}]
\ifchorded
\beginverse*
\vspace*{-0.8\versesep}
{\memorize \[C#m]\[E]\[A]\[B] \[C#m]\[E]\[F#m]\[B] \[C#m]\[E]\[F#m]\[B] \[C#m]\[E]\[F#m]\[B]}
\vspace*{-\versesep}
\endverse
\fi
\beginchorus
\memorize
\[C#m]Dall'au\[E]rora io \[F#m]cerco \[B]te,
\[C#m]fino al tra\[E]monto ti \[F#m]chia\[B]mo.
\[C#m]Ha sete \[G#m]solo di \[A]te l'\[B]anima \[C#m]mia
come \[G#m]terra de\[A]ser\[B]ta. \rep{2}
\endchorus
\beginverse
^Non mi ferme^rò un ^solo i^stante
^sempre cante^rò ^la tua ^lode.
^Perché sei il mio ^Dio, ^il mio ri^paro
^mi protegge^rai ^all'ombra delle tue \[B4]ali.
\endverse
\beginchorus
\[C#m]Dall'au\[E]rora io \[F#m]cerco \[B]te \dots
\endchorus
\beginverse
^Non mi ferme^rò un ^solo i^stante,
^io racconte^rò ^le tue ^opere
^perché sei il mio ^Dio, ^unico ^bene.
^Nulla mai po^trà ^la notte contro di \[B4]me.
\endverse
\beginchorus
^Dall'au^rora io ^cerco ^te,
^fino al tra^monto ti ^chia^mo.
^Ha sete ^solo di ^te l'^anima ^mia
come ^terra de^ser^ta.
\[C#m]Ha sete \[G#m]solo di \[A]te l'\[B]anima \[E]mia
come \[A]terra de\[B]ser\[E]ta.
\endchorus
\endsong

%titolo{Dalla tristezza alla danza}
%autore{Walker}
%album{Risplendi Gerusalemme}
%tonalita{Mi}
%gruppo{}
%momenti{Congedo;Salmi}
%identificatore{dalla_tristezza_alla_danza}
%data_revisione{2011_12_31}
%trascrittore{Francesco Endrici - Manuel Toniato}
\beginsong{Dalla tristezza alla danza}[by={Walker}]

\beginchorus
\[E] Egli \[A]ha  cam\[C#m]biato  \[B]il mio \[F#m7]pianto
E la \[E]mia tris\[A]tezza in \[B4]dan\[B]za.
\[E] Pro\[A]clame\[C#m]rò e \[B]cante\[F#m]rò 
la sua \[B]gioia in \[E]me \[A]\[(II volta: Mi)]  
\endchorus

\beginverse*
Dove \[E]c'era il \[A]dolore,\[E] \[A] 
ope\[E]rò la \[A]guarigione. \[E] \[A] 
Dove \[E]c'era \[A]soffe\[E]renza, \[A] 
ci do\[E]nò con\[A]sola\[E]zione. 
Sento il suo \[D]dolce amor per \[A]me che mi il\[E]lumina.
E la sua \[D]luce splende\[A]rà 
nell'o\[C]scurità del \[A]nostro \[B7]cuore\ldots
\endverse
\endsong


%titolo{Danza la vita}
%autore{}
%album{}
%tonalita{Re}
%gruppo{}
%momenti{Congedo}
%identificatore{danza_la_vita}
%data_revisione{2011_12_31}
%trascrittore{Francesco Endrici}
\beginsong{Danza la vita}
\beginverse
\[D]Canta con la \[G]voce e con il \[D]cuore, \[G]
\[D]con la bocca e \[G]con la vita, \[D] \[G]
\[D]canta senza \[G]stonature, \[D] \[G]
la \[D]verità \[G] del \[D]cuore. \[G]
\[D]Canta come \[G]cantano i viandanti
\echo{\[D]canta come \[G]cantano i viandanti}
non \[D]solo per riem\[G]pire il tempo,
\echo{non \[D]solo per \[G]riempire il tempo,}
\[D]Ma per soste\[G]nere lo sforzo 
\echo{\[D]Ma per soste\[G]nere lo sforzo.}
\[D]Canta \[G] e cam\[D]mina \[G]
\[D]Canta \[G] e cam\[D]mina \[G]
Se \[A]poi, credi non possa ba\[Bm]stare
segui il \[E]tempo, stai \[G]pron\[A]to e
\endverse
\beginchorus
\[D]Danza la \[G]vita, al \[A]ritmo dello \[D]Spirito.
\echo{Spirito che riempi i nostri }
\[Bm]Danza, \[G]danza al \[A]ritmo che c'è in \[D]te.
\echo{cuor, danza assieme a noi. Danza}
\[G]Spirito \[A]che \[D]riempi i nostri
\echo{ la vita al ritmo dello Spirito}
\[Bm]cuor. \[G]Danza assieme a \[A]no\[D]i.
\echo{Danza, danza al ritmo che c'è in te.}
\endchorus
\beginverse
\chordsoff
Cam^mina sulle ^orme del Si^gnore, ^
non ^solo con i ^piedi ^ma ^ \brk ^usa soprat^tutto il cuore.^^
^Ama ^ chi è con ^te. ^
Cam^mina con lo ^zaino sulle spalle
\echo{Cam^mina con lo ^zaino sulle spalle}
^la fatica a^iuta a crescere
\echo{^la fatica a^iuta a crescere}
^nella con^divisione
\echo{^nella con^divisione.}
^Canta ^ e cam^mina, ^
^canta  ^ e cam^mina. ^
Se ^poi, credi non possa ba^stare
segui il ^tempo, stai ^pron^to e
\endverse
\beginchorus
\chordsoff 
Rit. 
\endchorus
\endsong


%titolo{Davanti a questo amore}
%autore{De Luca, Marranzino, Conte}
%album{Risorto per amore}
%tonalita{Si-}
%gruppo{}
%momenti{Pasqua;Comunione}
%identificatore{davanti_a_questo_amore}
%data_revisione{2011_12_31}
%trascrittore{Francesco Endrici - Manuel Toniato}
\beginsong{Davanti a questo amore}[by={De\ Luca, Marranzino, Conte}]

\ifchorded
\beginverse*
\vspace*{-0.8\versesep}
{\nolyrics \[Bm] \[F#m] \[G7+] \[F#m]  \[Bm] \[F#m] \[G7+] \[F#4] \[F#] }
\vspace*{-\versesep}
\endverse
\fi

\beginverse*
Hai di\[Bm]steso le tue \[A]braccia anche per \[Bm]me Gesù;
\[A]dal tuo \[G7+]cuore, come fon\[A]te, hai ver\[D]sato \[A]pace in \[Bm]me, 
\[F#m]cerco an\[Em7]cora il mio pec\[F#7]cato, ma non \[G7+]c'è.
Tu da \[Em7]sempre vinci il \[A]mondo \brk dal tuo \[G]trono di do\[A]lore.
\endverse

\beginchorus
\[D]Dio, mia \[G]grazia, mia spe\[A]ranza, \brk \[Bm]ricco e grande Reden\[A]to\[F#]re.        
\[G]Tu, re \[A]umile e po\[D]tente, ri\[G]sorto per a\[D]more,
ri\[G]sorgi per la \[A]vita.

\[D]Vero a\[G]gnello senza \[A]macchia, \brk \[Bm7]mite e forte Salva\[A]{tore} \[F#7]sei.
\[G]Tu, Re \[A]povero e glo\[D]rioso ri\[G]sorgi con po\[D]tenza,
da\[G]vanti a questo a\[A]more la \[F#7]morte fuggi\[Bm]rà.
\endchorus

\ifchorded
\beginverse*
\vspace*{-\versesep}
{\nolyrics \[G7+] \[A] \[Bm] \[F#] \[G#7] }
\vspace*{-\versesep}
\endverse
\fi

\beginverse*
Hai di\[C#m]steso le tue \[B]braccia anche per \[C#m]me Gesù;
\[B]dal tuo \[A7+]cuore, come fon\[B]te, \brk hai ver\[E]sato \[B]pace in \[C#m]me,
cerco an\[F#m7]cora il mio pec\[G#7]cato, ma non \[A7+]c'è.
Tu da \[F#m7]sempre vinci il \[B]mondo \brk dal tuo \[A]trono di do\[B]lore.
\endverse

\beginchorus
\[E]Dio, mia \[A]grazia, mia spe\[B]ranza, \brk \[C#m]ricco e grande Reden\[B]to\[G#]re. 
\[A]Tu, re \[B]umile e po\[E]tente, ri\[A]sorto per a\[E]more,
ri\[A]sorgi per la \[B]vita.

\[E]Vero a\[A]gnello senza \[B]macchia, \brk \[C#m7]mite e forte Salva\[B]tore \[G#7]sei.
\[A]Tu, Re \[B]povero e glori\[E]oso ri\[A]sorgi con po\[E]tenza,
da\[A]vanti a questo a\[B]more la \[G#7]morte fuggi\[C#m]rà.
\endchorus

\ifchorded
\beginverse*
\vspace*{-\versesep}
{\nolyrics \[G#m] \[A7+] \[G#] \[C#m] \[G#] \[A7] \[G#4] \[C#m] }
\endverse
\fi
\endsong


%titolo{Davanti al Crocifisso}
%autore{}
%album{}
%tonalita{La-}
%gruppo{}
%momenti{Adorazione}
%identificatore{Davanti_al_crocifisso}
%data_revisione{2011_12_31}
%trascrittore{Francesco Endrici}
\beginsong{Davanti al Crocifisso}[ititle={Alto e glorioso Dio}]
\beginverse*
\[Am]O \[G]alto e glo\[C]rioso \[G]Dio,
il\[C]lumina el \[F]core \[G]mio.
\[C] Dame fede di\[G]ritta, \[C]speranza \[F]certa, cari\[G]tà per\[C]fetta,
\[Am] umiltà pro\[Dm]fonda, \[F] senno e cognose\[C]mento
che io \[G]servi li \[F]toi comanda\[C]menti.
\endverse
\endsong

%titolo{Del tuo Spirito, Signore}
%autore{Gen Verde, Gen Rosso}
%album{Come fuoco vivo}
%tonalita{Re}
%gruppo{}
%momenti{Salmi}
%identificatore{del_tuo_spirito_signore}
%data_revisione{2011_12_31}
%trascrittore{Francesco Endrici}
\beginsong{Del tuo Spirito, Signore}[by={Gen\ Verde, Gen\ Rosso}]
\meter{3}{4}
\ifchorded
\beginverse*
\vspace*{-0.8\versesep}
\[Am]\[Bm]\[G]\[D]\[Em] 
\vspace*{-\versesep}
\endverse
\fi
\beginchorus
|\[D] Del tuo |\[G]Spiri\[D]to, Si|\[G]gno\[D]re,
\[A]è |\[Bm]piena la \[F#m]ter|\[G]ra, è |\[D]piena la \[Em]ter|\[D]ra. 
\endchorus
\beginverse
|\[C]Benedici il Si|\[B&]gnore, \mbar{2}{4}\[Dm]anima \mbar{4}{4}\[Am]mi\[B&]a,
Si\[C]gnore, \mbar{3}{4}\[F]Dio,\[C] tu sei |\[G]gran\[C]de!
|\[C]Sono immense, splen|\[B&]denti
\mbar{2}{4}\[Dm]tutte le tue \mbar{3}{4}\[B&]ope\[F]re e |\[Gm]tutte le crea\[A]tu\mbar{4}{4}\[D]re.
\endverse
\beginverse
\chordsoff
Se tu togli il tuo soffio muore ogni cosa
e si dissolve nella terra.
Il tuo spirito scende:
tutto si ricrea e tutto si rinnova.
\endverse
\beginverse
\chordsoff
La tua gloria, Signore, resti per sempre.
Gioisci, Dio, del creato.
Questo semplice canto
salga a te Signore, sei tu la nostra gioia.
\endverse
\endsong


%titolo{Dio aprirà una via}
%autore{Moen}
%album{Cantiamo con gioia}
%tonalita{Sol}
%gruppo{}
%momenti{Avvento;Congedo}
%identificatore{dio_aprira_una_via}
%data_revisione{2011_12_31}
%trascrittore{Francesco Endrici - Manuel Toniato}
\beginsong{Dio aprirà una via}[by={Moen}]

\beginchorus
\[G]Dio aprirà una \[D]via dove \[C]sembra non ci \[G]sia
\[C]come ope\[G]ra non so, \[Am]ma una nuova \[G]via ve\[D]drò.
\[G]Dio mi guide\[D]rà, mi ter\[C]rà vicino a \[G]sé.
\[C]Per ogni giorno a\[G]more e \[Em7]forza
\[Am]Lui mi donerà, \[C]una via apri\[G]rà.
\endchorus

\beginverse*
\[E&]Aprirà una \[F]strada nel de\[E&]{ser}\[B&]to, 
\[E&]fiumi di acqua \[F]viva io ve\[G]drò.
\[C]Se tutto passe\[D]rà, la Sua pa\[Bm]rola \[B7]reste\[Em]rà,
\[C]una cosa \[D]nuova Lui fa\[E4]{rà.} \[E] 
\endverse
\endsong

%titolo{Dio dell'unità}
%autore{Mariano}
%album{Venite a me}
%tonalita{Do}
%gruppo{}
%momenti{Ingresso}
%identificatore{dio_dell_unita}
%data_revisione{2011_12_31}
%trascrittore{Francesco Endrici}
\beginsong{Dio dell'unità}[by={Mariano}]
\beginverse
\[C]Tu, Dio del \[Dm7]cielo e della \[C]terra,
che co\[Dm7]nosci i nostri \[C]giorni,
apri an\[Em7]cora le Tue \[F]braccia di bon\[G]tà.
\[C]Tu, che ci at\[Dm7]tendi dal prin\[C]cipio
e da \[Dm7]sempre ci accom\[C]pagni,
ecco at\[Dm7]torno alla Tua \[F]mensa i figli \[G4]Tuoi. \[G]
\endverse
\beginchorus
\[C]Noi \[Em7] cantiamo \[F]Te, \[G4]unico e im\[G]menso
\[C]Dio \[Em7] dell'uni\[F]tà, \[G4]Padre e Si\[G]gnore,
\[C]Tu \[Em7] sarai per \[F]noi \[G4]pane di \[G]liber\[C]tà. \[Em]\[F]\[G]
\endchorus
\beginverse
^Tu, che ci ^doni la Pa^rola
e sai ^diventare ^pane
per nu^trire di spe^ranza i nostri ^cuori.
^Tu, che hai of^ferto la Tua v^ita
e can^celli i nostri er^rori,
sai ri^dare la fi^ducia ad ogni ^uomo. ^
\endverse
\beginchorus
\[C]Noi \[Em7] cantiamo \[F]Te, \[G4]unico e im\[G]menso
\[C]Dio \[Em7] dell'uni\[F]tà, \[G4]Padre e Si\[G]gnore,
\[C]Tu \[Em7] sarai per \[F]noi \[G4]pane di \[G]liber\[C]tà. \[Em]\[F]\[G]\[D]
\endchorus
\beginchorus
\transpose{2}
\[C]Noi \[Em7] cantiamo \[F]Te, \[G4]unico e im\[G]menso
\[C]Dio \[Em7] dell'uni\[F]tà, \[G4]Padre e Si\[G]gnore,
\[C]Tu \[Em7] sarai per \[F]noi \[G4]pane di \[G]liber\[C]tà. \[Em]\[F]\[G]\[C]
\endchorus
\endsong

%titolo{Dio si è fatto come noi}
%autore{Stefani, Giombini}
%album{Messa Alleluia}
%tonalita{Re}
%gruppo{}
%momenti{Natale}
%identificatore{dio_si_e_fatto_come_noi}
%data_revisione{2011_12_31}
%trascrittore{Francesco Endrici - Manuel Toniato}
\beginsong{Dio si è fatto come noi}[by={Stefani, Giombini}]

\beginverse
\[D]Di\[G]o si è \[D]fatto \[A]come \[D]noi \[G] \[D]  \[A] \brk \[D]per \[F#m]farci \[G]co\[A]me \[D]lui \[G] \[D] 
\endverse

\beginchorus
\[G]Vie\[B]ni Ge\[Em]sù \[A]  \[F#m]resta con \[Bm]noi! \[D]   
\[G]\[Em]re\[A]sta \[A7]con \[D]noi.\[G] \[D] \[A] 
\endchorus

\beginverse
\chordsoff
Viene dal grembo di una donna la Vergine Maria.
\endverse

\beginverse
\chordsoff
Tutta la storia lo aspettava il nostro Salvatore.
\endverse

\beginverse
\chordsoff
Egli era un uomo come noi e ci ha chiamato amici.
\endverse

\beginverse
\chordsoff
Egli ci ha dato la sua vita insieme a questo pane.
\endverse

\beginverse
\chordsoff
Noi che mangiamo questo pane saremo tutti amici.
\endverse

\beginverse
\chordsoff
Noi, che crediamo nel suo amore, vedremo la sua gloria.
\endverse

\beginverse
\chordsoff
Vieni, Signore, in mezzo a noi: resta con noi per sempre.
\endverse
\endsong


%titolo{Dolce sentire}
%autore{Ortolani}
%album{Fratello sole, sorella luna}
%tonalita{}
%gruppo{}
%momenti{}
%identificatore{dolce_sentire}
%data_revisione{2011_12_31}
%trascrittore{Francesco Endrici}
\beginsong{Dolce sentire}[by={Ortolani}]
\beginverse
\[D]Dolce \[A7]sen\[F#m]tire \[G]come \[A7]nel mio \[D]cuore
\[G]o\[A]ra, u\[G]mil\[F#m]men\[Bm]te \[Em]sta nascendo a\[A7]mo\[A]re.
\[D]Dolce \[A7]ca\[F#m]pire \[G]che non \[A7]son più \[D]solo,
\[G]ma \[A]che \[G]son \[F#m]par\[Bm]te \[Em]di una im\[A7]mensa \[D]vita, \[Bm]
che \[Gm]gene\[D]rosa ri\[A7]splende intorno a \[D]me: \[Bm]
do\[Gm]no di \[D]Lui, del \[A7]suo immenso a\[D]more.
\endverse
\beginverse
^Ci ha da^to il ^cielo ^e le ^chiare ^stelle,
^fra^tel^lo ^so^le ^e sorella ^lu^na;
^la ma^dre ^terra, con ^frutti ^prati e ^fiori,
^il ^fuo^co, il ^ven^to, ^l’aria e ^l’acqua \[D]pura,
\[Bm]fon\[A]te \[G]di \[F#m]vi\[Bm]ta \[Em]per le \[A7]sue crea\[D]ture: \[Bm]
do\[Gm]no di \[D]Lui, del \[A7]suo immenso a\[D]more, \[Bm]
do\[Gm]no di \[D]Lui, del \[A7]suo immenso a\[D]more.
\endverse
\endsong

%titolo{Dov'è carità e amore}
%autore{Meloni, Zanettin, Zardini}
%album{}
%tonalita{Re}
%gruppo{}
%momenti{Comunione}
%identificatore{dov_e_carita_e_amore}
%data_revisione{2011_12_31}
%trascrittore{Francesco Endrici - Manuel Toniato}
\beginsong{Dov'è carità e amore}[by={Meloni, Zanettin, Zardini}]

\beginchorus
\[A7]Do\[D]v'è cari\[G]tà e a\[A]more, \[G] qui c'è \[A7]Di\[D]o.
\endchorus

\beginverse
\[Bm]Ci ha riu\[F#m]niti tutti in\[G]sieme Cristo a\[D]more:
\[G]go\[F#m]diamo esul\[Em]tanti nel Si\[A]gnore!
\[D]Temia\[Em]mo e ami\[G]amo il Dio vi\[A7]ven\[Bm]te,
\[Em]e a\[F#m]miamo\[G]ci tra \[F#m7]noi con \[Bm]cuore sin\[A7]cero.
\endverse

\beginverse
\chordsoff
Noi formiamo, qui riuniti, un solo corpo:
evitiamo di dividerci tra noi:
via le lotte maligne, via le liti!
e regni in mezzo a noi Cristo Dio.
\endverse

\beginverse
\chordsoff
Chi non ama resta sempre nella notte
e dall'ombra della morte non risorge;
ma se noi camminiamo nell'amore,
noi saremo veri figli della luce.
\endverse

\beginverse
\chordsoff
Nell'amore di colui che ci ha salvati,
rinnovati dallo Spirito del Padre,
tutti uniti sentiamoci fratelli,
e la gioia diffondiamo sulla terra.
\endverse
\endsong


%titolo{Down to the river to pray}
%autore{}
%album{}
%tonalita{Mi}
%gruppo{}
%momenti{Spiritual}
%identificatore{down_to_the_river_to_pray}
%data_revisione{2011_12_31}
%trascrittore{Francesco Endrici}
\beginsong{Down to the river to pray}
\beginverse
As I went \[E]down in the river to pray
\[B]studying a\[E]bout that \[A]good \[B]ol' \[E]way.
And who shall wear the starry crown?
Good \[B]Lord \[A]show me the \[E]way!
\endverse
\beginchorus
\[B]O sisters \[E]let's go down
\[A]let's go down, come on \[E]down.
\[B]O sisters \[E]let's go down,
down in the \[A]river \[B]to \[E]pray.
\endchorus
\chordsoff
\beginverse
As I went down in the river to pray
studying about that good ol' way.
And who shall wear the robe \& crown?
Good Lord show me the way.
\endverse
\beginchorus
O brothers let's go down
let's go down, come on down.
Come on brothers, let's go down
Down in the river to pray.
\endchorus
\beginverse
As I went down in the river to pray
studying about that good ol' way.
And who shall wear the starry crown?
Good Lord show me the way!
\endverse
\beginchorus
O fathers let's go down
let's go down, come on down
O fathers let's go down
Down in the river to pray.
\endchorus
\beginverse
As I went down in the river to pray
studying about that good ol' way.
And who shall wear the robe \& crown?
Good Lord show me the way.
\endverse
\beginchorus
O mothers let's go down
Come on down, don't you wanna go down?
Come on mothers, let's go down
Down in the river to pray.
\endchorus
\beginverse
As I went down in the river to pray
studying about that good ol' way.
And who shall wear the starry crown?
Good Lord show me the way!
\endverse
\beginchorus
O sinners, let's go down
let's go down, come on down
O sinners, let's go down
Down in the river to pray.
\endchorus
\beginverse
As I went down in the river to pray
studying about that good ol' way.
And who shall wear the robe \& crown?
Good Lord show me the way.
\endverse
\endsong

\lettera
%titolo{È bello}
%autore{Gen Rosso}
%album{Noi veniamo a Te}
%tonalita{Re}
%gruppo{}
%momenti{Congedo}
%identificatore{e_bello}
%data_revisione{2011_12_31}
%trascrittore{Francesco Endrici - Manuel Toniato}
\beginsong{È bello}[by={Gen\ Rosso}]

\beginverse
È \[D]bello andar con i miei fra\[A]telli
per le vie del \[D]mondo e poi scoprire \[G]Te
nascosto in ogni \[D]cuor.
E veder che ogni mat\[A]tino Tu
ci fai ri\[D]nascere e fino a \[G]sera
sei vicino \[D]nella gioia e nel do\[A7]lor.
\endverse

\beginchorus
\[D]Grazie per\[G]ché sei con \[D]me,
\[D]grazie per\[G]ché se ci a\[D]miamo
ri\[A7]mani tra \[D]noi.
\endchorus

\beginverse
\chordsoff
È bello udire la tua voce
che ci parla delle grandi cose
fatte dalla tua bontà.
Vedere l'uomo fatto a immagine
della tua vita, fatto per conoscere
in Te il mistero della Trinità.
\endverse

\beginverse
\chordsoff
È bello dare questa lode a Te
portando a tutto il mondo
il nome tuo Signor che sei l'amor.
Uscire e per le vie cantare
che abbiamo un Padre solo e tutti quanti
siamo figli veri nati dal Signor.
\endverse
\endsong



%titolo{È bello lodarti}
%autore{Gen Verde}
%album{È bello lodarti}
%tonalita{Sol}
%gruppo{}
%momenti{Ingresso}
%identificatore{e_bello_lodarti}
%data_revisione{2011_12_31}
%trascrittore{Francesco Endrici}
\beginsong{È bello lodarti}[by={Gen\ Verde}]
\beginchorus
\[G] È \[D]bello can\[C]tare il tuo a\[G]more,
\[Am] è \[7]bello lo\[D]dare il tuo nome.
\[G] È \[B4]bello can\[C]tare il tuo amore,
è \[G]bello lo\[D]darti, Si\[C]gnore,
è \[G]bello can\[D]tare a \[C]te!
\endchorus

\beginverse
\[Em]Tu che sei l'amore infi\[Bm6]nito
che nep\[C]pure il cielo può contenere,
ti \[Am]sei fatto \[7]uomo, \[D6]Tu sei venuto qui
ad \[B7]abitare in mezzo a \[C]noi, allora\dots \[G] 
\endverse

\beginverse
^Tu che conti tutte le ^stelle
e le ^chiami una ad una per nome,
da ^mille sen^tieri ^ci hai radunati qui,
^ci hai chiamati figli ^tuoi, allora\dots ^
\endverse
\endsong

%titolo{E chissà}
%autore{Spoladore}
%album{Inno di ``Ponti e Arcobaleni'' - incontro nazionale del settore giovani di AC, Roma 1997}
%tonalita{Sol}
%gruppo{}
%momenti{}
%identificatore{e_chissa}
%data_revisione{2011_12_31}
%trascrittore{Francesco Endrici - Manuel Toniato}
\beginsong{E chissà}[by={Spoladore}]

\ifchorded
\beginverse*
\vspace*{-0.8\versesep}
{\nolyrics \[B&] \[F] \[C7] \[G] \[F] \[B&] \[C] \[D] }
\vspace*{-\versesep}
\endverse
\fi

\beginverse
E chis\[G]sà, \[C]chi lo \[G]sa,
se questo mondo poi re\[C]siste\[G]rà \[D] 
a questa im\[Em]mensa corsa fatta a testa in giù
\[D]senza \[G]fiato e \[C]liber\[G]tà
e \[C]gira \[G]gira \[C]ma non \[G]va
c'è care\[G7]stia di gioia e fanta\[C]sia \[F] \[C] 
ma se \[Cm6]pace guiderà i pensieri \[G]tuoi \[D] \[Em] 
la nuova \[Am7]Vita ti sor\[D]prende\[G]rà. \[D] 
\endverse

\beginchorus
Come è vero che \[G]vivi \[C] e \[G]senti \[D] 
questo \[Em]mondo ha bi\[D]sogno di \[C]te,
questo \[Em]mondo ha bi\[D]sogno di \[C]noi
\[G]Vivo, \[C]  \[G]sento \[D]
una \[Em]luce più \[D]forte che \[C]mai
che può \[Em]vincere il \[D]buio che \[C]c'è
corre in \[G]te se lo vu\[Am7]oi
corre \[G]qui tra di \[D]noi
questo \[Em]ponte con l'\[D]umani\[C]tà
arcoba\[Em]leno \[D]di pace \[C]sarà \[G] \[B&] \[F] \[C7] \[G] 
\endchorus

\beginverse
\chordsoff
E chissà, chi lo sa
cosa ciascuno di noi sceglierà
dentro ogni cosa che ti segna nella vita
o impari amore o rabbia avrai.
Non temere, non temere mai
che un caldo abbraccio dentro arriverà
e tra gli inganni della mente il cuore sa
che Dio ci ama e mai ci lascerà.
\endverse

\beginverse
\chordsoff
E chissà se lo sai
quanta dolcezza e gioia puoi provare
se fai pace fino al centro del tuo cuore
non c'è tempesta che ti porta via.
E gira e gira dai che va
se cambi il cuore cambia la città
in ogni volto c'è l'umanità
in ogni volto Dio ti parla già.
\endverse
\endsong


%titolo{È festa grande}
%autore{Cento}
%album{Ricevi questo anello}
%tonalita{Re}
%gruppo{}
%momenti{Matrimoni}
%identificatore{e_festa_grande}
%data_revisione{2011_12_31}
%trascrittore{Francesco Endrici}
\beginsong{È festa grande}[by={Cento}]
\beginchorus
È festa \[D]grande oggi \[A]a Gerusa\[D]lemme,
lo sposo porta in casa \[A]la sua \[D]sposa; \[7]
è festa in \[G]cielo, festa \[A]sulla \[F#m]terra, \[Bm]
amica, \[G]vieni a stare in\[A]sieme a \[D]me.
\endchorus
\beginverse
Ri\[D]cordi, amica, \[A]la tua giovi\[Bm]nezza?
So\[G]gnavi mondi \[A]forse un po' ir\[D]reali \[7]
e un grande a\[G]more per dar \[A]senso
a \[F#m]questa vita \[Bm]tua, io ti par\[G]lai al \[A]cuore.
\endverse
\beginverse
\chordsoff
Dimentica la casa di tuo padre,
le strade di una terra e la tua gente;
il re ti aspetta, innamorato
della bellezza tua, vivrai con lui per sempre.
\endverse
\beginverse
\chordsoff
Rivestiti di amore e di dolcezza,
al tuo Signore offri il tuo sorriso,
i vostri figli sono fiori
di questa umanità, un dono del tuo Dio.
\endverse
\endsong

%titolo{È giunta l'ora}
%autore{Scaglianti}
%album{Cantare giovane}
%tonalita{Re}
%gruppo{}
%momenti{}
%identificatore{e_giunta_l_ora}
%data_revisione{2011_12_31}
%trascrittore{Francesco Endrici - Manuel Toniato}
\beginsong{È giunta l'ora}[by={Scaglianti}]

\beginverse
\[A7] È giunta l'\[D]ora \[G] Padre per \[D]me
ai miei a\[Bm]mici \[G] ho detto \[Em]che
\[A9] questa è la \[F#m]vita: \[Bm] conoscere \[F#m]Te
\[G] e il Figlio \[A]tuo \[A9] Cristo Ge\[D]sù.
\endverse

\beginverse
\chordsoff
Erano tuoi, li hai dati a me, \brk ed ora sanno che torno a te.
Hanno creduto, conservali tu, \brk nel tuo amore, nell'unità.
\endverse

\beginverse
\chordsoff
Tu mi hai mandato ai figli tuoi, \brk la tua parola è verità.
E il loro cuore sia pieno di gioia, \brk la gioia vera viene da te.
\endverse

\beginverse
\chordsoff
Io sono in loro e tu in me, \brk e sian perfetti nell'unità,
e il mondo creda che tu mi hai mandato, 
li hai amati come ami me.
\endverse
\endsong

%titolo{È il giorno del Signore}
%autore{Cento}
%album{È il giorno del Signore}
%tonalita{Re-}
%gruppo{}
%momenti{Ingresso}
%identificatore{e_il_giorno_del_signore}
%data_revisione{2011_12_31}
%trascrittore{Francesco Endrici - Manuel Toniato}
\beginsong{È il giorno del Signore}[by={Cento}]

\beginchorus
\[Dm]Fe\[C]sta \[B&7+]fe\[A]sta \[Dm]festa per \[G]noi
oggi è do\[B&]meni\[A4]ca   \[A] 
\[Dm]Fe\[C]sta \[B&7+]fe\[A]sta \[Dm]festa per \[G]noi
è il giorno \[A]del Si\[Dm]gnor.\[C] \[B&] \[C] \[Dm] \[C] \[B&] 
\endchorus

\beginverse
\[C] Ci ha chia\[Dm7]mato il Signore per par\[C]lare con lui
condi\[B&]videre la gioia che ci \[A4]dà \[A] 
ci ha chia\[Dm7]mato il Signore \brk per spez\[C]zare il suo pane
condi\[B&]videre la fraterni\[A4]{tà.} \[A] 
\endverse

\beginverse
\chordsoff
Dio scende fra noi, ci rivela il suo amore,
fa di noi una comunità.
Cuore a cuore ci parla e capisce i problemi
d'ogni uomo che lo ascolterà.
\endverse

\beginverse
\chordsoff
Ogni gioia e dolore ti portiamo, Signore,
la risposta troveremo in te.
Tu ci doni te stesso, vita dentro di noi,
Paradiso, vera eternità.
\endverse
\endsong



%titolo{È l'incontro della vita}
%autore{Ricci}
%album{È l'incontro della vita}
%tonalita{La}
%gruppo{}
%momenti{Ingresso}
%identificatore{e_l_incontro_della_vita}
%data_revisione{2011_12_31}
%trascrittore{Francesco Endrici}
\beginsong{È l'incontro della vita}[by={Ricci}]
\ifchorded
\beginverse*
\vspace*{-0.8\versesep}
{\nolyrics \[A]\[A]\[E]\[E]\[D]\[D]\[E4]\[E]}
\vspace*{-\versesep}
\endverse
\fi
\beginverse
\memorize
È l'in\[A]contro della \[A]vita \brk è l'in\[E]contro intorno a \[E]te.
Tu che \[F#m]sei realtà infi\[F#m]nita \brk tu ci \[D]chiami tutti a \[D]te.
E il tuo \[A]Spirito è una \[E]brezza \brk che dis\[F#m]solve ogni tri\[D]stezza
nell'a\[A]more che tu \[E]vuoi fra di \[D]noi \[D]
nell'a\[A]more che tu \[E]vuoi fra di \[D]noi. \[D]
\endverse
\beginverse
È l'in^contro della ^gioia \brk è l'in^contro fra di ^noi
tu ri^splendi nella ^gloria \brk sei pre^sente in mezzo a ^noi.
Non imp^orta noi chi ^siamo, \brk ciò che im^porta è che ci a^miamo
dell'a^more che tu ^vuoi fra di ^noi ^
dell'a^more che tu ^vuoi fra di ^noi. ^
\endverse
\beginchorus
Nel tuo \[A]cuo\[C#m]re \brk noi tro\[D]viamo il para\[E]diso.
Nel tuo \[A]cuo\[C#m]re \brk noi tro\[D]viamo l'uni\[E]tà.
Nel tuo \[A]cuo\[C#m]re \brk gli oriz\[D]zonti più splen\[A]denti,
nel tuo \[F#m]cuore è l'\[C#m]umani\[B]tà. \[E]
\endchorus
\beginverse
È l'in^contro dei fra^telli \brk tutti u^niti qui con ^te
e i pro^positi più ^belli \brk adesso ^nascono con ^te.
C'è la ^forza, la sor^gente, \brk la più ^pura delle ^fonti 
nell'a^more che tu ^vuoi fra di ^noi ^ \brk nell'a^more che tu ^vuoi fra di ^noi. ^
\endverse
\beginchorus
Nel tuo \[A]cuo\[C#m]re \brk noi tro\[D]viamo il para\[E]diso.
Nel tuo \[A]cuo\[C#m]re \brk noi tro\[D]viamo l'uni\[E]tà.
Nel tuo \[A]cuo\[C#m]re \brk gli oriz\[D]zonti più splen\[A]denti,
nel tuo \[F#m]cuore è l'\[C#m]umani\[D Bm A]tà. 
\endchorus
\endsong

%titolo{È la gioia che fa cantare}
%autore{Conte, Ferrante}
%album{Cantiamo con gioia}
%tonalita{Re}
%gruppo{}
%momenti{Ingresso;Congedo}
%identificatore{e_la_gioia_che_fa_cantare}
%data_revisione{2011_12_31}
%trascrittore{Francesco Endrici}
\beginsong{È la gioia che fa cantare}[by={Conte, Ferrante}]
\beginchorus
\[D]È la gioia che \[G]fa can\[A]tare \[D]
celebrando il Si\[G]gno\[A]re. \[D]
Il suo Spirito \[G]oggi canta in \[A]me! \[D]
È la gioia che \[G]fa can\[A]tare \[D]
celebrando il Si\[G]gno\[A]re. \[Bm]
Il suo Spirito \[G]oggi \[A]canta in \[D]me!
\endchorus
\beginverse
\[A] Io canto alla \[Bm]gloria \[G]tua, \[A]
perché hai vinto la \[Bm]mor\[G]te, \[A]
mia potente sal\[Bm]vez\[G]za,
mia \[C]forza sei \[A]Tu.
\endverse
\beginverse
\chordsoff
^ Tu raduni il tuo ^popo^lo ^ e sconfiggi le ^tene^bre. ^
Il tuo esercito ^siamo ^noi: "Vit^toria di ^Dio!".
\endverse
\beginchorus
\[D]È la gioia che \[G]fa can\[A]tare \[D]
celebrando il Si\[G]gno\[A]re. \[D]
Il suo Spirito \[G]oggi canta in \[A]me! \[D]
È la gioia che \[G]fa can\[A]tare \[D]
celebrando il Si\[G]gno\[A]re. \[Bm]
Il suo Spirito \[G]oggi canta in \[Bm]me! \[G]
Il suo Spirito \[A]oggi canta in \[D]me!
\endchorus
\endsong






%titolo{E la strada si apre}
%autore{Gen Arcobaleno}
%album{E la strada si apre}
%tonalita{Mi-}
%gruppo{}
%momenti{Congedo}
%identificatore{e_la_strada_si_apre}
%data_revisione{2011_12_31}
%trascrittore{Francesco Endrici}
\beginsong{E la strada si apre}[by={Gen\ Arcobaleno}]
\ifchorded
\beginverse*
\vspace*{-0.8\versesep}
{\nolyrics \[Em] \[D] \[G] \[D] \[Em] \[D] \[G] \[D]}
\vspace*{-\versesep}
\endverse
\fi
\beginverse
\memorize
\[Em]Raggio che \[D]buca le \[G]nubi \brk ed è \[D]già cielo a\[Em]perto \[D] \[G] \[D]
\[Em]acqua che \[D]scende de\[G]cisa sca\[D]vando da \[F]sé
l'argine \[E]per la vita, \[Am]
la traiet\[G]toria di un \[D]volo che \[Em]
sull'oriz\[D]zonte di \[C]sera
\[Am]tutto di \[Bm]questa na\[C]tura ha una \[D]strada per \[Em]sé. \brk \[D] \[G] \[D]
\[Em]Attimo \[D]che segue \[G]attimo un \[D]salto nel \[Em]tempo \[D] \[G] \[D]
\[Em]passi di un \[D]mondo che \[G]tende ora\[D]mai all'uni\[F]tà
che non è \[E]più domani, \brk \[Am] usiamo al\[G]lora queste \[D]mani \[Em]
scaviamo a \[D]fondo nel \[C]cuore
\[Am]solo \[Bm]scegliendo l'a\[C]more il \[D4]mondo ve\[D]{drà\dots}
\endverse
\beginchorus
Che la strada si \[G]apre
\[D]passo dopo \[C]passo
\[G]ora \[D] su questa strada \[C]noi. \[B7]
E si spalanca un \[Em]cielo
un \[D]mondo che ri\[C]nasce
si può \[Em]vivere \[C] per l'uni\[D4]tà. \[D]
\endchorus
\ifchorded
\beginverse*
\vspace*{-\versesep}
{\nolyrics \[Em] \[D] \[G] \[D] \[Em] \[D] \[G] \[D]}
\endverse
\fi
\beginverse
^Nave che ^segue una ^rotta \brk in ^mezzo alle ^onde ^^^
^uomo che s'^apre la ^strada in una ^giungla di i^dee
seguendo ^sempre il sole, ^
quando si ^sente asse^tato ^
deve rag^giungere l'^acqua 
^sabbia che ^nella ri^sacca ri^torna al ^mare \brk ^^\[E] \[Am]
Usiamo al\[G]lora queste \[D]mani \[Em]
scaviamo a \[D]fondo nel \[C]cuore
\[Am]solo \[Bm]scegliendo l'a\[C]more il \[D4]mondo ve\[D]{drà\dots}
\endverse
\endsong

%titolo{È pace intima}
%autore{Gen Rosso}
%album{Se siamo uniti}
%tonalita{Mi}
%gruppo{}
%momenti{}
%identificatore{e_pace_intima}
%data_revisione{2011_12_31}
%trascrittore{Francesco Endrici - Manuel Toniato}
\beginsong{È pace intima}[by={Gen\ Rosso}]

\ifchorded
\beginverse*
\vspace*{-0.8\versesep}
{\nolyrics \[E] \[A] \[B] \[A] \[B] \[E] \[F#7] \[D4/9] \[E] }
\vspace*{-\versesep}
\endverse
\fi
\beginverse
\[C]Le ore \[Dm7]volano via \[C]il tempo s'\[F]avvi\[G]cina,
\[C]lungo la \[Dm]strada \[C]canto per \[G4]te. \[G7] 
\[Am]Nella tua \[Em7]casa so \[F]che t'incontre\[C]rò
\[Dm7]e \[C] sa\[F]rà una \[D9]festa tro\[C]varti an\[G4]co\[G]ra.
\endverse

\beginchorus
\[E]È pace \[A]intima \[B]la tua pre\[A]sen\[B]za \[E]{qui} 
\[E]mistero \[F#7]che non so \[D4/9]spiegarmi \[E]mai.
\[E]È cielo \[A]limpido \[B]è gioia \[A]pu\[B]ra \[E]che 
\[E]mi fa cono\[F#7]scere \[D4/9]chi sei per \[E]me
\endchorus

\beginverse
\chordsoff
Sembra impossibile ormai pensare ad altre cose, 
non posso fare a meno di te.
Sembrano eterni gli attimi che non ci sei
ed aspetto solo di ritrovarti.
\endverse

\beginverse
\chordsoff
È la più bella poesia dirti il mio sì per sempre
e nel segreto parlare con te:
Semplici cose, parole che tu sai,
note del mio canto nel tuo silenzio.
\endverse
\endsong


%titolo{È più bello insieme}
%autore{Gen Verde}
%album{Accordi}
%tonalita{Re}
%gruppo{}
%momenti{Congedo}
%identificatore{e_piu_bello_insieme}
%data_revisione{2011_12_31}
%trascrittore{Francesco Endrici}
\beginsong{È più bello insieme}[by={Gen\ Verde},ititle={Insieme è più bello}]
\ifchorded
\beginverse*
\vspace*{-0.8\versesep}
{\nolyrics \[D]\[A]\[G]\[G]\[D]\[A]\[G]}
\vspace*{-\versesep}
\endverse
\fi
\beginverse
\memorize
\[D] Dietro i volti \[A]sconosciuti \[G]
della gente \[D4]che mi \[D]sfiora, \[G]
quanta vita, \[D]quante attese \[Em]di felici\[A4]tà, \[A]\[D]
quanti atti\[A]mi vissuti, \[G]
mondi da sco\[D4]prire an\[D]cora, \[G]
splendidi uni\[D]versi accanto a \[A4]me. \[A7]
\endverse
\beginchorus
\[G]È più \[A]bello insieme
è un \[Bm]dono grande l'\[7]altra gente,
\[G]è più \[A]bello insieme. \[Bm7 (G)] \rep{2}
\endchorus
\ifchorded
\beginverse*
\vspace*{-\versesep}
{\nolyrics \[D]\[A]\[G]\[G]\[D]\[A]\[G]\[G]}
\endverse
\fi
\beginverse
^ E raccolgo ^nel mio cuore ^
la speranza ed ^il do^lore ^
il silenzio, il ^pianto della ^gente attorno a ^me. ^^
In quel pianto, in ^quel sorriso ^
è il mio pianto, il ^mio sor^riso ^
chi mi vive ac^canto è un altro ^me. ^
\endverse
\ifchorded
\beginverse*
\vspace*{-\versesep}
{\nolyrics \[D]\[A]\[G]\[G]\[D]\[A]\[G]\[G]\[B]\[E]}
\endverse
\fi
\beginverse
\transpose{2}
^ Fra le case e i ^grattacieli, ^
fra le antenne ^lassù in ^alto, ^
così traspa^rente il cielo ^non l'ho visto ^mai. ^^
E la luce ^getta veli ^
di colore ^sull'a^sfalto ^
ora che can^tate assieme a ^me.
\endverse
\beginchorus
\transpose{2}
\[G]È più \[A]bello insieme
è un \[Bm]dono grande l'\[7]altra gente,
\[G]è più \[A]bello insieme. \[Bm7 (G)] \rep{2}
\endchorus
\ifchorded
\beginverse*
\vspace*{-\versesep}
{\nolyrics \[D]\[A]\[G]\[G]\[D]\[A]\[G]\[G]}
\endverse
\fi
\endsong

%titolo{E se non fosse un sogno}
%autore{}
%album{}
%tonalita{Re}
%gruppo{}
%momenti{}
%identificatore{e_se_non_fosse_un_sogno}
%data_revisione{2011_12_31}
%trascrittore{Francesco Endrici}
\beginsong{E se non fosse un sogno}
\beginchorus
E se non \[D]fosse un \[A]sogno se \[Bm]tutto fosse \[F#m]vero?
Se \[G]fosse vero a\[D]more che il \[Em]mondo fa gi\[A]rar?
E se non \[D]fosse un \[A]sogno se \[Bm]tutto fosse \[F#m]vero?
Po\[G]tresti far qual\[D]cosa per \[Em]rinnovar con \[A]noi
L'intera umani\[D]tà. \[Em] \[F#m] \[Em] 
\endchorus

\beginverse
\[D]Se coraggio a\[F#m]vrai, \[G]alto vole\[D]rai
\[G]verso un oriz\[D]zonte che non \[Em]è più uto\[A]pia.
\[D]E quel mondo in \[F#m]cui  \[G]non speravi \[D]più
\[Em]lo vedremo insieme io e \[A]te.
\endverse

\beginverse
Se coraggio avrai, seme ti farai
che caduto in terra marcisce in umiltà.
Piccolo così sembra nulla , ma
un grandioso albero sarà.
\endverse

\beginverse
Se coraggio avrai, certo crederai
che la vera vita dalla morte nascerà.
La fragilità forza diverrà
quando le sue orme seguirai.
\endverse
\endsong

%titolo{E sei rimasto qui}
%autore{Gen Rosso}
%album{Congresso Eucaristico nazionale, Bologna 1997}
%tonalita{Fa}
%gruppo{}
%momenti{Comunione}
%identificatore{e_sei_rimasto_qui}
%data_revisione{2011_12_31}
%trascrittore{Francesco Endrici - Manuel Toniato}
\beginsong{E sei rimasto qui}[by={Gen\ Rosso}]

\ifchorded
\beginverse*
\vspace*{-0.8\versesep}
{\nolyrics \[F] \[C] \[F] \[F] \[C] \[B&] }
\vspace*{-\versesep}
\endverse
\fi

\beginverse
\memorize
\[F] Perché la sete d'infi\[B&/F]nito? \[Gm] \brk Perché la fame d'immor\[Dm]tali\[C]tà?
\[F] Sei Tu che hai messo dentro \[B&/F]l'uomo \[Gm] \brk il desiderio dell'e\[F/C]terni\[C]tà!
Ma \[Gm]Tu sapevi \[F/A]che quel vuoto \[B&]lo colmavi \[F/A]Tu,
per \[Gm]questo sei ve\[F/A]nuto in mezzo a \[C]noi.
\endverse

\beginchorus
E \[F]sei rimasto qui, \[B&/F]visibile mistero.
E \[F]sei rimasto qui, \[Dm]cuore del mondo in\[C]tero.
E \[B&]rimarrai con noi fin\[Am]ché quest'uni\[Dm]verso gire\[G]rà.
Sal\[F]vezza dell'u\[C]mani\[F]tà. \[C] 
\endchorus

\beginverse
\[D] Si apre il cielo del fu\[G/D]turo,  \[Em] \brk il muro della morte or\[Bm]mai non \[A]c'è.
\[D]Tu, Pane vivo, ci fai \[G/B]Uno: \[Em] \brk richiami tutti i figli at\[D]torno a \[A]Te.
E \[Em]doni il tuo \[D/F#]Spirito che \[G]lascia dentro \[D/F#]noi
il \[Em]germe della  \[D/F#]sua immortali\[A]tà.
\endverse

\beginchorus
\[D]Sei rimasto qui, \[G/D]visibile mistero.
\[D]Sei rimasto qui, \[Bm]cuore del mondo in\[A]tero.
E \[G]rimarrai con noi fin\[F#m]ché quest'uni\[Bm]verso gire\[E]rà.
Sal\[D]vezza dell'u\[A]mani\[D]tà. \[C] 
\endchorus

\beginverse

^ Presenza vera nel mi^stero, ^ \brk ma più reale di ogni ^realtà, ^
^ da te ogni cosa prende ^vita  ^ \brk e tutto un giorno a te ri^torne^rà.
Var^cando l'infi^nito tutti ^troveremo in ^Te
un ^Sole immenso ^di felici\[Gm]tà. \[Am]\[B&]\[C]
\endverse

\beginverse*
\itshape \[F/A]Noi,  \[B&/D]trasformati in \[C]Te, sa\[F]remo il seme \[B&]che
fa\[Gm]rà fiorire l'\[F/A]universo \[B&]nella Trini\[C]tà.
\[F/A]Noi,  \[B&/D]trasformati in \[C]Te, sa\[F]remo il seme \[B&]che
fa\[Gm]rà fiorire \[F/A]tutto l'uni\[B&]verso insieme a \[C]Te.
\endverse

\beginchorus
E \[G]sei rimasto qui, vi\[C]sibile mistero.
\[G]Sei rimasto qui, \[Em]cuore del mondo in\[D]tero.
E \[C]rimarrai con noi \brk fin\[Bm]ché quest'uni\[Em]verso gire\[A]rà.  \[D] 
\[G]Sei rimasto qui, vi\[C]sibile mistero.
\[G]Sei rimasto qui, \[Em]cuore del mondo in\[D]tero.
E \[C]rimarrai con noi fin\[Bm]ché quest'uni\[Em]verso gire\[A]rà.
\[G/D]Ieri oggi e sempre. \[Am] \[G/B] 
\[C]Sal\[C7+/D]vezza dell'u\[D]manità. \[G] 
\[D] \[G] \[G/B] \[D] \[G] \[C] \[G] 
\endchorus
\endsong


%titolo{E sia sì}
%autore{Antonioli, Giorgi}
%album{}
%tonalita{Sol}
%gruppo{}
%momenti{Congedo}
%identificatore{e_sia_si}
%data_revisione{2011_12_31}
%trascrittore{Francesco Endrici}
\beginsong{E sia sì}[by={Antonioli, Giorgi}]
\ifchorded
\beginverse*
\vspace*{-0.8\versesep}
{\nolyrics \[G]\[Am]\[G]\[C]\[G]\[Am7]\[D] \[G]\[Am]\[G]\[C]\[G]\[Am7]\[D] \[G]\[C]\[G]\[C] }
\vspace*{-\versesep}
\endverse
\fi
\beginverse
\[G]Quando negli occhi si ac\[D]cende il sorriso
e un \[C]raggio di sole co\[Bm7]lora il mio viso
mi \[C]sento leggero e \[G]provo a volare,
in \[A]pace con tutti io \[D7]voglio giocare.
Ed \[C]ora io cam\[D]mino e canto \[G]in salita,
\[C]poche note e \[D]dire che è \[G]bella la vita.
Già \[C]nasce in si\[D]lenzio, \[B7]dentro il mio \[Em7]cuore
un \[C]piccolo \[G]seme per \[A]frutti d'a\[D]more.
\endverse
\beginchorus
\[G]Eccomi, Si\[D]gnore, sono \[G]qui, \[Am7]ma che pa\[D]ura \[C]dire \[D]sì
\[C]so che tu sei vi\[G]cino a me
\[C]chicco maturo sa\[G]rò per te
\[A]se dentro il campo del mio cuore
le \[Cdim]spighe al grano cambiano colore.
\[G]Sì, nel vento cante\[Am]rò
e \[C]sia \[G]sì, Si\[C]gnore, il mio \[G]sì.
\endchorus
\ifchorded
\beginverse*
\vspace*{-\versesep}
{\nolyrics \[G]\[Am]\[G]\[C]\[G]\[Am7]\[D] \[G]\[Am]\[G]\[C]\[G]\[Am7]\[D] \[G]\[C]\[G]\[C] }
\endverse
\fi
\beginverse
\chordsoff
Quando nel cuore si sente una voce
una mano che prende la mano e mi dice
parole d'amore, io provo a pensare:
la voce che chiama è il mio Signore.
Ancora io non vedo e cerco qui tra i fiori
d'ogni specie e frutti e di tanti colori:
già nasce in silenzio dentro il mio cuore
di un chicco di grano, una storia d'amore!
\endverse
\beginchorus
\chordsoff
Eccomi, Signore, sono qui, ma che paura dire sì
so che tu sei vicino a me seme buono sarò per te
se nel campo del mio cuore
le spighe al grano cambiano colore.
Sì, nel vento canterò
e sia sì, Signore, il mio sì
e sia sì, Signore, il mio sì.
\endchorus
\endsong

%titolo{E sono solo un uomo}
%autore{Sequeri}
%album{In cerca d'autore}
%tonalita{Re}
%gruppo{}
%momenti{Comunione}
%identificatore{e_sono_solo_un_uomo}
%data_revisione{2011_12_31}
%trascrittore{Francesco Endrici}
\beginsong{E sono solo un uomo}[by={Sequeri}]
\beginverse
\[D]Io lo so Si\[F#m]gnore che \[G]vengo da lon\[D]tano
\[D]prima nel pen\[F#m]siero e \[G]poi nella tua \[A7]mano
\[D]io mi rendo \[A]conto che \[G]Tu sei la mia \[D]vita
e \[G]non mi sembra \[Em]vero di pre\[E7]garti co\[A4/7]sì.
\endverse
\beginverse*
^Padre di ogni ^uomo e ^non ti ho visto ^mai.
Spirito di ^Vita e ^nacqui da una ^donna
^Figlio mio fra^tello e ^sono solo un ^uomo
ep^pure io ca^pisco che ^Tu sei veri\[A7]tà.
\endverse

\beginchorus
E im\[D]parerò a guar\[G]dare tutto il \[A]mondo \[D7]
con gli \[G]occhi traspa\[A]renti di un bam\[D]bino \[D7]
e in\[G]segnerò a chia\[A]marti Padre \[D]nostro \[B7]
ad \[Em]ogni figlio \[E7 (A7)]che diventa \[A7 (D)]uomo. \rep{2}
\endchorus
\beginverse
%\chordsoff
^Io lo so Si^gnore che ^Tu mi sei vi^cino,
^luce alla mia ^mente ^guida al mio cam^mino,
^mano che sor^regge ^sguardo che per^dona
e ^non mi sembra ^vero che Tu e^sista co^sì.
\endverse
\beginverse*
^Dove nasce a^more ^Tu sei la sor^gente.
^Dove c'è una ^croce ^Tu sei la spe^ranza
^dove il tempo ha ^fine ^Tu sei vita e^terna
e ^so che posso ^sempre con^tare su di \[A7]Te.
\endverse
\beginchorus
%\chordsoff
E ac\[D]coglierò la \[G]vita come un \[A]dono \[D7]
e a\[G]vrò il coraggio \[A]di morire an\[D]ch'io \[D7]
e in\[G]contro a Te ver\[A]rò col mio fra\[D]tello \[B7]
che \[Em]non si sente a\[E7 (A7)]mato da nes\[A7 (D)]suno. \rep{2}
\endchorus
\endsong

%titolo{Ecco il nostro sì}
%autore{Fossi}
%album{}
%tonalita{Re}
%gruppo{}
%momenti{Maria;Vocazione}
%identificatore{ecco_il_nostro_si}
%data_revisione{2011_12_31}
%trascrittore{Francesco Endrici - Manuel Toniato}
\beginsong{Ecco il nostro sì}[by={Fossi}]
\beginverse
\[D]Fra tutte le \[F#m]donne scelta in \[G]Nazareth, \[A] 
\[Bm] sul tuo volto ri\[G]splende 
\[Em] il coraggio di \[A]quando hai detto "Sì".
\[D] Insegna a \[F#m]questo cuore \[G]l'umiltà, \[A] 
\[Bm] il silenzio d'a\[G]more, 
\[Em] la Speranza nel \[A]Figlio tuo Gesù.
\endverse

\beginchorus
\[D]Ecco il nostro \[G]Sì, 
nuova \[Em]luce che ri\[A]schiara il giorno,
\[F#m]è bellissi\[G]mo rega\[Em]lare al mondo \[A]la Speranza.
\[D]Ecco il nostro \[G]Sì, 
cammi\[Em]niamo insieme a \[A]te Maria,
\[F#m]Madre di Ge\[G]sù, madre \[Em]dell'umanità.  \[A] 
\endchorus

\beginverse
\chordsoff
Nella tua casa il verbo si rivelò 
nel segreto del cuore il respiro del figlio Emmanuel.
In segna a queste mani la fedeltà,
a costruire la pace, una casa comune insieme a te.     
\endverse

\beginverse
\chordsoff
Donna dei nostri giorni sostienici,
guida il nostro cammino con la forza di quando hai detto "Sì".
Insegnaci ad accogliere Gesù, 
noi saremo Dimora, la più bella poesia dell'anima.
\endverse
\endsong

%titolo{Ecco il pane}
%autore{Ricci}
%album{Sei venuto dal cielo}
%tonalita{Fa#-}
%gruppo{}
%momenti{Offertorio}
%identificatore{ecco_il_pane}
%data_revisione{2011_12_31}
%trascrittore{Francesco Endrici - Manuel Toniato}
\beginsong{Ecco il pane}[by={Ricci}]

\ifchorded
\beginverse*
\vspace*{-0.8\versesep}
{\nolyrics \[F#m] \[E6] \[D7] \[E6] }
\vspace*{-\versesep}
\endverse
\fi
\beginverse*
\[F#m]Ecco il pane e il vino della \[E]Cena insieme a \[C#m]Te,
\[A]ecco questa \[E]vita che tra\[D]sformi: \[Bm] 
cieli e \[F#m]terre nuove in noi, tu di\[E]scendi dentro \[C#m]{noi\ldots}
fatti \[A]come \[E]te, noi con Te nel\[D]l'ani\[Bm]ma.
\endverse
\endsong

%titolo{Ecco l'uomo}
%autore{}
%album{}
%tonalita{Mi-}
%gruppo{}
%momenti{Triduo Pasquale}
%identificatore{ecco_l_uomo}
%data_revisione{2011_12_31}
%trascrittore{Francesco Endrici - Manuel Toniato}
\beginsong{Ecco l'uomo}[by={}]
\beginverse
\[Em7]Nella memoria di \[Am7]questa Passione
\[D7]noi ti chiediamo per\[G]dono, Si\[B7]gnore
\[Em7]per ogni volta che \[Am7]abbiamo lasciato
\[D7]il tuo fratello so\ch{B}{f}{f}{ff}rire da \[B7]solo.
\endverse

\beginchorus
\[Em]Noi ti pre\[Am]ghiamo \[D7]Uomo della \[G]Croce;
\[Em]Figlio e Fratel\[F#mdim]lo, \[B7/9]noi speriamo in \[Em]te! \rep{2}
\endchorus

\beginverse
\chordsoff
Nella memoria di questa tua Morte
noi ti chiediamo coraggio, Signore
per ogni volta che il dono d'amore
ci chiederà di soffrire da soli.
\endverse

\beginverse
\chordsoff
Nella memoria dell'Ultima Cena
noi spezzeremo di nuovo il tuo Pane
ed ogni volta il tuo Corpo donato
sarà la nostra speranza di vita.
\endverse
\endsong

%titolo{Ecco quel che abbiamo}
%autore{Gen Verde}
%album{Accordi}
%tonalita{Do}
%gruppo{}
%momenti{Offertorio}
%identificatore{ecco_quel_che_abbiamo}
%data_revisione{2011_12_31}
%trascrittore{Francesco Endrici}
\beginsong{Ecco quel che abbiamo}[by={Gen\ Verde}]
\beginverse
\[C]Ecco quel che ab\[G]biamo, nulla \[F]ci appartiene, or\[C]mai.
\[Em]Ecco i \[Am]frutti della \[Em]terra, che Tu \[F]moltipliche\[G]rai.
\[C]Ecco queste \[G]mani, puoi u\[F]sarle, se lo \[C]vuoi, 
\[Em]per di\[Am]videre nel \[Em]mondo il pane \[F]che Tu hai \[G]dato a \[C]noi. \[Am]
\endverse
\beginchorus
Solo una \[Am]goccia hai messo \[Em]fra le mani \[Em7]mie,
solo una goccia che tu \[A]ora chiedi a \[A7]me,
\ifchorded
{\nolyrics \[Dm7]\[G]\[Em]\[A]}
\fi
una \[Dm7]goccia che in mano a \[F7]te,
una \[Dm7]pioggia divente\[Em7]rà e la \[F]terra feconde\[C]rà.
\endchorus
\beginverse
^Sulle strade, il ^vento da lon^tano porte^rà
^il pro^fumo del fru^mento, che ^tutti avvolge^rà.
^E sarà l’a^more che il rac^colto sparti^rà
^e il mi^racolo del ^pane in terra ^si ri^pete^rà. ^
\endverse
\beginchorus
Le nostre \[Am]gocce, pioggia \[Em]fra le mani \[Em7]tue,
saranno linfa di una \[A]nuova civil\[A7]tà
\ifchorded
{\nolyrics \[Dm7]\[G]\[Em]\[A]}
\fi
e la \[Dm7]terra prepare\[F7]rà 
la \[Dm7]festa del pane \[Em7]che ogni \[F]uomo condivide\[C]rà.
\endchorus
\endsong

%titolo{Eccomi}
%autore{Frisina}
%album{Non temere}
%tonalita{Do}
%gruppo{}
%momenti{Comunione;Salmi}
%identificatore{eccomi}
%data_revisione{2011_12_31}
%trascrittore{Francesco Endrici - Manuel Toniato}
\beginsong{Eccomi}[by={Frisina}]

\beginchorus
\[C]Ecco\[G]mi, \[Dm]ecco\[Am]mi! \[F]Si\[C]gnore io ven\[G4]go. \[G] 
\[Am]Ecco\[Em]mi, \[F]ecco\[C]mi! \[F]si compia in \[Am]me \brk la tua \[G4]volon\[G]tà. \[C] 
\endchorus

\beginverse
\[C]Nel mio Si\[F]gnore ho spe\[C]rato \brk \[Am]e su di me si è chi\[G4]na\[G]to
\[Dm]ha dato a\[G7]scolto al mio \[Am]gri\[Em]do \brk m'\[F]ha libe\[D7]rato dalla \[G4]mor\[G]te.
\endverse

\beginverse
\chordsoff
I miei piedi ha reso saldi \brk sicuri ha reso i miei passi.
Ha messo sulla mia bocca \brk un nuovo canto di lode.
\endverse

\beginverse
\chordsoff
Il sacrificio non gradisci \brk ma mi hai aperto l'orecchio
non hai voluto olocausti, \brk allora ho detto: io vengo!
\endverse

\beginverse
\chordsoff
Sul tuo libro di me è scritto: \brk si compia il tuo volere.
Questo, mio Dio, desidero, \brk la tua legge è nel mio cuore.
\endverse

\beginverse
\chordsoff
La tua giustizia ho proclamato \brk non tengo chiuse le labbra;
non rifiutarmi, Signore, \brk la tua misericordia.
\endverse
\endsong


%titolo{Ed ho la gioia nel cuore}
%autore{Popolare}
%album{Sempre canterò}
%tonalita{Mi}
%gruppo{}
%momenti{}
%identificatore{ed_ho_la_gioia_nel_cuore}
%data_revisione{2011_12_31}
%trascrittore{Francesco Endrici - Manuel Toniato}
\beginsong{Ed ho la gioia nel cuore}[by={Popolare\ dell'America\ latina}]

\beginverse
Ed ho la \[E]gioia nel cuore, gioia nel cuore,
gioia nel cuore dentro \[B7]me,
alleluia, Gloria a \[E]Dio.
\endverse

\beginchorus
È come un \[E]fiume d'acqua viva, \echo{viva}
\[E]fiume d'acqua viva, \echo{viva}
fiume d'acqua viva dentro me.
Alza le brac\[E]cia e loda il tuo Si\[B7]gnor,
alza le braccia e loda il tuo Si\[E]gnor,
dà lode a Di\[E]o, gloria a Dio, lode a \[A]Lui.
Alza le \[E]braccia e \[B7]loda il tuo Si\[E]gnor.
\endchorus

\beginverse
\chordsoff
Ed ho la pace nel cuore, pace nel cuore,
pace nel cuore dentro me,
alleluia, Gloria a Dio.
\endverse

\beginverse
\chordsoff
Ho l'A.C.R. nel cuore, A.C.R. nel cuore,
A.C.R. nel cuore dentro me,
alleluia, Gloria a Dio.
\endverse
\endsong

%titolo{Effonderò il mio Spirito}
%autore{Frisina}
%album{Benedici il Signore}
%tonalita{Sol}
%gruppo{}
%momenti{Pentecoste;Ingresso}
%identificatore{effondero_il_mio_spirito}
%data_revisione{2011_12_31}
%trascrittore{Francesco Endrici - Manuel Toniato}
\beginsong{Effonderò il mio Spirito}[by={Frisina}]

\beginchorus
\[G]Effonde\[D]rò il mio \[C]Spirito, \[G]su ogni crea\[D]tura
\[Am]effonde\[Em]rò  \[C]la mia \[G]gioia, \[Am]la mia \[Em]pace sul \[Am7]mon\[D]do.
\endchorus

\beginverse
\[G]Vieni, o \[C]Spirito \[Am]Consola\[G]to\[D]re,
\[Em]vieni e\ch{C}{f}{f}{ff}ondi sul \[Am]mondo la \[G]tu\[Em]a dol\[Am]cez\[D]za.
\endverse
\beginverse
\chordsoff
Vieni e dona ai tuoi figli la pace,
vieni e donaci la tua forza.
\endverse
\beginverse
\chordsoff
Vieni, o Spirito Onnipotente,
vieni, e crea negli uomini un cuore nuovo.
\endverse
\beginverse
\chordsoff
Vieni, e dona ai tuoi figli l'amore,
vieni, riscalda il cuore del mondo.
\endverse
\endsong


%titolo{Emmanuel tu sei}
%autore{Buttazzo}
%album{Gloria all'Emmanuele}
%tonalita{Sol}
%gruppo{}
%momenti{Natale}
%identificatore{emmanuel_tu_sei}
%data_revisione{2011_12_31}
%trascrittore{Francesco Endrici - Manuel Toniato}
\beginsong{Emmanuel tu sei}[by={Buttazzo}]

\ifchorded
\beginverse*
\vspace*{-0.8\versesep}
{\nolyrics \[G] \[D/F#] \[Em7] \[G] \[C] \[G/B] \[D4] \[D] }
\vspace*{-\versesep}
\endverse
\fi
\beginverse
\[G]Emmanu\[D/F#]el, \[Em]tu sei qui con \[G/D]me,
\[C]vieni piccolis\[G]simo in\[Am7]contro a \[D4]me. \[D] 
\[G]Emmanu\[D/F#]el, \[Em]amico silen\[Bm7]zioso,
\[C]della tua pre\[G]senza riempi i \[Am7]giorni \[D4]miei\[D].
\endverse

\beginchorus
\[G]Sei la \[D]luce \[C]dentro \[G]me, \[C]sei la \[G]via da\[Am7]vanti a \[D4]me, \[D] 
\[G]nella \[D]storia ir\[B]rompi \[Em]tu: \[G]  \brk \[C]io ti ac\[D]colgo, mio Si\[C]gnor. 
\endchorus
\ifchorded
\beginverse*
\vspace*{-\versesep}
{\nolyrics \[G] \[Am7] \[D4/7] }
\vspace*{-\versesep}
\endverse
\fi
\beginverse
\[G]Emmanu\[D/F#]el, \[Em]tu sei qui con \[G/D]me.
\[C]La tua dolce \[G]voce parla \[Am7]dentro \[D4]me, \[D] 
\[G]il cuore \[D/F#]mio    \[Em]sente che ci \[Bm7]sei,
\[C]e nell'amore \[G]vuoi guidare i \[Am7]passi \[D4]miei\[D].
\endverse

\beginchorus
\[G]Sei la \[D]luce \[C]dentro \[G]me, \[C]sei la \[G]via da\[Am7]vanti a \[D4]me, \[D] 
\[G]nella \[D]storia ir\[B]rompi \[Em]tu: \[G]  \brk \[C]io ti ac\[D]colgo, mio Si\[G]gnor.
\endchorus

\ifchorded
\beginverse*
\vspace*{-\versesep}
{\nolyrics \[D/F#] \[Em7] \[G/D] \[C] \[G/B] \[Am7] \[D] \[E] }
\vspace*{-\versesep}
\endverse
\fi

\beginverse
\[A]Emmanu\[E/G#]el, \[F#m]tu sei qui con \[A/E]me,
\[D]riempi di spe\[A]ranza tutti i \[Bm7]sogni \[E4]miei\[E].
\[A]Sei tu il mio \[E/G#]Re,    il \[F#m]Dio della mia \[C#m7]vita,
\[D]fonte di sal\[A]vezza per l'u\[Bm7]manità.\[E4]  \[E] 
\endverse
\beginchorus
\[A]Sei la \[E]luce \[D]dentro \[A]me, \[D]sei la \[A]via da\[Bm7]vanti a \[E4]me, \[E] 
\[A]nella \[E]storia ir\[C#]rompi \[F#m]tu: \[A]  \brk \[D]io ti ac\[E]colgo, mio Si\[D]gnor.
\endchorus
\ifchorded
\beginverse*
\vspace*{-\versesep}
{\nolyrics \[A] \[Bm7] \[E4/7] \[A] }
\endverse
\fi
\endsong

%titolo{Esci dalla tua terra}
%autore{Cocquio}
%album{Agape}
%tonalita{Sol}
%gruppo{}
%momenti{Congedo}
%identificatore{esci_dalla_tua_terra}
%data_revisione{2011_12_31}
%trascrittore{Francesco Endrici}
\beginsong{Esci dalla tua terra}[by={Cocquio}]
\beginchorus
\[G]Esci dalla tua \[C]terra e va', \[D]dove ti mostre\[G]rò;
\[G]esci dalla tua \[C]terra e va', \[D]dove ti mostre\[G]rò.
\endchorus
\beginverse
A\[Gm]bramo, non andare, non par\[Cm]tire,
non lasciare la tua \[D]casa
cosa \[D7]speri di tro\[Gm]var?
La \[Gm]strada è sempre quella,
ma la \[Cm]gente è differente, ti è ne\[D]mica,
dove \[D7]speri di arri\[G]var?

Quello che lasci \[C]tu lo conosci,
\[D]il tuo Signore \[G]cosa ti dà?
Un \[G]popolo, la \[C]terra e la pro\[G]messa,
parola \[D7]di Jah\[G]vè.
\endverse
\beginverse
\chordsoff
La rete sulla spiaggia abbandonata
l'han lasciata i pescatori,
son partiti con Gesù.
La folla che osannava se n'è andata,
ma il silenzio una domanda
sembra ai dodici portar.

Quello che lasci tu lo conosci,
il tuo Signore cosa ti dà?
il centuplo quaggiù e l'eternità
Parola di Gesù.
\endverse
\beginverse
\chordsoff
Partire non è tutto
certamente c'è chi parte e non dà niente
cerca solo libertà.
Partire con la fede del Signore,
con l'amore aperto a tutti
può cambiar l'umanità.

Quello che lasci tu lo conosci,
quello che porti vele di più.
Andate e predicate il mio Vangelo
parola di Gesù.
\endverse
\beginchorus
\chordsoff
Esci dalla tua terra e va', dove ti mostrerò,
esci dalla tua terra e va', sempre con te sarò.
\endchorus
\endsong

%titolo{Evenu shalom}
%autore{Tradizionale}
%album{RnS 87}
%tonalita{Re-}
%gruppo{}
%momenti{Pace}
%identificatore{evenu_shalom}
%data_revisione{2011_12_31}
%trascrittore{Francesco Endrici}
\beginsong{Evenu shalom}[ititle={Evenu shalom}]
\beginverse
Evenu \[Dm]shalom alejem
Evenu \[Gm]shalom alejem
Evenu \[A7]shalom a\[Dm]lejem
Evenu \[A7]shalom shalom shalom alej\[Dm]em.
\endverse
\beginverse
\chordsoff
E sia la pace con noi. \rep{3}
Evenu shalom shalom shalom alejem.
\endverse
\beginverse
\chordsoff
Et la paix soit avec nous. \rep{3}
Evenu shalom shalom shalom alejem.
\endverse
\beginverse
\chordsoff
Und sei der Friede mit uns. \rep{3}
Evenu shalom shalom shalom alejem.
\endverse
\beginverse
\chordsoff
And the peace be with us. \rep{3}
Evenu shalom shalom shalom alejem.
\endverse
\beginverse
\chordsoff
Y sea la paz con nosotros. \rep{3}
Evenu shalom shalom shalom alejem.
\endverse
\beginverse
Diciamo ^pace al mondo,
cantiamo ^pace al mondo,
la nostra ^vita sia gio^iosa,
e il mio ^saluto, pace, giunga fino a ^voi.
\endverse
\endsong

%titolo{Exultet}
%autore{Comi}
%album{}
%tonalita{Sol-}
%gruppo{}
%momenti{Pasqua}
%identificatore{exultet}
%data_revisione{2011_12_31}
%trascrittore{Francesco Endrici}
\beginsong{Exultet}[by={Comi}]
\meter{12}{8}
\beginverse*
|\[Gm] |\[Gm] E|\[Gm]sultino le creature del |\[Gm]cielo,
per la vit|\[F]toria del loro creatore e Si|\[Gm]gnore.
Gioisca la |\[Gm]terra inondata da sì grande splen|\[Gm]dore,
sappia che dalle |\[F]tenebre
l'Universo è uscito vinci|\[Gm]tore.
Si rallegri la |\[Gm]Chiesa, nostra |\[E&]madre
\[F]perché risplende su di |\[F]essa una grande |\[Gm]luce. 
E questo |\[E&]tempio ri\[F]suoni dell'im|\[E&]mensa nostra 
acclama|\[D]zione. \[D7]
\endverse
\beginchorus
Alle|\[G]luia, alle|\[Am]luia, alle|\[C]luia, alle|\[G]luia
Alle|\[G]luia, alle|\[Am]luia, alle|\[C]luia, alle|\[G]luia
Alle|\[G]luia, alle|\[Am]luia, alle|\[C]luia, alle|\[G]luia
\chordsoff
(Celebriamo a gran voce colui che l'antico
peccato ha pagato per noi,
Cristo Gesù, risorto dai morti.)
\endchorus
\beginverse*
|\[Gm] |\[Gm] |\[Gm]Questa è la festa di Pasqua,
|\[Gm] in cui viene immolato il vero a|\[Gm]gnello,
che col suo |\[F]sangue protegge le porte dei cre|\[Gm]denti
Questa è la |\[Gm]notte in cui |\[Gm]Dio ha liberato
dall'Egitto i padri |\[F]nostri e li ha con|\[E&]dotti
al di là del mare a piedi a|\[D]sciutti. \[D7]
Questa è la |\[Gm]notte in |\[Gm]cui la colonna di |\[F]luce
dissi|\[E&]pò le tenebre del |\[D]male.
Questa è an\[D7]cora la |\[Gm]notte in cui tutti i cre|\[Gm]denti,
in Cristo ri|\[E&]sorto,
sono strap|\[F]pati dalle \[E&]tenebre del pec|\[Gm]cato
e della |\[A]morte.
\endverse
\beginverse*
Questa è la |\[F]notte in cui |\[F]Cristo ha distrutto
la |\[C]morte e dal se|\[B&]polcro
è risorto glo|\[Gm]rioso e vinci\[A]tore. 
\endverse
\beginverse*
\mbar{6}{8}\[A] Questa è la \mbar{12}{8}\[F]notte in cui |\[F]Cristo ha distrutto
la |\[C]morte e dal se|\[B&]polcro
è risorto glo|\[Gm]rioso e vinci\[A]tore. \mbar{6}{8}\[A]

\mbar{12}{8}\[F]Oh meravigliosa condiscendenza
del tuo amore \mbar{6}{8}\[F]per noi Si\mbar{12}{8}\[B&]gnore!
|\[C]O inestimabile tenerezza del tuo a\mbar{6}{8}\[C7]more Si\mbar{12}{8}\[F]gnore!
|\[A]Per riscattare lo schiavo,
|\[A7]hai consegnato alla \[Dm]morte tuo |\[B&]figlio,
il tuo unico |\[Gm]figlio Ge\[A7]sù. |\[Fm]
\endverse
\beginverse*
Felice colpa, felice |\[Fm]colpa,
felice \[B&m]colpa, felice |\[B&m]colpa, 
felice \[E&]colpa, |\[E&] felice \[Fm]colpa, |\[Fm]
felice colpa, felice |\[Fm]colpa,
felice \[B&m]colpa, felice |\[B&m]colpa,
felice \[E&]colpa,\[7] felice |\[A&]colpa,
felice \[C7]colpa. \mbar{4}{4}\[C7]

Che meri|\[F]tasti |\[F] un tale Reden|\[C]tore |\[C]
che dal se|\[B&]polcro è ri|\[Gm]sorto glorioso e vinci|\[A]tore. |\[A]
Che meri|\[F]tasti |\[F] un tale Reden|\[C]tore |\[C]
che dal se|\[B&]polcro è ri|\[Gm]sorto glorioso e vinci|\[A]tore. |\[A]
Alle\mbar{12}{8}\[F]luia, alle|\[Gm]luia, alle|\[B&]luia, alle|\[F]luia
\endverse
\endsong


\lettera
%titolo{Fa' che io creda}
%autore{De Luca, Conte}
%album{io scelgo te}
%tonalita{Mi}
%gruppo{}
%momenti{Comunione}
%identificatore{fa_che_io_creda}
%data_revisione{2011_12_31}
%trascrittore{Francesco Endrici - Manuel Toniato}
\beginsong{Fa' che io creda}[by={De\ Luca, Conte}]

\ifchorded
\beginverse*
\vspace*{-0.8\versesep}
{\nolyrics \[E] \[A7+] \[E] \[A7+] }
\vspace*{-\versesep}
\endverse
\fi

\beginverse
\[A]Nel mio \[E]cuore \[A]credo che Tu \[B]sei il \[C#m]Figlio di Dio. 
\[A]Tra le Tue mani, Si\[E/G#]gnore, 
quest'acqua in \[F#m]vino si trasforme\[B]rà,
non \[E]c'è un altro \[G#7]Dio come \[C#m]Te.
\[A]Gesù \[E]Santo, \[A]Figlio del Dio \[B]vi\[C#m]vente Tu sei.
\[A]Se oggi io guardo al tuo \[E/G#]cuore 
anche sul\[F#m]l'acqua io cammine\[B]rò,
non \[E]c'è un altro \[G#7]Dio \[C#m]come \[B]Te.
\endverse

\beginchorus
Fa' ch'io \[E]creda, o Si\[A/E]gnore, \brk nel po\[E]tere del Tuo a\[A/E]more,
Come in\[F#m]censo sale \[B]già \brk il mio \[A]grazie per le \[G#]cose che fa\[C#m]{ra}\[B]i.
Fa' ch'io \[E]veda, o Si\[A/E]gnore il po\[E]tere del Tuo \[A/E]nome,
che ogni \[F#m]uomo creda in \[B]Te.
Tu sei \[A]Cristo e nei \[G#]cuori regne\[C#m]rai,                  
il \[B]Figlio di Dio Tu \[E]sei. \[A7+] 
\endchorus

\beginverse
\chordsoff
Nel mio cuore credo che Tu sei il Figlio di Dio. 
Tra le Tue mani, Signore, 
la morte in vita si trasformerà, 
non c'è un altro Dio come Te.
Gesù santo, Figlio del Dio vivente Tu sei.
Se oggi io guardo al Tuo cuore, 
anche sull'acqua io camminerò,
Non c'è un altro Dio come Te.
\endverse

\beginchorus
Fa' ch'io \[E]creda, o Si\[A/E]gnore, \brk nel po\[E]tere del Tuo a\[A/E]more,
Come in\[F#m]censo sale \[B]già \brk il mio \[A]grazie per le \[G#]cose che fa\[C#m]{ra}\[B]i.
Fa' ch'io \[E]veda, o Si\[A/E]gnore il po\[E]tere del Tuo \[A/E]nome,
che ogni \[F#m]uomo creda in \[B]Te.
Tu sei \[A]Cristo e nei \[G#]cuori regne\[C#m]rai,                  
il \[B]Figlio di Dio Tu \[E/B]sei, il \[A/E]Figlio di Dio Tu \[E/B]sei, 
il \[A/B]mio Signore tu \[F#m]{se}\[B]i. \[A] \[G#7] \[C#m] \[B] 

Fa' ch'io \[G/D]veda, o Si\[C/D]gnore, il po\[G/D]tere del Tuo \[C/D]nome,
che ogni \[Am]uomo creda in \[D]Te.
Tu sei \[C]Cristo e nei \[B7]cuori regne\[Em]rai,                  
il \[C6/D]Figlio di Dio Tu \[G/D]sei, il \[C/D]Figlio di Dio Tu \[G/D]sei, 
il \[C/D]mio Signore tu \[G]sei.
\endchorus
\endsong




%titolo{Francesco vai}
%autore{Bizzeti}
%album{Vorrei amare}
%tonalita{Mi-}
%gruppo{}
%momenti{}
%identificatore{francesco_vai}
%data_revisione{2011_12_31}
%trascrittore{Francesco Endrici}
\beginsong{Francesco vai}[by={Bizzeti}]
\beginverse
\[Em]Quello che io vivo non mi \[D]basta \[Em]più,
tutto quel che avevo non mi \[D]serve \[Em]più:
io cerche\[B]rò quello che dav\[Em]vero vale,
e non più il \[Am]servo, ma il pa\[C]drone segui\[B]rò!
\endverse
\beginchorus
Francesco, \[Em]vai, ri\[D]para la mia \[Em]casa!
Fran\[D]cesco, \[Em]vai, non \[D]vedi che è in ro\[G]vina?
E non te\[Am]mere: \[C]io sarò con \[G]te do\[B]vunque an\[Em]drai.
\[D]Francesco, \[Em]vai!
\endchorus
\beginverse
\chordsoff
Nel buio e nel silenzio ti ho cercato, Dio;
dal fondo della notte ho alzato il grido mio
e griderò finché non avrò risposta
per conoscere la tua volontà.
\endverse
\beginverse
\chordsoff
Altissimo Signore, cosa vuoi da me?
Tutto quel che avevo l'ho donato a te.
Ti seguirò nella gioia e nel dolore
e della vita mia una lode a te farò.
\endverse
\beginverse
\chordsoff
Quello che cercavo l'ho trovato qui:
ora ho riscoperto nel mio dirti sì
la libertà di essere figlio tuo,
fratello e sposo di Madonna povertà.
\endverse
\endsong

%titolo{Frutto della nostra terra}
%autore{Buttazzo}
%album{Sorgente di unità}
%tonalita{Sol}
%gruppo{}
%momenti{Offertorio}
%identificatore{frutto_della_nostra_terra}
%data_revisione{2011_12_31}
%trascrittore{Francesco Endrici}
\beginsong{Frutto della nostra terra}[by={Buttazzo}]
\beginverse
\[G]Frutto della nostra \[C]terra \[G]del lavoro di ogni \[D]uomo
\[Em]pane della nostra \[Bm]vita, cibo \[C]della quotidiani\[D]tà.
\[G]Tu che lo prendevi un \[C]giorno, \brk \[G]lo spezzavi per i \[D]tuoi
\[Em]oggi vieni in questo \[Bm]pane, \brk cibo \[C]vero dell'umani\[D]tà.
\endverse
\beginchorus
E sarò \[G]pane, e sarò \[D]vino,
nella mia \[Em]vita, nelle tue \[Bm]mani.
Ti accoglie\[C]rò dentro di \[D]me,
farò di \[Em]me un'offerta \[C]viva,
un sacri\[Am]ficio \[D] gradito a \[G]Te.
\endchorus
\beginverse
%\chordsoff
^Frutto della nostra ^terra ^del lavoro di ogni ^uomo
^vino delle nostre ^vigne \brk sulla ^mensa dei fratelli ^tuoi.
^Tu che lo prendevi un ^giorno, \brk ^lo bevevi con i ^tuoi
^oggi vieni in questo ^vino \brk e ti ^doni alla vita ^mia.
\endverse
\beginchorus
E sarò \[G]pane, e sarò \[D]vino,
nella mia \[Em]vita, nelle tue \[Bm]mani.
Ti accoglie\[C]rò dentro di \[D]me,
farò di \[Em]me un'offerta \[C]viva,
un sacri\[Am]ficio \[D] gradito a \[G]Te,
\[C] un sacri\[Am]ficio \[D] gradito a \[G]Te.
\endchorus
\endsong

\lettera
%titolo{Genti tutte proclamate}
%autore{Perosi}
%album{}
%tonalita{Do}
%gruppo{}
%momenti{}
%identificatore{genti_tutte_proclamate}
%data_revisione{2011_12_31}
%trascrittore{Francesco Endrici - Manuel Toniato}
\beginsong{Genti tutte proclamate}[by={Perosi}]


\beginverse
\[C]Gen\[G]ti \[C]tut\[Am]te, \[G]pro\[C]cla\[F]ma\[C]te \[F]il \[C]mi\[G]ste\[C]ro \[F]del \[G]Si\[C]gnor,
\[C]del \[G]suo \[C]corpo e \[F]del \[C]suo \[Dm]san\[Am]gue, \[F]che \[G]la \[Am]ver\[Em]gi\[D]ne do\[G]nò,
\[C]e fu sparso in \[G]sa\[C]cri\[F]fi\[C]cio \[F]per \[C]sal\[Dm]var \[C]l'u\[G]mani\[C]tà.
\endverse

\beginverse
\chordsoff
Dato a noi da madre pura, per noi tutti si incarnò.
La feconda sua parola tra le genti seminò;
con amore generoso la sua vita consumò.
\endverse

\beginverse
\chordsoff
Nella notte della Cena coi fratelli si trovò.
Del pasquale sacro rito ogni regola compì
e agli apostoli ammirati come cibo si donò.
\endverse

\beginverse
\chordsoff
La parola del Signore pane e vino trasformò:
pane in Carne, vino in Sangue, \brk in memoria consacrò.
Non i sensi, ma la fede prova questa verità.
\endverse

\beginverse
\chordsoff
Adoriamo il Sacramento che Dio Padre ci donò.
Nuovo patto, nuovo rito nella fede si compì.
Al mistero è fondamento la parola di Gesù.
\endverse

\beginverse
\chordsoff
Gloria al Padre onnipotente, \brk gloria al Figlio Redentor,
lode grande, sommo onore all'eterna Carità.
Gloria immensa, eterno amore alla santa Trinità. 
\chordson\[C]\[Dm]A\[C]men. 
\endverse
\endsong


%titolo{Gerusalemme}
%autore{Ferretti}
%album{}
%tonalita{Sol}
%gruppo{}
%momenti{Ingresso;Quaresima}
%identificatore{gerusalemme}
%data_revisione{2011_12_31}
%trascrittore{Francesco Endrici - Manuel Toniato}
\beginsong{Gerusalemme}[by={Ferretti}]

\beginchorus
\[G]Ge\[Bm]rusa\[C]lem\[D]me, \[G]noi ti \[Bm]rive\[C]dre\[D]mo \[G] 
\[C6]la spe\[Bm]ran\[Em]za ci \[C]palpi\[A7]ta nel \[D]\[A7]cuo\[D]re. \[D7] 
\[Em]La \[Bm]strada è \[C]lun\[D]ga ep\[G]pure arrive\[C6]remo,  \[D] 
Ge\[Bm]rusa\[Em]lem\[C]me, \[Am]casa \[Em]del Si\[C]\[D9]gno\[G]re.   \[C] \[G] 
\endchorus

\beginverse
Quando \[Em]fummo sal\[D]vati dall'E\[G]gitto
cammi\[Em]nam\[Bm]mo e pre\[C]gammo \[Am]con Mo\[D]sè;
quando era\[Em]vamo sc\[D]hiavi in Babi\[G]lonia
noi piange\[Em]va\[Bm]mo e \[C]pensa\[Am]vamo a \[D]te. \[D7] 
\endverse

\beginverse
\chordsoff
Senza fermarsi in sterile rimpianto
i tuoi figli camminano nel mondo;
lungo le strade libere del canto
vengono avanti in cerca del tuo volto.
\endverse

\beginverse
\chordsoff
Cristo fratello, mostraci il cammino,
tu che sei morto, tu che sei risorto.
Tu che conosci il Padre da vicino,
Cristo fratello che ci vivi accanto.
\endverse
\endsong

%titolo{Gesù, Parola di Dio}
%autore{Comi}
%album{}
%tonalita{Sol}
%gruppo{}
%momenti{}
%identificatore{gesu_parola_dio}
%data_revisione{2011_12_31}
%trascrittore{Francesco Endrici - Manuel Toniato}
\beginsong{Gesù, Parola di Dio}[by={Comi}]


%\beginverse* \itshape
%\[G]No, non mormo\[D]rare, \[Em]popolo di \[B7]Dio,
%\[C]il \[D]Si\[G]gno\[Em]re \[C]è  \[A]con \[D]te per\[D7]ché\dots
%\endverse

\beginchorus
Ge\[F]sù pur es\[C]sendo \[Dm]Dio, si \[F]è \[C]fatto \[Dm]Uomo
per \[B&]esser Pa\[C]rola di \[F]Di\[Dm]o, \brk Pa\[B&]rola di \[Gm]Dio per \[C]noi. \[C7] 
\endchorus

\beginverse
\[F]Per ogni uomo che ha \[C]fame, \brk \[Dm]per ogni uomo che ha \[Am]sete,
\[B&]è la Pa\[C]rola di \[F]{Di}\[Dm]o che \[B&]oggi \[G7]lo sazie\[C]rà.
\[F]Alle\[C]luia, \[Dm]allelu\[Am]ia, \[B&]al\[C]le\[F]lu\[Dm]ia, \[B&]al\[G]lelu\[C]ia. \[D7]  
\endverse

\beginverse
\chordsoff
Per ogni uomo che piange, \brk per ogni uomo che soffre
è la Parola di Dio che oggi lo consolerà. 
Alleluia, alleluia, alleluia, alleluia.
\endverse

\beginverse
\chordsoff
Per ogni uomo che è schiavo, \brk per ogni uomo oppresso
è la Parola di Dio che oggi lo libererà.
Alleluia, alleluia, alleluia, alleluia.
\endverse

\beginverse
\chordsoff
Per ogni uomo che è cieco,
per ogni uomo che è sordo,
è la Parola di Dio che oggi lo guarirà. Rit.
Alleluia, alleluia, alleluia, alleluia.
\endverse
\endsong



%titolo{Gesù verrà}
%autore{Bruno}
%album{Io ti seguirò}
%tonalita{Do}
%gruppo{}
%momenti{Congedo}
%identificatore{gesu_verra}
%data_revisione{2011_12_31}
%trascrittore{Francesco Endrici}
\beginsong{Gesù verrà}[by={Bruno}]
\ifchorded
\beginverse*
\vspace*{-0.8\versesep}
{\nolyrics \[C]\[F]\[G]\[C]\[C]\[F]\[C]\[G]\[C]}
\vspace*{-\versesep}
\endverse
\fi
\beginverse*
\memorize
Grande \[G]gioia ci sa\[Am]rà, \[G]\[F7+]
la spe\[C2]ranza rivi\[Dm7]vrà, \[Dm7]\[C]
la tua \[G]vita cambie\[Am]rà, \[G]\[F7+]
devi \[C2]crederlo per\[Dm7]ché, \[Dm7]
Gesù ver\[C4]rà. \[C4]\[C4]\[C4]\[C]
\endverse
\beginverse*
Il tuo de^serto spari^rà, ^^
nuove ^vie Lui ti apri^rà, ^^
la tri^stezza passe^rà, ^^
e la ^notte più non ^torne^rà, Gesù ver\[C]rà. \[F]\[G]\[C]
\endverse
\beginchorus
Gesù ver\[C]rà, \[G] la sua \[F]gloria appari\[G]rà,
Gesù ver\[C]rà, \[G] come il \[F]sole splende\[G]rà,
Gesù ver\[C]rà \[F7+] e con \[E]noi Lui reste\[Am]rà,
come una \[F7+]stella Lui sa\[C]rà
ed il tuo \[Dm7]cuore oriente\[G]rà,
Gesù ver\[C]rà. \[F]\[G]\[C]\rep{2}
Gesù ver\[C]rà. \[F]\[C]\[G]\[C]
\endchorus
\beginverse*
La cit^tà di Dio ver^rà, ^^
per gli e^letti si apri^rà, ^^
la giu^stizia regne^rà ^^
e la ^pace torne^rà, ^
Gesù ver^rà. ^^^^
\endverse
\beginverse*
Ogni ^uomo adore^rà ^^
Gesù ^Cristo che ver^rà ^^
e lo ^Spirito sa^rà ^^
l'acqua ^viva che ci ^riempi^rà,
Gesù ver\[C]rà. \[F]\[G]\[C]
\endverse
\beginchorus
Gesù ver\[C]rà, \[G] la sua \[F]gloria appari\[G]rà,
Gesù ver\[C]rà, \[G] come il \[F]sole splende\[G]rà,
Gesù ver\[C]rà \[F7+] e con \[E]noi Lui reste\[Am]rà,
come una \[F7+]stella Lui sa\[C]rà
ed il tuo \[Dm7]cuore oriente\[G]rà,
Gesù ver\[C]rà. \[F]\[G]\[C]\rep{2}
Gesù ver\[C]rà. \[F]\[C]\[G]\[C]
\endchorus
\beginverse*
\[C]Vie\[F]ni, \[G]vie\[C]ni, \[C]vie\[F]ni \[C]Ge\[G]sù,
\[C]vie\[F]ni, \[G]vie\[C]ni, \[C]vie\[F]ni \[C]Ge\[G]sù,
\[C]vie\[F]ni, \[G]vie\[C]ni, \[C]vie\[F]ni \[C]Ge\[G]sù,
\[C]vie\[F]ni, \[G]vie\[C]ni, \[C]vie\[F]ni \[Dm7]Ge\[G]sù. \[A]
\endverse
\beginchorus
\transpose{2}
Gesù ver\[C]rà, \[G] la sua \[F]gloria appari\[G]rà,
Gesù ver\[C]rà, \[G] come il \[F]sole splende\[G]rà,
Gesù ver\[C]rà \[F7+] e con \[E]noi Lui reste\[Am]rà,
come una \[F7+]stella Lui sa\[C]rà
ed il tuo \[Dm7]cuore oriente\[G]rà,
Gesù ver\[C]rà. \[F]\[G]\[C]\rep{2}
Gesù ver\[C]rà. \[F]\[G]\[C]
\endchorus
\beginverse*
\[D]Vie\[G]ni, \[A]vie\[D]ni.
\endverse
\endsong

%titolo{Gioia che invade l'anima}
%autore{Ricci}
%album{È l'incontro della vita}
%tonalita{La}
%gruppo{}
%momenti{Congedo}
%identificatore{gioia_che_invade_l_anima}
%data_revisione{2011_12_31}
%trascrittore{Francesco Endrici}
\beginsong{Gioia che invade l'anima}[by={Ricci}]
\ifchorded
\beginverse*
\vspace*{-0.8\versesep}
{\nolyrics \[A]\[D]\[A]\[D]\[A]\[D]\[E4]\[E]}
\vspace*{-\versesep}
\endverse
\fi
\beginverse
\memorize
\[A]Gioia che invade l'\[D]anima e canta,
\[A]gioia di avere \[D]Te
\[A]resurrezione e \[D]vita infinita, \[E]vita dell'uni\[D]tà.
\[A]La porteremo al \[D]mondo che attende,
\[A]la porteremo \[D]là
\[A]dove si sta spe\[D]gnendo la vita, \[E]vita s'accende\[D]rà.
\endverse
\beginverse
Per^ché la tua casa è an^cora più grande,
^grande come sai ^tu,
\[F#m]grande come la \[D]terra nell'uni\[F#m]verso \brk che vive in \[C#m]Te.
\[D]Continueremo il canto \[A]delle tue \[E]lodi,
\[D]noi con la nostra \[A]vita, con \[E]Te. 
\endverse
\beginverse
\[A]Ed ora, \[D]via! A por\[A]tare l'a\[D]more nel \[A]mondo,
\[D]carità nelle \[E]case, nei \[D]campi, nella cit\[A]tà.
\[D]Liberi a por\[A]tare l'a\[D]more nel \[A]mondo,
\[D]verità nelle \[E]scuole, in u\ch{D}{f}{fi}{ffi}cio, dove sa\[A]rà.
\[D]E sa\[A]rà \[D]vita nuova. \[F#m]Fuori il \[D]mondo \[A]chiama \[E]
anche noi con il \[D]canto \[A]delle tue \[E]lodi,
\[D]nella \[A]vita con \[E]Te.
\endverse
\endsong


%titolo{Giovane donna}
%autore{Scaglianti, Bancolini}
%album{Andiamo a Betlemme}
%tonalita{Re}
%gruppo{}
%momenti{Maria}
%identificatore{giovane_donna}
%data_revisione{2011_12_31}
%trascrittore{Francesco Endrici}
\beginsong{Giovane donna}[by={Scaglianti, Bancolini}]
\beginverse
\[D]Giovane \[Em]donna, at\[G]tesa dell'umani\[D]tà,
un desi\[Em]derio d'a\[G]more e pura liber\[D]tà.
Il Dio lon\[F#m]tano è \[G]qui, vicino a \[A]te,
\[D]voce e si\[F#m]lenzio, an\[G]nuncio di novi\[A]tà. \[A4/3]
\endverse
\beginchorus
\[D]\[F#7]A\[Bm]ve, Ma\[G]\[Gm]ri\[D]a. \brk \[D]\[F#7]A\[Bm]ve, Ma\[G]\[Gm]ri\[D]a.
\endchorus
\beginverse
%\chordsoff
^Dio t'ha pre^scelta qual ^madre piena di bel^lezza, 
ed il suo a^more ti av^volgerà con la sua ^ombra.
Grembo per ^Dio ve^nuto sulla ^terra,
^tu sarai ^madre di un ^uomo nuo^vo. ^
\endverse
\beginverse
%\chordsoff
^Ecco l'an^cella che ^vive della tua Pa^rola
libero il ^cuore per^ché l'amore trovi ^casa.
Ora l'at^tesa è ^densa di pre^ghiera,
^e l'uomo ^nuovo è ^qui, in mezzo a ^noi. ^
\endverse
\endsong


%titolo{Gloria a Cristo}
%autore{Jef}
%album{}
%tonalita{Sol}
%gruppo{}
%momenti{Acclamazione al Vangelo}
%identificatore{gloria_a_cristo}
%data_revisione{2011_12_31}
%trascrittore{Francesco Endrici - Manuel Toniato}
\beginsong{Gloria a Cristo}[by={}]

\beginverse
\[D]Gloria a Cristo splen\[Em]dore e\[D]terno del \[C]Dio vi\[D]vente!
\endverse

\beginchorus
\[D]Glo\[Em]ria a \[D]te, \[Am]Si\[D]gnor!
\endchorus

\beginverse
\chordsoff
Gloria a Cristo sapienza eterna del Dio vivente!
\endverse

\beginverse
\chordsoff
Gloria a Cristo parola eterna del Dio vivente!
\endverse

\beginverse
\chordsoff
Gloria a Cristo che muore e risorge \brk per tutti i fratelli.
\endverse

\beginverse
\chordsoff
Gloria a Cristo che ascende nei cieli \brk alla destra del Padre.
\endverse
\endsong


%titolo{Gloria a te Cristo Gesù}
%autore{Lecot}
%album{Inno del giubileo}
%tonalita{Sol}
%gruppo{}
%momenti{Ingresso;Congedo}
%identificatore{gloria_a_te_cristo_gesu}
%data_revisione{2011_12_31}
%trascrittore{Francesco Endrici - Manuel Toniato}
\beginsong{Gloria a te Cristo Gesù}[by={Lecot}]
\beginchorus
\[G]Glo\[C]ria a \[G]te, Cristo \[C]Ge\[Bm]sù, \brk \[Em]og\[D]gi e \[G]sempre tu \[Am]regne\[B]rai.
\[D]Gloria a \[C]te! \[G]Presto ver\[D]rai: \brk \[Em]sei spe\[C]ranza \[Am]so\[D]lo \[G]tu.
\endchorus

\beginverse
\[G]Sia lode a \[D]te! \[Em]Cristo Si\[Bm]gnore, \brk \[C]offri per\[F]dono, chiedi giu\[C]stizia:
l'anno di \[Am]grazia \[F]apre le \[Em]porte. \brk \[Am]Solo in \[G]te \[D]pace e uni\[Bm]tà! 
\[G]Amen! Al\[C]le\[G]lu\[D]ia!
\endverse

\beginverse
\chordsoff
Sia lode a te! Prega con noi \brk la benedetta Vergine Madre:
tu l'esaudisci, tu la coroni. \brk Solo in te pace e unità 
Amen! Alleluia!
\endverse

\beginverse
\chordsoff
Sia lode a te! Tutta la Chiesa \brk celebra il Padre con la tua voce
e nello Spirito canta di gioia. \brk Solo in te pace e unità. 
Amen! Alleluia!
\endverse
\endsong

%titolo{Gloria al Signore che salva}
%autore{De Santis, Frigiola}
%album{Voglio vedere il tuo volto}
%tonalita{Fa}
%gruppo{}
%momenti{Ingresso;Salmi}
%identificatore{gloria_al_signore_che_salva}
%data_revisione{2011_12_31}
%trascrittore{Francesco Endrici - Manuel Toniato}
\beginsong{Gloria al Signore che salva}[by={De\ Santis, Frigiola}]

\beginverse*
\itshape \[B&]Gloria al Si\[C]gnore che \[Dm]salva, \[B&]gloria alla \[C]sua po\[F]tenza,
\[Gm]solo da \[Am]lui la vit\[B&]toria, \[Gm]gloria al suo \[Am]nome.
\endverse

\beginchorus
\[B&]Gloria al Si\[C]gnore che \[Dm]salva, \brk \[B&]gloria alla \[C]sua po\[F]tenza,
\[Gm]solo da \[Am]lui la vit\[B&]toria, \brk \[Gm]gloria al suo \[Am]nome per \[Dm]sempre.
\endchorus

\beginverse
Ha scon\[B&]fitto le nazioni, dis\[C]trutto ogni fortezza,
dal suo \[B&]trono di gloria tri\[Gm]onfa la giu\[A7]stizia.
\endverse

\beginverse
\chordsoff
\chordsoff
Ha disperso i suoi nemici, difeso chi lo teme,
la sua destra potente è scudo di salvezza.
\endverse

\beginverse
\chordsoff
Ha liberato i giusti dal laccio della morte,
con il suo braccio santo ha fatto meraviglie
\endverse
\endsong

%titolo{Gloria, gloria ao Pai Criador}
%autore{}
%album{}
%tonalita{Re}
%gruppo{}
%momenti{Lode}
%identificatore{gloria_ao_pai_criador}
%data_revisione{2011_12_31}
%trascrittore{Francesco Endrici}
\beginsong{Gloria, gloria ao Pai Criador}
\ifchorded
\beginverse*
\vspace*{-0.8\versesep}
{\nolyrics \[D]\[G]\[D] \[D]\[G]\[D]}
\vspace*{-\versesep}
\endverse
\fi
\beginchorus
\[D]\[G]Glo\[D]ria, \[D]\[A]glo\[D]ria ao \[G]Pai Cria\[Gm]dor,
ao \[D]Filho Reden\[Bm]tor e ao E\[Em7]spirito, \[A]glo\[D]ria.\rep{2}
\endchorus
\beginverse*
Ao \[D]Pai Cria\[Gm]dor do \[D]mundo;
ao \[G]Filho, Reden\[A]tor dos \[D]homens;
ao E\[G]spirito \[Gm]que a\[D]nima a sua i\[Bm]greja,
ao E\[Em7]spirito \[A]glo\[D]ria.
\endverse
\beginchorus
\[D]\[G]Glo\[D]ria, \[D]\[A]glo\[D]ria ao \[G]Pai Cria\[Gm]dor,
ao \[D]Filho Reden\[Bm]tor e ao E\[Em7]spirito, \[A]glo\[D]ria.\rep{2}
\[D]\[G]Glo\[D]ria, \[D]\[A]glo\[D]ria ao \[G]Pai Cria\[Gm]dor,
ao \[D]Filho Reden\[B]tor e ao E\[Em7]spirito, \[A]glo\[D]ria.
\endchorus
\endsong

%titolo{Grande è la tua bontà}
%autore{}
%album{}
%tonalita{Re}
%gruppo{}
%momenti{Comunione;Salmi}
%identificatore{grande_e_la_tua_bonta}
%data_revisione{2011_12_31}
%trascrittore{Francesco Endrici - Manuel Toniato}
\beginsong{Grande è la tua bontà}

\beginverse
\[D]Grande è la tua bon\[Em]tà, \[G]che riservi per chi ti \[D]teme.
\[F#m]Colmi della tua \[Bm]grazia \[Em]l'uomo che in te si ri\[A]fugia.
\[D]Solo tu, mio \[Em]Dio, \[G]sei la rupe che mi ac\[D]coglie,
\[F#m]Tu sei la mia for\[Bm]tezza, \[Em]confido in te mio Si\[A]gnor.
\endverse

\beginchorus
\[G]Tu  \[A]di\[D]rigi ogni mio \[A]passo, 
sei mia \[Em]roccia e mio ba\[D]luar\[A]do
\[G]E \[A]dal \[D]laccio che mi hanno \[A]teso 
mia di\[Em]fesa, tu \[G]mi \[A]scioglie\[D]rai.
\endchorus

\beginverse
\chordsoff
Porgi a me l'orecchio, vieni presto a liberarmi.
Per la tua giustizia, non privarmi del tuo volto.
In te io mi rifugio, mi affido alle tue mani.
Solo tu mi proteggi, ho fede in te mio Signor.
\endverse
\endsong

%titolo{Grandi cose}
%autore{Gen Rosso}
%album{Se siamo uniti}
%tonalita{Do}
%gruppo{}
%momenti{Ingresso}
%identificatore{grandi_cose}
%data_revisione{2011_12_31}
%trascrittore{Francesco Endrici}
\beginsong{Grandi cose}[by={Gen\ Rosso}]
\beginchorus
\[C]Grandi \[G]cose ha fatto \[Dm]il Si\[Am]gnore per noi,
\[C]ha fatto \[F]germogliare i \[C]fiori tra le \[G4]rocce. \[G]
\[C]Grandi \[G]cose ha fatto \[Dm]il Si\[Am]gnore per noi,
\[C]ci ha ripor\[F]tati liberi \[C]alla nostra \[G4]terra.
Ed \[Am]ora possiamo can\[Dm]tare, possiamo gri\[Em]dare
l'amore che \[F]Dio ha versato su \[G4]noi. \[G]
\endchorus
\beginverse*
\[C]Tu che \[G]sai strap\[Dm]pare dalla \[Am]morte, \[C]
hai solle\[F]vato il nostro \[C]viso dalla \[G]polvere.
\[C]Tu che \[G]hai sen\[Dm]tito il nostro \[Am]pianto, \[C]
nel nostro \[F]cuore hai messo un \[C]seme di \[G]felicità.
\endverse
\endsong

%titolo{Grazie Signore}
%autore{}
%album{}
%tonalita{Re}
%gruppo{}
%momenti{Ritornelli}
%identificatore{grazie_signore}
%data_revisione{2012_01_11}
%trascrittore{Francesco Endrici}
\beginsong{Grazie Signore}
\beginverse*
\[D]Grazie Si\[G]gno\[A]re, rendiamo \[D]grazie
a Te che \[G]regni nei \[A]secoli e\[D]terni.
\endverse
\endsong
%titolo{Guarda questa offerta}
%autore{Gen Rosso}
%album{}
%tonalita{Re-}
%gruppo{}
%momenti{Offertorio}
%identificatore{guarda_questa_offerta}
%data_revisione{2011_12_31}
%trascrittore{Francesco Endrici - Manuel Toniato}
\beginsong{Guarda questa offerta}[by={Gen\ Rosso}]
\beginverse
\[Dm]Guarda questa offerta \[C]guarda noi Si\[Dm]gnor,
\[B&]tutto noi t'o\ch{Gm}{f}{f}{ff}friamo per unirci a \[A]te.
\endverse

\beginchorus
\[Dm]Nella tua \[Gm]mensa \[C]la nostra \[F]mensa, 
\[B&]nella tua \[C]vita \[C7]la nostra \[F]vita.  \rep{2}
\endchorus

\beginverse
\chordsoff
Che possiamo offrirti nostro Creator?
Ecco il nostro niente prendilo Signor!
\endverse
\endsong

%titolo{Gustate e vedete}
%autore{Frisina}
%album{Pane di vita nuova}
%tonalita{Re}
%gruppo{}
%momenti{Comunione;Salmi}
%identificatore{gustate_e_vedete}
%data_revisione{2011_12_31}
%trascrittore{Francesco Endrici - Manuel Toniato}
\beginsong{Gustate e vedete}[by={Frisina}]

\beginchorus
Gus\[D]tate e ve\[Bm]dete come è \[G]buono il Si\[A]gnore,
Be\[D]ato l'uomo \[F#m]che trova il \[G]suo rifugio in \[A]Lui.
Te\[D]mete il Si\[G]gnore, suoi \[D]san\[A]ti,
Nulla \[Bm]manca a co\[G]loro che lo \[D]te\[G]mo\[D]no.
\endchorus

\beginverse
\[D] Benedi\[G]rò il Si\[D]gnore in ogni \[G]tem\[A]po, 
sulla \[Bm]mia \[Em]bocca la sua \[Bm]lo\[A]de.
Io mi \[D]glorio nel Si\[F#m]gnore, 
a\[Bm]scoltino gli \[G]umili e \[D]si ral\[G]legri\[Em]no.\[A] 
\endverse

\beginverse
\chordsoff
Celebrate con me il Signore, 
esaltiamo insieme il suo nome.
Ho cercato il Signore 
e m'ha risposto, m'ha liberato.
\endverse

\beginverse
\chordsoff
Guardate a lui e sarete raggianti, 
non saranno confusi i vostri volti.
Il Signore ascolta il povero, 
egli lo libera da ogni angoscia.
\endverse
\endsong

\lettera
%titolo{Ho chiesto a Lui}
%autore{Scaglianti, Stookey}
%album{Cammina con noi}
%tonalita{La}
%gruppo{}
%momenti{Congedo}
%identificatore{ho_chiesto_a_lui}
%data_revisione{2011_12_31}
%trascrittore{Francesco Endrici}
\beginsong{Ho chiesto a Lui}[by={Scaglianti, Stookey}]
\ifchorded
\beginverse*
\vspace*{-0.8\versesep}
{\nolyrics \[A]\[D] \rep{8}}
\vspace*{-\versesep}
\endverse
\fi
\beginverse
\memorize
Ho chiesto a \[A]Lui, \[D]\[A] a chi tutto \[A]sa, \[D]\[A]
al Si\[C#m]gnor, di pensare a \[Bm]me \[E7]
di pensare a \[A]me, \[D]\[A] di condurmi \[A]là, \[D]\[A]
alla terra \[F#m]mia che ha promesso \[B7]lui:
il corpo \[A]mio solleve\[C#m]rà \[E7] il suo a\[A]mor. \[D]\[A]
\endverse
\beginverse
\chordsoff
Stasera ^poi ^^ io ti preghe^rò, ^^
o mio Si^gnor, di venire a ^me, ^
il tuo a^mor  ^^ mi console^rà ^^
il tuo ^giogo non mi pese^rà
io sali^rò la scala ^che ^ mi porta a ^te. ^^
\endverse
\beginverse
\chordsoff
Quel dì ver^rà, ^^ mi chiame^rai, ^^
e il nome ^mio pronunce^rai. ^
Allora il ^sol ^^ più non splende^rà ^^
il cielo ^cupo esplode^rà
la tromba an^cor risuone^rà: ^ sei Tu Si^gnor. ^^
\endverse
\endsong

\lettera
%titolo{I cieli narrano}
%autore{Frisina}
%album{Benedici il Signore}
%tonalita{Re}
%gruppo{}
%momenti{Ingresso;Salmi}
%identificatore{i_cieli_narrano}
%data_revisione{2011_12_31}
%trascrittore{Francesco Endrici - Manuel Toniato}
\beginsong{I cieli narrano}[by={Frisina}]

\beginchorus
I \[D]cieli \[G]narrano la \[D]gloria di \[A]Dio
e il \[Bm]firma\[G]mento annunzia \[D]l'opera \[A]sua.
Al\[Bm]lelu\[Em]ia al\[A]lelu\[D]ia al\[Bm]lelu\[G]ia alle\[Em7]\[A]lu\[D]ia.
\endchorus

\beginverse
Il \[D]giorno al \[G]giorno ne a\ch{D}{f}{f}{ff}ida il mes\[A]saggio
la \[Bm]notte alla \[G]notte ne tras\[D]mette no\[A]tizia
non \[Bm]è un lin\[G]guaggio non \[A]sono pa\[D]role 
di \[Bm]cui non si \[E]oda il \[A]suo\[A7]no.
\endverse

\beginverse
\chordsoff
Là pose una tenda per il sole che sorge
è come uno sposo dalla stanza nuziale
esulta come un prode che corre 
con gioia la sua strada.
\endverse

\beginverse
\chordsoff
Lui sorge dall'ultimo estremo del cielo
e la sua corsa l'altro estremo raggiunge.
Nessuna delle creature potrà 
mai sottrarsi al suo calore. 
\endverse

\beginverse
\chordsoff
La legge di Dio rinfranca l'anima
la testimonianza del Signore è verace.
Gioisce il cuore ai suoi giusti precetti 
che danno la luce agli occhi.
\endverse
\endsong


%titolo{I frutti che ci dai}
%autore{Tranchida}
%album{Resta con noi Gesù}
%tonalita{Mi}
%gruppo{}
%momenti{Offertorio}
%identificatore{i_frutti_che_ci_dai}
%data_revisione{2011_12_31}
%trascrittore{Francesco Endrici}
\beginsong{I frutti che ci dai}[by={Tranchida}]
\ifchorded
\beginverse*
\vspace*{-0.8\versesep}
{\nolyrics \[E]\[C#m]\[A]\[B]}
\vspace*{-\versesep}
\endverse
\fi
\beginchorus
Bene\[E]detto sei Tu Signore,
per il \[A]pane e per il \[E]vino, i \[F#m]frutti che ci \[B]dai.
Bene\[E]detto sei Tu Signore,
per il \[A]pane e per il \[E]vino,
che in \[F#m]Corpo e Sangue \[B]Tuo trasforme\[E]rai.
\endchorus
\beginverse
\memorize
\[E]Dalla buona \[A]terra, dall'\[B]acqua irri\[E]gata,
\[A]nascerà la \[F#m]spiga che il \[A]grano ci da\[B]rà.
\[E]Dalla grande \[A]forza dell'\[B]uomo che la\[E]vora,
il \[F#m]grano in pane \[E]buono \[B]si trasforme\[E]rà. \[B]
\endverse
\beginverse
\chordsoff
^Dalla buona ^terra, dal ^sole illumi^nata,
^nascerà la ^vite che l'^uva ci da^rà.
^Dalla grande ^forza dell'^uomo che la^vora,
l'^uva in vino ^buono ^si trasforme^rà. ^
\endverse
\beginchorus
Bene\[E]detto sei Tu Signore,
per il \[A]pane e per il \[E]vino, i \[F#m]frutti che ci \[B]dai.
Bene\[E]detto sei Tu Signore,
per il \[A]pane e per il \[E]vino,
che in \[F#m]Corpo e Sangue \[B]Tuo trasforme\[E]rai. \[F#m] \[E]
\endchorus
\endsong

%titolo{Il canto dell'amore}
%autore{Russo}
%album{}
%tonalita{Mi}
%gruppo{}
%momenti{}
%identificatore{il_canto_dell_amore}
%data_revisione{2011_12_31}
%trascrittore{Francesco Endrici}
\beginsong{Il canto dell'amore}[by={Russo}]
\beginverse*
Se dovrai at\[E]traversare il de\[C#m7]serto
non te\[A]mere io sarò con \[E]te
se do\[E]vrai camminare nel \[C#m7]fuoco
la sua \[A]fiamma non ti bruce\[E]rà
segui\[B]rai la mia \[A]luce nella \[E]notte \[E]
senti\[F#m]rai la mia \[B]forza nel cam\[C#m]mino \[C#m]
io s\[D]ono il tuo Dio, \[A] il Signo\[E]re. \[C#m7]\[A]\[E]
\endverse
\beginverse*
Sono ^io che ti ho fatto e plas^mato
ti ho chi^amato per no^me
io da ^sempre ti ho cono^sciuto
e ti ho ^dato il mio amo^re
perché ^tu sei pre^zioso ai miei ^occhi ^
vali ^più del più ^grande dei te^sori ^
io sa^rò con te ^ dovunque an^drai. ^^^
\endverse
\ifchorded
\beginverse*
\vspace*{-\versesep}
{\nolyrics \[B]\[A]\[E]\[E]\[D]\[A]\[B]\[B]}
\endverse
\fi
\beginverse*
Non pen^sare alle cose di ^ieri
cose ^nuove fioriscono ^già
apri^rò nel deserto sen^tieri
darò ^acqua nell'aridi^tà
perché ^tu sei pre^zioso ai miei ^occhi ^
vali ^più del più ^grande dei te^sori ^
io sa^rò con te ^ dovunque an^drai ^^^
perché \[B]tu sei pre\[A]zioso ai miei \[E]occhi \[E]
vali \[F#m]più del più \[B]grande dei te\[C#m]sori \[C#m]
io sa\[D]rò con te \[A] dovunque an\[E]drai. \[C#m7]\[A]\[E]
\endverse
\beginverse*
\[E] Io ti sa\[C#m7]rò accanto \[A]sarò con \[E]te 
\[E] per tutto il \[C#m7]tuo viaggio \[A]sarò con \[E]te. 
\[E] Io ti sa\[C#m7]rò accanto \[A]sarò con \[E]te 
\[E] per tutto il \[C#m7]tuo viaggio \[A]sarò con \[E]te. 
\endverse
\endsong

%titolo{Il Dio della festa}
%autore{Naso}
%album{}
%tonalita{Re}
%gruppo{}
%momenti{}
%identificatore{il_dio_della_festa}
%data_revisione{2011_12_31}
%trascrittore{Francesco Endrici - Manuel Toniato}
\beginsong{Il Dio della festa}[by={Naso}]

\beginchorus
\[D]Il mio Dio è il Dio \[Bm]della festa
il Dio \[F#m]della gioia e dell'a\[G6]mor! \[A7] \rep{2}
\[D]Alle\[G]lu\[A]ia, \[D]alle\[G]lu\[A]ia, \[D]alle\[G]lu\[A7]ia!
\endchorus

\beginverse
\[D]Canterò \[Bm]tutta la vita \[G]canterò,
a lui che \[A]salva \[D]canterò,
a lui che \[Bm]ama chi è piccolo,
chi è \[G]povero, chi è \[Em7]solo e chi è \[A7]misero.
\endverse

\beginverse
\chordsoff
Canterò, a lui che vive canterò,
nel nostro cuore canterò,
a lui che chiama noi tutti suoi amici
e per noi dona la vita.
\endverse

\beginverse
\chordsoff
Canterò, al Dio fedele canterò
al mio Signore canterò
Perché è lui la mia forza e il mio canto,
è il mio cielo qui in terra.
\endverse
\endsong


%titolo{Il disegno}
%autore{Marani}
%album{}
%tonalita{La-}
%gruppo{}
%momenti{Comunione}
%identificatore{il_disegno}
%data_revisione{2011_12_31}
%trascrittore{Francesco Endrici}
\beginsong{Il disegno}[by={Marani}]
\beginverse
Nel \[Am]mare del si\[F]lenzio una \[G]voce si al\[C]zò, \[E7]
da una \[Am]notte senza con\[F]fini una \[G]luce bril\[C]lò, \[E7]
dove non \[Am]c'era niente quel \[E7]giorno.
\endverse
\beginchorus
A\[Am]vevi scritto \[Dm]già il mio \[G]nome lassù nel \[C]cielo \[E7]
a\[Am]vevi scritto \[Dm]già la mia \[G]vita insieme a \[C]te, \[E7]
avevi \[Am]scritto già di \[E7]me.
\endchorus
\beginverse
%\chordsoff
E ^quando la tua ^mente fece ^splendere le ^stelle ^
e ^quando le tue ^mani model^larono la ^ter^ra, 
dove non c'^era niente quel ^giorno.
\endverse
\beginverse
%\chordsoff
E ^quando hai calco^lato la pro^fondità del ^cie^lo
e ^quando hai colo^rato ogni ^fiore della ^ter^ra,
dove non ^c'era niente quel ^giorno.
\endverse
\beginverse
%\chordsoff
E ^quando hai dise^gnato le ^nubi e le mon^ta^gne,
e ^quando hai dise^gnato il cam^mino di ogni ^uo^mo,
l'avevi ^fatto anche per ^me.
\endverse
\beginchorus
Se \[Am]ieri non sa\[Dm]pevo oggi \[G]ho incontrato \[C]te \[E7]
e \[Am]la mia liber\[Dm]tà è il tuo di\[G]segno su di \[C]me \[E7]
non cerche\[Am]rò più niente per\[E7]ché Tu mi salve\[Am]rai.
\endchorus
\endsong

%titolo{Il mattino di Pasqua}
%autore{Sequeri}
%album{E mi sorprende}
%tonalita{Do}
%gruppo{}
%momenti{Pasqua}
%identificatore{il_mattino_di_pasqua}
%data_revisione{2011_12_31}
%trascrittore{Francesco Endrici - Manuel Toniato}
\beginsong{Il mattino di Pasqua}[by={Sequeri}]

\beginchorus
\[C]Il Signore è ri\[G]sorto: can\[C]tate con \[Am]noi!
Egli ha \[D7]vinto la \[G]morte, al\[D]le\[G4]luia!    \[G] 
\[C]Allelu\[G]ia, al\[C]lelu\[Am]ia, al\[D7]lelu\[G]ia, al\[D]{le}\[G4]luia! \[G] 
\endchorus

\beginverse
\[C]Il mattino di \[G]Pasqua, nel ri\[C]cordo di \[Em]Lui,
siamo an\[Am]date al se\[F]polcro: non \[D7]era più \[G]là.
Senza \[C]nulla spe\[G]rare, con il \[C]cuore so\[Em]speso,
siamo an\[Am]dati al se\[F]polcro: non \[D7]era più \[G]là.
\endverse

\beginverse
\chordsoff
Sulla strada di casa  parlavamo di lui
e l'abbiamo incontrato: ha parlato con noi!
Sulle rive del lago pensavamo a quei giorni
e l'abbiamo incontrato: ha mangiato con noi!
\endverse
\endsong

%titolo{Il pane}
%autore{Pianori}
%album{Agape}
%tonalita{Re}
%gruppo{}
%momenti{Offertorio}
%identificatore{il_pane}
%data_revisione{2011_12_31}
%trascrittore{Francesco Endrici}
\beginsong{Il pane}[by={Pianori}]
\beginchorus
\[D]Dove troveremo \[G]tutto il \[A]pane
\[D]per sfamare \[G]tanta \[A]gente,
\[D]dove troveremo \[G]tutto il \[A]pane
\[D]se non ab\[A7]biamo \[D]niente.
\endchorus
\beginverse
\[D]Io possiedo \[E7]solo cinque \[A7]pani
\[D]io possiedo \[G]solo due \[A7]pesci
\[D]io possiedo un \[G]soldo sol\[A7]tanto
\[D]io \[E7]non pos\[A7]siedo \[D]niente.
\endverse
\beginverse
\chordsoff
Io so suonare la chitarra,
io so dipingere e fare poesie,
io so scrivere e penso molto,
io non so fare niente.
\endverse
\beginverse
\chordsoff
Io sono un tipo molto bello,
io sono molto intelligente,
io sono molto furbo,
io non sono niente. 
\endverse
\beginchorus
\chordsoff
Dio ci ha dato tutto il pane
per sfamare tanta gente,
Dio ci ha dato tutto il pane
anche se non abbiamo niente.
\endchorus
\endsong

%titolo{Il pane del cammino}
%autore{Motta, Sequeri}
%album{Come un desiderio}
%tonalita{Sol}
%gruppo{}
%momenti{Ingresso;Comunione}
%identificatore{il_pane_del_cammino}
%data_revisione{2011_12_31}
%trascrittore{Francesco Endrici}
\beginsong{Il pane del cammino}[by={Motta, Sequeri}]
\beginchorus
\[G]Il tuo \[D]popolo in cam\[Em]mino
\[C]cerca in \[A]Te la \[D]gui\[D4]da.
\[G]Sulla \[D7]strada verso il \[Em]re\[G7]gno
\[C]sei so\[A]stegno col Tuo \[G]Cor\[D]po,
\[Bm]resta \[E]sempre con {\[Am]noi} {\[Cm]o} Si\[G]\[D]gno\[G]re.
\endchorus
\beginverse
È il Tuo \[Gm]pane Ge\[Cm]sù, che ci dà \[Gm]forza
e \[Gm]rende più si\[F]curo il nostro \[Gm]passo.
Se il vigore nel cam\[D7]mino si svi\[E&]lisce \[G7]
la Tua \[Cm]mano dona \[A7]lieta la spe\[D7]ranza.
\endverse
\beginverse
%\chordsoff
È il tuo ^vino Ge^sù, che ci dis^seta
e ^sveglia in noi l'ar^dore di se^guirti.
Se la gioia cede il ^passo alla stan^chezza, ^
la Tua ^voce fa ri^nascere fre^schezza.
\endverse
\beginverse
%\chordsoff
È il Tuo ^corpo, Ge^sù, che ci fa ^Chiesa
fra^telli sulla ^strada della ^vita.
Se il rancore toglie ^luce all'ami^cizia, ^
dal tuo ^cuore nasce ^giovane il per^dono.
\endverse
\beginverse
%\chordsoff
È il Tuo ^Sangue, Ge^sù, il segno e^terno
dell'^unico lin^guaggio dell'a^more.
Se il donarsi come ^Te richiede ^fede ^
nel tuo ^Spirito sfi^diamo l'incer^tezza.
\endverse
\endsong

%titolo{Il Signore è il mio pastore}
%autore{Turoldo, Passoni, De Marzi}
%album{Salmi e cantici}
%tonalita{La}
%gruppo{}
%momenti{Salmi;Comunione}
%identificatore{il_signore_e_il_mio_pastore_turoldo}
%data_revisione{2011_12_31}
%trascrittore{Francesco Endrici}
\beginsong{Il Signore è il mio pastore}[by={Turoldo, Passoni, De\ Marzi}]
\beginverse
Il Si\[A]gnore è il \[D]mio pa\[A]store,
nulla \[F#m]manca ad \[Bm]ogni at\[A]te\[D]sa, \[A]
in ver\[C#m]dissimi \[D]prati mi \[A]\[E7]pa\[A]sce,
mi dis\[D]seta a \[E]placide \[A]acque.
\endverse
\beginverse
%\chordsoff
È il ri^storo dell'^anima ^mia,
in sen^tieri di^ritti mi ^gui^da ^
per a^more del ^santo suo ^^no^me,
dietro ^lui mi ^sento si^curo.
\endverse
\beginverse
%\chordsoff
Pur se an^dassi per ^valle o^scura
non a^vrò a te^mere alcun ^ma^le: ^
perché ^sempre mi ^sei vi^^ci^no,
mi so^stieni col ^tuo vin^castro.
\endverse
\beginverse
%\chordsoff
Quale ^mensa per ^me tu pre^pari
sotto gli ^occhi dei ^miei ne^mi^ci! ^
E di ^olio mi ^ungi il ^^ca^po,
il mio ^calice è ^colmo di eb^brezza!
\endverse
\beginverse
%\chordsoff
Bontà e ^grazia mi ^sono com^pagne
quanto ^dura il ^mio cam^mi^no: ^
io sta^rò nella ^casa di ^^Di^o
lungo ^tutto il mi^grare dei ^giorni.
\endverse
\beginverse
%\chordsoff
Grazie al ^Padre che ci ^ha bene^detti
fin dall'^alba del ^mondo nel ^Cri^sto: ^
nello ^Spirito il ^solo pa^^sto^re
che nei ^cieli ci ^fa cammi^nare.
\endverse
\endsong


%titolo{Il Signore è la luce}
%autore{Giombini}
%album{Salmi per il nostro tempo}
%tonalita{Fa}
%gruppo{}
%momenti{Acclamazioni al Vangelo}
%identificatore{il_signore_e_la_luce}
%data_revisione{2011_12_31}
%trascrittore{Francesco Endrici}
\beginsong{Il Signore è la luce}[by={Giombini}]
\ifchorded
\beginverse*
\vspace*{-0.8\versesep}
{\nolyrics \[F]\[B&]\[E&]\[C4]}
\vspace*{-\versesep}
\endverse
\fi
\beginverse
\memorize
\[C] Il Si\[F]gnore è la \[C]luce che \[B&]vince la \[F]notte. \[F]
\endverse
\beginchorus
Gloria \[Am]glo\[B&]ria! Can\[\vline]\[Gm]tia\[F]\[\vline]mo al Si\[C]gno\[F]re! \[F]
Gloria \[Am]glo\[B&]ria! Can\[\vline]\[Gm]tia\[F]\[\vline]mo al Si\[C]gno\[B&]re! \[F]
\endchorus
\beginverse
\chordsoff
Il Signore è la vita che vince la morte.
\endverse
\beginverse
\chordsoff
Il Signore è il coraggio che vince il terrore.
\endverse
\beginverse
\chordsoff
Il Signore è il sereno che vince la pioggia.
\endverse
\beginverse
\chordsoff
Il Signore è l'amore che vince il peccato.
\endverse
\endsong

%titolo{In cammino}
%autore{Spoladore}
%album{Dacci pace}
%tonalita{La-}
%gruppo{}
%momenti{Comunione}
%identificatore{in_cammino}
%data_revisione{2011_12_31}
%trascrittore{Francesco Endrici - Manuel Toniato}
\beginsong{In cammino}[by={Spoladore}]

\beginverse
Si\[Am]gnore mio \[Dm7]sei la mia \[E7]strada si\[Am]cura
\[Dm7]la mia \[E7]strada \[Am]buona \[Dm]strada di ogni \[G/B]verità. \[F+7] 
\[Dm7]Signore guida il \[G/B+7]servo  \[F7+]tuo \brk e cam\[Dm7]mina in\[G/B]sieme a \[F7+]me
nel pe\[Dm7]ricolo \[G/B]stai con \[E4/7]me. \[E7] 
\endverse

\beginchorus
In cam\[Am]mino noi \[Em]siamo \brk fa' che un \[F]giorno tor\[C]niamo.
Noi tor\[Dm]niamo da \[Am]Te \[B&7+]ricchi di bon\[A4/7]tà. \[A7] 
In cam\[Am]mino noi \[Em]siamo \brk fa' che un \[F]giorno tor\[C]niamo.
Noi tor\[Dm]niamo da \[Am]Te \[Dm]ricchi di bon\[E4/7]tà. \[E7] 
\endchorus

\beginverse
\chordsoff
Signore mio concedimi la tua pace
il dolce tuo buon sorriso, difendimi da ogni male
per vivere la tua Parola proteggimi dalla paura
per vivere nel tuo amore.
\endverse
\endsong

%titolo{In principio}
%autore{Ricci}
%album{Venne nel mondo}
%tonalita{Do}
%gruppo{}
%momenti{Natale}
%identificatore{in_principio_ricci}
%data_revisione{2011_12_31}
%trascrittore{Francesco Endrici}
\beginsong{In principio}[by={Ricci}]
\ifchorded
\beginverse*
\vspace*{-0.8\versesep}
{\nolyrics \[C]\[F]\[G]\[G]\[C]\[F]\[G]\[G]\[C]}
\vspace*{-\versesep}
\endverse
\fi
\beginverse
\memorize
In prin\[F]cipio era il \[G]Verbo,
ed il \[G]Verbo era \[C]presso \[F]Dio \[G]\[G]\[C]
in prin\[F]cipio era il \[G]Verbo,
ed il \[G]Verbo era \[C]Di\[F]o. \[G]
E ogni \[G]cosa cre\[Am]a\[Am]ta \[Em] \brk fu per \[Em]mezzo di \[F]lu\[F]i:
\[D7]nulla \[D7]di ciò che e\[C]si\[E7]ste \brk \[Am]fu fatto \[G]senza \[C]lu\[F]i. \[G]\[G]\[C]
\endverse
\beginverse
E nel ^mondo arri^vava la ^luce ^ve^ra ^^^
luce ^vera che il^lumina ^ogni ^uo^mo. ^
Lui ve^niva nel ^mon^do ^ \brk che fu ^fatto per ^lu^i,
^ma non ^lo rico^nob^be \brk il ^mondo ^ed era ^lu^i. ^^
\endverse
\beginchorus
Ma a \[Am]quanti l'ac\[Am]colsero
\[C]diede il po\[C]tere di \[C]essere \[E7]figli di \[Am]Dio, \[Am]
a \[C]tutti co\[C]loro che \[C]credono \[C]nel nome \[C]\[E7]su\[Am]o.
Per\[Am]ché non son \[F]nati da \[F]sangue o da \[C]carne
o vo\[C]lere di \[E7]\[E]uo\[Am]mo,
\[Am]ma sono \[F]nati da \[F]Dio, \[C]
\[C]ma sono \[E7]nati da \[\vline]\[E]\[Am]\[\vline]Dio. \[Am]\[Am]\[Am]\[G]\[C]
\endchorus
\beginverse
Ed il ^Verbo è di^sceso tra ^noi sulla ^ter^ra ^^
^e si è ^fatto bam^bino come ^uno di ^no^i. ^
Contem^pliamo la ^gloria ^sua \[G] \brk di uni\[G]genito \[F7+]Figlio
\[D9]di Dio \[D9]Padre \[C]glo\[E7]ria \[Am]di grazia e \[G]veri\[C]tà. \[F]\[G]\[A7]
\endverse
\beginverse
\[D]Fra \[G]di \[A]noi, \[A] nes\[G]suno ha ve\[G]duto \[A]Dio, 
solo il \[A]Figlio.
\[D]Fra \[G]di \[A]noi, \[A] nes\[G]suno ha ve\[G]duto \[A]Dio. 
Lui \[Bm]sì, \[Bm]\[F#m]\[F#m]  \[G]ed è qui \[G]fra di \[E7]noi: \[E7] 
il \[D]Figlio che \[F#7]lo rive\[Bm]la sta \[A]qui fra \[A]noi. \[A]\[A]\[A]
\endverse
\textnote{[Sovrapposta alla prossima strofa si ricanta la precedente.]}
\beginverse
\[D]Ed il \[G]Verbo è di\[A]sceso tra \[A]noi sulla \[D]ter\[G]ra \[A]\[A]
\[D]e si è \[G]fatto bam\[A]bino come \[A]uno di \[D]no\[G]i. \[A]
Contem\[A]pliamo la \[Bm]gloria \[Bm]sua \[F#m] di uni\[F#m]genito \[G]Fi\[G]glio
\[E7]di Dio \[E7]Padre \[D]glo\[F#7]ria \[Bm]di grazia e \[\vline]\[A]veri\[D]\[\vline]tà. \[G]
\endverse
\ifchorded
\beginverse*
\vspace*{-\versesep}
{\nolyrics \[G]\[A]\[A]\[D]\[G]\[A]\[A]\[D] \brk \[G]\[A]\[A]\[D]\[G]\[A]\[A]\[D] \[G]\[D]\[D]}
\endverse
\fi
\endsong




%titolo{Incontro a Te}
%autore{Mariano}
%album{Venite a me}
%tonalita{Fa}
%gruppo{}
%momenti{Congedo}
%identificatore{incontro_a_te_mariano}
%data_revisione{2011_12_31}
%trascrittore{Francesco Endrici}
\beginsong{Incontro a Te}[by={Mariano}]
\ifchorded
\beginverse*
\vspace*{-0.8\versesep}
{\nolyrics \[F]\[C]\[B&]\[C] \[F]\[C]\[B&]\[C] \[F]}
\vspace*{-\versesep}
\endverse
\fi
\beginverse
\memorize
Ristorati dal tuo pane, \[C] \brk dissetati dal Tuo vino, \[B&]
rafforzati dalla Tua Pa\[Dm]rola \[C]\[F]
proseguiamo nel domani \[Gm] \brk tutti uniti nel tuo corpo \[B&]
fonte inesau\[E&]ribile di \[C]pace.
\endverse
\beginchorus
Incontro a \[F]Te \[C] noi cammi\[B&]niamo
e \[C]dentro noi cresce\[F]rà \[C] la liber\[B&]tà.
Nel \[C]mondo che amiamo \[B&]porte\[C]remo la spe\[F]ran\[Dm]za
dei \[B&]figli tuoi, \[F]figli del tuo a\[Gm7]more. \brk \[C]\[B&]\[F] \[Gm7] \[F]
\endchorus
\beginverse
\chordsoff
Nella gioia e nel dolore, ^ \brk nel lavoro e nel riposo ^
nella solitudine del ^cuore, ^^
sei compagno del cammino ^ \brk tenerezza immensa e vera ^
mano che accom^pagna tutti ^noi.
\endverse
\endsong

%titolo{Innalzate nei cieli}
%autore{Albisetti, Martorell}
%album{}
%tonalita{Sol}
%gruppo{}
%momenti{Avvento}
%identificatore{innalzate_nei_cieli}
%data_revisione{2011_12_31}
%trascrittore{Francesco Endrici - Manuel Toniato}
\beginsong{Innalzate nei cieli}[by={Albisetti, Martorell}]

\beginverse
\[B7]Innal\[Em]zate \[D]nei \[C]cieli \[D]lo \[Em]sguardo:
la sal\[G]vezza di \[C]Dio è vi\[G]cina.\[B7] 
Risve\[Em]gliate \[D]nel \[C]cuore l'at\[D]te\[G]sa
per ac\[Am]cogliere il \[Bm]Re della \[Em]gloria
\endverse

\beginchorus
\[C]Vieni \[D]Ge\[G]sù, \[Em]vieni \[C]Ge\[Bm]sù! Di\[Em]scendi \[D]dal \[G]cielo,
di\[Em]scendi \[D]dal \[C]\[Am]\[B7]cie\[Em]lo.
\endchorus

\beginverse
\chordsoff
Sorgerà dalla casa di David
il Messia da tutti invocato:
prenderà da una Vergine il corpo
per potenza di Spirito Santo.
\endverse

\beginverse
\chordsoff
Benedetta sei tu o Maria,
che rispondi all'attesa del mondo;
come aurora splendente di grazia
porti al mondo il sole divino.
\endverse
\endsong

%titolo{Innalziamo lo sguardo}
%autore{Buttazzo}
%album{Maranathà}
%tonalita{La}
%gruppo{}
%momenti{Ingresso;Natale}
%identificatore{innalziamo_lo_sguardo}
%data_revisione{2011_12_31}
%trascrittore{Francesco Endrici}
\beginsong{Innalziamo lo sguardo}[by={Buttazzo}]
\ifchorded
\beginverse*
\vspace*{-0.8\versesep}
{\nolyrics \[A]\[E]\[F#m]\[A]\[D]\[E]\[A4]\[A]}
\vspace*{-\versesep}
\endverse
\fi
\beginchorus
\[A]Innal\[E]ziamo lo \[F#m]sguar\[A]do, \brk \[D]rinno\[Bm]viamo l'at\[E4]te\[E]sa.
\[A]Ecco \[E]viene il Si\[F#m]gno\[E]re, \brk \[D]viene \[E]non tarde\[A]rà. \[A]
\endchorus
\beginverse
\memorize
\[A]Brille\[E]rà come \[D]lu\[E]ce \brk \[F#m]la sal\[D]vezza per \[E4]noi: \[E]
\[D]la Pa\[E]rola di \[A]\[(E)]Di\[F#m]\[(A)]o \brk \[D]nasce\[Bm]rà in mezzo a \[E4]noi. \[E]
\endverse
\beginverse
^Questo è ^tempo di ^gio^ia, \brk ^di spe^ranza per ^noi: ^
^il Crea^tore del ^^mon^^do \brk ^regne^rà in mezzo a ^noi. ^
\endverse
\beginverse
^Procla^miamo con ^for^za \brk ^il Van^gelo di ^Dio. ^
^Annun^ciamo con ^^gio^^ia \brk ^la sal^vezza di ^Dio. ^
\endverse
\beginverse
^Percor^riamo i sen^tie^ri \brk ^che ci ^portano a ^Dio. ^
^Nell’a^more ve^^dre^^mo \brk ^la pre^senza di ^Dio. ^
\endverse
\endsong

%titolo{Inno alla Parola}
%autore{Valenti, Farruggio}
%album{}
%tonalita{Do}
%gruppo{}
%momenti{}
%identificatore{inno_alla_parola}
%data_revisione{2011_12_31}
%trascrittore{Francesco Endrici}
\beginsong{Inno alla Parola}[by={Valenti, Farruggio}]
\beginverse
Nei \[C]giorni che non a\[F]vevano tempo,
viveva con Dio nel si\[C]lenzio.
Parola che era la \[F]gloria e l'amore
fiorita in segreto all'im\[C]menso.
In \[G]Lei è la forza del \[Am]mondo, la \[F]vita,
fu \[A7]fatto da Lei lo spazio e il \[Dm]sole;
in\[G]fuse la mente alla \[Am]carne dell'\[F]uomo,
la \[Am]terra per \[B7]casa do\[Em]nò. \[G7]
\endverse
\beginverse
\chordsoff
Nascendo poi nella storia del mondo,
vedemmo tra noi la sua gloria;
nel buio la luce era apparsa in un volto,
l'amore ebbe il nome di un uomo.
Il mondo di tenebra fugge la luce
l'accoglie chi il cuore aprirà,
credendo che quella Parola è la vita
Iddio per Padre avrà.
\endverse
\beginverse
\chordsoff
Vivendo le nostre giornate, ai poveri
annuncia il perdono e il suo Regno;
e come un seme, per crescere grano,
dovrà nella terra morire,
così, per dar vita, fu uomo di croce,
vivente per Dio ritornò;
Signore del cielo, speranza del mondo,
la forza all'uomo donò.
\endverse
\beginverse
\chordsoff
Ci rese le ali per farci salire
gli spazi abitati da Dio.
Ci disse che il mondo,
crescendo nel tempo,
matura il suo corpo di gloria.
L'immenso rimane con noi incarnato,
la luce tramonto non avrà,
la fede proclama il nostro Signore,
il Dio che vive con noi.
\endverse
\endsong

%titolo{Insieme è forte}
%autore{Spoladore}
%album{Come in cielo così in terra}
%tonalita{Do}
%gruppo{}
%momenti{}
%identificatore{insieme_e_forte}
%data_revisione{2011_12_31}
%trascrittore{Francesco Endrici - Manuel Toniato}
\beginsong{Insieme è forte}[by={Spoladore}]

\beginverse
\[C]Noi vogliamo Si\[G/B]gnore che \[Dm]tu ci insegni an\[Am]cora
ad \[F]amare il \[G]mondo in\[Am]tero \[F]e comin\[G]ciare tra \[C]noi.
La vita ci corre in\[G/B]contro è \[Dm]come una grande \[Am]corsa
sei \[F]tu il ma\[G]estro Si\[Am]gnore \brk ci \[F]dai le rego\[G]le del \[C]gioco. \[C7] 
\endverse

\beginchorus
In\[F]sieme è \[C]forte in\[E7]sieme è \[Am]grande
\[Fm]sotto questo \[C]cielo \[Fm]come un grande ab\[C/E]braccio
che fra\[Dm]tel\[G]li ci \[C]fa. \rep{2} 
\endchorus

\beginverse
\chordsoff
Chi cerca la gloria e le cose \brk è lontano dal cuore di Dio.
Chi pensa solo a sé stesso è senza ali per volare,
nel buio della tempesta \brk non si perde chi ha il cuore pulito
chi è umile e sa ringraziare \brk nel mondo è un segno di pace.
\endverse
\endsong

%titolo{Io accolgo te}
%autore{Riondato, Toniato}
%album{}
%tonalita{Re}
%gruppo{}
%momenti{Matrimonio}
%identificatore{io_accolgo_te}
%data_revisione{2011_12_31}
%trascrittore{Francesco Endrici - Manuel Toniato}
\beginsong{Io accolgo te}[by={Riondato, Toniato}]

\ifchorded
\beginverse*
\vspace*{-0.8\versesep}
{\nolyrics \[D] \[A] \[G] \[A] \[D] \[A] \[G] \[A] }
\vspace*{-\versesep}
\endverse
\fi

\beginverse
\[D]   A\[A]desso siamo in \[G]due nel cam\[A]mino della \[D]vita.
Sboc\[A]ciati nell'a\[G]more ad im\[A]magine di \[Em]Dio.
Siamo come \[Bm]angeli con un'ala sola\[G]mente.
Destinati a rima\[A]nere abbracciati per vo\[D]la\[A]re.
\endverse
\ifchorded
\beginverse*
\vspace*{-\versesep}
{\nolyrics \[G] \[A] \[D] \[A] \[G] \[A] }
\endverse
\fi
\beginverse*
\chordsoff
Dio mi ha affidato te per esserti vicino
Per darti ciò di cui hai bisogno nella vita
Ti aiuterò a sperare pregare e perdonare
E ringrazierò colui che ha voluto tutto ciò
\endverse

\beginchorus
\[D] \[A]Io accolgo \[G]te  \[A]nelle mie \[D]mani.
Cu\[A]stode sa\[G]rò  \[A]della tua \[Em]vita
L'amore sa\[Bm]rà la nostra \[G]casa.
Insieme \[A4]noi \[A]due verso \[D]Lui.
\endchorus
\ifchorded
\beginverse*
\vspace*{-\versesep}
{\nolyrics \[A] \[G] \[A] \[D] \[A] \[G] \[A] }
\endverse
\fi

\beginverse
\chordsoff
Siate come musica orchestra di strumenti
Sempre un solo canto ma voci differenti
Siate come l'acqua fresca e trasparente
Oasi d'amore per il tempo che verrà

Siate come alberi dritti verso l'alto
Danzino fra voi i venti del Suo cielo
Che ogni vostro gesto sia carico d'amore
Specchio di una mano che sempre vi unirà
\endverse

\beginchorus
\[D] \[A]Io accolgo \[G]te  \[A]nelle mie \[D]mani.
Cu\[A]stode sa\[G]rò  \[A]della tua \[Em]vita.
L'amore sa\[Bm]rà la nostra \[G]casa.
Insieme \[A4]noi \[A]due verso \[D]Lui.
\endchorus

\beginverse*
\itshape \[C]   Nella \[G]gioia e nel do\[D]lore
\[C]   nella sa\[G]lute e nella \[D]malattia
\[C]  prometto \[G]di esserti fe\[D]dele sempre
tutti i \[C]giorni della \[G]vita \[A]mia. \[B] 
\endverse

\beginchorus
\[E] \[B]Io accolgo \[A]te \[B]nelle mie \[E]mani.
Cu\[B]stode sa\[A]rò \[B]della tua \[F#m]vita.
L'amore \[C#m]sarà   la nostra \[A]casa.
Insieme \[B4]noi \[B]due verso \[E]Lui. \[B] \[A] \[B] 
\endchorus

\beginverse
\[E] Vi \[B]ho amati \[A]sempre il mio a\[B]more ora è in \[E]voi.
Vi ho \[B]scelti perché \[A]siate una \[B]sola cosa in \[F#m]me.
Io sarò con \[C#m]voi ogni giorno della \[A]vita.
Passo dopo \[B4]passo al vostro fianco sulla \[E]via.
\endverse
\endsong


%titolo{Io credo risorgerò}
%autore{Stefani}
%album{}
%tonalita{Re}
%gruppo{}
%momenti{Esequie}
%identificatore{io_credo_risorgero}
%data_revisione{2011_12_31}
%trascrittore{Francesco Endrici - Manuel Toniato}
\beginsong{Io credo risorgerò}[by={Stefani}]
\beginchorus
\[F#]Io \[G]credo \[A] risorge\[B7]rò, \brk \[Em]questo mio \[F#]corpo ved\[G]rà il \[A]Salva\[Bm]tore!
\endchorus

\beginverse
\[D]Prima che io na\[Bm]scessi, mio \[Em]Dio tu mi co\[D]nosci:
ri\[F#m]cordati, Si\[F#m]gnore, che l'\[G]uomo è come l'\[F#m]erba,\[D] 
come il \[Em]fiore del \[Bm]cam\[F#]po.
\endverse

\beginverse
\chordsoff
Ora è nelle tue mani quest'anima che mi hai data:
accoglila, Signore, da sempre tu l'hai amata,
è preziosa ai tuoi occhi.
\endverse
\endsong

%titolo{Io non sono degno}
%autore{Chieffo}
%album{Agape}
%tonalita{Re-}
%gruppo{}
%momenti{Penitenza}
%identificatore{io_non_sono_degno}
%data_revisione{2011_12_31}
%trascrittore{Francesco Endrici}
\beginsong{Io non sono degno}[by={Chieffo}]
\beginchorus
\[Dm]Io non sono degno di \[Gm]ciò che fai per \[Dm]me,
tu che ami tanto \[B&7]uno come \[A7]me.
\[Dm]Vedi non ho nulla \[Gm]da donare a \[Dm]te,
ma se tu lo vuoi, \[A7]prendi \[Dm]me.
\endchorus
\beginverse
\[Dm]Sono come la polvere al\[Am]zata dal \[Dm]vento,
\[Dm]sono come la pioggia pio\[Am]vuta dal \[Dm]cielo.
\[F]Sono come una canna spez\[C]zata dall'ura\[F]gano
\[Dm]se Tu, Signore, non \[A7]sei con \[Dm]me.
\endverse
\beginverse
\chordsoff
Contro i miei nemici tu mi fai forte,
io non temo nulla, nemmeno la morte.
Sento che sei vicino e che mi aiuterai,
ma non son degno di quello che mi dai.
\endverse
\endsong


%titolo{Io sarò con te}
%autore{Damonte}
%album{Ti ho incontrato}
%tonalita{Re}
%gruppo{}
%momenti{}
%identificatore{io_saro_con_te_damonte}
%data_revisione{2011_12_31}
%trascrittore{Francesco Endrici}
\beginsong{Io sarò con te}[by={Damonte}]
\beginchorus
\[D]Io sarò con \[Bm]te sulle \[Em]strade della \[A]vita,
\[D]io sarò con \[Bm]te anche \[Em]quando non lo \[A]sai.
\[D]Io sarò con \[F#m]te custo\[G]dendoti per \[A]sempre
\[G]nella fedel\[A]tà è il mio a\[A7]mo\[D]re.
\chordsoff
Tu sarai con me sulle strade della vita,
tu sarai con me anche quando non lo so.
Tu sarai con me custodendomi per sempre
nella fedeltà è il tuo amore.
\endchorus
\beginverse
Se \[D]forte e corag\[F#m]gioso tu sa\[Bm]rai
sce\[G]gliendo la \[A]via della \[D]vita,
ascol\[G]tando la \[A]mia pa\[D]ro\[Bm]la
custo\[G]dendola \[A]nel tuo \[D]cuo\[A]re.
\endverse
\beginverse
\chordsoff
Se forte e coraggioso tu sarai
annunciando ad ogni uomo la salvezza,
cantando la speranza che non muore,
camminerò con te dovunque andrai.
\endverse
\endsong

%titolo{Io scelgo te}
%autore{La Porta}
%album{Io scelgo te}
%tonalita{Mi}
%gruppo{}
%momenti{Adorazione}
%identificatore{io_scelgo_te}
%data_revisione{2011_12_31}
%trascrittore{Francesco Endrici - Manuel Toniato}
\beginsong{Io scelgo te}[by={La\ Porta}]

\ifchorded
\beginverse*
\vspace*{-0.8\versesep}
{\nolyrics \[E] \[A/E] \[E] }
\vspace*{-\versesep}
\endverse
\fi

\beginverse
\[E]Io scelgo te come Si\[A/E]gnore della mia vita, 
scelgo \[E]te per\[A6/E]ché il mondo a\[A6/B]more non mi \[E]dà.
\[B/D#]Io voglio amare \[A/C#]te, 
\[E/B] guar\[A]dare il tuo \[F#m7]volto e \[B]vivere per \[A6/B]te.
\endverse

\beginchorus
\[E]La mia \[B/D#]voce sale a \[A/C#]te,
as\[B]coltami Si\[E]gnor, per \[B]sempre in te vi\[A]vrò,
io \[G#4]ti bene\[G#]di\[C#m]rò, ac\[F#]cogli la mia \[A/B]lo\[B]de.
\[E]La mia \[B/D#]voce sale a \[A/C#]te,
a\[B]scoltami Si\[E]gnor, per \[B]sempre in \[B&7]te vi\[A7+]vrò,
io \[G#4]ti bene\[G#]di\[C#m]rò, per\[Am/C]ché tu sei il mio \[E/B]re,
ac\[F#/A#]cogli la mia \[A/B]lo\[B7]de.
\endchorus

\ifchorded
\beginverse*
\vspace*{-\versesep}
{\nolyrics \[E] \[A/E] \[E] }
\endverse
\fi

\beginverse
\chordsoff
Io cerco te, tu sei il Signore della mia vita,
cerco te perché il mondo amore non mi dà.
Io voglio amare te, 
guardare il tuoi volto e vivere per te.
\endverse

\beginchorus
\[E]La mia \[B/D#]voce sale a \[A/C#]te,
as\[B]coltami Si\[E]gnor, per \[B]sempre in te vi\[A]vrò,
io \[G#4]ti bene\[G#]di\[C#m]rò, ac\[F#]cogli la mia \[A/B]lo\[B]de.
\[E]La mia \[B/D#]voce sale a \[A/C#]te,
as\[B]coltami Si\[E]gnor, per \[B]sempre in \[B&7]te vi\[A7+]vrò,
io \[G#4]ti bene\[G#]di\[C#m]rò, per\[Am/C]ché tu sei il mio \[E/B]re,
ac\[F#/A#]cogli la mia \[B4]lo\[B]de \[C/D]mio \[D]{Si}\[G]gnor. \[D/F#] 
\[C/E] \[D] \[G] \[D] \[C] \[B4] \[B] \[Em] \[A] \[C/D] \[D] 
\[G]La mia \[D/F#]voce sale a \[C/E]te, 
as\[D]coltami Si\[G]gnor, per \[D]sempre in te \[C#7]vi\[C7+]vrò,
io \[B4]ti bene\[B]di\[Em]rò per\[Cm6/E&]ché tu sei il mio \[G/D]re,
ac\[A/C#]cogli la mia \[D4]lo\[D]\[G]de.  \[Am/G]  \[D/G]  \[G]  
\endchorus
\endsong

%titolo{Io ti offro}
%autore{Cloninger, Moen, Cimini}
%album{Vieni alla mensa}
%tonalita{Re}
%gruppo{}
%momenti{Comunione}
%identificatore{io_ti_offro_moen}
%data_revisione{2011_12_31}
%trascrittore{Francesco Endrici}
\beginsong{Io ti offro}[by={Cloninger, Moen, Cimini}]
\beginverse
\[D]Quello che \[D]sono, \[G]quello che \[G]ho
\[A]io lo de\[A]pongo ai Tuoi \[G7+]piedi \[D]Si\[A]gnor
\[Bm7]Ogni mio er\[Bm7]rore, \[G]io lascio a \[G]\[D]Te
Le \[Em7]gioie e i do\[D]lori io \[G]dono a \[Em7]Te. \[A]
\endverse
\beginchorus
\[D]Io Ti \[D]offro me \[Bm7]stesso e Tu, \[Bm7]Dio della \[Em9]vita \[D]mia
\[G]cambia questo \[G7+]cuo\[A]re.
\[D]Io Ti \[D]offro i miei \[Bm7]giorni e Tu \[Bm7]fonte di \[Em9]Santi\[D]tà 
\[G]ne farai la \[G7+]lo\[F#7]de a \[Bm]Te \[F#m7]  
\[Em7]nell'offerta \[G7+]di \[Em]Ge\[D]sù \[D]
\endchorus
\beginverse
^Quello che ^fui, ^che mai sa^rò
^ogni mio s^ogno e pro^getto ^Si^gnor 
^tutto in ^Te ^io ripor^^rò 
e un ^nuovo cam^mino con ^Te scopri^rò. ^
\endverse
\beginchorus
\[D]Io Ti \[D]offro\dots
\[Em7]nell'offerta \[G7+]di \[Em]Ge\[Gm9]sù
\endchorus
\beginverse
\[B&7+]Cosa Ti \[C]diamo che \[Am11]non sia Tuo \[C]do\[Dm]no
\[Gm9]e cosa ab\[C]biamo che \[Am7]non sia già \[Dm]Tuo
\[Gm9]ogni crea\[C]tura vi\[Am7]vendo \[C]can\[D]ti
le \[Gm7]tue mera\[F]viglie Si\[G]gnor.
\endverse
\beginchorus
\chordsoff 
Rit. 
\endchorus
\ifchorded
\beginverse*
\vspace*{-\versesep}
Finale: \[Em7]Offro la mia \[G7+]vi\[Em]ta a \[D]Te.
\endverse
\fi
\endsong

\lettera
%titolo{Jesus Christ you are my life}
%autore{Frisina}
%album{Jesus is my life}
%tonalita{Re}
%gruppo{}
%momenti{Congedo;Lode}
%identificatore{jesus_christ_you_are_my_life}
%data_revisione{2011_12_31}
%trascrittore{Francesco Endrici - Manuel Toniato}
\beginsong{Jesus Christ you are my life}[by={Frisina}]

\beginchorus
\[D]Jesus \[A]Christ \[G]you are my \[D]life 
\[Em]alle\[Bm]luia alle\[Em]luia. \[A] 
\[D]Jesus \[A]Christ \[Bm]you are my \[F#m]life 
\[G]you are my \[D]life alle\[A]{lu}\[D]ia.
\endchorus

\beginverse
\[F#]Tu sei \[Bm]via, \[F#]sei veri\[Bm]tà, 
\[G]tu sei la \[D]nostra \[Em]vi\[A]ta
\[F#]Cammi\[Bm]nando \[G]insieme a \[D]te, 
\[G]vivremo in \[D]te per \[Em]sem\[A]pre.
\endverse

\beginverse
\chordsoff
Tu ci raccogli nell'unità 
riuniti nell'amore
Nella gioia dinnanzi a te, 
cantando la tua gloria
\endverse

\beginverse
\chordsoff
Nella gioia camminerem 
portando il tuo vangelo
Testimoni di carità 
figli di Dio nel mondo
\endverse
\endsong


\lettera
%titolo{Kumbaya}
%autore{}
%album{}
%tonalita{Do}
%gruppo{}
%momenti{}
%identificatore{kumbaya}
%data_revisione{2011_12_31}
%trascrittore{Francesco Endrici}
\beginsong{Kumbaya}
\beginverse*
Kumba\[C]ya, my Lord, \[F] kumba\[C]ya;
Kumba\[Em]ya, my Lord, \[F] kumba\[G]ya;
Kumba\[C]ya, my Lord, \[F] kumba\[C]ya;
\[F] O \[C]Lord, \[G] kumba\[C]ya.
\endverse
\beginverse*
Someone's singing, Lord, kumbaya\dots
Someone's praying, Lord, kumbaya\dots 
Someone's crying, Lord, kumbaya\dots 
Someone's sleeping Lord, kumbaya\dots
\endverse
\endsong


\lettera
%titolo{L'acqua della salvezza}
%autore{Roncari, Capello}
%album{La celebrazione della liberazione}
%tonalita{Re-}
%gruppo{}
%momenti{Quaresima}
%identificatore{l_acqua_della_salvezza}
%data_revisione{2011_12_31}
%trascrittore{Francesco Endrici - Manuel Toniato}
\beginsong{L'acqua della salvezza}[by={Roncari, Capello}]

\beginverse
Il Si\[Dm]gnore ci ha sal\[Gm]vato dai \[Dm]nemici
nel pas\[Gm]saggio \[C7]del Mar \[F]Rosso:
\[Gm]l'acqua che ha travolto gli Egi\[Dm]ziani
fu per \[Gm]noi \[A7]la sal\[Dm]vezza.\[D] 
\endverse

\beginchorus
\[D]Se cono\[Gm]scessi il \[C]dono di \[F]Dio
e chi è Co\[Gm]lui che ti \[A]chiede da \[Dm]bere,
lo preghe\[B&]resti tu \[C]stesso di \[F]darti
quell'acqua \[Gm]viva che ti salve\[A]rà.
(Finale: che ti salve\[D]rà)
\endchorus

\beginverse
\chordsoff
Eravamo prostrati nel deserto,
consumati dalla sete:
quando fu percossa la roccia,
zampillò una sorgente.
\endverse

\beginverse
\chordsoff
Dalle mura del tempio di Dio
sgorga un fiume d' acqua viva:
tutto quello che l' acqua toccherà,
nascerà a nuova vita.
\endverse

\beginverse
\chordsoff
Venga a me chi ha sete e chi mi cerca,
si disseti colui che in me crede:
fiumi d' acqua viva scorreranno
dal mio cuore trafitto.
\endverse

\beginverse
\chordsoff
Sulla croce, il Figlio di Dio
fu trafitto da una lancia:
dal cuore dell'Agnello immolato
scaturì sangue ed acqua.
\endverse

\beginverse
\chordsoff
Chi berrà l'acqua viva che io dono
non avrà mai più sete in eterno:
in lui diventerà una sorgente
zampillante per sempre.
\endverse
\endsong

%titolo{L'acqua la terra il cielo}
%autore{Caruso}
%album{}
%tonalita{Do}
%gruppo{}
%momenti{Ingresso}
%identificatore{l_acqua_la_terra_il_cielo}
%data_revisione{2011_12_31}
%trascrittore{Francesco Endrici}
\beginsong{L'acqua la terra il cielo}[by={Caruso}]
\beginverse
\[C]In prin\[Am]cipio la \[Dm7]terra Dio cre\[G]ò 
\[C]con i \[Am]monti i \[Dm7]prati e i suoi co\[G]lor 
il pro\[C]fumo \[Em]dei suoi \[Am]fior 
che ogni \[Dm]giorno io ri\[G]vedo intorno a \[C]me 
che os\[Am]servo la \[Dm]terra respi\[G]rar 
\[C]attra\[Am]verso le \[Dm7]piante e gli ani\[G]mal 
che co\[C]noscer \[Em]io do\[Am]vrò 
per sen\[Dm]tirmi di essa \[G]parte almeno un \[C]po'.
\endverse
\beginchorus
\[Am]Quest'avven\[Em]tura, \[F]queste sco\[C]perte 
\[F]le voglio \[C]viver con \[G]te 
\[Am]guarda che in\[Em]canto è \[F]questa na\[C]tura 
e \[F]noi siamo \[C]parte di \[G]lei. 
\endchorus
\beginverse
\chordsoff
Le mie mani in te immergerò 
fresca acqua che mentre scorri via 
fra i sassi del ruscello 
una canzone lieve fai sentire 
o pioggia che scrosci fra le fronde 
o tu mare che infrangi le tue onde 
sugli scogli e sulla spiaggia 
e orizzonti e lunghi viaggi fai sognar. 
\endverse
\beginverse
\chordsoff
Guarda il cielo che colori ha 
è un gabbiano che in alto vola già 
quasi per mostrare che 
ha imparato a viver la sua libertà 
che anch'io a tutti canterò 
se nei sogni farfalla diverrò 
anche te inviterò 
a puntare il tuo dito verso il sol.
\endverse
\endsong

%titolo{L'amore più grande}
%autore{Galliano, Anselmi}
%album{Aprite le porte a Cristo}
%tonalita{Sol}
%gruppo{}
%momenti{}
%identificatore{l_amore_piu_grande}
%data_revisione{2011_12_31}
%trascrittore{Francesco Endrici}
\beginsong{L'amore più grande}[by={Galliano, Anselmi}]
\beginchorus
Nes\[G]suno è l'a\[C]more più \[D]grande di \[G]chi
\[G7]dà la vita \[C]per gli amici \[D]suoi. \[D7]
Nes\[C]suno è l'a\[D]more più \[G]grande di \[Em]Te,
hai \[C]dato la \[D]vita per \[G]noi. \[G] \[Am]
\endchorus
\beginverse
Signore \[D] nella sera della \[Bm]cena, \[Em]
Tu \[C]servo senza \[D]nome ai nostri \[G]piedi, \[G] \[Am]
il fiume del tuo a\[D]more river\[Bm]savi, \[Em]
pa\[C]ura e male e \[D]colpa ci to\[G]glievi. \[G7]
Co\[C]sì noi rina\[D]sciamo nel per\[Bm]dono \[Em]
e \[C]tu sei pace \[D]vera dentro \[G]noi. \[D]
\endverse
\beginverse
\chordsoff
A tavola svelando il Tuo segreto:
dal mondo al padre andava il Tuo pensiero:
amici, Tuoi fratelli ci chiamavi;
e pane e vino e gioia ti facevi.
Così scopriamo il volto dell'amico
e Tu sei nuovo amore dentro noi.
\endverse
\beginverse
\chordsoff
Sul monte della nube della morte,
guardando a Dio Luce senza fine.
Nell'essere di carne ti spegnevi
e in noi un soffio nuovo respiravi.
Così da morte a vita risorgiamo
e tu sei gloria viva dentro noi.
\endverse
\endsong

%titolo{L'Emmanuel}
%autore{Mammoli}
%album{}
%tonalita{Mi}
%gruppo{}
%momenti{}
%identificatore{l_emmanuel}
%data_revisione{2011_12_31}
%trascrittore{Francesco Endrici}
\beginsong{L'Emmanuel}[by={Mammoli}]
\beginverse
Dall'\[E]orizzonte una grande luce
\[B]viaggia nella storia
e \[A]lungo gli anni ha vinto il buio
fa\[B]cendosi Me\[B7]moria.
E il\[E]luminando la nostra vita
\[B]chiaro ci rivela
che \[A]non si vive se non si cerca
\[F#m]la veri\[B7]\[E]\[B]{tà\dots}
\endverse
\beginverse
Da ^mille strade arriviamo a Roma
sui ^passi della fede
sen^tiamo l'eco della Parola
^che risuona an^cora
da ^queste mura, da questo cielo
^per il mondo intero
è ^vivo oggi, è l'Uomo Vero
^Cristo tra ^noi.
\endverse
\beginchorus
Siamo \[E]qui \[A]
sotto la stessa luce \[F#m]
sotto la sua croce \[D]
cantando ad \[B]una \[B7]voce.
\[E]È l'Emmanu\[B]el, l'Emmanu\[A]el,
Emmanu\[B]el. \[B7]
\[E]È l'Emmanu\[B]el, Emmanu\[A]el.
\endchorus
\beginverse
\chordsoff
Dalla città di chi ha versato
il sangue per amore
ed ha cambiato il vecchio mondo
vogliamo ripartire
seguendo Cristo, insieme a Pietro
rinasce in noi la fede
Parola viva che ci rinnova
e cresce in noi.
\endverse
\beginchorus
\chordsoff 
Rit. 
\endchorus
\beginverse
\transpose{1}
Un ^grande dono che Dio che ci ha fatto
è ^Cristo il Suo Figlio
e l'^umanità è rinnovata
^è in lui sal^vata
è ^vero uomo, è vero Dio
è il ^pane della Vita
che ad ^ogni uomo ai suoi fratelli
^ridone^rà.
\endverse
\beginchorus
\transpose{1}
Siamo \[E]qui \[A]
sotto la stessa luce \[F#m]
sotto la sua croce \[D]
cantando ad \[B]una \[B7]voce.
\[E]È l'Emmanu\[B]el, l'Emmanu\[A]el,
Emmanu\[B]el. \[B7]
\[E]È l'Emmanu\[B]el, Emmanu\[A]el.
\endchorus
\beginverse
\chordsoff
La morte è uccisa, la vita ha vinto
è Pasqua in tutto il mondo
un vento soffia in ogni uomo
lo Spirito fecondo
che porta avanti nella storia
la Chiesa sua sposa
sotto lo sguardo di Maria
comunità.
\endverse
\beginchorus
\chordsoff 
Rit. 
\endchorus
\beginverse
\chordsoff
Noi debitori del passato, di secoli di storia,
di vite date per amore, di santi che han creduto
di uomini che ad alta quota insegnano a volare,
di chi la storia sa cambiare come Gesù.
\endverse
\beginchorus
\chordsoff 
Rit. 
\endchorus
\beginverse
\chordsoff
È giunta un'era di primavera,
è tempo di cambiare, è oggi il giorno sempre
nuovo per ricominciare,
per dare svolte, parole nuove
e convertire il cuore,
per dire al mondo, ad ogni uomo, Signore Gesù.
\endverse
\beginchorus
\chordsoff 
Rit. 
\endchorus
\endsong




%titolo{L'unico maestro}
%autore{Civico}
%album{Route nazionale capi 1997}
%tonalita{La-}
%gruppo{}
%momenti{}
%identificatore{l_unico_maestro}
%data_revisione{2011_12_31}
%trascrittore{Francesco Endrici}
\beginsong{L'unico maestro}[by={Civico}]
\beginverse
\[Am] Le mie mani, \[Em7] con le tue, \brk \[Am]possono fare mera\[E7]viglie,
 \[Am] Possono stringere, \[Em7] perdonare \[Am]
e costruire catte\[E7]drali. \[C]
Possono \[G]dare da man\[F]gia\[E]re \[Am]
e far fiorire una pre\[E7]ghiera.
\endverse
\beginchorus
Perché \[C]tu, solo \[Em7]tu,
solo \[Am7]tu sei il mio ma\[C]estro, e insegna\[F]mi
ad a\[Fm]mare come hai \[D]fatto tu con \[C]me.
Se lo \[Em7]vuoi, io lo \[Am7]grido a tutto il \[C]mondo
che tu \[F]sei, l'\[Fm]unico maestro sei per \[C]me. \[Am]
\endchorus
\beginverse
\chordsoff
Questi piedi, con i tuoi, possono fare strade nuove,
possono correre, riposare, 
sentirsi a casa in questo mondo.
Possono mettere radici e passo passo camminare. 
\endverse
\beginverse
\chordsoff
Questi occhi con i tuoi \brk potran vedere meraviglie,
potranno piangere e luccicare, \brk guardare oltre ogni frontiera.
Potranno amare più di ieri \brk se sanno insieme a Te sognare.
\endverse
\beginverse
\chordsoff
Tu sei il corpo, noi le membra: \brk diciamo un'unica preghiera.
Tu sei il Maestro, noi testimoni \brk della parola del Vangelo.
Possiamo vivere felici \brk in questa Chiesa che rinasce.
\endverse
\ifchorded
\textnote{In Si minore:}
\transpose{2}
\beginverse*
\[Am] Le mie mani, \[Em7] con le tue, \brk \[Am]possono fare mera\[E7]viglie,
 \[Am] Possono stringere, \[Em7] perdonare \[Am]
e costruire catte\[E7]drali. \[C]
Possono \[G]dare da man\[F]gia\[E]re \[Am]
e far fiorire una pre\[E7]ghiera.
\endverse
\beginchorus
Perché \[C]tu, solo \[Em7]tu,
solo \[Am7]tu sei il mio ma\[C]estro, e insegna\[F]mi
ad a\[Fm]mare come hai \[D]fatto tu con \[C]me.
Se lo \[Em7]vuoi, io lo \[Am7]grido a tutto il \[C]mondo
che tu \[F]sei, l'\[Fm]unico maestro sei per \[C]me. \[Am]
\endchorus
\fi
\endsong

%titolo{L'uomo nuovo}
%autore{Espinosa}
%album{Canzoni dell'uomo nuovo}
%tonalita{Si-}
%gruppo{}
%momenti{}
%identificatore{l_uomo_nuovo}
%data_revisione{2011_12_31}
%trascrittore{Francesco Endrici}
\beginsong{L'uomo nuovo}[by={Espinosa}]
\beginchorus
\[Bm]Dammi un \[F#]cuor, Si\[Bm]gnor,
\[G]grande \[A]per a\[D]mare.
\[Bm]Dammi un \[F#]cuor, Si\[Bm]gnor,
\[A]pronto a lottare con \[Bm]te.
\endchorus
\beginverse
\[Bm]L'uomo \[G]nuovo crea\[A]tore della \[D]storia,
costrut\[G]tore di \[A]nuova umani\[D]tà. \[Bm]
L'uomo \[G]nuovo che \[A]vive l'esi\[Bm]sten\[Bm7]za
come un \[G]rischio che il \[Em]mondo cambie\[F#]rà.
\endverse
\beginverse
^L'uomo ^nuovo che ^lotta con spe^ranza,
nella ^vita ^cerca veri^tà. ^
L'uomo ^nuovo non ^stretto da ca^te^ne,
l'uomo ^libero che e^sige liber^tà.
\endverse
\beginverse
^L'uomo ^nuovo che ^più non vuol fron^tiere,
né vio^lenze in ^questa socie^tà. ^
L'uomo ^nuovo al ^fianco di chi ^sof^fre
divi^dendo con ^lui il tetto e il ^pane.
\endverse
\endsong

%titolo{La legge della vita}
%autore{Gen Rosso}
%album{La legge della vita}
%tonalita{La}
%gruppo{}
%momenti{}
%identificatore{la_legge_della_vita}
%data_revisione{2011_12_31}
%trascrittore{Francesco Endrici - Manuel Toniato}
\beginsong{La legge della vita}[by={Gen\ Rosso}]

\beginverse*
\[A]C'è una legge vera \[D]nella vita \brk im\[A]pressa in ogni \[D]{co}\[E]sa
\[A]legge che muove gli a\[D]stri del \[E]cielo 
in un \[Em7]concerto d'armo\[D]nia; \[A] 
e in \[A]terra canta nei co\[G]lori della na\[D]tura, \brk canta nella na\[A]tura;
e in \[A]terra canta nei co\[G]lori della na\[D]tura, \brk canta nella na\[A]tura.
\endverse

\ifchorded
\beginverse*
\vspace*{-\versesep}
{\nolyrics  \[F] \[G] \[Am7]  \[G#] \[G] \[A] }
\vspace*{-\versesep}
\endverse
\fi
\beginverse*
Il \[A]giorno cede il passo alla \[F#m7]notte per a\[D]\[E]mo\[A]re,
la \[A]notte saluta il \[D]giorno per a\[A]\[E]mo\[F#m]re.
Dal \[A]mare sale l'acqua al cielo per a\[D]\[E]mo\[A]re,
e l\[D]'acqua dis\[A]cende dal \[Bm7]cielo al \[A]mare \brk per a\[D]\[E]mo\[F#m]re.
La \[A]pianta dà le foglie alla \[F#m7]terra per a\[D]\[E]mo\[A]re,
La \[A]terra ridona le \[D]foglie per a\[A]\[E]mo\[F#m]re,
un \[A]seme cade in terra e muore per a\[D]\[E]mo\[A]re,
la \[A]vita ger\[D]moglia dal \[A]solco della \[F#m]morte \brk per a\[D]\[E]mo\[C#]re.
\endverse

\beginchorus
\[A]Questa è la legge e\[D]terna, legge di \[A]\[E]Di\[A]o,
un Dio \[D/E]che per a\[E]more  \[A4] ha creato ogni \[A]cosa,
ed ha na\[C]scosto \[G]amo\[D]re
dietro appa\[F/G]renze di \[G]morte e di do\[E4]lo\[E]re.
\[A]Questa è la legge e\[D]terna, legge di \[A]\[E]Di\[A]o,
un Dio \[D/E]che per a\[E]more \[A4]  ha creato ogni \[A]cosa,
ed ha na\[C]scosto \[G]amo\[D]re \[B&7+] 
\[A]dietro apparenze di \[F]morte e \[G]di do\[A]lore.
\endchorus

\beginverse*
\chordsoffÈ la legge vera della vita impressa in ogni cosa
legge che muove gli astri del cielo \brk in un concerto d'armonia,
e in terra canta nei colori della natura, \brk canta nella natura,
e in terra canta nei colori della natura, \brk canta nella natura.
\endverse

\ifchorded
\beginverse*
\vspace*{-\versesep}
{\nolyrics  \[F] \[G] \[Am7]  \[G#] \[G] \[A] }
\endverse
\fi
\endsong


%titolo{La mia anima canta}
%autore{Gen Verde}
%album{Cerco il tuo volto}
%tonalita{Sol}
%gruppo{}
%momenti{}
%identificatore{la_mia_anima_canta}
%data_revisione{2011_12_31}
%trascrittore{Francesco Endrici}
\beginsong{La mia anima canta}[by={Gen\ Verde}]
\ifchorded
\beginverse*
\vspace*{-0.8\versesep}
{\nolyrics \[G]\[G]\[C]\[C]\[G]\[G]}
\vspace*{-\versesep}
\endverse
\fi
\beginchorus
\[C] La mia \[D]anima can\[G]ta
la gran\[Em]dezza del Si\[Am]gnore,
il mio \[B]spirito e\[C]sulta
nel \[D]mio salva\[G]tore. \[G]
\[C] Nella \[D]mia pover\[G]tà
l’Infi\[Em]nito mi ha guar\[Am7]data,
in e\[B]terno ogni crea\[C]tura 
mi \[D]chiamerà be\[G]ata. \[G]
\endchorus
\beginverse
\memorize
La mia \[Am]gioia è nel Si\[Bm]gnore 
che ha com\[C]piuto grandi \[Bm]cose in me,
la mia \[Am]lode al Dio fe\[Bm]dele 
che ha soc\[C]corso il suo \[Bm]popolo 
e non \[G]ha dimenti\[G]cato 
le \[C]sue pro\[\vline]\[Cm]messe \[Dm7]\[\vline] d’a\[G]more. \[G]
\endverse
\beginverse
Ha di^sperso i su^perbi 
nei pen^sieri inconfes^sabili,
ha de^posto i po^tenti,
ha ri^sollevato gli ^umili,
ha sa^ziato gli affa^mati
e a^perto ai ^^ricchi ^^ le ^mani. ^
\endverse
\endsong


%titolo{La pace sia con te}
%autore{Gen Verde}
%album{Cerco il tuo volto}
%tonalita{Re}
%gruppo{}
%momenti{Pace}
%identificatore{la_pace_sia_con_te}
%data_revisione{2011_12_31}
%trascrittore{Francesco Endrici - Manuel Toniato}
\beginsong{La pace sia con te}[by={Gen\ Verde}]

\ifchorded
\beginverse*
\vspace*{-0.8\versesep}
{\nolyrics \[D] \[G] \[C/D] \[G] \[D] }
\vspace*{-\versesep}
\endverse
\fi
\beginverse*
\[D]E la \[G]pace \[A]sia \[G]con \[D]te, \[G]pa\[A]ce \[G]pa\[D]ce
\chordsoff ogni giorno sia con te, pace pace.
La Sua pace data a noi
che dovunque andrai porterai.
\chordson
\[D]E la \[G]pace \[A]sia con \[D]te.
\endverse
\endsong





%titolo{La Tua parola}
%autore{}
%album{}
%tonalita{Do}
%gruppo{}
%momenti{Ritornelli}
%identificatore{la_tua_parola}
%data_revisione{2012_01_11}
%trascrittore{Francesco Endrici}
\beginsong{La Tua parola}
\beginverse*
La \[C]tua Pa\[G]rola, Si\[C]gnore, la tua Pa\[G]rola, Si\[C]gnore,
la \[F]tua Pa\[G]rola, la \[C]tu\[Em]a Pa\[Am]rola, la \[Dm]tua Pa\[G]rola, \brk ci rende \[F]libe\[C]ri.
\endverse
\endsong
%titolo{La vera gioia}
%autore{Frisina}
%album{Non di solo pane}
%tonalita{Reb}
%gruppo{}
%momenti{Comunione}
%identificatore{la_vera_gioia}
%data_revisione{2011_12_31}
%trascrittore{Francesco Endrici}
\beginsong{La vera gioia}[by={Frisina}]
\ifchorded
\beginverse*
\vspace*{-0.8\versesep}
{\nolyrics \[D&]\[A&]}
\vspace*{-\versesep}
\endverse
\fi
\beginverse
\memorize
La vera \[D&]gioia \[E&m]nasce nella \[A&]pace, \[A&7]
la vera \[D&]gioia \[E&m]non consuma il \[A&]cuore, \[A&7]
è come \[D&]fuoco \[A&]con il suo ca\[B&m]lore \[E&m]
e dona \[D&]vita \[^E&m]quando il cuore \[A&]muore; \[^A&7]
la vera \[D&]gioia \[A&]costruisce il \[B&m]mondo \[E&m]
e porta \[D&]lu\[B&m]ce \[E&m]nell'o\[^A&]scuri\[D&]tà. \[A7]
\endverse
\beginverse
\transpose{1}
\preferflats
La vera ^gioia ^nasce dalla ^luce ^
che splende ^viva ^in un cuore ^puro, ^
la veri^tà so^stiene la sua ^fiamma ^
perciò non ^teme \[E&m]ombra nè men^zogna, \[A&7]
la vera ^gioia ^libera il tuo ^cuore, ^
ti rende ^can^to ^nella \[A&7]liber^tà. ^
\endverse
\beginverse
\transpose{2}
La vera ^gioia ^vola sopra il ^mondo ^
ed il pec^cato ^non potrà fer^marla, ^
le sue ^ali ^splendono di ^grazia, ^
dono di ^Cristo e \[E&m7]della sua sal^vezza \[A&]
e tutti u^nisce ^come in un ab^braccio ^
e tutti ^a^ma ^nella \[A&]cari^tà. ^
\endverse
\beginverse
\transpose{3}
{\nolyrics ^^^^
^^^^
^^^^
^\[E&m7]^\[A&]}
e tutti u^nisce ^come in un ab^braccio ^
e tutti ^a^ma ^nella \[A&7]cari\[D&]tà.
\endverse
\endsong



%titolo{Laudato sii mio dolcissimo Signore}
%autore{Gualano}
%album{Nel saio di Francesco}
%tonalita{Re}
%gruppo{}
%momenti{Lode;San Francesco}
%identificatore{laudato_sii_mio_dolcissimo_signore}
%data_revisione{2011_12_31}
%trascrittore{Francesco Endrici}
\beginsong{Laudato sii mio dolcissimo Signore}[by={Gualano}]
\beginverse
\[D] Laudato \[F#m]sii mio dol\[G]cissimo Si\[A]gnore, \[D]
con tutte \[F#m]quante le tue \[G]splendide crea\[A]tu\[F#m]re \[Bm]
special\[Gm]mente frate \[D]sole, che fa \[E]giorno e illumi\[G]na
bello e \[A4]grande \[A]come \[D]Te.
\endverse
\ifchorded
\beginverse*
\vspace*{-\versesep}
{\nolyrics \[F#m] \[G]\[A]\[D]\[F#m]\[G]\[A]}
\endverse
\fi
\beginverse
^ Laudato ^sii mio dol^cissimo Si^gnore, ^
per la ^luna e per le ^stelle lumi^nose e ^belle ^
per il ^vento e, l'aria, il ^cielo, 
per le ^nubi ed il se^reno, 
Tu sostieni tutto il \[D]mondo 
e il \[Bm]mondo vive \[A]grazie a \[A]Te. \[F]
\endverse
\beginchorus
Sii lau\[B&]dato mio Si\[F]gnore per la \[B&]vita che mi \[Am]dai
per la \[Dm]gioia di tro\[Gm]varti 
tutti i \[C]giorni accanto a \[F]me.
Sii lau\[B&]dato mio Si\[F]gnore, cante\[B&]rò per sempre a \[Am]Te, 
Te che \[Dm]abiti da \[Gm]sempre 
e che re\[C]spiri dentro \[D]me, \[F#m]\[G] 
che re\[A]spiri dentro \[D]me. \[F#m]\[G]\[A] 
\endchorus
\beginverse
^ Laudato ^sii mio dol^cissimo Si^gnore 
an^che per ^l'umile pre^ziosa e dolce ^ac^qua, ^
per il ^fuoco che ri^scalda, dona ^luce 
al nostro ^cuore nella ^notte ^scura. \brk ^ \[F#m] \[G]\[A] 
\endverse
\beginverse
^ Laudato ^sii mio dol^cissimo Si^gnore, per la \brk ^nostra madre ^terra che ci ^nutre e a ^tutti ^dà, 
frutti e ^fiori, colo^rati dei co^lori della ^vita, 
quella vita che da \[D]sempre, 
che da \[Bm]sempre esiste \[A]grazie a \[A]Te. \[F]
\endverse
\beginchorus
Sii lau\[B&]dato mio Si\[F]gnore per la \[B&]vita che mi \[Am]dai
per la \[Dm]gioia di tro\[Gm]varti 
tutti i \[C]giorni accanto a \[F]me.
Sii lau\[B&]dato mio Si\[F]gnore, cante\[B&]rò per sempre a \[Am]Te, 
Te che \[Dm]abiti da \[Gm]sempre 
e che re\[C]spiri dentro \[D]me, \[F#m]\[G] 
che re\[A]spiri dentro \[D]me. \[F#m]\[G]\[A] 
\endchorus
\beginverse
^ Laudato ^sii mio dol^cissimo Si^gnore, ^ 
per quelli ^che perdona^no per il tuo a^mo^re. ^
Danno ^pace ed in si^lenzio, sanno an^che morire 
^dentro, sempre ^lì vi^cino a ^Te.
\[D]Laudato \[F#m]sii mio dol\[G]cissimo Si\[A]gnore, \brk \[D]\[F#m] \[G]\[A]
\[D] Laudato \[F#m]sii mio dol\[G]cissimo Si\[A]gnore, \brk \[D]\[F#m] \[G]\[A]\[F]
\endverse
\beginchorus
Sii lau\[B&]dato mio Si\[F]gnore per la \[B&]vita che mi \[Am]dai
per la \[Dm]gioia di tro\[Gm]varti 
tutti i \[C]giorni accanto a \[F]me.
Sii lau\[B&]dato mio Si\[F]gnore, cante\[B&]rò per sempre a \[Am]Te, 
Te che \[Dm]abiti da \[Gm]sempre 
e che re\[C]spiri dentro \[D]me, \[F#m]\[G] 
che re\[A]spiri dentro \[D]me. \[F#m]\[G]\[A]  \[D]
\endchorus
\beginverse*
Finale: Laudato \[F#m]sii mio dol\[G]cissimo Si\[A]gnore. \brk \[D]\[F#m] \[G]\[A]\[D]
\endverse
\endsong

%titolo{Laudato sii, o mi Signore}
%autore{Agape}
%album{}
%tonalita{Mi}
%gruppo{}
%momenti{Lode}
%identificatore{laudato_sii_o_mi_signore}
%data_revisione{2011_12_31}
%trascrittore{Francesco Endrici}
\beginsong{Laudato sii, o mi Signore}[by={Agape}]
\beginchorus
\[E]Laudato sii, o mi Signore,
\[C#m]laudato sii, o mi Signore,
\[A]laudato sii, o mi Signore,
\[B]laudato sii, o mi Signore.
\endchorus
\beginverse
%\chordsoff
\[E]E per tutte le tue creature
\[C#m]per il sole e per la luna
\[A]per le stelle e per il vento
\[B]e per l'acqua e per il fuoco.
\endverse
\beginverse
%\chordsoff
^Per sorella madre terra
^ci alimenta e ci sostiene
^per i frutti, i fiori e l'erba
^per i monti e per il mare.
\endverse
\beginverse
%\chordsoff
^Perchè il senso della vita
^è cantare e lodarti
^e perchè la nostra vita
^sia sempre una canzone.
\endverse
\endsong

%titolo{Laudato sii, Signore mio}
%autore{Cento}
%album{Guarda laggiù l'orizzonte}
%tonalita{Mi}
%gruppo{}
%momenti{Lode;San Francesco}
%identificatore{laudato_sii_signore_mio}
%data_revisione{2011_12_31}
%trascrittore{Francesco Endrici}
\beginsong{Laudato sii, Signore mio}[by={Cento},ititle={Il canto della creazione}]
\beginchorus
\[E]Laudato sii Signore \[F#m]mio \[B] 
Laudato sii Signore \[C#m]mio \[A]
Laudato sii Signore \[B]mio \[A] 
Laudato \[F#m]sii Si\[B]gnore \[E]mio.
\endchorus
\beginverse
\[E]Per il sole d'ogni \[F#m]giorno, \[B] 
che riscalda e dona \[C#m]vita. \[A]
Egli illumina il cam\[B]mino \[A] 
di chi \[F#m]cerca te Si\[B7]gnore. \[E]
Per la luna e per le \[F#m]stelle, \[B] 
io le sento mie so\[C#m]relle \[A]
le hai formate su nel \[B]cielo \[A] 
e le \[F#m]doni a chi è nel \[B]buio. 
\endverse
\beginverse
\chordsoff
^Per la nostra madre ^terra, ^ 
che ci dona fiori ed ^erba, ^
su di lei noi fati^chiamo, ^ 
per il ^pane d'ogni ^giorno. ^
Per chi soffre con co^raggio, ^ 
e perdona nel tuo a^more, ^
Tu gli dai la pace ^tua, ^ 
alla ^sera della ^vita.
\endverse
\beginverse
\chordsoff
^Per la morte che è di ^tutti, ^ 
io la sento ogni i^stante, ^
ma se vivo nel tuo a^more, ^ 
dona un ^senso alla mia ^vita. ^
Per l'amore che è nel ^mondo, ^ 
tra una donna e l'uomo ^suo, ^
per la vita dei bam^bini ^ 
che il mio ^mondo fanno ^nuovo.
\endverse
\beginverse
\chordsoff
^Io ti canto mio Si^gnore ^ 
e con me la crea^zione ^
ti ringrazia umil^mente ^ 
perché ^tu sei il Si^gnore. ^\rep{2}
\endverse
\endsong


%titolo{Le mani alzate}
%autore{Vercruysse , Meloni}
%album{Dio della mia lode}
%tonalita{Re}
%gruppo{}
%momenti{Offertorio}
%identificatore{le_mani_alzate}
%data_revisione{2011_12_31}
%trascrittore{Francesco Endrici - Manuel Toniato}
\beginsong{Le mani alzate}[by={Vercruysse , Meloni}]
\beginchorus
\[D]Le mani al\[G]zate verso \[D]Te \[F#m]Si\[Bm]gnor,
\[G]per o\ch{D}{f}{f}{ff}rirti il \[E]mon\[A7]do.
\[D]Le mani al\[G]zate verso \[D]Te \[F#m]Si\[Bm]gnor,
\[G]gioia è in \[D]me nel pro\[A7]fon\[D]do.
\endchorus

\beginverse
\[Dm]Guardaci Tu Si\[Gm]gnore siamo \[Dm]\[A]tuo\[Dm]i,
\[B&]piccoli \[F]siam davanti a \[A]Te. \[A7] 
\[Dm]Come ruscelli \[Gm]siamo d'acqua \[Dm]lim\[A]pi\[B&]da,
\[Gm]semplici e \[A7]puri innanzi a \[Dm]Te.
\endverse

\beginverse
\chordsoff
Guardaci tu, Signore, siamo tuoi
sei via, vita e verità.
Se ci terrai la mano nella mano,
il cuore più non temerà.
\endverse
\endsong



%titolo{Le tue mani}
%autore{Giombini}
%album{Camminiamo nella speranza}
%tonalita{Re}
%gruppo{}
%momenti{Pasqua}
%identificatore{le_tue_mani_giombini}
%data_revisione{2011_12_31}
%trascrittore{Francesco Endrici}
\beginsong{Le tue mani}[by={Giombini}]
\beginverse
\[A7]Le tue \[D]mani son \[A]piene di \[D]fiori:
\[Bm]dove li por\[F#m]tavi fra\[G]tello \[D]mio? \[A7]
Li por\[D]tavo alla \[A7]tomba di \[D]Cristo,
ma l'\[G]ho trovata \[D]vuota, so\[A7]rella \[D]mia!
\endverse
\beginchorus
\[G]Al\[A]le\[F#m]lu\[Bm]ia, \[Em] Alle\[A7]lu\[D]ia! \[A]
\[G]Al\[A]le\[F#m]lu\[Bm]ia, \[Em] Al\[D]le\[A7]lu\[D]ia! \[G]\[D]
\endchorus
\beginverse
\chordsoff
I tuoi occhi riflettono gioia:
dimmi, cos'hai visto, fratello mio?
Ho veduto morire la morte,
ecco cosa ho visto, sorella mia!
\endverse
\beginverse
\chordsoff
Hai portato una mano all'orecchio:
dimmi cosa ascolti, fratello mio?
Sento squilli di trombe lontane,
sento cori d'angeli, sorella mia!
\endverse
\beginverse
\chordsoff
Stai cantando un'allegra canzone:
dimmi, perché canti, fratello mio?
Perché so che la vita non muore,
ecco perché canto, sorella mia!
\endverse
\endsong

%titolo{Le tue meraviglie}
%autore{Casucci, Balduzzi}
%album{Verbum Panis}
%tonalita{La-}
%gruppo{}
%momenti{Congedo;Natale}
%identificatore{le_tue_meraviglie}
%data_revisione{2011_12_31}
%trascrittore{Francesco Endrici - Manuel Toniato}
\beginsong{Le tue meraviglie}[by={Casucci, Balduzzi}]

\ifchorded
\beginverse*
\vspace*{-0.8\versesep}
{\nolyrics \[Am] \[Em] \[F] \[C] \[Dm] \[Am] \[F] \[G] 
\[Am] \[Em] \[F] \[C] \[Dm] \[Am] \[F] \[G] \[Am] }
\vspace*{-\versesep}
\endverse
\fi

\beginchorus
Ora \[F]lascia, o Si\[G]gnore, che io \[Em]vada in pa\[Am]ce,
perché ho \[Dm]visto le tue \[C]mera\[B&]vi\[G]glie.
Il tuo \[F]popolo in \[G]festa per le \[Em]strade corre\[Am]rà
a por\[Dm]tare le tue \[C]mera\[B&]vi\[G]glie!
\endchorus

\beginverse
\[Am]La tua pre\[Em]senza ha riem\[F]pito d'a\[C]more
\[Am]le nostre \[Em]vite, le \[F]nostre gior\[C]nate.
\[B&]In te una sola \[F]anima, \[Gm]un solo cuore \[F]siamo noi:
\[B&]con te la luce ri\[F]splende, \brk \[Gm]splende più chiara che \[C]mai!
\endverse

\beginverse
\chordsoff
La tua presenza ha inondato d'amore
le nostre vite, le nostre giornate,
fra la tua gente resterai, \brk per sempre vivo in mezzo a noi
fino ai confini del tempo: così ci accompagnerai.
\endverse

\beginchorus
Ora \[F]lascia, o Si\[G]gnore, che io \[Em]vada in pa\[Am]ce,
perché ho \[Dm]visto le tue \[C]mera\[B&]vi\[G]glie.
Il tuo \[F]popolo in \[G]festa per le \[Em]strade corre\[Am]rà
a por\[Dm]tare le tue \[C]mera\[B&]vi\[G]glie!
Ora \[F]lascia, o Si\[G]gnore, che io \[Em]vada in pa\[Am]ce,
perché ho \[F]visto le \[G]tue mera\[Em]vi\[Am]glie.
E il tuo \[F]popolo in \[G]festa per le \[Em]strade corre\[Am]rà
a por\[F]tare le \[G]tue mera\[F]vi\[C]glie!
\endchorus

\ifchorded
\beginverse*
\vspace*{-\versesep}
{\nolyrics \[Am] \[Em] \[F] \[C] \[Dm] \[Am] \[F] \[G] 
\[Am] \[Em] \[F] \[C] \[Dm] \[Am] \[F] \[G] \[Am] }
\endverse
\fi
\endsong

%titolo{Lo Spirito del Signore è su di me}
%autore{Comi}
%album{Preghiera di un povero}
%tonalita{Sol}
%gruppo{}
%momenti{Pentecoste}
%identificatore{lo_spirito_del_signore_e_su_di_me}
%data_revisione{2011_12_31}
%trascrittore{Francesco Endrici - Manuel Toniato}
\beginsong{Lo Spirito del Signore è su di me}[by={Comi}]

\beginchorus
Lo \[G]Spirito del Si\[D]gnore è su di \[Em]me, \[D] 
lo \[G]Spirito del Si\[D]gnore mi ha consa\[G]crato,
lo \[C]Spirito del Si\[D]gnore mi ha invi\[G]ato
a por\[C]tare il lieto an\[D]nuncio ai \[C]pove\[G]ri.
\endchorus

\beginverse
A fa\[G]sciare le \[D]piaghe dei \[Em]cuori spez\[G]zati,
a pro\[C]clamare la liber\[D]tà degli \[G]schiavi,
a promul\[C]gare l'anno di \[D7]grazia del Si\[B7]gnore,
e per conso\[G]lare \[D]tutti gli a\ch{Em}{f}{fl}{ffl}itti
dando \[G]loro \[D]una co\[G]rona,
olio di gioia, \[D]canto di \[Em]lode
invece di \[C]lutto e di do\[B7]lore.
\endverse

\chordsoff
\beginverse
Essi si chiameranno querce di giustizia,
la piantagione gradita al Signore,
segno per tutti della sua gloria.
E ricostruiranno le vecchie rovine,
rialzeranno gli antichi ruderi,
restaureranno città desolate
e devastate da più generazioni.
\endverse

\beginverse
Ed essi saranno chiamati sacerdoti del Signore,
saranno detti ministri del nostro Dio,
e dalle nazioni saranno serviti.
Ed essi godranno le loro ricchezze,
trarranno vanto dai loro beni.
Avranno gloria e non vergogna,
grida di gioia e non di oppressione.
\endverse

\beginverse
Poiché io sono il Signore che ama la giustizia,
darò loro fedelmente il giusto salario,
concluderò con loro un'alleanza.
E saranno famosi tra tutti i popoli,
la loro stirpe tra le nazioni.
Chi li vedrà ne avrà stima,
perché sono benedetti da Dio.
\endverse
\endsong

%titolo{Lo Spirito del Signore}
%autore{Frisina}
%album{Tu sarai profeta}
%tonalita{Re}
%gruppo{}
%momenti{Pentecoste}
%identificatore{lo_spirito_del_signore_frisina}
%data_revisione{2011_12_31}
%trascrittore{Francesco Endrici}
\beginsong{Lo Spirito del Signore}[by={Frisina}]
\beginchorus
\memorize
Lo \[D]Spirito del Si\[Em]gnore \[A]è su di \[D]me, \[G]\[A]
lo \[D]Spirito con l'un\[G]zione m'\[Em]ha consa\[A4]cra\[A]to,
lo \[G]Spirito m'ha man\[D]dato \[Em]ad annunziare ai \[A]poveri
\[Bm]un \[G]lieto mes\[D]sag\[Bm]gio di sal\[Em]\[A]vez\[D]za.
\endchorus
\beginverse
Lo ^Spirito di sa^pienza ^è su di ^me, ^^
per ^essere luce e ^guida ^sul mio cam^mi^no,
mi ^dona un linguaggio ^nuovo
^per annunziare agli ^uomini
^la ^tua pa^ro^la di sal^^vez^za.
\endverse
\beginverse
Lo ^Spirito di for^tezza ^è su di ^me, ^^
per ^testimoniare al ^mondo ^la Sua Pa^ro^la,
mi ^dona il Suo co^raggio ^per annunciare al ^mondo,
l'^av^vento glo^rio^so del tuo ^^re^gno.
\endverse
\beginverse
Lo ^Spirito del ti^more ^è su di ^me, ^^
per ^rendermi testi^mone ^del Suo per^do^no,
^purifica il mio ^cuore ^per annunciare agli ^uomini,
^le ^opere ^gran^di del Si^^gno^re.
\endverse
\beginverse
Lo ^Spirito della ^pace ^è su di ^me, ^^
^e mi ha colmato il ^cuore ^della sua ^gio^ia,
mi ^dona un canto ^nuovo ^per annunziare al ^mondo,
^il ^giorno di ^gra^zia del Si^^gno^re.
\endverse
\beginverse
Lo ^Spirito dell'A^more ^è su di ^me, ^^
per^ché possa dare al ^mondo ^la mia ^vi^ta,
mi ^dona la Sua ^forza ^per consolare i ^poveri,
^per ^farmi stru^men^to di sal^^vez^za.
\endverse
\endsong

%titolo{Lo spirito di Cristo vive in noi}
%autore{Mellino}
%album{Sono giovane Signore}
%tonalita{Sol}
%gruppo{}
%momenti{Pentecoste}
%identificatore{lo_spirito_di_cristo_vive_in_noi}
%data_revisione{2011_12_31}
%trascrittore{Francesco Endrici}
\beginsong{Lo spirito di Cristo vive in noi}[by={Mellino}]
\beginchorus
Lo \[G]Spirito di Cristo vive in \[C]noi,
ci \[D]rende sua dimora nella veri\[G]tà
e noi siamo \[C]suoi \[D] nella liber\[G]tà,
\[C]Chiesa viva \[D]nella cari\[G]tà. \[C]\[G]
\endchorus
\beginverse
La tua Pa\[C]rola \[D] ci risuona \[G]dentro,
in noi di\[C]venta \[D]forza di spe\[G]ranza;
illumi\[C]nati \[D]dalla tua sa\[Em]pienza
tu ci con\[C]duci verso l'uni\[D4]tà. \[D]
\endverse
\beginverse
\chordsoff
Riuniti in te siamo un solo corpo,
il tempio santo della gloria tua;
tu ci disseti, sei la sola fonte
e un solo amore cresce dentro noi.
\endverse
\beginverse
\chordsoff
Tu vivo fuoco, dono del Risorto,
ci rendi nuovi, figli dell'Eterno;
in te preghiamo Dio nostro Padre
e la sua gloria ora brilla in noi.
\endverse
\endsong




%titolo{Lodate Dio}
%autore{Gen Rosso}
%album{Dove Tu sei}
%tonalita{Sol}
%gruppo{}
%momenti{}
%identificatore{lodate_dio_gen_rosso}
%data_revisione{2011_12_31}
%trascrittore{Francesco Endrici - Manuel Toniato}
\beginsong{Lodate Dio}[by={Gen\ Rosso}]

\beginchorus
\[G] Lodate \[C]Di\[G]o \[F]cieli im\[C]mensi ed \[G]infiniti.
Lodate \[C]Di\[G]o \[Dm7]cori eterni \[Am]d'Angeli.\[C] 
Lodate \[D7]Dio \[G]Santi \[Am7]del Suo \[D4]Re\[D]gno.
Lo\[G]datelo \[Am7]uomi\[D]ni, \[Em]Dio vi \[D]ama.
Lo\[G]datelo \[Am7]uomi\[D]ni, \[C]Dio è con \[G]voi.
\endchorus

\ifchorded
\beginverse*
\vspace*{-\versesep}
{\nolyrics \[G] \[C] \[G] \[F] \[C] \[G] 
\[G] \[C] \[G] \[F] \[Dm] \[Am] 
\[C] \[D] \[G] \[Am] \[D] 
\[G] \[C] \[G] \[F] \[C] \[G] }
\endverse
\fi

\beginverse
\chordsoff
(Parlato) Ti ringraziamo, Dio nostro Padre
perché sei Amore.
Tu ci hai fatto dono della vita
e ci hai creati per essere figli tuoi.
Ti ringraziamo perché ci fai partecipi
della tua opera creatrice
dandoci un mondo da plasmare
con le nostre mani.
\endverse

\beginverse
\chordsoff
(Parlato) Ti ringraziamo, Signore Gesù,
perché per amore nostro
sei venuto in questo mondo
per riscattarci dal ogni male
e riportarci al Padre.
Ti ringraziamo perché sei rimasto
in mezzo a noi per sempre
per far di tutti gli uomini una sola famiglia,
un corpo solo in te.
\endverse

\beginverse
\chordsoff
(Parlato) Ti ringraziamo, Spirito d'amore,
perché rinnovi la faccia della terra.
Tu dai luce e conforto ad ogni cuore.
Ti ringraziamo perché con i tuoi santi doni
ci dai la forza di avanzare nel nostro cammino
per giungere uniti alla gioia della tua casa.
\endverse
\endsong

%titolo{Lodate Iddio}
%autore{Cento}
%album{È il giorno del Signore}
%tonalita{Re-}
%gruppo{}
%momenti{Salmi;Lode}
%identificatore{lodate_iddio}
%data_revisione{2011_12_31}
%trascrittore{Francesco Endrici - Manuel Toniato}
\beginsong{Lodate Iddio}[by={Cento}]
\beginverse
\[Dm]Sole, \[F]vento e \[C]fiori del \[Dm]campo, \brk lo\[Dm]date, lo\[F]date, lo\[C]date Id\[Dm]dio.
\[Dm]Terra, \[F]uomo, uc\[C]celli del \[Dm]cielo, \brk lo\[Dm]date, lo\[F]date, lo\[C]date Id\[Dm]dio.
\[Dm]Voi \[C]che a\[F]mate la vita e i fra\[C]telli, \brk lodate, lo\[B&]date, lo\[A]date Id\[Dm]dio.
\[Dm]Voi \[C]che \[F]siete felici o \[C]tristi, \brk lodate, lo\[B&]date, lo\[A]date Id\[Dm]dio.
\endverse

\beginverse
\chordsoff
Fuoco e nebbia e cime dei monti, \brk lodate, lodate, lodate Iddio
Nevi eterne e acque dei fiumi, \brk lodate, lodate, lodate Iddio
Voi che avete la pace nel cuore, \brk lodate, lodate, lodate Iddio
Voi che lottate sul posto di lavoro, \brk lodate, lodate, lodate Iddio.
\endverse

\beginverse
\chordsoff
Con le stelle accese nel cielo, \brk lodate, lodate, lodate Iddio
con i bimbi felici nel mondo, \brk lodate, lodate, lodate Iddio
Con i ragazzi che cercano amore, \brk lodate, lodate, lodate Iddio
Con gli oppressi di ogni colore, \brk lodate, lodate, lodate Iddio.
\endverse

\beginverse
\chordsoff
È Gesù la speranza dell'uomo \brk lodate, lodate, lodate Iddio
Noi cristiani viviamo di Lui \brk lodate, lodate, lodate Iddio
E cantiamo la gioia e l'amore \brk lodate, lodate, lodate Iddio
Che rinasce in chi crede in Lui, \brk lodate, lodate, lodate Iddio.
\endverse
\endsong

%titolo{Lode e gloria}
%autore{Spoladore}
%album{Dacci pace}
%tonalita{Sol}
%gruppo{}
%momenti{Acclamazione al Vangelo;Salmi}
%identificatore{lode_e_gloria}
%data_revisione{2011_12_31}
%trascrittore{Francesco Endrici - Manuel Toniato}
\beginsong{Lode e gloria}[by={Spoladore}]

\ifchorded
\beginverse*
\vspace*{-0.8\versesep}
{\nolyrics \[G] \[D] \[Am] \[Em] }
\vspace*{-\versesep}
\endverse
\fi

\beginchorus
\[G]Lode e \[D]gloria a \[Am]te o Si\[Em]gnore
\[G]Lode e \[D]Gloria a \[Am]Te o Si\[C]\[D]gno\[Em]re
\endchorus

\beginverse
\[D]Date \[G]lode al Signore o \[D]Figli di Dio
bene\[Am]dite il suo nome su \[Em]tutta la terra.
\[D]Il suo a\[G]more per noi è fe\[D]dele per sempre,
il suo a\[Am]more per noi è per \[C]\[D]sem\[Em]pre.
\endverse

\beginverse
\chordsoff
Acclamate al Signore da tutta la Terra
ci ha creati e voluti a lui apparteniamo
il Signore ci guida e ci tiene per mano
senza fine è il suo amore per noi
\endverse
\endsong

%titolo{Lode e gloria a te}
%autore{}
%album{Dio della mia lode}
%tonalita{Mi}
%gruppo{}
%momenti{Acclamazione al Vangelo}
%identificatore{lode_e_gloria_a_te}
%data_revisione{2011_12_31}
%trascrittore{Francesco Endrici - Manuel Toniato}
\beginsong{Lode e gloria a te}[by={Sconosciuto}]

\beginchorus
\[E]Lode e \[A]gloria a \[E]te, \[A]  \[E]lode e \[B]gloria a \[E]te. \[B] 
\[C#m]Luce del mat\[F#m]tino, \[E]lode e \[B]gloria a \[E]te.
\endchorus

\beginverse
\chordsoff
M'ha fatto camminare, m'ha fatto camminare,
per questo canto: lode e gloria a te.
\endverse

\beginverse
\chordsoff
Lodatelo nel tempio, lodatelo nel ciel,
io sempre canto: lode e gloria a te.
\endverse

\beginverse
\chordsoff
Lo loderò con l'arpa, io loderò il Signore,
ha fatto grandi cose: lode e gloria a te.
\endverse

\beginverse
\chordsoff
Lo loderò con danze, m'ha fatto camminar
per questo canto: lode e gloria a te.
\endverse
\endsong

%titolo{Lodi all'Altissimo}
%autore{Frisina}
%album{Tu sei bellezza}
%tonalita{Re-}
%gruppo{}
%momenti{San Fracesco;Comunione}
%identificatore{lodi_all_altissimo_frisina}
%data_revisione{2011_12_31}
%trascrittore{Francesco Endrici - Manuel Toniato}
\beginsong{Lodi all'Altissimo}[by={Frisina}]

\beginverse
Tu sei \[Dm]Santo, Signore \[Am]Dio,
Tu sei \[B&]forte, Tu sei \[Gm]grande, \[A] 
Tu sei l'Al\[F]tissimo, l'Onnipo\[C]tente, \[A] 
Tu Padre \[Dm]Santo, \[B&]Re \[Am7]del \[Dm]cielo.
\endverse

\beginverse
Tu sei \[B&]trino, \[C] uno Si\[F]gnore, \[Am] 
Tu sei il \[Gm]bene, \[A]tutto il \[Dm]bene,
Tu \[Am]sei l'A\[Gm/B&]more, \[C] Tu sei il \[F]vero,
Tu sei umil\[B&]tà, Tu \[Gm6]sei sa\[A]pienza.
\endverse
\chordsoff
\beginverse
Tu sei bellezza, Tu sei la pace,
la sicurezza, il gaudio, la letizia,
Tu sei speranza, Tu sei giustizia,
Tu temperanza e ogni ricchezza.
\endverse

\beginverse
Tu sei il Custode, Tu sei mitezza,
Tu sei rifugio, Tu sei fortezza,
Tu carità, fede e speranza,
Tu sei tutta la nostra dolcezza.
\endverse

\beginverse
Tu sei la vita, eterno gaudio,
Signore grande, Dio ammirabile,
Onnipotente, o Creatore,
o Salvatore di misericordia.
\endverse
\endsong

%titolo{Lodi di Dio Altissimo}
%autore{}
%album{}
%tonalita{La}
%gruppo{}
%momenti{San Francesco}
%identificatore{lodi_di_dio_altissimo_assisi}
%data_revisione{2011_12_31}
%trascrittore{Francesco Endrici - Manuel Toniato}
\beginsong{Lodi di Dio Altissimo}

\beginchorus
\[A2]Tu sei Santo, Si\[E/A]gnore Dio, fai \[A4]cose grandi, me\[A]ravigliose,
\[A2]Tu sei il bene, il \[E/A]sommo bene, \brk Tu \[A4]sei il Signore on\[A]nipotente!
\endchorus

\beginverse
\[F#m]Tu sei forte, Tu sei \[C#m]grande, \brk \[F#m]Tu sei l'Altissimo, Onnipo\[C#m]tente,
Tu \[D]sei presente, sei, per \[E]sempre Dio presente \[A]sei.
\endverse

\beginverse
\chordsoff
Tu sei trino e un solo Dio
Tu sei il Re del cielo e della terra \brk Tu sei bellezza, sei per sempre Dio bellezza sei
\endverse

\beginverse
\chordsoff
Tu sei amore e carità \brk Tu sei sapienza ed umiltà,
Tu sei letizia sei, per sempre Dio letizia sei
\endverse

\beginverse
\chordsoff
Tu sei la mia speranza e la mia forza \brk Tu sei la mia ricchezza e la mia gioia
Tu sei la pace, sei per sempre Dio la pace sei
\endverse
\endsong

%titolo{Luce}
%autore{Comunità del Cenacolo}
%album{Dio dell'amore}
%tonalita{Sol}
%gruppo{}
%momenti{}
%identificatore{luce}
%data_revisione{2011_12_31}
%trascrittore{Francesco Endrici - Manuel Toniato}
\beginsong{Luce}[by={Comunità\ del\ Cenacolo}]

\beginverse
\[G]C'è il segreto della liber\[D]tà \brk quella \[Em]vera nasce \[C]dentro di \[D]te.
È \[G]come risvegliarsi un mat\[D]tino con il \[Em]sole \brk dopo un \[C]lungo in\[D]verno
\[G] nel soffrire \[D] mio Signore \[Em] \brk ho incontrato \[C]Te Dio a\[D]more
\[G] nel perdono \[D] nel gioire \[Em] \brk ho capito \[C]che sei \[D]luce per me.
\endverse

\beginchorus
\[G] Signore sono \[D]qui per dirti ancora \[Em]sì \[C]lu\[D]ce
\[G] fammi scoppiare \[D]di gioia di vive\[Em]re \[C]lu\[D]ce.
\[G] Fammi strumento \[D]per portare attorno a \[Em]me \[C]lu\[D]ce
\[G] e chi è vicino a \[D]me sappia che tutto in \[Em]te è \[C]lu\[D]ce.
\endchorus

\beginverse
\chordsoff
Voglio ringraziarti Signore \brk per la Vita che mi hai ridonato
so che sei nell'amore degli amici \brk che ora ho incontrato
nel soffrire mio Signore \brk ho incontrato Te Dio amore
nel perdono nel gioire ho capito \brk che sei luce per me.
\endverse

\beginchorus
\chordsoff
Signore sono qui per dirti ancora sì luce\ldots
\endchorus

\beginverse*
\[Em] E con le lacrime agli \[C]occhi \brk e le mie mani al\[G]zate verso te Ge\[D]sù
\[Em] con la speranza nel \[C]cuore \brk e la tua luce in \[D4]me paura non ho \[D]più.
\endverse
\endsong



%titolo{Luce che sorgi nella notte}
%autore{Buttazzo}
%album{Gloria all'Emmanuele}
%tonalita{Re-}
%gruppo{}
%momenti{Natale}
%identificatore{luce_che_sorgi_nella_notte}
%data_revisione{2011_12_31}
%trascrittore{Francesco Endrici - Manuel Toniato}
\beginsong{Luce che sorgi nella notte}[by={Buttazzo}]


\ifchorded
\beginverse*
\vspace*{-0.8\versesep}
{\nolyrics \[Dm] \[B&] \[C] \[Dm] \[Dm] \[B&] \[C] \[D] }
\vspace*{-\versesep}
\endverse
\fi
\beginchorus
\[Dm]Luce che sorgi \[C]nella \[Dm]notte, \brk can\[B&]tiamo a \[C]te, o Si\[Dm]gnore!
\[Dm]Stella che splendi \[C]nel mat\[Dm]tino \brk di un \[B&]nuovo \[Gm]gior\[A]no.
Can\[B&]tiamo a \[F]{te,} \[Gm]Cristo Ge\[Dm]sù, \brk can\[B&]tiamo a \[C]te, o Si\[Dm]gnore.
\endchorus
\beginverse
\[Dm]Mentre il si\[B&]lenzio av\[C]volge la \[Dm]terra
tu \[F]vieni in mezzo a \[Gm]noi, Pa\[B&]rola del \[C]Padre.
Ri\[B&]veli ai \[C]nostri \[F]cuori l'a\[Gm]more di \[A4]{Di}\[A]o.
A \[B&]te la lo\[F]de, a \[Gm]te la \[Dm]glo\[C]ria, \[B&]nostro \[Gm]Salva\[A]tore!
\endverse
\beginverse
\chordsoff
Mentre la notte si fa più profonda
tu vieni in mezzo a noi, Splendore del Padre:
E doni ai nostri cuori la luce di Dio.
A te la lode, a te la gloria, nostro Salvatore!
\endverse
\ifchorded
\beginverse*
\vspace*{-\versesep}
{\nolyrics \[Dm] \[B&] \[C] \[Dm] \[Dm] \[B&] \[C] \[D] }
\endverse
\fi
\endsong


%titolo{Luce di Verità}
%autore{Becchimanzi, Scordari, Giordano}
%album{Inno dell'Agorà dei Giovani}
%tonalita{Mi}
%gruppo{}
%momenti{Ingresso;Congedo;Pentecoste}
%identificatore{luce_di_verita}
%data_revisione{2011_12_31}
%trascrittore{Francesco Endrici - Manuel Toniato}
\beginsong{Luce di Verità}[by={Becchimanzi, Scordari, Giordano}]

\beginchorus
\[E]Luce di veri\[A/E]tà, \[E]fiamma di cari\[E/G#]tà,
\[A]vincolo di uni\[E/G#]tà, \[C#m7]Spirito \[F#7]Santo A\[A/B]mo\[B]re.
\[E]Dona la liber\[A/E]tà, \[E]dona la santi\[E/G#]tà,
\[A]fa' dell'umani\[E/G#]tà \[C#m7]il tuo \[D]canto di \[A/B]lo\[B]de.
\endchorus

\beginverse
\[C#m] Ci poni come \[B/D#]luce sopra un \[E]mon\[A/C#]te:
\[F#m] in noi l'umani\[E/B]tà vedrà il tuo \[B4]vol\[B]to. \[E/G] 
\[A] Ti testimonie\[B/A]remo fra le \[E/G#]gen\[A]ti:
\[F#m] in noi l'umani\[E/B]tà vedrà il tuo \[B4]volto, Spirito, vieni.
\endverse

\beginverse
\chordsoff
Cammini accanto a noi lungo la strada,
si realizzi in noi la tua missione.
Attingeremo forza dal tuo cuore,
si realizzi in noi la tua missione, Spirito, vieni.
\endverse

\beginverse
\chordsoff
Come sigillo posto sul tuo cuore,
ci custodisci, Dio, nel tuo amore.
Hai dato la tua vita per salvarci,
ci custodisci, Dio, nel tuo amore, Spirito, vieni.
\endverse

\beginverse
\chordsoff
Dissiperai le tenebre del male,
esulterà in te la creazione.
Vivremo al tuo cospetto in eterno,
esulterà in te la creazione, Spirito, vieni.
\endverse

\beginverse
\chordsoff
Vergine del silenzio e della fede
l'Eterno ha posto in te la sua dimora.
Il tuo ``sì'' risuonerà per sempre:
l'Eterno ha posto in te la sua dimora, Spirito, vieni.
\endverse

\beginverse
\chordsoff
Tu nella Santa Casa accogli il dono,
sei tu la porta che ci apre il Cielo
Con te la Chiesa canta la sua lode,
sei tu la porta che ci apre il Cielo, Spirito, vieni.
\endverse

\beginverse
\chordsoff
Tu nella brezza parli al nostro cuore:
ascolteremo, Dio, la tua parola;
ci chiami a condividere il tuo amore:
ascolteremo, Dio, la tua parola, Spirito, vieni.
\endverse
\endsong


%titolo{Luce in noi}
%autore{Buttazzo, Beltrami}
%album{Esultiamo nel Signore}
%tonalita{Sol}
%gruppo{}
%momenti{Parola;Lode}
%identificatore{luce_in_noi}
%data_revisione{2011_12_31}
%trascrittore{Francesco Endrici - Manuel Toniato}
\beginsong{Luce in noi}[by={Buttazzo, Beltrami}]

\ifchorded
\beginverse*
\vspace*{-0.8\versesep}
{\nolyrics \[G] \[Am] \[D] \[G] }
\vspace*{-\versesep}
\endverse
\fi
\beginchorus
\[G]Luce in noi sa\[Am]rà \[D]questa tua Parola, Si\[G]gnore,
\[G]e ci \[Em]guide\[Am]rà \[D]con sapienza e veri\[G]tà.
\endchorus
\beginverse
\memorize
\[C] Beato \[D]l'uomo che a\[G]scolte\[Em]rà
\[C] la tua Pa\[D/C]rola Si\[G]gnore: 
\[F/G] nella tua \[G]legge cam\[C]mina già 
e con\[Am]forme al tuo cuore vi\[Am/D]vrà. \[C/D]  \[D] 
\endverse

\beginverse
^ Tu hai par^lato a ^noi, Si^gnore, 
^ per rive^larci la ^via; 
^ e siano ^scritti nei ^nostri cuori 
i tuoi ^giusti precetti d'a^more. ^^
\endverse

\beginverse
^ Ti lode^rò con sin^ceri^tà 
^ perché ho fi^ducia in ^te, 
^ e segui^rò la tua ^volontà 
perché ^so che mi ami, Si^gnore. ^\[E]
\endverse

\beginchorus
\[A]Luce in noi sa\[Bm]rà \[E]questa tua Parola, Si\[A]gnore,
\[A]e ci \[F#m]guide\[Bm]rà \[E]con sapienza e veri\[A]tà. \[A] \[Bm]  \[E]  \[A] 
\[A]Luce in \[F#m]noi sa\[Bm]rà \[E]con sapienza e veri\[A]tà.
\endchorus

\endsong


%titolo{Lui verrà e ti salverà}
%autore{Fitts, Sadler}
%album{Eterna è la sua misericordia}
%tonalita{Mi}
%gruppo{}
%momenti{Avvento;Quaresima;Congedo}
%identificatore{lui_verra_e_ti_salvera}
%data_revisione{2011_12_31}
%trascrittore{Francesco Endrici - Manuel Toniato}
\beginsong{Lui verrà e ti salverà}[by={Fitts, Sadler}]

\ifchorded
\beginverse*
\vspace*{-0.8\versesep}
{\nolyrics \[E/G#] \[A] \[B] \[E] }
\vspace*{-\versesep}
\endverse
\fi
\beginverse
\[E]A chi \[B]è nell'an\[C#m]goscia \[A]tu \[B]dirai: \[E] \brk ``Non \[B]devi te\[A]{me}\[B]re,
\[E] il tuo Si\[B]gnore è \[C#m]qui, con la \[A]for\[B]za su\[E]a.
Quando in\[B]vochi il suo \[A]no\[B]me  \[A] Lui \[B4]ti sa\[B]lve\[E]rà.''
\endverse

\beginchorus
\[E]Lui verrà e ti {\[A]\[B]sal}ve\[E]rà, \brk \[E]Dio verrà e ti {\[A]\[B]sal}ve\[C#m]rà,
di' a chi è smar\[A]rito che certo Lui \[B4]torne\[B]rà. 
Lui verrà e ti {\[A]\[B]sal}ve\[E]rà,
\[E]Lui verrà e ti {\[A]\[B]sal}ve\[E]rà, \brk \[E]Dio verrà e ti {\[A]\[B]sal}ve\[C#m]rà,
alza i tuoi \[A]occhi a lui, presto ri\[B4]torne\[B]rà, 
lui verrà e ti {\[A]\[B]sal}ve\[E]rà.
\endchorus

\beginverse
\chordsoff
A chi ha il cuore ferito tu dirai: ``Confida in Dio,
il tuo Signore è qui,  con il suo grande amore.
Quando invochi il suo nome Lui ti salverà.''
\endverse

\beginverse*
\itshape \[C#m]Egli è rifugio nelle av\[B/D#]versità, \[E]dalla tempesta ti ri\[F#m7]parerà.
\[E/G#]È il tuo baluardo e ti di\[A]fenderà, \brk la \[F#m7]forza sua Lui ti da\[B4]rà. \[B] 
\endverse
\ifchorded
\beginverse*
\vspace*{-\versesep}
{\nolyrics \[C4] \[C] \[F/A] }
\vspace*{-\versesep}
\endverse
\fi
\beginchorus
\[F] Lui verrà e ti {\[B&]\[C]sal}ve\[F]rà, \brk \[F]Dio verrà e ti {\[B&]\[C]sal}ve\[Dm]rà,
di' a chi è smar\[B&]rito che certo Lui \[C4]torne\[C]rà. 
Lui verrà e ti {\[B&]\[C]sal}ve\[F]rà,
\[F] Lui verrà e ti {\[B&]\[C]sal}ve\[F]rà, \brk \[F]Dio verrà e ti {\[B&]\[C]sal}ve\[Dm]rà,
alza i tuoi \[B&]occhi a lui, presto ri\[C4]torne\[C]rà, 
lui verrà e ti {\[B&]\[C]sal}ve\[F]rà.
\endchorus
\endsong

\lettera
%titolo{Ma il vostro posto è là}
%autore{Chieffo}
%album{La tenda}
%tonalita{Mi-}
%gruppo{}
%momenti{}
%identificatore{ma_il_vostro_posto_e_la}
%data_revisione{2011_12_31}
%trascrittore{Francesco Endrici}
\beginsong{Ma il vostro posto è là}[by={Chieffo},ititle={Signore com'è bello}, ititle={La tenda}]
\beginverse
Si\[Em]gnore, come è \[D]bello, non andiamo \[Em]via,
faremo delle \[D]tende e dormiremo \[Em]qua,
non scendiamo a \[D]valle dove l'altra \[Em]gente
non vuole ca\[D]pire quello che tu \[Em]sei.
\endverse
\chordsoff
\beginchorus
Ma il vostro posto è là, là in mezzo a loro,
l'amore che vi ho dato portatelo nel mondo.
Io sono venuto a salvarvi dalla morte,
Il Padre mi ha mandato e io mando voi.
\endchorus
\beginverse
Adesso che capite cos'è la vera gioia
Volete stare soli e non pensare a loro.
A cosa servirà l'amore che vi ho dato
Se la vostra vita da soli vivrete.
\endverse
\beginchorus
\chordsoff 
Rit. 
\endchorus
\beginverse
Quando vi ho chiamati eravate come loro,
duri di cuore, tardi a capire.
Ciò che io dicevo non lo sentivate:
“È pazzo --- si diceva --- non sa quello che dice”.
\endverse
\beginchorus
\chordsoff 
Rit. 
\endchorus
\beginverse
Scendete nella valle vivete nel mio amore,
da questo capiranno che siete miei fratelli.
Parlategli di me arriveranno al Padre,
se li saprete amare la strada troveranno.
\endverse
\beginchorus
Ma il nostro posto è là, là in mezzo a loro,
l'amore che ci ha dato portiamolo nel mondo.
Lui è venuto a salvarci dalla morte,
Il Padre l'ha mandato e Lui manda noi.
\endchorus
\endsong


%titolo{Madonna nera}
%autore{Bagniewski}
%album{In attesa dell'alba}
%tonalita{Sol}
%gruppo{}
%momenti{Maria}
%identificatore{madonna_nera}
%data_revisione{2011_12_31}
%trascrittore{Francesco Endrici}
\beginsong{Madonna nera}[by={Bagniewski},ititle={Alla madonna di Czestochowa}]
\beginverse
C'è una \[G]terra silen\[G7]ziosa dove o\[C]gnuno vuol tor\[E]nare
una \[Am]terra un dolce \[A7]volto
con due \[D]segni di vio\[D7]lenza
Sguardo in\[G]tenso e premu\[G7]roso
che ti \[C]chiede di affi\[Am]dare
la tua \[D]vita e il tuo \[D7]mondo in \[C]mano a \[G]Lei.
\endverse
\beginchorus
Ma\[G]donna, Madonna \[C]Nera, è \[C]dolce esser tuo \[G]figlio! \[D7]
Oh \[G]lascia, Madonna \[C]Nera, ch'io \[D]viva vicino a \[G]Te.
\endchorus
\beginverse
\chordsoff
Lei ti calma e rasserena, Lei ti libera dal male,
perché sempre ha un cuore grande \brk per ciascuno dei suoi figli.
Lei t'illumina il cammino \brk se le offri un po' d'amore,
se ogni giorno parlerai a Lei così.
\endverse
\beginverse
\chordsoff
Questo mondo in subbuglio \brk cosa all'uomo potrà offrire?
Solo il volto di una Madre \brk pace vera può donare.
Nel suo sguardo noi cerchiamo \brk quel sorriso del Signore
che ridesta un po' di bene in fondo al cuor.
\endverse
\endsong

%titolo{Madre del cielo}
%autore{Hermana Glenda}
%album{}
%tonalita{Sol}
%gruppo{}
%momenti{Maria}
%identificatore{madre_del_cielo}
%data_revisione{2011_12_31}
%trascrittore{Francesco Endrici}
\beginsong{Madre del cielo}[by={Hermana\ Glenda}]
\ifchorded
\beginverse*
\vspace*{-0.8\versesep}
{\nolyrics \[G]\[D] \[C] \[D]}
\vspace*{-\versesep}
\endverse
\fi
\beginverse
\memorize
\[G]Madre, \[G] dono di vita \[C]nuova, \[C]
Madre del \[G]cielo, \[G]madre dell'umani\[D]tà. \[D]
\[G]Madre, \[G] piena di \[C]luce, 
il\[C]lumina la nostra \[Em]vita, ri\[D]mani accanto a \[G]noi. \[D]
\endverse
\beginchorus
Da te \[G]viene la \[D]gioia, da te \[Em]viene la spe\[C]ranza, 
dolce \[G]Madre, \[Em] del figlio di \[D]Dio. \[D]
Tu re\[G]gina della \[D]pace, tu mi\[Em]stero di sal\[C]vezza, 
veglia \[G]sempre, \[D] su questi figli \[C]tuoi. \[G] \[D] \[D]
\endchorus
\beginverse
^Madre, ^ tu che sei piena di ^grazia  ^
Dio ti ha pre^scelta, ^il Signore è con ^te. ^
Ma^ria, ^ tu sei bene^detta ^
non ci abbando^nare, ri^mani accanto a ^noi. ^
\endverse
\beginchorus
Da te \[G]viene la \[D]gioia, da te \[Em]viene la spe\[C]ranza, 
dolce \[G]Madre, \[Em] del figlio di \[D]Dio. \[D]
Tu re\[G]gina della \[D]pace, tu mi\[Em]stero di sal\[C]vezza, 
veglia \[G]sempre, \[D] su questi figli \[C]tuoi, \[C]
veglia \[G]sempre, \[D] su questi figli \[C]tuoi. \[G] \[C] \[G] \[C] \[G]
\endchorus

\beginverse
Se^ñor, ^ toma mi vida ^nueva ^
Antes de que la e^spera ^
Desgaste años en ^mí. ^
E^stoy ^ dispuesto a lo que quie^ras ^
No importa lo que ^sea 
Tú ^llámame a ser^vir. ^
\endverse
\beginchorus
Lléva\[G]me donde los \[D]hombres nece\[Em]siten tus pa\[C]labras
nece\[G]siten \[Em] mis ganas de vi\[D]vir. \[D]
Donde \[G]falte la espe\[D]ranza donde \[Em]falte la ale\[C]gria
simple\[G]mente \[D] por no saber de \[C]tí. \[C]
\endchorus

\beginverse
Te ^doy, ^ mi corazón sin^cero ^
Para gritar sin ^miedo ^
Tu grandeza Se^ñor. ^
Ten^dré ^ mis manos sin can^sancio ^
Tú historia entre mis ^labios
Y ^fuerza en la ora^ción. ^
\endverse
\beginchorus
Lléva\[G]me donde los \[D]hombres nece\[Em]siten tus pa\[C]labras
nece\[G]siten \[Em] mis ganas de vi\[D]vir. \[D]
Donde \[G]falte la espe\[D]ranza donde \[Em]falte la ale\[C]gria
simple\[G]mente \[D] por no saber de \[C]tí. \[C]
\endchorus

\beginverse
Y a^sí, ^ en marcha iré can^tando ^
Por pueblos predi^cando ^
Lo bello que es tu a^mor ^
Se^ñor ^ tengo alma misio^nera ^
Condúceme a la ^tierra 
Que ^tenga sed de ^Dios. ^
\endverse
\beginchorus
Lléva\[G]me donde los \[D]hombres nece\[Em]siten tus pa\[C]labras
nece\[G]siten \[Em] mis ganas de vi\[D]vir. \[D]
Donde \[G]falte la espe\[D]ranza donde \[Em]falte la ale\[C]gria
simple\[G]mente \[D] por no saber de \[C]tí. \[C]
\endchorus
\beginchorus
Lléva\[G]me donde los \[D]hombres nece\[Em]siten tus pa\[C]labras
nece\[G]siten \[Em] mis ganas de vi\[D]vir. \[D]
Donde \[G]falte la espe\[D]ranza donde \[Em]falte la ale\[C]gria
simple\[G]mente \[D] por no saber de \[C]tí. \[C] \[G]
\endchorus
\endsong

%titolo{Madre della speranza}
%autore{Bizzeti}
%album{Vorrei amare}
%tonalita{Re}
%gruppo{}
%momenti{Maria}
%identificatore{madre_della_speranza}
%data_revisione{2011_12_31}
%trascrittore{Francesco Endrici}
\beginsong{Madre della speranza}[by={Bizzeti}]
\beginchorus
\[D]Madre \[G]della spe\[D]ranza, \[G]veglia sul \[A]nostro cam\[D]mino
\[G]guida i nostri \[A]passi \[Bm]verso il Figlio tuo, Ma\[Em]ri\[A]a.
\[D]Regina \[G]della \[D]pace, \[G]proteggi il \[A]nostro \[D]mondo \[D7]
\[G]prega per \[A]questa umani\[D]tà, Ma\[Bm]ria
\[G]Madre del\[A]la spe\[Bm]ranza, \[G]Madre del\[A]la spe\[D]ranza. \[F#]
\endchorus
\beginverse
\[Bm]Docile \[Em]serva del \[Bm]Padre, (Maria)
\[A]Piena di Spirito \[Bm]Santo, (Maria)
\[G]umile \[A]vergine \[D]madre del \[Em]Figlio di \[Bm]\[Em]Di\[F#]o! 
\[Bm]Tu sei la \[Em]piena di \[Bm]grazia, (Tutta bella sei)
\[A]scelta fra tutte le \[Bm]donne, (non c'è ombra in te)
\[G]Madre di \[A]Miseri\[D]cordia, \[Em]porta del \[Bm]\[Em]cie\[F#]lo.
\endverse
\beginverse
%\chordsoff
^Noi che cre^diamo alla ^vita, (Maria)
^noi che crediamo all'a^more, (Maria)
^sotto il tuo ^sguardo met^tiamo il ^nostro do^^ma^ni.
^Quando la ^strada è più ^dura, (ricorriamo a te)
^quando più buia è la ^notte, (veglia su di noi)
^stella del ^giorno ri^splendi sul ^nostro sen^^tie^ro.
\endverse
\endsong

%titolo{Madre io vorrei}
%autore{Sequeri}
%album{E mi sorprende}
%tonalita{Re-}
%gruppo{}
%momenti{Maria}
%identificatore{madre_io_vorrei}
%data_revisione{2011_12_31}
%trascrittore{Francesco Endrici}
\beginsong{Madre io vorrei}[by={Sequeri}]
\beginverse
\[Dm7]Io vorrei tanto parlare con te
di quel \[Gm7]Figlio che a\[Gm6]mavi:
\[C7]io vorrei tanto ascoltare da te
quello \[F7+]che pen\[A7/4]savi. \[A7]
\[Dm7]Quando hai udito che tu non saresti
più \[Gm7]stata \[Gm6]tua \[C7]e questo Figlio
che non aspettavi non \[F7+]era per \[A7]te.
\endverse
\beginchorus
\[Dm7]Ave Ma\[Gm]ria, \[C7]Ave Ma\[F7+]ria, \[F6]
\[Dm7]Ave Ma\[Gm]ria, \[C7]Ave Ma\[A7]ria.
\endchorus
\beginverse
\chordsoff
Io vorrei tanto sapere da te
se quand'era bambino 
tu gli hai spiegato che cosa sarebbe 
successo di Lui
e quante volte anche tu, di nascosto,
piangevi, Madre, quando sentivi
che presto l'avrebbero ucciso per noi.
\endverse
\beginverse
\chordsoff
Io ti ringrazio per questo silenzio
che resta tra noi 
io benedico il coraggio divivere
sola con Lui ora capisco che
fin da quei giorni pensavi a noi
per ogni figlio dell'uomo che muore ti prego così.
\endverse
\endsong



%titolo{Magnifica il Signore}
%autore{Amadei}
%album{La nostra festa è Cristo}
%tonalita{Re}
%gruppo{}
%momenti{Magnificat}
%identificatore{magnifica_il_signore}
%data_revisione{2011_12_31}
%trascrittore{Francesco Endrici - Manuel Toniato}
\beginsong{Magnifica il Signore}[by={Amadei}]

\beginchorus
Ma\[D]gnifica il Si\[F#m]gnore anima \[Bm]mia; 
il mio \[Em]spirito \[A7] esulta in \[D]Dio.
Alle\[D]luia, alle\[F#m]luia, alle\[Bm]luia, alle\[Em7]luia,\[A7] alle\[D]luia.
\endchorus

\beginverse
Perché ha guar\[D]dato l'umiltà della sua \[D7]serva:
ecco \[G]ora mi \[A]chiameran be\[D7]ata
perché il Po\[G]tente mi ha \[A]fatto grandi \[D]cose
e \[F#m]santo è il suo \[Bm]nome. Alle\[Em7]luia, \[A7] alle\[D]luia!
\endverse

\beginverse
\chordsoff
Ha spiegato la potenza del suo braccio
ha disperso i superbi di cuore,
i potenti rovescia dai troni
e innalza gli umili e li ricolma di ogni bene.
\endverse

\beginverse
\chordsoff
Il suo servo Israele egli solleva
ricordando la sua misericordia
promessa ad Abramo e ai nostri padri
e a tutti i suoi figli, perché santo è il suo nome.
\endverse
\endsong


%titolo{Magnificat}
%autore{Castaldi}
%album{}
%tonalita{Sol}
%gruppo{}
%momenti{Magnificat}
%identificatore{magnificat_castaldi}
%data_revisione{2011_12_31}
%trascrittore{Francesco Endrici}
\beginsong{Magnificat}[by={Castaldi}]
\transpose{-2}
\ifchorded
\beginverse*
\vspace*{-0.8\versesep}
{\nolyrics \[Em]\[C]\[G]\[D]\[Em]}
\vspace*{-\versesep}
\endverse
\fi
\beginverse
\memorize
La voce degli \[C]ultimi \[G] magnifi\[D]ca il Si\[Em]gnore,
sa esul\[C]tare solo in \[G]Dio. \[D] \[Em]
L'umiltà che \[C]lui ha guardato \[G] \brk l'ha chia\[D]mata be\[Em]ata,
lui l'ha \[C]scelta, fatta \[G]sua. \[D] \[C]
E grandi cose ha fatto in \[D]noi.
\endverse
\beginchorus
Magnifi\[Em]cat! Magnifi\[C]cat!
Cante\[G]remo il suo nome al \[D]cielo di questa cit\[Em]tà!
(Magnificat!) Magnifi\[C]cat!
Questa \[G]notte cantiamo il suo \[D]nome la sua fedel\[Em]tà!
(Magnificat!) Magnifi\[C]cat!
Sa\[G]remo un popolo u\[D]nito, la sua eredi\[Em]tà!
(Magnificat!) Magnifi\[C]cat!
Questa \[G]notte cantiamo il suo \[D]nome\dots Magnifi\[Em]cat! 
\[C]\[G]\[D]\[Em]
\endchorus
\beginverse
\chordsoff
La Sua misericordia si è stesa da sempre
sopra quanti sono in Dio,
la potenza del Suo braccio il Signore l'ha
spiegata per la nostra libertà.
E Lui ci ha fatto liberi.
\endverse
\beginverse
\chordsoff
I superbi li ha umiliati nei loro stessi
pensieri, nelle loro ideologie.
I potenti e i loro troni il Signore ha
rovesciato, ha innalzato gli umili
Lui l'ha detto e lo farà.
\endverse
\beginverse
\chordsoff
Ha soccorso gli affamati, Lui gli ha reso
giustizia, Lui gli ha dato dignità.
I ricchi a mani vuote lui li ha fatti tornare
alla loro vanità.
Era promessa ora è realtà.
\endverse
\endsong

%titolo{Magnificat}
%autore{Gragnani, Casini, Ciardella}
%album{}
%tonalita{Sol}
%gruppo{}
%momenti{Magnificat}
%identificatore{magnificat_dio_ha_fatto}
%data_revisione{2011_12_31}
%trascrittore{Francesco Endrici}
\beginsong{Magnificat}[by={Gragnani, Casini, Ciardella}]
\beginverse
\[G]Dio \[C7+] ha fatto in \[Bm7]me cose \[Em7]grandi.
\[C7+]Lui \[Am7] che guarda l'\[D]umile \[G7+]servo
e di\[C7+]sperde i su\[Am6]perbi
nell'or\[B7]goglio del \[Em]cuore. \[E]
\endverse
\beginchorus
\[Am]\[D7]L'anima \[G7+]mia \[C7+] esulta in \[Am]Dio \[B7]
mio salva\[Em]tore. \[E]
\[Am]\[D7]L'anima \[G7+]mia \[C7+] esulta in \[Am]Dio \[B7]
mio salva\[Em]tore. \[C]
La sua sal\[Am]vezza \[D]cante\[G]rò.
\endchorus
\beginverse
%\chordsoff
^Lui, ^ Onnipo^tente e ^santo.
^Lui ^ abbatte i ^grandi dai ^troni 
e sol^leva dal ^fango il suo ^umile ^servo. ^
\endverse
\beginverse
%\chordsoff
^Lui ^ miseri^cordia infi^nita,
^Lui ^ che rende ^povero il ^ricco 
e ri^colma di ^beni chi si af^fida al suo a^more. ^
\endverse
\beginverse
%\chordsoff
^Lui, ^ Amore ^sempre fe^dele.
^Lui ^ guida il suo ^servo Isra^ele 
e ri^corda il suo ^patto stabi^lito per ^sempre. ^
\endverse
\endsong

%titolo{Magnificat}
%autore{Frisina}
%album{Non temere}
%tonalita{Do}
%gruppo{}
%momenti{Magnificat}
%identificatore{magnificat_frisina}
%data_revisione{2011_12_31}
%trascrittore{Francesco Endrici}
\beginsong{Magnificat }[by={Frisina}]
\beginchorus
\[C]L'anima \[7+]mia ma\[F]gnifi\[G]ca, ma\[Am]gnifica il Si\[G]gnore
\[Dm]e il mio spirito e\[Am]sulta in \[F]Dio mio salva\[G]tore.
\endchorus
\beginverse*
\[Dm]Per\[G7]ché ha guar\[Am]dato l'umil\[F]tà della sua \[C]serva,
\[F]tutte le \[Dm]genera\[Am]zioni mi \[F]chiame\[D]ranno be\[G]ata.
\endverse
\beginchorus
\chordsoff
Cose grandiose ha compiuto in me l'Onnipotente,
cose grandi ha compiuto colui il cui nome è santo.
\endchorus
\beginverse*
\chordsoff
Di età in età è la sua misericordia,
la sua misericordia si stende su chi lo teme.
\endverse
\beginchorus
\chordsoff
Ha spiegato con forza la potenza del suo braccio,
i superbi ha disperso nei pensieri del loro cuore.
\endchorus
\beginverse*
\chordsoff
Ha rovesciato i potenti dai loro troni,
mentre ha innalzato, ha innalzato gli umili.
\endverse
\beginchorus
\chordsoff
Gli affamati ha ricolmato dei suoi beni,
mentre ha rimandato i ricchi a mani vuote.
\endchorus
\beginverse*
\chordsoff
Egli ha sollevato Israele, il suo servo,
ricordandosi della sua grande misericordia.
\endverse
\beginchorus
\chordsoff
Secondo la sua promessa fatta ai nostri padri,
la promessa ad Abramo e alla sua discendenza.
\endchorus
\beginverse*
\chordsoff
A Te sia gloria padre che dai la salvezza,
gloria al Figlio amato e allo Spirito Santo.
\endverse

\beginverse*
\[Dm]A\[Am]men, Alle\[F]lu\[C]ia \[F]A\[Am]men, Alle\[G]lu\[C]ia
\endverse
\endsong

%titolo{Mani}
%autore{Colombo}
%album{}
%tonalita{Re}
%gruppo{}
%momenti{}
%identificatore{mani_colombo}
%data_revisione{2011_12_31}
%trascrittore{Francesco Endrici}
\beginsong{Mani}[by={Colombo}]
\beginverse
Vor\[D]rei che le pa\[G]role mu\[A]tassero in pre\[D]ghiera
e \[G]rivederti o \[Bm]Padre che \[G]dipingevi il \[A]cielo.
Sa\[D]pessi quante \[G]volte guar\[A]dando questo \[D]mondo
vor\[G]rei che tu tor\[Bm]nassi a \[G]ritoc\[A]carne il \[D]cuore.
Vor\[Bm]rei che le mie \[F#m]mani a\[G]vessero la \[D]forza
per \[G]sostenere chi \[E]non può cammi\[A]nare.
Vor\[Bm]rei che questo \[F#m]cuore che e\[G]splode in senti\[D]menti
\[G]diventasse \[Bm]culla per \[G]chi non ha più \[A]madre.
\endverse
\beginchorus
\[D]Mani, prendi queste mie \[A]mani,
fanne vita, fanne a\[G]more
braccia aperte per ri\[Bm]ceve\[A]re chi è solo.
\[D]Cuore, prendi questo mio \[A]cuore,
fa' che si spalanchi al \[G]mondo
germogliando per quegli \[Bm]occhi
che non \[A]sanno pianger \[G]\[(A)]più.
\endchorus
\beginverse
\chordsoff
Sei tu lo spazio che desidero da sempre,
so che mi stringerai e mi terrai la mano.
Fa' che le mie strade si perdano nel buio
ed io cammini dove cammineresti Tu.
Tu soffio della vita prendi la mia giovinezza
con le contraddizioni e le falsità.
Strumento fa che sia per annunciare il Regno
a chi per queste vie Tu chiami Beati.
\endverse
\beginchorus
\chordsoff 
Rit. 
\endchorus
\beginverse
\chordsoff
Noi giovani di un mondo che cancella i sentimenti
e inscatola le forze nell'asfalto di città.
Siamo stanchi di guardare 
siamo stanchi di gridare
ci hai chiamati siamo Tuoi cammineremo insieme.
\endverse
\beginchorus
\transpose{2}
\[D]Mani, prendi queste mie \[A]mani,
fanne vita, fanne a\[G]more
braccia aperte per ri\[Bm]ceve\[A]re chi è solo.
\[D]Cuore, prendi questo mio \[A]cuore,
fa' che si spalanchi al \[G]mondo
germogliando per quegli \[Bm]occhi
che non \[A]sanno pianger \[G]più.
\endchorus
\endsong

%titolo{Mani}
%autore{Goia}
%album{Spalancate le porte a Cristo}
%tonalita{Do}
%gruppo{}
%momenti{}
%identificatore{mani_goia}
%data_revisione{2011_12_31}
%trascrittore{Francesco Endrici}
\beginsong{Mani}[by={Goia}]
\beginverse
Il \[C]sole scende \[F] è quasi notte or\[C]mai, \[C7]
dai re\[F]stiamo ancora insieme un po' \[G]
meno buio sa\[C]rà. \[C7]
La pa\[F]rola del Si\[G]gnore, come \[Em7]luce in mezzo a \[Am]noi
ci ri\[D7]schiara \[F7+] e ci unisce a \[G4]Lui. \[G]
\endverse
\beginchorus
\[C]Mani \[F]che si stringono \[C]forte,
\[F]in un cerchio di \[C]sguardi \[F]che s'incrocia\[G]no.
Come un ab\[Am]braccio stret\[F]to,
per sen\[G]tire che la \[C7+]chiesa vive,
che \[F7+]vive dentro \[B&]noi, insieme a \[G4]noi. \[G]
E sono \[C]mani \[F]che si tendono in \[C]alto,
\[F]che si aprono \[C]grandi \[F]per raccoglie\[G4]re. \[G]
Quella \[Am]forza immensa, \[F]
che il Si\[G]gnore mette in \[C7+]fondo al cuore,
e annun\[F7+]ciare a tutti \[Dm]che Dio ci \[G4]ama. \[G]
\endchorus
\beginverse
\chordsoff
E veglieremo con le lampade,
aspettando nella notte fin che giorno sarà;
e la voce del Signore all'improvviso giungerà,
saremo pronti, saremo amici suoi.
\endverse
\beginchorus
\dots Dio ci a\[F]\[G]\[C]ma.
\endchorus
\endsong

%titolo{Maranathà}
%autore{Frisina}
%album{Non temere}
%tonalita{Fa}
%gruppo{}
%momenti{Avvento}
%identificatore{maranatha_frisina}
%data_revisione{2011_12_31}
%trascrittore{Francesco Endrici}
\beginsong{Maranathà}[by={Frisina}]
\beginchorus
Marana\[C]thà, \[Dm] marana\[B&]thà, \[F]vieni, \[B&]vieni Si\[C]gnore Ge\[F]sù. \rep{2}
\endchorus
\beginverse
Il mondo at\[B&]tende la \[C]luce del tuo \[F]vol\[Dm]to,
le sue \[Gm]strade son \[F]solo oscuri\[C]tà.
Rischiara i \[Dm]cuori \[B&] di chi ti \[F]cerca, \[B&]
di chi è in cam\[Dm]mino \[Gm]incontro a \[C]te.
\endverse
\beginverse
\chordsoff
Vieni per l'uomo che cerca la sua strada,
per chi soffre, per chi non ama più.
Per chi non spera, per chi è perduto,
e trova il buio attorno a sé.
\endverse
\beginverse
\chordsoff
Tu ti sei fatto compagno nel cammino
ci conduci nel buio assieme a Te,
tu pellegrino sei per amore,
mentre cammini accanto a noi.
\endverse
\endsong



%titolo{Maria}
%autore{Pecoraro, De Filippis}
%album{Voglio vedere il tuo volto}
%tonalita{La}
%gruppo{}
%momenti{Maria}
%identificatore{maria_rns_402}
%data_revisione{2011_12_31}
%trascrittore{Francesco Endrici - Manuel Toniato}
\beginsong{Maria}[by={Pecoraro, De\ Filippis}]
\ifchorded
\beginverse*
\vspace*{-0.8\versesep}
{\nolyrics \[A] \[E/G#] \[D/F#] \[C#/F] \[F#m] \[C#/E] \[D7+] \[E] }
\vspace*{-\versesep}
\endverse
\fi

\beginverse
\[A]Ora che \[Bm]sei qui, \[C#m]  
la \[D]tua dolcezza \[F#m]sento dentro \[C#m/E]me.
\[D7+]  Co\[C#7]me l'aurora \[F#m]sor\[C#m/E]gi, 
ri\[D7+]{splen}\[A/C#]di fra \[Bm]noi.\[C#] 
\[A]Ora che \[Bm]sei qui, \[C#m] 
tu \[D]figlia e madre, \[F#m]fra le braccia \[C#m/E]tue
\[D7+]  Di\ch{[C#7}{f}{f}{ff}ondi la tua \[F#m]{lu}\[C#m/E]ce
ce\[D7+]le\[C#]ste di \[F#m]pa\[A/C#]ce e di a\[D7+]mor. \[E]
\endverse

\beginchorus
\[A]Tu Ma\[E/G#]ria, 
tu \[D/F#]sei di\[C#]mora per Ge\[F#m]sù figlio \[C#m]tuo. \[D7+] 
Re\[C#]gina che dal \[F#m]cie\[A/E]lo 
spe\[D7+]ran\[C#7]za e a\[F#m]mo\[A]re ci \[D7+]da\[E]i.
\endchorus

\beginverse
\[C]Ora che \[Dm]sei qui, \[Em] 
tu \[F]dolce madre \[Am]sono figlio \[Em/G]tuo,
\[F7+]  tra\[E7]sforma il mio \[Am]pian\[Em]to 
nel \[F7+]can\[E7]to d'a\[Am]mo\[C/G]re per \[F7+]te. \[G]         
\endverse

\beginchorus
\[C]Tu Ma\[G/B#]ria, 
tu \[F/A]sei di\[E]mora per Ge\[Am]sù figlio \[Em]tuo. \[F7+] 
Re\[E7]gina che dal \[Am]cie\[C/B]lo 
spe\[F7+]ran\[E7]za e a\[Am]mo\[C/G]re ci \[F7+]da\[G]i.
\endchorus

\ifchorded
\beginverse*
\vspace*{-\versesep}
{\nolyrics \[A]  \[Bm]  \[C#m]  \[D] \[F#m]  \[C#m/E]  
\[D7+]  \[C#7]  \[F#m]  \[C#m/E]  
\[D7+]  \[C#7]  \[F#m7]  \[A/E]  \[D7+]  \[E]}
\vspace*{-\versesep}
\endverse
\fi

\beginverse
\[A]Ora che \[Bm]sei qui, \[C#m]  
in\[D]segnami ad a\[F#m]mare come \[C#m/E]sai,
\[D7+]  ra\[C#7]diosa fra le  \[F#m]stel\[C#m/E]le, 
pre\[D]ghiera \[Bm/E]ora \[A]sei.
\endverse
\endsong



%titolo{Messa dei fanciulli}
%autore{}
%album{}
%tonalita{Sol}
%gruppo{}
%momenti{Consacrazione}
%identificatore{messa_dei_fanciulli}
%data_revisione{2011_12_31}
%trascrittore{Francesco Endrici}
\beginsong{Messa dei fanciulli}
\beginverse
\[G]Gloria a Te Si\[C]gnore che ci vuoi \[D]bene.
\endverse
\beginverse
\[F]È il Si\[C]gnore Ge\[F]sù, si \[B&]offre per \[C]no\[Dm]i.
\endverse
\beginverse
\[C]Un cuor solo, un'\[F]anima sola, \brk \[G]per la Tua \[F]gloria, Si\[G]gnore.
\endverse
\endsong

%titolo{Mi basta la tua grazia}
%autore{Sanfratello, Cucuzza}
%album{Parola d'amore}
%tonalita{Sol}
%gruppo{}
%momenti{Congedo}
%identificatore{mi_basta_la_tua_grazia}
%data_revisione{2011_12_31}
%trascrittore{Francesco Endrici - Manuel Toniato}
\beginsong{Mi basta la tua grazia}[by={Sanfratello, Cucuzza}]

\ifchorded
\beginverse*
\vspace*{-0.8\versesep}
{\nolyrics \[G]  \[Am7]  \[C2]  \[G] }
\vspace*{-\versesep}
\endverse
\fi
\beginverse
\[G]Quando sono debole, al\[Am]lora sono forte
per\[C2]ché tu sei la mia \[G]forza.
\[G]Quando sono triste è in \[Am]te che trovo gioia
per\[C2]ché tu sei la mia \[G]gioia.
\[D]Ge\[Em]sù io confido in \[C]Te,
Ge\[G]sù mi basta la tua \[G]gra\[D]zia.
\endverse

\beginchorus
\[C]Sei la mia \[G]forza, la mia sal\[C2]vezza,
sei la mia \[Em]pace, sicuro ri\[D4]fugio.
\[C]Nella tua \[G]grazia voglio re\[C2]stare,
Santo Si\[Em]gnore, sempre con \[D4]Te. \[D] 
\endchorus
\ifchorded
\beginverse*
\vspace*{-\versesep}
{\nolyrics \[C2]  \[D]   \[G]  \[Am]  \[C2]  \[G] }
\endverse
\fi

\beginverse
\chordsoff
Quando sono povero, allora sono ricco
perché sei la mia ricchezza.
Quando so malato, è in te che trovo vita
perché tu sei guarigione. 
Gesù io confido in te,
Gesù mi basta la tua grazia.
\endverse

\beginverse*
\chordsoff
Quando sono debole, allora sono forte
perché tu sei la mia forza.
\endverse
\endsong



%titolo{Mi pensamiento}
%autore{Com. de Dios}
%album{RnS 209}
%tonalita{Sol}
%gruppo{}
%momenti{}
%identificatore{mi_pensamiento}
%data_revisione{2011_12_31}
%trascrittore{Francesco Endrici}
\beginsong{Mi pensamiento}[by={Com.\ de\ Dios}]
\beginverse
\[G] Mi pensamiento eres \[Am]tu, Señor \[D7]
mi pensamiento eres \[G]tu, Señor \[Em]
Mi pensamiento eres \[Am]tu, Señor \[D7]
mi pensamiento eres \[G]tu. \[E7]\rep{2}
\endverse
\beginchorus
Porque \[Am]tu me has dado la \[D7]vida,
Porque \[G]tu me has \[B7]dado el exi\[Em]stir,
Poque \[Am]tu me has dado ca\[D7]riño,
me has dado a\[G]mor. \[C]\[G]
\endchorus
\beginverse
\chordsoff
Mi alegria eres tu, Señor\dots
\endverse
\beginverse
\chordsoff
Mi fortaleza eres tu Señor\dots
\endverse
\endsong

%titolo{Mia forza e mio canto}
%autore{Comi}
%album{Sulle orme d'Israele}
%tonalita{Mi-}
%gruppo{}
%momenti{Pasqua;Salmi}
%identificatore{mia_forza_e_mio_canto_comi}
%data_revisione{2011_12_31}
%trascrittore{Francesco Endrici}
\beginsong{Mia forza e mio canto}[by={Comi}]
\ifchorded
\beginverse*
\vspace*{-0.8\versesep}
{\nolyrics \[Em]\[C]\[D]\[G]}
\vspace*{-\versesep}
\endverse
\fi
\beginchorus
Mia \[Em]forza e mio canto \[C]è il Si\[D]gno\[G]re, 
d'Isra\[C]ele in eterno \[Am]è il Salva\[D]to\[Em]re.
\endchorus
\beginverse
\memorize
Voglio can\[Em]tare in onore del Si\[Am]gnore 
perché \[D]lui è il mio Salva\[Em]tore. 
È il mio \[Em]Dio, lo voglio lo\[Am]dare, 
è il \[D]Dio di mio padre, lo \[C]voglio \[Am]esal\[B7]tare.
\endverse
\beginverse
\chordsoff
Il suo nome è “il Signore”, 
la sua destra è ricolma di potenza. 
La sua destra annienta il nemico 
e lo schiaccia con vittoria infinita.
\endverse
\beginverse
\chordsoff
Il faraone in cuor suo diceva, 
li inseguirò e li raggiungerò. 
Ma col tuo soffio alzasti le acque 
perché il tuo popolo attraversasse il mare.
\endverse
\beginverse
\chordsoff
Soffiasti ancora e il mare ricoprì 
il faraone e il suo potere. 
Cavalli e carri e tutti i cavalieri 
furono sommersi nel profondo del mare.
\endverse
\beginverse
\chordsoff
Chi è come te, o Signore?
Chi è come te fra gli dei?
Sei maestoso, Signore, e santo.
Tremendo nelle imprese, operatore di prodigi.
\endverse
\beginverse
\chordsoff
Hai guidato il tuo popolo nel deserto,
il popolo che tu hai riscattato.
Lo conducesti con forza, o Signore,
e con amore alla tua santa dimora.
\endverse
\endsong



%titolo{Mio Signore}
%autore{}
%album{}
%tonalita{Sol}
%gruppo{}
%momenti{}
%identificatore{mio_signore}
%data_revisione{2011_12_31}
%trascrittore{Francesco Endrici}
\beginsong{Mio Signore}
\beginverse
\[G]Mio Signore, ricordati di me,
o \[C]mio Signore, ri\[G]cordati di me,
Mio Signore, ri\[Bm]cordati di \[Em]me,
non la\[Am7]sciarmi \[D7]solo quag\[G]giù! \[C]\[G]
\endverse
\beginverse
\chordsoff
Mio Signore, sei qui, rimani in me,
o mio Signore, sei qui, rimani in me,
mio Signore, sei qui, rimani in me,
la mia gioia vera sei tu.
\endverse
\beginverse
\chordsoff
Vieni, Signore, a vivere con me,
o vieni Signore, a vivere con me,
vieni, Signore, a vivere con me,
ch'io mi senta vivo per te.
\endverse
\endsong

%titolo{Mistero della fede}
%autore{Bonfitto}
%album{Sei grande nell'amore}
%tonalita{Re}
%gruppo{}
%momenti{Mistero della fede}
%identificatore{mistero_della_fede_bonfitto}
%data_revisione{2011_12_31}
%trascrittore{Francesco Endrici}
\beginsong{Mistero della fede}[by={Bonfitto}]
\beginverse
Mi\[D]stero della \[A]Fede.
Annun\[D]ciamo la Tua \[Em]morte Si\[Bm]gnore, \[G]
procla\[D]miamo la \[G]Tua risurre\[A]zione, \[F#m]
nell'at\[Bm]tesa della \[G]Tua ve\[D]nuta.
\endverse
\endsong

%titolo{Annunciamo la tua morte}
%autore{Giombini}
%album{Camminiamo nella speranza}
%tonalita{Do}
%gruppo{Anamnesi}
%momenti{Anamnesi}
%identificatore{mistero_della_fede_giombini}
%data_revisione{2011_12_31}
%trascrittore{Francesco Endrici - Manuel Toniato}
\beginsong{Annunciamo la tua morte}[by={Giombini}]

\beginverse*
{\itshape \[C]Mistero \[F]della \[G]fede}
Annun\[C]ciamo la Tua \[G7]morte Si\[Am]gnore,
\[C]procla\[Em]miamo la \[F]Tua Risurre\[G4]zione \[G7] 
nell'at\[Am]tesa \[D]della Tua ve\[Em]nuta \[Am] 
nell'at\[F]tesa \[G7]della Tua ve\[C]\[F]nu\[C]ta.
\endverse
\endsong

%titolo{Annunciamo la tua morte}
%autore{Rossi}
%album{}
%tonalita{Sol}
%gruppo{Anamnesi}
%momenti{Anamnesi}
%identificatore{mistero_della_fede_rossi}
%data_revisione{2011_12_31}
%trascrittore{Francesco Endrici - Manuel Toniato}
\beginsong{Annunciamo la tua morte}[by={Rossi}]

\beginverse*
{\itshape \[G]Mistero della fede.}
Annun\[G]cia\[D]mo la Tua \[C]morte Si\[D]gnore.
Procla\[G]mia\[D]mo la Tua \[C]Risurre\[D]zione
nell'at\[Em]tesa \[C]della Tua ve\[G4]nuta. \[G] 
\endverse
\endsong

%titolo{Musica di festa}
%autore{Sands, Jotti}
%album{}
%tonalita{Mi-}
%gruppo{}
%momenti{Congedo}
%identificatore{musica_di_festa}
%data_revisione{2011_12_31}
%trascrittore{Francesco Endrici}
\beginsong{Musica di festa}[by={Sands, Jotti}]
\beginverse
\[Em]Cantate al Si\[Bm]gnore un \[Em]cantico \[Bm]nuovo,
\[Em]splende la sua \[Bm]glori\[Em]a! \[Bm]
\[Em]Grande è la sua \[Bm]forza, \[Em]grande la sua \[Bm]pace,
\[C]grande la Sua \[D]Santi\[Em]tà! \[E]
\endverse
\beginchorus
In \[Am]tutta la \[D7]terra, \[G7]popoli del \[C7+]mondo,
gri\[Am6]date la \[B]sua \[Em]fe\[D]del\[G]tà! \[E7]
\[Am]Musica di \[D7]festa, \[G7+]musica di \[C7+]lode,
\[Am6]Musica di \[B7]liber\[Em]tà!
\endchorus
\beginverse
\chordsoff
Agli occhi del mondo ha manifestato
la Sua salvezza!
Per questo si canti, per questo si danzi,
per questo si celebri!
\endverse
\beginverse
\chordsoff
Con l'arpa ed il corno, con timpani e flauti,
con tutta la voce!
Canti di dolcezza, canti di salvezza,
canti d'immortalità!
\endverse
\beginverse
\chordsoff
I fiumi e i monti, battono le mani
davanti al Signore!
La sua giustizia giudica la terra
giudica le genti!
\endverse
\beginverse
\chordsoff
Al Dio che ci salva, gloria in eterno
Amen! Alleluja!
Gloria a Dio Padre, gloria a Dio Figlio,
gloria a Dio Spirito!
\endverse
\endsong

\lettera
%titolo{Nada te turbe}
%autore{Taizè}
%album{}
%tonalita{La-}
%gruppo{}
%momenti{}
%identificatore{nada_te_turbe}
%data_revisione{2011_12_31}
%trascrittore{Francesco Endrici}
\beginsong{Nada te turbe}
\beginverse
\[Am]Nada te \[Dm7]turbe, \[G]nada te e\[C]spante,
\[F]quien a Dios \[D-6]tiene \[E]nada la \[Am]falta.
\[Am]Nada te \[Dm7]turbe, \[G]nada te e\[C]spante,
\[F]solo \[Dm6]Dios \[E]ba\[Am]sta.
\endverse
\endsong

%titolo{Nascerà}
%autore{Gen\ Rosso}
%album{In concerto per la pace}
%tonalita{Re}
%gruppo{}
%momenti{}
%identificatore{nascera}
%data_revisione{2011_12_31}
%trascrittore{Francesco Endrici - Manuel Toniato}
\beginsong{Nascerà}[by={Gen\ Rosso}]
\beginverse
\[D] Non c'è al mondo chi mi ami, 
\[D] non c'è stato mai nessuno
in fondo alla mia \[A]vita, \[G7+]come te.
\[D] È con te la mia partita; 
come sabbia fra le dita
scorrono i miei \[A]giorni in\[G7+]sieme a te.

\[C6/9] Inquietudine, o malinconia:
non c'è posto \[G]per loro \[D]in casa \[A]mia.
\[C6/9] Sempre nuovo è il tuo modo
di inventare il \[G]gioco del \[D]tempo per \[Bm]me.
\endverse

\ifchorded
\beginverse*
\vspace*{-\versesep}
{\nolyrics \[Bm] \[A] \[Bm] \[A] }
\vspace*{-\versesep}
\endverse
\fi
\beginchorus
\[D]Nasce\[A]rà \[Bm] dentro \[G]me, 
\[D] sul si\[A]lenzio che \[Bm]abita \[G]qui,
\[D]fiori\[A]rà \[Bm] un canto \[G]che 
\[D] mai nes\[A]suno ha can\[Bm]tato per \[G]te.
\endchorus

\ifchorded
\beginverse*
\vspace*{-\versesep}
{\nolyrics \[D]  \[A4] \[A] }
\endverse
\fi

\beginverse
\chordsoff
Se la strada si fa dura, 
come posso aver paura?
Nel buio della notte ci sei Tu.
Se mi assale la fatica 
di cancellare la sconfitta,
dietro ogni ferita sei ancora Tu.

È una cosa che non mi spiego mai:
cosa ho tatto perché Tu scegliessi me?
Cosa mai dirò quando mi vedrai,
quando dai confini del mondo verrai?
\endverse

\beginchorus
\chordsoff
Nascerà dentro me, 
sul silenzio che abita qui,
fiorirà un canto che 
mai nessuno ha cantato per te.
\endchorus

\ifchorded
\beginverse*
\vspace*{-\versesep}
{\nolyrics \[D]  \[C]  \[G] \[D]  \[C] }
\vspace*{-\versesep}
\endverse
\fi
\beginchorus
\[G]Nasce\[D]rà \[Em] dentro \[C]me, 
\[G] sul si\[D]lenzio che \[Em]abita \[C]qui,
\[G]fiori\[D]rà \[Em] un canto \[C]che 
\[G] mai nes\[D]suno ha can\[Em]tato per \[C]te.
\endchorus
\endsong

%titolo{Nel canto di te}
%autore{Cavallin}
%album{}
%tonalita{Mi-}
%gruppo{}
%momenti{Comunione;Adorazione}
%identificatore{nel_canto_di_te}
%data_revisione{2011_12_31}
%trascrittore{Francesco Endrici - Manuel Toniato}
\beginsong{Nel canto di te}[by={Cavallin}]

\ifchorded
\beginverse*
\vspace*{-0.8\versesep}
{\nolyrics \[D]  \[Em7]  \[F#m7] }
\vspace*{-\versesep}
\endverse
\fi
\beginverse
\[Em7] Davanti al mi\[D]stero \[Em7] svelato per \[F#m7]noi 
\[Em7] sei tu che ci in\[D]contri \[F#m]dentro al si\[Bm7]lenzio 
nei giorni pas\[G]sati, nel cuori ormai \[D/A]stanchi, 
sei pane di \[G]vita che toglie il ran\[D/F#]core   
tu apri una \[G]strada e \[A/C#]scendi tra \[Bm7]noi, 
Signore Ge\[Em7]{sù.} \[A] 
\endverse
                        
\beginverse
\chordsoff
Abbiamo perduto la forza di amare 
perfino il tuo volto ci sembra lontano  
le mani non hanno più nulla da dare 
sei tu la ricchezza in questo deserto 
noi poveri in viaggio veniamo da te, 
Signore Gesù.
\endverse

\beginchorus 
\[F] Tu abiti \[C/E]terre dove noi cammi\[Dm]niamo 
dovunque sa\[Am]remo tu ci sa\[B&]rai 
proteggi i tuoi \[F/A]figli e sogna di \[Gm]noi. \[C]
\[D/F#] Nel canto di \[G]Te trove\[A]remo la \[Bm7]vita 
perché il nostro \[G]vivere è \[A]solo per \[D/F#]Te. 
Tu abbracci ogni cosa, Signore Ge\[A]sù.
\endchorus
 
 \beginverse
\chordsoff
Agnello di Dio, Amore indifeso 
Tu parli nel cuore di chi non ha pace 
ridoni la vita a chi l'ha perduta 
ci fai camminare sopra ogni morte 
accendi quel fuoco che è amore per Te, 
Signore Gesù 
\endverse
                        
\beginverse
\chordsoff
Tu solo conosci da dove veniamo 
per paura dell'altro scappiamo da te  
nel buio  profondo ci vieni a trovare, 
di benedizioni ci riempi le mani 
con misericordia ti curvi su noi; 
Signore Gesù 
\endverse
\endsong

%titolo{Nel tuo silenzio}
%autore{Gen Verde, Gen Rosso}
%album{Come fuoco vivo}
%tonalita{Re}
%gruppo{}
%momenti{Comunione}
%identificatore{nel_tuo_silenzio}
%data_revisione{2011_12_31}
%trascrittore{Francesco Endrici}
\beginsong{Nel tuo silenzio}[by={Gen\ Verde, Gen\ Rosso}]
\ifchorded
\beginverse*
\vspace*{-0.8\versesep}
{\nolyrics \[D]\[G]\[A]\[D]\[Bm]\[Em]\[A7]\[D]}
\vspace*{-\versesep}
\endverse
\fi
\beginverse
\memorize
\[D]Nel tuo si\[G]lenzio ac\[A]colgo il mi\[D]stero
\[Bm7]venuto a \[Em7]vivere \[A7]dentro di \[D]me.
\[D]Sei tu che \[G]vieni, o \[A]forse è più \[D]vero
\[Bm7]che tu mi ac\[Em7]cogli in \[G]te, Ge\[D]sù. \[D]
\endverse
\beginverse
%\chordsoff
^Sorgente ^viva che ^nasce nel ^cuore
^è questo ^dono che ^abita in ^me.
^La tua pre^senza è un ^fuoco d'a^more
^che avvolge ^l'anima ^mia, Ge^sù. ^
\endverse
\beginverse
\[F]Ora il tuo \[B&7]Spirito in \[C]me dice: “\[F]Padre”,
\[F]non sono \[B&7+]io a par\[C7]lare, sei \[Cm7]tu.
\[F7]Nell'infi\[B&]nito o\[C7]ceano di \[F]pace
Tu vivi in \[B&7+]me, io in \[B&m]te, Ge\[F]sù.
\endverse
\endsong

%titolo{Nel tuo silenzio}
%autore{Cavallin}
%album{}
%tonalita{Fa}
%gruppo{}
%momenti{Adorazione;Comunione}
%identificatore{nel_tuo_silenzio_cavallin}
%data_revisione{2011_12_31}
%trascrittore{Francesco Endrici - Manuel Toniato}
\beginsong{Nel tuo silenzio}[by={Cavallin}]

\beginverse
\[F] Nel tuo silen\[Am]zio \[B&] ti doni a \[C]noi
\[Dm] Pane di \[Am]vita, \[B&] Cristo Ge\[C]sù.
\[Dm] Amore e\[Am]terno \[B&] che strappi \[Am]via
\[B&] la morte e il \[Dm]male \[E&] dai figli \[C]tuoi.
\endverse

\beginverse
\chordsoff
Di amore riempi il grembo e il cuore 
senza difese di fronte a noi
Tu Dio e Signore Pane che dai
la vita piena la gioia in noi.
\endverse

\beginchorus
\[F]Io sono con voi \[F7+]tutti i giorni del mondo.           
\[B&]Io sarò con voi non vi \[C]lascerò \[F]soli.
\[Dm]Vita sarò che tra\[G]sforma anche voi,
\[B&]questo mio \[C]corpo è la mia \[Am]vita per \[Dm]voi
\[B&]pane spez\[F]zato \[B&]dono di \[C]vita per \[F]voi. 
\endchorus

\beginverse
\chordsoff
Nel fuoco ardente troviamo Te
stupiti e scalzi guardiamo a Te.
Tu parli e dici ``Sono per voi
la mano forte che libera.''
\endverse

\beginverse
\chordsoff
Volto nascosto, Presenza oscura
lungo la notte ci stringi a te
lotti e colpisci chi scappa via
Tu ci guarisci ferendoci.
\endverse

\beginverse
\chordsoff
Dal sonno vinti lungo la via
stanchi di andare verso di Te
per noi prepari Pane dal cielo,
la forza ancora per credere.
\endverse

\beginverse
\chordsoff
Lungo la strada ti accosti a noi
Cristo Risorto\ldots{} Dio con noi
la tua Parola spezzata in noi
diventa pane: resta con noi!
\endverse
\endsong



%titolo{Nella Tua presenza}
%autore{Ricci}
%album{La Tua dimore}
%tonalita{Re}
%gruppo{}
%momenti{Comunione}
%identificatore{nella_tua_presenza}
%data_revisione{2011_12_31}
%trascrittore{Francesco Endrici}
\beginsong{Nella Tua presenza}[by={Ricci}]
\ifchorded
\beginverse*
\vspace*{-0.8\versesep}
{\nolyrics \[D]\[F#m]\[D]\[F#m]}
\vspace*{-\versesep}
\endverse
\fi
\beginverse
\memorize
\[D]Nella tua presenza avvolti da \[F#m]Te,
\[D]nella tua dimora insieme con \[F#m]Te,
con la vita Tua che \[G]sboccia
nell'anima, in \[F#m]noi,
con la linfa tua, la \[G]stessa,
in ciascuno di \[Bm]no\[A]i.
\endverse
\beginverse
^Eccoci fratelli, parte di ^Te,
^eccoci famiglia, una sola con ^Te,
che risorto dai la ^vita che non muore ^mai,
che risorto dentro al ^cuore
accendi il tuo ^cie^lo. \[F]
\endverse
\beginchorus
Come il Padre che ha mandato me
pos\[C]siede la vita in sé \[E&] e come grazie al Padre,
grazie a Lui, io \[Gm]vivo \[Dm]
così colui, così colui che \[Am]mangia di me
vi\[G]vrà grazie a me, lui vivrà, vivrà per \[A]me. \[A]
\endchorus
\beginverse
\chordsoff
Tu che ci hai mostrato il Padre, Gesù,
tu che hai dato un nome perfino al dolore,
ora tu ci dai te stesso e ci dai l'unità,
ci spalanchi la tua casa dove abita il cielo.
\endverse
\beginverse
\chordsoff
Nella tua dimora insieme con Te,
nella tua presenza avvolti da Te,
con la vita Tua che sboccia
nell'anima, in noi,
con la linfa tua, la stessa, in ciascuno di noi.
\endverse
\beginchorus
\chordsoff 
Rit. 
\endchorus
\ifchorded
\beginverse*
\vspace*{-\versesep}
{\nolyrics \[D]\[D]\[F#m]\[D]}
\endverse
\fi
\endsong


%titolo{Noi crediamo in te}
%autore{Branca, Ciancio}
%album{Sentieri di speranza}
%tonalita{Lab}
%gruppo{}
%momenti{Ingresso;Congedo}
%identificatore{noi_crediamo_in_te_rns}
%data_revisione{2011_12_31}
%trascrittore{Francesco Endrici - Manuel Toniato}
\beginsong{Noi crediamo in te}[by={Branca, Ciancio}]

\beginverse
\[Ab]Gesù nostro Si\[B&7]gnore, \[Ab]tu sei qui \[B&7]mezzo a noi,
\[E&]allontana il ti\[B&]more e mostraci il \[E&]tuo grande a\[F]mor.
\[Ab]Solo tu sei la \[B&7]vita, \[Ab]tu sei via e \[B&7]verità,
\[E&]resta qui insieme a \[B&]noi e nulla \[E&/G]più manche\[F]rà.
\endverse

\beginchorus
\[C]Noi cre\[F]diamo in \[Am]Te Ge\[G]sù, 
\[C]Tu sei \[F]l'unico \[Am]Salva\[G]tor.
\[Dm7]chi crede in Te grandi \[Am]cose farà, 
\[F]ed il tuo volto ve\[G4]drà.  \[G] 
\[C]Noi cre\[F]diamo in \[Am]Te Ge\[G]sù 
\[C]noi spe\[F]riamo nel \[Am]nome \[G]tuo.
\[Dm7]Tu sei nel Padre e il \[Am]Padre è in Te,
\[F]Tu sei Signor, \[G4]sei il {\[C]\[F]\[Am]\[G]Re,}  il \[C]Re.
\endchorus

\ifchorded
\beginverse*
\vspace*{-\versesep}
{\nolyrics \[F] \[Am] \[G] \[Dm7] \[Am] \[F] \[G4] \[G] }
\endverse
\fi

\beginverse
\chordsoff
Soli non ci hai lasciati, il tuo Spirito è in noi, 
egli è il consolatore e nulla più mancherà.
Noi ti amiamo Signore, la tua legge seguiam,
tu ci doni la pace e nulla più mancherà.
\endverse
\endsong

%titolo{Noi siamo gente di festa}
%autore{A.C.R.}
%album{}
%tonalita{Sol}
%gruppo{}
%momenti{}
%identificatore{noi_siamo_gente_di_festa}
%data_revisione{2011_12_31}
%trascrittore{Francesco Endrici - Manuel Toniato}
\beginsong{Noi siamo gente di festa}[by={A.C.R.}]
\ifchorded
\beginverse*
\vspace*{-0.8\versesep}
{\nolyrics \[G] \[C] \[G] }
\vspace*{-\versesep}
\endverse
\fi
\beginchorus
\[G]Noi siamo \[C]gente di \[G]festa, noi, 
siamo \[C]gente di \[G]gioia, noi,
e \[A]camminiamo insieme \brk sulle \[D]strade della \[D7]vita. \rep{2}
\endchorus

\beginverse
\[G]Anche (\textit{anche}) \[Em]quando (\textit{quando}) 
\[A]niente sembra par\[D]lare;
\[G]Anche (\textit{anche}) \[Em]quando (\textit{quando}) 
\[A]non si vede un sor\[D]riso,
E \[G]tu (\textit{e tu}) vor\[Am]resti scap\[D]pare, 
vor\[Em]resti vo\[Bm]lare, vor\[C]resti an\[D]dar via (\textit{via, via, via\ldots})
\endverse

\beginverse
\chordsoff
Questa (\textit{questa}) vita (\textit{vita}) 
è nelle tue mani;
Vivi (\textit{vivi}) ora (\textit{ora}) 
la speranza e la gioia;
E tu (\textit{e tu}) trasforma la tua noia, 
dona il tuo futuro, \brk canta insieme a noi (\textit{noi, noi, noi\ldots})
\endverse

\beginverse
\chordsoff
Per le (\textit{strade}) per le (\textit{piazze}) 
giriamo tutti insieme
Canta (\textit{grida}) corri (\textit{vieni}) 
insieme a noi;
E noi (\textit{con te}) ci metteremo a danzare, 
diventeremo più amici, \brk saremo sempre di più (\textit{u, u, u\ldots}).
\endverse
\endsong


%titolo{Noi veglieremo}
%autore{Machetta}
%album{}
%tonalita{Re}
%gruppo{}
%momenti{Avvento}
%identificatore{noi_veglieremo}
%data_revisione{2011_12_31}
%trascrittore{Francesco Endrici - Manuel Toniato}
\beginsong{Noi veglieremo}[by={Machetta}]

\beginchorus
Nella \[D]notte, o \[G]Dio, \[Em7] noi \[A]veglie\[D]remo,
con le \[Bm]lampade, vestiti a \[F#m]festa: \[B]  
presto \[Em]arri\[F#m]verai \[G6]  e \[A]sarà \[D]giorno.
\endchorus

\beginverse
\[Bm]Ralle\[Em]gratevi in at\[A]tesa del Si\[D7+]gnore:
improv\[Bm]visa giungerà la sua \[Em7]voce. \[A9]
Quando \[Gm]Lui verrà, sarete \[D7+]pronti
e vi \[Em]chiamerà "\[G]amici" per \[F#]sempre. \[A7]  
\endverse

\beginverse
\chordsoff
Raccogliete per il giorno della vita
dove tutto sarà giovane in eterno.
Quando lui verrà sarete pronti
e vi chiamerà amici per sempre.
\endverse
\endsong

%titolo{Noi veniamo a Te}
%autore{Buttazzo}
%album{Vita nuova con Te}
%tonalita{Fa}
%gruppo{}
%momenti{Ingresso}
%identificatore{noi_veniamo_a_te}
%data_revisione{2011_12_31}
%trascrittore{Francesco Endrici - Manuel Toniato}
\beginsong{Noi veniamo a Te}[by={Buttazzo}]

\ifchorded
\beginverse*
\vspace*{-0.8\versesep}
{\nolyrics \[F] \[C/E] \[Dm] \[B&] \[F/A] \[B&] \[B&/C] \[Gm7/C] }
\vspace*{-\versesep}
\endverse
\fi
\beginchorus
Noi ve\[F]niamo a \[C]te, ti se\[Dm]guiamo Si\[B&]gnor, 
solo \[F]Tu hai pa\[Gm]role di \[B&]vi\[Gm/C]ta
e ri\[F]nasce\[C/E]rà dall'in\[Dm]contro con \[B&]Te 
una \[F]nuova u\[B&]mani\[F]tà.
\endchorus

\beginverse
\[Gm]Tu, ma\[C]estro degli \[F]uomi\[Dm7]ni, \brk \[Gm]tu ci \[C]chiami all'a\[F]scolto \[E&] \[F] 
\[B&]e rin\[A7]novi con \[Dm7]noi \brk l'alle\[G7]anza d'amore infi\[Gm7]nito.   \[Gm7/C] 
\endverse

\beginverse
\chordsoff
Tu, speranza degli uomini, \brk ti ci apri alla vita
e rinnovi per noi \brk la promessa del mondo futuro.
\endverse

\beginverse
\chordsoff
Tu, amico degli uomini, \brk tu ci chiami fratelli
e rivivi con noi \brk l'avventura di un nuovo cammino.
\endverse

\ifchorded
\beginverse*
\vspace*{-\versesep}
{\nolyrics \[F] \[C/E] \[Dm] \[Gm] \[F] }
\endverse
\fi
\endsong

%titolo{Non temere}
%autore{Frisina}
%album{Non temere}
%tonalita{Do}
%gruppo{}
%momenti{Maria}
%identificatore{non_temere}
%data_revisione{2011_12_31}
%trascrittore{Francesco Endrici - Manuel Toniato}
\beginsong{Non temere}[by={Frisina}]

\ifchorded
\beginverse*
\vspace*{-0.8\versesep}
{\nolyrics \[C] \[G] \[Am] \[G] }
\vspace*{-\versesep}
\endverse
\fi

\beginverse
\[C]Non te\[G]mere, Ma\[Am]ri\[Am7]a, 
\[F]perché hai trovato \[Em]gra\[Am]zia
\[Dm]presso il \[G]tuo Si\[E]gno\[Am]re, 
\[Dm7]che si dona a \[G4]te. \[G] 
\endverse

\beginchorus
Apri il \[Dm7]cuo\[E]re, non te\[Am]me\[F]re, 
\[C]Egli sa\[G]rà con \[C]te. \[Ab7] 
\endchorus

\beginverse
\[D&]Non te\[Ab]mere, A\[B&m9]bra\[B&m]mo 
\[G&]la tua \[E&m]debo\[Fm]lez\[B&m]za:
\[G&]padre di un \[Ab]nuovo \[F]popo\[B&m]lo 
\[E&m7]nella fede sa\[Ab4]rai. \[Ab] 
\endverse

\beginchorus
Apri il \[E&m7]cuo\[F]re, non te\[B&m]me\[G&]re, 
\[D&]Egli sa\[Ab]rà con \[D&]te. \[A7] 
\endchorus

\beginverse
\[D]Non te\[A]mere, Mo\[Bm]sè, \[Bm7]  
\[G]se tu non \[Em]sai par\[F#m]la\[Bm]re,
\[G]perché la \[A]voce \[F#]del Si\[Bm]gnore 
\[Em7]parlerà per \[A4]te. \[A] 
\endverse

\beginchorus
Apri il \[Em]cuo\[F#]re, non te\[Bm]me\[G]re, 
\[D]Egli sa\[A]rà con \[D]te. \[B&7] 
\endchorus

\beginverse
\[E&]Non te\[B&]mere, Giu\[Cm]sep\[Cm7]pe, 
di \[Ab]prende\[Fm]re Ma\[Gm]ri\[Cm]a,
\[Ab]perché in \[B&]lei Dio \[G]compi\[Cm]rà
\[Fm7]il mistero d'A\[B&4]mo\[B&]re.
\endverse

\beginchorus
Apri il \[Fm]cuo\[G]re, non te\[Cm]me\[Ab]re,
\[E&]Egli sa\[B&]rà con \[E&]te. \[B7] 
\endchorus

\beginverse
\[E]Pietro, \[B]no, non te\[C#m]me\[C#m7]re,
\[A]se il Si\[F#m]gnore ha \[G#m]scel\[C#m]to
\[A]la tua \[B]fede \[G#]povera, 
\[F#m]per convincere il \[B4]mon\[B]do.
\endverse

\beginchorus
Apri il \[F#m]cuo\[G#]re, non te\[C#m]me\[A]re, 
\[E]Egli sa\[B]rà con \[E]te. \[A7+]  \[A6]  \[E] 
\endchorus
\endsong


%titolo{Nostra Signora della speranza}
%autore{Spoladore}
%album{Dacci pace}
%tonalita{Do}
%gruppo{}
%momenti{Maria}
%identificatore{nostra_signora_della_speranza}
%data_revisione{2011_12_31}
%trascrittore{Francesco Endrici - Manuel Toniato}
\beginsong{Nostra Signora della speranza}[by={Spoladore}]

\beginverse
Nostra Si\[C]gnora della Spe\[Dm]ranza
noi ti invo\[G7]chiamo o nostra \[C]Madre
Tu sei la \[Am]stella, Stella del \[Em]mare
nella tem\[F]pesta di questa \[C]vita
splendi su \[B&]noi, splendi su \[F]noi come il \[G4]sole \[G] 
l'oscura \[F]nuvo\[G]la del \[C]male
che ci \[B&]copre mente e \[F]cuore
scaccia \[G4]via o Ma\[G]ria.
\endverse

\beginverse
\chordsoff
Nostra Signora della Speranza
ricolma il vuoto del nostro cuore
con la tua presenza e il tuo calore
o Mamma nostra non ci lasciare
risplenda nuova in questo mondo la Tua Pace
portaci al Figlio Tuo Gesù
e tra le braccia di nostro Padre
nella luce dell'amore.
\endverse
\endsong


%titolo{Notte di luce}
%autore{Rainoldi, Akepsimas}
%album{}
%tonalita{Fa}
%gruppo{}
%momenti{Natale}
%identificatore{notte_di_luce}
%data_revisione{2011_12_31}
%trascrittore{Francesco Endrici}[by={Rainoldi, Akepsimas}]
\beginsong{Notte di luce}
\beginverse
\[F]Notte di \[C4]lu\[C7]ce, colma è l'at\[A4]te\[A7]sa!
\[B&]Notte di spe\[F]ranza: \[Dm7]vie\[G]ni Ge\[C]sù!
\[Gm]Ver\[C7]bo del \[Dm]Padre, \[G7]ve\[A7]sti il si\[Dm]lenzio.
\endverse
\beginchorus
\[F7]Sia  \[B&]glo\[C7]ria nei \[F]cieli, sia \[C]pa\[Dm]ce quag\[C]giù!
\[F7]Sia  \[B&]glo\[C7]ria nei \[Dm]cieli, sia \[Gm7]pa\[C]ce quag\[F]giù!
\endchorus
\beginverse
%\chordsoff
^Giorno d'a^mo^re, nuova alle^an^za!
^Giorno di sal^vezza: ^vie^ni, Ge^sù!
^Spo^so fe^dele, ^ve^sti la ^carne.
\endverse
\beginverse
%\chordsoff
^Alba di ^pa^ce, Regno che ir^rom^pe!
^Alba di per^dono: ^vie^ni, Ge^sù!
^San^to di ^Dio, ^ve^sti il pec^cato.
\endverse
\endsong



\lettera
%titolo{Oggi è nato}
%autore{Ricci}
%album{Venne nel mondo}
%tonalita{Do}
%gruppo{}
%momenti{Natale}
%identificatore{oggi_e_nato}
%data_revisione{2011_12_31}
%trascrittore{Francesco Endrici - Manuel Toniato}
\beginsong{Oggi è nato}[by={Ricci}]


\beginverse*
\itshape \[Am]Oggi per noi è nato il Salvatore, 
tutta la terra canta a lui, benedice lui.
\endverse

\beginverse*
\[F] E ogni \[G]giorno si \[Am]leva un canto, 
\[F] canto \[G]nuovo di \[Am]gioia e vita.
\[F] E ogni \[G]giorno si \[Am]narra la gloria 
ed osan\[F]nano i popoli che oggi è \[F]na\[G]to.
\endverse

\beginchorus
\[C]Tutto il mondo \[F]canta \[Am] alle\[G]luia.
\[C]Anche il cielo \[F]canta \[Am]alle\[G]luia.
\[C] Canta la \[F]terra,\[Am] canta il \[G]mare
e nei \[F]campi le \[G]messi ri\[F]spondono agli \[G]alberi. \rep{2}
\endchorus

\beginverse*
\chordsoff
\itshape Oggi per noi è nato il Salvatore.
Tutta la terra canta a lui, benedice lui.
\endverse

\beginverse*
\chordsoff
E ogni giorno si leva un canto,
canto nuovo di gioia e vita.
e ogni giorno si narra la gloria,
ed osannano i popoli che oggi è nato.
\endverse

\beginchorus
\chordsoff
Tutto il mondo canta alleluia.
Anche il cielo canta alleluia.
Canta la terra, canta il mare
e nei campi le messi rispondono agli alberi. \rep{2}
\endchorus
\beginchorus
\[C]   \[F]   \[Am]   Alle\[G]luia. \[C]   \[F]   \[Am]   Alle\[G]luia
\endchorus
\ifchorded
\beginverse*
\vspace*{-\versesep}
{\nolyrics \[C]  \[F]   \[Am]   \[G]  \[F]   \[F]   \[G]  \[F] \[G]  }
\endverse
\fi
\beginchorus
\[C]Tutto il mondo \[F]canta \[Am] alle\[G]luia.
\[C]Anche il cielo \[F]canta \[Am]alle\[G]luia.
\[C] Canta la \[F]terra,\[Am] canta il \[G]mare
Alle\[F]luia, alle\[G]luia, alle\[F]luia, alle\[G]lu\[C]ia. \[Am]\[C] 
\endchorus
\endsong

%titolo{Oggi ti chiamo}
%autore{Fanelli}
%album{Canterò felicità}
%tonalita{Do}
%gruppo{}
%momenti{Ingresso;Vocazione}
%identificatore{oggi_ti_chiamo}
%data_revisione{2011_12_31}
%trascrittore{Francesco Endrici}
\beginsong{Oggi ti chiamo}[by={Fanelli}]
\beginverse
\[C]Oggi ti chiamo alla \[B&]vita,
ti invito a se\[Am]guirmi, a venire con \[C]me.
Apri i tuoi occhi, il tuo \[B&]cuore,
dimentica \[Am]tutto e segui \[C]me.
E \[Am]non avere più pa\[F]ura di la\[G]sciare quel che \[C]hai
il \[Am]senso vero della \[F]vita trove\[G]rai. \[7]
\endverse
\beginchorus
Segui\[C]rò la Tua Pa\[F]rola, mio Si\[C]gnore io ver\[F]rò
con la \[Am]mano nella \[F]Tua sempre \[Dm]io cammine\[G]rò;
dammi \[C]oggi la Tua \[Dm]forza ed il Tuo a\[G]more.
Cante\[C]rò canzoni \[F]nuove, cante\[C]rò felici\[F]tà,
ed il \[Am]fuoco del Tuo a\[F]more nel mio \[Dm]mondo porte\[G]rò,
cante\[C]rò che solo \[Dm]tu sei liber\[G]tà.
\endchorus
\beginverse
%\chordsoff
^Oggi ti chiamo all'a^more,
ti invito a se^guirmi, a stare con ^me.
Apri le porte del ^cuore,
allarga i con^fini del dono di ^te.
Ac^cogli tutti nella ^pace con fi^ducia e veri^tà:
l'a^more vero ti da^rà la liber^tà. ^
\endverse
\beginverse
%\chordsoff
^Oggi ti chiamo alla ^gioia,
ti invito a se^guirmi, a venire con ^me.
Sai quanto vale un sor^riso:
può dare spe^ranza a chi non ne ^ha.
La ^gioia è segno della ^vita che ri^nasce dentro ^te
e an^nuncia ad ogni uomo ^pace e liber^tà. ^
\endverse
\endsong

%titolo{Ogni mia parola}
%autore{Gen Verde}
%album{È bello lodarti}
%tonalita{Do}
%gruppo{}
%momenti{Acclamazioni al Vangelo}
%identificatore{ogni_mia_parola}
%data_revisione{2011_12_31}
%trascrittore{Francesco Endrici}
\beginsong{Ogni mia parola}[by={Gen\ Verde}]
\beginverse*
\[C]Come la \[G]pioggia e la \[C]ne\[G]ve
\[C]scendono \[F]giù dal \[G]cielo
e \[Am]non vi ri\[G]tornano \[F]senza irri\[G]gare
e \[F]far germo\[G]gliare la \[F]ter\[G]ra,
\[C]così ogni mia Pa\[G]rola non ri\[C]tornerà a \[G]me
\[C]senza operare \[F]quanto de\[G]sidero,
\[Am]senza aver compiuto \[Em]ciò per cui l'a\[F]vevo man\[C]data.
\[F]Ogni mia Pa\[G]rola, \[F]ogni mia Pa\[G]rola \[C]
\endverse
\endsong

%titolo{Ogni volta che mangiamo}
%autore{}
%album{}
%tonalita{Do}
%gruppo{Anamnesi}
%momenti{Anamnesi}
%identificatore{ogni_volta_che_mangiamo}
%data_revisione{2011_12_31}
%trascrittore{Francesco Endrici - Manuel Toniato}
\beginsong{Ogni volta che mangiamo}
\beginverse*
\[C]Ogni volta che man\[F]giamo di questo \[C]pane \[C7] 
e be\[Dm]viamo a questo \[C]calice \[F] 
annun\[B&]ciamo la tua \[Gm]morte Si\[A]gnore \[Dm] 
nell'at\[B&]tesa \[Gm]della tua ve\[B&]nu\[F]ta.
\endverse
\endsong

%titolo{Oh, oh canterò}
%autore{A.C.R.}
%album{}
%tonalita{Re}
%gruppo{}
%momenti{}
%identificatore{oh_oh_cantero}
%data_revisione{2011_12_31}
%trascrittore{Francesco Endrici - Manuel Toniato}
\beginsong{Oh, oh canterò}[by={A.C.R.}]

\beginverse
\[D]Santa è la spe\[Bm]ranza di rag\[Em]giungere il \[A7]cielo:
Fra\[D]tello \[G]spera con \[D]me. \rep{2} \[G]  \[A] 
\endverse

\beginchorus
\[G]Oh, \[A]oh \[Bm]cante\[A]rò, un \[G]cantico nuovo da\[A]vanti al Si\[D]gnore. \rep{2}
\endchorus

\beginverse
\chordsoff
Solo con l'amore puoi raggiungere il cielo: 
Fratello ama con me. \rep{2}
\endverse

\beginverse
\chordsoff
Tutti noi insieme per raggiungere il cielo: 
Fratello canta con me. \rep{2}
\endverse

\endsong

%titolo{Ora è tempo di gioia}
%autore{Gen Rosso}
%album{Se siamo uniti}
%tonalita{Re}
%gruppo{}
%momenti{}
%identificatore{ora_e_tempo_di_gioia}
%data_revisione{2011_12_31}
%trascrittore{Francesco Endrici}
\beginsong{Ora è tempo di gioia}[by={Gen\ Rosso}]
\beginverse
L'\[D]eco \[Em]torna d'an\[D]tiche \[G]val\[A]li
\[D]la sua \[Em]voce \[D7+]non porta \[C7+]più,
\[Bm]ricordo \[F#m]di som\[G]messe \[E7]lacri\[A]me
\[D]di e\[Em7]si\[D]li in terre \[A4]lonta\[D]ne.
\endverse
\beginchorus
\[G]Ora è \[D]tempo di \[C]gio\[D]ia, \brk \[G]non \[Am7]ve \[G]ne ac\[C]cor\[D4]ge\[D]te
\[G]ecco \[D]faccio una \[Am]cosa \[Em]nuova
\[B7]nel de\[C7+]serto una \[Bm]strada apri\[Em]rò.
\endchorus
\beginverse
\chordsoff
Come l'onda che sulla sabbia
copre le orme e poi passa e va,
così nel tempo si cancellano
le ombre scure del lungo inverno.
\endverse
\beginverse
\chordsoff
Tra i sentieri dei boschi il vento
con i rami ricomporrà
nuove armonie che trasformano
i lamenti in canti di festa.
\endverse
\endsong


%titolo{Ora lasciateci cantare}
%autore{Sequeri}
%album{E mi sorprende}
%tonalita{Re-}
%gruppo{}
%momenti{Pasqua}
%identificatore{ora_lasciateci_cantare}
%data_revisione{2011_12_31}
%trascrittore{Francesco Endrici - Manuel Toniato}
\beginsong{Ora lasciateci cantare}[by={Sequeri}]

\beginverse*
\itshape \[Dm7/9]O filii et fili\ae, Rex c\ae lestis, rex glori\ae,
morte surrexit hodie, alleluia.
\[Dm]{Cri}\[C]sto \[B&]{Si}\[A]gnore \[Dm]{risu}\[C]{sci}\[Dm]tò.
\endverse

\beginchorus
\[Dm]Ora lasciate\[C]ci can\[Dm]tare \[F]la tene\[C]rezza \[Dm]dell'a\[C]more,
\[Dm]ora lasciate\[C]ci can\[Dm]tare \[F]tutta la \[C]forza \[Dm]della \[C]vita!
\[Dm]Ora lasciate\[C]ci can\[Dm]tare \[F]tutta la \[C]nostra \[Dm]{gio}\[C]ia,
\[Dm]ora lasciate\[C]ci can\[Dm]tare: \[F]Cristo ri\[C]susci\[Dm]tò!
\endchorus

\beginverse
\[Dm]Padre dell'u\[Gm]omo \[B&]io \[C]ti rin\[Dm]grazio,
\[F]Fi\[Dm]glio e fra\[Gm]tello \[C]ti \[Gm]bene\[A7]dico,
\[B&7]Spirito \[Am7]Santo, \[F]se\[A7]me di \[D]vi\[C]ta,
\[B&7]oltre la \[Gm]mor\[A7]te \[Dm]so \[C]che tu \[Dm]sei.

\endverse

\beginverse
\chordsoff
E questo canto, come il tuo pane,
semplice e lieto ora ci unisce
nella memoria, nella speranza
d'essere insieme quando verrai.
\endverse
\endsong


%titolo{Osanna al figlio di David}
%autore{}
%album{}
%tonalita{Sol}
%gruppo{}
%momenti{}
%identificatore{osanna_al_figlio_di_david}
%data_revisione{2011_12_31}
%trascrittore{Francesco Endrici - Manuel Toniato}
\beginsong{Osanna al figlio di David}

\beginchorus
\[D7]O\[G]sanna al \[D]Figlio di \[G]Da\[Em]vid, o\[Am]sanna al \[C]Re\[D]den\[G]tor!
\endchorus

\beginverse
\[Em]Apritevi, o \[D]porte e\[G]terne, a\[C]vanzi il \[D]Re della \[G]gloria,
a\[Am]dori \[Em]cielo e \[C]terra, l'e\[G]terno \[D]suo po\[G]ter.
\endverse

\beginverse
\chordsoff
O monti stillate dolcezza il re d'amor s'avvicina
si dona pane vivo ed offre pace al cuor.
\endverse

\beginverse
\chordsoff
O vergine presso l'Altissimo trovasti grazia ed onor:
soccorri i tuoi figlioli donando il Salvator.
\endverse
\endsong

%titolo{Osanna al Re dei Re}
%autore{Pesare}
%album{Gerico, le tue mura crolleranno}
%tonalita{Re}
%gruppo{}
%momenti{Ingresso;Congedo}
%identificatore{osanna_al_re_dei_re}
%data_revisione{2011_12_31}
%trascrittore{Francesco Endrici - Manuel Toniato}
\beginsong{Osanna al Re dei Re}[by={Pesare}]

\beginchorus
\[D] \[A]O\[G]san\[A]na \[D] \[A]O\[G]san\[A]na
\[D] \[A]O\[G]san\[A]na o\[G]sanna al \[A]Re dei \[D]re.
\endchorus

\beginverse
\[D] Su cantiamo \[A]al Signore \[G] un canto \[A]nuovo
\[D] e gioiosi \[A]proclamiamo \[G]che  \[D]Ge\[G]sù \[A]è il \[D]Re.
\endverse

\beginverse
\chordsoff
Lui ci porta alla vittoria, è un Dio potente.
Innalziamo a Lui la gloria: egli è il Signor
\endverse

\beginchorus
\[D] \[A]O\[G]san\[A]na \[D] \[A]O\[G]san\[A]na
\[D] \[A]O\[G]san\[A]na o\[G]sanna al \[A]Re dei \[E]re.
\[E] \[B]O\[A]san\[B]na \[E] \[B]O\[A]san\[B]na
\[E] \[B]O\[A]san\[B]na o\[A]sanna al \[B]Re dei \[E]re.
\endchorus

\beginverse
\[E] Eleviamo  \[B]su nel cielo \[A] la nostra \[B]lode
\[E] e con gli ange\[B]li cantiamo: ``\[A]Glo\[E]ria \[A]al \[B]Si\[E]gnor!''
\endverse

\beginchorus
\[E] \[B]O\[A]san\[B]na \[E] \[B]O\[A]san\[B]na
\[E] \[B]O\[A]san\[B]na o\[A]sanna al \[B]Re dei \[E]re.
\endchorus
\endsong




\lettera
%titolo{Pace a te, fratello mio}
%autore{Giombini}
%album{Camminiamo nella speranza}
%tonalita{Re}
%gruppo{}
%momenti{Pace}
%identificatore{pace_a_te_fratello_mio}
%data_revisione{2011_12_31}
%trascrittore{Francesco Endrici - Manuel Toniato}
\beginsong{Pace a te, fratello mio}[by={Giombini}]

\beginchorus
\[D]Pace a te, fra\[G]tello \[D]mio, pa\[Bm]ce a te, so\[A]rella \[D]mia,
\[D]pace a \[F#m]tutti gli \[Bm]uomi\[F#m]ni \[G]di buona \[A]volon\[D]tà.
\endchorus

\beginverse*
\[D]Pace nella scuola e \[Bm]nella fabbrica, 
\[D]nella politica e \[A7]nello sport.
\[G]Pace in famiglia \[F#m]pace in auto\[Bm]mobile, 
\[G]pace nella \[A7]Chie\[D]sa.
\endverse
\endsong

%titolo{Pace a te, pace a te}
%autore{Costa, Varnavà}
%album{E se anche non ci conosciamo}
%tonalita{Do}
%gruppo{}
%momenti{Pace}
%identificatore{pace_a_te_pace_a_te}
%data_revisione{2011_12_31}
%trascrittore{Francesco Endrici}
\beginsong{Pace a te, pace a te}[by={Costa, Varnavà}]
\ifchorded
\beginverse*
\vspace*{-0.8\versesep}
{\nolyrics \[C]\[F]\[C]}
\vspace*{-\versesep}
\endverse
\fi
\beginverse
\memorize
\[C] Nel Signore \[F]io ti do la \[C]pace,
pace a \[F]te, pace a \[C]te.
\ifchorded
Nel Si\[Am]gnore \[F]io ti do la \[C]pace,
pace a \[Am]te, \[F]pace a \[C]te.
\fi
Nel suo nome \[F]resteremo u\[C]niti
pace a \[F]te, pace a \[C]te.
\ifchorded
Nel suo nome \[E7]resteremo u\[Am]niti
\[F]pace a \[C]te, \[F]pace a \[C]te. \[F]\[C]
\fi
\endverse
\beginverse
%\chordsoff
^E se anche ^non ci cono^sciamo,
pace a ^te, pace a ^te.
\ifchorded
E se ^anche ^non ci cono^sciamo,
pace a ^te, ^pace a ^te.

Lui conosce ^tutti i nostri ^cuori,
pace a ^te, pace a ^te.
\fi
Lui conosce ^tutti i nostri ^cuori,
^pace a ^te, ^pace a ^te. ^^
\endverse
\beginverse
%\chordsoff
^Se il pensiero ^non è sempre u^nito,
pace a ^te, pace a ^te.
\ifchorded
Se il pen^siero ^non è sempre u^nito,
pace a ^te, ^pace a ^te.

Siamo uniti ^nella stessa ^fede,
pace a ^te, pace a ^te.
\fi
Siamo uniti ^nella stessa ^fede,
^pace a ^te, ^pace a ^te. ^^
\endverse
\beginverse
%\chordsoff
^E se noi ^non giudiche^remo,
pace a ^te, pace a ^te.
\ifchorded
E se ^noi ^non giudiche^remo,
pace a ^te, ^pace a ^te.

Il Signore ^ci vorrà sal^vare,
pace a ^te, pace a ^te.
\fi
Il Signore ^ci vorrà sal^vare,
^pace a ^te, ^pace a ^te,
\[F]pace a \[C]te, \[F]pace a \[C]te. \[F] \[C]
\endverse
\endsong

%titolo{Pace alle genti}
%autore{Mariano}
%album{Venite a me}
%tonalita{La}
%gruppo{}
%momenti{Pace}
%identificatore{pace_alle_genti}
%data_revisione{2011_12_31}
%trascrittore{Francesco Endrici}
\beginsong{Pace alle genti}[by={Mariano}]
\beginverse*
\[A]Spezze\[C#m]rai \[D]l'arco della \[E]guerra, \chordsoff
annunzierai pace alle genti
e regnerai da mare a mare
fino ai confini di questa terra.
\endverse
\endsong

%titolo{Pace Shalom}
%autore{}
%album{}
%tonalita{La}
%gruppo{}
%momenti{Pace}
%identificatore{pace_shalom}
%data_revisione{2011_12_31}
%trascrittore{Francesco Endrici}
\beginsong{Pace Shalom}
\beginverse
Vi \[A]lascio la \[E]pace vi \[F#m]do la mia \[C#m]pace
sha\[D]lom sha\[E]lom sha\[A]lom.
\endverse
\beginverse
^Rimanete in ^me e ^porterete ^frutto
sha^lom sha^lom sha^lom.
Sha\[D]lom sha\[E]lom sha\[A]lom.
\endverse
\endsong

%titolo{Pace sia, pace a voi}
%autore{Gen Verde, Gen Rosso}
%album{Come fuoco vivo}
%tonalita{Mi}
%gruppo{}
%momenti{Pace}
%identificatore{pace_sia_pace_a_voi}
%data_revisione{2011_12_31}
%trascrittore{Francesco Endrici}
\beginsong{Pace sia, pace a voi}[by={Gen\ Verde, Gen\ Rosso}]
\beginchorus
\[B] “Pace \[E]sia, pace a voi”: la tua \[A]pace sarà
sulla \[C#m]terra com'è nei \[B]cieli.
“Pace \[E]sia, pace a voi”: la tua \[A]pace sarà
gioia \[G]nei nostri \[D]occhi, nei \[A]cuo\[B]ri.
“Pace \[E]sia, pace a voi”: la tua \[A]pace sarà
luce \[C#m]limpida nei pen\[B]sieri.
“Pace \[E]sia, pace a voi”: la tua \[A]pace sarà
una \[E]casa per \[B]tutti. \[E]\[A]\[E]
\endchorus
\beginverse
“\[A]Pace a \[E]voi”: sia il tuo \[B]dono vi\[C#m]sibile.
“\[A]Pace a \[E]voi”: la tua e\[B]redi\[C#m]tà.
“\[A]Pace a \[E]voi”: come un \[B]canto all'u\[C#m]nisono
che \[D]sale dalle nostre cit\[B]tà.
\endverse
\beginverse
%\chordsoff
“^Pace a ^voi”: sia un im^pronta nei ^secoli.
“^Pace a ^voi”: segno d'^uni^tà.
“^Pace a ^voi”: sia l'ab^braccio tra i ^popoli,
la ^tua promessa all'umani^tà.
\endverse
\endsong

%titolo{Padre mio}
%autore{Gen Rosso}
%album{Dove Tu sei}
%tonalita{Re}
%gruppo{}
%momenti{Comunione}
%identificatore{padre_mio}
%data_revisione{2011_12_31}
%trascrittore{Francesco Endrici - Manuel Toniato}
\beginsong{Padre mio}[by={Gen\ Rosso}]


\beginverse
Padre \[D]mio \[D5+]mi abbandono a \[D]Te, \[D5+] 
di \[Em]me fa quello che ti \[A]piace,
\[F#]grazie di ciò che fai per \[Bm]me, \[B] 
\[E]spero solamente in \[A]Te.
Pur\[D]ché \[D5+]si compia il tuo vo\[D]lere \[D5+] 
in \[Em]me e in tutti i miei fra\[A]telli,
\[F#]niente desidero di \[Bm]più,\[B] 
\[E]fare quello che vuoi \[A]Tu.
\endverse

\beginchorus
\[D]Dammi che ti rico\[E7]nosca,
\[A7]dammi che ti possa a\[D]mare sempre più,
\[D]dammi che ti resti ac\[E7]canto.
\[A7]dammi d'essere l'A\[D]mor.
\endchorus

\beginverse
\chordsoff
Fra le tue mani depongo la mia anima, 
con tutto l'amore del mio cuore,
mio Dio la dono a te
perché ti amo immensamente.
Sì ho bisogno di donarmi a te
senza misura affidarmi alle tue mani
Perché sei Padre mio,
perché sei Padre mio!
\endverse
\endsong


%titolo{Pane del cielo}
%autore{Gen Rosso}
%album{Dove tu sei}
%tonalita{Do}
%gruppo{}
%momenti{Comunione}
%identificatore{pane_del_cielo}
%data_revisione{2011_12_31}
%trascrittore{Francesco Endrici}
\beginsong{Pane del cielo}[by={Gen\ Rosso}]
\beginchorus
\[C]Pane del \[Em]cielo, \[F]sei Tu Ge\[C]sù,
\[Am]via d'a\[Dm]more: \[F]Tu ci fai come \[C]Te. \rep{2}
\endchorus
\beginverse
\[F]No, non è ri\[Dm]masta fredda la \[G]terra;
\[Em]Tu sei ri\[F]masto con \[C]noi \[F] per nutrirci di \[C]Te.
\[Am]Pane di \[G]vita, \[Am] ed infiam\[G]mare col tuo a\[E]more
\[G]tutta l'u\[F]mani\[C]tà.
\endverse
\beginverse
\chordsoff
Sì, il cielo è qui su questa terra;
Tu sei rimasto con noi ma ci porti con Te
nella tua casa dove vivremo insieme a Te
tutta l'eternità.
\endverse
\beginverse
\chordsoff
No, la morte non può farci paura;
Tu sei rimasto con noi, e chi vive di Te
vive per sempre.
Sei Dio con noi, sei Dio per noi,
Dio in mezzo a noi.
\endverse
\endsong

%titolo{Pane della vita}
%autore{Tranchida}
%album{Con voci di gioia}
%tonalita{Re}
%gruppo{}
%momenti{Comunione}
%identificatore{pane_della_vita_tranchida}
%data_revisione{2011_12_31}
%trascrittore{Francesco Endrici}
\beginsong{Pane della vita}[by={Tranchida}]
\ifchorded
\beginverse*
\vspace*{-0.8\versesep}
{\nolyrics \[D]\[G]\[D]\[A]}
\vspace*{-\versesep}
\endverse
\fi
\beginchorus
\[D]Pane della vita sei \[G]tu Ge\[D]sù,
vino di salvezza tu \[G]offri a \[A]noi.
\[G]Questa comu\[A]nione con \[D]te Si\[Bm]gnore
Chiesa \[D]viva \[A]ci fa\[D]rà.
\endchorus
\beginverse
\memorize
Invitati alla tua \[A]festa
noi for\[G]miamo un solo \[D]corpo
vero \[Em]cibo è la Pa\[Bm]rola che noi \[G]tutti uni\[A]rà.
\endverse
\beginverse
Invitai alla tua ^cena cele^briamo la tua ^Pasqua
il tuo ^corpo e il tuo ^sangue
tu Si^gnore done^rai. \[B&7]
\endverse
\beginchorus
\transpose{1}
\[D]Pane della vita sei \[G]tu Ge\[D]sù,
vino di salvezza tu \[G]offri a \[A]noi.
\[G]Questa comu\[A]nione con \[D]te Si\[Bm]gnore
Chiesa \[D]viva \[A]ci fa\[D]rà.
\endchorus
\beginverse
\transpose{1}
Invitati alla tua ^mensa tu ci ^nutri col tuo ^pane
Tu ^Signore per a^more nuova ^vita ci offri^rai.
\endverse
\beginverse
\chordsoff
Invitati al tuo banchetto col tuo vino ci rinnovi
Tu signore in eterno gioia immensa ci darai.
\endverse
\beginchorus
\chordsoff
Fine: Chiesa viva ci farà.
\endchorus
\endsong


%titolo{Pane di vita}
%autore{Marranzino, Pesare}
%album{Il Tuo amore è grande}
%tonalita{Sol}
%gruppo{}
%momenti{Offertorio;Comunione}
%identificatore{pane_di_vita}
%data_revisione{2011_12_31}
%trascrittore{Francesco Endrici - Manuel Toniato}
\beginsong{Pane di vita}[by={Marranzino, Pesare}]

\ifchorded
\beginverse*
\vspace*{-0.8\versesep}
{\nolyrics \[C/E] \[D/F#] \[C7+] \[C6]  \[C/E] \[D/F#] \[C7+] 
\[C/E] \[D/F#] \[G7+] \[C7+] \[Am7] \[Am/D] \[G4] \[G] }
\vspace*{-\versesep}
\endverse
\fi
\beginverse
\[G]Pane di vita \[C7+]sei, \brk spez\[G]zato per tutti \[C7+]noi,
chi ne \[C]man\[D]gia per \[G7+]sempre in \[C7+]te vi\[D4]vrà. \[D]  \[Am/D] 
\chordsoff
Veniamo al tuo santo altare, 
mensa del tuo amore.
Come pane vieni in mezzo a noi.
\endverse

\beginchorus
Il tuo \[G]corpo ci sazie\[C7+]rà \[Am7/D] 
Il tuo \[G]sangue ci salve\[C7+]rà
\[A7/C#]perché Si\[G/D]gnor tu sei \[B/D#]morto per a\[Em7]more
e ti \[A7]offri \[Am7]og\[D]gi \[C]per \[D]noi. \[Am7/D] \rep{2}
({\footnotesize II volta} e ti \[A7]offri \[Am7/D]oggi \[D7]per \[C/E]no\[D/F#]i. )
\endchorus

\ifchorded
\beginverse*
\vspace*{-\versesep}
{\nolyrics \[G7+] \[C7+] \[Am7] \[Am/D] \[G4] \[G] }
\endverse
\fi

\beginverse
\chordsoff
Fonte di vita sei, 
immensa carità,
il tuo sangue ci dona l'eternità.

Veniamo al tuo santo altare, 
mensa del tuo amore.
Come vino vieni in mezzo a noi.
\endverse

\beginchorus
Il tuo \[G]corpo ci sazie\[C7+]rà \[Am7/D] 
Il tuo \[G]sangue ci salve\[C7+]rà
\[A7/C#]perché Si\[G/D]gnor tu sei \[B/D#]morto per a\[Em7]more
e ti \[A7]offri \[Am7]{og}\[D]{gi} \[C]{per} \[D]{noi.} \[Am7/D]\rep{2}
({\footnotesize II volta} e ti \[A7]offri \[Am7/D]{oggi per} \[G4]{no}\[G]{i.})
\endchorus
\endsong


%titolo{Pane di vita nuova}
%autore{Frisina}
%album{Pane di vita nuova}
%tonalita{Re}
%gruppo{}
%momenti{Comunione}
%identificatore{pane_di_vita_nuova}
%data_revisione{2011_12_31}
%trascrittore{Francesco Endrici}
\beginsong{Pane di vita nuova}[by={Frisina}]
\beginverse
\[D]Pane \[G]di vita \[A]nuo\[D]va, 
\[G]vero \[D]cibo dato agli \[Em]uomi\[A]ni,
\[G]nutri\[D]mento \[Em]che sostiene il \[A]mondo, 
\[Bm]do\[G]no  \[D]splendi\[Em]do  \[A]di gra\[D]zia.
\endverse
\beginverse
\chordsoff
Tu sei sublime frutto 
di quell'albero di vita
che Adamo non potè toccare:
Ora è in  Cristo a noi donato.
\endverse
\beginchorus
\[G]Pane \[D]della \[G]vi\[A]ta, 
\[D]sangue \[F#]di sal\[Em]vez\[A]za,
\[G]vero \[D]corpo, \[Em]vera be\[Bm]vanda,
\[Em]cibo di \[D]grazia per il \[A]mon\[D]do.
\endchorus
\beginverse
\chordsoff
Sei l'Agnello immolato
nel cui Sangue è la salvezza,
memoriale della vera Pasqua
della nuova Alleanza.
\endverse
\beginverse
\chordsoff
Manna che nel deserto
nutri il popolo in cammino,
sei sostegno e forza nella prova
per la Chiesa in mezzo al mondo.
\endverse
\beginchorus
\chordsoff 
Rit. 
\endchorus
\beginverse
\chordsoff
Vino che ci dà gioia,
che riscalda il nostro cuore,
sei per noi il prezioso frutto
della vigna del Signore.
\endverse
\beginverse
\chordsoff
Dalla vite ai tralci
scorre la vitale linfa
che ci dona la vita divina,
scorre il sangue dell'amore.
\endverse
\beginchorus
\chordsoff 
Rit. 
\endchorus
\beginverse
\chordsoff
Al banchetto ci inviti
che per noi hai preparato,
doni all'uomo la tua Sapienza,
doni il Verbo della vita.
\endverse
\beginverse
\chordsoff
Segno d'amore eterno
pegno di sublimi nozze,
comunione nell'unico corpo
che in Cristo noi formiamo.
\endverse
\beginchorus
\chordsoff 
Rit. 
\endchorus
\beginverse
\chordsoff
Nel tuo Sangue è la vita
ed il fuoco dello Spirito,
la sua fiamma incendia il nostro cuore 
e purifica il mondo.
\endverse
\beginverse
\chordsoff
Nel prodigio dei pani
tu sfamasti ogni uomo,
nel tuo amore il povero è nutrito
e riceve la tua vita.
\endverse
\beginchorus
\chordsoff 
Rit. 
\endchorus
\beginverse
\chordsoff
Sacerdote eterno
Tu sei vittima ed altare,
offri al Padre tutto l'universo,
sacrificio dell'amore.
\endverse
\beginverse
\chordsoff
Il tuo Corpo è tempio
della lode della Chiesa,
dal costato tu l'hai generata,
nel tuo Sangue l'hai redenta.
\endverse
\beginchorus
\chordsoff 
Rit. 
\endchorus
\beginverse
\chordsoff
Vero Corpo di Cristo
tratto da Maria Vergine,
dal tuo fianco doni a noi la grazia, 
per mandarci tra le genti.
\endverse
\beginverse
\chordsoff
Dai confini del mondo,
da ogni tempo e ogni luogo
il creato a te renda grazie,
per l'eternità ti adori.
\endverse
\beginchorus
\chordsoff 
Rit. 
\endchorus
\beginverse
\chordsoff
A te Padre la lode,
che donasti il Redentore,
e al Santo Spirito di vita 
sia per sempre onore e gloria. 
\endverse
\beginverse*
\[D]\[A]A\[D]men.
\endverse
\endsong

%titolo{Per la vita che ci dai}
%autore{}
%album{}
%tonalita{Re}
%gruppo{}
%momenti{}
%identificatore{per_la_vita_che_ci_dai}
%data_revisione{2011_12_31}
%trascrittore{Francesco Endrici - Manuel Toniato}
\beginsong{Per la vita che ci dai}

\beginverse
Per la \[D]vita \[G]che ci \[D]dai, alle\[G]luia\dots\[D]
\endverse

\beginchorus
Ti rin\[D]grazio o \[Em]mio Si\[A]gnor, al\[D]le\[A]lu\[D]ia.
\endchorus

\beginverse
\chordsoff
Per la gioia e per l'amor, alleluia\dots
\endverse

\beginverse
\chordsoff
Il tuo amore ci riunì, alleluia\dots
\endverse

\beginverse
\chordsoff
E se Dio mi ha dato il sol, alleluia\dots
\endverse

\beginverse
\chordsoff
E se Dio mi ha dato il mar, alleluia\dots
\endverse

\beginverse
\chordsoff
E se Dio mi ha dato te, alleluia\dots
\endverse

\beginverse
\chordsoff
Per la gioia che ci dai, alleluia\dots
\endverse

\beginverse
\chordsoff
Per il bene che ci vuoi, alleluia\dots
\endverse

\beginverse
\chordsoff
Per la vita e per l'amor, alleluia\dots
\endverse

\beginverse
\chordsoff
La tua morte ci salvò, alleluia\dots
\endverse

\beginverse
\chordsoff
Nella pace pregherò, alleluia\dots
\endverse
\endsong

%titolo{Perché la vostra gioia sia piena}
%autore{Valoppi}
%album{MGS Triveneto, Festa dei Giovani 2004}
%tonalita{Re}
%gruppo{}
%momenti{}
%identificatore{perche_la_vostra_gioia}
%data_revisione{2011_12_31}
%trascrittore{Francesco Endrici - Manuel Toniato}
\beginsong{Perché la vostra gioia sia piena}[by={Valoppi}]
\beginchorus
Per\[D]ché la vostra \[A]gioia sia piena
Per\[Bm]ché la vostra \[G]gioia sia \[A]piena. \rep{2}
\endchorus

\beginverse
\[D]Prendi tra le mani prendi il mio Amore
\[A]donalo a chi soffre nel suo dolore
\[Bm]vivi sulla strada è il tuo destino
\[A]lascia che Dio guidi, guidi il tuo cammino.
\[D]Apri le tue mani dona la tua vita
\[A]non tenerla stretta tra le tue dita
\[Bm]ora tocca al cuore aprilo al mondo
\[A]gioca la tua vita e sia fino in fondo.
\endverse

\beginverse
\chordsoff
Chiedi ciò che è vero, ti sarà dato
per vivere l'Amore Dio ci ha creato
resta unito a me vivendo le parole
se così farai sarà ciò che Lui vuole.
Sentirai che scende dal cielo bellezza
riconoscerai la sua tenerezza
se tu porti in te le mie parole
da te fiorirà ciò che Dio vuole.
\endverse

\beginverse
\chordsoff
Sogno di Dio, da Lui sei nato
vita divina ti ha generato
se poi rimani nel mio Amore
vivrà pienezza nel cuore.
\endverse

\beginchorus
Per\[D]ché la vostra \[A]gioia sia piena
re\[Bm]state uniti a \[G]me nel pro\[A]fondo.
Per\[D]ché la vostra \[A]gioia sia piena
vi\[Bm]vete il mio A\[G]more nel \[A]mondo. \rep{2}

Per\[E]ché la vostra \[B]gioia sia piena
re\[C#m]state uniti a \[A]me nel pro\[B]fondo.
Per\[E]ché la vostra \[B]gioia sia piena
vi\[C#m]vete il mio A\[A]more nel \[B]mondo. \rep{2}

Per\[F]ché la vostra \[C]gioia sia piena
re\[Dm]state uniti a \[B&]me nel pro\[C]fondo.
Per\[F]ché la vostra \[C]gioia sia piena
vi\[Dm]vete il mio A\[B&]more nel \[C]mondo
\chordsoff
(vivete il mio amore profondo per voi).
\endchorus
\endsong


%titolo{Perché tu sei con me}}
%autore{Gen Verde}
%album{Cerco il Tuo volto}
%tonalita{Sol}
%gruppo{}
%momenti{Comunione;Salmi}
%identificatore{perche_tu_sei_con_me}
%data_revisione{2011_12_31}
%trascrittore{Francesco Endrici - Manuel Toniato}
\beginsong{Perché tu sei con me}[by={Gen\ Verde}]

\ifchorded
\beginverse*
\vspace*{-0.8\versesep}
{\nolyrics \[Am]  \[Em]  \[D]  \[Em]  \[D] }
\vspace*{-\versesep}
\endverse
\fi
\beginchorus
\[C]Solo \[Em]Tu sei il \[D]mio pa\[C]store, \[Am]niente \[G]mai mi \[C]manche\[D]rà,
solo \[Em]Tu sei il \[D]mio pa\[C]store, o Si\[D]gno\[C]re. \[D]  \[Em] 
\endchorus

\beginverse
\[G]Mi con\[Em]duci \[C]dietro \[D]Te \[Em]sulle verdi al\[Bm]ture.
\[C]Ai ruscelli tran\[G]quilli las\[Am]sù
\[G]dov'è più \[D]limpida \[Em]l'acqua per \[C]me
\[G]dove mi \[D]fai ripo\[C]{sa}\[G]re.
\endverse

\beginverse
\chordsoff
Anche fra le tenebre di un abisso oscuro,
io non temo alcun male perché,
Tu mi sostieni sei sempre con me,
rendi il sentiero sicuro.
\endverse

\beginverse
\chordsoff
Siedo alla tua tavola che mi hai preparato,
ed il calice è colmo per me
di quella linfa di felicità
che per amore hai versato.
\endverse

\beginverse
\chordsoff
Sempre mi accompagnano lungo estati e inverni,
la tua grazia, la tua fedeltà.
Nella tua casa io abiterò
fino alla fine dei giorni.
\endverse
\endsong

%titolo{Popoli tutti}
%autore{}
%album{}
%tonalita{Re}
%gruppo{}
%momenti{Congedo}
%identificatore{popoli_tutti}
%data_revisione{2011_12_31}
%trascrittore{Francesco Endrici}
\beginsong{Popoli tutti}
\beginchorus
\[D]Popoli \[G]tutti, bat\[D]tete le \[A]mani,
can\[G]tate al Si\[D]gnore con \[G]voci di \[A]gioia.
\[D]Grande è il suo \[G]nome su \[D]tutta la \[A]terra,
la \[G]sua tene\[D]rezza per \[A]tutte le na\[D]zioni.
\[G]Alleluia, \[D] \[A]alleluia, \[D] \[G]alleluia, \[D] \[A]alleluia. \[D]
\endchorus
\beginverse
Chi \[Bm]è come il Signore nostro Dio
che \[G]siede nell'\[A]alto dei \[D]cieli \[F#]
e \[G]volge lo \[A]sguardo su \[F#m7]tutta la \[Bm]terra?
\[G]Alleluia, \[D]allelu\[A]ia.
\endverse
\beginverse
\chordsoff
Chi è come il Signore nostro Dio
che rialza con amore il povero
per dargli onore in mezzo al suo popolo?
Alleluia, alleluia.
\endverse
\endsong

%titolo{Popoli tutti acclamate}
%autore{Zschech}
%album{Eterna è la Sua misericordia}
%tonalita{La}
%gruppo{}
%momenti{Ingresso;Congedo;Comunione}
%identificatore{popoli_tutti_acclamate}
%data_revisione{2011_12_31}
%trascrittore{Francesco Endrici - Manuel Toniato}
\beginsong{Popoli tutti acclamate}[by={Zschech}]

\beginverse
\[A] Mio Dio, \[E] Signore, \[F#m]nulla è \[E]pari a \[D]te.
Ora e per \[A/F#]sempre, \[D]voglio lo\[A/E]dare
\[F#m7]il Tuo grande \[G]amor \[D]per \[E4]noi. \[E] 
\[A]  Mia roccia \[E] Tu sei, \[F#m]pace e con\[E]forto mi \[D]dai.
Con tutto il \[A/F#]cuore \[D]e le mie \[A/E]forze,
\[F#m7]sempre io Ti a\[G]do\[D]re\[E4]rò. \[E] 
\endverse

\beginchorus
\[A]Popoli \[F#m]tutti accla\[D]mate al Si\[E]gnore.
\[A]Gloria e po\[F#m]tenza can\[D7+]tiamo al \[E]re.
\[F#m]Mari e monti si \[D]prostrino a te,
al tuo \[E]nome, \[F#m]o Si\[E]gnore.
\[A]Canto di \[F#m]gioia per \[D]quello che \[E]fai,
per \[A]sempre Si\[F#m]gnore con \[D7+]te reste\[E]rò.
\[F#m]Non c'è promessa non \[D]c'è fedel\[E7]tà che in \[A]te.
\endchorus

\beginverse
\chordsoff
Tu luce d'amore, Spirito di Santità
entra nei cuori di questi tuoi figli
chiamati ad annunciare il Re.
Tu forza d'amore nuova speranza ci dai
in questo giorno a te consacrato
gioia immensa canterà.
\endverse
\endsong


%titolo{Preghiera a Maria}
%autore{Semprini}
%album{Dimmi dolce Maria}
%tonalita{Re}
%gruppo{}
%momenti{Maria}
%identificatore{preghiera_a_maria}
%data_revisione{2011_12_31}
%trascrittore{Francesco Endrici}
\beginsong{Preghiera a Maria}[by={Semprini}]
\beginverse
Ma\[D]ria, \[Bm] tu che hai atteso nel si\[G]len\[Em]zio \[A]
la sua pa\[F#m]rola per \[Bm]noi.
\endverse
\beginchorus
A\[G]iuta\[A]ci ad ac\[F#m]coglie\[Bm]re
il \[G]Figlio Tuo che ora \[Em]vive in \[A]noi. \[A7]
\endchorus
\beginverse
\chordsoff
Maria, tu che sei stata così docile 
davanti al Tuo Signor.
\endverse
\beginverse
\chordsoff
Maria, tu che hai portato dolcemente
l'immenso dono d'amor.
\endverse
\beginverse
\chordsoff
Maria madre, umilmente tu hai sofferto 
il suo ingiusto dolor.
\endverse
\beginverse
\chordsoff
Maria, tu che vivi nella gloria
insieme al tuo Signor.
\endverse
\endsong





%titolo{Preghiera dei poveri di Jahvè}
%autore{Amadei}
%album{Giovani in festa}
%tonalita{Mi-}
%gruppo{}
%momenti{Avvento}
%identificatore{preghiera_dei_poveri_di_jahve}
%data_revisione{2011_12_31}
%trascrittore{Francesco Endrici - Manuel Toniato}
\beginsong{Preghiera dei poveri di Jahvè}[by={Amadei}, ititle={O cieli piovete dall'alto}]

\beginchorus
O \[Em]cieli, piovete dall'\[C]alto, o \[D]nubi mandateci il \[Em]Santo,
o \[B7]terra, apriti o \[Em]terra e \[B7]germina il Salva\[Em]tore.
\endchorus

\beginverse
Siamo il de\[Em]serto, \[C]siamo l'ar\[D]sura: \[D]Marana\[B]thà,\[D] Mara\[B]nathà!
\endverse

\beginverse
\chordsoff
Siamo il vento, nessuno ci ode: Maranathà, Maranathà!
\endverse

\beginverse
\chordsoff
Siamo le tenebre, nessuno ci guida: Maranathà, Maranathà!
\endverse

\beginverse
\chordsoff
Siam le catene, nessuno ci scioglie: Maranathà, Maranathà!
\endverse
\endsong


%titolo{Preghiera sui doni}
%autore{}
%album{}
%tonalita{La}
%gruppo{}
%momenti{Offertorio}
%identificatore{preghiera_sui_doni}
%data_revisione{2011_12_31}
%trascrittore{Francesco Endrici}
\beginsong{Preghiera sui doni}
\ifchorded
\beginverse*
\vspace*{-0.8\versesep}
{\nolyrics \[D]\[Dm]\[A]\[F#m]\[Bm]\[A]}
\vspace*{-\versesep}
\endverse
\fi
\beginverse
\memorize
Nel pane \[A]che ti presen\[E]tiamo
offriamo a \[F#m]te \[B]tutto quello che \[E]siamo.
Il nostro \[C#7]cuore, la nostra \[D]vita,
accogli\[A]la nella \[E]tua bon\[A]tà. \[A7]
E crea in \[D]noi \[Dm7]un cuore \[C#m]nuovo \[F#7]
che sappia a\[Bm]mar come \[E]tu fai con \[A]noi.
\[A7]Fa' di questo \[D]pane \[Dm]cibo che dà \[A]vita
\[F#m]segno di uni\[Bm]tà tra noi \[E7]figli \[A]tuoi.
\endverse
\beginverse
\chordsoff
Nel vino che ti presentiamo
offriamo a Te l'amore che sta in noi.
Insegnaci a donarlo al mondo
come segno del tuo grande amor.
E dona a noi speranza viva
per diventar testimoni di te.
Fa' di questo vino sangue di salvezza
noi te lo chiediamo nel tuo figlio Gesù.
\endverse
\endsong


%titolo{Prenderemo il largo}
%autore{Scarpa, Buttazzo}
%album{Vieni soffio di dio}
%tonalita{La}
%gruppo{}
%momenti{Congedo}
%identificatore{prenderemo_il_largo}
%data_revisione{2011_12_31}
%trascrittore{Francesco Endrici}
\beginsong{Prenderemo il largo}[by={Scarpa, Buttazzo}]
\ifchorded
\beginverse*
\vspace*{-0.8\versesep}
{\nolyrics \[A]\[D]\[F#m]\[D]}
\vspace*{-\versesep}
\endverse
\fi
\beginverse
\memorize
|\[A]Questo è il nostro tempo per o|\[E]sare, per andare,
la pa|\[D7+]rola che ci chiama è quella |\[E4]tua! \[E]
|\[D]Come un giorno a Pietro, anche |\[Bm7]oggi dici a noi:
“Getta al |\[D7+]largo le tue reti insieme a |\[E]me”.
\endverse
\beginchorus
Sali|\[A]remo in questa \[E]barca anche |\[F#m]noi, \[7] 
il tuo |\[D7+]vento soffia |\[Bm7]già sulle |\[E]vele. \[F7dim]
Prende|\[A]remo il largo |\[C#m7]dove vuoi |\[F#m]tu \[7]
navi|\[D7+]gando insieme a |\[Bm7]te, Ge|\[A]sù. 
\endchorus
\beginverse
^Questo è il nostro tempo, \brk questo è il ^mondo che ci dai:
oriz^zonti nuovi, vie di umani^tà. ^
^Come un giorno a Pietro, anche ^oggi dici a noi: 
“Se mi ^ami più di tutto, segui ^me”.
\endverse
\beginchorus
Sali|\[A]remo in questa \[E]barca anche |\[F#m]noi, \[7] 
il tuo |\[D7+]vento soffia |\[Bm7]già sulle |\[E]vele. \[F7dim]
Prende|\[A]remo il largo |\[C#m7]dove vuoi |\[F#m]tu \[7]
navi|\[D7+]gando insieme a |\[Bm7]te, Ge|\[A]sù. |\[D]|\[F#m]|\[D]
\endchorus
\beginverse
^Navigando il mare della ^storia insieme a te, 
la tua ^barca in mezzo a forti venti ^va. ^
^Come un giorno a Pietro, anche ^oggi dici a noi: 
“Se tu ^credi in me, tu non affonde^rai”.
\endverse
\beginchorus
Sali|\[A]remo in questa \[E]barca anche |\[F#m]noi, \[7] 
il tuo |\[D7+]vento soffia |\[Bm7]già sulle |\[E]vele. \[F7dim]
Prende|\[A]remo il largo |\[C#m7]dove vuoi |\[F#m]tu \[7]
navi|\[D7+]gando insieme a |\[Bm7]te, Ge|\[A]sù.
\endchorus
\beginchorus
Sali|\[A]remo in questa \[E]barca anche |\[F#m]noi, \[7] 
il tuo |\[D7+]vento soffia |\[Bm7]già sulle |\[E]vele. \[F7dim]
Prende|\[A]remo il largo |\[C#m7]dove vuoi |\[F#m]tu \[7]
navi|\[D7+]gando insieme a |\[Bm7]te, Ge|\[A]sù,
navi|\[D7+]gando insieme a |\[Bm7]te, Ge|\[A]sù. |\[A]
\endchorus
\endsong


%titolo{Proteggimi o Dio}
%autore{Gallotta}
%album{Insieme venticinque}
%tonalita{Re}
%gruppo{}
%momenti{Salmi;Ingresso}
%identificatore{proteggimi_o_dio}
%data_revisione{2011_12_31}
%trascrittore{Francesco Endrici - Manuel Toniato}
\beginsong{Proteggimi o Dio}[by={Gallotta}]

\beginverse
\[D]Proteggimi, o \[A]Dio: in \[G]te io \[E]mi \[A]rifugio.
\[Bm]Ho detto a \[D]lui: "Sei \[G]tu il mi\[Em]o Signo\[A]re,
\[D]senza di \[A]te \[G]non ho bene a\[A]lcuno".
\[Bm]Nelle tue \[D]mani, Si\[G]gnore, è \[E]la mia \[A]vita!
\endverse

\beginchorus
\[F]Tu mi \[B&]indicherai il sen\[C]tiero della \[F]vita
\[B&]gioia piena nella \[F]tua presenza,
\[Gm]dolcezza senza \[C]fine.
\[F]Tu mi indiche\[B&]rai il sen\[C]tiero della \[F]vita
\[B&]gioia \[C]piena nella \[Dm]tua presenza,
\[Gm]dolcezza senza \[A]fine.
\endchorus

\beginverse
\chordsoff
Benedico Dio che m'ha dato consiglio;
anche di notte il mio cuore m'istruisce.
Innanzi a me sempre il Signore,
sta alla mia destra, non posso vacillare.
\endverse

\beginverse
\chordsoff
Mia eredità, mio calice è il Signore,
per me la sorte è su luoghi deliziosi.
Lieto e sereno è il cuore mio,
luce e speranza ai miei passi tu darai.
\endverse
\endsong






\lettera
%titolo{Quale gioia}
%autore{Bagniewski}
%album{Venite dal profondo}
%tonalita{Fa}
%gruppo{}
%momenti{Ingresso;Salmi}
%identificatore{quale_gioia}
%data_revisione{2011_12_31}
%trascrittore{Francesco Endrici}
\beginsong{Quale gioia}[by={Bagniewski}]
\beginverse
\[F]Quale \[C]gioia, mi \[Gm]disse\[B&]ro,
an\[F]dremo alla \[C]casa del \[Gm]Si\[B&]gnore. \[C]
\[F]Ora i \[C]piedi, o Ge\[Gm]rusa\[B&]lemme,
si \[F]fermano da\[C7]vanti a \[F]Te.
\endverse
\beginchorus
\[F]Ora Gerusa\[C]lemme \[Gm]è ri\[B&]costru\[C]ita \[C7]
\[F]come città \[C]salda, \[Gm]for\[B&]te e u\[C]nita. \[C7]
\endchorus
\beginverse
\chordsoff
Salgono insieme le tribù di Jahvè
per lodare il nome del Signore d'Israele.
\endverse
\beginverse
\chordsoff
Là sono posti i seggi della sua giustizia,
i seggi della casa di Davide.
\endverse
\beginverse
\chordsoff
Domandate pace per Gerusalemme,
sia pace a chi ti ama, pace alle tue mura.
\endverse
\beginverse
\chordsoff
Su di Te sia pace, chiederò il tuo bene,
per la casa di Dio chiederò la gioia.
\endverse
\beginverse
\chordsoff
Noi siamo il tuo popolo, Egli è il nostro Dio,
possa rinnovarci la felicità.
\endverse
\endsong



%titolo{Quale gioia è star con te}
%autore{Branca, Ciancio}
%album{Voglio vedere il Tuo volto}
%tonalita{Re}
%gruppo{}
%momenti{Comunione;Congedo}
%identificatore{quale_gioia_e_star_con_te}
%data_revisione{2011_12_31}
%trascrittore{Francesco Endrici - Manuel Toniato}
\beginsong{Quale gioia è star con te}[by={Branca, Ciancio}]

\ifchorded
\beginverse*
\vspace*{-0.8\versesep}
{\nolyrics \[D] \[A] \[Em] \[Bm] \[D] \[A] \[G] \[A] \[D] \[A] \[Bm] \[Em] \[D] \[A] \[G] \[A] \[Bm] \[A] \[D4/A] }
\vspace*{-\versesep}
\endverse
\fi
\beginverse
\[D4]Ogni volta che \[D]ti cerco, \[Em]ogni volta che ti invoco,
\[D]Sempre mi ac\[Bm]cogli Si\[A4]gnor. \[A] 
\[Bm7]Grandi sono i tuoi prodigi,\brk \[G2]Tu sei buono verso tutti, 
\[D/F#]Santo Tu \[G]regni tra \[A4]noi. \[A] 
\endverse

\beginchorus
Quale \[D]gioia è star con \[A]Te Gesù \[Em]vivo e vi\[Bm]cino,
\[D]bello è dar \[A]lode a Te, \[G]Tu sei il Si\[A]gnor.
Quale \[D]dono è aver cre\[A]duto in Te \brk \[Em]che non mi abban\[Bm]doni,
\[D]Io per sempre a\[A]biterò \[G]la \[A]Tu\[Bm]a \[A]casa, mio \[E]Re.
\endchorus

\ifchorded
\beginverse*
\vspace*{-\versesep}
{\nolyrics \[A] \[Em] \[Bm] \[D] \[A] \[G] \[A] }
\endverse
\fi

\beginverse
\chordsoff
Hai guarito il mio dolore, \brk hai cambiato questo cuore,
oggi rinasco, Signor. 
Grandi sono i tuoi prodigi, \brk Tu sei buono verso tutti, 
santo Tu regni tra noi.
\endverse

\beginverse
\[E]Hai salvato la mia vita, \[F#m7]hai aperto la mia bocca,
\[E/G#]Canto per \[C#m]Te, mio Si\[B4]gnor.\[B] 
\[C#m7]Grandi sono i tuoi prodigi, \brk \[A]tu sei buono verso tutti,
\[E]santo Tu \[A]regni tra \[C4]noi. \[C] 
\endverse

\beginchorus
Quale \[E]gioia è star con \[B]Te Gesù \[F#m]vivo e \[C#m]vicino,
\[E]Bello è dar \[B7]lode a Te, \[A]Tu sei il Si\[B7]gnor.
Quale \[E]dono è aver creduto in \[B]Te \brk \[F#m]che non mi abban\[C#m]doni.
\[E]Io per sempre a\[B7]biterò \[A]la \[B]Tu\[C#m]a \[B]casa, \brk mio \[E4/C#]Re.

Ti lode\[B4/G#]rò \echo{ti loderò}, \brk Ti adore\[C#4]rò \echo{ti adorerò}
Ti cante\[A/B]rò, cante\[B&/C]remo.

Quale \[F]gioia è star con \[C7]Te Gesù \[Gm]vivo e \[Dm]vicino,
\[F]bello è dar \[C7]lode a Te, \[B&]Tu sei il Si\[C7]gnor.
Quale \[F]dono è aver cre\[C7]duto in Te \brk \[Gm]che non mi abban\[Dm]doni.
\[F]Io per sempre a\[C7]biterò \[B&]la Tua \[C]casa, \brk mio \[C4/A]Re. \[C4/B&] 
\[C4]la Tua casa, mio \[C4/A]Re. \[C4/B&] 
\[C4]Tu sei il Signor, mio \[F]Re.
\endchorus
\endsong



%titolo{Quando busserò}
%autore{Giombini}
%album{Salmi per il nostro tempo}
%tonalita{Re}
%gruppo{}
%momenti{}
%identificatore{quando_bussero}
%data_revisione{2011_12_31}
%trascrittore{Francesco Endrici}
\beginsong{Quando busserò}[by={Giombini}]
\beginverse
Quando \[D]busserò \[G] alla tua \[D]porta
avrò \[G]fatto \[E]tanta \[A]strada;
avrò \[Bm]piedi \[F#m]stanchi e \[G]nu\[D]di, \[G]
avrò \[D]mani \[A7]bianche e \[D7]pure.
Avrò \[G]fatto \[A]tanta \[D]strada,
avrò \[G]piedi \[A]stanchi e \[D]nudi, \[Bm]
avrò \[Em]mani \[A7]bianche e \[Bm]pure, \[G] \brk o mio Si\[D]\[A]gno\[D]re.
\endverse
\beginverse
\chordsoff
Quando busserò alla tua porta
avrò frutti da portare,
avrò ceste di dolore,
avrò grappoli d'amore.
Avrò frutti da portare,
avrò ceste di dolore,
avrò grappoli d'amore, o mio Signore.
\endverse
\beginverse
\chordsoff
Quando busserò alla tua porta
avrò amato tanta gente,
avrò amici da ritrovare,
e nemici per cui pregare.
Avrò amato tanta gente,
avrò amici da ritrovare,
e nemici per cui pregare, o mio Signore.
\endverse
\endsong

%titolo{Quando cammino}
%autore{Giombini}
%album{}
%tonalita{Fa}
%gruppo{}
%momenti{}
%identificatore{quando_cammino}
%data_revisione{2011_12_31}
%trascrittore{Francesco Endrici}
\beginsong{Quando cammino}[by={Giombini}]
\beginverse
\[F]Quando cammino per il \[Dm]mondo
il Si\[B&]gnore cam\[C7]mina avanti a \[F]me.
Lo riconosco tra la \[Dm]gente
d'ogni \[B&]razza e \[C7]nazionali\[F]tà
A \[B&]volte pe\[C]rò mi \[F]fer\[Dm]mo
per\[Gm]ché la \[C7]strada è \[F7]faticosa,
Al\[B&]lora anche \[C]lui si \[F]siede lag\[Dm]giù
e m'a\[Gm]spetta \[G]sorri\[F]den\[C]te.
\endverse
\beginverse
\chordsoff
Quando cammino per il mondo
il Signore cammina avanti a me.
E per le strade della vita
Grido a tutti la mia felicità.
\endverse
\endsong


%titolo{Quando venne la sua ora}
%autore{Machetta}
%album{}
%tonalita{Re-}
%gruppo{}
%momenti{Pasqua}
%identificatore{quando_venne_la_sua_ora}
%data_revisione{2011_12_31}
%trascrittore{Francesco Endrici - Manuel Toniato}
\beginsong{Quando venne la sua ora}[by={Machetta}]

\beginverse                 
\[Dm]Quando \[Gm]venne la \[Dm]sua \[F]ora \[Am]di pas\[G6]sare dal mondo al \[Am]Padre
\[F]volle a\[C]marci \[Dm]sino alla \[Am]fine \[B&]Cristo,  \[Gm7]nostra \[D4]vi\[Dm]ta.
\endverse

\beginchorus
\[F]Dà la \[C]vita \[Dm]solo chi \[Am]muore, \[B&]ama \[Gm]chi sa \[C4]perde\[C]re;
\[F]è Si\[C]gnore \[Dm]solo chi \[Am]serve: \[Gm]farsi \[F]schiavo è \[C]liber\[F]tà.
\endchorus

\beginverse
\chordsoff
Ha lavato le nostre piaghe, disprezzato e umiliato, 
respinto dalla sua gente, Cristo, il Salvatore.
\endverse

\beginverse
\chordsoff
Mi chiamate Rabbì e Signore: ho tracciato la vostra via. 
Annunciate in tutta la terra questa mia parola.
\endverse
\endsong

%titolo{Quello che abbiamo udito}
%autore{Buttazzo}
%album{Alleluia è risorto il Signor}
%tonalita{Sol}
%gruppo{}
%momenti{Pasqua;Ingresso}
%identificatore{quello_che_abbiamo_udito}
%data_revisione{2011_12_31}
%trascrittore{Francesco Endrici - Manuel Toniato}
\beginsong{Quello che abbiamo udito}[by={Buttazzo}]

\beginchorus
\[G]Quello che abbiamo u\[Am]di\[D]to, 
\[G]quello che abbiam ve\[Em]du\[D]to,
\[Am]quello che ab\[B]biam \[Em4]tocca\[Em]to 
dell'a\[C]mo\[D7]re in\[G]finito \[C]l'annun\[D4]cia\[D]mo a \[C]voi! \[G] 
\endchorus

\beginverse
\[C]Grandi \[G]cose ha \[Am]fatto il Si\[G]gnore!
\[C]Del suo a\[G]more vo\[C]gliamo par\[D]lare:
\[Am7]Dio  \[B7]Padre il suo Figlio ha do\[Em7]nato,
\[C]sulla \[G]croce l'ab\[C]biamo ve\[D4/7]du\[C]to. \[C/G]  \[G] 
\endverse

\beginverse
\chordsoff
In Gesù tutto il cielo si apre,
ogni figlio conosce suo Padre;
alla vita rinasce ogni cosa
e l'amore raduna la Chiesa.
\endverse

\beginverse
\chordsoff
Nello Spirito il mondo è creato
e si apre al suo dono infinito;
il fratello al fratello dà mano
per aprire un nuovo cammino.
\endverse

\beginverse
\chordsoff
Viene il regno di Dio nel mondo
e l'amore rivela il suo avvento;
come un seme germoglia nell'uomo
che risponde all'invito divino.
\endverse
\endsong


%titolo{Questa è la mia fede}
%autore{Branca}
%album{Ad una voce}
%tonalita{Mi}
%gruppo{}
%momenti{Ingresso;Congedo}
%identificatore{questa_e_la_mia_fede}
%data_revisione{2011_12_31}
%trascrittore{Francesco Endrici}
\beginsong{Questa è la mia fede}[by={Branca}]
\beginchorus
\[E]Questa è la mia \[G#m]fede, procl\[A]amarti mio \[E]re,
\[C#m]unico \[E]Dio, \[A7]grande Si\[B]gnore.
\[E]Questa è la spe\[G#m]ranza so che \[A]risorge\[G#]rò
\[A]e in Te di\[B]more\[E]rò. \rep{2}
\endchorus
\beginverse
\[Em]Canterò la gioia di esser \[C7+]figlio, canterò
che Tu non \[G]abbandoni, \[C]non tradisci \[D]mai.
\[Em]Dammi sempre la tua grazia e in \[C7+]Te dimorerò
per ado\[G]rarti, per ser\[C]virti in veri\[D]tà, mio \[B]re.
\endverse
\beginverse
%\chordsoff
^Canterò che solo Tu sei ^vita e verità, 
che sei sal^vezza, che sei ^vera liber^tà.
^Io porrò la mia fiducia in ^te che sei la via,
cammine^rò nella Tua ^santa volon^tà, mio ^re.
\endverse
\endsong


%titolo{Questa famiglia}
%autore{Giombini}
%album{Salmi per il nostro tempo}
%tonalita{Re}
%gruppo{}
%momenti{Ingresso}
%identificatore{questa_famiglia}
%data_revisione{2011_12_31}
%trascrittore{Francesco Endrici - Manuel Toniato}
\beginsong{Questa famiglia}[by={Giombini}]

\beginchorus
\[D]Questa Fa\[A]miglia \[Bm]Ti \[E7]Bene\[A]dice,
\[G]Ti Bene\[D]dice Si\[C]gno\[A]re! 
\[D]Questa Fa\[A]miglia \[Bm]Ti \[E7]Bene\[A]dice,
\[G]Ti Bene\[D]dice Si\[A7]gno\[D]re! 
\endchorus

\beginverse
\[D]Ti benedice perché ci hai fatti incon\[A]trare,
perché ci hai \[Em]dato amore e gioia \brk per vivere in\[Bm]sieme,
Perché ci hai \[F#m]dato uno scopo per conti\[G]nuare:
questa fa\[D]miglia ti bene\[A7]dice\[G].
\endverse

\beginverse
\chordsoff
Ti benedice perché ci doni pazienza
e nel dolore ci dai la forza di sperare
Perché lavoro e pane non ci fai mancare:
questa famiglia ti benedice.
\endverse
\endsong




\lettera
%titolo{Rallegratevi nel Signore}
%autore{Colucci}
%album{Nelle tue mani}
%tonalita{Mi}
%gruppo{}
%momenti{}
%identificatore{rallegratevi_nel_signore}
%data_revisione{2011_12_31}
%trascrittore{Francesco Endrici - Manuel Toniato}
\beginsong{Rallegratevi nel Signore}[by={Colucci}]

\beginchorus
Ralle\[E]gratevi nel Signore sempre, rallegratevi 
\[A] non angustiatevi per \[B]nulla. 
Ralle\[E]gratevi nel Signore sempre, rallegratevi 
\[A] il Signore è vi\[B]cino. 
\endchorus

\beginverse
E la \[A]pace di \[B]Dio sa\[G#m]rà con \[C#m]voi,  \brk \[A]non angustiatevi per \[B]nulla. 
E la \[A]pace di \[B]Dio sa\[G#m]rà con \[C#m]voi, \[A]al\[B]lelu\[E]ia.
\endverse

\beginchorus
Ralle\[F]gratevi nel Signore sempre, rallegratevi 
\[B&] non angustiatevi per \[C]nulla. 
Ralle\[F]gratevi nel Signore sempre, rallegratevi 
\[B&] il Signore è vi\[C]cino. 
\endchorus
 
\beginverse 
E la \[B&]luce di \[C]Dio ri\[Am]splenda in \[Dm]voi,  \brk \[B&]non angustiatevi per \[C]nulla. 
E la \[B&]luce di \[C]Dio ri\[Am]splenda in \[Dm]voi,  \[B&]al\[C]lelu\[F]ia. 
\endverse

\beginverse*\itshape
\[F]Il Signore è la \[Gm]gioia, Alleluia
Il Signore è la speranza, Alleluia
Il Signore è libertà, Alleluia
Il Signore è verità, Alle\[E&7]lu\[D4]ia
\endverse

\beginchorus
Ralle\[G]gratevi nel Signore sempre, rallegratevi 
\[C] non angustiatevi per \[D]nulla. 
Ralle\[G]gratevi nel Signore sempre, rallegratevi 
\[C] il Signore è vi\[D]cino. 
\endchorus

\beginverse 
E la \[C]grazia di \[D]Dio di\[Bm]mori in \[Em]voi,  \brk \[C]non angustiatevi per \[D]nulla. 
E la \[C]grazia di \[D]Dio di\[Bm]mori in \[Em]voi,  \[C]al\[D]lelu\[G]ia. 
\endverse 

\beginverse* \itshape \bfseries
Ralle\[G]gratevi nel Signore sempre 
ralle\[D]gratevi nel Signore sempre. \rep{3}
Ralle\[G]gratevi nel Si\[G/F]gnore sempre 
\[C]perché il Signore è vi\[D]cino.
E la \[C]pace di \[D]Dio sa\[Bm]rà con \[Em]voi, 
\[C]non angustiatevi per \[D]nulla. 
E la \[C]pace di \[D]Dio sa\[Bm]rà con \[Em]voi, 
\[C]al\[D]lelu\[G]ia. 
\endverse 
\endsong


%titolo{Rallegriamoci}
%autore{Auricchio}
%album{Gloria all'Emmanuele}
%tonalita{Re}
%gruppo{}
%momenti{Ingresso;Congedo;Natale}
%identificatore{rallegriamoci}
%data_revisione{2011_12_31}
%trascrittore{Francesco Endrici}
\beginsong{Rallegriamoci}[by={Auricchio}]
\beginverse
\[G] Ralle\[D]griamoci, \[A] \brk non c'è \[G]spazio alla tri\[D]stezza in questo \[A]giorno, \[G]
Ralle\[D]griamoci, \[A] \brk è la \[G]vita che di\[D]strugge ogni pa\[A]ura \[G]
Ralle\[F#m]griamoci, \[G] \brk che si compie in questo \[D]giorno la pro\[A]messa
ralle\[Bm]griamoci, \[G] \brk ogni \[D]uomo lo ve\[A]drà: la salvezza di \[D]Dio. \[G]\[D]
\endverse
\beginchorus
Gloria a \[D]Te Emma\[G]nue\[A]le
Gloria a \[D]Te Figlio di \[G]Di\[A]o.
Gloria a \[Bm]Te Cristo Si\[G]gnore che nasci per \[D]noi
e torna la \[D]gio\[A]ia. \rep{2}
(e torna la \[D]\[A]gio\[D]ia.)
\endchorus
\beginverse
\chordsoff
Rallegriamoci, \brk Egli viene a liberarci da ogni male.
Rallegriamoci, \brk è il momento di gustare il suo perdono,
rallegriamoci, \brk con coraggio riceviamo la sua vita,
rallegriamoci, \brk perché è giunta in mezzo a noi la presenza di Dio.
\endverse
\beginverse
\chordsoff
Rallegriamoci, \brk tutti i popoli del mondo lo vedranno
rallegriamoci, \brk nel Signore è la nostra dignità.
Rallegriamoci, \brk nella luce del suo regno in cui viviamo,
Rallegriamoci, \brk siamo tempio vivo suo, siamo chiesa di Dio.
\endverse
\endsong


%titolo{Re di gloria}
%autore{Marranzino, De Luca}
%album{Cantiamo con gioia}
%tonalita{Sol}
%gruppo{}
%momenti{Comunione}
%identificatore{re_di_gloria}
%data_revisione{2011_12_31}
%trascrittore{Francesco Endrici - Manuel Toniato}
\beginsong{Re di gloria}[by={Marranzino, De\ Luca}]

\ifchorded
\beginverse*
\vspace*{-0.8\versesep}
{\nolyrics \[Am7] \[C] \[D] \[G] }
\vspace*{-\versesep}
\endverse
\fi
\beginverse
\[G]Ho incontrato Te Gesù \brk e ogni \[D]cosa in me è cambiata
\[Am]tutta la mia \[Am7]vita ora ti \[D]appar\[Am7]tie\[D]ne
\[G]tutto il mio passato io lo a\ch{C}{f}{f}{ff}i\[B7]do a \[Em]Te \[Em7] 
Ge\[Am]sù Re di \[C7+]gloria mio Si\[D4]gnor\[D7].
\endverse

\beginverse
\chordsoff
Tutto in Te riposa, la mia mente il mio cuore
trovo pace in Te Signor,Tu mi dai la gioia
voglio stare insieme a Te,non lasciarti mai
Gesù Re di gloria mio Signor.
\endverse

\beginchorus
\[D7]Dal Tuo a\[G]more chi \[C7+]mi separe\[D4]rà \[D] 
sulla \[Am7]croce hai \[C7+]dato la \[D4]vita per \[D]me
u\[D7]na co\[G]rona di \[C7+]gloria mi da\[D4]rai \[D] 
quando un \[Am7]gior\[C]no \[D]ti ve\[G]drò.
\endchorus

\beginverse
\chordsoff
Tutto in Te riposa,la mia mente il mio cuore
trovo pace in Te Signor, Tu mi dai la gioia vera
voglio stare insieme a Te, non lasciarti mai
Gesù Re di gloria mio Signor.
\endverse

\beginchorus
\[D7]Dal Tuo a\[G]more chi \[C7+]mi separe\[D4]rà \[D] 
sulla \[Am7]croce hai \[C7+]dato la vita per \[D]me
u\[D7]na co\[G]rona di \[C7+]gloria mi da\[D4]rai \[D] 
quando un \[Am7]gior\[C]no \[D]ti ve\[E&]drò. \[(Sol)]
\[E&7]Dal Tuo a\[Ab]more chi \[D&7+]mi separe\[E&]rà \ldots
\endchorus
\ifchorded
\beginverse*
\vspace*{-\versesep}
{\nolyrics \[B&m7] \[D&7+] \[E&]  \[E&7] \[Ab] \[D&7+] \[E&4] \[E&] }
\endverse
\fi
\beginverse*\bfseries
Io ti a\[B&m7]spet\[D&7+]to \[E&]mio Si\[Fm]gnor \[Ab] 
io ti a\[B&m7]spet\[Ab]to \[E&]mio \[C]Si\[Fm]gnor \[Ab] 
io ti a\[B&m7]spet\[Ab]to \[E&]mio \[Ab4]Re! \[Ab] 
\endverse
\textnote{Nel caso non si cambi tonalità, riportiamo il finale in Sol maggiore:}
\beginverse*\bfseries
\[D7]Dal Tuo a\[G]more chi \[C7+]mi separe\[D]rà \ldots
\endverse
\ifchorded
\beginverse*
\vspace*{-\versesep}
{\nolyrics \[Am7]  \[C7+]  \[D]  \[D7]  \[G]  \[C7+]  \[D4]  \[D] }
\endverse
\fi
\beginverse*\bfseries
Io ti a\[Am7]spet\[C7+]to \[D]mio Si\[Em]gnor \[G] 
io ti a\[Am7]spet\[G]to \[D]mio \[B]Si\[Em]gnor \[G] 
io ti a\[Am7]spet\[G]to \[D]mio \[G4]Re! \[G] 
\endverse

\endsong


%titolo{Receive the power}
%autore{Sebastian, Pinto}
%album{Inno della XXIII GMG, Sydney 2008}
%tonalita{Mi-}
%gruppo{}
%momenti{}
%identificatore{receive_the_power}
%data_revisione{2011_12_31}
%trascrittore{Francesco Endrici - Manuel Toniato}
\beginsong{Receive the power}[by={Sebastian, Pinto}]

\beginverse
\[Em]Ogni \[D]popolo e tri\[G]bù, si ingi\[C]noc\[G]chia da\[C]vanti a \[D]te.
\[Em]Esul\[D]tiamo in te, Ge\[G]sù, ti se\[C]guia\[G]mo: \brk solo \[C]tu sei la \[D]Via.
\endverse

\beginchorus
\[G]Alle\[G7+]luia, \[Em]alle\[Em2]luia
Re\[C]ceive the Power, from the \[D]Holy Spirit
\[G]Alle\[G7+]luia, \[Em]alle\[Em2]luia
Re\[C]ceive the Power to be a \[D]light unto the \[G]world
\endchorus

\beginverse
\chordsoff
Nello Spirito ci dai, la fortezza, la santità.
Nel battesimo ci fai, testimoni della tua fedeltà.
\endverse

\beginverse*\itshape
Vero \[C]Dio, la \[D]lode a \[G]te, nostro \[C]Re, la \[D]lode a \[G]te,
Dio-con-\[C]noi, la \[D]lode a \[G]te, Cari\[C]tà, la \[D]lode a \[G]te,
Vero \[C]Dio, la \[D]lode a \[G]te, nostro \[C]Re, la \[D]lode a \[G]te,
Dio-con-\[C]noi, la \[D]lode a \[G]te, Trini\[Am]tà, cante\[C]rò per \[D]sempre.
\endverse

\endsong



%titolo{Rendete grazie}
%autore{Smith}
%album{Come Davide}
%tonalita{Fa}
%gruppo{}
%momenti{Ringraziamento}
%identificatore{rendete_grazie}
%data_revisione{2011_12_31}
%trascrittore{Francesco Endrici - Manuel Toniato}
\beginsong{Rendete grazie}[by={Smith}]

\beginverse*
\[F]Rendete \[C]grazie a Co\[Dm]lui che è \[Am]Santo, 
ren\[B&]dete grazie a Dio \[F]per suo \[E&]Figlio \[Gm]Ge\[C7]sù! \rep{2}
E \[Am]possa il \[Dm]debole dir son \[Gm]forte e possa
il povero \[C]dir son \[F7+]ricco per quel che ha \[Dm]fatto il Si\[E&]gnore per \[C]me. \rep{2}
\endverse
\endsong





%titolo{Resta accanto a me}
%autore{Gen Verde}
%album{Cerco il tuo volto}
%tonalita{Mi}
%gruppo{}
%momenti{Congedo}
%identificatore{resta_accanto_a_me}
%data_revisione{2011_12_31}
%trascrittore{Francesco Endrici}
\beginsong{Resta accanto a me}[by={Gen\ Verde}]
\beginchorus
\[E]Ora \[B]vado \[A]sulla mia \[E]strada
\[F#m]con l'a\[G#m]more \[A]tuo che mi \[B]guida.
\[E]O Si\[B]gnore, o\[A]vunque io \[E]vada,
\[A]resta ac\[B]canto a \[E]me.
\[E]Io ti \[B]prego, \[A]stammi vi\[E]cino
\[F#m]ogni \[G#m]passo \[A]del mio cam\[B]mino,
\[E]ogni \[B]notte, \[A]ogni mat\[E]tino,
\[A]resta ac\[B]canto a \[E]me.
\endchorus
\beginverse
\[B]Il tuo sguardo \[A]puro sia luce per \[C#m]me
\[B]e la tua Pa\[A]rola sia voce per \[E]me.
\[A]Che io trovi il \[B]senso del mio andare
\[C#m]so\[B]lo in \[E]te, nel \[B]tuo fedele amare il mio per\[E]ché.
\endverse
\beginverse
^Fa' che chi mi ^guarda non veda che ^te.
^Fa' che chi mi a^scolta non senta che ^te,
^e chi pensa a ^me, fa' che nel cuore
^pen^si a ^te e ^trovi quell'amore \brk che hai dato a ^me.
\endverse
\endsong

%titolo{Resta qui con noi}
%autore{Gen Rosso}
%album{Una storia che cambia}
%tonalita{Re}
%gruppo{}
%momenti{Congedo}
%identificatore{resta_qui_con_noi}
%data_revisione{2011_12_31}
%trascrittore{Francesco Endrici}
\beginsong{Resta qui con noi}[by={Gen\ Rosso}]
\beginverse
Le \[D]ombre si distendono, \[G]scende ormai la sera,
\[D]e s'allontanano dietro i \[Em]monti
i riflessi di un \[Bm]giorno che non \[F#m]finirà,
di un \[E]giorno che ora \[G]correrà \[A]sempre, \[D]
Perché sap\[F#m]piamo che \[G]una nuova \[Em]vita
da qui è par\[D]tita e mai \[G]più si ferme\[A4]rà. \[A]
\endverse
\beginchorus
\[D]Resta qui con \[F#m]noi, il \[G]sole scende \[D]già,
\[Em]resta qui con \[A]noi, Si\[G]gnore è \[A]sera or\[D]mai.
\[D]Resta qui con \[F#m]noi, il \[G]sole scende \[D]già,
\[Em]se tu sei fra \[A]noi, la \[G]notte \[A]non ver\[D]rà.
\endchorus
\beginverse
%\chordsoff
S'al^larga verso il mare ^il tuo cerchio d'onda,
^che il vento spingerà fino a ^quando
giungerà ai con^fini di ogni ^cuore,
alle ^porte dell'a^more ^vero. ^
Come una ^fiamma che ^dove passa ^brucia
così il tuo a^more tutto il ^mondo invade^rà. ^
\endverse
\beginverse
%\chordsoff
Da^vanti a noi l'umanità ^lotta, soffre e spera
^come una terra che nell'ar^sura
chiede l'acqua da un ^cielo senza ^nuvole,
ma che ^sempre le può ^dare ^vita. ^
Con te sa^remo sor^gente d'acqua ^pura,
con te fra ^noi il de^serto fiori^rà. ^
\endverse
\endsong

%titolo{Restate qui}
%autore{}
%album{}
%tonalita{Fa}
%gruppo{}
%momenti{Ritornelli}
%identificatore{restate_qui}
%data_revisione{2012_01_11}
%trascrittore{Francesco Endrici}
\beginsong{Restate qui}
\beginverse*
\[F]Restate \[Am]qui ve\[Dm]gliate con \[Am]me,
\[B&]voglio do\[F]narvi il mio a\[C4]mo\[C]re;
\[F]Restate \[Am]qui ve\[Dm]gliate con \[Am]me,
\[B&]voglio riem\[Gm]pirvi di \[B&]gio\[F]ia.
\endverse
\endsong
%titolo{Resurrezione}
%autore{}
%album{}
%tonalita{Sol}
%gruppo{}
%momenti{Pasqua}
%identificatore{resurrezione_la_notte}
%data_revisione{2011_12_31}
%trascrittore{Francesco Endrici}
\beginsong{Resurrezione}
\beginchorus
\[C]Alleluia alleluia \[G] \[D]Alleluia alleluia
\[G]Alleluia \[G7]alleluia \[C]alleluia alleluia \[G]alle\[D7]lu\[G]ia.
\endchorus
\beginverse
La \[G]notte, la notte se n'è an\[C]data via
la \[G]notte, la notte non c'è \[D]più.
\[G]Tu ci hai donato l'\[C]alba nuova
\[G]Cristo Si\[D]gnore Ge\[G]sù.
\endverse
\beginverse
\chordsoff
Cantate cantate al Signore della vita
cantate cantate a Lui
Egli ha sconfitto la morte per sempre
grande è la sua bontà.
\endverse
\beginchorus
\chordsoff 
Rit. 
\endchorus
\ifchorded
\beginverse*
\vspace*{-\versesep}
{\nolyrics \[E7]}
\endverse
\fi
\beginverse
\transpose{2}
Lo^date lodate il Si^gnore voi viventi
lo^datelo con veri^tà
^tutte le genti ve^dranno la sua gloria
^gioia per l'^umani^tà.
\endverse
\beginverse
\chordsoff
Cantate cantate al Pastore della vita
risorto dai morti per noi
che ci conduce ai pascoli eterni
amen alleluia.
\endverse
\beginchorus
\transpose{2}
\[C]Alleluia alleluia \[G] \[D]Alleluia alleluia
\[G]Alleluia \[G7]alleluia \[C]alleluia alleluia \[G]alle\[D7]lu\[G]ia.
\endchorus
\endsong

%titolo{Ricordate}
%autore{Spoladore}
%album{Tu sei}
%tonalita{Sol}
%gruppo{}
%momenti{}
%identificatore{ricordate}
%data_revisione{2011_12_31}
%trascrittore{Francesco Endrici - Manuel Toniato}
\beginsong{Ricordate}[by={Spoladore}]

\beginverse
Non ab\[G]bando\[Am7]nate \[G/B]mai
la \[C]strada che \[G/B]porta alla \[Am7]{Vi}\[G]ta.   \[Am7] \[G] \[Am7] 
\[G]Non rincor\[Am7]rete la \[G/B]strada
la \[C]strada che \[G/B]porta alla \[Am7]mor\[G]te.  \[Am7] \[G] \[Am7] 
\[G]Non distrug\[Am7]gete voi \[G/B]stessi
e la \[C]luce immor\[G/B]tale \brk che è \[Am7]dentro di \[G]voi.\[Am7]  \[G] \[Am7] 
\endverse

\beginchorus
\[D]Ricordate sempre, \[C9]Dio non ha creato la morte
ma ha \[D]posto in tutte le cose \brk un \[C9]seme di vita e di luce.
E non \[G]vuole la morte degli uomini \brk degli uomini \[D]mai.
\endchorus

\beginverse
\chordsoff
Dio ha creato ogni cosa
perché tutto esista e non muoia mai.
Ogni forza presente nel mondo
è per la Vita e per la Vita sarà.
La Vita non è un ombra che passa
ma è un dono che non finirà.
\endverse
\endsong

%titolo{Riempici di te}
%autore{Spoladore}
%album{Tu sei}
%tonalita{Re-}
%gruppo{}
%momenti{Pentecoste}
%identificatore{riempici_di_te}
%data_revisione{2011_12_31}
%trascrittore{Francesco Endrici - Manuel Toniato}
\beginsong{Riempici di te}[by={Spoladore}]

\beginchorus
\[Dm]Riempici di Te \[C]Padre Creatore, \brk \[F]Riempici di Te \[B&9]Figlio Salvatore
\[F]Riempici di Te \[C9]Spirito d'Amore, \brk \[B&7+]riempici di Te
\endchorus

\beginverse*
Sciogli il \[Dm7]cuore dei tuoi \[C]figli \brk dalle ca\[F]tene dell'in\[B&9]ganno
dalla \[F]cieca indiffe\[C9]renza dalla va\[B&9]nità del mondo.

Sciogli il \[Dm7]cuore dei tuoi \[C]figli con la \[F]vera liber\[B&9]tà
con la dol\[F]cezza del per\[C9]dono con la \[B&7+]lieta povertà.
\endverse
\endsong


%titolo{Rimani con me}
%autore{Gen Rosso}
%album{}
%tonalita{Sol}
%gruppo{}
%momenti{Maria}
%identificatore{rimani_con_me}
%data_revisione{2011_12_31}
%trascrittore{Francesco Endrici}
\beginsong{Rimani con me}[by={Gen\ Rosso}, ititle={Maria tu sei}]
\beginverse
Maria, tu \[G]sei la vita per \[D]me, \[D4]
sei la spe\[C]ranza, la gioia, l'a\[D]more, \[G]tutto sei.
Maria, tu sai quello che \[D]vuoi,
sai con che \[C]forza d'amore in \[D]cielo mi \[G]porterai.
\endverse
\beginchorus
Maria ti \[Am]do
il mio cuore per \[C]sempre se vuoi.
Tu dammi l'a\[G]more che non passa \[D]mai.
Rimani con \[Em]me \brk e andiamo nel \[D]mondo insieme.
La tua pre\[C]senza sarà
goccia di para\[G]diso per l'umani\[D]tà.
\endchorus
\beginverse
\chordsoff
Maria, con te sempre vivrò
in ogni momento, giocando, cantando, ti amerò.
Seguendo i tuoi passi in te io avrò la luce che
illumina i giorni e le notti dell'anima.
\endverse
\endsong

%titolo{Risurrezione}
%autore{Gen Rosso}
%album{La vita di ogni cosa}
%tonalita{Re}
%gruppo{}
%momenti{Pasqua}
%identificatore{risurrezione_gen}
%data_revisione{2011_12_31}
%trascrittore{Francesco Endrici}
\beginsong{Risurrezione}[by={Gen\ Rosso}]
\beginverse
Che \[D]gioia ci hai \[G]dato, Si\[D]gnore del \[G]cielo,
Si\[D]gnore del \[G]grande uni\[A4]ver\[7]so,
che \[D]gioia ci hai \[G]dato, ve\[D]stito di \[G]luce,
ve\[D]stito di \[A7]gloria infi\[Bm]ni\[G]ta,
ve\[D]stito di \[A7]gloria infi\[G]ni\[D]ta. \[G]\[D]\[G]
\endverse
\beginverse
%\chordsoff
Ve^derti ri^sorto, ve^derti Si^gnore
il ^cuore sta ^per impaz^zi^re
Tu ^sei ritor^nato, Tu ^sei qui tra ^noi
e a^desso ti a^vremo per ^sem^pre.
E a^desso ti a^vremo per ^sem^pre. ^
\endverse
\beginverse
^ Chi cercate, ^donne, quag^giù,
chi cercate, ^donne, quag^giù,
quello ch'era ^morto non è ^qui ^
è ri^sorto! Sì, come a^veva detto ^anche a voi.
^Voi gridate a ^tutti ^che ^è risorto ^Lui,
^tutti ^che è ^risorto ^Lui. ^
\endverse
\beginverse
%\chordsoff
^ Tu hai vinto il ^mondo, Ge^sù,
Tu hai vinto il ^mondo, Ge^sù,
liberiamo ^la felici^tà ^
e la ^morte, no, non e^siste più, l'hai ^vinta Tu 
e ^hai salvato ^tutti noi, ^ ^uomini con ^Te, 
^tutti noi,^ ^uomini con ^Te.
\endverse
\beginverse*
\[G]Uomini con \[G]te, \[G]uomini con \[G]te
\textit{che \[D]gioia ci hai \[G]dato, ti a\[D]vremo per \[G]sem\[D]pre.}
\endverse
\endsong

%titolo{Riuniti nel suo nome}
%autore{Zardini}
%album{}
%tonalita{Sol}
%gruppo{}
%momenti{Comunione}
%identificatore{riuniti_nel_suo_nome}
%data_revisione{2011_12_31}
%trascrittore{Francesco Endrici - Manuel Toniato}
\beginsong{Riuniti nel suo nome}[by={Zardini}]
\beginverse
Padre \[C]nostro, Padre \[D]buono: 
nel tuo \[C]nome ci ra\[D]duni nel tuo \[C]Figlio ci re\[B]dimi
nello \[Em]Spirito ci u\[D]nisci in un \[Am]popolo pre\[D]gante.
\endverse

\beginchorus
Dove \[G]siamo  riu\[D]niti \[B]nel Suo \[Em]nome, \[C] 
Dio ver\[Am]rà, \[D] Dio ver\[G]rà 
\[C] e in mezzo a \[Am]noi a\[Em]bi\[Am]te\[B]rà. \[E] 
Dio ver\[Am]rà, \[D]Dio ver\[G]rà 
\[C] e in mezzo a \[Am]noi a\[C]bi\[D]te\[G]rà.
\endchorus

\beginverse
\chordsoff
Padre nostro, Padre buono:
a noi venga sulla terra, il tuo regno nella pace;
il tuo regno nella gloria, a noi venga su nel cielo.
\endverse
\endsong

\lettera
%titolo{Saldo è il mio cuore}
%autore{Remigi, Martinez}
%album{Vittoria}
%tonalita{Re}
%gruppo{}
%momenti{Ingresso;Congedo;Salmi}
%identificatore{saldo_e_il_mio_cuore}
%data_revisione{2011_12_31}
%trascrittore{Francesco Endrici - Manuel Toniato}
\beginsong{Saldo è il mio cuore}[by={Remigi, Martinez}]

\ifchorded
\beginverse*
\vspace*{-0.8\versesep}
{\nolyrics \[Bm7] \[F#m7] \[G] \[D] \[Em7] \[D] \[G] }
\vspace*{-\versesep}
\endverse
\fi
\beginchorus
\[D]Saldo è il mio \[G]cuore, \[D]Dio, \[G]  
\[D]voglio can\[G]tare al Si\[A4]gnor.   \[A7] 
\[Bm7]Svegliati mio \[F#m7]cuore, sve\[G]gliatevi arpa e \[D]cetra,
\[Em]voglio sve\[E]gliare l'au\[F#4/7]ro\[F#]ra. \[F#] 
\[Bm7]Svegliati mio \[F#m7]cuore, sve\[G]gliatevi arpa e \[D]cetra,
\[Em7]voglio sve\[D]gliare l'au\[G]ro\[D]ra. \[G] \[D] 
\endchorus

\beginverse
\[G]Ti loderò tra i popoli, Si\[D4]gno\[D]re,
a \[G]te cante\[A]rò inni fra le \[Gm6]gen\[D]ti,
poi\[G]ché la tua bontà è \[F#4/7]grande fi\[F#7]no ai \[Bm7]cie\[E]li
e \[Em7]la tua fedel\[G]tà fino alle \[A7/6]nu\[A9]bi.
\endverse

\beginverse
\chordsoff
Innalzati sopra i cieli, o Signore,
su tutta la terra splenda la tua gloria.
Perché siano liberati i tuoi amici,
salvaci con la tua potente destra.
\endverse

\beginverse
\chordsoff
Dio ha parlato nel suo santuario,
contro il nemico ci darà soccorso.
Con Dio noi faremo cose grandi
ed egli annienterà chi ci opprime.
\endverse
\endsong

%titolo{Salve Regina}
%autore{Gen Verde}
%album{È bello lodarti}
%tonalita{Sol}
%gruppo{}
%momenti{Maria}
%identificatore{salve_regina_gen}
%data_revisione{2011_12_31}
%trascrittore{Francesco Endrici}
\beginsong{Salve Regina}[by={Gen\ Verde}]
\beginverse*
\[G]Salve Re\[D]gina, \[C]Madre di miseri\[G]cordia
vita dol\[D]cezza speranza \[C]nostra \[G]salve
Salve Re\[D]gina. \[D7] \rep{2}
\[G] A te ricor\[Am]riamo \[D]esuli figli di \[G]Eva
A te sospi\[Bm]riamo
pian\[C]genti in questa valle di \[D]lacrime.
\[G] Avvocata \[Am]nostra
\[D]volgi a noi gli occhi \[G]tuoi
mostraci dopo questo e\[Bm]silio
il \[C]frutto del tuo seno Ge\[D]sù.
\[G]Salve Re\[D]gina, \[C]Madre di miseri\[G]cordia
o Clemente, o \[D]Pia, o dolce \[C]Vergine Ma\[G]ria.
Salve Re\[D]gina. \[D7]
\[G]Salve Re\[D]gina. \[C]Sal\[G]ve. \[C]Sal\[G]ve.
\endverse
\endsong

%titolo{San Francesco}
%autore{Spoladore}
%album{Così}
%tonalita{Re}
%gruppo{}
%momenti{Comunione;San Francesco}
%identificatore{san_francesco}
%data_revisione{2011_12_31}
%trascrittore{Francesco Endrici}
\beginsong{San Francesco}[by={Spoladore}]
\beginverse
O Si\[D]gnore fa' di \[D5]me un tuo stru\[D6]mento, \[D5]
fa' di \[D]me uno stru\[D5]mento della tua \[G]pace, \[B7]
dov'è \[Em]odio che io \[Gm]porti l'a\[A]more,
dov'è o\ch{B&}{f}{f}{ff}esa che io \[A]porti il per\[D]dono, \[G]\[A]
dov'è \[D]dubbio che io \[D5]porti la \[D6]fede, \[D5]
dov'è di\[D]scordia che io \[D5]porti l'u\[G]nione, \[B7]
dov'è er\[Em]rore che io \[A]porti veri\[F#m]tà, \[B7]
a chi di\[Em]spera che io \[A]porti la spe\[D]ranza. \[D7]
Dov'è er\[G]rore che io \[A]porti veri\[F#m]tà, \[B7]
a chi di\[Em]spera che io \[A]porti la spe\[D]ranza. \[D7]
\endverse
\beginchorus
O Ma\[G]estro dammi \[B7]tu un cuore \[Em]grande, \[G7]
che sia \[C]goccia di ru\[Am7]giada per il \[D]mondo, \[D7]
che sia \[C]voce \[D]di speranza,
\[Bm]che sia un \[Em]buon mattino
\[C]per il \[D]giorno di ogni \[G]uomo. \[7]
\[C]E con gli \[D]ultimi del \[Bm]mondo
\[Em]sia il mio passo \[C7+]lieto \[D]nella pover\[G]tà,
\[Am]nella pover\[D7]tà. ( \[Am]Nella \[Cm]pover\[G]tà)
\endchorus
\beginverse
%\chordsoff
O Si^gnore fa' di ^me il tuo ^canto, ^
fa' di ^me il tuo ^canto di ^pace; ^
a chi è ^triste che io ^porti la ^gioia,
a chi è nel ^buio che io ^porti la ^luce. ^^
È do^nando che si ^ama la ^vita, ^
è ser^vendo che si ^vive con ^gioia, ^
perdo^nando che si ^trova il per^dono, ^
è mo^rendo che si ^vive in e^terno. ^
Perdo^nando che si ^trova il per^dono, ^
è mo^rendo che si ^vive in e^terno. ^
\endverse
\endsong

%titolo{Santa Maria del cammino}
%autore{D'Andrea}
%album{RnS 60}
%tonalita{Do}
%gruppo{}
%momenti{Maria}
%identificatore{santa_maria_del_cammino}
%data_revisione{2011_12_31}
%trascrittore{Francesco Endrici}
\beginsong{Santa Maria del cammino}[by={D'Andrea}]
\beginchorus
\[C]Mentre trascorre la \[F]vi\[C]ta
\[G]solo tu \[7]non sei \[C]mai \[7]
\[F]Santa Ma\[Dm]ria del cam\[C]mi\[Am]no
\[Dm]sempre sa\[G7]rà con \[C]te. \[7]
\endchorus
\beginverse
\[F]Vieni, o Madre, in \[C]mezzo a noi
\[G]vieni Ma\[7]ria quag\[C]giù, \[7]
\[F]cammineremo in\[Em]sieme a \[Am]te
\[Dm]verso la \[G7]liber\[C]tà.
\endverse
\beginverse
\chordsoff
Quando qualcuno ti dice: “nulla mai cambierà”,
lotta per un mondo nuovo, lotta per la verità.
\endverse
\beginverse
\chordsoff
Lungo la strada la gente chiusa in sé stessa va;
offri per primo la mano a chi è vicino a te.
\endverse
\beginverse
\chordsoff
Quando ti senti ormai stanco \brk e sembra inutile andar,
tu vai tracciando un cammino: un altro ti seguirà.
\endverse
\endsong

%titolo{Scende la sera}
%autore{Gen Verde}
%album{Prime pagine}
%tonalita{Re}
%gruppo{}
%momenti{}
%identificatore{scende_la_sera}
%data_revisione{2011_12_31}
%trascrittore{Francesco Endrici}
\beginsong{Scende la sera}[by={Gen\ Verde}]
\beginverse
\[D]Scende se\[A]rena la \[Bm]sera
ca\[Bm7]rezza leg\[G]gera \[Em7]
che la città at\[A4]ten\[A]de.
\[D]Scende più \[Em7]dolce sta\[Bm7]sera
come \[D]questa pre\[G]ghiera \[Em7]
che l'anima ac\[A4]cen\[A]de.
Per \[F7+]te, non basta que\[G]sta città
smargi\[F7dim]nato universo
il tuo \[E7]cuore ha sognato
ha \[Am7]chiesto di \[A]più,
do\[F7+]mani chis\[G]sà
fino ai \[Em7]limiti del \[Am7]mondo
un'\[B&7+]onda di vita an\[Em]drà. \[A4 A]
\endverse
\beginchorus
\[D]Quando il vento s'alza e va
chi \[Bm]mai sa \[7]come?
\[G6]Soffia dove vuole e \[Em7]va
chi \[A4]mai sa \[A]dove?
\[D]Sopra mille strade va
\[F#m7]laggiù \[Bm7] lon\[7]tano, \[G]
chi mai fer\[Em7]marlo po\[A4]trà? \[A]
\endchorus
\beginverse
\chordsoff
Cuore che abita i cieli
Le case i cortili
Le vie i grattacieli
Cuore poeta del vento
Di stelle e del cemento
Dei bimbi e del tempo
Per te\dots
\endverse
\beginchorus
\chordsoff
Quando il vento s'alza e va\dots
\endchorus
\beginverse
\chordsoff
[Scende — Quando il vento —
Cuore — Quando il vento]
\endverse
\beginchorus
\[F]Quando il vento s'alza e va e va
\[D7+]chi mai fer\[E]marlo po\[F#]trà?
\endchorus
\endsong



%titolo{Scusa Signore}
%autore{Biagioli, Aliscioni}
%album{Celebriamo la nostra speranza}
%tonalita{Mi}
%gruppo{}
%momenti{Riconciliazione;Quaresima}
%identificatore{scusa_signore}
%data_revisione{2011_12_31}
%trascrittore{Francesco Endrici}
\beginsong{Scusa Signore}[by={Biagioli, Aliscioni}]
\beginverse
\[E]Scusa, Si\[B7]gnore, se bus\[E]siamo
alla \[A]porta del tuo \[E]cuore, siamo \[B7]noi.
\[E]Scusa, Si\[B7]gnore, se chie\[E]diamo,
mendi\[A]canti dell'a\[E]more,
un ri\[B7]storo da \[E]te. \[B7] 
\endverse
\beginchorus
Co\[E]sì la \[F#m]foglia quando è \[A]stanca cade \[E]giù,
ma \[C#m]poi la \[G#m]terra ha una \[A]vita sempre in \[B7]più.
Co\[E]sì la \[F#m]gente quando è \[A]stanca vuole \[E]te,
e \[C#m]tu, Si\[G#m]gnore hai una \[A]vita sempre in \[B7]più,
sempre in \[E]più.
\endchorus
\beginverse
\chordsoff
Scusa, Signore, se entriamo
nella reggia della luce, siamo noi.
Scusa, Signore, se sediamo
alla mensa del tuo corpo
per saziarci di te.
\endverse
\beginverse
\chordsoff
Scusa, Signore, quando usciamo
dalla strada del tuo amore, siamo noi.
Scusa, Signore, se ci vedi
solo all'ora del perdono
ritornare da te.
\endverse
\endsong

%titolo{Se avessi mai commesso}
%autore{S. Teresa di Lisieux}
%album{}
%tonalita{Sol}
%gruppo{}
%momenti{}
%identificatore{se_avessi_mai_commesso}
%data_revisione{2011_12_31}
%trascrittore{Francesco Endrici}
\beginsong{Se avessi mai commesso}[by={S.\ Teresa\ di\ Lisieux}]
\beginverse
\[G]Se avessi mai com\[C]messo il peg\[D]giore dei \[G]crimini
per \[Em]sempre manter\[D]rei la \[G]stessa fi\[D]ducia
poi\[Em]ché io so che \[C]questa molti\[D]tudine di o\ch{Em}{f}{f}{ff}ese
non \[C]è che goccia d'\[G]acqua in \[D7]un braciere ar\[Em]dente
non \[C]è che goccia d'\[G]acqua in \[D7]un braciere ar\[G]dente.
\endverse
\beginverse
\chordsoff
Oh, se potessi aver un cuore ardente d'amore
che resti il mio sostegno, non m'abbandoni mai
che ami tutto in me, persino la mia debolezza
e non mi lasci mai, né il giorno né la notte.
e non mi lasci mai, né il giorno né la notte.
\endverse
\beginverse
\chordsoff
Non ho trovato mai creatura capace
d'amarmi a tal punto, e senza mai morire
di un Dio ho bisogno, che assunta la mia natura
si faccia mio fratello capace di soffrir.
si faccia mio fratello capace di soffrir.
\endverse
\beginverse
\chordsoff
Io so fin troppo bene che le nostre giustizie
non hanno ai tuoi occhi il minimo valore
ed io per dare un prezzo ad ogni mio sacrificio
gettare lo vorrei nel tuo divino cuor,
gettare lo vorrei nel tuo divino cuor.
\endverse
\beginverse
\chordsoff
No, tu non hai trovato creatura senza macchia
dettasti la tua legge, tra i fulmini del cielo
e nel tuo Sacro Cuore, Gesù mi nascondo
non tremo perché sei la sola mia virtù,
non tremo perché sei la sola mia virtù.
\endverse
\endsong


%titolo{Se cade una stella}
%autore{Nobili}
%album{}
%tonalita{La-}
%gruppo{}
%momenti{Congedo}
%identificatore{se_cade_una_stella}
%data_revisione{2011_12_31}
%trascrittore{Francesco Endrici}
\beginsong{Se cade una stella}[by={Nobili}]
\beginverse\[Am]Come il cielo a volte è scuro, \brk non vuole la\[Am7]sciare
spazio al \[Am]sole che freme lassù, sopra le \[E]nubi
così \[F]spesso è questa \[Em]vita, \[Dm]piena di mille pro\[Am]blemi \brk e la \[F]sera restiamo da \[Em]soli,
vuoto il \[Dm]cuore senza più a\[E7]more. \[C]
\endverse
\beginchorus
Ma niente potrà fer\[F]mare la voglia di andare
basta solo un i\[C]stante per tornare sperare
nell'amore di un \[G]Padre che si fa sen\[F]tire,
che si fa tro\[C]vare. \[C7]
Se poi cade una \[F]stella
e in alto nel \[G]cielo
non può più tor\[Am]nare, \[G]che ne sa\[F]rà?
Se cade una \[C]stella di certo, Si\[G]gnore,
l'accoglierai \[C]tu. 
\[C7]Se poi cade una \[F]stella
e in alto nel \[G]cielo
non può più tor\[Am]nare, \[G]che ne sa\[F]rà?
Siamo noi quella \[C]stella,ti prego, Si\[G]gnore, \brk accoglici \[F]Tu. \[C]\[E7]
\endchorus\beginverse
\chordsoff
Quando tutto ormai sembra lontano \brk e non ha più senso 
c'è qualcosa che nasce da dentro, che non dà pace 
è il ricordo del primo respiro, \brk nostalgia di un amore infinito
ogni gesto riprende colore, \brk nuova vita si accende nel cuore.
\endverse
\endsong

%titolo{Se il chicco di frumento}
%autore{Machetta}
%album{Una voce che ti cerca}
%tonalita{Re}
%gruppo{}
%momenti{Offertorio;Quaresima}
%identificatore{se_il_chicco_di_frumento}
%data_revisione{2011_12_31}
%trascrittore{Francesco Endrici - Manuel Toniato}
\beginsong{Se il chicco di frumento}
\ifchorded
\beginverse*
\vspace*{-0.8\versesep}
{\nolyrics \[D] \[Em] \[F#] \[Bm] \[Em] \[A7] \[D] \[G] \[D] }
\vspace*{-\versesep}
\endverse
\fi

\beginchorus
Se il \[D]chicco di fru\[F#m]mento \[Em6]  
non \[A7]cade nella \[Bm]terra e non \[F#m]muore 
ri\[G]mane da \[D]solo \[Bm] 
se \[Em]muore \[A7]cresce\[D]rà. \[G] \[D] 
\endchorus

\beginverse
\[Bm]Troverà la sua \[F#m]vita 
\[G]chi la perde per \[A]me
\[D]Viene la \[G]prima\[F#m]{ve}\[Bm]ra, 
l'in\[Em]verno \[A7]se ne \[D]va. \[G]  \[D] 
\endverse

\beginverse
\chordsoff
Come il tralcio che piange, 
anche tu fiorirai
Viene la primavera, 
l'inverno se ne va.
\endverse
\endsong


%titolo{Se m'accogli}
%autore{Sequeri}
%album{Eppure tu sei qui}
%tonalita{Do}
%gruppo{}
%momenti{Offertorio}
%identificatore{se_m_accogli}
%data_revisione{2011_12_31}
%trascrittore{Francesco Endrici}
\beginsong{Se m'accogli}[by={Sequeri}]
\beginverse
Tra le \[C]mani non ho \[F]niente \brk spero \[Em]che mi accoglie\[Am]rai
chiedo \[C]solo di re\[F]stare accanto a \[G4]Te. \[G]
Sono \[C]ricco sola\[F]mente dell'a\[Em]more che mi \[Am]dai:
è per \[C]quelli che non l'\[F]hanno avuto \[G4]mai. \[G]
\endverse
\beginchorus
Se m'ac\[C]cogli, mio Si\[E7]gnore, altro \[Am]non Ti chiede\[C7]rò
e per \[F]sempre la Tua \[Dm]strada la mia \[E7]strada reste\[Am]rà
nella \[F]gioia e nel do\[C]lore fino a \[F]quando \[Dm]Tu vor\[E]rai
con la \[Am]mano nella \[G]Tua cammine\[F]rò. \[C]
\endchorus
\beginverse
%\chordsoff
Io Ti ^prego con il ^cuore, so che ^Tu mi ascolte^rai
rendi ^forte la mia ^fede più che ^mai. ^
Tieni ac^cesa la mia ^luce fino al ^giorno che Tu ^sai, 
con i ^miei fratelli in^contro a Te ver^rò. ^
\endverse
\endsong

%titolo{Segni del Tuo amore}
%autore{Gen Verde, Gen Rosso}
%album{Come fuoco vivo}
%tonalita{Do}
%gruppo{}
%momenti{Offertorio}
%identificatore{segni_del_tuo_amore}
%data_revisione{2011_12_31}
%trascrittore{Francesco Endrici}
\beginsong{Segni del Tuo amore}[by={Gen\ Verde, Gen\ Rosso}]
\ifchorded
\beginverse*
\vspace*{-0.8\versesep}
{\nolyrics \[C]\[Dm]\[C]\[Dm]\[Em]\[Dm]\[C]\[Dm]}
\vspace*{-\versesep}
\endverse
\fi
\beginverse
\[C]Mille e mille grani nelle spighe \[Dm]d'o\[C]\[Dm]ro
\[C]mandano fragranza e danno gioia al \[Dm]cuo\[C]\[Dm]re,
\[C]quando, macinati, fanno un pane \[Dm]so\[C]\[Dm]lo,
\[C]pane quotidiano, dono tuo, Si\[Dm]gno\[C]re.
\endverse
\beginchorus\[G]Ecco il pane e il vino, segni del tuo a\[F]mo\[C]re.
\[G]Ecco questa offerta, accoglila Si\[F]gno\[C]re,
\[F]tu di mille e mille \[G]cuori fai un cuore \[C]solo,
un corpo solo in \[G]te
e il \[F]Figlio tuo verrà, vi\[G]vrà ancora in mezzo a \[C]noi.
\endchorus
\beginverse
\chordsoff
Mille grappoli maturi sotto il sole, 
festa della terra donano vigore, 
quando da ogni perla stilla il vino nuovo, 
vino della gioia, dono tuo, Signore.
\endverse
\endsong

%titolo{Segni nuovi}
%autore{Gen Rosso}
%album{}
%tonalita{Mi}
%gruppo{}
%momenti{Congedo}
%identificatore{segni_nuovi}
%data_revisione{2011_12_31}
%trascrittore{Francesco Endrici}
\beginsong{Segni nuovi}[by={Gen\ Rosso}]
\ifchorded
\beginverse*
\vspace*{-0.8\versesep}
{\nolyrics \[C#m]\[B]\[E]\[A]\[A]\[A]\[A]\[E]}
\vspace*{-\versesep}
\endverse
\fi
\beginverse
\memorize
E segni \[E]nuovi \[A]oggi \[B]nascono \[E]già
e c'è più \[E]sole \[A]nelle \[B]nostre cit\[C#m]tà
il mondo u\[C#m]nito \[A]splende \[B]qui fra di \[E]noi
è un ide\[E]ale \[A]che la \[B]storia fa\[C#m]rà
un ide\[B]ale che \[E]storia si \[A]fa. \[A]\[A]\[A]
\endverse
\beginchorus
\[C#m]Vedo cambiare le cose che stanno at\[A]torno 
\[C#m]crollare muri e barriere fin dal pro\[A]fondo 
vedo \[B]gente che vive la \[E]vita 
vedo in \[C#m]alto, un \[A]cielo \[B]chiaro.
\endchorus
\beginverse
\chordsoff
Ormai non è così strano sentir parlare
di una casa comune dove abitare
e l'amore fra noi lo farà
invadendo il mondo intero.
\endverse
\endsong

%titolo{Sei il mio pastore}
%autore{}
%album{}
%tonalita{Do}
%gruppo{}
%momenti{Ritornelli}
%identificatore{sei_il_mio_pastore}
%data_revisione{2012_01_11}
%trascrittore{Francesco Endrici}
\beginsong{Sei il mio pastore}
\beginverse*
\[C]Sei il \[F]mio pa\[C]store, \[Am]nulla mi \[Dm]manche\[G]rà,
\[Dm]sei il \[E]mio pa\[Am]store, \[F]nulla mi \[G]manche\[C]rà.
\endverse
\endsong
%titolo{Sei il mio pastore}
%autore{Spoladore}
%album{Siamo nati liberi}
%tonalita{Re}
%gruppo{}
%momenti{Comunione;Salmi}
%identificatore{sei_il_mio_pastore_spoladore}
%data_revisione{2011_12_31}
%trascrittore{Francesco Endrici - Manuel Toniato}
\beginsong{Sei il mio pastore}[by={Spoladore}]

\beginverse
Il Si\[D]gnore è il mio pastore: \brk io non \[G]manco mai di nulla
e su \[D]prati d'erba fresca \[G]Lui mi fa' riposare
e ad \[D]acque tranquille mi conduce, \brk mi rin\[G]franca e mi dà forza.
Lui mi \[D]guida per il buon sentiero \brk per A\[G]more dei suo nome. \[A] 
\endverse
\beginverse*
\chordsoff
Anche se un giorno andassi per la valle più oscura
io di nulla avrei paura perché sempre sei con me.
Io mi appoggio al tuo bastone \brk con fiducia e sicurezza
mi abbandono al tuo Amore \brk mi abbandono tutto a Te.
\endverse

\beginchorus
E sei \[F]Tu il \[C]mio Pa\[B&]store, e sei \[Gm]Tu il \[B&]mio Pa\[C]store \[A] 
e sei \[Dm]Tu il \[Am]mio pa\[B&]{sto}\[F]re
mia vita mio \[B&]sole mia sola spe\[F]ranza 
o mio Si\[C]gnore. \rep{2} \[(B&)] 
\endchorus

\beginverse
\chordsoff
Tu prepari per me un banchetto \brk una festa nella tua gioia
Tu prepari per me un banchetto \brk sotto gli occhi dei miei nemici;
e con olio profumato Tu profumi il mio capo
e con olio profumato fai risplendere il mio volto.
\endverse
\beginverse*
\chordsoff
Il mio calice trabocca e trabocca della tua Vita
e lo riempi fino all'orlo e lo riempi della tua gioia
mie compagne saranno sempre \brk la felicità e il tuo Amore
e abiterò per sempre nella casa tua Signore.
\endverse

\endsong

%titolo{Sei per noi cibo di eternità}
%autore{Buttazzo}
%album{Sorgente di unità}
%tonalita{Do}
%gruppo{}
%momenti{Comunione}
%identificatore{sei_per_noi_cibo_di_eternita}
%data_revisione{2011_12_31}
%trascrittore{Francesco Endrici}
\beginsong{Sei per noi cibo di eternità}[by={Buttazzo}]
\ifchorded
\beginverse*
\vspace*{-0.8\versesep}
{\nolyrics \[C]\[C]\[G]\[Am]\[Dm]\[F]\[G4]\[G3]}
\vspace*{-\versesep}
\endverse
\fi
\beginchorus
\[C]Sei per \[G]noi cibo di e\[Am]ternità,
vera be\[Dm]vanda che
colma la \[G]sete in noi.
\[C]Sei per \[G]noi luce di \[Am]verità,
presenza \[Dm]viva del \[F]Dio \[G]con \[C]noi. \[Am]
\endchorus
\beginverse
\memorize
Tu Signore sei vi\[Em]cino \[F]
sei presente ancora in \[C]mezzo a \[G]noi. \[Am]
Tu, l'eterno onnipo\[Em]tente \[F]
ora \[Dm]vieni incontro a \[G4]noi.
\endverse
\beginverse
Infinita Cari^tà ^
l'universo intero ^vive in ^Te. ^
Tu ci guardi con a^more ^
e ci ^chiami insieme a ^Te. \[La]
\endverse
\beginchorus
\transpose{2}
\[C]Sei per \[G]noi cibo di e\[Am]ternità,
vera be\[Dm]vanda che
colma la \[G]sete in noi.
\[C]Sei per \[G]noi luce di \[Am]verità,
presenza \[Dm]viva del \[F]Dio \[G]con \[C]noi. \[Am]
\endchorus
\beginverse
\transpose{2}
Come cervo alla sor^gente ^
il nostro cuore anela ^sempre a ^Te. ^
A tua immagine ci hai ^fatti, ^
ora ^noi veniamo a ^Te.
\endverse
\beginchorus
\[C]Sei per \[G]noi cibo di e\[Am]ternità,
vera be\[Dm]vanda che
colma la \[G]sete in noi.
\[C]Sei per \[G]noi luce di \[Am]verità,
presenza \[Dm]viva del \[F]Dio \[G]con \[C]noi \[Am]
presenza \[Dm]viva del \[F]Dio \[G]con \[C]noi. \[Am]
\endchorus
\endsong



%titolo{Seme dell'eternità}
%autore{Gen Verde, Gen Rosso}
%album{Come fuoco vivo}
%tonalita{Mi}
%gruppo{}
%momenti{Comunione}
%identificatore{seme_dell_eternita}
%data_revisione{2011_12_31}
%trascrittore{Francesco Endrici}
\beginsong{Seme dell'eternità}[by={Gen\ Verde, Gen\ Rosso}]
\ifchorded
\beginverse*
\vspace*{-0.8\versesep}
{\nolyrics \[E]\[E4]\[E]\[E4]\[A]}
\vspace*{-\versesep}
\endverse
\fi
\beginverse
\memorize
\[E]Pane di {\[C#m7]vita} o\ch{A}{f}{f}{ff}erto per \[E4]noi,
\[E]forza del \[A]nostro cam\[A]mi\[B]no,
\[G]cibo del \[D]cielo che il \[A]Padre ci \[C#m7]dà,
\[A]per ogni \[E]uomo \[E4]sei “Dio vi\[E]cino”. \[E4]\[E]\[E4]
\endverse
\beginverse
^In questa ^fonte di ^felici^tà
^c'è il tuo di^segno di^vi^no,
^sei tu che ^vieni a trasfor^marci in ^te,
^questo è l'im^menso ^nostro de^stino.
\endverse
\beginverse
\ifchorded
\[B]\[E]\[A]\[G]\[D]\[A]\[C#m7]\[A]\[B7]\[B]
\fi
^Tu sei la ^Luce ve^nuta tra ^noi,
l'^Amore, E^terno Pre^sen^te.
^Tu ci fai ^una cosa ^sola in ^te,
^figli nel ^Figlio ^del Dio vi\[A]vente.
\endverse
\beginverse
\chordsoff
Tu che ti sei fatto cibo per noi
col Pane e la tua Parola,
facci risorgere un giorno con te
da questa vita a vita nuova.
\endverse
\beginverse
\chordsoff
Tu sei la Vita, sei la Verità,
tu sei la Strada sicura,
Pane che nutre in noi la carità,
certezza della vita futura.
\endverse
\ifchorded
\beginverse*
\vspace*{-\versesep}
{\nolyrics \[B]\[E]\[A]\[G]\[D]\[A]\[C#m7]\[A]\[B7]\[B]}
\endverse
\fi
\beginverse
\chordsoff
Noi siamo i tralci nutriti da te
con il tuo amore fecondo,
dacci il tuo Spirito e saremo con te
messe divina che nutre il mondo.
\endverse
\beginverse
^Hai messo il ^seme dell'^eterni^tà
^nel corpo ^che tu ci hai ^da^to
^e il nostro ^corpo un giorno ^riporte^rà
^nella tua ^gloria ^tutto il cre^ato.
\endverse
\endsong

%titolo{Semina la pace}
%autore{Gen Rosso}
%album{Inspiration}
%tonalita{Do}
%gruppo{}
%momenti{Pace}
%identificatore{semina_la_pace}
%data_revisione{2011_12_31}
%trascrittore{Francesco Endrici}
\beginsong{Semina la pace}[by={Gen\ Rosso}, ititle={Hopes of peace}]
%\transpose{3}
\beginverse
\[C]Senti il cuore \[F]della tua cit\[C]tà, 
\[C]batte nella \[F]notte intorno a \[C]Te,
\[C]sembra una can\[F7]zone \[Am]muta \[G]che 
\[C]cerca un'alba di \[F] sereni\[C]tà
\endverse
\beginverse
^Semina la ^pace e tu ve^drai 
^che la tua spe^ranza rivi^vrà;
^spine tra le ^mani ^piange^rai, 
^ma un mondo ^nuovo ^nasce\[C7]rà.
\endverse
\transpose{5}
\beginverse
^Semina la ^pace e tu ve^drai 
^che la tua spe^ranza rivi^vrà;
^spine tra le ^mani ^piange^rai, 
^ma un mondo ^nuovo ^nasce\[C7]rà.
\endverse
\beginchorus
\[F] Sì, \[G]nascerà
il \[C]mondo della pace, \[F]
di guerra \[G]non si parlerà \[Em7] mai \[Am]più,
\[F] la pace è un \[G]dono
che la \[C]vita \[E7]ci da\[Am]rà, 
un \[F]sogno che \[C] si avvere\[G4]\[G]{rà\dots}
\endchorus
\beginverse
\[C]Senti il cuore \[F]della tua cit\[C]tà, 
\[C]batte nella \[F]notte intorno a \[C]Te,
\[C]sembra una can\[F7]zone \[Am]muta \[G]che 
\[C]cerca un'\[F]alba di se\[C]reni\[C7]tà
\endverse
\beginchorus
\[F] Sì, \[G]nascerà
il \[C]mondo della pace, \[F]
di guerra \[G]non si parlerà \[Em7] mai \[Am]più,
\[F] la pace è un \[G]dono
che la \[C]vita \[E7]ci da\[Am]rà, 
un \[F]sogno che \[C] si avvere\[G4]\[G]{rà\dots}
\endchorus

\beginverse
^Semina la ^pace e tu ve^drai 
^che la tua spe^ranza rivi^vrà;
^spine tra le ^mani piange^rai, 
^ma un mondo ^nuovo ^nasce\[C7]rà.
\endverse
\endsong

%titolo{Servo per amore}
%autore{Gen Rosso}
%album{Se siamo uniti}
%tonalita{Si-}
%gruppo{}
%momenti{Offertorio}
%identificatore{servo_per_amore}
%data_revisione{2011_12_31}
%trascrittore{Francesco Endrici}
\beginsong{Servo per amore}[by={Gen\ Rosso}]
\beginverse 
Una \[Bm]notte di sudore sulla barca in mezzo al \[D]mare
e mentre il \[A]cielo s'imbianca \[F#m]già
tu guardi \[G]le tue reti \[Bm]vuote.
Ma la \[D]voce che ti chiama un altro \[A]mare ti mostre\[D]rà
e sulle \[G]rive di ogni \[D]cuore le tue \[Em]reti \[G]gette\[D]rai.
\endverse
\beginchorus
\[Bm]Offri la vita \[D]tua
come Ma\[A]ria ai \[F#m]piedi della \[G]croce
\[Bm]e sarai \[D]servo di ogni \[A]uomo,
\[D]servo per a\[G]more, 
\[D]sacerdote \[Em]dell'u\[G]mani\[D]tà.
\endchorus
\beginverse
Avan^zavi nel silenzio, fra le lacrime spe^ravi
che il seme ^sparso davanti a ^te 
cadesse ^sulla buona ^terra. 
Ora il ^cuore tuo è in festa \brk perché il ^grano biondeggia or^mai, 
è matu^rato sotto il ^sole, puoi ri^porlo ^nei gra^nai.
\endverse
\endsong


%titolo{Si accende una luce}
%autore{Rohr, Fant}
%album{}
%tonalita{Fa}
%gruppo{}
%momenti{Avvento}
%identificatore{si_accende_una_luce}
%data_revisione{2011_12_31}
%trascrittore{Francesco Endrici}
\beginsong{Si accende una luce}[by={Rohr, Fant}]
\beginverse
Si ac\[F]cende una luce all'\[C]uomo quag\[F]giù,
presto verrà tra \[C]noi Ge\[F]sù.
Annuncia il profeta \[C]la novi\[F]tà:
il re Messia ci \[C]salve\[F]rà.
\endverse
\beginchorus
\[C]Lieti can\[Dm]tate \[C]gloria al Si\[F]gnor,
\[B&]nasce\[C]rà il reden\[F]tor.
\endchorus
\beginverse
\chordsoff
Si accende una luce all'uomo quaggiù,
presto verrà tra noi Gesù.
Un'umile grotta solo offrirà
Betlemme, piccola città.
\endverse
\beginverse
\chordsoff
Si accende una luce all'uomo quaggiù,
presto verrà tra noi Gesù.
Pastori, adorate con umiltà
Cristo, che nasce in povertà.
\endverse
\beginverse
\chordsoff
Si accende una luce all'uomo quaggiù,
presto verrà tra noi Gesù.
Il coro celeste «Pace - dirà -
a voi di buona volontà».
\endverse
\beginverse
\chordsoff
Si accende una luce all'uomo quaggiù,
presto verrà tra noi Gesù.
Vegliate, lo sposo non tarderà;
se siete pronti, vi aprirà.
\endverse
\endsong

%titolo{Sì, ma insieme}
%autore{A.C.R.}
%album{}
%tonalita{La}
%gruppo{}
%momenti{}
%identificatore{si_ma_insieme}
%data_revisione{2011_12_31}
%trascrittore{Francesco Endrici - Manuel Toniato}
\beginsong{Sì, ma insieme}[by={A.C.R.}]

\beginchorus
\[A]Sì, ma in\[D]sieme, \[A]sì ma in\[D]sieme,
\[F#m]sì, ma in\[Bm]sieme, \[E] insieme a \[A]noi. \rep{2}
\endchorus

\beginverse
\[A]Tanti ra\[D]gazzi siamo \[A]qui
per can\[E]tare e stare in\[D]sieme nella \[A]gioia.
\[A]Siamo di\[D]versi e si \[A]sa
ma vi\[E]viamo tutti un'\[D]unica spe\[A]ranza.
\endverse

\beginverse
\chordsoff
Per fare festa siamo qui,
per pregare e diventare veri amici.
Tutta la gioia che c'è in noi
la vogliamo far vedere sempre più.
\endverse

\beginverse
\chordsoff
Da tanti posti siamo qui:
siamo certi che è più bello stare insieme.
E c'è bisogno anche di noi
per cambiare e fare un mondo più felice.
\endverse

\beginverse
\chordsoff
Grazie, Signore, perché tu
sei la forza che ci fa crescere insieme.
Grazie, Signore, perché tu
sei l'amico che ci fa esser felici.
\endverse
\endsong

%titolo{Siamo arrivati da mille strade diverse}
%autore{Giombini}
%album{Da mille strade}
%tonalita{Re}
%gruppo{}
%momenti{Ingresso}
%identificatore{siamo_arrivati}
%data_revisione{2011_12_31}
%trascrittore{Francesco Endrici}
\beginsong{Siamo arrivati da mille strade diverse}[by={Giombini}]
\beginchorus
\[D]Siamo arrivati da mille \[A]strade diverse,
in mille \[G]modi diversi
in mille mo\[D]menti diversi,
perché il Si\[A]gnore \[G] ha voluto co\[D]sì.
\endchorus
\beginverse
\[G] Ci ha chia\[A]mato per \[D]nome, \[G]
ci ha detto: “\[A]siete \[D]liberi \[G]
se cer\[A]cate la mia \[D]strada, \[B7]
la mia \[Em]strada \[A7] è l'a\[D]more!”
\endverse
\beginverse
\chordsoff
Ci donato questa casa,
ci ha detto: “Siate uniti!
Se amate la mia casa,
la mia casa è la pace!”
\endverse
\endsong


%titolo{Signora della pace}
%autore{Spoladore}
%album{Così}
%tonalita{La}
%gruppo{}
%momenti{Maria}
%identificatore{signora_della_pace}
%data_revisione{2011_12_31}
%trascrittore{Francesco Endrici}
\beginsong{Signora della pace}[by={Spoladore}]
\beginverse
\[A] Dolce Si\[D]gnora vestita di \[A]cielo, \[D]\[A]
Madre \[D]dolce della spe\[F#m]ranza, \[E]\[A]
gli uomini \[D]corrono senza fu\[A]turo, \[E]
ma \[G]nelle loro \[D]mani
c'è an\[C#m]cora quella \[B]forza
per \[G#7]stringere la \[C#m]Pace
e non \[B]farla andare \[E]via
dal \[A]cuore della \[B]gente.
\endverse
\beginchorus
Ma \[E]tu \[B]portaci a \[C#m]Dio
nel \[Am7]mondo cambie\[E]remo
le \[F#m]strade gli oriz\[A]zon\[B]ti
e \[E]noi apri\[B]remo nuove \[C#m]vie
che \[Am]partono dal \[E]cuore
e ar\[F#m]rivano alla \[A]pa\[B]ce.
E \[C]noi non ci \[D]fermeremo \[G]mai
per\[B7]ché insieme a \[Em]Te
l'a\[B7]more vince\[E]rà.
\endchorus
\beginverse
%\chordsoff 
^ Dolce Si^gnora vestita di ^cielo ^^
Madre ^dolce dell'inno^cenza, ^^
libera il ^mondo dalla pa^ura, ^
dal ^buio senza ^fine,
dalla ^guerra e dalla ^fame,
dall'^odio che di^strugge
gli oriz^zonti della ^vita
dal ^cuore della ^gente.
\endverse
\endsong

%titolo{Signore aiutaci}
%autore{}
%album{}
%tonalita{Re}
%gruppo{}
%momenti{Ritornelli}
%identificatore{signore_aiutaci}
%data_revisione{2012_01_11}
%trascrittore{Francesco Endrici}
\beginsong{Signore aiutaci}
\beginverse*
Si\[D]gnore a\[A]iutaci a\[G]mar\[A]ti di \[D]più,
Si\[D]gnore a\[A]iutaci ad a\[G]mar\[A]ci di \[D]più.
\endverse
\endsong
%titolo{Solo chi ama}
%autore{Machetta}
%album{}
%tonalita{Fa}
%gruppo{}
%momenti{Comunione;Pasqua}
%identificatore{solo_chi_ama}
%data_revisione{2011_12_31}
%trascrittore{Francesco Endrici - Manuel Toniato}
\beginsong{Solo chi ama}[by={Machetta}]

\beginverse
\[F]Noi annun\[Dm]ciamo la pa\[Gm]rola e\[C]terna: \[F]Dio {\[Gm]è} a\[C]more.
\[F]Questa è la \[Dm]voce che ha var\[Gm]cato i \[C]tempi: \brk \[F]Dio è \[Gm]cari\[C]tà.  \[C7] 
\endverse

\beginchorus
\[F]Passa questo \[Dm]mondo, \[Gm7]passano i se\[C]coli,
\[F7+]solo chi \[B&]ama non \[C]passerà \[F]mai. \rep{2}
\endchorus

\beginverse
\chordsoff
Dio è luce e in Lui non c'è la notte: Dio è Amore.
Noi camminiamo lungo il suo sentiero: \brk Dio è carità.
\endverse

\beginverse
\chordsoff
Noi ci amiamo perché Lui ci ama: Dio è amore.
Egli per primo diede a noi la vita: \brk Dio è carità.
\endverse

\beginverse
\chordsoff
Giovani forti, avete vinto il male: Dio è amore.
In voi dimora la parola eterna: \brk Dio è carità.
\endverse
\endsong

%titolo{Solo in Dio}
%autore{Zerlotti}
%album{Sempre canterò}
%tonalita{Re-}
%gruppo{}
%momenti{Quaresima;Salmi}
%identificatore{solo_in_dio}
%data_revisione{2011_12_31}
%trascrittore{Francesco Endrici - Manuel Toniato}
\beginsong{Solo in Dio}[by={Zerlotti}]

\beginchorus
\[Dm]Solo in \[B&]Dio ri\[Am7]posa l'anima \[Dm]mia, 
da \[B&]lui la \[Am7]mia spe\[G]ran\[Dm]za.
\endchorus

\beginverse
\[F]Lui solo è mia \[C]rupe e \[B&]mia sal\[Dm]vezza.
\[F]Mia roccia di di\[C]fesa, non po\[B&]trò vacil\[Dm]lare.
\endverse

\beginverse
\chordsoff
In Dio è la mia salvezza, la mia gloria,
il mio saldo rifugio, la mia difesa.
\endverse

\beginverse
\chordsoff
Confida sempre in Lui o popolo
davanti a Lui effondi il tuo cuore.
\endverse

\beginverse
\chordsoff
Poiché il potere appartiene a Dio
Tua Signore è la grazia.
\endverse
\endsong

%titolo{Spirito di Dio}
%autore{Rangoni, Soldano}
%album{Celebrate}
%tonalita{Re}
%gruppo{}
%momenti{Pentecoste}
%identificatore{spirito_di_dio_rangoni}
%data_revisione{2011_12_31}
%trascrittore{Francesco Endrici - Manuel Toniato}
\beginsong{Spirito di Dio}[by={Rangoni, Soldano}]
\
\ifchorded
\beginverse*
\vspace*{-0.8\versesep}
{\nolyrics %
\[D] \[G] \[D] \[F#] \[Bm] \[D] \[G] \[D] \[A7] 
}
\vspace*{-\versesep}
\endverse
\fi
\beginverse
\[D]Spirito di \[G]Dio riempi\[D]mi, \[G]  \brk \[D]Spirito di \[G]Dio battezza\[A]mi, \[A7] 
\[D]Spirito di \[G]Dio con\[D]sa\[F#]cra\[Bm]mi, \brk \[D]vieni ad abi\[G]tare dentro \[D]me!  \[G] 
\endverse

\beginverse
\chordsoff
Spirito di Dio guariscimi, Spirito di Dio rinnovami
Spirito di Dio consacrami, \brk vieni ad abitare dentro me!
\endverse

\beginverse
\chordsoff
Spirito di Dio guariscici, Spirito di Dio rinnovaci,
Spirito di Dio consacraci, \brk vieni ad abitare dentro noi!
\endverse

\beginverse
\chordsoff
Spirito di Dio riempici, Spirito di Dio battezzaci
Spirito di Dio consacraci, \brk vieni ad abitare dentro noi!
\endverse

\endsong

%titolo{Spirito di Dio}
%autore{Iverson}
%album{Sempre canterò}
%tonalita{Re}
%gruppo{}
%momenti{Pentecoste}
%identificatore{spirito_di_dio_vivenzio}
%data_revisione{2011_12_31}
%trascrittore{Francesco Endrici - Manuel Toniato}
\beginsong{Spirito di Dio}[by={Iverson}]
\
\beginchorus
\[D]Spiri\[B7]to di \[Em]Di\[A7]o \[D]scendi su \[A]di \[D]noi. 
\[D]Spiri\[B7]to di \[Em]Di\[A7]o \[D]scendi su \[A]di \[D]noi. 
\endchorus

\beginverse
\[G]Fondici, \[D]plasmaci, \[E]riempi\[E7]ci, \[A]usa\[A7]ci.
\[D]Spiri\[B7]to di \[Em]Di\[A7]o \[D]scendi su \[A]di \[D]noi. 
\endverse

\chordsoff
\beginverse
Rendici docili, umili, semplici.
Spirito di Dio scendi su di noi.
\endverse

\beginverse
Tu, che ti librasti sulla creazione.
Spirito di Dio scendi su di noi.
\endverse

\beginverse
Tu, che fecondasti le acque del Giordano.
Spirito di Dio scendi su di noi.
\endverse

\beginverse
Tu, che discendesti sulla prima Chiesa.
Spirito di Dio scendi su di noi.
\endverse

\beginverse
Guidaci, Spirito, salvaci, formaci!
Spirito di Dio scendi su di noi.
\endverse

\beginverse
Suscita vergini, donaci apostoli!
Spirito di Dio scendi su di noi.
\endverse

\beginverse
Libera i poveri, dà pace ai popoli.
Spirito di Dio scendi su di noi.
\endverse
\endsong

%titolo{Spirito di vita}
%autore{Lucadello}
%album{}
%tonalita{Mi}
%gruppo{}
%momenti{Ingresso;Congedo;Pentecoste}
%identificatore{spirito_di_vita}
%data_revisione{2011_12_31}
%trascrittore{Francesco Endrici - Manuel Toniato}
\beginsong{Spirito di vita}[by={Lucadello}]

\ifchorded
\beginverse*
\vspace*{-0.8\versesep}
{\nolyrics \[A/C#] \[B/D#] \[E] \[A/F#] \[B/G#] \[A] \[B4] \[B] }
\vspace*{-\versesep}
\endverse
\fi
\beginchorus
\[E]Vieni \[A]vie\[E]ni \[E]Spirito \[A]San\[B]to.
\[G#m7]Posati \[A]so\[(E)]pra \[A]l'anima \[B4]mi\[B]a
\[E]fai germo\[A]glia\[E]re in \[A]me \[(E)]i tuoi \[A]do\[B]ni
\[G#m7]sii in me re\[A]spiro e mia \[E]o\[B]pe\[E]ra.
\endchorus

\beginverse
\[A]Tu che nella crea\[B]zione del \[C#m7]mon\[B]do
\[F#m7]ti posasti \[G#m7]sulle  \[B/A]ac\[A]que
\[A/C#]per ani\[B/D#]marle di \[E]vi\[B]ta 
\[F#m7]posati sulle \[G#m7]acque dell'\[A]a\[E/B]nima \[A/C#]mi\[B/D#]a.
\endverse

\beginverse
\chordsoff
Suscita in me quel fervore 
stabilisci la tua dimora
nel mio cuore che ti consacro 
Spirito Santo, dammi i tuoi sette doni
\endverse

\beginverse
\chordsoff
Sii tu il mio respiro 
sii tu la mia vita
motivo di ogni mia opera 
tutto per me oggi e sempre
\endverse

\beginverse*\itshape
\[A/C#]Vie\[B/D#]ni \[E]Vie\[A/F#]ni \[B/G#]\[A]Vie\[B4]ni. \[B] 
\endverse
\endsong

%titolo{Spirito Santo, soffia su di noi}
%autore{}
%album{}
%tonalita{La}
%gruppo{}
%momenti{Pentecoste}
%identificatore{spirito_santo_soffia_su_di_noi}
%data_revisione{2011_12_31}
%trascrittore{Francesco Endrici}
\beginsong{Spirito Santo, soffia su di noi}
\beginverse
\[A]Spirito \[E]Santo, \[D]soffia su \[A]noi
\[D]Fiume di \[A]pace, un \[D]mare d’a\[E]more
\[A]Come una \[E]fonte,	 \[D]dai la Tua \[A]gioia
\[F#m]Spiri\[C#m]to \[F#m]San\[D]to, riempi\[A]ci \[E]di \[A]Te.
\endverse
\endsong

%titolo{Spirito santo vieni}
%autore{Bordini}
%album{}
%tonalita{Re}
%gruppo{}
%momenti{Pentecoste}
%identificatore{spirito_santo_vieni}
%data_revisione{2011_12_31}
%trascrittore{Francesco Endrici - Manuel Toniato}
\beginsong{Spirito santo vieni}[by={Bordini}]

\beginchorus
\[D]Spirito \[Bm]San\[A]to, \[D]Spirito San\[A]to,
%\textit{(Spirito Santo, vieni, vieni!)}
\[Bm]Spirito Santo \[F#m]vieni!
%\textit{(Spirito Santo vieni)}
\[Bm]Vieni dai quattro \[A]venti,
%\textit{(vieni, vieni!)}
\[D]Spirito \[A]del Si\[Bm]gnore,
%\textit{(Spirito del Signore, vieni!)}
\[G]Spirito \[F#m]dell'A\[Bm]more,
%\textit{(Spirito Santo, vieni!)}
\[G]Spirito \[A]San\[D]to \[Em7Re]vieni!
\endchorus

\beginverse
\[D]Vieni Santo \[F#m]Spiri\[Bm]to, \[G]riempi il cuore dei \[Em]fede\[A]li,
\[D]accendi il fuoco del tuo a\[F#m]mo\[Bm]re.
\endverse

\beginverse
\chordsoff
Lava le nostre colpe trasformaci in primizia
di creazione nuova.
\endverse

\beginverse
\chordsoff
Vieni Santo Spirito fa'splendere la tua luce
rinnova il volto della terra.
\endverse

\beginverse
\chordsoff
Dal regno delle tenebre guidaci alla sorgente
del primo eterno Amore.
\endverse
\endsong

%titolo{Su ali d'aquila}
%autore{Joncas}
%album{Su ali d'aquila}
%tonalita{Do}
%gruppo{}
%momenti{Salmi;Comunione}
%identificatore{su_ali_d_aquila}
%data_revisione{2011_12_31}
%trascrittore{Francesco Endrici}
\beginsong{Su ali d'aquila}[by={Joncas}]
\beginverse
\[F7+]Tu che abiti al ri\[C7+]paro del Signore
e \[F7+]che dimori alla sua \[C7+]ombra, 
\[E&]dì al Signore:«Mio ri\[G#]fugio, mia \[Fm]roccia in cui con\[G]fido».
\endverse
\beginchorus
E ti ri\[C]alzerà, ti solleverà su ali d'\[Dm]aquila
\[G]ti reggerà sulla \[Gm]brezza dell'\[C]alba ti fa\[F]rà bril\[Dm]lar
come il \[Am]sole, co\[Dm]sì nelle sue \[G]mani vi\[C]vrai.
\endchorus
\beginverse
\chordsoff
Dal laccio del cacciatore ti libererà
e dalla carestia che distrugge
poi ti coprirà con le sue alie rifugio troverai.
\endverse
\beginverse
\chordsoff
Non devi temere i terrori della notte
né freccia che vola di giorno
mille cadranno al tuo fianco,ma nulla ti colpirà.
\endverse
\beginverse
\chordsoff
Perché ai suoi angeli ha dato un comando
di preservarti in tutte le tue vie,
ti porteranno sulle loro mani
contro la pietra non inciamperai.
\endverse
\beginchorus
\chordsoff
E ti rialzerà\dots
E ti rialzerò, ti solleverò su ali d'aquila
ti reggerò sulla brezza dell'alba ti farò brillar
come il sole, così nelle mie mani vivrai.
\endchorus
\endsong

%titolo{Sulla tua parola}
%autore{Paci, Varnavà}
%album{Trionferemo, trionferemo}
%tonalita{La-}
%gruppo{}
%momenti{Comunione}
%identificatore{sulla_tua_parola}
%data_revisione{2011_12_31}
%trascrittore{Francesco Endrici}
\beginsong{Sulla tua parola}[by={Paci, Varnavà},ititle={Pietro vai}]
\beginverse
Si\[Am]gnore, ho pe\[Dm]scato tutto il \[Am]giorno,
le \[F]reti son ri\[G]maste sempre \[C]vuote, \[E]
s'è \[Dm]fatto \[E]tardi, a \[Am]casa \[E]ora ri\[Am]torno,
Si\[F]gnore, son de\[G]luso e me ne \[Am]vado.
\endverse
\beginverse

La ^vita con me è ^sempre stata ^dura
e ^niente mai mi ^dà soddisfa^zione, ^
la ^strada in cui mi ^guidi è ^in^si^cura,
sono ^stanco e ora ^non aspetto \[E]più.
\endverse
\beginchorus
\[A]Pietro, \[E]vai, \[F#m]fidati di \[C#m]me,
\[D]getta ancora in \[A]acqua le tue \[E]reti,
\[F#m]prendi ancora il \[C#m]largo, sulla \[D]mia pa\[A]rola,
\[D]con la mia po\[A]tenza, \[Bm]io ti fa\[F#m]rò \[D]
pesca\[E]tore di \[D]uomi\[A]ni.
\endchorus
\beginverse
\chordsoff
Maestro, dimmi cosa devo fare,
insegnami, Signore, dove andare,
Gesù, dammi la forza di partire,
la forza di lasciare le mie cose:
questa famiglia che mi son creato,
le barche che a fatica ho conquistato,
la casa, la mia terra, la mia gente,
Signore, dammi tu una fede forte.
\endverse
\beginchorus
\chordsoff
Pietro, vai, fidati di me,
la mia Chiesa su te fonderò;
manderò lo Spirito, ti darà coraggio,
donerà la forza dell'amor
per il Regno di Dio.
\endchorus
\endsong

%titolo{Svegliati Sion}
%autore{Comunità\ Shalom}
%album{Prostrati adoriamo}
%tonalita{Sol}
%gruppo{}
%momenti{Salmi}
%identificatore{sveglaiti_sion}
%data_revisione{2011_12_31}
%trascrittore{Francesco Endrici}
\beginsong{Svegliati Sion}[by={Comunità\ Shalom}]
\beginchorus
\[G]Svegliati, svegliati o \[C]Si\[D]on,
\[G]metti le \[Em7]vesti più \[A]bel\[D]le.
\[D7]Scuoti la polvere e \[G]alza\[C]ti, \[G]Santa Ge\[D7]rusa\[G]lem\[B7]me.
\endchorus
\beginverse
\[Em]Ecco ti tolgo di \[C]mano il \[D]calice \[Bm7]della ver\[Em]tigine.
La \[G]coppa \[C]della mia \[D]ira, \[C]tu non \[D7]berrai \[Em]più. \[D7]
\endverse
\beginverse
\chordsoff
Sciogli dal collo i legami \brk e leva al cielo i tuoi occhi. 
Schiava figlia di Sion, io ti libererò.
\endverse
\beginverse
\chordsoff
Come sono belli sui monti i piedi del messaggero.
Colui che annunzia la pace è messaggero di pace.
\endverse
\endsong



%titolo{Symbolum ‘77}
%autore{Sequeri}
%album{In cerca d'autore}
%tonalita{Mi-}
%gruppo{}
%momenti{Comunione}
%identificatore{symbolum_77}
%data_revisione{2011_12_31}
%trascrittore{Francesco Endrici}
\beginsong{Symbolum ‘77}[by={Sequeri}]
\beginverse
\[Em]Tu sei la mia \[C]vita, \[D]altro io non \[G]ho.
\[Em]Tu sei la mia \[C]strada, \[D]la mia veri\[B7]tà.
\[Am]Nella tua pa\[D7]rola \[G]io cammine\[Em]rò, \[7]
\[C]finché avrò re\[D]spiro, fino a \[G]quando tu vor\[B7]rai.
\[Am]Non avrò pa\[D7]ura, sai, \[G]se tu sei con \[Em]me:
\[C]io ti prego, \[D]resta con \[Em]me.
\endverse
\beginverse
%\chordsoff
^Credo in Te, Si^gnore, ^nato da Ma^ria,
^figlio eterno e ^santo, ^uomo come ^noi.
^Morto per a^more, ^vivo in mezzo a ^noi:	 ^
^una cosa ^sola con il ^Padre e con i ^tuoi,
^fino a quando, ^io lo so, ^tu ritorne^rai
^per aprirci il ^regno di ^Dio.
\endverse
\beginverse
%\chordsoff
^Tu sei la mia ^forza, ^altro io non ^ho,
^tu sei la mia ^pace, ^la mia liber^tà.
^Niente nella ^vita ^ci separe^rà. ^
^So che la tua ^mano forte ^non mi lasce^rà.
^So che da ogni ^male tu ^mi libere^rai
^e nel tuo per^dono vi^vrò.
\endverse
\beginverse
%\chordsoff
^Padre della ^vita, ^noi crediamo in ^Te.
^Figlio Salva^tore, ^noi speriamo in ^Te.
^Spirito d'a^more ^vieni in mezzo a ^noi. ^
^Tu da mille ^strade ci ra^duni in uni^tà.
^E per mille ^strade poi, ^dove tu vor^rai,
^noi saremo il ^seme di ^Dio.
\endverse
\endsong

%titolo{Symbolum ‘79}
%autore{Sequeri}
%album{E mi sorprende}
%tonalita{Do}
%gruppo{}
%momenti{Comunione}
%identificatore{symbolum_79}
%data_revisione{2011_12_31}
%trascrittore{Francesco Endrici}
\beginsong{Symbolum ‘79}[by={Sequeri}]
\beginverse
\[C]Tu sei prima d'\[G]ogni cosa, \[C]prima d'ogni \[G]tempo
d'ogni \[C]mio pensiero, \[G]prima della \[C]vi\[G]ta.
\[C]Una voce u\[G]dimmo che gri\[C]dava nel de\[G]serto:
prepa\[C]rate la ve\[G]nuta del Si\[C]gno\[G]re.
\endverse
\beginverse
\chordsoff
Tu sei la parola eterna, della quale vivo
che mi pronunciò soltanto per amore.
E ti abbiamo udito predicare per le strade
della nostra incomprensione senza fine.
\endverse
\beginchorus
\[C]\[G]\[C]\[G]
\[C]Io \[E7]ora so chi \[Am]sei, \[7] \[F]io \[G]sento la tua \[Em7]voce
\[F]io \[Am]vedo la tua \[Em]luce, \[C] \[F]io so che tu sei \[G7]qui.
\[C]E \[E7]sulla tua pa\[Am]rola, \[7] \[F]io \[G]credo nell'a\[Em7]more
\[F]io \[Am]vivo nella \[Em]pace, \[C] \[F]io \[G]so che torne\[C]rai.
\endchorus
\beginverse
\chordsoff
Tu sei l'apparire dell'immensa tenerezza
di un Amore che nessuno ha visto mai.
Ci fu dato il lieto annuncio della tua venuta
noi abbiamo visto un uomo come noi.
\endverse
\beginverse
\chordsoff
Tu sei verità che non tramonta, sei la vita
che non muore, sei la via d'un mondo nuovo.
E ti abbiamo visto stabilire la tua tenda
tra la nostra indifferenza d'ogni giorno.
\endverse
\endsong

%titolo{Symbolum ‘80}
%autore{Sequeri}
%album{Symbolum ‘80}
%tonalita{Re-}
%gruppo{}
%momenti{Comunione}
%identificatore{symbolum_80}
%data_revisione{2011_12_31}
%trascrittore{Francesco Endrici}
\beginsong{Symbolum ‘80}[by={Sequeri}]
\ifchorded
\beginverse*
\vspace*{-0.8\versesep}
{\nolyrics \[Dm]\[F]\[C]\[Dm]\[Dm]}
\vspace*{-\versesep}
\endverse
\fi
\beginverse
\memorize
\[Dm]Oltre le me\[F]morie del \[C]tempo che ho vis\[Dm]suto,
oltre la spe\[C]ranza che serve al mio do\[A7]mani,
\[Dm]oltre il desi\[F]derio di \[C]vivere il pre\[Dm]sente,
anch'io, confesso, ho \[C]chiesto che cosa è veri\[A]tà?
\[D]E \[A]Tu \[Bm]come un desi\[F#m]derio,
\[G]che non \[A]ha me\[D]morie, \[E7]Padre \[A]buono,
\[D]come una spe\[A]ranza \[Bm]che non ha con\[F#m]fini,
\[G]come un \[A]tempo e\[D]terno \[E]sei per \[A]me.
\endverse
\beginchorus
\[F]Io \[C]so \[Dm]quanto amore \[Am]chiede
\[B&]questa \[C]lunga at\[F]tesa \[Gm]del tuo giorno, o \[C]Dio;
\[F]luce in ogni \[C]cosa \[Dm]io non vedo an\[Am]cora,
\[B&]ma la \[C]tua pa\[F]rola \[Gm]mi rischiare\[A7]rà.
\endchorus
\beginverse
\chordsoff
Quando le parole non bastano all'amore,
quando il mio fratello domanda più del pane,
quando l'illusione promette un mondo nuovo,
anch'io rimango incerto nel mezzo del cammino.
E Tu Figlio tanto amato
verità dell'uomo, mio Signore,
come la promessa di un perdono eterno,
libertà infinità sei per me.
\endverse
\beginverse
\chordsoff
Chiedo alla mia mente coraggio di cercare
chiedo alle mie mani la forza di donare,
chiedo al cuore incerto passione per la vita
e chiedo a te fratello di credere con me!
E Tu forza della vita,
Spirito dell'amore, dolce Iddio,
grembo d'ogni cosa, tenerezza immensa,
verità del mondo sei per me.

\endverse
\endsong

\lettera
%titolo{Te al centro del mio cuore}
%autore{Gen Verde}
%album{È bello lodarti}
%tonalita{Mi-}
%gruppo{}
%momenti{Comunione}
%identificatore{te_al_centro_del_mio_cuore}
%data_revisione{2011_12_31}
%trascrittore{Francesco Endrici}
\beginsong{Te al centro del mio cuore}[by={Gen\ Verde}]
\beginverse
\[Em]Ho bisogno di incontrarti nel mio \[G]cuore,
\[Bm]di trovare Te, di stare insieme a \[C7+]Te:
\[Am]unico riferimento del mio an\[Em]dare,
\[C]unica ragione \[D]Tu, \[Bm]unico sostegno \[Em]Tu.
Al \[C]centro del mio cuore \[D]ci sei solo \[G]Tu.
\endverse
\beginverse
%\chordsoff
^Anche il cielo gira intorno e non ha ^pace,
^ma c'è un punto fermo è quella stella ^là.
^La stella polare è fissa ed è la ^sola,
^la stella polare ^Tu, ^la stella sicura ^Tu.
Al ^centro del mio cuore ^ci sei solo ^Tu.
\endverse
\beginchorus
\[G]Tutto \[Bm]ruota intorno a \[C]Te, in funzione di \[Bm]Te, \[Em]
e poi \[Bm]non importa il \[C]come, il dove e il \[D4]se. \[D]
\endchorus
\beginverse
%\chordsoff
^Che Tu splenda sempre al centro del mio ^cuore,
^il significato allora sarai ^Tu,
^quello che farò sarà soltanto a^more.
^Unico sostegno ^Tu, ^la stella polare ^Tu.
Al ^centro del mio cuore ^ci sei solo ^Tu.
\endverse
\endsong



%titolo{Ti esalto Dio mio re}
%autore{Cavalieri, Uva}
%album{Sempre canterò}
%tonalita{Sol}
%gruppo{}
%momenti{Ingresso}
%identificatore{ti_esalto_dio_mio_re}
%data_revisione{2011_12_31}
%trascrittore{Francesco Endrici - Manuel Toniato}
\beginsong{Ti esalto Dio mio re}[by={Cavalieri, Uva}]

\beginchorus
\[D7]Ti e\[G]salto \[D7]Dio mio \[G]re, \[Em]  
cante\[Am]rò in e\[G]terno a \[D]{te} \[D7] 
io \[G]voglio lo\[Em]darti Si\[Am]gnor \[D7]  
e bene\[G]dirti, \[D7]allelu\[G]ia! \[C] \[G] 
\endchorus

\beginverse
Il Si\[C]gnore è degno di ogni \[G]lode,
non si \[C]può misurar la sua gran\[G]dezza;
ogni vi\[Em]vente pro\[D7]clami la sua \[G]gloria\[C],
la sua \[G]opera è giu\[D]stizia e veri\[G]tà.
\endverse

\beginverse
\chordsoff
Il Signore è paziente e pietoso.
Lento all'ira e ricco di Grazia.
Tenerezza per ogni creatura,
il Signore è buono verso tutti.
\endverse

\beginverse
\chordsoff
Il Signore sostiene chi vacilla
e rialza chiunque è caduto.
Gli occhi di tutti ricercano il suo volto,
la sua mano provvede loro il cibo.
\endverse
\endsong

%titolo{Ti offriamo}
%autore{Gen Verde}
%album{Gratitudine}
%tonalita{La-}
%gruppo{}
%momenti{Offertorio}
%identificatore{ti_offriamo}
%data_revisione{2011_12_31}
%trascrittore{Francesco Endrici - Manuel Toniato}
\beginsong{Ti offriamo}[by={Gen\ Verde}]

\beginverse*
\[Am]Su questo al\[G]tare ti o\ch{F7+}{f}{f}{ff}riamo il \[E]nostro \[Am]giorno
tutto quello che ab\[G]biamo lo do\[F7+]niamo a \[E]Te.
L'a\[Am]mare, il gio\[G]ire, il do\[F7]lore di \[E]questo \[Am]giorno
su questo al\[G]tare doniamo a \[Am]Te.
\endverse
\beginverse*
\chordsoff
Fa' di tutti noi un corpo, un'anima sola;
che porta a te tutta l'umanità,
e fa che il tuo amore ci trasformi in te,
come il pane e il vino
che ora ti offriamo. \rep{2}
\endverse
\endsong

%titolo{Ti offriamo, Signore}
%autore{Spoladore}
%album{Come in cielo così in terra}
%tonalita{Sol}
%gruppo{}
%momenti{Offertorio}
%identificatore{ti_offriamo_signore}
%data_revisione{2011_12_31}
%trascrittore{Francesco Endrici - Manuel Toniato}
\beginsong{Ti offriamo, Signore}[by={Spoladore}]

\beginverse
Ti offriamo \[Em]Signore il nostro \[D]vivere
con tutto il \[Am7]peso \[B] e la gioia dei \[Em]giorni. \[D]
Ti offriamo \[Em]Signore le nostre \[D]mani
molte volte son \[Am7]vuote \[B] ma ricercano \[Em]te. \[D] 
\endverse

\beginchorus
Ti offriamo \[G]Signore \[C] la vita del \[G]mondo
\[Cm6]e nelle tue \[Em7]mani \[D] ricominciamo con \[G]te. \[D] 
\endchorus

\beginverse
\chordsoff
Ti offriamo Signore le nostre croci
il coraggio di amare dei santi di oggi
ti offriamo Signore le nostre forze
perché il tuo volere sia in cielo e in terra.
\endverse

\beginverse
\chordsoff
Ti offriamo Signore queste speranze
le nostre famiglie e i bambini del mondo
ti offriamo Signore la nostra Chiesa
la tua famiglia su tutta la terra.
\endverse

\beginverse
\chordsoff
Ti offriamo Signore i nostri errori
peccato e tristezza perdonaci ancora
Ti offriamo Signore chi soffre ancora
catene e ingiustizia e cerca la pace.
\endverse
\endsong

%titolo{Ti ringraziamo}
%autore{Gen Verde}
%album{Gratitudine}
%tonalita{Re}
%gruppo{}
%momenti{Offertorio;Congedo}
%identificatore{ti_ringraziamo}
%data_revisione{2011_12_31}
%trascrittore{Francesco Endrici - Manuel Toniato}
\beginsong{Ti ringraziamo}[by={Gen\ Verde}]

\beginchorus
\[D] Veniamo da \[Em]Te o Si\[A]gnore \brk con il \[F#m]cuore pieno di \[Bm]gioia
ed in\[D]sieme vo\[Em]gliamo ringra\[A7]ziarti.
\[D] Veniamo da \[Em]Te o Si\[A]gnore \brk con il \[F#m]cuore pieno di \[Bm]gioia
ed in\[D]sieme vo\[Em]gliamo  \[A7]ringra\[D]ziarti.
\endchorus

\beginverse
\[D] Per i giorni che ci \[Em]doni: \[A7] Ti ringra\[D]ziamo.
\[D] Per i frutti della \[Em]terra: \[A7] Ti ringra\[D]ziamo.
\[F#] Per il la\[Bm]voro, \[F#] le gioie della \[Bm]vita: \[G] \brk ti ringra\[A7]ziamo.
\endverse

\beginverse
\chordsoff
Per le tue parole: ti ringraziamo.
Perché hai dato la tua vita: ti ringraziamo.
E per la Chiesa che tutti ci unisce ti ringraziamo.
\endverse
\endsong



%titolo{Ti ringrazio mio signore}
%autore{Sequeri}
%album{}
%tonalita{Re}
%gruppo{}
%momenti{Congedo}
%identificatore{ti_ringrazio_mio_signore}
%data_revisione{2011_12_31}
%trascrittore{Francesco Endrici}
\beginsong{Ti ringrazio mio signore}[by={Sequeri}]
\beginchorus
\[D] Ti ringrazio mio Si\[G]gnore \brk e \[A7]non ho più pa\[D]ura \[7]
per\[G]ché \[A7]con la mia \[D]mano \brk nella \[G]mano \[A7]degli amici \[D]miei, \[7]
cam\[G]mino tra la \[A7]gente della \[D]mia \[F#]cit\[Bm]tà
e \[G]non mi \[A]sento più \[D]solo, \[7] non \[G]sento la stan\[A7]chezza
guardo \[D]dritto a\[F#]vanti a \[Bm]me,
per\[G]ché sulla mia \[A7]strada ci sei \[D]tu. \[G]\[D]
\endchorus
\beginverse
A\[D]matevi l'un l'\[Bm]altro come \[G]lui ha amato \[A]noi,
e \[D]siate per \[Bm]sempre suoi a\[G]mi\[A]ci.
E \[Em]quello che fa\[A]rete al più \[F#]piccolo tra \[Bm]voi,
cre\[G]dete, l'a\[A]vete \[7]fatto a \[D]lui. \[G]\[D]
\endverse
\beginverse
\chordsoff
Se amate veramente perdonatevi tra voi
nel cuore di ognuno ci sia pace;
il Padre che è nei cieli vede tutti i figli suoi
con gioia a voi perdonerà.
\endverse
\beginverse
\chordsoff
Sarete suoi amici se vi amate tra di voi
e questo è tutto il suo Vangelo:
l'amore non ha prezzo non misura ciò che dà,
l'amore confini non ne ha.
\endverse
\endsong


%titolo{Ti ringrazio o mio Signore}
%autore{Gabarain}
%album{Ammira e ringrazia}
%tonalita{Re}
%gruppo{}
%momenti{Congedo}
%identificatore{ti_ringrazio_o_mio_signore_gabarain}
%data_revisione{2011_12_31}
%trascrittore{Francesco Endrici}
\beginsong{Ti ringrazio o mio Signore}[by={Gabarain}]
\beginverse
Ti rin\[D]grazio, o \[A]mio Si\[Bm]gnore,
per le \[Em]cose che sono nel \[A]mondo,
per la \[D7]vita che tu mi hai do\[G]nato, \[Gm]
per l'a\[D]more \[A7] che tu nutri per \[D]me. \[7]
\endverse
\beginchorus
Alle\[G]luia, o mio Si\[D]gnore,
alle\[A7]luia, o Dio del \[D]\[A7]cie\[D]lo. \[D7]
Alle\[G]luia, o mio Si\[D]gnore,
alle\[A7]luia, o Dio del \[D]ciel.
\endchorus
\beginverse
\chordsoff
Quando il cielo si tinge d'azzurro
io ti penso e tu sei con me.
Non lasciarmi vagare nel buio,
nelle tenebre che la vita ci dà.
\endverse
\endsong


%titolo{Ti saluto o croce santa}
%autore{Gazzera, Damilano}
%album{}
%tonalita{Re-}
%gruppo{}
%momenti{Quaresima}
%identificatore{ti_saluto_o_croce_santa}
%data_revisione{2011_12_31}
%trascrittore{Francesco Endrici}
\beginsong{Ti saluto o croce santa}[by={Gazzera, Damilano}]
\beginchorus
Ti sa\[Dm]luto, o Croce \[F]santa
che por\[Dm]tasti il Reden\[A7]tor;
gloria, \[F]lode, o\[Gm]nor ti \[F]canta
ogni \[Gm]lingua ed \[A7]ogni \[Dm]cuor.
\endchorus
\beginverse
Sei ves\[F]sillo glorioso di \[C]Cristo,
sei sal\[F]vezza del \[Gm]popol fe\[F]del.
Grondi \[Gm]sangue inno\[A7]cente sul \[Dm]tristo
che ti \[A7]volle martirio cru\[Dm]del.
\endverse
\beginverse
\chordsoff
Tu nascesti fra braccia amorose
d'una Vergine Madre, o Gesù.
Tu moristi fra braccia pietose
d'una croce che data ti fu.
\endverse
\beginverse
\chordsoff
O Agnello divino immolato
sull'altar della croce, pietà!
Tu, che togli dal mondo il peccato,
salva l'uomo che pace non ha.
\endverse
\beginverse
\chordsoff
Del giudizio nel giorno tremendo,
sulle nubi del cielo verrai.
Piangeranno le genti vedendo
qual trofeo di gloria sarai.
\endverse
\endsong

%titolo{Ti seguirò}
%autore{Frisina}
%album{Benedici il Signore}
%tonalita{La}
%gruppo{}
%momenti{Comunione}
%identificatore{ti_seguiro_frisina}
%data_revisione{2011_12_31}
%trascrittore{Francesco Endrici - Manuel Toniato}
\beginsong{Ti seguirò}[by={Frisina}]

\beginverse
\[A]Ti segui\[E]rò, ti \[F#m]seguirò, o Si\[D]gnore,
\[A]e \[E]nella \[C#]tua \[F#m]strada \[D]cammine\[A]rò.
\endverse

\beginverse
\chordsoff
Ti seguirò nella via dell'amore
e donerò al mondo la vita.
\endverse

\beginverse
\chordsoff
Ti seguirò nella via del dolore
e la tua croce ci salverà.
\endverse

\beginverse
\chordsoff
Ti seguirò nella via della gioia
e la tua luce ci guiderà.
\endverse
\endsong





%titolo{Tu accompagnali}
%autore{}
%album{}
%tonalita{Mi}
%gruppo{}
%momenti{Matrimoni}
%identificatore{tu_accompagnali}
%data_revisione{2011_12_31}
%trascrittore{Francesco Endrici}
\beginsong{Tu accompagnali}[ititle={Per questi nostri amici}]
\beginverse
Per \[E]questi nostri a\[B]mici ti pre\[C#m]ghiamo:
a\[A]iutali a \[B]cammi\[E]nare, a ripen\[B]sare ogni \[C#m]giorno
a \[A]questo giorno di \[E]fe\[C#m]sta, \[A]che Tu doni a \[D^B]noi.
\endverse
\beginchorus
\[E]Tu \[B]accompagna\[A]li, \[E]Dio \[C#m] dell'a\[D]more! \[B]
\[E]Tu \[B]accompagna\[A]li, \[E]Dio \[C#m] dell'a\[D]more! \[B]
\endchorus
\beginverse
\chordsoff
Per questi nostri amici ti preghiamo:
siano segno del tuo amore \brk in un mondo senza sogni
siano sale, siano luce, nostalgia di Te. 
\endverse
\beginverse
\chordsoff
Per questi nostri amici ti preghiamo:
coprili di gioia come solo Tu sai fare,
per accogliere la vita e ringraziare Te.
\endverse
\beginverse
\chordsoff
C'è qualcosa in noi, che Tu ci hai dato,
che solo Tu, Signore, hai potuto inventare,
grande come il cielo e forse anche di più.
\endverse
\endsong

%titolo{Tu ci hai redenti}
%autore{Cerino}
%album{}
%tonalita{}
%gruppo{Anamnesi}
%momenti{Anamnesi}
%identificatore{tu_ci_hai_redenti}
%data_revisione{2011_12_31}
%trascrittore{Francesco Endrici - Manuel Toniato}
\beginsong{Tu ci hai redenti}[by={Cerino}]
\beginverse*
{\itshape \[D]Mistero della fede.}
\[D]Tu ci hai re\[Em]denti con la tua \[A]Croce e risurre\[D]zione,
\[Bm]salvaci o \[B]Salva\[Em]tore, \[A]salvaci o Salva\[D]tore
o \[Em]Salva\[G]tore del \[F#]mon\[B]do.
\endverse
\endsong



%titolo{Tu mi segui}
%autore{Ricci}
%album{È l'incontro della vita}
%tonalita{Re}
%gruppo{}
%momenti{Comunione}
%identificatore{tu_mi_segui}
%data_revisione{2011_12_31}
%trascrittore{Francesco Endrici}
\beginsong{Tu mi segui}[by={Ricci}]
\ifchorded
\beginverse*
\vspace*{-0.8\versesep}
{\nolyrics \[D]\[F#m]\[G]\[D]\[A]\[G]}
\vspace*{-\versesep}
\endverse
\fi
\beginverse
\memorize
Dalle \[D]mani mi sfuggono i \[F#m]giorni veloci:
rac\[G]coglili tu, mio Si\[D]gnore, la \[A]vita ti \[G]dono.
Dentro al \[D]cuore ti tengo sei il \[F#m]grande tesoro,
sei \[G]tu che dai forma ai miei \[D]gesti
e di\[A]segni il cam\[G]mino.
\endverse
\beginchorus
Tu mi \[A]segui, mi guardi ed è \[D]tua la mia gioia, \[G]
il mio dolore, il la\[D]voro; \[G]
Tu che fai del mio \[Bm]povero es\[F#m]sere
un \[G]tuo taber\[D]nacolo \[A]vivo.
\endchorus
\beginverse
%\chordsoff
Fa' che il ^tocco di queste mie ^dita sia il tuo,
sia ca^rezza di luce sui ^volti oscu^rati dal ^pianto.
Fa' che il ^timbro di questa mia ^voce sia il tuo,
sia spe^ranza di vita che il^lumina i ^cuori più ^soli.
\endverse
\beginverse
%\chordsoff
Che con ^questa mia vita il tuo a^more ritorni
nel ^mondo assetato dell'^acqua che ^solo tu ^doni.
Fa' che io ^possa portare sa^pienza e bellezza,
che ^splenda sul nulla di ^me la real^tà di Ma^ria.
\endverse
\ifchorded
\beginverse*
\vspace*{-\versesep}
{\nolyrics \[D]\[F#m]\[G]\[A]\[D]}
\endverse
\fi
\endsong


%titolo{Tu scendi dalle stelle}
%autore{Alfonso De' Liguori}
%album{}
%tonalita{Re}
%gruppo{}
%momenti{Natale}
%identificatore{tu_scendi_dalle_stelle}
%data_revisione{2011_12_31}
%trascrittore{Francesco Endrici}
\beginsong{Tu scendi dalle stelle}[by={Alfonso De'\ Liguori}]
\beginverse
Tu \[D]scendi dalle stelle, o Re del \[A]cielo
e vieni in una grot\[G]ta al \[D]freddo e al \[A]ge\[D]lo,
e \[A]vieni in una grot\[G]ta al \[D]freddo e al \[A]ge\[D]lo.
O Bam\[A]bino, mio Di\[D]vino,
io Ti \[A]vedo qui a tre\[D]mar. O Dio be\[A]ato!
Ah quanto Ti costò \[G]l'a\[D]vermi a\[A]ma\[D]to!
Ah \[A]quanto Ti costò \[G]l'a\[D]vermi a\[A]ma\[D]to!
\endverse
\beginverse
%\chordsoff
A ^Te che sei del mondo il Crea^tore,
non sono panni e fuo^co o ^mio Si^gno^re,
non ^sono panni e fuo^co o ^mio Si^gno^re.
Caro e^letto Pargo^letto,
quanto ^questa pover^tà più m'inna^mora.
Giacchè Ti fece amor ^po^vero an^co^ra!
Giac^chè Ti fece amor ^po^vero an^co^ra!
\endverse
\beginverse
%\chordsoff
Tu ^lasci il bel gioire del divin ^seno,
per giungere a pena^re su ^questo ^fie^no;
per ^giungere a pena^re su ^questo ^fie^no.
Dolce a^more del mio ^cuore, 
dove a^mor ti traspo^rtò! O Gesù ^mio, 
perchè tanto patir, ^^per amor ^mi^o.
Per^chè tanto patir, ^^per amor ^mi^o.
\endverse
\endsong


%titolo{Tu sei}
%autore{Spoladore}
%album{Tu sei}
%tonalita{Do}
%gruppo{}
%momenti{Comunione}
%identificatore{tu_sei}
%data_revisione{2011_12_31}
%trascrittore{Francesco Endrici}
\beginsong{Tu sei}[by={Spoladore}]
\beginverse
Tu \[C]sei la prima \[Dm7]stella del mattino,
Tu \[Em7]sei la nostra \[F]grande nostalgia,
Tu \[Em7]sei il cielo \[Dm]chiaro \[G]dopo la pa\[C]ura,
\[G]dopo la pa\[Am7]ura d'esserci per\[F]duti
e \[Dm7]tornerà la \[C]vita in questo \[G]mare.
\endverse
\beginchorus
Soffie\[F]rà, soffie\[C]rà, il vento \[G]forte della \[Am]vita,
soffie\[F]rà sulle \[C]vele e le \[F]gonfie\[G]rà di \[C]te.
Soffie\[F]rà, soffie\[C]rà, il vento \[G]forte della \[Am]vita,
soffie\[A&]rà sulle \[C]vele e le \[F]gonfie\[G]rà di \[C]te.
\endchorus
\beginverse
%\chordsoff
Tu ^sei l'unico ^volto della pace,
Tu ^sei speranza ^nelle nostre mani,
Tu ^sei il vento ^nuovo ^sulle nostre ^ali,
^sulle nostre ^ali soffierà la ^vita
e ^gonfierà le ^vele per questo ^mare.
\endverse
\endsong

%titolo{Tu sei l'offerta}
%autore{}
%album{}
%tonalita{La-}
%gruppo{}
%momenti{Offertorio}
%identificatore{tu_sei_l_offerta}
%data_revisione{2011_12_31}
%trascrittore{Francesco Endrici}
\beginsong{Tu sei l'offerta}
\beginverse
L'|\[Am]Agnello è ve|\[Dm]nuto a |\[Am]noi \brk dalla |\[Dm]casa di |\[Am]David |\[Am] 
dalla |\[C]stirpe di A|\[F]bramo l'e|\[G]terno \[F]sacer|\[C]dote, |\[C] 
dal |\[F]cielo è di|\[Dm7]sceso come |\[Am]luce, \brk |\[Dm7]nato da Ma|\[E7]ria. |\[E7]
\endverse
\beginchorus
|\[A]Tu sei l'o\mch{C#m}{f}{f}{ff}erta del |\[D]cielo e \[Bm7]della |\[E4]ter\[E7]ra,
|\[A]tu immo|\[C#m]lato, |\[D]tu \[Bm7]ado|\[E4]ra\[E7]to.
|\[A]Cosa posso o\mch{A}{f}{f}{ff}rir\[D]ti, Si|\[E7]gnor? |\[E7]
Gloria a |\[Am]te, \[Em7]a|\[G7]mico dell'|\[C]uomo! |\[E7]
Gloria a |\[Am]te, |\[G7]misericor|\[C]dioso!
|\[Am]Gloria a |\[E7]te, |\[Am]venu\[Dm]to a sal|\[E7]varci! |\[E7]
|\[A]Tu sei l'o\mch{C#m}{f}{f}{ff}erta del |\[D]cielo e \[Bm7]della |\[E4]ter\[E7]ra,
|\[A]tu immo|\[C#m]lato, |\[D]tu \[Bm7]ado|\[E4]ra\[E7]to. |\[E7]|\[E7]
\endchorus
\beginverse
\chordsoff
L'Agnello è venuto a noi, germoglio divino, 
elevato ministro del nuovo sacrificio, 
portando l'offerta del suo corpo nato da Maria.
\endverse
\endsong

%titolo{Tu sei per me}
%autore{Spoladore}
%album{Dacci pace}
%tonalita{Fa}
%gruppo{}
%momenti{Congedo}
%identificatore{tu_sei_per_me}
%data_revisione{2011_12_31}
%trascrittore{Francesco Endrici - Manuel Toniato}
\beginsong{Tu sei per me}[by={Spoladore}]

\beginverse
\[C]Tu sei per \[F]me Padre e \[C]\[Gm7]Ma\[F]dre,
\[C]Tu sei per \[F]me Fratello e A\[G]mico,
\[Em]Tu sei per \[F]me Servo e Si\[C]gno\[Am]re.
\[C]Tu sei il mio \[G]tutto e tutto è in \[F]me. \[C] \[Dm] \[C] \[F] 
\endverse

\beginverse
\chordsoff
Tu sei il Vivente ed io vivo in te
nulla può esistere fuori di Te
anche voi fratelli anche voi
\chordson
rifu\[C]giatevi \[G]solo, solo in \[F]\[C]lui. \[Dm] \[C] \[B&] \[F] 
\endverse

\beginverse
\[D]Noi ti ringra\[G]ziamo e ti lo\[D]\[Am7]dia\[G]mo 
ti chie\[D]diamo quello \[G]che Tu hai pro\[A]messo
anche \[F#m]se deboli \[G]uomini noi \[D]sia\[Bm]mo
\[D]amiamo in \[A]Te il solo \[G]Amo\[D]re. \[Em] \[D] \[G] \[D] \[Em] \[D]
\endverse

\beginverse
\[E]Tu sei per \[A]me Padre e \[E]\[Bm7]Ma\[A]dre,
\[E]Tu sei per \[A]me Fratello e A\[B]mico,
\[G#m]Tu sei per me \[A]Servo e Si\[E]gnore. \[C#m] 
\[E]Tu sei il mio \[B]tutto e tutto è in \[A]me. \[E] \[F#m] \[E] \[A] \[E] \[F#m] \[E] 
\endverse

\beginverse
\[F]Tu sei per \[B&]me Padre e \[F]\[Cm7]Ma\[B&]dre, 
\[F]Tu sei per \[B&]me Fratello e A\[C]mico,
\[Am]Tu sei per \[B&]me Servo e Si\[F]gno\[Dm]re.
\[F]Tu sei il mio \[C]tutto e tutto è in \[B&]me. \[F] \[Gm] \[F] \[B&] \[F] \[Gm7] \[F7] 
\endverse
\endsong

%titolo{Tu sei prezioso ai miei occhi}
%autore{Crestani, Refosco}
%album{MGS Triveneto, Festa dei Giovani 2008}
%tonalita{}
%gruppo{}
%momenti{}
%identificatore{tu_sei_prezioso_ai_miei_occhi}
%data_revisione{2011_12_31}
%trascrittore{Francesco Endrici - Manuel Toniato}
\beginsong{Tu sei prezioso ai miei occhi}[by={Crestani, Refosco}]
\ifchorded
\beginverse*
\vspace*{-0.8\versesep}
{\nolyrics \[D] \[A]  \[D] \[A] \[Bm] \[G]  \[D] \[A] \[G]  \[D] \[A] \[Bm] \[G]  \[D] \[A] }
\vspace*{-\versesep}
\endverse
\fi

\beginverse
\[D] Apro gli occhi e scopri\[Bm]rò \brk le tante cose che non \[D]so
nascoste nella gente ac\[A]canto a me
\[Em] chi guarda bene capi\[Bm]rà è dentro noi la veri\[D]tà
siamo pronti a condi\[A]viderla?
\[Em] Specchiandoci\[A] negli occhi tuoi\[Bm]  \[D] 
\[Em] seguiamo il tempo che ci \[D]dai \brk sulla musica che a\[A]desso vuoi!
\endverse

\beginchorus
\[D]Tu sei pre\[A]zioso ai miei \[Bm]occhi, \[G]  \brk \[D]Tu sei pre\[A]zioso per \[G]me.
\[D]Tu sei pre\[A]zioso ai miei \[Bm]occhi, \[G]  \brk \[D]mi sento \[A]dire da \[E]te.
\endchorus

\beginverse
\chordsoff
Mi sento pronto a svolgere l'incarico affidato a me
supererò tutti gli ostacoli.
La forza non mi manca sai \brk ho un cuore vivo e lo vedrai
e c'è il tuo sguardo che ora è su di me.
Specchiandomi negli occhi tuoi
il tempo che ora tu mi dai \brk è musica da vivere per noi!
\endverse

\beginchorus
\chordsoff
Tu sei prezioso ai miei occhi, \brk Tu sei prezioso per me.
Tu sei prezioso ai miei occhi, \brk Ti sento urlare per me.
\endchorus

\beginverse*\itshape
\[G]Sogno di viaggiare, supe\[A]rare le barriere e i limi\[Bm]ti
(io sarò il tuo co\[D]raggio se mi cerchi)
\[A]insegnami a volare tu che \[Em]puoi
(per andare in alto tu chiama con lo sguardo)
per arri\[A]vare a gridare più forte.
\endverse

\beginchorus
\chordsoff
Tu sei prezioso ai miei occhi, \brk Tu sei prezioso per me.
Tu sei prezioso ai miei occhi, \brk voglio gridare ora Te.

Tu sei prezioso ai miei occhi, \brk Tu sei prezioso per me.
Tu sei prezioso ai miei occhi, \brk Tu sei prezioso per noi.
\endchorus
\endsong


%titolo{Tu sei vivo fuoco}
%autore{}
%album{Tu sei come roccia}
%tonalita{Re}
%gruppo{}
%momenti{}
%identificatore{tu_sei_vivo_fuoco}
%data_revisione{2011_12_31}
%trascrittore{Francesco Endrici - Manuel Toniato}
\beginsong{Tu sei vivo fuoco}

\beginverse
\[D]Tu sei vivo \[A]fuoco \[Bm]che trionfi a \[F#m]sera
\[G]del mio \[D]giorno \[Em]sei la \[A]bra\[D]ce.
\[D]Ecco già ros\[A]seggia \[Bm]di bellezza e\[F#m]terna,
\[G]questo \[D]giorno \[Em]che si \[A]spe\[D]gne.
\[Bm]Se con \[Em]te, \[A]come \[D]vuoi, 
l'\[G]anima ri\[A]scal\[D]do, \[G]sono nella \[A]pa\[D]ce.
\endverse

\beginverse
\chordsoff
Tu sei fresca nube che ristori a sera,
del mio giorno sei rugiada.
Ecco già rinasce di freschezza eterna,
questo giorno che sfiorisce.
Se con te, come vuoi, 
cerco la sorgente, sono nella pace.
\endverse

\beginverse
\chordsoff
Tu sei l'orizzonte che s'allarga a sera,
del mio giorno sei dimora.
Ecco già riposa in ampiezza eterna,
questo giorno che si chiude.
Se con te, come vuoi, 
m'avvicino a casa, sono nella pace.
\endverse

\beginverse
\chordsoff
Tu sei voce amica che mi parli a sera,
del mio giorno sei conforto.
Ecco già risuona d'allegrezza eterna
questo giorno che ammutisce.
Se con te, come vuoi, 
cerco la parola, sono nella pace.
\endverse

\beginverse
\chordsoff
Tu sei sposo ardente che ritorni a sera, 
del mio giorno sei l'abraccio.
Ecco già esulta di ebrezza eterna
questo giorno che sospira.
Se con te, come vuoi, 
mi consumo amando sono nella pace.
\endverse
\endsong



%titolo{Tuo è il regno}
%autore{}
%album{}
%tonalita{Re}
%gruppo{Dossologia}
%momenti{Dossologia}
%identificatore{tuo_e_il_regno}
%data_revisione{2011_12_31}
%trascrittore{Francesco Endrici - Manuel Toniato}
\beginsong{Tuo è il regno}
\beginverse*
\[D]Tuo è il \[A]Regno, \[Bm]tua la \[F#m]potenza
\[G]e \[A]la \[D]glo\[G]ria nei \[D]se\[A]co\[D]li.
\endverse
\endsong



%titolo{Tutta la terra}
%autore{Ricci}
%album{Sei venuto dal cielo}
%tonalita{Re}
%gruppo{}
%momenti{Natale;Salmi}
%identificatore{tutta_la_terra}
%data_revisione{2011_12_31}
%trascrittore{Francesco Endrici - Manuel Toniato}
\beginsong{Tutta la terra}[by={Ricci}]

\ifchorded
\beginverse*
\vspace*{-0.8\versesep}
{\nolyrics \[D] \[A] \[Bm] \[G] }
\vspace*{-\versesep}
\endverse
\fi
\beginchorus
\[D]Tutta la \[A]terra ha ve\[Bm]duto la sal\[G]vezza \brk del Si\[D]gnore \[A] \[Bm] \[G] 
\endchorus

\chordsoff
\beginverse
Can\[Em]tate al Signore un canto \[Bm]nuovo,
un canto \[A]nuovo perché ha compiuto pro\[D]digi.
Gli ha \[F#7]dato vittoria la sua \[Bm]destra,
la sua \[F#7]destra e il suo braccio \[Bm]santo.\[A] 
\endverse

\beginverse
Il Signore ha mostrato la salvezza,
a tutti i popoli qual è la sua giustizia.
Lui si è ricordato del suo amore
verso il popolo di Israele.
\endverse

\beginverse
Tutti i confini della terra
hanno veduto la salvezza di Dio.
Acclami al Signore tutto il mondo,
esultate con canti di gioia.
\endverse

\beginverse
Cantate inni al Signore
Con l'arpa e con suoni, suoni melodiosi,
con trombe e con suoni di corno
acclamate al Re, il Signore.
\endverse
\endsong

%titolo{Tutta la terra attende}
%autore{Baggio}
%album{Maranathà}
%tonalita{Do#-}
%gruppo{}
%momenti{Natale}
%identificatore{tutta_la_terra_attende}
%data_revisione{2011_12_31}
%trascrittore{Francesco Endrici}
\beginsong{Tutta la terra attende}[by={Baggio}]
\ifchorded
\beginverse*
\vspace*{-0.8\versesep}
{\nolyrics \[C#m]\[B]\[A]\[E]\[F#m7] \[B]\[E]\[E]\[E]}
\vspace*{-\versesep}
\endverse
\fi
\beginchorus
Tutta la \[B]terra at\[C#m]tende impa\[A]ziente \[E]
che si ri\[B]velino i \[E]figli di \[B]Dio \[E]
e soffre an\[B]cora le \[C#m]doglie del \[A]parto, \[E] \brk aspetta \[B7]il suo Mes\[E]sia. \[E]\[A]
\endchorus
\beginverse
\memorize
Germoglio \[A]della ra\[E]dice di \[E]Jesse \[B4] \brk ti innalzi \[B]segno per \[E]noi. \[E] \[B]
vieni Si\[B]gnore a sal\[A]vare il tuo \[A]popolo  \brk \[E]dona \[E] la liber\[\vline]\[B4]\[B]\[\vline]\[E]\[B]tà 
Vieni Si\[E]gnore a sal\[A]vare il tuo \[A]popolo  \brk \[E]dona \[E] la liber\[B4]\[B]\[E]tà 
\endverse
\beginchorus
\chordsoff 
Rit. 
\endchorus
\beginverse
Oh chiave ^della fa^miglia di ^Davide ^  \brk ci apri il ^regno di ^Dio ^^
vieni Si^gnore ri^schiara le ^tenebre  \brk ^vinci ^ l'oscuri^^^^tà. ^^
Vieni Si^gnore ri^schiara le ^tenebre  \brk ^vinci ^ l'oscuri^^^tà.
\endverse
\beginchorus
\chordsoff 
Rit. 
\endchorus
\beginverse
O stella ^che fai bril^lare la ^notte ^  \brk splendi di ^luce per ^noi, ^^
vieni Si^gnore e il^lumina il ^misero, \brk  ^sana ^ la ceci^^^^tà. ^^
Vieni Si^gnore e il^lumina il ^misero,  \brk ^sana ^ la ceci^^^tà.
\endverse
\beginchorus
\chordsoff 
Rit. 
\endchorus
\beginverse
Re delle ^genti so^stieni la ^chiesa, ^  \brk pietra ango^lare sei ^tu ^^
vieni Si^gnore e ^salva il tuo ^popolo,  \brk ^tutta ^ l'umani^^^^tà. ^^ 
Vieni Si^gnore e ^salva il tuo ^popolo,  \brk ^tutta ^ l'umani^^^tà. 
\endverse
\beginchorus
Tutta la \[B]terra at\[C#m]tende impa\[A]ziente \[E]
che si ri\[B]velino i \[E]figli di \[B]Dio \[E]
e soffre an\[B]cora le \[C#m]doglie del \[A]parto, \[E] \brk aspetta \[B7]il suo Mes\[E]sia. \[B]\[E]
Tutta la \[B]terra at\[C#m]tende impa\[E]ziente \[A] \brk aspetta \[B7]il suo Mes\[E]sia. 
\endchorus
\endsong

%titolo{Tutta la vita è un dono}
%autore{Gen Rosso}
%album{}
%tonalita{Do}
%gruppo{}
%momenti{Congedo}
%identificatore{tutta_la_vita_e_un_dono}
%data_revisione{2011_12_31}
%trascrittore{Francesco Endrici - Manuel Toniato}
\beginsong{Tutta la vita è un dono}[by={Gen\ Rosso}]

\ifchorded
\beginverse*
\vspace*{-0.8\versesep}
{\nolyrics \[C] \[B&] \[C] \[C4/7] \[C] \[B&] }
\vspace*{-\versesep}
\endverse
\fi

\beginverse
\[C]Tutta la vita è un dono \[G]per o\[F]gni \[C]uomo
\[Dm]tutta la vita è un \[Am]dono \[G]in ogni \[F]momen\[C]to
\[Am]tutta la vita è un \[B&]dono, canta per \[G]lei, \[F]canta per \[C]lei.
\endverse

\beginchorus
\[C]Corri corri a perdifiato \brk contro la cor\[Em7]rente che trascina
in\[Am7]frangi la marea di \[Dm7]chi ti porta giù
di\[C]struggi le barriere del\[Em7]l'indifferenza
sor\[F]ridi a chi non sa \[Am]che la vita \[B&]non si ferma \[G4]mai.
\endchorus

\chordsoff
\beginverse
Tutta la vita è un dono, non la sciupare;
tutta la vita è un dono, non la bruciare
tutta la vita è un dono: vivi per lei, vivi per lei.
\endverse

\beginverse
Anche nei giorni tristi la vita è amore;
anche tra le bufere la vita è amore;
la vita è sempre un dono che Dio ci dà, \brk che Dio ci dà.
\endverse
\endsong

\lettera
%titolo{Un cuore nuovo}
%autore{Buttazzo}
%album{Vita nuova con Te}
%tonalita{Mi-}
%gruppo{}
%momenti{Quaresima}
%identificatore{un_cuore_nuovo}
%data_revisione{2011_12_31}
%trascrittore{Francesco Endrici - Manuel Toniato}
\beginsong{Un cuore nuovo}[by={Buttazzo}]

\beginchorus
Ti da\[Em]rò un cuore \[Bm7]nuovo popolo \[Em7]mio.
Il mio \[G]Spirito e\ch{C}{f}{f}{ff}onderò in \[G]te.
Toglie\[Am]rò da \[Em]te il \[Bm]cuore di \[Em]pietra,
un \[Am]cuore di \[Em]carne ti da\[Bm]rò popolo \[Em]mio.
\endchorus

\beginverse
Da \[D]tutte le na\[G]zioni \[D/F#]vi radune\[G]rò,
vi \[Am]mostrerò la \[G6]strada della \[D/F#]{vita}   \[D] 
e viv\[Em]rà chi la \[Bm7]segui\[Em]rà.
\endverse

\chordsoff
\beginverse
Vi aspergerò con acqua e puri vi farò.
Dagli idoli sarete liberati.
Questa è la mia libertà.
\endverse

\beginverse
Mio popolo sarete, le genti lo vedranno.
Abiterete dentro la mia casa.
E vedrete il mio volto.
\endverse
\endsong


%titolo{Un giorno fra le mie mani}
%autore{Gen Rosso}
%album{Se siamo uniti}
%tonalita{Re}
%gruppo{}
%momenti{}
%identificatore{un_giorno_fra_le_mie_mani}
%data_revisione{2011_12_31}
%trascrittore{Francesco Endrici}
\beginsong{Un giorno fra le mie mani}[by={Gen\ Rosso}]
\beginverse
Un \[D]giorno fra le mie \[A]mani,
un \[D]giorno qui davanti a \[F#m7]me,
che cosa \[G]mai fa\[D]rò perché alla \[Em7]fine \[D]tu
ne sia fe\[A4]lice? \[A]
Oh! come vor\[D]rei in ogni mo\[A]mento
strap\[D]pare questa oscuri\[F#m7]tà
che scende e \[G]non mi \[D]fa
guardare \[Em7]al di \[D]là dei passi \[A4]miei. \[A]
\endverse
\beginchorus
Come vor{\[Bm7]rei} a\[Em7]marti
in \[F#m7]chi cammina accanto a \[G]me,
in \[Em7]chi incrocia la mia \[A]vita,
in \[F#m7]chi mi sfiora ma non \[G]sa
che tu sei \[D]lì con \[Em7]lui.
È \[D]quello che più vor\[G]rei,
È \[E7]quello che più vor\[D]rei per \[A4]te.
\endchorus
\beginverse
\chordsoff
La strada piena di gente,
ma l'orizzonte è tutto lì,
la folla se ne va tra un negozio e un bar \brk indifferente.
Oh! come vorrei parlare ad ognuno,
così come faresti tu,
della felicità, di quella pace che \brk tu solo dai.
\endverse
\beginchorus
\chordsoff
Così vorrò amarti
negli ultimi della città,
nel buio di chi muore solo,
in chi dispera e non sa
che tu sei lì con lui.
Così oggi ti amerò,
così oggi ti amerò di più.
\endchorus
\endsong


%titolo{Una cosa sola in Te}
%autore{Giovannini}
%album{}
%tonalita{Do}
%gruppo{}
%momenti{Pace}
%identificatore{una_cosa_sola_in_te}
%data_revisione{2011_12_31}
%trascrittore{Francesco Endrici}[by={Giovannini}]
\beginsong{Una cosa sola in Te}
\beginchorus
\[C]Pace, nascerà la \[Em]pace, \[F]se Tu resti in mezzo a \[C]noi
\[Dm]se sa\[G]rò \[C]cari\[Am]tà
\[Dm7]oltre la spe\[D7]ranza io ti in\[G4]contre\[7]rò.
\[C]Apri le tue porte a \[Em]Cristo
\[F]e nel mondo cresce\[C]rà
\[Dm]l'uni\[G]tà \[C]fra di \[Am]noi
\[F]una cosa \[G]sola noi sa\[F]remo in \[C]Te.
\endchorus
\beginverse
Io sono \[Em]giovane Si\[Am]gnore, \[C]
le ali al \[G]mondo spieghe\[F]rei. \[C]
Ma in fondo \[Em]vince la pa\[Am]ura \[C]
mi siedo e \[Em]resto lì \[F] a guardare il cielo. \[G]
A \[F]volte mi ri\[G]trovo solo e \[Am]cresce la mia povertà
e al\[F]lora c'è bisogno di for\[Dm]mare una catena
con le \[G]mani tue, fratello mio.
\endverse
\beginverse
%\chordsoff
Io sento il ^canto della ^vita ^
che nasce ^dentro il cuore ^mio. ^
E in fondo ^nasce la mia ^gioia ^
se credo ^che si può ^ incontrare amore. ^
Se ^guarderò di^ritto al cuore
le ^tue ricchezze troverò
e ^sentirò il bisogno di for^mare una catena
con le ^mani tue, fratello mio.
\endverse
\endsong



%titolo{Una voce nell'anima}
%autore{Lacchin}
%album{}
%tonalita{Do#-}
%gruppo{}
%momenti{}
%identificatore{una_voce_nell_anima}
%data_revisione{2011_12_31}
%trascrittore{Francesco Endrici}
\beginsong{Una voce nell'anima}[by={Lacchin}]
\ifchorded
\beginverse*
\vspace*{-0.8\versesep}
{\nolyrics \[C#m]\[A]\[E]\[B] \rep{4}}
\vspace*{-\versesep}
\endverse
\fi
\beginverse
\[C#m]Tutto inizia sempre per a\[A]more,
per a\[E]more di qual\[B]cuno.
\chordsoff
Come la rugiada del mattino
gonfia il cuore della terra.
E allora dimmi se è possibile sperare
in questa vita, in un temporale,
che riempia tutti i cuori in una volta sola
e a far cadere ogni paura.
\endverse
\beginverse
\chordsoff
Continuate stolti a zoppicare,
sul bastone vi curverete.
Vi ostinate ad esser ciechi e sordi,
e non berrete e non vivrete.
Che questa carestia vi asciughi fino al cuore,
si inaridisca finanche il mare
se non seguite quel mio vecchio vagabondo,
che vi è profeta e insegna al mondo.
\endverse
\beginchorus
Ad amare \[C#m]te, ad amare \[A]me \[E]\[B] \chordsoff 
\echo{Come se fossimo io e te}
Ad amare te, ad amare me 
\echo{Un viaggio solo ed unico}
Ad amare te, ad amare me 
\echo{Senza più correre chissà}
Ad amare te, ad amare me 
\echo{Come fai da un'eternità.}
\endchorus
\beginverse
\chordsoff
Abbassa o re la cresta del tuo orgoglio, 
non ferirti con la tua spada. 
Richiama tutti i sacerdoti del tuo dio, 
del tuo dio peccatore. 
Ed apri gli occhi nella fede del Signore, 
di quello vero, che sa creare 
dal nulla monti, mari, fiumi e le farfalle, 
abbraccia l'Uomo che è Salvatore.
\endverse
\beginchorus
\chordsoff
Ad amare te, ad amare me 
\echo{Come se fossimo io e te}
Ad amare te, ad amare me 
\echo{Un viaggio solo ed unico}
Ad amare te, ad amare me 
\echo{Senza più correre chissà}
Ad amare te, ad amare me 
\echo{Come fai da un'eternità.}
\endchorus
\beginverse
\chordsoff
Una voce nell'anima,
sembra un tuono, una freccia va,
sopra i campi e le città
la mia gente ti seguirà. \rep{4}
\endverse
\textnote{Sulle ultime due ripetizioni si canterà:}
\beginchorus
\chordsoff
Ad amare te, ad amare me 
\echo{Come se fossimo io e te}
Ad amare te, ad amare me 
\echo{Un viaggio solo ed unico}
Ad amare te, ad amare me 
\echo{Senza più correre chissà}
Ad amare te, ad amare me 
\echo{Come fai da un'eternità.}
\endchorus
\endsong


%titolo{Uno siamo noi}
%autore{Spoladore}
%album{Tu sei}
%tonalita{Mi}
%gruppo{}
%momenti{}
%identificatore{uno_siamo_noi}
%data_revisione{2011_12_31}
%trascrittore{Francesco Endrici - Manuel Toniato}
\beginsong{Uno siamo noi}[by={Spoladore}]

\beginverse
\[E9]Uno è il giorno che \[Am6/E]dura la Vita.
Mille \[E9]albe e tramonti per il \[Am6/E]giorno che vivi.
\[C#m9]Uno è il fuoco di mi\[D9]liardi di \[D]stelle
\[A]unico il coro delle \[E]onde dei mari,
\[E9]uno è l'amore di mi\[Am6/E]lioni di abbracci,
mi\[E9]lioni di pianti e \[Am6/E]Uno è il Dolore.
\[C#m9]Uno è il tempo esiste \[D9]solo il Pre\[D]sente,
\[A]uno è la Vita la \[E]Vita che hai. 
\endverse

\beginchorus
Uno siamo \[E]\[E4]\[E]noi \[]e non siamo \[A]so\[D9]li \[A]mai.
Uno siamo \[C#m]noi siamo \[B]noi
Impa\[D]rare ad amare è il \[B]nostro destino.
Metti \[A]tutta la forza che \[C]hai,
vivi \[A/B]tutto l'Amore che \[E]puoi.
\endchorus
\ifchorded
\beginverse*
\vspace*{-\versesep}
{\nolyrics \[E4] \[E] \[D9] \[A] \[F#m7] \[A7] }
\endverse
\fi
\beginverse
\chordsoff
Solo il Male divide e ci fa dubitare
Non ci sono nemici ci son solo infelici
Infelici da amare e basta anche un sorriso
Uno è il Sogno Uno le nostre mani
Molto è nascosto ma nulla è per caso
Tutto è diverso ma nulla è diviso
Uno è il Respiro che riempie la Vita
Uno o Nessuno è il nostro destino.
\endverse

\beginchorus
Uno siamo \[E]\[E4]\[E]noi \[]e non siamo \[A]so\[D9]li \[A]mai.
Uno siamo \[C#m]noi siamo \[B]noi
Impa\[D]rare ad amare è il \[B]nostro destino
Metti \[A]tutta la forza che \[C]hai
Vivi \[A/B]tutto l'Amore che puoi.
\endchorus
\ifchorded
\beginverse*
\vspace*{-\versesep}
{\nolyrics \[F#] \[F#4] \[F#]  \[B] \[E9] \[B] \[D#m] \[C#] \[E] \[C#] }
\endverse
\fi
\beginchorus
Metti \[B]tutta la forza che \[D]hai
Vivi \[E9]tutto l'Amore che \[F#]puoi 
Metti \[B]tutta la forza che \[D]hai
Vivi \[E9]tutto l'Amore che \[F#]puoi. \[F#4] \[F#] \[E9] \[B] \[F#] 
\endchorus
\endsong

%titolo{Uomini nuovi}
%autore{Fanelli}
%album{Uomini nuovi}
%tonalita{Mi}
%gruppo{}
%momenti{Congedo}
%identificatore{uomini_nuovi}
%data_revisione{2011_12_31}
%trascrittore{Francesco Endrici}
\beginsong{Uomini nuovi}[by={Fanelli}]
\transpose{3}
\beginchorus
\[A]Uomini \[E]nuovi sa\[B]remo in \[E]te
\[A]cammine\[E]remo nella \[A]tua novi\[B4]tà \[B]
\[A]e cresce\[E]remo ogni \[B]giorno nel \[E]sole
\[A]gridando al \[E]mondo la \[B]tua liber\[E]tà. \[F#m7]
\endchorus
\beginverse
Se trasfor\[B]mate il vostro \[E]cuore \[F#m7]
se accoglie\[B]rete la Pa\[E]rola
come la \[A7+]pioggia che scende giù dal \[G#m7]cielo
nasce\[F#m7]rà un mondo di \[B]luce. \[F#m7]
Se accoglie\[B]rete in voi l'a\[E]more \[F#m7]
se lo vi\[B]vrete tra di \[E]voi
come fra\[A7+]telli con la mano nella \[G#m7]mano
ritorne\[F#m7]rà la nuova prima\[A7+]vera. \[B]
\endverse
\beginverse
Se lotte^rete per la ^pace ^
se cerche^rete la giu^stizia
se difen^dete la dignità dell'^uomo
rinasce^rà nel mondo la spe\[A7+]ranza. \[B]
\endverse
\beginchorus
\transpose{2}
\[(B)] \[A]Uomini \[E]nuovi sa\[B]remo in \[E]te
\[A]cammine\[E]remo nella \[A]tua novi\[B4]tà \[B]
\[A]e cresce\[E]remo ogni \[B]giorno nel \[E]sole
\[A]gridando al \[E]mondo la \[B]tua liber\[E]tà.
\endchorus
\endsong

\lettera
%titolo{Va' per le vie del mondo}
%autore{Pettenon, Spiritual}
%album{}
%tonalita{Mi}
%gruppo{}
%momenti{Congedo}
%identificatore{va_per_le_vie_del_mondo}
%data_revisione{2011_12_31}
%trascrittore{Francesco Endrici - Manuel Toniato}
\beginsong{Va' per le vie del mondo}[by={Pettenon, Spiritual}]

\beginchorus
\[E]Va' \[A]per le vie del \[E]mondo \brk \[F#m]dove la \[B]gente \[E]cerca \[A]pace.
\[E]Va', \[A]porta la pa\[E]rola: \brk è \[A]Cristo in \[B7]mezzo a \[E]noi.
\endchorus

\beginverse
\[E]Quando cercavo \[G#m]{luce} \[F#m]vedevo il mondo \[E]triste,
\[E]pregavo notte e \[G#m]giorno e \[A]Dio mi ha \[F#7]det\[B7]o:
\endverse

\chordsoff
\beginverse
Oggi mi sento un altro, sono un cristiano nuovo,
rinato in questo giorno mi sento l'ultimo.
\endverse

\beginverse
Ora che abbiamo Dio vicino a casa nostra,
non lo cerchiamo in cielo ma nel fratello.
\endverse

\beginverse
Uniti nel suo nome, Gesù è in mezzo a noi;
Così la chiesa splende per ogni uomo.
\endverse
\endsong

%titolo{Venimus adorare eum}
%autore{Linssen}
%album{Inno della XX GMG, Colonia 2005}
%tonalita{Re}
%gruppo{}
%momenti{}
%identificatore{venimus_adorare_eum}
%data_revisione{2011_12_31}
%trascrittore{Francesco Endrici - Manuel Toniato}
\beginsong{Venimus adorare eum}[by={Linssen}]
\ifchorded
\beginverse*
\vspace*{-0.8\versesep}
{\nolyrics \[D] \[G] \[A4] \[A] }
\vspace*{-\versesep}
\endverse
\fi

\beginverse
\[D]Chiedi perché par\[G]tire \[Em]dal proprio \[A]regno
\[Bm]solo per inse\[G]guire \[Em]una stella e per\[A]ché
\[G]per un Bimbo pi\[E7]egano quelle gi\[D]nocchia da \[Bm]Re?
\[Em] Tu la risposta  sai che \[A]è:
\endverse

\beginchorus
Ve\[D]nimus adorare Eum, Em\[C]manuel \[G]Dio con noi
\[D]Venimus adorare \[C]Eum, Em\[G]manuel \rep{2}
\endchorus

\chordsoff
\beginverse
Chiedi perché lasciare sui monti il gregge
solo per ascoltare un canto e perché
per un Bimbo piegano quelle ginocchia, perché?
Tu la risposta sai che è:
\endverse

\beginverse
Ecco da lontano per adorarlo \brk siamo giunti anche noi,
noi, tutti figli suoi, profeti e sacerdoti ormai.
Nel pane e nel vino noi siamo in lui e lui è in noi:
e un canto qui si alza già.
\endverse
\endsong





%titolo{Venite fedeli}
%autore{Stefani, Wade}
%album{}
%tonalita{Fa}
%gruppo{}
%momenti{Natale}
%identificatore{venite_fedeli}
%data_revisione{2011_12_31}
%trascrittore{Francesco Endrici}
\beginsong{Venite fedeli}[by={Stefani, Wade}]
\beginverse
Ve\[F]nite, fe\[C]deli, l'\[F]an\[C]ge\[F]lo \[B&]ci in\[F]vi\[C]ta, 
ve\[Dm]ni\[C]te, \[B&]ve\[C]\[Dm]ni\[C]te \[F]a Be\[C]\[G7]tlem\[C]me.
\endverse
\beginchorus
\[F]Na\[Gm]sce per \[C7]no\[F]i \[Gm]Cristo \[Dm]Salva\[C]\[B&]to\[C]re.
Ve\[F]ni\[C]te, \[F]a\[C]do\[F]ria\[C]mo, ve\[F]ni\[C]te, \[F]a\[B&]do\[F]ria\[C]mo, 
\[F]ve\[B&]ni\[F]te, \[Gm]ado\[C]ria\[Dm]mo \[Gm]il Si\[F]gno\[C7]re Ge\[F]sù!
\endchorus
\beginverse
\chordsoff
La luce del mondo brilla in una grotta: 
la fede ci guida a Betlemme.
\endverse
\beginverse
\chordsoff
La notte risplende, tutto il mondo attende; 
seguiamo i pastori a Betlemme.
\endverse
\beginverse
\chordsoff
Il Figlio di Dio, Re dell'universo, 
si é fatto bambino a Betlemme.
\endverse
\beginverse
\chordsoff
«Sia gloria nei cieli, pace sulla terra» 
un angelo annuncia a Betlemme.
\endverse
\endsong

%titolo{Verbum panis}
%autore{Casucci, Balduzzi}
%album{Verbum panis}
%tonalita{Mi-}
%gruppo{}
%momenti{Comunione}
%identificatore{verbum_panis}
%data_revisione{2011_12_31}
%trascrittore{Francesco Endrici}
\beginsong{Verbum panis}[by={Casucci, Balduzzi}]
\ifchorded
\beginverse*
\vspace*{-0.8\versesep}
{\nolyrics \[Em]\[(6x)]\[\vline]\[C7+]\[D]\[\vline]\[Em]}
\vspace*{-\versesep}
\endverse
\fi
\beginverse
\memorize
\[Em]Prima del \[D]tempo
prima an\[Em]cora che la \[D]terra
comin\[Em]ciasse a vive\[D]re 
\[Em]il Verbo \[D]era presso \[Em]Dio. \[D]\[Em]\[D]
\[Em]Venne nel \[D]mondo
e per \[Em]non abbando\[D]narci
in questo \[Em]viaggio ci la\[D]sciò
\[Em]tutto sé \[D]stesso come \[Em]pane. \[D] \[Em]
\endverse
\beginchorus
Verbum \[Em]caro factum est \[Em]
Verbum \[Em]panis factum est \[Em]
Verbum \[Em]caro factum est \[Em]
Verbum \[Em]panis factum \[C7+]est. \[C7+]\[D4]\[D]
\endchorus
\beginverse
\[G]Qui \[D]spezzi ancora il \[C]pane in mezzo a \[D]noi
e chi\[G]unque mange\[D]rà \[C]non avrà più \[D]fame.
\[G]Qui \[D]vive la tua \[C]chiesa intorno a \[D]te
dove o\[G]gnuno trove\[D]rà \brk \[C]la sua vera \[D]casa. \[Em]
\endverse
\beginchorus
Verbum \[Em]caro factum est \[Em]
Verbum \[Em]panis factum est \[Em]
Verbum \[Em]caro factum est \[Em]
Verbum \[Em]panis 
\endchorus
\beginverse
^Prima del ^tempo
quando l'^universo ^fu creato
^dall'oscuri^tà
^il Verbo ^era presso ^Dio. ^^^
^Venne nel ^mondo
nella ^sua miseri^cordia
Dio ha man^dato il Figlio ^suo
^tutto sé ^stesso come ^pane. ^ ^
\endverse
\beginchorus
Verbum \[Em]caro factum est \[Em]
Verbum \[Em]panis factum est \[Em]
Verbum \[Em]caro factum est \[Em]
Verbum \[Em]panis factum \[C7+]est. \[C7+]\[D4]\[D]
\endchorus
\beginverse
\[G]Qui \[D]spezzi ancora il \[C]pane in mezzo a \[D]noi
e chi\[G]unque mange\[D]rà \[C]non avrà più \[D]fame.
\[G]Qui \[D]vive la tua \[C]chiesa intorno a \[D]te
dove o\[G]gnuno trove\[D]rà \brk \[C]la sua vera \[D]casa. \[Em]
\endverse
\beginchorus
Verbum \[Em]caro factum est \[Em]
Verbum \[Em]panis factum est \[Em]
Verbum \[Em]caro factum est \[Em]
Verbum \[Em]panis factum \[Em]est. 
\endchorus
\endsong

%titolo{Vi amo così}
%autore{Brusati, Versaci}
%album{}
%tonalita{Re}
%gruppo{}
%momenti{Pasqua;Congedo}
%identificatore{vi_amo_cosi}
%data_revisione{2011_12_31}
%trascrittore{Francesco Endrici - Manuel Toniato}
\beginsong{Vi amo così}[by={Brusati, Versaci}]

\ifchorded
\beginverse*
\vspace*{-0.8\versesep}
{\nolyrics \[D]   \[A]  \[Bm9]   \[G] }
\vspace*{-\versesep}
\endverse
\fi

\beginverse
\[D]C'è una novità \[G]nella nostra storia
\[D]Dio che si fà uomo e porta il \[G]cielo sulla terra
\[Bm]Con la nostra \[A]vita si ri\[G]vela 
nello \[Bm]Spirito che \[A]soffia veri\[G]tà.

\[F9]C'è una novità: \[B&]che la morte è vinta
\[F9]e la gioia in cuore a tutto il \[B&]mondo poi racconta,
\[Dm]annunciando an\[C]cora la Pa\[B&]rola 
che ci a\[Dm]iuta a ritor\[C]nare insieme \[B&]qui:
\endverse

\beginchorus
``Vi amo co\[F]sì e vi ame\[C]rò come vi ho a\[Dm]mati, \brk con \[C]voi reste\[B&]rò
e via da \[F]qui ognuno sa\[C]rà \brk il testi\[Dm]mone del\[C]la Cari\[B&7+]tà,
il testi\[Dm]mone di \[C]Me, Veri\[F]tà.'' \[C]    \[Dm]     \[B&]   
\endchorus

\beginverse
\[F9]C'è una novità: \[B&]Cristo è la speranza,
\[F9]che da noi dilaga, fino ai con\[B&]fini della terra
\[Dm]Figli della \[C]Chiesa, madre e a\[B&]mica
che ri\[Dm]vela la pro\[C]messa ancora \[B&]qui:
\endverse

\beginchorus
``Vi amo co\[F]sì e vi ame\[C]rò come vi ho a\[Dm]mati, \brk con \[C]voi reste\[B&]rò
e via da \[F]qui ognuno sa\[C]rà \brk il testi\[Dm]mone del\[C]la Cari\[B&7+]tà,
il testi\[Dm]mone di \[C]Me, Veri\[F]tà.'' \[C]    \[Dm]     \[B&]  
\endchorus

\beginverse*\itshape
\[Gm7]Senti nel vento il \[A/F]grido del mondo
\[B&9]che si alza e arriva qui, \[Dm7/9]fino a \[C]noi!
\[Gm7]Chiede da sempre, \[A/F]lo sai, d'esser\[C4]ci (d'esserci)
\endverse

\ifchorded
\beginverse*
\vspace*{-\versesep}
{\nolyrics \[F]  \[C]    \[Dm]     \[C]    \[B&] }
\endverse
\fi

\beginchorus
\ldots E via da \[F]qui ognuno sa\[C]rà \brk il testi\[Dm]mone del\[C]la Cari\[B&7+]tà,
il testi\[Dm]mone di \[C]Me, Veri\[B&]tà.
Vi amo co\[G]sì e vi ame\[D]rò come vi ho a\[Em]mati, \brk con \[D]voi reste\[C]rò
e via da \[G]qui ognuno sa\[D]rà \brk il testi\[Em]mone del\[D]la Cari\[C7+]tà,
il testi\[Em]mone di \[D]Me, Veri\[C]tà.
Vi amo co\[G]sì e vi ame\[D]rò come vi ho a\[Em]mati, \brk con \[D]voi reste\[C]rò
e via da \[G]qui ognuno sa\[D]rà \brk il testi\[Em]mone del\[D]la Cari\[C7+]tà,
il testi\[Em]mone di \[D]Me, Veri\[C]tà.
\endchorus
\endsong 


%titolo{Vieni al Signor}
%autore{Calisi}
%album{Benedici il Signore}
%tonalita{Re-}
%gruppo{}
%momenti{Salmi;Quaresima}
%identificatore{vieni_al_signor}
%data_revisione{2011_12_31}
%trascrittore{Francesco Endrici - Manuel Toniato}
\beginsong{Vieni al Signor}[by={Calisi}]

\ifchorded
\beginverse*
\vspace*{-0.8\versesep}
{\nolyrics \[Dm] \[D4] \[Dm] \[D4] }
\vspace*{-\versesep}
\endverse
\fi

\beginverse
\[Dm]Benedici \[C]il Signor \[Dm]anima \[C]mia 
\[B&7+]quanto è in \[Am7]me lo bene\[Dm]dica.\[D4] 
\[Dm]Non dimenti\[C]care \[Dm]i suoi bene\[C]fici, 
\[B&7+]quanto è in \[Am7]me lo bene\[Dm]dica.
\endverse

\beginchorus
\[F]Egli per\[Am7]dona \[B&]tut\[Gm]te le \[A]tue \[A7]colpe
\[Dm]buono e pie\[C]toso è il Si\[B&7+]gnore \[Am7]lento all'\[A7]ira.
\[Dm] Vieni \[Dm7]al Signor,\[B&7+] ri\[Am7]cevi il suo a\[Dm]mor.
\endchorus

\chordsoff
\beginverse
Salva dalla fossa la tua vita 
e ti incorona di grazia.
Come il cielo è alto sopra la terra 
così è la sua misericordia.
\endverse

\beginverse
Ma la grazia del Signor dura in eterno 
per quelli che lo temono.
Benedici il Signor, anima mia 
quanto è in me lo benedica.
\endverse
\endsong






%titolo{Vieni e seguimi}
%autore{Gen Rosso}
%album{Se siamo uniti}
%tonalita{La}
%gruppo{}
%momenti{Congedo}
%identificatore{vieni_e_seguimi}
%data_revisione{2011_12_31}
%trascrittore{Francesco Endrici}
\beginsong{Vieni e seguimi}[by={Gen\ Rosso}]
\beginverse
Lascia \[A]che il mondo \[Bm]vada per la sua \[A]strada.
Lascia \[C#m]che l'uomo ri\[F#m]torni alla sua \[E]casa.
Lascia \[D]che la gente accumuli la sua for\[A]tuna.
\endverse
\beginchorus
Ma \[E]tu, tu \[D]vieni e \[A]seguimi, \[E]tu, \[D]vieni e \[A]seguimi.
\endchorus
\beginverse
Lascia ^che la barca in ^mare spieghi la ^vela.
Lascia ^che trovi af^fetto chi segue il ^cuore.
Lascia ^che dall'albero cadano i frutti ma^turi.
\endverse
\beginchorus
Ma \[E]tu, tu \[D]vieni e \[A]seguimi, \[E]tu, \[D]vieni e \[F#m]seguimi.
\endchorus
\beginverse
E sa\[F#]rai luce per gli \[B]uomini
e sa\[F#]rai sale della \[C#m]terra \[E]
e nel mondo de\[F#]serto aprirai
una \[B]strada nuova. \rep{2}
E per \[F#]questa strada, \[G#m]va', \[F#]va',
e \[B]non voltarti indietro, \[F#]va'
e \[B]non voltarti indietro\[F#]{\dots}
\endverse
\endsong


%titolo{Vieni nasci ancora}
%autore{Ricci}
%album{Venne nel mondo}
%tonalita{Re}
%gruppo{}
%momenti{Natale}
%identificatore{vieni_nasci_ancora}
%data_revisione{2011_12_31}
%trascrittore{Francesco Endrici}
\beginsong{Vieni nasci ancora}[by={Ricci}]
\beginverse
\textnote{Donne}
Torni Si\[D]gnore, torni nel \[G]cuore
col tuo si\[Bm]lenzio denso di \[A]Te
e come i pa\[G]stori un tempo \brk ora noi \[Bm]ti adoriamo
e i nostri \[A]doni sono ciò che siamo \[F#]noi.
\endverse
\beginverse
%\chordsoff
\textnote{Uomini}
Eri la ^luce, venivi nel ^mondo,
venivi tra i ^tuoi e i tuoi però
loro non ^ti hanno accolto.
Ma noi ti invo^chiamo, vieni
ma noi ti vo^gliamo accanto
la nostra ^casa è tua t'accoglieremo ^noi.
\endverse
\beginverse
%\chordsoff
\textnote{Donne}
E tu che ri^torni, tu che ri^nasci
dove c'è a^more e carità \brk qui sei pre^sente.
Tu per dav^vero vieni,
Tu per dav^vero nasci
noi siamo u^niti nel tuo nome e tu sei ^qui.
\endverse
\beginchorus
\[D] Vieni nasci ancora \[G] dentro l'anima \[Bm]
vieni nasci sempre, \[A] nasci in mezzo a noi. \[G]
Per le strade luci, \[Bm] feste e musiche \[A]
ma Betlemme è \[F#]qui. \rep{2} \[(B)]
\endchorus
\beginverse
\transpose{2}
\textnote{Uomini}
Torni Si^gnore, torni nel ^cuore
col tuo si^lenzio denso di ^Te
e come i pa^stori un tempo ora noi ^ti adoriamo
e i nostri ^doni sono ciò che siamo ^noi.
\endverse
\beginverse
%\chordsoff
\transpose{2}
\textnote{Donne}
Eri la ^luce, venivi nel ^mondo,
venivi tra i ^tuoi e i tuoi però
loro non ^ti hanno accolto.
\textnote{Uomini}
Ma noi ti invo^chiamo, vieni
ma noi ti vo^gliamo accanto
la nostra ^casa è tua t'accoglieremo ^noi.
\endverse
\beginverse
\transpose{2}
%\chordsoff
\textnote{Donne}
E tu che ri^torni, tu che ri^nasci
dove c'è a^more e carità \brk qui sei pre^sente.
\textnote{Uomini}
Tu per dav^vero vieni, \brk Tu per dav^vero nasci
noi siamo u^niti nel tuo nome e tu sei ^qui.
\endverse
\textnote{[Sovrapposto a queste ultime 4 strofe gli uomini cantano:]}
\beginchorus
\transpose{2}
\[D] Vieni nasci ancora \[G] dentro l'anima \[Bm]
vieni nasci sempre, \[A] nasci in mezzo a noi. \[G]
Per le strade luci, \[Bm] feste e musiche \[A]
ma Betlemme è \[F#]qui. \rep{2} 
\[A] Ma Betlemme è \[D]qui. \[D4]\[D]
\endchorus
\endsong



%titolo{Vieni qui tra noi}
%autore{Gen Verde, Gen Rosso}
%album{Come fuoco vivo}
%tonalita{Sol}
%gruppo{}
%momenti{Pentecoste}
%identificatore{vieni_qui_tra_noi}
%data_revisione{2011_12_31}
%trascrittore{Francesco Endrici}
\beginsong{Vieni qui tra noi}[by={Gen\ Verde, Gen\ Rosso}]
\beginverse*
\[C]Vie\[G]ni \[D]qui tra \[G]noi
come \[C]fiamma che \[Em]scende dal \[D4]cielo.
\[C]Vie\[G]ni \[D]qui tra \[Em7]noi,
\[Am7]rin\[G]nova il \[C]cuore del \[D4]mondo.
\[C]Vie\[G]ni \[D]qui tra \[G]noi,
col tuo a\[C]more ri\[Em]schiara la \[D4]terra.
\[C]Vie\[G]ni \[D]qui \[Em7]noi,
\[Am7]soffio di \[D]liber\[G7]tà. \[C7 G]
\endverse
\beginverse*
Nel si\[G]lenzio tu \[C]sei pa\[G]ce,
nella notte \[C]lu\[G]ce,
Dio nascosto, \[C]vi\[G]ta,
Dio tu sei, A\[C]mo\[G]re.
Tutto si ri\[C]crea in \[G]te,
tutto \[D]vive in \[Em7]te.
Scalda col tuo \[C]fuo\[G]co \brk terra e \[D]cie\[Em7]lo.
Tu, che sai rac\[C]coglie\[G]re \brk ogni \[D]ge\[B]mi\[Em]to, 
semina nel \[C]nostro \[G]cuore \brk una spe\[Am7]ranza d'eterni\[C]tà.
\endverse
\beginverse*
\chordsoff
Vieni qui tra noi
come fiamma che scende dal cielo.
Vieni qui tra noi,
rinnova il cuore del mondo.
\endverse
\beginverse*
\[C]Vie\[G]ni \[D]qui tra \[G]noi,
col tuo a\[C]more ri\[Em]schiara la \[D4]terra.
\[C]Vie\[G]ni \[D]qui \[Em7]noi,
\[C]soffio di \[Cm]liber\[A7]tà
amore, \[C]Dio in mezzo a \[G]noi!
\endverse
\endsong

%titolo{Vieni santo spirito}
%autore{Arguello}
%album{}
%tonalita{Do}
%gruppo{}
%momenti{Pentecoste}
%identificatore{vieni_santo_spirito_arguello}
%data_revisione{2011_12_31}
%trascrittore{Francesco Endrici}
\beginsong{Vieni santo spirito}[by={Arguello}]
\beginverse
\[C]Vieni \[G]Santo \[Am]Spirito, \[F]manda a \[C]noi dal \[G]cielo,
un \[C]rag\[F]gio di \[C]luce, un \[G]raggio \[F]di \[C]luce.
\[C]Vieni \[G]Padre dei \[Am]poveri, \[F]vieni da\[C]tore dei \[G]doni,
\[C]lu\[F]ce dei \[C]cuori, \[G]luce \[F] dei cuo\[C]ri.
\[Am]Consola\[G]tore per\[Am]fetto, \[F]ospite \[G]dolce dell'\[C]ani\[E]ma,
\[Am]dolcissimo sol\[G]lievo, dol\[Am]cissimo sol\[E]lievo.
\[Am]Nella fa\[G]tica ri\[Am]poso, \[F]nel ca\[G]lore ri\[C]pa\[E]ro,
\[Am]nel pianto con\[G]forto, nel \[Am]pianto con\[E]forto.
\endverse
\beginverse
\chordsoff
Luce beatissima, invadi i nostri cuori,
senza la tua forza nulla, nulla è nell'uomo.
Lava ciò che è sordido, scalda ciò che è gelido,
rialza chi è caduto, rialza chi è caduto.
Dona ai tuoi fedeli, che in te confidano,
i sette santi doni, i sette santi doni.
Dona virtù e premio, dona morte santa,
dona eterna gioia, dona eterna gioia.
\endverse
\endsong

%titolo{Vieni, Santo Spirito di Dio}
%autore{Scarpa}
%album{Vieni soffio di Dio}
%tonalita{Sol}
%gruppo{}
%momenti{Pentecoste}
%identificatore{vieni_santo_spirito_di_dio_scarpa}
%data_revisione{2011_12_31}
%trascrittore{Francesco Endrici}
\beginsong{Vieni, Santo Spirito di Dio}[by={Scarpa}]
\beginchorus
\[G]Vieni, Santo \[Am]Spirito di \[Bm7]Dio, \[\vline]\[C]\[D]\[\vline] 
\[G]come vento \[Am7]soffia sulla \[B4]Chie\[7]sa! 
\[C]Vieni come \[Bm7]fuoco, \[Bm7]ardi in \[Em]noi 
\[C]e con te sa\[G]remo \[Am]veri testi\[Am7]moni di Ge\[G]sù. \[C]\[G]\[Am] 
\endchorus
\beginverse
Sei \[G]vento, spazza il \[Em7]cielo dalle \[Am]nubi del ti\[C2]more.
sei \[G]fuoco, sciogli il \[Em7]gelo e ac\[Am]cendi il nostro ar\[D]dore.
\[Bm]Spirito crea\[Em]tore, \[\vline]\[C]scendi \[A7]\[\vline]su di \[D4]noi! \[D]
\endverse
\beginverse
\chordsoff
Tu bruci tutti i semi di morte e di peccato; 
tu scuoti le certezze che ingannano la vita. 
Fonte di sapienza, scendi su di noi!
\endverse
\beginverse
\chordsoff
Tu sei coraggio e forza nelle lotte della vita; 
tu sei l'amore vero, sostegno nella prova. 
Spirito d'amore, scendi su di noi!
\endverse
\beginverse
\chordsoff
Tu, fonte di unità, rinnova la tua Chiesa, 
illumina le menti, dai pace al nostro mondo. 
O Consolatore, scendi su di noi!
\endverse
\endsong

%titolo{Vieni Santo Spirito}
%autore{Fioravanti}
%album{Spirito Santo vieni}
%tonalita{Mi}
%gruppo{}
%momenti{Pentecoste}
%identificatore{vieni_santo_spirito_fioravanti}
%data_revisione{2011_12_31}
%trascrittore{Francesco Endrici - Manuel Toniato}
\beginsong{Vieni Santo Spirito}[by={Fioravanti}]

\ifchorded
\beginverse*
\vspace*{-0.8\versesep}
{\nolyrics \[E]}
\vspace*{-\versesep}
\endverse
\fi

\beginchorus
\[E]Vieni Santo Spirito, vieni in noi. 
\[E]Vieni Santo Spirito, vieni in noi.
\endchorus
\beginverse
\[E]Padre dei \[A7]poveri, \[E]ricco di \[A7]doni, 
\[C#m]vieni \[F#m]Spiri\[B4/7]to d'a\[E]mor.
\endverse
\beginchorus
\[E]Vieni Santo Spirito, \[B6]vie\[B7]ni in \[E]noi. 
\[E]Vieni Santo Spirito, \[B6]vie\[B7]ni in \[E]noi.
\endchorus
\beginverse
\[E]Piega ciò che è \[A7/9]arido, \[E]scalda ciò che è \[A7/9]gelido
\[C#m]vieni \[F#m]Spiri\[B4/7]to d'a\[E]mor.
\endverse
\beginchorus
\[E]Vieni Santo Spirito, vieni in noi. 
\[E]Vieni Santo Spirito, vieni in noi.
\endchorus
\beginverse
\[F]Lava ciò che è \[B&7]sordido, \[F]bagna ciò che è \[B&7]arido
\[Dm7]Vieni \[Gm]Spiri\[C4/7]to d'a\[F]mor
\endverse
\beginchorus
\[F]Vieni Santo Spirito, \[C6]vieni in \[F]noi.
\[F]Vieni Santo Spirito, \[C6]vie\[C7]ni in \[F]noi.
\endchorus
\beginverse
\[F]Dona virtù e \[B&7]premio, \[F]dona sapienza e \[B&7]gioia
\[Dm7]Vieni \[Gm]Spiri\[C4/7]to d'a\[F]mor
\endverse
\endsong

%titolo{Vieni Santo Spirito}
%autore{Piatti}
%album{}
%tonalita{Re}
%gruppo{}
%momenti{Pentecoste}
%identificatore{vieni_santo_spirito_piatti}
%data_revisione{2011_12_31}
%trascrittore{Francesco Endrici - Manuel Toniato}
\beginsong{Vieni Santo Spirito}[by={Piatti}]

\beginchorus
\[D]Vieni, \[Bm]Santo \[Em]Spiri\[A7]to!
\[D]Vieni, \[F#m]Santo \[Em]Spiri\[A]to,
\[D]riempi i \[G]cuori \[A]dei tuoi fe\[F#m]deli
ac\[Bm]cendi il \[Em]fuoco \[D]del tuo a\[A7]mor. \rep{2} \[D]del \[A7]tuo a\[D]mor
\endchorus

\beginverse
\[D]Ovunque sei pre\[Bm]sente, \[Em]Spirito di \[A7]Dio,
in \[D]tutto ciò che \[F#m]vive in\[B7]fondi \[Em]la tua \[A]forza;
Tu \[D7]sei parola \[G]ve\[Gm]ra, \[D]fonte \[Em]di spe\[A]ran\[D]za
e \[G]guida al nostro \[D]cuo\[A7]re.
\endverse

\beginverse
\chordsoff
Tu vivi in ogni uomo, Spirito di Dio,
in chi di giorno in giorno
lotta per il pane,
in chi senza paura cerca la giustizia
e vive nella pace.
\endverse
\endsong



%titolo{Vieni Spirito di Cristo}
%autore{Amadei}
%album{La nostra festa è Cristo}
%tonalita{Mi-}
%gruppo{}
%momenti{Pentecoste}
%identificatore{vieni_spirito_di_cristo}
%data_revisione{2011_12_31}
%trascrittore{Francesco Endrici - Manuel Toniato}
\beginsong{Vieni Spirito di Cristo}[by={Amadei}]

\beginchorus
\[Em]Vieni, vieni \[Am]Spirito d'amore,
ad i\[Em]nsegnar le cose di \[Bm]Dio, \[Bm7]
\[Em]vieni, vieni, \[Am]Spirito di pace
a \[Em]suggerire le cose che \[Bm]Lui \[Bm7] ha detto a \[Em]noi.
\endchorus

\beginverse
\[Em]Noi ti invochiamo \[Am]Spirito di Cristo,
\[Em]vieni Tu dentro di \[Bm]noi.\[Bm7]
\[Em]Cambia i nostri occhi, \[Am]fa che noi vediamo
\[Em]la bontà di Dio per \[Bm]noi.\[Bm7]
\endverse

\beginverse
\chordsoff
Vieni o Spirito dai quattro venti
e soffia su chi non ha vita.
Vieni o Spirito, soffia su di noi
perché anche noi riviviamo.
\endverse

\beginverse
\chordsoff
Insegnaci a sperare, insegnaci ad amare,
insegnaci a lodare Iddio.
Insegnaci a pregare, insegnaci la via,
insegnaci Tu l'unità.
\endverse
\endsong

%titolo{Vieni spirito di Dio}
%autore{}
%album{}
%tonalita{Re}
%gruppo{}
%momenti{Pentecoste}
%identificatore{vieni_spirito_di_dio}
%data_revisione{2011_12_31}
%trascrittore{Francesco Endrici}
\beginsong{Vieni spirito di Dio}
\ifchorded
\beginverse*
\vspace*{-0.8\versesep}
{\nolyrics \[D]\[G]\[D]\[A]}
\vspace*{-\versesep}
\endverse
\fi
\beginverse
\memorize
Vieni \[D]Spirito di Dio \[G]scendi su di \[D]noi \[G]
vieni \[D]Spirito di Dio \[G]resta dentro di \[A]noi
rimani \[D]qui per conso\[G]lar
rimani \[D]qui per libe\[G]rar
rimani \[D]a santificare \[A]Spirito di Dio sta \[G]qui. \[D]
\endverse
\beginchorus
Scendi su \[G]noi, scendi su \[D]noi
tocca le \[G]menti e i nostri \[A]cuor.
Spirito di \[F#m]pace e d'a\[Bm]mor
scendi su \[G]noi Santo \[A]Spirito scendi su \[D]noi.
\endchorus
\beginverse
Vieni ^Spirito di Dio ^forza di liber^tà ^
vieni ^Spirito di Dio ^luce di veri^tà
rimani ^qui per conso^lar
rimani ^qui per libe^rar
rimani ^a santificare ^Spirito di Dio sta ^qui. ^
\endverse
\beginchorus
Scendi su \[G]noi, scendi su \[D]noi
tocca le \[G]menti e i nostri \[A]cuor.
Spirito di \[F#m]pace e d'a\[Bm]mor
scendi su \[G]noi Santo \[A]Spirito scendi su \[D]noi. \rep{2}
\endchorus
\ifchorded
\beginverse*
\vspace*{-\versesep}
{\nolyrics \[G]\[A]\[G]\[D]}
\endverse
\fi
\endsong

%titolo{Vivere la vita}
%autore{Gen Verde}
%album{È bello lodarti}
%tonalita{Do}
%gruppo{}
%momenti{Ingresso}
%identificatore{vivere_la_vita}
%data_revisione{2011_12_31}
%trascrittore{Francesco Endrici}
\beginsong{Vivere la vita}[by={Gen\ Verde}]
\beginverse
\[C]Vivere la \[G]vita con le \[Dm]gioie 
e coi do\[F]lori di ogni \[Am]giorno, \[G]
è quello che Dio \[C]vuole da te. \[G]
\[C]Vivere la \[G]vita e inabis\[Dm]sarsi 
nell'a\[F]more è il tuo de\[Am]stino, \[G]
è quello che Dio \[C]vuole da \[G]te.
\[F]Fare insieme agli \[G]altri la tua \[C]strada verso \[Em]Lui,
\[F]correre con \[G]i fratelli \[C]tuo\[Em]i.
\[F]Scoprirai al\[G]lora il \[G7]cielo \[C]dentro di \[Em]te,
\[F]una scia di \[Dm]luce lasce\[G]rai.
\endverse
\beginverse
%\chordsoff
^Vivere la ^vita è l'avven^tura 
più stu^penda dell'a^more, ^ 
è quello che Dio ^vuole da te. ^
^Vivere la ^vita e gene^rare 
ogni mo^mento il para^diso, ^ 
è quello che Dio ^vuole da ^te.
^Vivere per^ché ritorni al ^mondo l'uni^tà,
^perché Dio sta ^nei fratelli ^tuo^i.
^Scoprirai al^lora il ^cielo ^dentro di ^te,
^una scia di ^luce lasce^rai,
\[F]una scia di \[Dm]luce lasce\[C]rai.
\endverse
\endsong

%titolo{Vocazione}
%autore{Sequeri}
%album{In cerca d'autore}
%tonalita{Do}
%gruppo{}
%momenti{Ingresso}
%identificatore{vocazione}
%data_revisione{2011_12_31}
%trascrittore{Francesco Endrici}
\beginsong{Vocazione}[by={Sequeri}]
\beginverse
\[C]Era un giorno \[G]come tanti \[F]altri,
e quel \[G]giorno Lui pas\[C]sò. \[F]\[C]\[G]
\[C]Era un uomo \[G]come tutti gli \[F]altri,
e pas\[G]sando mi chia\[C]mò \[F]\[C]\[E]
\[Am]come lo sa\[Em]pesse che il mio \[F]nome \brk era \[G]proprio quello
\[C]come mai ve\[G]desse proprio \[F]me
nella sua \[G]vita, non lo \[C]so. \[F]\[C]\[G]
\[C]Era un giorno \[G]come tanti \[F]altri
e quel \[G]giorno mi chia\[C]mò. \[F]\[C]\[E]
\endverse
\beginchorus
\[Am]Tu \[Em]Dio, \[F]che conosci il \[G]nome mio
\[Am]fa' \[Em]che \[F]ascoltando \[G]la tua voce
\[C]io ri\[G]cordi dove \[F]porta la mia \[G]strada
\[C]nella \[G]vita, all'in\[F]contro con \[C]Te. \[F]\[C]\[G]
\endchorus
\beginverse
^Era l'alba ^triste e senza ^vita,
e ^qualcuno mi chia^mò ^^^
^era un uomo ^come tanti ^altri,
ma la ^voce, quella ^no. ^^^
^Quante volte un ^uomo
con il ^nome giusto ^mi ha chiamato
^una volta ^sola l'ho sen^tito
pronun^ciare con a^more. ^^^
^Era un uomo ^come nessun ^altro
e quel ^giorno mi chia^mò. ^^^
\endverse
\endsong

%titolo{Voglio cantare}
%autore{Milan}
%album{}
%tonalita{Do}
%gruppo{}
%momenti{Ingresso;Congedo;Salmi}
%identificatore{voglio_cantare}
%data_revisione{2011_12_31}
%trascrittore{Francesco Endrici - Manuel Toniato}
\beginsong{Voglio cantare}[by={Milan}]


\beginchorus
\[C]Voglio cantare al \[G]mio Signore, 
\[Am]finché esisto, fin\[F]ché ho vita
e il mio \[C]canto sia gra\[G]dito a lui 
\[F] che è la mia \[C]gioia. \rep{2}
\endchorus

\beginverse
\[Am]Benedici il Signore anima \[Dm]mia: \[D4] \[Dm] 
\[G]Signore mio Dio quanto sei \[C]grande!
\[Am]Rivestito di maestà e di splen\[Dm]dore, \[D4] \[Dm] 
av\[G]volto di luce come di un \[C]manto,
tu \[F]stendi il cielo come una tenda,
costru\[G]isci sull'acqua la tua dimora,
\[F]fai delle nubi il tuo \[C]carro 
cam\[F]mini sulle ali del \[C]vento;
\[F]fai eseguire i \[G]tuoi comandi
\[F]al soffio del vento, alla \[G]luce dei lampi\[G7].
\endverse

\beginverse
\chordsoff
Hai fondato la terra sulle sue basi
ed essa non potrà più vacillare.
Come un manto l'avvolgeva il mare,
i monti dalle acque erano invasi;
al tuo comando sono fuggite,
al fragore del tuono hanno tremato:
d'incanto i mari sono emersi,
si sono aperte le valli,
hai arginato le acque:
non passeranno,
a coprire la terra non torneranno.
\endverse
\endsong

%titolo{Voglio cantare al Signore}
%autore{Villani}
%album{Gerico, le tue mura crolleranno}
%tonalita{Fa}
%gruppo{}
%momenti{Salmi;Ingresso;Congedo}
%identificatore{voglio_cantare_al_signore}
%data_revisione{2011_12_31}
%trascrittore{Francesco Endrici - Manuel Toniato}
\beginsong{Voglio cantare al Signore}[by={Villani}]


\beginchorus
Voglio can\[F]tare al Si\[C]gnor e \[B&]dare gloria a \[C]Lui,
voglio can\[F]tare per \[C]sempre al Si\[B&]gnor! \rep{2}
\endchorus

\beginverse
Ca\[C]vallo e cavaliere ha get\[F]tato nel mare.
\[C]Egli è il mio Dio e lo \[F]voglio esaltare.
\[B&]Chi è come Te lassù nei \[C]cieli o Signor?
\[B&]Chi è come Te lassù mae\[C]stoso in santità? 
\endverse

\beginverse
\chordsoff
La destra del Signore ha annientato il nemico
le sue schiere ha rigettato in fondo al mare.
Sull'asciutto tutto il suo popolo passò
con timpani e con danze al Signore s'inneggiò.
\endverse

\beginverse
\chordsoff
Con la tua potenza Israele hai salvato,
per la tua promessa una terra gli hai dato.
Per i suoi prodigi al Signore canterò,
con un canto nuovo il suo nome esalterò!
\endverse
\endsong



%titolo{Voi siete di Dio}
%autore{Balduzzi, Casucci}
%album{Verbum panis}
%tonalita{Sol}
%gruppo{}
%momenti{}
%identificatore{voi_siete_di_dio}
%data_revisione{2011_12_31}
%trascrittore{Francesco Endrici}
\beginsong{Voi siete di Dio}[by={Balduzzi, Casucci}]
\ifchorded
\beginverse*
\vspace*{-0.8\versesep}
{\nolyrics \[G]\[G]\[D]\[C]\[Em7]\[Em7]\[C]\[D]\[G]}
\vspace*{-\versesep}
\endverse
\fi
\beginverse
\memorize
Tutte le \[G]stelle della \[D]not\[G]te \[(C)]\[G]
le nebulose le co\[D]me\[Em]te \[G]
il sole su una ragna\[D]te\[G]la \[C]
è tutto vostro e voi \[G]sie\[D]te di \[G]Dio. \[E]
\endverse
\beginverse
\transpose{-3}
Tutte le ^rose della ^vi^ta, ^^
il grano i prati i fili d'^er^ba ^
il mare i fiumi le mon^ta^gne ^
è tutto vostro e voi ^sie^te di ^Dio. \[B&]
\endverse
\beginverse
Tutte le ^musiche e le ^dan^ze ^^
i grattacieli le astro^na^vi ^
i quadri i libri le cul^tu^re ^
è tutto vostro e voi ^sie^te di ^Dio.
\endverse
\ifchorded
\beginverse*
\vspace*{-\versesep}
{\nolyrics ^^^^^
^^^
^^^
^^^^}
\endverse
\fi
\beginverse
\transpose{-3}
Tutte le ^volte che per^do^no ^^
quando sorrido e quando ^pian^go ^
quando mi accorgo di chi ^so^no ^
è tutto vostro e voi \[Em]sie\[D]te di \[C]Dio
è tutto nostro e noi \[G]sia\[D]mo di \[G]Dio.
\endverse
\endsong

%titolo{Vorrei amarti}
%autore{}
%album{}
%tonalita{Re}
%gruppo{}
%momenti{}
%identificatore{vorrei_amarti}
%data_revisione{2011_12_31}
%trascrittore{Francesco Endrici}
\beginsong{Vorrei amarti}
\beginverse
\[D]Io vorrei sa\[Bm]perti amare \[G]come Dio, \[A]
che ti prende per \[D]mano ma ti \[Bm]lascia anche an\[G]dare.
Vor\[A]rei saperti a\[D]mare senza \[Bm]farti mai do\[G]mande,
fe\[A]lice perché e\[D]sisti e co\[Bm]sì io posso \[G]darti \brk il \[A]meglio di me.
\endverse
\beginchorus
\[D]Con la \[A]forza del \[Bm]mare, l'e\[G]ternità dei \[D]giorni,
la \[A]gioia dei \[Bm]voli, la \[G]pace della \[D]sera,
l'im\[F#]mensità del \[Bm]cielo, \[G]come ti ama \[D]Dio.
\endchorus
\beginverse
\chordsoff
Io vorrei saperti amare come Dio
che ti conosce e ti accetta come sei
tenerti tra le mani come i voli nell'azzurro,
felice perché esisti e così io posso darti \brk il meglio di me.
\endverse
\beginverse
\chordsoff
Io vorrei saperti amare come Dio
che ti fa migliore con l'amore che ti dona;
seguirti tra la gente con la gioia che hai dentro
felice perché esisti e così io posso darti \brk il meglio di me.
\endverse
\beginchorus
\[D]Con la \[A]forza del \[Bm]mare, l'e\[G]ternità dei \[D]giorni,
la \[A]gioia dei \[Bm]voli, la \[G]pace della \[D]sera,
l'im\[F#]mensità del \[Bm]cielo, \[G]come ti ama \[D]Dio,\brk \[Bm]\[G] \[A]come ti ama \[D]Dio.
\endchorus
\endsong

%titolo{Vorrei essere}
%autore{Bizzeti}
%album{Io sono un cantastorie}
%tonalita{Sol}
%gruppo{}
%momenti{}
%identificatore{vorrei_essere}
%data_revisione{2011_12_31}
%trascrittore{Francesco Endrici}
\beginsong{Vorrei essere}[by={Bizzeti}]
\ifchorded
\beginverse*
\vspace*{-0.8\versesep}
{\nolyrics \[G]\[Bm]\[C]\[D7]}
\vspace*{-\versesep}
\endverse
\fi
\beginverse
\memorize
\[G] Vorrei essere un \[Bm]chicco di grano,
\[C] affidarmi alla \[D]terra accogliente
\[G] Esser pane del \[Bm]genere umano,
\[C] esser cibo di \[D]tutta la gente.
\[G] Vorrei essere un \[D]tralcio di vite,
\[Em] dare grappoli \[A]dolci e ma\[D]turi
\[G] Rallegrare le \[Bm]mense imbandite,
\[C] render gli animi \[D]forti e si\[Em]curi.
\endverse
\beginverse
^ Vorrei essere un ^raggio di sole,
^ asciugare i bu^cati distesi
^ Risvegliare le ^rose e le viole,
^ riscaldare cam^pagne e paesi.
^ Vorrei essere un ^ramo d’ulivo,
^ verde simbolo ^di fratel^lanza
^ Dare pace a chi an^cora ne è privo
^ e portare nel ^mondo spe^ranza.
\endverse
\endsong


