
%....................................................................................
%
% ██╗  ██╗ ██████╗ ██████╗ ███████╗██████╗ ███╗   ██╗██╗██╗  ██╗██╗   ██╗███████╗
% ██║ ██╔╝██╔═══██╗██╔══██╗██╔════╝██╔══██╗████╗  ██║██║██║ ██╔╝██║   ██║██╔════╝
% █████╔╝ ██║   ██║██████╔╝█████╗  ██████╔╝██╔██╗ ██║██║█████╔╝ ██║   ██║███████╗
% ██╔═██╗ ██║   ██║██╔═══╝ ██╔══╝  ██╔══██╗██║╚██╗██║██║██╔═██╗ ██║   ██║╚════██║
% ██║  ██╗╚██████╔╝██║     ███████╗██║  ██║██║ ╚████║██║██║  ██╗╚██████╔╝███████║
% ╚═╝  ╚═╝ ╚═════╝ ╚═╝     ╚══════╝╚═╝  ╚═╝╚═╝  ╚═══╝╚═╝╚═╝  ╚═╝ ╚═════╝ ╚══════╝
% Font: Ansi Shadow
%....................................................................................



%% CHORDED Settings
% > A4vert + palatino + titleindex + tematicindex + chorded

\documentclass[a4vert, libertine,titleindex,  tematicindex, chorded, cover]{canzoniereonline}

%opzioni formato: singoli, standard (A4), a5vert, a5oriz, a6vert;
%opzioni accordi: lyric, chorded {quelli d Songs}
%opzioni font: palatino, libertine
%opzioni segno minore: "minorsign=quel che vuoi"
%opzioni indici: authorsindex, titleindex, tematicindex
%opzioi copertina: cover e nocover



% SETTINGS
%-------------------------------------------------------------------------------
% !TEX encoding = UTF-8 Unicode


% STILE DOCUMENTO
%������������������������������������������������������������������������


\def\canzsongcolumsnumber{1} %# coolonne lungo cui disporre le canzoni




% PACCHETTI DA IMPORTARE
%������������������������������������������������������������������������
%\usepackage[T1]{fontenc}
%\usepackage[utf8]{inputenc}
%\usepackage[italian]{babel}
\usepackage{pdfpages}
\usepackage{hyperref}
\usepackage{wasysym}






% NUOVI COMANDI E VARIABILI GLOBALI
%������������������������������������������������������������������������

%Coomando per la suddivisione in capitoli
\renewcommand{\songchapter}{\chapter*}


%Counter globale per tenere traccia di una numerazione progressiva
%Si affianca a un altro counter gi� utilizzato nella classe CanzoniereOnLine "songnum" 
%che, tuttavia, si riazzera ognivolta viene definito un nuovo ambiente \beginsongs{}
\newcounter{GlobalSongCounter} 



%New command to make colophon appear on even (closing) page
\newcommand*\cleartoleftpage{%
  \clearpage
  \ifodd\value{page}\hbox{}\newpage\fi
}










% FRONTESPIZIO
%-------------------------------------------------------------------------------


% RIGHE PER LA COPERTINA
%������������������������������������������������������������������������


%titoletto
\renewcommand{\titolettocop}{Gino} 	


%TITOLO					
\renewcommand{\titolocop}{Trollolobrigido}  			


%Sottotitolo
\renewcommand{\sottotitolocop}{Anchesi detto Guzzo} 	


%fondo pagina
\renewcommand{\piede}{\today}							







%Starting Document
\begin{document}







 % COLOPHON
%-------------------------------------------------------------------------------
\ifcover
	\firstpage
	\colophon
\else
	\relax
\fi




% CAPITOLI DEL CANZONIERE
%-------------------------------------------------------------------------------

%Ciascun capitolo contiene già tutta la logica di gestione della 
%numerazione progressiva, del DB locale da cui attingere le canzoni
%e la creazione/chiusura dell'ambiente in cui vengono importate 
%tutte le canzoni relative
\addtocounter{GlobalSongCounter}{1} %set starting song counter to 1 (0 otherwise)


\songchapter{Alleluia}
%...............................................................................
%
%   █████╗ ██╗     ██╗     ███████╗██╗     ██╗   ██╗██╗ █████╗     
%  ██╔══██╗██║     ██║     ██╔════╝██║     ██║   ██║██║██╔══██╗    
%  ███████║██║     ██║     █████╗  ██║     ██║   ██║██║███████║    
%  ██╔══██║██║     ██║     ██╔══╝  ██║     ██║   ██║██║██╔══██║    
%  ██║  ██║███████╗███████╗███████╗███████╗╚██████╔╝██║██║  ██║    
%  ╚═╝  ╚═╝╚══════╝╚══════╝╚══════╝╚══════╝ ╚═════╝ ╚═╝╚═╝  ╚═╝
% Font: ANSI Shadow
%...............................................................................
\begin{songs}{}
\songcolumns{\canzsongcolumsnumber}
\setcounter{songnum}{\theGlobalSongCounter} %set songnum counter, otherwise would be reset

%set the default path inside current folder
\makeatletter
\def\input@path{{Songs/Alleluia/}}
\makeatother


%***** INSERT SONGS HERE ******

%-------------------------------------------------------------
%			INIZIO	CANZONE
%-------------------------------------------------------------


%titolo: 	Alleluia a te canto
%autore: 	Giombini
%tonalita: 	Do 



%%%%%% TITOLO E IMPOSTAZONI
\beginsong{Alleluia a Te canto}[by={Giombini}] 	% <<< MODIFICA TITOLO E AUTORE
\transpose{-3} 						% <<< TRASPOSIZIONE #TONI (0 nullo)
\momenti{Acclamazione al Vangelo;}							% <<< INSERISCI MOMENTI	
% momenti vanno separati da ; e vanno scelti tra:
% Ingresso; Atto penitenziale; Acclamazione al Vangelo; Dopo il Vangelo; Offertorio; Comunione; Ringraziamento; Fine; Santi; Pasqua; Avvento; Natale; Quaresima; Canti Mariani; Battesimo; Prima Comunione; Cresima; Matrimonio; Meditazione;
\ifchorded
	%\textnote{Tonalità originale }	% <<< EV COMMENTI (tonalità originale/migliore)
\fi


%%%%%% INTRODUZIONE
\ifchorded
\vspace*{\versesep}
\textnote{Intro: \qquad \qquad  }%(\eighthnote 116) % << MODIFICA IL TEMPO
% Metronomo: \eighthnote (ottavo) \quarternote (quarto) \halfnote (due quarti)
\vspace*{-\versesep}
\beginverse*

\nolyrics

%---- Prima riga -----------------------------
\vspace*{-\versesep}
\[C] \[G] \[C*]	 % \[*D] per indicare le pennate, \rep{2} le ripetizioni

%---- Ogni riga successiva -------------------
%\vspace*{-\versesep}
%\[G] \[C]  \[D]	

%---- Ev Indicazioni -------------------------			
\textnote{\textit{(Con stop e attacco solo voce)} }	

\endverse
\fi




%%%%% STROFA
\beginverse		%Oppure \beginverse* se non si vuole il numero di fianco
\memorize 		% <<< DECOMMENTA se si vuole utilizzarne la funzione
%\chordsoff		& <<< DECOMMENTA se vuoi una strofa senza accordi

\[(C)] A te canto alle\[G]luia! 
\[A-] A te dono la mia \[E-]gioia,
\[F] a te grido mio Si\[C]gnore,
\[D7] a te offro ogni do\[G]lo\[7]re!

\endverse




%%%%% RITORNELLO
\beginchorus
\textnote{\textbf{Rit.}}

\[C]Al-\[G]le \[A-]lu ia! \[F]ah! \[C] 
Alle\[E-]luia! \[F]Alle\[G]luia!
\[C]Al-\[G]le \[A-]lu ia! \[F]ah! \[C] 
Alle\[E-]luia! Al\[F]le\[G]lu\[C]ia!  \[C] \[C*]

\endchorus



%%%%% STROFA
\beginverse		%Oppure \beginverse* se non si vuole il numero di fianco
%\memorize 		% <<< DECOMMENTA se si vuole utilizzarne la funzione
%\chordsoff		% <<< DECOMMENTA se vuoi una strofa senza accordi

^ A te dico io ti ^amo,
^ a te dedico la ^vita
^ a te chiedo dammi ^pace,
^ a te grido la mia ^fe^de!


\endverse




\endsong
%------------------------------------------------------------
%			FINE CANZONE
%------------------------------------------------------------



%-------------------------------------------------------------
%			INIZIO	CANZONE
%-------------------------------------------------------------


%titolo: 	Alleluia Canto per Cristo
%autore: 	Costa
%tonalita: 	Re 



%%%%%% TITOLO E IMPOSTAZONI
\beginsong{Alleluia Canto per Cristo}[by={E. Costa}] 	% <<< MODIFICA TITOLO E AUTORE
\transpose{0} 						% <<< TRASPOSIZIONE #TONI (0 nullo)
\momenti{Acclamazione al Vangelo}							% <<< INSERISCI MOMENTI	
% momenti vanno separati da ; e vanno scelti tra:
% Ingresso; Atto penitenziale; Acclamazione al Vangelo;  Dopo il Vangelo; Offertorio; Comunione; Ringraziamento; Fine; Santi; Pasqua; Avvento; Natale; Quaresima; Canti Mariani; Battesimo; Prima Comunione; Cresima; Matrimonio; Meditazione; Spezzare del pane;
\ifchorded
	%\textnote{Tonalità migliore }	% <<< EV COMMENTI (tonalità originale/migliore)
\fi


%%%%%% INTRODUZIONE
\ifchorded
\vspace*{\versesep}
\musicnote{
\begin{minipage}{0.48\textwidth}
\textbf{Intro}
\hfill 
%( \eighthnote \, 80)   % <<  MODIFICA IL TEMPO
% Metronomo: \eighthnote (ottavo) \quarternote (quarto) \halfnote (due quarti)
\end{minipage}
} 	
\vspace*{-\versesep}
\beginverse*

\nolyrics

%---- Prima riga -----------------------------
\vspace*{-\versesep}
\[D] \[F#-] \[G] \[D]\[D]	 % \[*D] per indicare le pennate, \rep{2} le ripetizioni

%---- Ogni riga successiva -------------------
%\vspace*{-\versesep}
%\[G] \[C]  \[D]	

%---- Ev Indicazioni -------------------------			
%\textnote{\textit{(Oppure tutta la strofa)} }	

\endverse
\fi



%%%%% RITORNELLO
\beginchorus
\textnote{\textbf{Rit.}}
\[D]Allelu\[F#-]ia, alle\[G]luia, allelu\[D]ia,
\[G]allelu\[D]ia, alle\[A]lu\[A]ia!
\[D]Allelu\[F#-]ia, alle\[G]luia, allelu\[D]ia,
\[G]allelu\[D]ia, alle\[A]lu\[D]ia!
\endchorus


%%%%% STROFA
\beginverse		%Oppure \beginverse* se non si vuole il numero di fianco
\memorize 		% <<< DECOMMENTA se si vuole utilizzarne la funzione
%\chordsoff		% <<< DECOMMENTA se vuoi una strofa senza accordi
\[D]Canto per \[F#-]Cristo che \[G]mi libere\[D]rà
\[G]quando ver\[D]rà nella \[A]glo\[A]ria,
\[D]quando la \[F#-]vita con \[G]Lui rinasce\[D]rà,
\[G]allelu\[D]ia, alle\[A]lu\[D]ia!
\endverse

%%%%% STROFA
\beginverse		%Oppure \beginverse* se non si vuole il numero di fianco
%\memorize 		% <<< DECOMMENTA se si vuole utilizzarne la funzione
%\chordsoff		% <<< DECOMMENTA se vuoi una strofa senza accordi
^Canto per ^Cristo: in ^Lui rifiori^rà
^ogni spe^ranza per^du^ta,
^ogni crea^tura con ^Lui risorge^rà,
^allelu^ia, alle^lu^ia!
\endverse

%%%%% STROFA
\beginverse		%Oppure \beginverse* se non si vuole il numero di fianco
%\memorize 		% <<< DECOMMENTA se si vuole utilizzarne la funzione
%\chordsoff		% <<< DECOMMENTA se vuoi una strofa senza accordi
^Canto per ^Cristo: un ^giorno torne^rà!
^Festa per ^tutti gli a^mi^ci,
^festa di un ^mondo che ^più non mori^rà,
^allelu^ia, alle^lu^ia!
\endverse


\endsong
%------------------------------------------------------------
%			FINE CANZONE
%------------------------------------------------------------

%-------------------------------------------------------------
%			INIZIO	CANZONE
%-------------------------------------------------------------


%titolo: 	Alleluia e poi
%autore: 	Luca Diliberto, Giuliana Monti
%tonalita: 	Do 


%%%%%% TITOLO E IMPOSTAZONI
\beginsong{Alleluia E poi}[by={L. Diliberto, G. Monti}]
\transpose{0} 						% <<< TRASPOSIZIONE #TONI (0 nullo)
\momenti{Acclamazione al Vangelo}							% <<< INSERISCI MOMENTI	
% momenti vanno separati da ; e vanno scelti tra:
% Ingresso; Atto penitenziale; Acclamazione al Vangelo; Dopo il Vangelo; Offertorio; Comunione; Ringraziamento; Fine; Santi; Pasqua; Avvento; Natale; Quaresima; Canti Mariani; Battesimo; Prima Comunione; Cresima; Matrimonio; Meditazione;
\ifchorded
	%\textnote{Tonalità originale }	% <<< EV COMMENTI (tonalità originale/migliore)
\fi




%%%%%% INTRODUZIONE
\ifchorded
\vspace*{\versesep}
\musicnote{
\begin{minipage}{0.48\textwidth}
\textbf{Intro:}
\hfill 
%( \eighthnote \, 80)   % <<  MODIFICA IL TEMPO
% Metronomo: \eighthnote (ottavo) \quarternote (quarto) \halfnote (due quarti)
\end{minipage}
} 	
\vspace*{-\versesep}
\beginverse*

\nolyrics

%---- Prima riga -----------------------------
\vspace*{-\versesep}
\[C] \[G]  \[C]	 % \[*D] per indicare le pennate, \rep{2} le ripetizioni

%---- Ogni riga successiva -------------------
%\vspace*{-\versesep}
%\[G] \[C]  \[D]	

%---- Ev Indicazioni -------------------------			
%\textnote{\textit{(Oppure tutta la strofa)} }	

\endverse
\fi



%%%%% STROFA
\beginverse		%Oppure \beginverse* se non si vuole il numero di fianco
\memorize 		% <<< DECOMMENTA se si vuole utilizzarne la funzione
%\chordsoff		& <<< DECOMMENTA se vuoi una strofa senza accordi
\[C]Chiama, ed \[G]io ver\[A-]rò da \[E-]Te:
\[F]Figlio, nel si\[C]lenzio, mi \[D]accoglie\[G]rai.
\[C]Voce e \[G]{poi\dots} la \[A-]liber\[E-]tà,
\[F]nella Tua Pa\[C]rola cam\[D7]mine\[G]rò.
\endverse



%%%%% RITORNELLO
\beginchorus
\textnote{\textbf{Rit.}}
\[C]Alleluia, \[G]alleluia, \[A-]allelu\[E-]ia,
\[F]alleluia, \[C]alle\[D7]lu\[G]ia,
\[C]Alleluia, \[G]alleluia, \[A-]allelu\[E-]ia,
\[F]alleluia, \[C]alle\[G]lu\[C]ia.

\endchorus




%%%%% STROFA
\beginverse		%Oppure \beginverse* se non si vuole il numero di fianco
%\memorize 		% <<< DECOMMENTA se si vuole utilizzarne la funzione
%\chordsoff		& <<< DECOMMENTA se vuoi una strofa senza accordi

^Danza, ed ^io ver^rò con ^Te:
^Figlio, la Tua ^strada com^prende^rò.
^Luce, e ^poi, nel ^tempo ^tuo,
^oltre il desi^derio ri^pose^rò. 
\endverse



\endsong
%------------------------------------------------------------
%			FINE CANZONE
%------------------------------------------------------------

%-------------------------------------------------------------
%			INIZIO	CANZONE
%-------------------------------------------------------------


%titolo: 	Alleluia Ed oggi ancra
%autore: 	Sequerii
%tonalita: 	Re-



%%%%%% TITOLO E IMPOSTAZONI
\beginsong{Alleluia Ed oggi ancora}[by={Sequeri}]% <<< MODIFICA TITOLO E AUTORE
\transpose{0} 						% <<< TRASPOSIZIONE #TONI (0 nullo)
\momenti{Acclamazione al Vangelo;}							% <<< INSERISCI MOMENTI	
% momenti vanno separati da ; e vanno scelti tra:
% Ingresso; Atto penitenziale; Acclamazione al Vangelo; Dopo il Vangelo; Offertorio; Comunione; Ringraziamento; Fine; Santi; Pasqua; Avvento; Natale; Quaresima; Canti Mariani; Battesimo; Prima Comunione; Cresima; Matrimonio; Meditazione; Spezzare del pane;
\ifchorded
	%\textnote{Tonalità migliore }	% <<< EV COMMENTI (tonalità originale/migliore)
\fi


%%%%%% INTRODUZIONE
\ifchorded
\vspace*{\versesep}
\textnote{Intro: \qquad \qquad  (\eighthnote 132) }% <<  MODIFICA IL TEMPO
% Metronomo: \eighthnote (ottavo) \quarternote (quarto) \halfnote (due quarti)
\vspace*{-\versesep}
\beginverse*

\nolyrics

%---- Prima riga -----------------------------
\vspace*{-\versesep}
\[(D-)] \[G-] \[C7] \[F]   % \[*D] per indicare le pennate, \rep{2} le ripetizioni

%---- Ogni riga successiva -------------------
\vspace*{-\versesep}
\[B&6] \[G-] \[A7]\[D-]

%---- Ev Indicazioni -------------------------			
%\textnote{\textit{(Oppure tutta la strofa)} }	

\endverse
\fi



\beginchorus
\[(D-)] Alle\[G-]luia, \[C7] allelu\[F]ia, \quad \[B&6] 
alle\[G-]luia, \[A7] alle\[D-]luia,
\[D-7] alle\[G-]luia, \[C7] allelu\[F]ia, \quad \[B&6]
  alle\[G-]luia, \[A7] alle\[D-]luia.
\endchorus

\beginverse*
Ed oggi an\[D-]cora, mio Si\[G-]gnore, \brk ascolte\[C]rò la tua pa\[F]rola,
che mi \[B&]guida nel cam\[G-]mino della \[A4]vi\[A]ta.
\endverse


\endsong
%------------------------------------------------------------
%			FINE CANZONE
%------------------------------------------------------------




%-------------------------------------------------------------
%			INIZIO	CANZONE
%-------------------------------------------------------------


%titolo: 	Alleluia, festa con Te
%autore: 	Fabio Avolio
%tonalita: 	Do 



%%%%%% TITOLO E IMPOSTAZONI
\beginsong{Alleluia, festa con Te}[by={Fabio Avolio}] 	% <<< MODIFICA TITOLO E AUTORE
\transpose{0} 						% <<< TRASPOSIZIONE #TONI (0 nullo)
\momenti{Acclamazione al Vangelo}							% <<< INSERISCI MOMENTI	
% momenti vanno separati da ; e vanno scelti tra:
% Ingresso; Atto penitenziale; Acclamazione al Vangelo; Dopo il Vangelo; Offertorio; Comunione; Ringraziamento; Fine; Santi; Pasqua; Avvento; Natale; Quaresima; Canti Mariani; Battesimo; Prima Comunione; Cresima; Matrimonio; Meditazione; Spezzare del pane;
\ifchorded
	%\textnote{Tonalità originale }	% <<< EV COMMENTI (tonalità originale/migliore)
\fi

%%%%%% INTRODUZIONE
\ifchorded
\vspace*{\versesep}
\textnote{Intro: \qquad \qquad  }%(\eighthnote 116) % <<  MODIFICA IL TEMPO
% Metronomo: \eighthnote (ottavo) \quarternote (quarto) \halfnote (due quarti)
\vspace*{-\versesep}
\beginverse*

\nolyrics

%---- Prima riga -----------------------------
\vspace*{-\versesep}
\[C] \[F] \[G] \[A-] \[F] \[C] \[D-] \[G]	 % \[*D] per indicare le pennate, \rep{2} le ripetizioni

%---- Ogni riga successiva -------------------
%\vspace*{-\versesep}
%\[G] \[C]  \[D]	

%---- Ev Indicazioni -------------------------			
%\textnote{\textit{(Oppure tutta la strofa)} }	

\endverse
\fi

%%%%% RITORNELLO
\beginchorus
\textnote{\textbf{Rit.}}

\[C]Allelu\[F]ia, \[G]allelu\[A-]ia, \[F]oggi è \[C]festa con \[D-]te, Ges\[G]ù.
\[C]Tu sei con \[F]noi, \[G]gioia ci \[A-]dai, \[F]allelu\[C]ia, alle\[G]lu\[C]ia.

\endchorus

%%%%% STROFA
\beginverse		%Oppure \beginverse* se non si vuole il numero di fianco
\memorize 		% <<< DECOMMENTA se si vuole utilizzarne la funzione
%\chordsoff		% <<< DECOMMENTA se vuoi una strofa senza accordi

\[C]Nella tua \[G]casa \[C]siamo ve\[F]nuti \[C]per incon\[D-]trar\[G]ti.
\[C]A te can\[G]tiamo \[A-]la nostra \[F]lode, \[C]gloria al tuo \[G]no\[C]me. \[G]

\endverse

%%%%% STROFA
\beginverse		%Oppure \beginverse* se non si vuole il numero di fianco
%\memorize 		% <<< DECOMMENTA se si vuole utilizzarne la funzione
\chordsoff		% <<< DECOMMENTA se vuoi una strofa senza accordi

^Il pane ^vivo ^che ci hai pro^messo ^dona la ^vi^ta.
^A te can^tiamo ^la nostra ^lode, ^gloria al tuo ^no^me. ^

\endverse

%%%%% STROFA
\beginverse		%Oppure \beginverse* se non si vuole il numero di fianco
%\memorize 		% <<< DECOMMENTA se si vuole utilizzarne la funzione
\chordsoff		% <<< DECOMMENTA se vuoi una strofa senza accordi

^Tu sei l'^mico ^che ci accom^pagna ^lungo il cam^mi^no.
^A te can^tiamo ^la nostra ^lode, ^gloria al tuo ^no^me. ^

\endverse

\endsong
%------------------------------------------------------------
%			FINE CANZONE
%------------------------------------------------------------
%-------------------------------------------------------------
%			INIZIO	CANZONE
%-------------------------------------------------------------


%titolo: 	Alleluia, gente di tutto il mondo
%autore: 	Pierangelo Sequeri
%tonalita: 	Fa



%%%%%% TITOLO E IMPOSTAZONI
\beginsong{Alleluia Gente di tutto il mondo}[by={P. Sequeri}] 	% <<< MODIFICA TITOLO E AUTORE
\transpose{0} 						% <<< TRASPOSIZIONE #TONI (0 nullo)
\momenti{Acclamazione al Vangelo}							% <<< INSERISCI MOMENTI	
% momenti vanno separati da ; e vanno scelti tra:
% Ingresso; Atto penitenziale; Acclamazione al Vangelo; Dopo il Vangelo; Offertorio; Comunione; Ringraziamento; Fine; Santi; Pasqua; Avvento; Natale; Quaresima; Canti Mariani; Battesimo; Prima Comunione; Cresima; Matrimonio; Meditazione; Spezzare del pane;
\ifchorded
	\textnote{$\bigstar$ Tonalità migliore }	% <<< EV COMMENTI (tonalità originale/migliore)
\fi


%%%%%% INTRODUZIONE
\ifchorded
\vspace*{\versesep}
\musicnote{
\begin{minipage}{0.48\textwidth}
\textbf{Intro}
\hfill 
%( \eighthnote \, 80)   % <<  MODIFICA IL TEMPO
% Metronomo: \eighthnote (ottavo) \quarternote (quarto) \halfnote (due quarti)
\end{minipage}
} 	
\vspace*{-\versesep}
\beginverse*

\nolyrics

%---- Prima riga -----------------------------
\vspace*{-\versesep}
\[F] \[B&] \[F]	 % \[*D] per indicare le pennate, \rep{2} le ripetizioni

%---- Ogni riga successiva -------------------
%\vspace*{-\versesep}
%\[G] \[C]  \[D]	

%---- Ev Indicazioni -------------------------			
%\textnote{\textit{(Oppure tutta la strofa)} }	

\endverse
\fi

%%%%% STROFA
\beginverse		%Oppure \beginverse* se non si vuole il numero di fianco
\memorize 		% <<< DECOMMENTA se si vuole utilizzarne la funzione
%\chordsoff		% <<< DECOMMENTA se vuoi una strofa senza accordi

\[F]Gente di \[A7]tutto il \[G-]mondo
ascol\[C7]tate il nostro \[F]can\[F7]to,
\[B&]lieti vi annun\[F]ciamo:
il Si\[B&-]gno\[C7]re è ri\[F]sorto!

\endverse

%%%%% RITORNELLO
\beginchorus
\textnote{\textbf{Rit.}}

\[F]Allelu\[D-]ia, \[G-]allelu\[C]ia, 
\[F]allelu, allelu, \[B&]allelu\[C7]ia.
\[F]Allelu\[D-]ia, \[G-]allelu\[C]ia,
\[F]allelu, \[C7]allelu\[F*]ia. \[B&*] \[F]

\endchorus

%%%%% STROFA
\beginverse		%Oppure \beginverse* se non si vuole il numero di fianco
%\memorize 		% <<< DECOMMENTA se si vuole utilizzarne la funzione
%\chordsoff		% <<< DECOMMENTA se vuoi una strofa senza accordi

^Il Figlio ^tanto a^mato
che il Dio ^nostro ci ha do^na^to
^l'ha risusci^tato
per la ^vi^ta del ^mondo!

\endverse

%%%%% STROFA
\beginverse		%Oppure \beginverse* se non si vuole il numero di fianco
%\memorize 		% <<< DECOMMENTA se si vuole utilizzarne la funzione
\chordsoff		% <<< DECOMMENTA se vuoi una strofa senza accordi

^Diede la ^propria ^vita
per a^more dei ^fra^telli.
^Vinta ormai la ^morte
è per ^sem^pre con ^noi.

\endverse

%%%%% STROFA
\beginverse		%Oppure \beginverse* se non si vuole il numero di fianco
%\memorize 		% <<< DECOMMENTA se si vuole utilizzarne la funzione
\chordsoff		% <<< DECOMMENTA se vuoi una strofa senza accordi

^Vivere ^del suo a^more
nell'at^tesa che ri^tor^ni:
^questa è la pa^rola
che ci ^do^na spe^ranza.

\endverse

\endsong
%------------------------------------------------------------
%			FINE CANZONE
%------------------------------------------------------------

%++++++++++++++++++++++++++++++++++++++++++++++++++++++++++++
%			CANZONE TRASPOSTA
%++++++++++++++++++++++++++++++++++++++++++++++++++++++++++++
\ifchorded
%decremento contatore per avere stesso numero
\addtocounter{songnum}{-1} 
\beginsong{Alleluia Gente di tutto il mondo}[by={P. Sequeri}]  	% <<< COPIA TITOLO E AUTORE
\transpose{+2} 						% <<< TRASPOSIZIONE #TONI + - (0 nullo)
%\preferflats  %SE VOGLIO FORZARE i bemolle come alterazioni
%\prefersharps %SE VOGLIO FORZARE i # come alterazioni
\ifchorded
	\textnote{$\lozenge$ Tonalità originale}	% <<< EV COMMENTI (tonalità originale/migliore)
\fi



%%%%%% INTRODUZIONE
\ifchorded
\vspace*{\versesep}
\musicnote{
\begin{minipage}{0.48\textwidth}
\textbf{Intro}
\hfill 
%( \eighthnote \, 80)   % <<  MODIFICA IL TEMPO
% Metronomo: \eighthnote (ottavo) \quarternote (quarto) \halfnote (due quarti)
\end{minipage}
} 	
\vspace*{-\versesep}
\beginverse*

\nolyrics

%---- Prima riga -----------------------------
\vspace*{-\versesep}
\[F] \[B&] \[F]	 % \[*D] per indicare le pennate, \rep{2} le ripetizioni

%---- Ogni riga successiva -------------------
%\vspace*{-\versesep}
%\[G] \[C]  \[D]	

%---- Ev Indicazioni -------------------------			
%\textnote{\textit{(Oppure tutta la strofa)} }	

\endverse
\fi

%%%%% STROFA
\beginverse		%Oppure \beginverse* se non si vuole il numero di fianco
\memorize 		% <<< DECOMMENTA se si vuole utilizzarne la funzione
%\chordsoff		% <<< DECOMMENTA se vuoi una strofa senza accordi

\[F]Gente di \[A7]tutto il \[G-]mondo
ascol\[C7]tate il nostro \[F]can\[F7]to,
\[B&]lieti vi annun\[F]ciamo:
il Si\[B&-]gno\[C7]re è ri\[F]sorto!

\endverse

%%%%% RITORNELLO
\beginchorus
\textnote{\textbf{Rit.}}

\[F]Allelu\[D-]ia, \[G-]allelu\[C]ia, 
\[F]allelu, allelu, \[B&]allelu\[C7]ia.
\[F]Allelu\[D-]ia, \[G-]allelu\[C]ia,
\[F]allelu, \[C7]allelu\[F*]ia. \[B&*] \[F]

\endchorus

%%%%% STROFA
\beginverse		%Oppure \beginverse* se non si vuole il numero di fianco
%\memorize 		% <<< DECOMMENTA se si vuole utilizzarne la funzione
%\chordsoff		% <<< DECOMMENTA se vuoi una strofa senza accordi

^Il Figlio ^tanto a^mato
che il Dio ^nostro ci ha do^na^to
^l'ha risusci^tato
per la ^vi^ta del ^mondo!

\endverse

%%%%% STROFA
\beginverse		%Oppure \beginverse* se non si vuole il numero di fianco
%\memorize 		% <<< DECOMMENTA se si vuole utilizzarne la funzione
\chordsoff		% <<< DECOMMENTA se vuoi una strofa senza accordi

^Diede la ^propria ^vita
per a^more dei ^fra^telli.
^Vinta ormai la ^morte
è per ^sem^pre con ^noi.

\endverse

%%%%% STROFA
\beginverse		%Oppure \beginverse* se non si vuole il numero di fianco
%\memorize 		% <<< DECOMMENTA se si vuole utilizzarne la funzione
\chordsoff		% <<< DECOMMENTA se vuoi una strofa senza accordi

^Vivere ^del suo a^more
nell'at^tesa che ri^tor^ni:
^questa è la pa^rola
che ci ^do^na spe^ranza.

\endverse

\endsong


\fi
%++++++++++++++++++++++++++++++++++++++++++++++++++++++++++++
%			FINE CANZONE TRASPOSTA
%++++++++++++++++++++++++++++++++++++++++++++++++++++++++++++

%titolo{Alleluia La nostra festa}
%autore{Varnavà}
%album{La nostra festa}
%tonalita{Do}
%famiglia{Liturgica}
%gruppo{Alleluia}
%momenti{Alleluia}
%identificatore{alleluia_la_nostra_festa}
%data_revisione{2011_12_31}
%trascrittore{Francesco Endrici}
\beginsong{Alleluia La nostra festa}[by={Varnavà}]
\beginchorus

\[C]Alle\[F]luia, alleluia, \[C]alle\[G]luia, alleluia,
\[C]alle\[F]luia, alleluia, \[C]al\[G]lelu\[C]ia. \rep{2}
\endchorus
\beginverse*
La nostra \[F]festa non \[G]deve fi\[C]nire
non \[A-]deve fi\[D-]nire e \[G]non fini\[C]\[(C7)]rà. \rep{2}
Per\[F]ché la \[G]festa \[A-]siamo \[E-]noi
che \[F]cammi\[D-]niamo verso \[G]Te, \[C7]
per\[F]ché la \[G]festa \[A-]siamo \[E-]noi
can\[F]tando in\[D7]sieme co\[G7]sì:
\endverse
\beginchorus
\[C]Alle\[F]luia, alleluia, \[C]alle\[G]luia, alleluia,
\[C]alle\[F]luia, alleluia, \[C]al\[G]lelu\[C]ia. \rep{2}
\endchorus
\endsong


%titolo{Alleluia lodate il Signore}
%autore{Frisina}
%album{Benedici il Signore}
%tonalita{Re}
%famiglia{Liturgica}
%gruppo{Alleluia}
%momenti{Alleluia}
%identificatore{alleluia_lodate_il_signore}
%data_revisione{2013_11_28}
%trascrittore{Francesco Endrici - Manuel Toniato}
\beginsong{Alleluia lodate il Signore}[by={Frisina}]
\beginchorus
\[D]Alle\[G]lu\[D]ia alle\[E-]lu\[B-]ia, lo\[G]date il Si\[A]gno\[D]re
alle\[G]lu\[D]ia alle\[E-]lu\[B-]ia, lo\[G]date il Si\[A]gno\[D]re.
\endchorus

\beginverse
Lo\[B&]date il Si\[F]gnore nel \[B&]suo tempio \[F]santo
lo\[G-]datelo nell'\[D-]alto firma\[C]mento
lo\[B&]datelo nei \[F]grandi pro\[B&]digi del suo a\[F]more
lo\[G]datene l'ec\[A-]celsa sua \[E]mae\[A]stà.
\endverse

\beginverse
\chordsoff
Lodatelo col suono gioioso delle trombe,
lodatelo sull'arpa e sulla cetra.
Lodatelo col suono dei timpani e dei sistri,
lodatelo coi flauti e sulle corde. 
\endverse

\beginverse
\chordsoff
Lodatelo col suono dei cimbali sonori,
lodatelo con cimbali squillanti.
Lodate il Signore voi tutte creature,
Lodate e cantate al Signore. 
\endverse

\beginverse
\chordsoff
Lodate voi tutti suoi angeli dei cieli,
Lodatelo voi tutte sue schiere.
Lodatelo voi cieli, voi astri e voi stelle,
lodate il Signore Onnipotente. 
\endverse

\beginverse
\chordsoff
Voi tutti governanti e genti della terra,
lodate il nome santo del Signore.
Perché solo la sua gloria risplende sulla terra,
lodate e benedite il Signore. 
\endverse
\endsong



%titolo{Alleluia Lode cosmica}
%autore{Puri}
%album{}
%tonalita{Re-}
%famiglia{Liturgica}
%gruppo{Alleluia}
%momenti{Alleluia}
%identificatore{alleluia_lode_cosmica}
%data_revisione{2015_10_25}
%trascrittore{Francesco Endrici}
\beginsong{Alleluia Lode cosmica}[by={Puri}]
\beginchorus
\[D-]Alleluia, \[F]allelu\[C]ia! 	\[C]
\[G-]Alleluia, \[B&]allelu\[D-]ia! \[C6]
Alle\[B&]lu\[F]ia, al\[G-]lelu\[D-]ia!
Alle\[C]lu\[D-]ia,	alle\[A4/7]lu\[A7]ia! \iflyric \rep{2} \fi
\ifchorded
\[D-]Alleluia, \[F]allelu\[C]ia! 	\[C]
\[G-]Alleluia, \[B&]allelu\[D-]ia! \[C6]
Alle\[B&]lu\[F]ia, al\[G-]lelu\[D-]ia!
Alle\[C]lu\[D-]ia,	alle\[A4/7]\[A4/7]lu\[A7]\[A7]ia! 
\fi
\endchorus
\beginverse
\[D-]Lodino il Si\[B&]gnor i \[F]cie\[C]li, 
\[D-]lodino il Si\[B&]gnor i \[F]ma\[C]ri,
gli \[G-]angeli, i \[G-]cieli dei \[B&]cieli:
\[B&]il Suo \[D-]nome è \[D-]grande e su\[A4/7]bli\[A7]me.

\[D-]Sole, luna e \[B&]stelle ar\[F]den\[C]ti, 
\[D-]neve, pioggia, \[B&]nebbia, e \[F]fuo\[C]co 
\[G-]lodino \[G-] il Suo \[B&]nome \[B&]in e\[D-]ter\[D-]no! 
\[A4/7]Sia \[A7]lode al Si\[B&7+]gnor! \[B&]
\[C6]Sia	\[B&7+]lode al Suo nome
\[A4]Sia \[A]lode al Signor!
\endverse
\beginchorus
\[D-]Alleluia, \[F]allelu\[C]ia! 	\[C] \iflyric \dots \fi
\ifchorded
\[G-]Alleluia, \[B&]allelu\[D-]ia! \[C6]
Alle\[B&]lu\[F]ia, al\[G-]lelu\[D-]ia!
Alle\[C]lu\[D-]ia,	alle\[A4/7]lu\[A7]ia! 
\[D-]Alleluia, \[F]allelu\[C]ia! 	\[C]
\[G-]Alleluia, \[B&]allelu\[D-]ia! \[C6]
Alle\[B&]lu\[F]ia, al\[G-]lelu\[D-]ia!
Alle\[C]lu\[D-]ia,	alle\[A4/7]\[A4/7]lu\[A7]\[A7]ia! 
\fi
\endchorus
\beginverse
^Lodino il Si^gnor le ^ter^re, 
^lodino il Si^gnor i ^mon^ti,
il ^vento ^ della tem^pesta
^che obbe^disce al^la Sua ^vo^ce,

^giudici, so^vrani ^tut^ti, 
^giovani, fan^ciulle, ^vec^chi 
^lodino ^ la Sua ^Gloria ^in e^ter^no! 
^Sia ^lode al Si^gnor! ^
^Sia ^lode al Suo nome!
^Sia ^lode al Signor!
\endverse
\beginchorus
\[D-]Alleluia, \[F]allelu\[C]ia! 	\[C]{\iflyric \dots \fi}
\ifchorded
\[G-]Alleluia, \[B&]allelu\[D-]ia! \[C6]
Alle\[B&]lu\[F]ia, al\[G-]lelu\[D-]ia!
Alle\[C]lu\[D-]ia,	alle\[A4/7]lu\[A7]ia! 
\[D-]Alleluia, \[F]allelu\[C]ia! 	\[C]
\[G-]Alleluia, \[B&]allelu\[D-]ia! \[C6]
Alle\[B&]lu\[F]ia, al\[G-]lelu\[D-]ia!
Alle\[C]lu\[D-]ia,	alle\[A4/7]\[A4/7]\[A7]\[A7]lu\[B&7]\[B&7]ia! %%%%%%%%%%%
Alle\[B&6]\[B&6]lu\[D]ia!
\fi
\endchorus
\endsong


%-------------------------------------------------------------
%			INIZIO	CANZONE
%-------------------------------------------------------------


%titolo: 	Alleluia Passeranno i cieli
%autore: 	Costa, Varnavà
%tonalita: Fa 
\beginsong{Alleluia Passeranno i cieli}[by={Costa, Varnavà}]
\transpose{3} 						% <<< TRASPOSIZIONE #TONI (0 nullo)
\momenti{Acclamazione al Vangelo}							% <<< INSERISCI MOMENTI	
% momenti vanno separati da ; e vanno scelti tra:
% Ingresso; Atto penitenziale; Acclamazione al Vangelo; Dopo il Vangelo; Offertorio; Comunione; Ringraziamento; Fine; Santi; Pasqua; Avvento; Natale; Quaresima; Canti Mariani; Battesimo; Prima Comunione; Cresima; Matrimonio; Meditazione;
\ifchorded
	%\textnote{Tonalità originale }	% <<< EV COMMENTI (tonalità originale/migliore)
\fi





%%%%%% INTRODUZIONE
\ifchorded
\vspace*{\versesep}
\textnote{Intro: \qquad \qquad  (\eighthnote 120) }% << MODIFICA IL TEMPO
% Metronomo: \eighthnote (ottavo) \quarternote (quarto) \halfnote (due quarti)
\vspace*{-\versesep}
\beginverse*

\nolyrics

%---- Prima riga -----------------------------
\vspace*{-\versesep}
\[D] \[A]  \[(D)]	 % \[*D] per indicare le pennate, \rep{2} le ripetizioni

%---- Ogni riga successiva -------------------
%\vspace*{-\versesep}
%\[G] \[C]  \[D]	

%---- Ev Indicazioni -------------------------			
%\textnote{\textit{(Oppure tutta la strofa)} }	

\endverse
\fi






\beginchorus
\[D]Alle\[A]luia, \[B-]alleluia, \[F#-]alleluia,
\[G]alleluia, \[D]allelu\[E-]ia, 
\[A7*]a-\[D]alle\[G]luia, al\[A7]lelu\[D]ia.
\endchorus


\beginverse*
\[D]Passeranno i \[A]cieli e \[B-]passerà la \[F#-]terra,
\[G]la Tua parola \[D]non passe\[E-]rà. 
\[A7*]a-\[D]alle\[G]luia, al\[A7]lelu\[D]ia.
\endverse




\endsong


%-------------------------------------------------------------
%			INIZIO	CANZONE
%-------------------------------------------------------------


%titolo: 	Alleluia Rendete grazie
%autore: 	Gen Verde
%tonalita: 	Sib


%%%%%% TITOLO E IMPOSTAZONI
\beginsong{Alleluia Rendete grazie}[by={Gen Verde}] 	% <<< MODIFICA TITOLO E AUTORE
\transpose{0} 						% <<< TRASPOSIZIONE #TONI (0 nullo)
\momenti{Acclamazione al Vangelo;}							% <<< INSERISCI MOMENTI	
% momenti vanno separati da ; e vanno scelti tra:
% Ingresso; Atto penitenziale; Acclamazione al Vangelo; Dopo il Vangelo; Offertorio; Comunione; Ringraziamento; Fine; Santi; Pasqua; Avvento; Natale; Quaresima; Canti Mariani; Battesimo; Prima Comunione; Cresima; Matrimonio; Meditazione;
\ifchorded
	%\textnote{Tonalità originale }	% <<< EV COMMENTI (tonalità originale/migliore)
\fi


%%%%%% INTRODUZIONE
\ifchorded
\vspace*{\versesep}
\musicnote{
\begin{minipage}{0.48\textwidth}
\textbf{Intro:}
\hfill 
%( \eighthnote \, 80)   % <<  MODIFICA IL TEMPO
% Metronomo: \eighthnote (ottavo) \quarternote (quarto) \halfnote (due quarti)
\end{minipage}
} 	
\vspace*{-\versesep}
\beginverse*

\nolyrics

%---- Prima riga -----------------------------
\vspace*{-\versesep}
\[B&] \[C] \[D-] \[A-]	 % \[*D] per indicare le pennate, \rep{2} le ripetizioni

%---- Ogni riga successiva -------------------
\vspace*{-\versesep}
\[B&] \[C] \[F] \[F]

%---- Ev Indicazioni -------------------------			
%\textnote{\textit{(Con stop e attacco solo voce)} }	

\endverse
\fi





%%%%% RITORNELLO
\beginchorus

Alle\[B&]luia, alle\[C]luia, alle\[D-]luia, \[A-]
alle\[B&]luia, alle\[C4]l\[C]u\[F]ia. \[A-]
Alle\[B&]luia, alle\[C]luia, alle\[D-]luia, \[A-]
alle\[B&]luia, alle\[C4]l\[C]u\[F]ia. \[F]

\endchorus




%%%%% STROFA
\beginverse		%Oppure \beginverse* se non si vuole il numero di fianco
\memorize 		% <<< DECOMMENTA se si vuole utilizzarne la funzione
%\chordsoff		& <<< DECOMMENTA se vuoi una strofa senza accordi
Ren\[B&]dete grazie a \[G-]Dio, Egli è bu\[C]ono,
e\[A7]terno e fedele è il suo a\[D-]more.
\[B&]Sì, è così: lo \[F]dica Israele,
\[G-]dica che il suo a\[F]more è per \[E&]sempre. \[C]
\endverse





%%%%% STROFA
\beginverse		%Oppure \beginverse* se non si vuole il numero di fianco
%\memorize 		% <<< DECOMMENTA se si vuole utilizzarne la funzione
\chordsoff		% <<< DECOMMENTA se vuoi una strofa senza accordi
La destra del Signore si è innalzata
a compiere grandiose meraviglie.
Non morirò, ma resterò in vita
e annuncerò i prodigi del Signore.
\endverse




%%%%% STROFA
\beginverse		%Oppure \beginverse* se non si vuole il numero di fianco
%\memorize 		% <<< DECOMMENTA se si vuole utilizzarne la funzione
\chordsoff		% <<< DECOMMENTA se vuoi una strofa senza accordi
La pietra che avevano scartato
è divenuta pietra angolare.
Questo prodigio ha fatto il Signore,
una meraviglia ai nostri occhi.
\endverse




\endsong
%------------------------------------------------------------
%			FINE CANZONE
%------------------------------------------------------------



%-------------------------------------------------------------
%			INIZIO	CANZONE
%-------------------------------------------------------------


%titolo: 	Alleluia Signore Sei Venuto
%autore: 	Scaglianti
%tonalita: 	LA



%%%%%% TITOLO E IMPOSTAZONI
\beginsong{Alleluia Signore sei venuto}[by={Scaglianti}]
	% <<< MODIFICA TITOLO E AUTORE
\transpose{0} 						% <<< TRASPOSIZIONE #TONI (0 nullo)
\momenti{Acclamazione al Vangelo;}							% <<< INSERISCI MOMENTI	
% momenti vanno separati da ; e vanno scelti tra:
% Ingresso; Atto penitenziale; Acclamazione al Vangelo; Dopo il Vangelo; Offertorio; Comunione; Ringraziamento; Fine; Santi; Pasqua; Avvento; Natale; Quaresima; Canti Mariani; Battesimo; Prima Comunione; Cresima; Matrimonio; Meditazione;
\ifchorded
	%\textnote{Tonalità originale }	% <<< EV COMMENTI (tonalità originale/migliore)
\fi


%%%%%% INTRODUZIONE
\ifchorded
\vspace*{\versesep}
\textnote{Intro: \qquad \qquad  }%(\eighthnote 116) % << MODIFICA IL TEMPO
% Metronomo: \eighthnote (ottavo) \quarternote (quarto) \halfnote (due quarti)
\vspace*{-\versesep}
\beginverse*

\nolyrics

%---- Prima riga -----------------------------
\vspace*{-\versesep}
\[A] \[A]  \[D]	 \[A] % \[*D] per indicare le pennate, \rep{2} le ripetizioni

%---- Ogni riga successiva -------------------
%\vspace*{-\versesep}
%\[G] \[C]  \[D]	

%---- Ev Indicazioni -------------------------			
\textnote{\textit{(con il ritmo della prima riga)} }	

\endverse
\fi





\beginverse
Si\[A]gnore, sei ve\[C#-]nuto fra\[D]tello in mezzo a \[E]noi,
Si\[F#-]gnore, hai por\[D]tato a\[B7]more e liber\[E]tà.
Si\[F#-]gnore sei vis\[F#-]suto \[E]nella \[D]pover\[C#-]tà:
\[A]noi ti ringra\[D]ziamo, Ge\[E4/7]sù.
\endverse

\beginchorus
\[A]Alle\[F#-]luia, \[D] allelu\[E]ia, \[A]alle\[F#-]luia, \[B7] allelu\[E]ia.
\[A]Alle\[F#-]luia, \[D] allelu\[E]ia, \[C#-]Alle\[D]lu\[*A]ia. \[*D]  \[*A] 
\endchorus

\beginverse
\chordsoff
Signore sei venuto, fratello nel dolore.
Signore, hai parlato del regno dell'amore.
Signore, hai donato la tua vita a noi.
Noi ti ringraziamo Gesù.
\endverse

\beginverse
\chordsoff
Sei qui con noi Signore, fratello in mezzo a noi.
Tu parli al nostro cuore d'amore e libertà
tu vuoi che ti cerchiamo nella povertà.
Noi ti ringraziamo Gesù.
\endverse
\endsong
%------------------------------------------------------------
%			FINE CANZONE
%------------------------------------------------------------




%titolo{Alleluia venite a me}
%autore{}
%album{}
%tonalita{Re}
%famiglia{Liturgica}
%gruppo{Alleluia}
%momenti{Alleluia}
%identificatore{alleluia_venite_a_me}
%data_revisione{2011_12_31}
%trascrittore{Francesco Endrici - Manuel Toniato}
\beginsong{Alleluia venite a me}
\beginverse
Ve\[D]nite a me: allelu{\[A/D]ja!} \brk Cre\[D]dete in me: allelu\[A/D]ja!
Io \[G]sono la \[A]via, la \[F#-]veri\[B-]tà: al\[F#-]lelu\[G]ja, al\[A7]lelu\[D]ja!
\endverse

\beginverse
\chordsoff
Restate in me: alleluia! \brk Vivete in me: alleluia!
Io sono la vita, la santità, alleluia, alleluia.
\endverse

\beginverse
\chordsoff
Cantate con me: alleluia! \brk Danzate con me: alleluia!
Io sono la gioia, la libertà: alleluia, alleluia.
\endverse
\endsong






%******* END SONGS ENVIRONMENT ******
\setcounter{GlobalSongCounter}{\thesongnum}
\end{songs}


\songchapter{Gloria}
%...............................................................................
%
%  ██████╗ ██╗      ██████╗ ██████╗ ██╗ █████╗     
% ██╔════╝ ██║     ██╔═══██╗██╔══██╗██║██╔══██╗    
% ██║  ███╗██║     ██║   ██║██████╔╝██║███████║    
% ██║   ██║██║     ██║   ██║██╔══██╗██║██╔══██║    
% ╚██████╔╝███████╗╚██████╔╝██║  ██║██║██║  ██║    
%  ╚═════╝ ╚══════╝ ╚═════╝ ╚═╝  ╚═╝╚═╝╚═╝  ╚═╝
% Font ANSI Shadow                                                          
%...............................................................................
\begin{songs}{}
\songcolumns{\canzsongcolumsnumber}
\setcounter{songnum}{\theGlobalSongCounter} %set songnum counter, otherwise would be reset

%set the default path inside current folder
\makeatletter
\def\input@path{{Songs/Gloria/}}
\makeatother


%***** INSERT SONGS HERE ******

%-------------------------------------------------------------
%			INIZIO	CANZONE
%-------------------------------------------------------------


%titolo: 	GGloria a Dio e pace in terra
%autore: 	A. Monti
%tonalita: 	Fa



%%%%%% TITOLO E IMPOSTAZONI
\beginsong{Gloria a Dio e pace in terra}[by={A. Monti}] 	% <<< MODIFICA TITOLO E AUTORE
\transpose{0} 						% <<< TRASPOSIZIONE #TONI (0 nullo)
\momenti{}							% <<< INSERISCI MOMENTI	
% momenti vanno separati da ; e vanno scelti tra:
% Ingresso; Atto penitenziale; Acclamazione al Vangelo; Dopo il Vangelo; Offertorio; Comunione; Ringraziamento; Fine; Santi; Pasqua; Avvento; Natale; Quaresima; Canti Mariani; Battesimo; Prima Comunione; Cresima; Matrimonio; Meditazione; Spezzare del pane;
\ifchorded
	\textnote{$\bigstar$ Tonalità migliore  }	% <<< EV COMMENTI (tonalità originale/migliore)
\fi


%%%%%% INTRODUZIONE
\ifchorded
\vspace*{\versesep}
\musicnote{
\begin{minipage}{0.48\textwidth}
\textbf{Intro:}
\hfill 
%( \eighthnote \, 80)   % <<  MODIFICA IL TEMPO
% Metronomo: \eighthnote (ottavo) \quarternote (quarto) \halfnote (due quarti)
\end{minipage}
} 	
\vspace*{-\versesep}
\beginverse*

\nolyrics

%---- Prima riga -----------------------------
\vspace*{-\versesep}
 \[F] \[C] \[F] \[F*]



%---- Ev Indicazioni -------------------------			
%\textnote{\textit{(Come la seconda parte della prima strofa)} }	

\endverse
\fi


%%%%% RITORNELLO
\beginchorus
\textnote{\textbf{Rit.}}
\[F]Gloria a Dio e \[C]pace in terra
a \[D-]chi ha riconosc\[A-]iuto
il \[B&]grande A\[C]more di un \[F]Dio \[D]
\[G-]che \[C]non ha a\[F]vuto pa\[B&]ura
 \[G-]a farsi \[C]uomo per \[F]noi, \[D]
\[G-]e che ch\[C]iede ad \[F]ogni crea\[B&]tura
\[G-]di farsi n\[C]uova ogni \[F]giorno. \[A]
\endchorus



%%%%% STROFA
\beginverse
\memorize 
\[D-]Un amore i\[A-]naspet\[D-]tato 
\[D-]silenzioso \[A-]si ri\[F]vela
\[C]spirito di \[F]vita \[C]e di veri\[D-]tà;
\endverse
\beginverse*
^è il progetto ^per il ^mondo, 
^è il disegno ^del cre^ato,
^segno di un mis^tero ^vivo in mezzo a ^noi. \[C]
\endverse



%%%%% STROFA
\beginverse
^Voce lieve ^mai ud^ita 
^porta all’uomo ^l’uomo ^vero:
^chi vorrà accet^tare ^con lui reste^rà.
\endverse
\beginverse*
^Corre l’eco ^tra le ^valli 
^trasportando ^la no^tizia:
^lode a chi ha do^nato ^gioia a tutti ^noi!  \[C]
\endverse


%%%%% STROFA
\beginverse
^Luce chiara ^nella ^notte, 
^alba senza ^più tra^monto,
^tenda del per^dono ^e di liber^tà.
\endverse
\beginverse*
^Trova forza ^lo scon^fitto, 
^trova senso ^la na^tura,
^nella via di ^Cristo ^vivo in mezzo a ^noi. \[C]
\endverse


\ifchorded
%%%%% RITORNELLO
\beginchorus
\textnote{\textbf{Rit.}}
\[F]Gloria a Dio e \[C]pace in terra
a \[D-]chi ha riconosc\[A-]iuto
il \[B&]grande A\[C]more di un \[F]Dio \[D]
\[G-]che \[C]non ha a\[F]vuto pa\[B&]ura
 \[G-]a farsi \[C]uomo per \[F]noi, \[D]
\[G-]e che ch\[C]iede ad \[F]ogni crea\[B&]tura
\[G-]di farsi n\[C]uova ogni \[F]giorno. \[F*]
\endchorus
\fi
\endsong
%------------------------------------------------------------
%			FINE CANZONE
%------------------------------------------------------------



%titolo{Gloria a te Cristo Gesù}
%autore{Lecot}
%album{Inno del giubileo}
%tonalita{Sol}
%famiglia{Liturgica}
%gruppo{}
%momenti{Ingresso;Congedo}
%identificatore{gloria_a_te_cristo_gesu}
%data_revisione{2011_12_31}
%trascrittore{Francesco Endrici - Manuel Toniato}
\beginsong{Gloria a te Cristo Gesù}[by={Lecot}]
\beginchorus
\[G]Glo\[C]ria a \[G]te, Cristo \[C]Ge\[B-]sù, \brk \[E-]og\[D]gi e \[G]sempre tu \[A-]regne\[B]rai.
\[D]Gloria a \[C]te! \[G]Presto ver\[D]rai: \brk \[E-]sei spe\[C]ranza \[A-]so\[D]lo \[G]tu.
\endchorus

\beginverse
\[G]Sia lode a \[D]te! \[E-]Cristo Si\[B-]gnore, \brk \[C]offri per\[F]dono, chiedi giu\[C]stizia:
l'anno di \[A-]grazia \[F]apre le \[E-]porte. \brk \[A-]Solo in \[G]te \[D]pace e uni\[B-]tà! 
\[G]Amen! Al\[C]le\[G]lu\[D]ia!
\endverse

\beginverse
\chordsoff
Sia lode a te! Prega con noi \brk la benedetta Vergine Madre:
tu l'esaudisci, tu la coroni. \brk Solo in te pace e unità 
Amen! Alleluia!
\endverse

\beginverse
\chordsoff
Sia lode a te! Tutta la Chiesa \brk celebra il Padre con la tua voce
e nello Spirito canta di gioia. \brk Solo in te pace e unità. 
Amen! Alleluia!
\endverse
\endsong


%-------------------------------------------------------------
%			INIZIO	CANZONE
%-------------------------------------------------------------


%titolo: 	Gloria (Buttazzo)
%autore: 	Buttazzo
%tonalita: 	Sol 



%%%%%% TITOLO E IMPOSTAZONI
\beginsong{Gloria Buttazzo}[by={F. Buttazzo}]	% <<< MODIFICA TITOLO E AUTORE
\transpose{0} 						% <<< TRASPOSIZIONE #TONI (0 nullo)
\momenti{Gloria}							% <<< INSERISCI MOMENTI	
% momenti vanno separati da ; e vanno scelti tra:
% Ingresso; Atto penite0nziale; Acclamazione al Vangelo; Dopo il Vangelo; Offertorio; Comunione; Ringraziamento; Fine; Santi; Pasqua; Avvento; Natale; Quaresima; Canti Mariani; Battesimo; Prima Comunione; Cresima; Matrimonio; Meditazione; Spezzare del pane;
\ifchorded
	%\textnote{Tonalità migliore }	% <<< EV COMMENTI (tonalità originale/migliore)
\fi

%%%%%% INTRODUZIONE
\ifchorded
\vspace*{\versesep}
\musicnote{
\begin{minipage}{0.48\textwidth}
\textbf{Intro}
\hfill 
%( \eighthnote \, 80)   % <<  MODIFICA IL TEMPO
% Metronomo: \eighthnote (ottavo) \quarternote (quarto) \halfnote (due quarti)
\end{minipage}
} 	
\vspace*{-\versesep}
\beginverse*

\nolyrics

%---- Prima riga -----------------------------
\vspace*{-\versesep}
\[G] \[D]  \[C] \[G]	% \[*D] per indicare le pennate, \rep{2} le ripetizioni

%---- Ogni riga successiva -------------------
\vspace*{-\versesep}
\[E-] \[C]  \[D4]	\[D]	

%---- Ev Indicazioni -------------------------			
\textnote{\textit{(oppure tutto il ritornello)} }	

\endverse
\fi



%%%%% RITORNELLO
\beginchorus
\textnote{\textbf{Rit.}}
\[G]Gloria a \[D]Dio nell'\[C]alto dei \[G]cieli 
e \[E-]pace in \[C]terra agli \[D4]uomi\[D]ni.
\[G]Gloria a \[F]Dio nell'\[C]alto dei \[G]cieli 
e \[E-]pace in \[C]terra agli \[D4]uomi\[G]ni. \[F] \[C] \[B7] 
\endchorus




%%%%% STROFA
\beginverse		%Oppure \beginverse* se non si vuole il numero di fianco
\memorize
\[E-]Noi Ti lo\[C]diamo, Ti \[D]benedi\[E-]ciamo.
\[E-]Ti ado\[C]riamo, Ti \[D]glorifi\[G]chiamo.
\[B]Ti ren\[E-]diamo \[C]gra\[G]zie 
\[C]per la Tua \[A-]Gloria im\[D4]men\[D]sa.
\endverse



%%%%% STROFA
\beginverse		%Oppure \beginverse* se non si vuole il numero di fianco
%\memorize 		% <<< DECOMMENTA se si vuole utilizzarne la funzione
%\chordsoff		% <<< DECOMMENTA se vuoi una strofa senza accordi
^Signore ^Dio, ^Re del c^ielo,
^Dio ^Padre ^Onnipo^tente.
^Gesù ^Cristo, Ag^nello di ^Dio, 
^Tu, ^Figlio del ^Pa-^dre.
\endverse



%%%%% STROFA
\beginverse		%Oppure \beginverse* se non si vuole il numero di fianco
%\memorize 		% <<< DECOMMENTA se si vuole utilizzarne la funzione
%\chordsoff		% <<< DECOMMENTA se vuoi una strofa senza accordi
^Tu che ^togli i pec^cati del ^mondo,  
^la nostra ^supplica as^colta, Sig^nore.
^Tu che s^iedi alla ^destra del ^Padre, 
^abbi pie^tà di  ^no-^i.
\endverse


%%%%% STROFA
\beginverse		%Oppure \beginverse* se non si vuole il numero di fianco
%\memorize 		% <<< DECOMMENTA se si vuole utilizzarne la funzione
%\chordsoff		% <<< DECOMMENTA se vuoi una strofa senza accordi
^Tu solo il ^Santo, Tu ^solo il Si^gnore, 
^Tu, l’Al^tissimo ^Gesù ^Cristo,
^con lo ^Spirito ^San-^to    
^nella ^Gloria del ^Pa-^dre
\endverse


\ifchorded
%%%%% RITORNELLO
\beginchorus
\[G]Gloria a \[D]Dio nell'\[C]alto dei \[G]cieli 
e \[E-]pace in \[C]terra agli \[D4]uomi\[D]ni.
\[G]Gloria a \[F]Dio nell'\[C]alto dei \[G]cieli 
e \[E-]pace in \[C]terra agli \[D4]uomi\[G]ni. 
\endchorus
\fi




%%%%%% EV. INTERMEZZO
\beginverse*
\vspace*{1.3\versesep}
{
	\nolyrics
	\musicnote{Chiusura strumentale}
	
	\ifchorded

    %---- Prima riga -----------------------------
    \vspace*{-\versesep}
    \[F] \[C] \[D]  	% \[*D] per indicare le pennate, \rep{2} le ripetizioni

    %---- Ogni riga successiva -------------------
    \vspace*{-\versesep}
    \[G] \[F] \[C] \[D]  \[G*]		

	\fi
	%---- Ev Indicazioni -------------------------			
	%\textnote{\textit{(ripetizione della strofa)}} 
	 
}
\vspace*{\versesep}
\endverse


\endsong
%------------------------------------------------------------
%			FINE CANZONE
%------------------------------------------------------------



%-------------------------------------------------------------
%			INIZIO	CANZONE
%-------------------------------------------------------------


%titolo: 	Gloria dal basso della terra
%autore: 	Sermig
%tonalita: 	Fa



%%%%%% TITOLO E IMPOSTAZONI
\beginsong{Gloria dal basso della terra}[by={Sermig}]% <<< MODIFICA TITOLO E AUTORE
\transpose{0} 						% <<< TRASPOSIZIONE #TONI (0 nullo)
\momenti{}							% <<< INSERISCI MOMENTI	
% momenti vanno separati da ; e vanno scelti tra:
% Ingresso; Atto penitenziale; Acclamazione al Vangelo; Dopo il Vangelo; Offertorio; Comunione; Ringraziamento; Fine; Santi; Pasqua; Avvento; Natale; Quaresima; Canti Mariani; Battesimo; Prima Comunione; Cresima; Matrimonio; Meditazione; Spezzare del pane;
\ifchorded
	%\textnote{Tonalità migliore }	% <<< EV COMMENTI (tonalità originale/migliore)
\fi


%%%%%% INTRODUZIONE
\ifchorded
\vspace*{\versesep}
\textnote{Intro: \qquad \qquad  }%(\eighthnote 116) % <<  MODIFICA IL TEMPO
% Metronomo: \eighthnote (ottavo) \quarternote (quarto) \halfnote (due quarti)
\vspace*{-\versesep}
\beginverse*

\nolyrics

%---- Prima riga -----------------------------
\vspace*{-\versesep}
 |\[G]\[D7] | \[G]\[D7] | \[E-]\[D] | \[C] \[D] |
  % \[*D] per indicare le pennate, \rep{2} le ripetizioni

%---- Ogni riga successiva -------------------
\vspace*{-\versesep}
\[G]\[D7] | \[G]\[D7] | \[E-]\[D]| \[C]  \[D] |	

%---- Ev Indicazioni -------------------------			
%\textnote{\textit{(Oppure tutta la strofa)} }	

\endverse
\fi





\beginverse*
\memorize
|\[G]Gloria dal basso \[D7]della |\[G]terra,
gloria \[D7]dal più in|\[E-]fame degli ster\[D]mi-i\[C]ni.  \[D]
|\[G]Gloria nella \[D7]care|\[G]stia,
gloria \[D7]nella |\[E-]guerra più a\[D]tro-o\[C]ce. \[D]
\endverse



\beginverse*
|^Gloria, gloria, ^gloria,
|^solo tu hai la ^forza |^con la tua ^glori|^a ^
|^di-asciugare-le ^lacri|^me,
di por^tare |^nella tua ^glori|^a  ^
|\[E-]nell'alto dei \[D]cie-e|\[C]li  
i |\[C]vinti della \[D]te-er|\[E-]ra, 
i |\[E-]vinti della \[D]te-er|\[G]ra, 
i |\[C]vinti della \[D]te-er|\[G]ra. \[D7]
\endverse



%%%%%% EV. INTERMEZZO
\beginverse*
\vspace*{1.3\versesep}
{
	\nolyrics
	\textnote{Intermezzo strumentale}
	
	\ifchorded

	%---- Prima riga -----------------------------
	\vspace*{-\versesep}
	\[G]\[D7] | \[E-]\[D] | \[C]\[D]

	%---- Ogni riga successiva -------------------
	\vspace*{-\versesep}
	 \[G]\[D7] | \[G]\[D7] | \[E-]\[D] | \[C]\[D]


	\fi
	%---- Ev Indicazioni -------------------------			
	%\textnote{\textit{(ripetizione della strofa)}} 
	 
}
\vspace*{\versesep}
\endverse




\beginverse*
|\[G]Gloria dal basso \[D7]della |\[G]terra,
gloria \[D7]dal più in|\[E-]fame degli ster\[D]mi-i\[C]ni. \[D]
|\[G]Gloria nella \[D7]care|\[G]stia,
gloria \[D7]nella |\[E-]guerra più a\[D]tro-o|\[C]ce. \[D]
\endverse


\beginverse*
|^Gloria, gloria, ^gloria,
|^solo tu hai la ^forza |^con la tua ^glori|^a ^
|^di-asciugare-le ^lacri|^me,
di por^tare |^nella tua ^glori|^a ^
|\[E-]nell'alto dei \[D]cie-e|\[C]li  
i |\[C]vinti della \[D]te-er|\[E-]ra, 
i |\[E-]vinti della \[D]te-er|\[G]ra, 
i |\[C]vinti della \[D]te-er|\[G]ra. \[D7]
\endverse


%%%%%% EV. INTERMEZZO
\beginverse*
\vspace*{1.3\versesep}
{
	\nolyrics
	\textnote{Chiusura strumentale}
	
	\ifchorded

	%---- Prima riga -----------------------------
	\vspace*{-\versesep}
	\[G]\[D7] | \[E-]\[D] | \[C]\[D]

	%---- Ogni riga successiva -------------------
	\vspace*{-\versesep}
	 \[G]\[D7] | \[G]\[D7] | \[E-]\[D] | \[G*]


	\fi
	%---- Ev Indicazioni -------------------------			
	%\textnote{\textit{(ripetizione della strofa)}} 
	 
}
\vspace*{\versesep}
\endverse



\endsong
%------------------------------------------------------------
%			FINE CANZONE
%------------------------------------------------------------



%-------------------------------------------------------------
%			INIZIO	CANZONE
%-------------------------------------------------------------


%titolo: 	Gloria
%autore: 	M. Giombini
%tonalita: 	Do



%%%%%% TITOLO E IMPOSTAZONI
\beginsong{Gloria (Giombini)}[by={M. Giombini}] 	% <<< MODIFICA TITOLO E AUTORE
\transpose{0} 						% <<< TRASPOSIZIONE #TONI (0 nullo)
\momenti{}							% <<< INSERISCI MOMENTI	
% momenti vanno separati da ; e vanno scelti tra:
% Ingresso; Atto penitenziale; Acclamazione al Vangelo; Dopo il Vangelo; Offertorio; Comunione; Ringraziamento; Fine; Santi; Pasqua; Avvento; Natale; Quaresima; Canti Mariani; Battesimo; Prima Comunione; Cresima; Matrimonio; Meditazione; Spezzare del pane;
\ifchorded
	%\textnote{Tonalità migliore }	% <<< EV COMMENTI (tonalità originale/migliore)
\fi


%%%%%% INTRODUZIONE
\ifchorded
\vspace*{\versesep}
\textnote{Intro: \qquad \qquad  }%(\eighthnote 116) % <<  MODIFICA IL TEMPO
% Metronomo: \eighthnote (ottavo) \quarternote (quarto) \halfnote (due quarti)
\vspace*{-\versesep}
\beginverse*

\nolyrics

%---- Prima riga -----------------------------
\vspace*{-\versesep}
\[C*] \[G]  \[A-] \[F] \quad \[C] \[G] \[C] \quad \[F*] \[G*]



%---- Ev Indicazioni -------------------------			
\textnote{\textit{(Come la seconda parte della prima strofa)} }	

\endverse
\fi


\beginverse*		%Oppure \beginverse* se non si vuole il numero di fianco
\memorize 		% <<< DECOMMENTA se si vuole utilizzarne la funzione
%\chordsoff		% <<< DECOMMENTA se vuoi una strofa senza accordi
\[C]Glo-\[G]o-o-\[C]ria, \quad \[F*] \[G*]
\[C]Glo-\[G]o-o-\[C]ria \quad \[F*] \[G*]
\[C] a \[G]Dio nell'\[A-]alto dei \[F]Cieli, 
\[C]Glo-\[G]o-o-\[C]ria! \quad \[F*] \[G*]
\endverse

\beginverse*
^E ^pa-a-a-^ce, \quad ^ ^
^e ^pa-a-a-^ce \quad ^ ^
^ in ^terra agli ^uomi^ni
di ^buona ^volon^tà. \quad \[C7]
\endverse

\beginverse*
Noi \[F]ti lodiamo \echo{noi ti lodiamo}
\[C]ti benediciamo \echo{ti benediciamo}
ti \[F]adoriamo \echo{ti adoriamo}
\[G]ti glorifichiamo \echo{ti glorifichiamo}.
\endverse

\beginverse*
^Ti rendi^a-a-a-^mo  \quad ^ ^
^gra-^a-a-^zie  \quad ^ ^
^ per ^la tua ^Glori^a
im^me-^e-en^sa.  \quad \[E7]
\endverse

\beginverse*
Si\[A-]gnore Figlio Uni\[E-]genito
Gesù \[F]Cristo, Si\[G]gnore \[C]Dio \quad \[E7]
\[A-] Agnello \[G] di Dio,
\[F] Figlio del \[E7]Padre.
\endverse

\beginverse*
\[A-] Tu che togli i pec\[E-]cati, \echo{tu che togli i peccati}
\[A-] i peccati del \[E]mondo, \echo{i peccati del mondo}
\[F] abbi pie\[C]tà  di noi,
\[D7] abbi pie\[G]tà  di noi!
\[A-] Tu che togli i pecca\[E-]ti,
\[A-] i peccati del \[E]mondo,
\[F] accogli, \[C] accogli,
\[D7] la nostra \[G]supplica!
\endverse

\beginverse*
\[C] Tu che siedi alla \[G]destra \[A-]
\[(A-)] Alla destra del \[E-]Padre \[F]
\[(F)] Abbi pie\[C]tà  di noi, \[D7] abbi pie\[G]tà  di noi
\endverse 


%%%%%% INTERMEZZO
\beginverse*
\vspace*{1.3\versesep}
{
	\nolyrics
	\textnote{Intermezzo strumentale}
	
	\ifchorded

	%---- Prima riga -----------------------------
	\vspace*{-\versesep}
	\[C] \[G] \[C] \quad \[F*] \[G*]	
	%---- Ogni riga successiva -------------------
	\vspace*{-\versesep}
	\[C] \[G] \[C] \quad \[F*] \[G*]
	%---- Ogni riga successiva -------------------
	\vspace*{-\versesep}
	\[C] \[G]  \[A-] \[F] \quad \[C] \[G] \[C] \quad \[C7]
	
	\fi
	%---- Ev Indicazioni -------------------------			
	%\textnote{\textit{(ripetizione della strofa)}} 
	 
}
\vspace*{\versesep}
\endverse


\beginverse*
Per\[F]chè tu solo il Santo \echo{perchè tu solo il Santo}
tu \[C]solo il Signore \echo{tu solo il Signore}
tu \[F]solo l'Altissimo \echo{tu solo l'Altissimo}
\[G]Gesù Cristo \echo{Gesù Cristo}.
\endverse

\beginverse*
^Con lo ^Spiri^to  \quad ^ ^
^Sa-^a-an^to  \quad ^ ^
^ nella ^gloria ^di Dio ^Padre 
^A-^a-a^men \quad \[F*] \[G*]
\endverse

\beginverse*
^Con lo ^Spiri^to  \echo{con lo ^Spiri^to Santo} 
^Sa-^a-an^to   \echo{nella ^glo^ria di Dio Padre}
^ nella ^gloria ^di Dio ^Padre 
^A-^a-a^men! \echo{nella \[F*]glo\[G*]ria di Dio Padre} 
\endverse



\beginverse*
\[C] nella \[G]gloria \[A-]di Dio \[F]Padre 
\[C]A-\[G7]a-a\[C]men! \quad \[G] \quad \[C*]
\endverse

\endsong
%------------------------------------------------------------
%			FINE CANZONE
%------------------------------------------------------------



%-------------------------------------------------------------
%			INIZIO	CANZONE
%-------------------------------------------------------------


%titolo: 	Gloria (esme)
%autore: 	Gen Verde
%tonalita: 	FA e RE 



%%%%%% TITOLO E IMPOSTAZONI
\beginsong{Gloria nell'alto dei cieli}[by={Gen Verde, Esme}] 	% <<< MODIFICA TITOLO E AUTORE
\transpose{0} 						% <<< TRASPOSIZIONE #TONI (0 nullo)
\momenti{}							% <<< INSERISCI MOMENTI	
% momenti vanno separati da ; e vanno scelti tra:
% Ingresso; Atto penitenziale; Acclamazione al Vangelo; Dopo il Vangelo; Offertorio; Comunione; Ringraziamento; Fine; Santi; Pasqua; Avvento; Natale; Quaresima; Canti Mariani; Battesimo; Prima Comunione; Cresima; Matrimonio; Meditazione;
\ifchorded
	%\textnote{Tonalità originale }	% <<< EV COMMENTI (tonalità originale/migliore)
\fi




%%%%%% INTRODUZIONE
\ifchorded
\vspace*{\versesep}
\textnote{Intro: \qquad \qquad  }%(\eighthnote 116) % << MODIFICA IL TEMPO
% Metronomo: \eighthnote (ottavo) \quarternote (quarto) \halfnote (due quarti)
\vspace*{-\versesep}
\beginverse*

\nolyrics

%---- Prima riga -----------------------------
\vspace*{-\versesep}
\[F]   \[B&]  \[C]  \[C]	 \rep{2} % \[*D] per indicare le pennate, \rep{2} le ripetizioni

%---- Ogni riga successiva -------------------
%\vspace*{-\versesep}
%\[G] \[C]  \[D]	

%---- Ev Indicazioni -------------------------			
%\textnote{\textit{(Oppure tutta la strofa)} }	

\endverse
\fi





%%%%% RITORNELLO
\beginchorus
%\textnote{\textbf{Rit.}}

\[F]Gloria, \[B&]gloria a \[D-]Dio. \[C]
Gloria, \[F]gloria nel\[B&]l'alto dei \[D-]cie\[C]li.
\[F]Pace in \[B&]terra agli \[D-]uomi\[C]ni
di \[F]buona \[B&]volon\[F]tà. \[B&] 
\[F]Gl\[B&]o\[F]ria!
\endchorus







%%%%% STROFA
\beginverse*		%Oppure \beginverse* se non si vuole il numero di fianco
\memorize 		% <<< DECOMMENTA se si vuole utilizzarne la funzione
%\chordsoff		& <<< DECOMMENTA se vuoi una strofa senza accordi
Noi \[B&]ti lo\[F]diamo, \[G-]ti benedi\[F]ciamo,
ti \[B&]ador\[F]iamo, glo\[E&]rifichiamo \[F]te,
\[B&]ti ren\[F]diamo \[G-7]grazie per la \[F]tua immensa
\[E&]glor\[C4]ia. \[C]

\endverse


%%%%% STROFA
\beginverse*
Si^gnore ^Dio, ^glor^ia!  ^Re del ci^elo, ^glor^ia!
^Dio ^Padre, ^Dio onnipo^tente, ^glor\[C]ia! \[G-] \[E&] \[C]
\endverse



%%%%% RITORNELLO
\beginchorus
%\textnote{\textbf{Rit.}}

\[F]Gloria, \[B&]gloria a \[D-]Dio. \[C]
Gloria, \[F]gloria nel\[B&]l'alto dei \[D-]cie\[C]li.
\[F]Pace in \[B&]terra agli \[D-]uomi\[C]ni
di \[F]buona \[B&]volon\[F]tà. \[B&] 
\[F]Gl\[B&]o\[F]ria!
\endchorus



%%%%% STROFA
\beginverse*
Si\[F]gnore, Figlio uni\[E&]genito, \[B&]Gesù Cri\[F]sto,
Si\[F]gnore, Agnello di \[E&]Dio, \[B&]Figlio del Pad\[F]re.
\[F]Tu che togli i pec\[E&]cati del mondo,
a\[B&]bbi pietà  di no\[F]i;
\[F]tu che togli i pec\[E&]cati del mondo,
a\[B&]ccogli la nostra su\[F]pplica;
\[F]tu che siedi alla \[E&]destra del Padre,
\[B&]abbi pietà  di n\[C4]oi. \[C]
\endverse




%%%%% RITORNELLO
\beginchorus
%\textnote{\textbf{Rit.}}

\[F]Gloria, \[B&]gloria a \[D-]Dio. \[C]
Gloria, \[F]gloria nel\[B&]l'alto dei \[D-]cie\[C]li.
\[F]Pace in \[B&]terra agli \[D-]uomi\[C]ni
di \[F]buona \[B&]volon\[F]tà. \[B&] 
\[F]Gl\[B&]o\[F]ria!
\endchorus


%%%%% STROFA
\beginverse*
Per^chè tu ^solo il ^Santo, il Si^gnore,
tu ^solo l'Al^tissimo, ^Cristo G^esù
^con lo ^Spirito ^Santo nella ^gloria
del ^Pad\[C]re. \[G-] \[E&] \[C]
\endverse



%%%%% RITORNELLO
\beginchorus
%\textnote{\textbf{Rit.}}

\[F]Gloria, \[B&]gloria a \[D-]Dio. \[C]
Gloria, \[F]gloria nel\[B&]l'alto dei \[D-]cie\[C]li.
\[F]Pace in \[B&]terra agli \[D-]uomi\[C]ni
di \[F]buona \[B&]volon\[F]tà. \[B&] 
\[F]Gl\[B&]o\[F]ria! \[F] \[*F]
\endchorus




\endsong
%------------------------------------------------------------
%			FINE CANZONE
%------------------------------------------------------------
%++++++++++++++++++++++++++++++++++++++++++++++++++++++++++++
%			CANZONE TRASPOSTA
%++++++++++++++++++++++++++++++++++++++++++++++++++++++++++++
\ifchorded
%decremento contatore per avere stesso numero
\addtocounter{songnum}{-1} 
\beginsong{Gloria nell'alto dei cieli}[by={Gen Verde, Esme}] 	% <<< COPIA TITOLO E AUTORE
\transpose{-3} 						% <<< TRASPOSIZIONE #TONI + - (0 nullo)
\ifchorded
	\textnote{Tonalità migliore per le chitarre}	% <<< EV COMMENTI (tonalità originale/migliore)
\fi




%%%%%% INTRODUZIONE
\ifchorded
\vspace*{\versesep}
\textnote{Intro: \qquad \qquad  }%(\eighthnote 116) % << MODIFICA IL TEMPO
% Metronomo: \eighthnote (ottavo) \quarternote (quarto) \halfnote (due quarti)
\vspace*{-\versesep}
\beginverse*

\nolyrics

%---- Prima riga -----------------------------
\vspace*{-\versesep}
\[F]   \[B&]  \[C]  \[C]	 \rep{2} % \[*D] per indicare le pennate, \rep{2} le ripetizioni

%---- Ogni riga successiva -------------------
%\vspace*{-\versesep}
%\[G] \[C]  \[D]	

%---- Ev Indicazioni -------------------------			
%\textnote{\textit{(Oppure tutta la strofa)} }	

\endverse
\fi





%%%%% RITORNELLO
\beginchorus
%\textnote{\textbf{Rit.}}

\[F]Gloria, \[B&]gloria a \[D-]Dio. \[C]
Gloria, \[F]gloria nel\[B&]l'alto dei \[D-]cie\[C]li.
\[F]Pace in \[B&]terra agli \[D-]uomi\[C]ni
di \[F]buona \[B&]volon\[F]tà. \[B&] 
\[F]Gl\[B&]o\[F]ria!
\endchorus







%%%%% STROFA
\beginverse*		%Oppure \beginverse* se non si vuole il numero di fianco
\memorize 		% <<< DECOMMENTA se si vuole utilizzarne la funzione
%\chordsoff		& <<< DECOMMENTA se vuoi una strofa senza accordi
Noi \[B&]ti lo\[F]diamo, \[G-]ti benedi\[F]ciamo,
ti \[B&]ador\[F]iamo, glo\[E&]rifichiamo \[F]te,
\[B&]ti ren\[F]diamo \[G-7]grazie per la \[F]tua immensa
\[E&]glor\[C4]ia. \[C]

\endverse


%%%%% STROFA
\beginverse*
Si^gnore ^Dio, ^glor^ia!  ^Re del ci^elo, ^glor^ia!
^Dio ^Padre, ^Dio onnipo^tente, ^glor\[C]ia! \[G-] \[E&] \[C]
\endverse



%%%%% RITORNELLO
\beginchorus
%\textnote{\textbf{Rit.}}

\[F]Gloria, \[B&]gloria a \[D-]Dio. \[C]
Gloria, \[F]gloria nel\[B&]l'alto dei \[D-]cie\[C]li.
\[F]Pace in \[B&]terra agli \[D-]uomi\[C]ni
di \[F]buona \[B&]volon\[F]tà. \[B&] 
\[F]Gl\[B&]o\[F]ria!
\endchorus



%%%%% STROFA
\beginverse*
Si\[F]gnore, Figlio uni\[E&]genito, \[B&]Gesù Cri\[F]sto,
Si\[F]gnore, Agnello di \[E&]Dio, \[B&]Figlio del Pad\[F]re.
\[F]Tu che togli i pec\[E&]cati del mondo,
a\[B&]bbi pietà  di no\[F]i;
\[F]tu che togli i pec\[E&]cati del mondo,
a\[B&]ccogli la nostra su\[F]pplica;
\[F]tu che siedi alla \[E&]destra del Padre,
\[B&]abbi pietà  di n\[C4]oi. \[C]
\endverse




%%%%% RITORNELLO
\beginchorus
%\textnote{\textbf{Rit.}}

\[F]Gloria, \[B&]gloria a \[D-]Dio. \[C]
Gloria, \[F]gloria nel\[B&]l'alto dei \[D-]cie\[C]li.
\[F]Pace in \[B&]terra agli \[D-]uomi\[C]ni
di \[F]buona \[B&]volon\[F]tà. \[B&] 
\[F]Gl\[B&]o\[F]ria!
\endchorus


%%%%% STROFA
\beginverse*
Per^chè tu ^solo il ^Santo, il Si^gnore,
tu ^solo l'Al^tissimo, ^Cristo G^esù
^con lo ^Spirito ^Santo nella ^gloria
del ^Pad\[C]re. \[G-] \[E&] \[C]
\endverse



%%%%% RITORNELLO
\beginchorus
%\textnote{\textbf{Rit.}}

\[F]Gloria, \[B&]gloria a \[D-]Dio. \[C]
Gloria, \[F]gloria nel\[B&]l'alto dei \[D-]cie\[C]li.
\[F]Pace in \[B&]terra agli \[D-]uomi\[C]ni
di \[F]buona \[B&]volon\[F]tà. \[B&] 
\[F]Gl\[B&]o\[F]ria! \[F] \[*F]
\endchorus




\endsong

\fi
%++++++++++++++++++++++++++++++++++++++++++++++++++++++++++++
%			FINE CANZONE TRASPOSTA
%++++++++++++++++++++++++++++++++++++++++++++++++++++++++++++


%-------------------------------------------------------------
%			INIZIO	CANZONE
%-------------------------------------------------------------


%titolo: 	Gloria (Ricci)
%autore: 	D. Ricci
%tonalita: 	re



%%%%%% TITOLO E IMPOSTAZONI
\beginsong{Gloria (Ricci)}[by={D. Ricci}] 	% <<< MODIFICA TITOLO E AUTORE
\transpose{0} 						% <<< TRASPOSIZIONE #TONI (0 nullo)
\momenti{}							% <<< INSERISCI MOMENTI	
% momenti vanno separati da ; e vanno scelti tra:
% Ingresso; Atto penitenziale; Acclamazione al Vangelo; Dopo il Vangelo; Offertorio; Comunione; Ringraziamento; Fine; Santi; Pasqua; Avvento; Natale; Quaresima; Canti Mariani; Battesimo; Prima Comunione; Cresima; Matrimonio; Meditazione;
\ifchorded
	%\textnote{Tonalità migliore per le chitarre }	% <<< EV COMMENTI (tonalità originale/migliore)
\fi




%%%%%% INTRODUZIONE
\ifchorded
\vspace*{\versesep}
\textnote{Intro: \qquad \qquad  }%(\eighthnote 116) % << MODIFICA IL TEMPO
% Metronomo: \eighthnote (ottavo) \quarternote (quarto) \halfnote (due quarti)
\vspace*{-\versesep}
\beginverse*

\nolyrics

%---- Prima riga -----------------------------
\vspace*{-\versesep}
\[D]    \[C] \[D]  \[C]	 \rep{2} % \[*D] per indicare le pennate, \rep{2} le ripetizioni

%---- Ogni riga successiva -------------------
%\vspace*{-\versesep}
%\[G] \[C]  \[D]	

%---- Ev Indicazioni -------------------------			
%\textnote{\textit{(Oppure tutta la strofa)} }	

\endverse
\fi





%%%%% RITORNELLO
\beginchorus
%\textnote{\textbf{Rit.}}

\[D]Gloria \echo{nell'\[C]alto dei cieli},
\[D]Gloria \echo{nell'\[C]alto dei cieli},
\[D]Gloria \echo{nell'\[C]alto dei cieli},
\[G]Gloria Gloria a \[D]Dio!
\[D]Pace-e \echo{agli \[C]uomini in terra},
\[D]Pace-e \echo{agli \[C]uomini in terra},
\[D]Pace-e \echo{agli \[C]uomini di},
\[G]buona volon\[D]tà !
\endchorus







%%%%% STROFA
\beginverse*		%Oppure \beginverse* se non si vuole il numero di fianco
\memorize 		% <<< DECOMMENTA se si vuole utilizzarne la funzione
%\chordsoff		& <<< DECOMMENTA se vuoi una strofa senza accordi

\[F] Noi ti lodiamo, \[G]ti benediciamo,
\[D] ti adoriamo, ti glorifichiamo.
\[F] Ti rendiamo grazie \[G]per la tua gloria 
im\[D]mensa, gloria immensa.
\[F] Signore Dio, \[G]Re del Cielo, 
 Dio \[D]Padre Onnipotente.
\[F] Signore Dio, \[G]Re del Cielo, 
Dio \[D]Padre Onnipo\[A]tente.

\endverse






%%%%% RITORNELLO
\beginchorus
%\textnote{\textbf{Rit.}}

\[D]Gloria \echo{nell'\[C]alto dei cieli},
\[D]Gloria \echo{nell'\[C]alto dei cieli},
\[D]Gloria \echo{nell'\[C]alto dei cieli},
\[G]Gloria Gloria a \[D]Dio!
\[D]Pace-e \echo{agli \[C]uomini in terra},
\[D]Pace-e \echo{agli \[C]uomini in terra},
\[D]Pace-e \echo{agli \[C]uomini di},
\[G]buona volon\[D]tà !
\endchorus





%%%%% STROFA
\beginverse*
^ Signore Figlio uni^genito Gesù Cristo,
Sig^nore Dio agnello di Dio, figlio del Padre.
^ Tu che togli i pec^cati del mondo, 
ab^bi pietà di noi,
^ Tu che togli i pec^cati del mondo, 
ac^cogli la nostra supplica,
^ Tu che siedi alla ^destra di Dio Padre, 
^abbi pietà di ^noi.
\endverse







%%%%% RITORNELLO
\beginchorus
%\textnote{\textbf{Rit.}}

\[D]Gloria \echo{nell'\[C]alto dei cieli},
\[D]Gloria \echo{nell'\[C]alto dei cieli},
\[D]Gloria \echo{nell'\[C]alto dei cieli},
\[G]Gloria Gloria a \[D]Dio!
\[D]Pace-e \echo{agli \[C]uomini in terra},
\[D]Pace-e \echo{agli \[C]uomini in terra},
\[D]Pace-e \echo{agli \[C]uomini di},
\[G]buona volon\[D]tà !
\endchorus





%%%%% STROFA
\beginverse*
^ Perchè tu solo il Santo, ^tu solo il Signore,
^tu solo l’Altissimo 
Gesù Cristo,
^ con lo Spirito ^Santo nella Gloria 
di Dio ^Padre, 
\[A]A-a-\[D]men. \[C] \[D] \[C] \[D] \[C] \[D*]
\endverse




\endsong
%------------------------------------------------------------
%			FINE CANZONE
%------------------------------------------------------------






%******* END SONGS ENVIRONMENT ******
\setcounter{GlobalSongCounter}{\thesongnum}
\end{songs}


\songchapter{Liturgia}
%...............................................................................
%
% ██╗     ██╗████████╗██╗   ██╗██████╗  ██████╗ ██╗ █████╗ 
% ██║     ██║╚══██╔══╝██║   ██║██╔══██╗██╔════╝ ██║██╔══██╗
% ██║     ██║   ██║   ██║   ██║██████╔╝██║  ███╗██║███████║
% ██║     ██║   ██║   ██║   ██║██╔══██╗██║   ██║██║██╔══██║
% ███████╗██║   ██║   ╚██████╔╝██║  ██║╚██████╔╝██║██║  ██║
% ╚══════╝╚═╝   ╚═╝    ╚═════╝ ╚═╝  ╚═╝ ╚═════╝ ╚═╝╚═╝  ╚═╝
% Font: ANSI Shadow                                                                               
%...............................................................................
\begin{songs}{}
\songcolumns{\canzsongcolumsnumber}
\setcounter{songnum}{\theGlobalSongCounter} %set songnum counter, otherwise would be reset

%set the default path inside current folder
\makeatletter
\def\input@path{{Songs/Liturgia/}}
\makeatother





%***** INSERT SONGS HERE ******

%AAA
%-------------------------------------------------------------
%			INIZIO	CANZONE
%-------------------------------------------------------------

%titolo: A volte le parole
%autore: Colombo, Racz
%tonalita: Sol 


%%%%%% TITOLO E IMPOSTAZONI
\beginsong{A volte le parole}[ititle={La tua parola è  vita per noi}, by={S. Colombo, A. Racz}] 	% <<< MODIFICA TITOLO E AUTORE
\transpose{0} 						% <<< TRASPOSIZIONE #TONI (0 nullo)
\momenti{ Dopo il Vangelo}							% <<< INSERISCI MOMENTI	
% momenti vanno separati da ; e vanno scelti tra:
% Ingresso; Atto penitenziale; Acclamazione al Vangelo; Dopo il Vangelo; Offertorio; Comunione; Ringraziamento; Fine; Santi; Pasqua; Avvento; Natale; Quaresima; Canti Mariani; Battesimo; Prima Comunione; Cresima; Matrimonio; Meditazione;

%%%%%% INTRODUZIONE
\ifchorded
\vspace*{\versesep}
\musicnote{
\begin{minipage}{0.48\textwidth}
\textbf{Intro}
\hfill 
%( \eighthnote \, 80)   % <<  MODIFICA IL TEMPO
% Metronomo: \eighthnote (ottavo) \quarternote (quarto) \halfnote (due quarti)
\end{minipage}
} 	
\vspace*{-\versesep}
\beginverse*

\nolyrics

%---- Prima riga -----------------------------
\vspace*{-\versesep}
\[G] \[D] \[G] 	 % \[*D] per indicare le pennate, \rep{2} le ripetizioni

%---- Ogni riga successiva -------------------
%\vspace*{-\versesep}
%\[G] \[C]  \[D]	

%---- Ev Indicazioni -------------------------			
%\textnote{\textit{(Oppure tutta la strofa)} }	

\endverse
\fi




%%%%% STROFA
\beginverse*
\[G]La tua pa\[D]rola è \[E-]vita per \[C]noi.
\[D] Luce del cam\[B-]mi\[E-]no,
\[A-]forza che il cuore non \[D]ha.
\[G]La tua pa\[D]rola la \[E-]pace ci \[C]da.
\[D] Vince ogni \[B-]ma\[E-]le,
\[A-] crea uni\[D]tà,
\[C] crea \[D]uni\[G]tà. 
\endverse



\endsong
%------------------------------------------------------------
%			FINE CANZONE
%------------------------------------------------------------
%-------------------------------------------------------------
%			INIZIO	CANZONE
%-------------------------------------------------------------


%titolo: 	Accendi la vita
%autore: 	Bertoglio, Testa
%tonalita: 	Sol 



%%%%%% TITOLO E IMPOSTAZONI
\beginsong{Accendi la vita}[by={E. Bertoglio, A. Testa}] 	% <<< MODIFICA TITOLO E AUTORE
\transpose{0} 						% <<< TRASPOSIZIONE #TONI (0 nullo)
\momenti{Ingresso; Cresima; Prima Comunione; Congedo; Santi; Matrimonio;}							% <<< INSERISCI MOMENTI	
% momenti vanno separati da ; e vanno scelti tra:
% Ingresso; Atto penitenziale; Acclamazione al Vangelo; Dopo il Vangelo; Offertorio; Comunione; Ringraziamento; Fine; Santi; Pasqua; Avvento; Natale; Quaresima; Canti Mariani; Battesimo; Prima Comunione; Cresima; Matrimonio; Meditazione;
\ifchorded
	%\textnote{Tonalità originale }	% <<< EV COMMENTI (tonalità originale/migliore)
\fi


%%%%%% INTRODUZIONE
\ifchorded
\vspace*{\versesep}
\musicnote{
\begin{minipage}{0.48\textwidth}
\textbf{Intro}
\hfill 
%( \eighthnote \, 80)   % <<  MODIFICA IL TEMPO
% Metronomo: \eighthnote (ottavo) \quarternote (quarto) \halfnote (due quarti)
\end{minipage}
} 	
\vspace*{-\versesep}
\beginverse*

\nolyrics

%---- Prima riga -----------------------------
\vspace*{-\versesep}
\[D] \[A]  \[D]	 % \[*D] per indicare le pennate, \rep{2} le ripetizioni

%---- Ogni riga successiva -------------------
%\vspace*{-\versesep}
%\[G] \[C]  \[D]	

%---- Ev Indicazioni -------------------------			
%\textnote{\textit{(Oppure tutta la strofa)} }	

\endverse
\fi




%%%%% STROFA
\beginverse		%Oppure \beginverse* se non si vuole il numero di fianco
\memorize 		% <<< DECOMMENTA se si vuole utilizzarne la funzione
%\chordsoff		& <<< DECOMMENTA se vuoi una strofa senza accordi

\[D]Come il vento in\[A]frange il mare a \[B-]riva \[B-]
\[G]così il tempo \[D]agita il mio cu\[A]ore \[A]
\[D]cerca il senso \[A]della sua esis\[B-]tenza \[B-]
\[G]cerca il volto \[D]mite del Si\[A]gnore. \[A]
Ed ho cer\[G]cato \echo{ed ho cercato}
per ogni \[D]via \echo{per ogni via}
su ogni \[F#-]vetta i miei \[G]piedi
han cammi\[A]nato
e nel do\[G]lore \echo{e nel dolore} 
mi son pie\[D]gato \echo{mi son piegato}
nella fa\[F#-]tica il tuo \[G]nome io ne\[A]gai
ma \[B-]poi ...

\endverse



%%%%% STROFA
\beginverse*		%Oppure \beginverse* se non si vuole il numero di fianco
%\memorize 		% <<< DECOMMENTA se si vuole utilizzarne la funzione
%\chordsoff		% <<< DECOMMENTA se vuoi una strofa senza accordi

^Ancora ho prepa^rato le mie ^cose ^
^pronto per un vi^aggio che ver^rà ^
^affidando al ^sonno della ^notte ^
^sogni di un in^contro che sa^rà ^
finché una ^voce \echo{finché una voce} 
mi ha des^tato \echo{mi ha destato}
finché il mio ^nome nel ^vento ha risuo^nato
è come un ^fuoco \echo{è come un fuoco}
che incendia il ^cuore \echo{che incendia il cuore}
un fuoco ^che caldo ^si libere^rà.

\endverse



%%%%% RITORNELLO
\beginchorus
\textnote{\textbf{Rit.}}

Accendi la \[D]vita che Dio ti \[G]dà
brucia d’a\[F#-]mo\[B-]re \[A] non perderti \[D]mai \[A]
accendi la \[D]vita perché ora \[G]sai
che il nostro \[F#-]viag\[B-]gio \[A] porta al Si\[D]gnor.

\endchorus



%%%%% STROFA
\beginverse		%Oppure \beginverse* se non si vuole il numero di fianco
%\memorize 		% <<< DECOMMENTA se si vuole utilizzarne la funzione
%\chordsoff		% <<< DECOMMENTA se vuoi una strofa senza accordi

^Come il sole ^dona il suo ca^lore ^
^tu o Signore ^doni veri^tà ^
^luce di una f^iamma senza ^fine ^
^alba di una ^nuova umani^tà. ^
Ed ho ascol^tato \echo{Ed ho ascoltato}
le tue pa^role \echo{le tue parole}
mi son nu^trito di ^nuovo del tuo a^more
ho aperto gli ^occhi \echo{ho aperto gli occhi}
alla mia ^gente \echo{alla mia gente}
con te vi^cino la ^vita esplode^rà.

\endverse


%%%%% RITORNELLO
\beginchorus
\textnote{\textbf{Rit.}}

Accendi la \[D]vita ... \quad \quad \rep{3}

\endchorus



\endsong
%------------------------------------------------------------
%			FINE CANZONE
%------------------------------------------------------------


%-------------------------------------------------------------
%			INIZIO	CANZONE
%-------------------------------------------------------------


%titolo: 	Acqua siamo noi
%autore: 	Cento
%tonalita: 	Re



%%%%%% TITOLO E IMPOSTAZONI
\beginsong{Acqua siamo noi}[by={Cento}] 	% <<< MODIFICA TITOLO E AUTORE
\transpose{0} 						% <<< TRASPOSIZIONE #TONI (0 nullo)
\momenti{Ingresso}							% <<< INSERISCI MOMENTI	
\ifchorded
	%\textnote{Tonalità originale }	% <<< EV COMMENTI (tonalità originale/migliore)
\fi


%%%%%% INTRODUZIONE
\ifchorded
\vspace*{\versesep}
\textnote{Intro: \qquad \qquad } % << MODIFICA IL TEMPO
\vspace*{-\versesep}
\beginverse*

\nolyrics

%---- Prima riga -----------------------------
\vspace*{-\versesep}
 \[D] \[G]\[A4]\[A]

%---- Ogni riga successiva -------------------
\vspace*{-\versesep}
\[G] \[C]  \[D]	

%---- Ogni riga successiva -------------------
\vspace*{-\versesep}
\[G] \[C]  \[D]	

%---- Ev Indicazioni -------------------------			
\textnote{\textit{(Oppure tutta la strofa)} }	

\endverse
\fi








\beginverse
\memorize
|\[D]Acqua \[A]siamo \[D]noi, \brk dall'an|\[G6]tica sor\[A]gente ve\[D]niamo,
\[D]fiumi \[A]siamo \[D]noi \brk se i ru\[G6]scelli si \[A]mettono in\[D]sieme,
\[D]mari \[G]siamo \[D]noi \brk se i tor\[G6]renti si \[A]danno la \[D]mano,
\[D]vita nuova \[B-7]c'è \brk se Ge\[G]sù è in \[A]mezzo a \[D]noi.
\endverse
\textbf{Rit.}


\beginchorus
E allora \[F#-7]diamoci la \[B-7]mano
e tutti in\[F#-7]sieme cammi\[G7+]niamo
ed un o\[F#-7]ceano di \[B-7]pace nasce\[A4]rà. \[A]
E l'ego\[E-]ismo cancel\[C7+]liamo
e un cuore \[E-]limpido sen\[C]tiamo
è Dio che \[C7+]bagna del suo a\[A7]mor l'umani\[D]tà. \[A4]\[A]
\endchorus




\beginverse
^Su nel ^cielo ^c'è \brk Dio ^Padre che ^vive per ^l'uomo
^crea ^tutti ^noi \brk e ci ^ama di a^more infi^nito,
^figli ^siamo ^noi \brk e fra^telli di ^Cristo Si^gnore,
^vita nuova ^c'è \brk quando ^Lui è in ^mezzo a ^noi.
\endverse
\textbf{Rit.}


\beginverse
%\chordsoff
^Nuova u^mani^tà \brk oggi ^nasce da ^chi crede in ^Lui,
^nuovi ^siamo ^noi \brk se l'a^more è la ^legge di ^vita,
^figli ^siamo ^noi \brk se non ^siamo di^visi da ^niente,
^vita eterna ^c'è \brk quando ^Lui è ^dentro ^noi.
\endverse
\textbf{Rit.}

\endsong

%------------------------------------------------------------
%			FINE CANZONE
%------------------------------------------------------------




%++++++++++++++++++++++++++++++++++++++++++++++++++++++++++++
%			CANZONE TRASPOSTA
%++++++++++++++++++++++++++++++++++++++++++++++++++++++++++++
\ifchorded
%decremento contatore per avere stesso numero
\addtocounter{songnum}{-1} 
\beginsong{Acqua siamo noi }[by={Cento}] 	% <<< COPIA TITOLO E AUTORE
\transpose{-2} 						% <<< TRASPOSIZIONE #TONI + - (0 nullo)
\ifchorded
	\textnote{Tonalità più figa }	% <<< EV COMMENTI (tonalità originale/migliore)
\fi


%%%%%% INTRODUZIONE
\ifchorded
\vspace*{\versesep}
\textnote{Intro: \qquad \qquad  } % << MODIFICA IL TEMPO
\vspace*{-\versesep}
\beginverse*

\nolyrics

%---- Prima riga -----------------------------
\vspace*{-\versesep}
 \[D] \[G]\[A4]\[A]

%---- Ogni riga successiva -------------------
\vspace*{-\versesep}
\[G] \[C]  \[D]	

%---- Ogni riga successiva -------------------
\vspace*{-\versesep}
\[G] \[C]  \[D]	

%---- Ev Indicazioni -------------------------			
\textnote{\textit{(Oppure tutta la strofa)} }	

\endverse
\fi








\beginverse
\memorize
|\[D]Acqua \[A]siamo \[D]noi, \brk dall'an|\[G6]tica sor\[A]gente ve\[D]niamo,
\[D]fiumi \[A]siamo \[D]noi \brk se i ru\[G6]scelli si \[A]mettono in\[D]sieme,
\[D]mari \[G]siamo \[D]noi \brk se i tor\[G6]renti si \[A]danno la \[D]mano,
\[D]vita nuova \[B-7]c'è \brk se Ge\[G]sù è in \[A]mezzo a \[D]noi.
\endverse
\textbf{Rit.}


\beginchorus
E allora \[F#-7]diamoci la \[B-7]mano
e tutti in\[F#-7]sieme cammi\[G7+]niamo
ed un o\[F#-7]ceano di \[B-7]pace nasce\[A4]rà. \[A]
E l'ego\[E-]ismo cancel\[C7+]liamo
e un cuore \[E-]limpido sen\[C]tiamo
è Dio che \[C7+]bagna del suo a\[A7]mor l'umani\[D]tà. \[A4]\[A]
\endchorus




\beginverse
^Su nel ^cielo ^c'è \brk Dio ^Padre che ^vive per ^l'uomo
^crea ^tutti ^noi \brk e ci ^ama di a^more infi^nito,
^figli ^siamo ^noi \brk e fra^telli di ^Cristo Si^gnore,
^vita nuova ^c'è \brk quando ^Lui è in ^mezzo a ^noi.
\endverse
\textbf{Rit.}


\beginverse
%\chordsoff
^Nuova u^mani^tà \brk oggi ^nasce da ^chi crede in ^Lui,
^nuovi ^siamo ^noi \brk se l'a^more è la ^legge di ^vita,
^figli ^siamo ^noi \brk se non ^siamo di^visi da ^niente,
^vita eterna ^c'è \brk quando ^Lui è ^dentro ^noi.
\endverse
\textbf{Rit.}

\endsong

\fi
%++++++++++++++++++++++++++++++++++++++++++++++++++++++++++++
%			FINE CANZONE TRASPOSTA
%++++++++++++++++++++++++++++++++++++++++++++++++++++++++++++




%-------------------------------------------------------------
%			INIZIO	CANZONE
%-------------------------------------------------------------


%titolo: 	Alla tua presenza
%autore: 	F. Tiddia
%tonalita: 	Do



%%%%%% TITOLO E IMPOSTAZONI
\beginsong{Alla tua presenza}[by={F. Tiddia}] 	% <<< MODIFICA TITOLO E AUTORE
\transpose{0} 						% <<< TRASPOSIZIONE #TONI (0 nullo)
%\preferflats  %SE VOGLIO FORZARE i bemolle come alterazioni
%\prefersharps %SE VOGLIO FORZARE i # come alterazioni
\momenti{}							% <<< INSERISCI MOMENTI	
% momenti vanno separati da ; e vanno scelti tra:
% Ingresso; Atto penitenziale; Acclamazione al Vangelo; Dopo il Vangelo; Offertorio; Comunione; Ringraziamento; Fine; Santi; Pasqua; Avvento; Natale; Quaresima; Canti Mariani; Battesimo; Prima Comunione; Cresima; Matrimonio; Meditazione; Spezzare del pane;
\ifchorded
	%\textnote{$\bigstar$ Tonalità migliore }	% <<< EV COMMENTI (tonalità originale/migliore)
\fi


%%%%%% INTRODUZIONE
\ifchorded
\vspace*{\versesep}
\musicnote{
\begin{minipage}{0.48\textwidth}
\textbf{Intro}
\hfill 
%( \eighthnote \, 80)   % <<  MODIFICA IL TEMPO
% Metronomo: \eighthnote (ottavo) \quarternote (quarto) \halfnote (due quarti)
\end{minipage}
} 	
\vspace*{-\versesep}
\beginverse*

\nolyrics

%---- Prima riga -----------------------------
\vspace*{-\versesep}
\[C] \[A-7] \[D-]\[G4] % \[*D] per indicare le pennate, \rep{2} le ripetizioni

%---- Ogni riga successiva -------------------
%\vspace*{-\versesep}
%\[G] \[C]  \[D]	

%---- Ev Indicazioni -------------------------			
%\textnote{\textit{[oppure tutta la strofa]} }	

\endverse
\fi




%%%%% STROFA
\beginverse		%Oppure \beginverse* se non si vuole il numero di fianco
\memorize 		% <<< DECOMMENTA se si vuole utilizzarne la funzione
%\chordsoff		% <<< DECOMMENTA se vuoi una strofa senza accordi

\[C]Alla tua presenza \[A-7]portaci Signore, 
nei tuoi \[D-]atri noi vogliamo dimo\[G4]rar. 
\[C]Nel tuo tempio intoneremo \[A-7]inni a te, 
\[D-7]canti di lode alla tua \[G4]maestà. 

\endverse




%%%%% RITORNELLO
\beginchorus
\textnote{\textbf{Rit.}}

\[F]Il tuo Santo \[G]Spirito \brk ci \[E-]guidi là \[A-]dove sei tu, 
\[D-7]alla tua pre\[G]senza Si\[C]gnore Ge\[E&7]su-\[C]ù. 
\[F]In eterno \[G]canteremo \[E-7]gloria a te, Si\[A-7]gnor. 
\[D-7]Alla tua pre\[G]senza, \[D-7]alla tua pre\[G]senza 
per \[D-7]sempre insie\[G]me a te Ge\[C]sù.

\endchorus



\endsong
%------------------------------------------------------------
%			FINE CANZONE
%------------------------------------------------------------


%titolo{Amare questa vita}
%autore{Meregalli}
%album{Mi hai tenuto per mano}
%tonalita{Fa}
%famiglia{Liturgica}
%gruppo{}
%momenti{}
%identificatore{amare_questa_vita}
%data_revisione{2011_12_31}
%trascrittore{Francesco Endrici}
\beginsong{Amare questa vita}[by={Meregalli}]
\beginverse
\[F]Erano \[B&]uomini \[C]senza pa\[7]ura,
di \[D-]solcare il \[7]mare pen\[A-]sando alla \[7]riva,
\[B&]barche sotto il \[C]cielo, \[F] tra montagne e si\[C]lenzio,
\[B&]davano le \[C]reti al \[F]ma\[D-]re, \brk \[B&]vita dalle \[G-]mani di \[C4]Dio \[C]
\endverse
\beginverse
\chordsoff
^Venne nell'^ora più ^lenta del ^giorno,
^quando le ^reti si ^sdraiano a ^riva,
^l'aria senza ^vento, ^ si riempì di una ^voce,
^mani cari^che di ^sa^le, ^sale nelle ^mani di ^Dio. ^
\endverse
\beginchorus
Lo se\[F]guimmo fi\[G-7]dandoci degli \[F]occhi \[7]
gli cre\[B&]demmo a\[A7]mando le pa\[D-]role. \[D7]
Fu il \[G-]sole caldo a \[C7]riva
o fu il \[F]vento sulla \[C]vela
o il \[B&]gusto e la fa\[F]tica di ri\[G-7]schiare
e accettare quella \[C]sfida.
\endchorus
\beginverse
^Prima che un ^sole più ^alto vi ^insidi, 
^prima che il ^giorno vi ^lasci de^lusi,
^riprendete il ^largo ^ e gettate le ^reti, 
^barche cari^che di ^pe^sci, \brk ^vita dalle ^mani di ^Dio. ^
\endverse
\beginchorus
Lo se\[F]guimmo fi\[G-7]dandoci degli \[F]occhi \[7]
gli cre\[B&]demmo a\[A7]mando le pa\[D-]role. \[D7]
Lui \[G-]voce lui no\[C7]tizia
lui \[F]strada e lui sua \[C]meta
lui \[B&]gioia impreve\[F]dibile e sin\[G-7]cera
di amare questa \[C]vita.
\endchorus
\beginverse
^Erano ^uomini ^senza pa^ura
di s^olcare il ^mare pen^sando alla ^riva,
^anche quella ^sera, ^ senza dire pa^role,
^misero le ^barche in ^ma^re, ^vita dalle ^mani di ^Dio, ^
\[B&]misero le \[C]barche in \[F]ma\[D-]re, \brk \[B&]vita dalle \[C]mani di \[F]Dio.
\endverse
\endsong


%-------------------------------------------------------------
%			INIZIO	CANZONE
%-------------------------------------------------------------


%titolo: 	Andate per le strade
%autore: 	Roncari, Capello
%tonalita: 	Si-



%%%%%% TITOLO E IMPOSTAZONI
\beginsong{Andate per le strade}[by={Roncari, Capello}] 	% <<< MODIFICA TITOLO E AUTORE
\transpose{0} 						% <<< TRASPOSIZIONE #TONI (0 nullo)
\momenti{Fine}							% <<< INSERISCI MOMENTI	
% momenti vanno separati da ; e vanno scelti tra:
% Ingresso; Atto penitenziale; Acclamazione al Vangelo; Dopo il Vangelo; Offertorio; Comunione; Ringraziamento; Fine; Santi; Pasqua; Avvento; Natale; Quaresima; Canti Mariani; Battesimo; Prima Comunione; Cresima; Matrimonio; Meditazione;
\ifchorded
	%\textnote{Tonalità originale }	% <<< EV COMMENTI (tonalità originale/migliore)
\fi


%%%%%% INTRODUZIONE
\ifchorded
\vspace*{\versesep}
\textnote{Intro: \qquad \qquad  }%(\eighthnote 116) % << MODIFICA IL TEMPO
% Metronomo: \eighthnote (ottavo) \quarternote (quarto) \halfnote (due quarti)
\vspace*{-\versesep}
\beginverse*

\nolyrics

%---- Prima riga -----------------------------
\vspace*{-\versesep}
\[B-] \[A] \[*B-]	 % \[*D] per indicare le pennate, \rep{2} le ripetizioni

%---- Ogni riga successiva -------------------
%\vspace*{-\versesep}
%\[G] \[C]  \[D]	

%---- Ev Indicazioni -------------------------			
%\textnote{\textit{(Oppure tutta la strofa)} }

\endverse
\fi

%%%%% RITORNELLO
\beginchorus
\textnote{\textbf{Rit.}}

An\[(B-)]date per le \[D]strade in \[G]tutto il \[A]mondo,
chia\[F#]mate i miei a\[B-]mici \[A]per far \[D]festa,
c'è un \[F#]posto per cia\[G]scuno \[A]alla mia \[B-]mensa.  \[*B-]

\endchorus

%%%%% STROFA
\beginverse		%Oppure \beginverse* se non si vuole il numero di fianco
\memorize 		% <<< DECOMMENTA se si vuole utilizzarne la funzione
%\chordsoff		& <<< DECOMMENTA se vuoi una strofa senza accordi

Nel \[(D)]vostro cam\[G]mino annun\[A]ciate il Van\[D]gelo 
di\[B-]cendo “È vi\[E-]cino il \[F#]Regno dei \[B-]cieli!”.
Gua\[D7]rite i ma\[G]lati, mon\[A]date i leb\[D]brosi,
ren\[B-]dete la \[F#-]vita a \[C#]chi l'ha per\[F#]duta. \[F#] \quad \[*B-] 

\endverse

\beginverse
%\chordsoff
^Vi è stato do^nato con a^more gra^tuito: 
u^gualmente do^nate con ^gioia e per a^more. 
Con ^voi non pren^dete ^né oro né ar^gento, 
per^ché l'ope^raio ha di^ritto al suo ^cibo. ^ \quad ^
\endverse
\beginverse
%\chordsoff
En^trando in una ^casa do^natele la ^pace: 
se ^c'è chi vi ri^fiuta e ^non accoglie il ^dono, 
la ^pace torni a ^voi e u^scite dalla ^casa, 
^scuotendo la ^polvere dai ^vostri cal^zari.  ^ \quad ^
\endverse
\beginverse
%\chordsoff
^Ecco, io vi ^mando a^gnelli in mezzo ai ^lupi: 
siate ^dunque avve^duti come ^sono i ser^penti, 
ma ^liberi e ^chiari co^me le co^lombe; 
do^vrete soppor^tare pri^gioni e tribu^nali. ^ \quad ^
\endverse
\beginverse
%\chordsoff
Nes^suno è più ^grande del ^proprio ma^estro, 
né il ^servo è più im^portante ^del suo pa^drone. 
Se ^hanno odiato ^me, odie^ranno anche ^voi; 
ma ^voi non te^mete, io non ^vi lascio ^soli! ^ \quad ^
\endverse

\endsong
%------------------------------------------------------------
%			FINE CANZONE
%------------------------------------------------------------
%-------------------------------------------------------------
%			INIZIO	CANZONE
%-------------------------------------------------------------


%titolo: 	Andiamo ed annunciamo
%autore: 	V. Di Mauro
%tonalita: 	Re 



%%%%%% TITOLO E IMPOSTAZONI
\beginsong{Andiamo ed annunciamo}[by={V. Di Mauro}] 	% <<< MODIFICA TITOLO E AUTORE
\transpose{0} 						% <<< TRASPOSIZIONE #TONI (0 nullo)
\momenti{Fine; Cresima}							% <<< INSERISCI MOMENTI	
% momenti vanno separati da ; e vanno scelti tra:
% Ingresso; Atto penitenziale; Acclamazione al Vangelo; Dopo il Vangelo; Offertorio; Comunione; Ringraziamento; Fine; Santi; Pasqua; Avvento; Natale; Quaresima; Canti Mariani; Battesimo; Prima Comunione; Cresima; Matrimonio; Meditazione; Spezzare del pane;
\ifchorded
	%\textnote{Tonalità migliore }	% <<< EV COMMENTI (tonalità originale/migliore)
\fi

%%%%%% INTRODUZIONE
\ifchorded
\vspace*{\versesep}
\textnote{Intro: \qquad \qquad  }%(\eighthnote 116) % <<  MODIFICA IL TEMPO
% Metronomo: \eighthnote (ottavo) \quarternote (quarto) \halfnote (due quarti)
\vspace*{-\versesep}
\beginverse*

\nolyrics

%---- Prima riga -----------------------------
\vspace*{-\versesep}
\[D] \[D] \[B-] 	 % \[*D] per indicare le pennate, \rep{2} le ripetizioni

%---- Ogni riga successiva -------------------
\vspace*{-\versesep}
\[G] \[A] \[D]

%---- Ev Indicazioni -------------------------			
\textnote{\textit{(Come la prima metà del ritornello)} }	

\endverse
\fi

%%%%% RITORNELLO
\beginchorus
\textnote{\textbf{Rit.}}

Andi\[D]amo e annunciamo a tutto il \[B-]mondo
che il Si\[G]gnore è ve\[A]nuto in mezzo a \[D]noi
e se ami\[G]amo come \[A]Lui ci \[D*]ha a\[F#-*]ma\[B-]to,
ogni \[G]giorno con \[A]noi cammine\[D*]rà. \[G*] \[D]

\endchorus

%%%%% STROFA
\beginverse		%Oppure \beginverse* se non si vuole il numero di fianco
\memorize 		% <<< DECOMMENTA se si vuole utilizzarne la funzione
%\chordsoff		% <<< DECOMMENTA se vuoi una strofa senza accordi

La \[D]terra percor\[B-]rete inseg\[E-]nando ad ogni \[A7]uomo
il \[D]mio comanda\[B-]mento, la \[E-]legge dell’A\[A7]more:
par\[D]late con la \[A]vita e \[G]non con le pa\[A]role;
chi \[D]vede il vostro a\[B-]more da \[E-]solo capi\[A7]rà.

\endverse

%%%%% STROFA
\beginverse		%Oppure \beginverse* se non si vuole il numero di fianco
%\memorize 		% <<< DECOMMENTA se si vuole utilizzarne la funzione
\chordsoff		% <<< DECOMMENTA se vuoi una strofa senza accordi

Com^prenderà chi ^sbaglia e ^chi non spera ^più,
a ^Me ritorne^rà se ^voi l’accoglie^rete
nel ^nome di quel ^Padre, che at^tende e poi per^dona,
del ^Figlio e dello ^Spirito, che a ^voi è stato ^dato!

\endverse

%%%%% STROFA
\beginverse		%Oppure \beginverse* se non si vuole il numero di fianco
%\memorize 		% <<< DECOMMENTA se si vuole utilizzarne la funzione
\chordsoff		% <<< DECOMMENTA se vuoi una strofa senza accordi

Se ^lungo e fati^coso vi ^sembrerà il ca^mino
che agli altri vi con^duce per ^far conoscer ^Me,
non ^rallentate il ^passo, ma ^ricordate ^sempre
che ^ovunque in ogni ^strada, con ^voi cammine^rò!

\endverse

\endsong
%------------------------------------------------------------
%			FINE CANZONE
%------------------------------------------------------------
%-------------------------------------------------------------
%			INIZIO	CANZONE
%-------------------------------------------------------------


%titolo: 	Angelo di Dio
%autore: 	G. Mezzalira
%tonalita: 	Mib



%%%%%% TITOLO E IMPOSTAZONI
\beginsong{Angelo di Dio}[by={G. Mezzalira}]
\transpose{-3} 						% <<< TRASPOSIZIONE #TONI (0 nullo)
%\preferflats  %SE VOGLIO FORZARE i bemolle come alterazioni
%\prefersharps %SE VOGLIO FORZARE i # come alterazioni
\momenti{}							% <<< INSERISCI MOMENTI	
% momenti vanno separati da ; e vanno scelti tra:
% Ingresso; Atto penitenziale; Acclamazione al Vangelo; Dopo il Vangelo; Offertorio; Comunione; Ringraziamento; Fine; Santi; Pasqua; Avvento; Natale; Quaresima; Canti Mariani; Battesimo; Prima Comunione; Cresima; Matrimonio; Meditazione; Spezzare del pane;
\ifchorded
	%\textnote{Tonalità migliore }	% <<< EV COMMENTI (tonalità originale\migliore)
\fi


%%%%%% INTRODUZIONE
\ifchorded
\vspace*{\versesep}
\musicnote{
\begin{minipage}{0.48\textwidth}
\textbf{Intro}
\hfill 
%( \eighthnote \, 80)   % <<  MODIFICA IL TEMPO
% Metronomo: \eighthnote (ottavo) \quarternote (quarto) \halfnote (due quarti)
\end{minipage}
} 	
\vspace*{-\versesep}
\beginverse*

\nolyrics

%---- Prima riga -----------------------------
\vspace*{-\versesep}
\[E&] \[A&] \[B&7]	\[E&] % \[*D] per indicare le pennate, \rep{2} le ripetizioni

%---- Ogni riga successiva -------------------
%\vspace*{-\versesep}
%\[G] \[C]  \[D]	

%---- Ev Indicazioni -------------------------			
%\textnote{\textit{(Oppure tutta la strofa)} }	

\endverse
\fi


%%%%% RITORNELLO
\beginchorus
\textnote{\textbf{Rit.}}
\[E&]Angelo di \[A&]Dio, \brk che \[B&7]sei il mio cus\[G-]tode,
il\[C-]lumina custo\[F7]disci \brk \[B&]reggi-e governa \[E&]me 
che ti \[A&]fui affi\[Ddim/B&]dato 
\[G-]dalla pietà ce\[C-]leste 
\[E&]A-a-am\[A&]en \[C7*] \[F-]A-a-\[B&*]a\[E&]men 
\endchorus




%%%%% STROFA
\beginverse		%Oppure \beginverse* se non si vuole il numero di fianco
\memorize 		% <<< DECOMMENTA se si vuole utilizzarne la funzione
%\chordsoff		% <<< DECOMMENTA se vuoi una strofa senza accordi

\[C]Guida i miei \[F]pensieri, \[D-]segui i miei \[G]passi,
\[C]dammi la tua \[A-]forza per \[G]restare nella \[C]luce
\[C]la gioia \[F]incontrerò \[G]la pace \[C]porterò
\[A-]Angelo \[F]Santo \[C]veglia \[G]su di \[C]me.


\endverse





%%%%% STROFA
\beginverse		%Oppure \beginverse* se non si vuole il numero di fianco
%\memorize 		% <<< DECOMMENTA se si vuole utilizzarne la funzione
%\chordsoff		% <<< DECOMMENTA se vuoi una strofa senza accordi
^Guida la mia ^vita, ^segui il mio la^voro,
^dammi la tua ^forza per dif^fondere l’a^more
^Gesù io ^annunzierò, ^paura ^non avrò 
^Angelo ^Santo ^veglia ^su di ^me.
\endverse





\endsong
%------------------------------------------------------------
%			FINE CANZONE
%------------------------------------------------------------




%-------------------------------------------------------------
%			INIZIO	CANZONE
%-------------------------------------------------------------


%titolo: 	Annunceremo che tu
%autore: 	Auricchio
%tonalita: 	Do 


%%%%%% TITOLO E IMPOSTAZONI
\beginsong{Annunceremo che Tu}[by={Auricchio}]
\transpose{-2}
\momenti{Ingresso; Fine}							% <<< INSERISCI MOMENTI	
% momenti vanno separati da ; e vanno scelti tra:
% Ingresso; Atto penitenziale; Acclamazione al Vangelo; Dopo il Vangelo; Offertorio; Comunione; Ringraziamento; Fine; Santi; Pasqua; Avvento; Natale; Quaresima; Canti Mariani; Battesimo; Prima Comunione; Cresima; Matrimonio; Meditazione;
\ifchorded
	%\textnote{Tonalità originale }	% <<< EV COMMENTI (tonalità originale/migliore)
\fi




%%%%%% INTRODUZIONE
\ifchorded
\vspace*{\versesep}
\textnote{Intro: \qquad \qquad  }%(\eighthnote 116) % << MODIFICA IL TEMPO
% Metronomo: \eighthnote (ottavo) \quarternote (quarto) \halfnote (due quarti)
\vspace*{-\versesep}
\beginverse*

\nolyrics

%---- Prima riga -----------------------------
\vspace*{-\versesep}
\[G] \[D]  \[G]	 \[D]% \[*D] per indicare le pennate, \rep{2} le ripetizioni

%---- Ogni riga successiva -------------------
\vspace*{-\versesep}
\[A-] \[C]  \[D]	

%---- Ev Indicazioni -------------------------			
\textnote{\textit{(Come il ritornello)} }	

\endverse
\fi




%%%%% RITORNELLO
\beginchorus
\textnote{\textbf{Rit.}}
\[(D7)] Annunceremo che \[G]Tu sei Veri\[D]tà, 
lo grideremo dai \[G]tetti della nostra cit\[D]tà, 
senza paura \[A-]anche tu \[C] lo puoi can\[D]tare.
\endchorus




%%%%% STROFA
\beginverse
\memorize
\[G] E non te\[C]mere dai,  
\[G] che non ci \[D]vuole poi tanto, 
\[G] quello che \[C]non si sa  
\[G] non reste\[D]rà nascosto. 
\[B-] Se ti parlo nel \[C]buio, lo dirai nella \[D]luce,
ogni giorno è il mo\[C]mento di credere in \[D]me.
\endverse





%%%%% STROFA
\beginverse
%\memorize 		% <<< DECOMMENTA se si vuole utilizzarne la funzione
%\chordsoff		& <<< DECOMMENTA se vuoi una strofa senza accordi
^ Con il co^raggio tu  
^ porterai la Pa^rola che salva,  
^ anche se ^ci sarà 
^ chi non vuole ac^cogliere il dono, 
^ tu non devi fer^marti, ma continua a lot^tare, 
il mio Spirito ^sempre ti accompagne^rà.
\endverse





%%%%% STROFA
\beginverse
%\memorize 		% <<< DECOMMENTA se si vuole utilizzarne la funzione
%\chordsoff		& <<< DECOMMENTA se vuoi una strofa senza accordi
^ Non ti abban^dono mai,  
^ io sono il ^Dio fedele, 
^ conosco il ^cuore tuo. 
^ Ogni tuo pen^siero mi è noto,  
^ la tua vita è pre^ziosa, vale più di ogni ^cosa,
il segno più ^grande del mio amore per ^te.
\endverse

\endsong
%------------------------------------------------------------
%			FINE CANZONE
%------------------------------------------------------------
%-------------------------------------------------------------
%			INIZIO	CANZONE
%-------------------------------------------------------------


%titolo: 	Antica eterna danza
%autore: 	Gen Verde
%tonalita: 	Sol 



%%%%%% TITOLO E IMPOSTAZONI
\beginsong{Antica eterna danza}[by={Gen Verde}] 	% <<< MODIFICA TITOLO E AUTORE
\transpose{0} 						% <<< TRASPOSIZIONE #TONI (0 nullo)
\momenti{Offertorio; Prima Comunione}							% <<< INSERISCI MOMENTI	
% momenti vanno separati da ; e vanno scelti tra:
% Ingresso; Atto penitenziale; Acclamazione al Vangelo; Dopo il Vangelo; Offertorio; Comunione; Ringraziamento; Fine; Santi; Pasqua; Avvento; Natale; Quaresima; Canti Mariani; Battesimo; Prima Comunione; Cresima; Matrimonio; Meditazione;
\ifchorded
	%\textnote{Tonalità originale }	% <<< EV COMMENTI (tonalità originale/migliore)
\fi





%%%%%% INTRODUZIONE
\ifchorded
\vspace*{\versesep}
\textnote{Intro: \qquad \qquad  }%(\eighthnote 116) % << MODIFICA IL TEMPO
% Metronomo: \eighthnote (ottavo) \quarternote (quarto) \halfnote (due quarti)
\vspace*{-\versesep}
\beginverse*

\nolyrics

%---- Prima riga -----------------------------
\vspace*{-\versesep}
\[*G] \[*D]  \[*C]	 % \[*D] per indicare le pennate, \rep{2} le ripetizioni

%---- Ogni riga successiva -------------------
\vspace*{-\versesep}
\[G] \[C]  \[G]	

%---- Ev Indicazioni -------------------------			
%\textnote{\textit{(Oppure tutta la strofa)} }	

\endverse
\fi



%%%%% STROFA
\beginverse
\memorize

\[G]Spighe \[D]d'oro al \[E-]vento, 
an\[E-]tica, e\[D]terna \[C]danza
per \[A-]fare un \[D]solo \[E-]pa\[D]ne
spez\[C]zato \[A-]sulla m\[B]ensa.
\[G]Grappoli \[D]dei \[E-]colli, 
pro\[E-]fumo \[D]di le\[C]tizia
per \[A-]fare un \[D]solo \[E-]vi\[D]no 
be\[A-]vanda \[B-]della \[E-]grazia.  

\endverse



%%%%%% EV. INTERMEZZO
\beginverse*
\vspace*{1.3\versesep}
{
	\nolyrics
	
	\ifchorded
	\textnote{Intermezzo strumentale}

		
	%---- Prima riga -----------------------------
	\vspace*{-\versesep}
	\[*G] \[*D]  \[*C]	 % \[*D] per indicare le pennate, \rep{2} le ripetizioni

	%---- Ogni riga successiva -------------------
	\vspace*{-\versesep}
	\[G] \[C]  \[G] 


	\fi
	%---- Ev Indicazioni -------------------------			
	%\textnote{\textit{(ripetizione della strofa)}} 
	 
}
\vspace*{\versesep}
\endverse




%%%%% STROFA
\beginverse

^Con il pa^ne e il vi^no 
Si^gnore ^ti doni^amo
le ^nostre gi^oie ^pu^re, 
le at^tese e ^le pa^ure.
^Frutti ^del la^voro 
e ^fede n^el fu^turo,
la ^voglia ^di cam^bia^re 
e ^di ri^cominci^are.  

\endverse



%%%%%% EV. INTERMEZZO
\beginverse*
\vspace*{1.3\versesep}
{
	\nolyrics
	
	\ifchorded
	\textnote{Intermezzo strumentale}

		
	%---- Prima riga -----------------------------
	\vspace*{-\versesep}
	\[*G] \[*D]  \[*C]	 % \[*D] per indicare le pennate, \rep{2} le ripetizioni

	%---- Ogni riga successiva -------------------
	\vspace*{-\versesep}
	\[G] \[C]  \[G]	


	\fi
	%---- Ev Indicazioni -------------------------			
	%\textnote{\textit{(ripetizione della strofa)}} 
	 
}
\vspace*{\versesep}
\endverse




%%%%% STROFA
\beginverse

^Dio del^la spe^ranza, 
sor^gente ^d'ogni ^dono
ac^cogli q^uesta o^ffer^ta 
che in^sieme ^Ti porti^amo.
^Dio dell'^uni^verso 
rac^cogli ^chi è dis^perso
e ^facci ^tutti ^Chie^sa, 

\vspace*{1.3\versesep}
\textnote{\textit{(rallentando)}}
u^na ^cosa in ^Te.
\endverse
\endsong


%titolo{Apri le tue braccia}
%autore{Machetta}
%album{Una voce che ti cerca}
%tonalita{Re-}
%famiglia{Liturgica}
%gruppo{}
%momenti{Ingresso;Quaresima;Riconciliazione;Conversione}
%identificatore{apri_le_tue_braccia}
%data_revisione{2011_12_31}
%trascrittore{Francesco Endrici}
\beginsong{Apri le tue braccia}[by={Machetta}]
\beginverse
Hai cer\[D-]cato la \[C]libertà lon\[D-]tano,
hai tro\[A-]vato la \[E&]noia e le ca\[B&]tene;
hai va\[G-]ga\[D-]to \[G-]senza \[D-]via, \[G-]solo, \[E&] con la tua \[C]fame.
\endverse
\beginchorus
\[F]A\[C]pri le tue \[D-]brac\[A-]cia, \[B&]corri in\[G-]contro al \[C]Padre;
\[F]oggi \[D] la sua \[G-]ca\[B&]sa sarà in \[F]fe\[C7]sta per \[F]te.
\endchorus
\beginverse
\chordsoff
Se vor^rai spez^zare le ca^tene
trove^rai la ^strada dell'a^more;
la tua ^gio^ia ^cante^rai: ^questa ^ è liber^tà.
\endverse
\beginverse
\chordsoff
I tuoi ^occhi ri^cercano l'az^zurro;
c'è una ^casa che a^spetta il tuo ri^torno,
e la ^pa^ce ^torne^rà: ^questa ^ è liber^tà.
\endverse
\endsong



%-------------------------------------------------------------
%			INIZIO	CANZONE
%-------------------------------------------------------------


%titolo: 	Ave Maria
%autore: 	Casucci, Balduzzi
%tonalita: 	Re 


%%%%%% TITOLO E IMPOSTAZONI
\beginsong{Ave Maria}[by={C. Casucci, M. Balduzzi}] 	% <<< MODIFICA TITOLO E AUTORE
\transpose{0} 						% <<< TRASPOSIZIONE #TONI (0 nullo)
\momenti{Canti Mariani; Ringraziamento}							% <<< INSERISCI MOMENTI	
% momenti vanno separati da ; e vanno scelti tra:
% Ingresso; Atto penitenziale; Acclamazione al Vangelo; Dopo il Vangelo; Offertorio; Comunione; Ringraziamento; Fine; Santi; Pasqua; Avvento; Natale; Quaresima; Canti Mariani; Battesimo; Prima Comunione; Cresima; Matrimonio; Meditazione;
\ifchorded
	%\textnote{Tonalità originale }	% <<< EV COMMENTI (tonalità originale/migliore)
\fi


%%%%%% INTRODUZIONE
\ifchorded
\vspace*{\versesep}
\musicnote{
\begin{minipage}{0.48\textwidth}
\textbf{Intro}
\hfill 
(\quarternote \, 72)
%( \eighthnote \, 80)   % <<  MODIFICA IL TEMPO
% Metronomo: \eighthnote (ottavo) \quarternote (quarto) \halfnote (due quarti)
\end{minipage}
} 
\musicnote{\textit{(dolce, arpeggiato)}}	
\vspace*{-\versesep}
\beginverse*


\nolyrics

%---- Prima riga -----------------------------
\vspace*{-\versesep}
\[D]\[A]\[B-]\[G]	 % \[*D] per indicare le pennate, \rep{2} le ripetizioni

%---- Ogni riga successiva -------------------
\vspace*{-\versesep}
\[D]\[A]\[E-] \[G]

%---- Ev Indicazioni -------------------------			
\textnote{\textit{(come metà ritornello)} }	

\endverse
\fi




%%%%% RITORNELLO
\textnote{\textbf{Rit.}}
\beginchorus

\[D]A\[A]ve Ma\[B-]ria, \[G] 
\[D]\[A]a\[E-]ve, \[G]
\[D]a\[A]ve Ma\[B-]ria, \[G] 
\[D]\[A]a\[D4]ve. \[D]

\endchorus



%%%%% STROFA
\beginverse
\memorize
\[D]Donna dell'at\[D]tesa e \[B-]madre di spe\[B-]ranza
\[A]ora pro no\[G]bis.
\[D]Donna del sor\[D]riso e \[B-]madre del si\[B-]lenzio
\[A]ora pro no\[G]bis.
\[D]Donna di fron\[D]tiera e \[A]madre dell'ar\[A]dore
\[B-]ora pro no\[G]bis.
\[D]Donna del ri\[D]poso e \[A]madre del sen\[A]tiero
\[G]ora pro no\[A]bis.
\endverse




%%%%% STROFA
\beginverse
^Donna del de^serto e ^madre del re^spiro
^ora pro no^bis.
^Donna della ^sera e ^madre del ri^cordo
^ora pro no^bis.
^Donna del pre^sente e ^madre del ri^torno
^ora pro no^bis.
^Donna della ^terra e ^madre dell'a^more
^ora pro no^bis.
\endverse


\endsong
%------------------------------------------------------------
%			FINE CANZONE
%------------------------------------------------------------




%BBB
%-------------------------------------------------------------
%			INIZIO	CANZONE
%-------------------------------------------------------------


%titolo: 	Beati quelli 
%autore: 	Sequeri
%tonalita: 	Fa >Re



%%%%%% TITOLO E IMPOSTAZONI
\beginsong{Beati quelli}[by={Sequeri}]% <<< MODIFICA TITOLO E AUTORE
\transpose{-3} 						% <<< TRASPOSIZIONE #TONI (0 nullo)
\momenti{Ingresso}							% <<< INSERISCI MOMENTI	
% momenti vanno separati da ; e vanno scelti tra:
% Ingresso; Atto penitenziale; Acclamazione al Vangelo; Dopo il Vangelo; Offertorio; Comunione; Ringraziamento; Fine; Santi; Pasqua; Avvento; Natale; Quaresima; Canti Mariani; Battesimo; Prima Comunione; Cresima; Matrimonio; Meditazione; Spezzare del pane;
\ifchorded
	%\textnote{Tonalità migliore }	% <<< EV COMMENTI (tonalità originale/migliore)
\fi


%%%%%% INTRODUZIONE
\ifchorded
\vspace*{\versesep}
\textnote{Intro: \qquad \qquad  (\eighthnote 116)} % <<  MODIFICA IL TEMPO
% Metronomo: \eighthnote (ottavo) \quarternote (quarto) \halfnote (due quarti)
\vspace*{-\versesep}
\beginverse*

\nolyrics

%---- Prima riga -----------------------------
\vspace*{-\versesep}
\[F]\[B&]\[C7]	 % \[*D] per indicare le pennate, \rep{2} le ripetizioni

%---- Ogni riga successiva -------------------
%\vspace*{-\versesep}
%\[G] \[C]  \[D]	

%---- Ev Indicazioni -------------------------			
%\textnote{\textit{(Oppure tutta la strofa)} }	

\endverse
\fi

\beginverse
\memorize
Beati \[F]quelli che \[B&]poveri \[F]sono, \[A7]
beati \[D-]quanti son \[C7]puri di \[F]cuore. \[C7]
Beati \[F]quelli che \[B&]vivono in \[F]pena \[A7]
nell'at\[D-]tesa d'un \[G-]nuovo mat\[C7]tino.
\endverse



%%%%% RITORNELLO
\beginchorus
\textnote{\textbf{Rit.}}
Saran be\[F]ati, vi \[B&]dico, be\[F*]\[A7*]a\[D-]ti
perché di \[B&]essi è il \[B&-]regno dei \[C7]cieli.
Saran be\[F]ati, vi \[B&]dico, be\[F*]\[A7*]a\[D-]ti
perché di \[B&]essi è il \[G7]regno dei \[F*]\[C9*]cie\[F]li.
\endchorus

\beginverse
Beati ^quelli che ^fanno la ^pace, ^
beati ^quelli che in^segnano l'a^more. ^
Beati ^quelli che ^hanno la ^fame ^
e la ^sete di ^vera giu^stizia.
\endverse


\beginverse
Beati ^quelli che un ^giorno sa^ranno ^
persegui^tati per ^causa di ^Cristo, ^
perché nel ^cuore non ^hanno vio^lenza, ^
ma la ^forza di ^questo Van^gelo.
\endverse

\endsong
%------------------------------------------------------------
%			FINE CANZONE
%------------------------------------------------------------





%-------------------------------------------------------------
%			INIZIO	CANZONE
%-------------------------------------------------------------


%titolo: 	Beato il cuore GMG 2016
%autore: Jakub Blycharz
%tonalita: 	Fa 



%%%%%% TITOLO E IMPOSTAZONI
\beginsong{Beato il cuore }[ by={Inno GMG Cracovia 2016 — J. Blycharz}, ititle={Inno GMG Cracovia 2016}]
\transpose{-2} 						% <<< TRASPOSIZIONE #TONI (0 nullo)
\momenti{Comunione; Fine; Ingresso}							% <<< INSERISCI MOMENTI	
% momenti vanno separati da ; e vanno scelti tra:
% Ingresso; Atto penitenziale; Acclamazione al Vangelo; Dopo il Vangelo; Offertorio; Comunione; Ringraziamento; Fine; Santi; Pasqua; Avvento; Natale; Quaresima; Canti Mariani; Battesimo; Prima Comunione; Cresima; Matrimonio; Meditazione;
\ifchorded
	\textnote{$\bigstar$ Tonalità migliore}	% <<< EV COMMENTI (tonalità originale/migliore)
\fi


%%%%%% INTRODUZIONE
\ifchorded
\vspace*{\versesep}
\musicnote{
\begin{minipage}{0.48\textwidth}
\textbf{Intro}
\hfill 
%( \eighthnote \, 80)   % <<  MODIFICA IL TEMPO
% Metronomo: \eighthnote (ottavo) \quarternote (quarto) \halfnote (due quarti)
\end{minipage}
} 	
\vspace*{-\versesep}
\beginverse*

\nolyrics

%---- Prima riga -----------------------------
\vspace*{-\versesep}
\[C#-]  \[A]	\[E]  % \[*D] per indicare le pennate, \rep{2} le ripetizioni

%---- Ogni riga successiva -------------------
\vspace*{-\versesep}
\[B] \[(B)] \[(B)]  \[C#-]

%---- Ev Indicazioni -------------------------			
%\textnote{\textit{(Oppure tutta la strofa)} }	

\endverse
\fi




%%%%% STROFA
\beginverse		%Oppure \beginverse* se non si vuole il numero di fianco
\memorize 		% <<< DECOMMENTA se si vuole utilizzarne la funzione
%\chordsoff		& <<< DECOMMENTA se vuoi una strofa senza accordi

\[C#-]Sei sceso \[A]dalla tua immensi\[E]tà
\[D]in nostro a\[A]iu\[E]to.
Miseri\[B]cordia  scorre  da \[F#]te
\[A]sopra \[B]tutti \[C#]noi.


\endverse


%%%%% STROFA
\beginverse*	%Oppure \beginverse* se non si vuole il numero di fianco
%\memorize 		% <<< DECOMMENTA se si vuole utilizzarne la funzione
%\chordsoff		& <<< DECOMMENTA se vuoi una strofa senza accordi

^ Persi in un ^mondo d’oscuri^tà
^lì Tu ci ^tro^vi.
Nelle tue ^braccia ci stringi e ^poi
^dai la ^vita per ^noi.


\endverse



%%%%% RITORNELLO
\beginchorus
\textnote{\textbf{Rit.}}

Beato è il \[E]cuo\[B]re che per\[C#-]do\[A]na!
Miseri\[E]cordia riceve\[B]rà da Dio in ci\[F#]elo! \rep{2}

\endchorus



%%%%% STROFA
\beginverse		%Oppure \beginverse* se non si vuole il numero di fianco
%\memorize 		% <<< DECOMMENTA se si vuole utilizzarne la funzione
%\chordsoff		% <<< DECOMMENTA se vuoi una strofa senza accordi

^ Solo il per^dono riporte^rà
^pace nel ^mon^do.
Solo il per^dono ci svele^rà
^come f^igli t^uoi.

\endverse



%%%%% RITORNELLO
\beginchorus
\textnote{\textbf{Rit.}}

Beato è il \[E]cuo\[B]re che per\[C#-]do\[A]na!
Miseri\[E]cordia riceve\[B]rà da Dio in ci\[F#]elo! \rep{2}

\endchorus




%%%%% STROFA
\beginverse		%Oppure \beginverse* se non si vuole il numero di fianco
%\memorize 		% <<< DECOMMENTA se si vuole utilizzarne la funzione
%\chordsoff		% <<< DECOMMENTA se vuoi una strofa senza accordi

^ Col sangue in ^croce hai pagato ^Tu
^le nostre ^pover^tà.
Se noi ci am^iamo e restiamo in^ te
^il mondo ^crede^rà!

\endverse



%%%%% RITORNELLO
\beginchorus
\textnote{\textbf{Rit.}}

Beato è il \[E]cuo\[B]re che per\[C#-]do\[A]na!
Miseri\[E]cordia riceve\[B]rà da Dio in ci\[F#]elo! \rep{2}

\endchorus




%%%%% BRIDGE
\beginverse*		%Oppure \beginverse* se non si vuole il numero di fianco
%\memorize 		% <<< DECOMMENTA se si vuole utilizzarne la funzione
%\chordsoff		% <<< DECOMMENTA se vuoi una strofa senza accordi
\textnote{\textbf{Bridge}}
\[A]Le nostre an\[B]gosce ed ansie\[C#-]tà
get\[A]tiamo ogni \[B]attimo in \[A]te.
Amore \[B]che non abbandona \[C#-]mai,
\[A]vivi in \[B]mezzo a \[C#]noi!

\endverse



%%%%% RITORNELLO
\beginchorus
\textnote{\textbf{Rit.}}

Beato è il \[A]cuo\[E]re che per\[F#-]do\[D]na!
Miseri\[A]cordia riceve\[E]rà da Dio in ci\[B]elo! \rep{4}

\endchorus



%%%%%% EV. INTERMEZZO
\beginverse*
\vspace*{1.3\versesep}
{
	\nolyrics
	\musicnote{Chiusura strumentale}
	
	\ifchorded

	%---- Prima riga -----------------------------
	\vspace*{-\versesep}
	\[C#-]  \[A]	\[E]  % \[*D] per indicare le pennate, \rep{2} le ripetizioni


	%---- Ogni riga successiva -------------------
	\vspace*{-\versesep}
	\[B] \[(B)] \[(B)]  \[C#-]


	\fi
	%---- Ev Indicazioni -------------------------			
	%\textnote{\textit{(ripetizione della strofa)}} 
	 
}
\vspace*{\versesep}
\endverse


\endsong
%------------------------------------------------------------
%			FINE CANZONE
%------------------------------------------------------------




%++++++++++++++++++++++++++++++++++++++++++++++++++++++++++++
%			CANZONE TRASPOSTA
%++++++++++++++++++++++++++++++++++++++++++++++++++++++++++++
\ifchorded
%decremento contatore per avere stesso numero
\addtocounter{songnum}{-1} 
\beginsong{Beato il cuore }[ by={Inno GMG Cracovia 2016 — J. Blycharz}]	% <<< COPIA TITOLO E AUTORE
\transpose{0} 						% <<< TRASPOSIZIONE #TONI + - (0 nullo)
%\preferflats SE VOGLIO FORZARE i bemolle come alterazioni
%\prefersharps SE VOGLIO FORZARE i # come alterazioni
\ifchorded
	\textnote{$\lozenge$ Tonalità originale}	% <<< EV COMMENTI (tonalità originale/migliore)
\fi


%%%%%% INTRODUZIONE
\ifchorded
\vspace*{\versesep}
\musicnote{
\begin{minipage}{0.48\textwidth}
\textbf{Intro}
\hfill 
%( \eighthnote \, 80)   % <<  MODIFICA IL TEMPO
% Metronomo: \eighthnote (ottavo) \quarternote (quarto) \halfnote (due quarti)
\end{minipage}
} 	
\vspace*{-\versesep}
\beginverse*

\nolyrics

%---- Prima riga -----------------------------
\vspace*{-\versesep}
\[C#-]  \[A]	\[E]  % \[*D] per indicare le pennate, \rep{2} le ripetizioni

%---- Ogni riga successiva -------------------
\vspace*{-\versesep}
\[B] \[(B)] \[(B)]  \[C#-]

%---- Ev Indicazioni -------------------------			
%\textnote{\textit{(Oppure tutta la strofa)} }	

\endverse
\fi




%%%%% STROFA
\beginverse		%Oppure \beginverse* se non si vuole il numero di fianco
\memorize 		% <<< DECOMMENTA se si vuole utilizzarne la funzione
%\chordsoff		& <<< DECOMMENTA se vuoi una strofa senza accordi

\[C#-]Sei sceso \[A]dalla tua immensi\[E]tà
\[D]in nostro a\[A]iu\[E]to.
Miseri\[B]cordia  scorre  da \[F#]te
\[A]sopra \[B]tutti \[C#]noi.


\endverse


%%%%% STROFA
\beginverse*	%Oppure \beginverse* se non si vuole il numero di fianco
%\memorize 		% <<< DECOMMENTA se si vuole utilizzarne la funzione
%\chordsoff		& <<< DECOMMENTA se vuoi una strofa senza accordi

^ Persi in un ^mondo d’oscuri^tà
^lì Tu ci ^tro^vi.
Nelle tue ^braccia ci stringi e ^poi
^dai la ^vita per ^noi.


\endverse



%%%%% RITORNELLO
\beginchorus
\textnote{\textbf{Rit.}}

Beato è il \[E]cuo\[B]re che per\[C#-]do\[A]na!
Miseri\[E]cordia riceve\[B]rà da Dio in ci\[F#]elo! \rep{2}

\endchorus



%%%%% STROFA
\beginverse		%Oppure \beginverse* se non si vuole il numero di fianco
%\memorize 		% <<< DECOMMENTA se si vuole utilizzarne la funzione
%\chordsoff		% <<< DECOMMENTA se vuoi una strofa senza accordi

^ Solo il per^dono riporte^rà
^pace nel ^mon^do.
Solo il per^dono ci svele^rà
^come f^igli t^uoi.

\endverse



%%%%% RITORNELLO
\beginchorus
\textnote{\textbf{Rit.}}

Beato è il \[E]cuo\[B]re che per\[C#-]do\[A]na!
Miseri\[E]cordia riceve\[B]rà da Dio in ci\[F#]elo! \rep{2}

\endchorus




%%%%% STROFA
\beginverse		%Oppure \beginverse* se non si vuole il numero di fianco
%\memorize 		% <<< DECOMMENTA se si vuole utilizzarne la funzione
%\chordsoff		% <<< DECOMMENTA se vuoi una strofa senza accordi

^ Col sangue in ^croce hai pagato ^Tu
^le nostre ^pover^tà.
Se noi ci am^iamo e restiamo in^ te
^il mondo ^crede^rà!

\endverse



%%%%% RITORNELLO
\beginchorus
\textnote{\textbf{Rit.}}

Beato è il \[E]cuo\[B]re che per\[C#-]do\[A]na!
Miseri\[E]cordia riceve\[B]rà da Dio in ci\[F#]elo! \rep{2}

\endchorus




%%%%% BRIDGE
\beginverse*		%Oppure \beginverse* se non si vuole il numero di fianco
%\memorize 		% <<< DECOMMENTA se si vuole utilizzarne la funzione
%\chordsoff		% <<< DECOMMENTA se vuoi una strofa senza accordi
\textnote{\textbf{Bridge}}
\[A]Le nostre an\[B]gosce ed ansie\[C#-]tà
get\[A]tiamo ogni \[B]attimo in \[A]te.
Amore \[B]che non abbandona \[C#-]mai,
\[A]vivi in \[B]mezzo a \[C#]noi!

\endverse



%%%%% RITORNELLO
\beginchorus
\textnote{\textbf{Rit.}}

Beato è il \[A]cuo\[E]re che per\[F#-]do\[D]na!
Miseri\[A]cordia riceve\[E]rà da Dio in ci\[B]elo! \rep{4}

\endchorus



%%%%%% EV. INTERMEZZO
\beginverse*
\vspace*{1.3\versesep}
{
	\nolyrics
	\musicnote{Chiusura strumentale}
	
	\ifchorded

	%---- Prima riga -----------------------------
	\vspace*{-\versesep}
	\[C#-]  \[A]	\[E]  % \[*D] per indicare le pennate, \rep{2} le ripetizioni


	%---- Ogni riga successiva -------------------
	\vspace*{-\versesep}
	\[B] \[(B)] \[(B)]  \[C#-]


	\fi
	%---- Ev Indicazioni -------------------------			
	%\textnote{\textit{(ripetizione della strofa)}} 
	 
}
\vspace*{\versesep}
\endverse


\endsong


\fi
%++++++++++++++++++++++++++++++++++++++++++++++++++++++++++++
%			FINE CANZONE TRASPOSTA
%++++++++++++++++++++++++++++++++++++++++++++++++++++++++++++

%-------------------------------------------------------------
%			INIZIO	CANZONE
%-------------------------------------------------------------


%titolo: 	Benedetto sei Signor
%autore: 	La Rocca, Capacchione
%tonalita: 	Sol 



%%%%%% TITOLO E IMPOSTAZONI
\beginsong{Benedetto sei Signor}[by={La Rocca, Capacchione}] 	% <<< MODIFICA TITOLO E AUTORE
\transpose{0} 						% <<< TRASPOSIZIONE #TONI (0 nullo)
\momenti{Offertorio; Pasqua}							% <<< INSERISCI MOMENTI	
% momenti vanno separati da ; e vanno scelti tra:
% Ingresso; Atto penitenziale; Acclamazione al Vangelo; Dopo il Vangelo; Offertorio; Comunione; Ringraziamento; Fine; Santi; Pasqua; Avvento; Natale; Quaresima; Canti Mariani; Battesimo; Prima Comunione; Cresima; Matrimonio; Meditazione; Spezzare del pane;
\ifchorded
	%\textnote{Tonalità migliore }	% <<< EV COMMENTI (tonalità originale/migliore)
\fi


%%%%%% INTRODUZIONE
\ifchorded
\vspace*{\versesep}
\textnote{Intro: \qquad \qquad  }%(\eighthnote 116) % <<  MODIFICA IL TEMPO
% Metronomo: \eighthnote (ottavo) \quarternote (quarto) \halfnote (due quarti)
\vspace*{-\versesep}
\beginverse*

\nolyrics

%---- Prima riga -----------------------------
\vspace*{-\versesep}
\[F] \[D-] \[B&*]  \[C*] \[F] \rep{2}	 % \[*D] per indicare le pennate, \rep{2} le ripetizioni

%---- Ogni riga successiva -------------------
%\vspace*{-\versesep}
%\[G] \[C]  \[D]	

%---- Ev Indicazioni -------------------------			
%\textnote{\textit{(Oppure tutta la strofa)} }	

\endverse
\fi




%%%%% RITORNELLO
\beginchorus
\textnote{\textbf{Rit.}}

\[F]Gloria, \[D-]gloria, bene\[B&*]detto \[C*]sei Si\[F]gnor!
\[F]Gloria, \[D-]gloria, bene\[B&*]detto \[C*]sei Si\[F]gnor!

\endchorus



%%%%% STROFA
\beginverse		%Oppure \beginverse* se non si vuole il numero di fianco
\memorize 		% <<< DECOMMENTA se si vuole utilizzarne la funzione
%\chordsoff		% <<< DECOMMENTA se vuoi una strofa senza accordi

Questo \[F]pane e questo \[D-]vino 
noi li \[G]presentiamo a \[C]Te:
sono i \[F]doni del tuo \[D-]amore,
 nutri\[G]mento dell'umani\[C]tà.
Bene\[A]detto sei Tu, Si\[D-]gnore, 
per la \[B&]mensa che prepari a \[C]noi;
fà che in\[A]torno a questo al\[D-]tare 
sia \[B&*]grande la fra\[C*]terni\[F]tà.

\endverse









%%%%% STROFA
\beginverse		%Oppure \beginverse* se non si vuole il numero di fianco
%\memorize 		% <<< DECOMMENTA se si vuole utilizzarne la funzione
%\chordsoff		% <<< DECOMMENTA se vuoi una strofa senza accordi

Questa ^vita che ci ^doni 
noi la ^presentiamo a ^Te,
nella ^fede i nostri ^giorni 
noi vi^vremo con semplici^tà.
Bene^detto sei Tu, Si^gnore, 
per il ^pane che ci done^rai;
fà che al ^mondo possiamo ^dare 
una ^vera soli^darie^tà.
\endverse




\endsong
%------------------------------------------------------------
%			FINE CANZONE
%------------------------------------------------------------


%-------------------------------------------------------------
%			INIZIO	CANZONE
%-------------------------------------------------------------


%titolo: 	Benedici il Signore anima mia
%autore: 	Frisina
%tonalita: 	Sol 



%%%%%% TITOLO E IMPOSTAZONI
\beginsong{Benedici il Signore anima mia}[by={M. Frisina}] 	% <<< MODIFICA TITOLO E AUTORE
\transpose{0} 						% <<< TRASPOSIZIONE #TONI (0 nullo)
\momenti{}							% <<< INSERISCI MOMENTI	
% momenti vanno separati da ; e vanno scelti tra:
% Ingresso; Atto penitenziale; Acclamazione al Vangelo; Dopo il Vangelo; Offertorio; Comunione; Ringraziamento; Fine; Santi; Pasqua; Avvento; Natale; Quaresima; Canti Mariani; Battesimo; Prima Comunione; Cresima; Matrimonio; Meditazione; Spezzare del pane;
\ifchorded
	%\textnote{Tonalità originale }	% <<< EV COMMENTI (tonalità originale/migliore)
\fi

%%%%%% INTRODUZIONE
\ifchorded
\vspace*{\versesep}
\musicnote{
\begin{minipage}{0.48\textwidth}
\textbf{Intro}
\hfill 
%( \eighthnote \, 80)   % <<  MODIFICA IL TEMPO
% Metronomo: \eighthnote (ottavo) \quarternote (quarto) \halfnote (due quarti)
\end{minipage}
} 	
\vspace*{-\versesep}
\beginverse*

\nolyrics

%---- Prima riga -----------------------------
\vspace*{-\versesep}
\[A-] \[G] \[A-]	 % \[*D] per indicare le pennate, \rep{2} le ripetizioni

%---- Ogni riga successiva -------------------
%\vspace*{-\versesep}
%\[G] \[C]  \[D]	

%---- Ev Indicazioni -------------------------			
\textnote{\textit{[oppure tutta la prima sequenza]} }	

\endverse
\fi


%%%%% RITORNELLO
\beginchorus
\textnote{\textbf{Rit.}}
Bene\[A-]dici il Si\[G]gnore, anima \[C]mia
Quant'è in \[F]me bene\[G]dica il suo \[C]nome
Non di\[D-]mentiche\[G]rò tutti i \[E]suoi bene\[A-]fici.
Bene\[F]dici il Si\[E-]gnore, anima \[A-]mia.
\endchorus

%%%%% STROFA
\beginverse
\memorize
Lui per\[F]dona \[G]tutte le tue \[C]colpe
e ti \[D-]salva dalla \[G]mor\[A-]te.
Ti co\[F]rona di \[G]grazia e ti \[E]sazia di \[A-]beni
nella \[F]tua giovi\[G]nez\[E]za.
\endverse


%%%%% STROFA
\beginverse
%\chordsoff
Il Si^gnore a^gisce con giu^stizia
con a^more verso i ^pove^ri
rive^lò a Mo^sè le sue ^vie, ad Isra^ele
le sue ^grandi ^ope^re.
\endverse

%%%%% STROFA
\beginverse
\chordsoff
Il Signore è buono e pietoso
lento all'ira e grande nell'amor.
Non conserva in eterno il suo sdegno e la sua ira
verso i nostri peccati.
\endverse

%%%%% STROFA
\beginverse
\chordsoff
Come dista Oriente da Occidente
allontana le tue colpe.
Perché sa che di polvere siam tutti noi plasmati,
come l'erba i nostri giorni.
\endverse


\beginchorus
\vspace*{1.3\versesep}
\textnote{\textbf{Finale }}
%\chordsoff
Bene^dite il Si^gnore voi  ^angeli,
voi ^tutti su^oi mi^nistri.
Bene^ditelo voi ^tutte sue ^opere e do^míni.
Bene\[F]dicilo \[E-]tu, anima \[A-]mia.
\endchorus


\endsong


%titolo{Benedici, o Signore}
%autore{Gen Rosso}
%album{Se siamo uniti}
%tonalita{Si-}
%famiglia{Liturgica}
%gruppo{}
%momenti{Offertorio}
%identificatore{benedici_o_signore}
%data_revisione{2013_12_30}
%trascrittore{Francesco Endrici}
\beginsong{Benedici, o Signore}[by={Gen\ Rosso}]
\beginverse
\[B-]Nebbia e freddo, giorni lunghi e a\[A]mari
mentre il seme \[B-]muore.
\[D]Poi prodigio, antico e sempre \[A]nuovo,
del primo filo d'\[G7+]erba.
E nel \[D]vento dell'e\[A]state on\[B-]deggiano le \[D]spighe
a\[A]vremo ancora \[F#]pa\[B]ne.
\endverse
\beginchorus
\[E]Bene\[B]dici, \[E]o Si\[B]gnore,
\[A]questa o\ch{E}{f}{f}{ff}erta che por\[F#4]tiamo a \[F#]te
\[E]Facci \[B]uno \[G#-]come il \[E&-]pane
\[C#]che anche \[E]oggi hai \[B]dato a noi.
\endchorus
\beginverse
\chordsoff
^Nei filari, dopo il lungo in^verno \brk fremono le ^viti.
^La rugiada avvolge nel si^lenzio \brk i primi tralci ^verdi.
Poi co^lori dell'au^tunno, coi ^grappoli ma^turi
a^vremo ancora ^vi^no.
\endverse
\endsong


%-------------------------------------------------------------
%			INIZIO	CANZONE
%-------------------------------------------------------------


%titolo: 	Benedite e acclamate
%autore: 	A. Casale
%tonalita: 	Fa 



%%%%%% TITOLO E IMPOSTAZONI
\beginsong{Benedite e acclamate}[by={A. Casale}] 	% <<< MODIFICA TITOLO E AUTORE
\transpose{0} 						% <<< TRASPOSIZIONE #TONI (0 nullo)
\momenti{Ingresso; Matrimonio; }							% <<< INSERISCI MOMENTI	
% momenti vanno separati da ; e vanno scelti tra:
% Ingresso; Atto penitenziale; Acclamazione al Vangelo; Dopo il Vangelo; Offertorio; Comunione; Ringraziamento; Fine; Santi; Pasqua; Avvento; Natale; Quaresima; Canti Mariani; Battesimo; Prima Comunione; Cresima; Matrimonio; Meditazione; Spezzare del pane;
\ifchorded
	\textnote{$\bigstar$ Tonalità migliore }	% <<< EV COMMENTI (tonalità originale/migliore)
\fi



%%%%%% INTRODUZIONE
\ifchorded
\vspace*{\versesep}
\musicnote{
\begin{minipage}{0.48\textwidth}
\textbf{Intro}
\hfill 
(\eighthnote \, 88)
%( \eighthnote \, 80)   % <<  MODIFICA IL TEMPO
% Metronomo: \eighthnote (ottavo) \quarternote (quarto) \halfnote (due quarti)
\end{minipage}
} 	
\vspace*{-\versesep}
\beginverse*


\nolyrics

%---- Prima riga -----------------------------
\vspace*{-\versesep}
\[F] \[C] \[D-]	 % \[*D] per indicare le pennate, \rep{2} le ripetizioni

%---- Ogni riga successiva -------------------
\vspace*{-\versesep}
\[B&] \[F]  \[C] \[B&]  \[B&] 	

%---- Ev Indicazioni -------------------------			
%\textnote{\textit{(Oppure tutta la strofa)} }	

\endverse
\fi








%%%%% RITORNELLO
\beginchorus
\textnote{\textbf{Rit.}}

Bene\[F]dite e accla\[C]mate
il Si\[D-]gnore di tutta la terra,
egli \[G-]compie prodigi
chia\[B&]mando ogni uomo all'a\[C]more. \[C]
Bene\[F]dite e accla\[C]mate
il Si\[D-]gnore di tutta la terra,
egli \[G-]compie prodigi 
u\[B&]nendo un uomo e una \[C]donna.

\endchorus



%%%%% STROFA
\beginverse		%Oppure \beginverse* se non si vuole il numero di fianco
\memorize 		% <<< DECOMMENTA se si vuole utilizzarne la funzione
%\chordsoff		% <<< DECOMMENTA se vuoi una strofa senza accordi

\[D-]Guarda, Signore, 
questo \[G-]patto d'amore
che per \[B&]grazia questi \[E&]sposi
con\[C4]sacrano da\[C]vanti a Te.
E sia \[D-]segno di speranza
e sia \[G-]segno che rivela
il mi\[B&]stero dell'a\[E&]more:
la \[C4]fonte del cre\[C]ato.

\endverse



%%%%% STROFA
\beginverse		%Oppure \beginverse* se non si vuole il numero di fianco
%\memorize 		% <<< DECOMMENTA se si vuole utilizzarne la funzione
%\chordsoff		% <<< DECOMMENTA se vuoi una strofa senza accordi

^Padre di ogni bene,
Tu, che ^guidi i loro passi,
bene^dici nel tuo ^nome
i ^sogni che li u^niscono.
E sia ^pace sullo sposo
e sia ^vita nella sposa, 
la Tua ^mano mostri an^cora
la ^Tua benevo^lenza.

\endverse



\endsong
%------------------------------------------------------------
%			FINE CANZONE
%------------------------------------------------------------

%++++++++++++++++++++++++++++++++++++++++++++++++++++++++++++
%			CANZONE TRASPOSTA
%++++++++++++++++++++++++++++++++++++++++++++++++++++++++++++
\ifchorded
%decremento contatore per avere stesso numero
\addtocounter{songnum}{-1} 
\beginsong{Benedite e acclamate}[by={A. Casale}] 	% <<< COPIA TITOLO E AUTORE
\transpose{+2} 						% <<< TRASPOSIZIONE #TONI + - (0 nullo)
%\preferflats  %SE VOGLIO FORZARE i bemolle come alterazioni
%\prefersharps %SE VOGLIO FORZARE i # come alterazioni
\ifchorded
	\textnote{$\lozenge$ Tonalità originale}	% <<< EV COMMENTI (tonalità originale/migliore)
\fi



%%%%%% INTRODUZIONE
\ifchorded
\vspace*{\versesep}
\musicnote{
\begin{minipage}{0.48\textwidth}
\textbf{Intro}
\hfill 
(\eighthnote \, 88)
%( \eighthnote \, 80)   % <<  MODIFICA IL TEMPO
% Metronomo: \eighthnote (ottavo) \quarternote (quarto) \halfnote (due quarti)
\end{minipage}
} 	
\vspace*{-\versesep}
\beginverse*


\nolyrics

%---- Prima riga -----------------------------
\vspace*{-\versesep}
\[F] \[C] \[D-]	 % \[*D] per indicare le pennate, \rep{2} le ripetizioni

%---- Ogni riga successiva -------------------
\vspace*{-\versesep}
\[B&] \[F]  \[C] \[B&]  \[B&] 	

%---- Ev Indicazioni -------------------------			
%\textnote{\textit{(Oppure tutta la strofa)} }	

\endverse
\fi








%%%%% RITORNELLO
\beginchorus
\textnote{\textbf{Rit.}}

Bene\[F]dite e accla\[C]mate
il Si\[D-]gnore di tutta la terra,
egli \[G-]compie prodigi
chia\[B&]mando ogni uomo all'a\[C]more. \[C]
Bene\[F]dite e accla\[C]mate
il Si\[D-]gnore di tutta la terra,
egli \[G-]compie prodigi 
u\[B&]nendo un uomo e una \[C]donna.

\endchorus



%%%%% STROFA
\beginverse		%Oppure \beginverse* se non si vuole il numero di fianco
\memorize 		% <<< DECOMMENTA se si vuole utilizzarne la funzione
%\chordsoff		% <<< DECOMMENTA se vuoi una strofa senza accordi

\[D-]Guarda, Signore, 
questo \[G-]patto d'amore
che per \[B&]grazia questi \[E&]sposi
con\[C4]sacrano da\[C]vanti a Te.
E sia \[D-]segno di speranza
e sia \[G-]segno che rivela
il mi\[B&]stero dell'a\[E&]more:
la \[C4]fonte del cre\[C]ato.

\endverse



%%%%% STROFA
\beginverse		%Oppure \beginverse* se non si vuole il numero di fianco
%\memorize 		% <<< DECOMMENTA se si vuole utilizzarne la funzione
%\chordsoff		% <<< DECOMMENTA se vuoi una strofa senza accordi

^Padre di ogni bene,
Tu, che ^guidi i loro passi,
bene^dici nel tuo ^nome
i ^sogni che li u^niscono.
E sia ^pace sullo sposo
e sia ^vita nella sposa, 
la Tua ^mano mostri an^cora
la ^Tua benevo^lenza.

\endverse



\endsong
\fi
%++++++++++++++++++++++++++++++++++++++++++++++++++++++++++++
%			FINE CANZONE TRASPOSTA
%++++++++++++++++++++++++++++++++++++++++++++++++++++++++++++


%-------------------------------------------------------------
%			INIZIO	CANZONE
%-------------------------------------------------------------


%titolo: 	Blessed be your name
%autore: 	Matt Redman
%tonalita: 	Sol > La 



%%%%%% TITOLO E IMPOSTAZONI
\beginsong{Blessed be Your name}[by={M. Redman, B. Redman}] 	% <<< MODIFICA TITOLO E AUTORE
\transpose{0} 						% <<< TRASPOSIZIONE #TONI (0 nullo)
%\preferflats  %SE VOGLIO FORZARE i bemolle come alterazioni
%\prefersharps %SE VOGLIO FORZARE i # come alterazioni
\momenti{Ingresso; Meditazione; Ringraziamento; Congedo; Matrimonio}							% <<< INSERISCI MOMENTI	
% momenti vanno separati da ; e vanno scelti tra:
% Ingresso; Atto penitenziale; Acclamazione al Vangelo; Dopo il Vangelo; Offertorio; Comunione; Ringraziamento; Fine; 
% Santi; Pasqua; Avvento; Natale; Quaresima; Canti Mariani; Battesimo; Prima Comunione; Cresima; Matrimonio; Meditazione; Spezzare del pane;
\ifchorded
	%\textnote{Tonalità migliore }	% <<< EV COMMENTI (tonalità originale/migliore)
\fi


%%%%%% INTRODUZIONE
\ifchorded
\vspace*{\versesep}
\musicnote{
\begin{minipage}{0.48\textwidth}
\textbf{Intro}
\hfill 
%( \eighthnote \, 80)   % <<  MODIFICA IL TEMPO
% Metronomo: \eighthnote (ottavo) \quarternote (quarto) \halfnote (due quarti)
\end{minipage}
} 	
\vspace*{-\versesep}
\beginverse*

\nolyrics

%---- Prima riga -----------------------------
\vspace*{-\versesep}
\[G] \[D] \[E-] \[C] 	 % \[*D] per indicare le pennate, \rep{2} le ripetizioni

%---- Ogni riga successiva -------------------
\vspace*{-\versesep}
\[G] \[D] \[E-] \[C] \[C]	

%---- Ev Indicazioni -------------------------			
%\textnote{\textit{(Oppure tutta la strofa)} }	

\endverse
\fi




%%%%% STROFA
\beginverse		%Oppure \beginverse* se non si vuole il numero di fianco
\memorize 		% <<< DECOMMENTA se si vuole utilizzarne la funzione
%\chordsoff		% <<< DECOMMENTA se vuoi una strofa senza accordi

\[G] Blessed \[D]Be Your Name,  \brk in the \[E-]land that is \[C]plentiful, 
where Your \[G]streams of a\[D]bundance flow,
blessed \[C]be Your name.

\endverse
\beginverse*	

^ Blessed ^be Your name,  \brk when i'm ^found in the ^desert place,
though i ^walk through the ^wilderness,
blessed ^be Your name.


\endverse
\beginverse*		

^ Every blessing ^you pour out, i'll 
^ turn back to ^praise,
^ when the darkness ^closes in, Lord,
\[E-] still i will \[C]say:

\endverse


%%%%% RITORNELLO
\beginchorus
\textnote{\textbf{Rit.}}

Blessed be the \[G]name of the \[D]Lord!
Blessed be Your \[E-]name! \[C]
Blessed be the \[G]name of the \[D]Lord!
Blessed be Your \[E-]glorious \[C]name! 	\rep{2}

\endchorus


%%%%% STROFA
\beginverse		%Oppure \beginverse* se non si vuole il numero di fianco
%\memorize 		% <<< DECOMMENTA se si vuole utilizzarne la funzione
%\chordsoff		% <<< DECOMMENTA se vuoi una strofa senza accordi

^ Blessed ^be Your name, \brk when the ^sun's shining ^down on me,
when the ^world is all as ^it should be,
blessed b^e Your name.

\endverse
\beginverse*	

^ Blessed ^be Your name, \brk on the ^road marked with ^suffering,
though there is ^pain in the ^offering,
blessed ^be your name.

\endverse
\beginverse*		

^ Every blessing ^you pour out, i'll 
^ turn back to ^praise,
^ when the darkness ^closes in, Lord,
\[E-] still i will \[C]say:

\endverse


%%%%% RITORNELLO
\beginchorus
\textnote{\textbf{Rit.}}

Blessed be the \[G]name of the \[D]Lord!
Blessed be Your \[E-]name! \[C]
Blessed be the \[G]name of the \[D]Lord!
Blessed be Your \[E-]glorious \[C]name! 	\rep{2}

\endchorus


%%%%% BRIDGE
\beginverse*		%Oppure \beginverse* se non si vuole il numero di fianco
%\memorize 		% <<< DECOMMENTA se si vuole utilizzarne la funzione
%\chordsoff		% <<< DECOMMENTA se vuoi una strofa senza accordi
\vspace*{1.3\versesep}
\textnote{\textbf{Bridge}} %<<< EV. INDICAZIONI

You ^give and take ^away.
You ^give and take ^away.
My ^heart will choose to ^say,
Lord, bl\[E-]essed be Your \[C]name!  \rep{2}

\endverse
\beginverse*		

^ Every blessing ^you pour out, i'll 
^ turn back to ^praise,
^ when the darkness ^closes in, Lord,
\[E-] still i will \[C]say:

\endverse



%%%%% RITORNELLO
\beginchorus
\textnote{\textbf{Rit.}}

Blessed be the \[G]name of the \[D]Lord!
Blessed be Your \[E-]name! \[C]
Blessed be the \[G]name of the \[D]Lord!
Blessed be Your \[E-]glorious \[C]name! 	\rep{2}

\endchorus





%%%%% BRIDGE
\beginverse*		%Oppure \beginverse* se non si vuole il numero di fianco
%\memorize 		% <<< DECOMMENTA se si vuole utilizzarne la funzione
%\chordsoff		% <<< DECOMMENTA se vuoi una strofa senza accordi

You ^give and take ^away.
You ^give and take ^away.
My ^heart will choose to ^say,
Lord, bl\[E-]essed be Your \[C]name!  \rep{2}

\endverse



%%%%%% EV. INTERMEZZO
\beginverse*
\vspace*{1.3\versesep}
{
	\nolyrics
	\musicnote{Chiusura strumentale}
	
	\ifchorded

	%---- Prima riga -----------------------------
	\vspace*{-\versesep}
	\[G] \[D] \[E-] \[C] 

	%---- Ogni riga successiva -------------------
	\vspace*{-\versesep}
	\[G] \[D] \[C*]  \textit{(sospeso...)}


	\fi
	%---- Ev Indicazioni -------------------------			
	%\musicnote{\textit{sospeso}} 
	 
}
\vspace*{\versesep}
\endverse


\endsong
%------------------------------------------------------------
%			FINE CANZONE
%------------------------------------------------------------


 
%CCC
%-------------------------------------------------------------
%			INIZIO	CANZONE
%-------------------------------------------------------------


%titolo: 	Camminerò
%autore: 	
%tonalita: 	Do



%%%%%% TITOLO E IMPOSTAZONI
\beginsong{Camminerò}[by={}] 	% <<< MODIFICA TITOLO E AUTORE
\transpose{0} 						% <<< TRASPOSIZIONE #TONI (0 nullo)
\momenti{Fine; Prima Comunione}							% <<< INSERISCI MOMENTI	
% momenti vanno separati da ; e vanno scelti tra:
% Ingresso; Atto penitenziale; Acclamazione al Vangelo; Dopo il Vangelo; Offertorio; Comunione; Ringraziamento; Fine; Santi; Pasqua; Avvento; Natale; Quaresima; Canti Mariani; Battesimo; Prima Comunione; Cresima; Matrimonio; Meditazione; Spezzare del pane;
\ifchorded
	%\textnote{Tonalità originale }	% <<< EV COMMENTI (tonalità originale/migliore)
\fi

%%%%%% INTRODUZIONE
\ifchorded
\vspace*{\versesep}
\textnote{Intro: \qquad \qquad  }%(\eighthnote 116) % <<  MODIFICA IL TEMPO
% Metronomo: \eighthnote (ottavo) \quarternote (quarto) \halfnote (due quarti)
\vspace*{-\versesep}
\beginverse*

\nolyrics

%---- Prima riga -----------------------------
\vspace*{-\versesep}
\[C] \[F] \[G7] \[C]	 % \[*D] per indicare le pennate, \rep{2} le ripetizioni

%---- Ogni riga successiva -------------------
%\vspace*{-\versesep}
%\[G] \[C]  \[D]	

%---- Ev Indicazioni -------------------------			
%\textnote{\textit{(Oppure tutta la strofa)} }	

\endverse
\fi

%%%%% STROFA
\beginverse		%Oppure \beginverse* se non si vuole il numero di fianco
\memorize 		% <<< DECOMMENTA se si vuole utilizzarne la funzione
%\chordsoff		% <<< DECOMMENTA se vuoi una strofa senza accordi

Mi hai chia\[C]mato dal nulla Si\[F]gnore
e mi hai \[G7]dato il dono della \[C]vita
tu mi hai preso e mi hai messo per \[F]strada
e mi hai \[G7]detto di cammi\[C]nar.
Verso un mondo che non ha con\[F]fini no
verso \[C]mete da raggiunge\[G7]re oramai
verso il \[C]regno dell'A\[G7]more
che è \[F]sempre un pò più in \[C]là!

\endverse

%%%%% RITORNELLO
\beginchorus
\textnote{\textbf{Rit.}}

Cammine\[F]rò senza stan\[C]carmi
e vole\[F]rò sui monti più \[C]alti
e trove\[F]rò la forza di an\[C]dare, sempre più a\[F]vanti, \[*G7]sì!
io \[C]camminerò, \[F]camminerò
\[C]con te vicino, \[F]io non cadrò
e \[C]camminerò, \[G]camminerò. \[C]

\endchorus

%%%%% STROFA
\beginverse		%Oppure \beginverse* se non si vuole il numero di fianco
%\memorize 		% <<< DECOMMENTA se si vuole utilizzarne la funzione
%\chordsoff		% <<< DECOMMENTA se vuoi una strofa senza accordi

In ogni is^tante ti sento vi^cino,
Tu dai il ^senso alle cose che io ^faccio.
La tua luce mi indica la ^strada
e mi in^vita a cammi^nar.
Verso un mondo che non ha con^fini no
verso ^mete da raggiunge^re oramai
verso il ^regno dell'A^more
che è ^sempre un pò più in ^là!

\endverse

%%%%% RITORNELLO
\beginchorus
\textnote{\textbf{Rit.}}

Cammine\[F]rò senza stan\[C]carmi
e vole\[F]rò sui monti più \[C]alti
e trove\[F]rò la forza di an\[C]dare, sempre più a\[F]vanti, \[*G7]sì!
io \[C]camminerò, \[F]camminerò
\[C]con te vicino, \[F]io non cadrò
e \[C]camminerò, \[G]camminerò.\[C]


\endchorus

\endsong
%------------------------------------------------------------
%			FINE CANZONE
%------------------------------------------------------------


%++++++++++++++++++++++++++++++++++++++++++++++++++++++++++++
%			CANZONE TRASPOSTA
%++++++++++++++++++++++++++++++++++++++++++++++++++++++++++++
\ifchorded
%decremento contatore per avere stesso numero
\addtocounter{songnum}{-1} 
\beginsong{Camminerò}[by={}] 	% <<< COPIA TITOLO E AUTORE
\transpose{-3} 						% <<< TRASPOSIZIONE #TONI + - (0 nullo)
%\preferflats  %SE VOGLIO FORZARE i bemolle come alterazioni
%\prefersharps %SE VOGLIO FORZARE i # come alterazioni
\ifchorded
	%\textnote{Tonalità originale}	% <<< EV COMMENTI (tonalità originale/migliore)
\fi


%%%%%% INTRODUZIONE
\ifchorded
\vspace*{\versesep}
\textnote{Intro: \qquad \qquad  }%(\eighthnote 116) % <<  MODIFICA IL TEMPO
% Metronomo: \eighthnote (ottavo) \quarternote (quarto) \halfnote (due quarti)
\vspace*{-\versesep}
\beginverse*

\nolyrics

%---- Prima riga -----------------------------
\vspace*{-\versesep}
\[C] \[F] \[G7] \[C]	 % \[*D] per indicare le pennate, \rep{2} le ripetizioni

%---- Ogni riga successiva -------------------
%\vspace*{-\versesep}
%\[G] \[C]  \[D]	

%---- Ev Indicazioni -------------------------			
%\textnote{\textit{(Oppure tutta la strofa)} }	

\endverse
\fi

%%%%% STROFA
\beginverse		%Oppure \beginverse* se non si vuole il numero di fianco
\memorize 		% <<< DECOMMENTA se si vuole utilizzarne la funzione
%\chordsoff		% <<< DECOMMENTA se vuoi una strofa senza accordi

Mi hai chia\[C]mato dal nulla Si\[F]gnore
e mi hai \[G7]dato il dono della \[C]vita
tu mi hai preso e mi hai messo per \[F]strada
e mi hai \[G7]detto di cammi\[C]nar.
Verso un mondo che non ha con\[F]fini no
verso \[C]mete da raggiunge\[G7]re oramai
verso il \[C]regno dell'A\[G7]more
che è \[F]sempre un pò più in \[C]là!

\endverse

%%%%% RITORNELLO
\beginchorus
\textnote{\textbf{Rit.}}

Cammine\[F]rò senza stan\[C]carmi
e vole\[F]rò sui monti più \[C]alti
e trove\[F]rò la forza di an\[C]dare, sempre più a\[F]vanti, \[*G7]sì!
io \[C]camminerò, \[F]camminerò
\[C]con te vicino, \[F]io non cadrò
e \[C]camminerò, \[G]camminerò. \[C]

\endchorus

%%%%% STROFA
\beginverse		%Oppure \beginverse* se non si vuole il numero di fianco
%\memorize 		% <<< DECOMMENTA se si vuole utilizzarne la funzione
%\chordsoff		% <<< DECOMMENTA se vuoi una strofa senza accordi

In ogni is^tante ti sento vi^cino,
Tu dai il ^senso alle cose che io ^faccio.
La tua luce mi indica la ^strada
e mi in^vita a cammi^nar.
Verso un mondo che non ha con^fini no
verso ^mete da raggiunge^re oramai
verso il ^regno dell'A^more
che è ^sempre un pò più in ^là!

\endverse

%%%%% RITORNELLO
\beginchorus
\textnote{\textbf{Rit.}}

Cammine\[F]rò senza stan\[C]carmi
e vole\[F]rò sui monti più \[C]alti
e trove\[F]rò la forza di an\[C]dare, sempre più a\[F]vanti, \[*G7]sì!
io \[C]camminerò, \[F]camminerò
\[C]con te vicino, \[F]io non cadrò
e \[C]camminerò, \[G]camminerò.\[C]


\endchorus

\endsong
%------------------------------------------------------------
%			FINE CANZONE
%------------------------------------------------------------



\fi
%++++++++++++++++++++++++++++++++++++++++++++++++++++++++++++
%			FINE CANZONE TRASPOSTA
%++++++++++++++++++++++++++++++++++++++++++++++++++++++++++++
%-------------------------------------------------------------
%			INIZIO	CANZONE
%-------------------------------------------------------------


%titolo: 	Camminerò, camminerò
%autore: 	Marani, Gen Rosso
%tonalita: 	Re 



%%%%%% TITOLO E IMPOSTAZONI
\beginsong{Camminerò, camminerò}[by={Marani, Gen Rosso}] 	% <<< MODIFICA TITOLO E AUTORE
\transpose{0} 						% <<< TRASPOSIZIONE #TONI (0 nullo)
\momenti{Fine}							% <<< INSERISCI MOMENTI	
% momenti vanno separati da ; e vanno scelti tra:
% Ingresso; Atto penitenziale; Acclamazione al Vangelo; Dopo il Vangelo; Offertorio; Comunione; Ringraziamento; Fine; Santi; Pasqua; Avvento; Natale; Quaresima; Canti Mariani; Battesimo; Prima Comunione; Cresima; Matrimonio; Meditazione; Spezzare del pane;
\ifchorded
	%\textnote{Tonalità originale }	% <<< EV COMMENTI (tonalità originale/migliore)
\fi


%%%%%% INTRODUZIONE
\ifchorded
\vspace*{\versesep}
\textnote{Intro: \qquad \qquad  }%(\eighthnote 116) % <<  MODIFICA IL TEMPO
% Metronomo: \eighthnote (ottavo) \quarternote (quarto) \halfnote (due quarti)
\vspace*{-\versesep}
\beginverse*

\nolyrics

%---- Prima riga -----------------------------
\vspace*{-\versesep}
\[D] \[B-] 	 % \[*D] per indicare le pennate, \rep{2} le ripetizioni

%---- Ogni riga successiva -------------------
\vspace*{-\versesep}
\[*G] \[*A] \[D]	

%---- Ev Indicazioni -------------------------			
%\textnote{\textit{(Oppure tutta la strofa)} }	

\endverse
\fi

%%%%% RITORNELLO
\beginchorus
\textnote{\textbf{Rit.}}

\[D]Camminerò, \[B-]camminerò, 
\[G]nella tua strada \[A]Signor.
\[D]Dammi la mano, \[B-]voglio restar, 
per \[E-]sempre \[A]insieme a \[D]te.

\endchorus

%%%%% STROFA
\beginverse		%Oppure \beginverse* se non si vuole il numero di fianco
\memorize 		% <<< DECOMMENTA se si vuole utilizzarne la funzione
%\chordsoff		% <<< DECOMMENTA se vuoi una strofa senza accordi

\[D]Quando ero solo, solo e \[B-]stanco del mondo
\[E-]quando non c’era l’A\[A]mor,
\[D]tante persone vidi \[B-]intorno a me;
\[E-]sentivo can\[A]tare co\[D]sì.

\endverse

%%%%% STROFA
\beginverse		%Oppure \beginverse* se non si vuole il numero di fianco
%\memorize 		% <<< DECOMMENTA se si vuole utilizzarne la funzione
%\chordsoff		% <<< DECOMMENTA se vuoi una strofa senza accordi

^Io non capivo ma ri^masi a sentire
^quando il Signore mi par^lò:
^lui mi chiamava, chia^mava anche me,
e la ^mia ris^posta si al^zò.

\endverse

%%%%% STROFA
\beginverse		%Oppure \beginverse* se non si vuole il numero di fianco
%\memorize 		% <<< DECOMMENTA se si vuole utilizzarne la funzione
\chordsoff		% <<< DECOMMENTA se vuoi una strofa senza accordi

^ Or non mi importa se uno ^ride di me,
^lui certamente non ^sa,
^del gran regalo che ^ebbi quel dì,
che ^dissi al Si^gnore co^si.

\endverse

%%%%% STROFA
\beginverse		%Oppure \beginverse* se non si vuole il numero di fianco
%\memorize 		% <<< DECOMMENTA se si vuole utilizzarne la funzione
\chordsoff		% <<< DECOMMENTA se vuoi una strofa senza accordi

^A volte son triste ma ^mi guardo intorno,
^scopro il mondo e l’a^mor;
^son questi i doni che ^lui fa a me,
fe^lice ri^torno a can^tar.

\endverse

\endsong
%------------------------------------------------------------
%			FINE CANZONE
%------------------------------------------------------------

%-------------------------------------------------------------
%			INIZIO	CANZONE
%-------------------------------------------------------------


%titolo: 	Camminiamo incontro al Signore
%autore: 	Galliano, Semprini
%tonalita: 	Re



%%%%%% TITOLO E IMPOSTAZONI
\beginsong{Camminiamo incontro al Signore}[by={A. M. Galliano, D. Semprini}] 	% <<< MODIFICA TITOLO E AUTORE
\transpose{0} 						% <<< TRASPOSIZIONE #TONI (0 nullo)
\momenti{Comunione; Avvento; Cresima; Prima Comunione}							% <<< INSERISCI MOMENTI	
% momenti vanno separati da ; e vanno scelti tra:
% Ingresso; Atto penitenziale; Acclamazione al Vangelo; Dopo il Vangelo; Offertorio; Comunione; Ringraziamento; Fine; Santi; Pasqua; 
% Avvento; Natale; Quaresima; Canti Mariani; Battesimo; Prima Comunione; Cresima; Matrimonio; Meditazione; Spezzare del pane;
\ifchorded
	%\textnote{Tonalità migliore }	% <<< EV COMMENTI (tonalità originale/migliore)
\fi


%%%%%% INTRODUZIONE
\ifchorded
\vspace*{\versesep}
\musicnote{
\begin{minipage}{0.48\textwidth}
\textbf{Intro}
\hfill 
%( \eighthnote \, 80)   % <<  MODIFICA IL TEMPO
% Metronomo: \eighthnote (ottavo) \quarternote (quarto) \halfnote (due quarti)
\end{minipage}
} 	
\vspace*{-\versesep}
\beginverse*

\nolyrics

%---- Prima riga -----------------------------
\vspace*{-\versesep}
\[D] \[A] \[B-]\[F#-]  % \[*D] per indicare le pennate, \rep{2} le ripetizioni

%---- Ogni riga successiva -------------------
\vspace*{-\versesep}
 \[G] \[D] \[A] \[A]


%---- Ev Indicazioni -------------------------			
%\textnote{\textit{(Oppure tutta la strofa)} }	

\endverse
\fi






%%%%% RITORNELLO
\beginchorus
\textnote{\textbf{Rit.}}
\[D]Cammini\[A]amo in\[B-]contro al Si\[F#-]gnore,
\[G]cammin\[D]iamo con \[A]gio\[B-]ia:
\[G]egli \[D]viene non \[A]tarde\[B-]rà,
\[G]egli \[D]viene, ci \[A]salve\[B-]ra-\[A7]à.
\endchorus



%%%%% STROFA
\beginverse		%Oppure \beginverse* se non si vuole il numero di fianco
\memorize 		% <<< DECOMMENTA se si vuole utilizzarne la funzione
%\chordsoff		% <<< DECOMMENTA se vuoi una strofa senza accordi
\[D]Egli \[A]viene il \[B-]giorno è vi\[F#-]cino
\[G]E la \[D]notte va \[A]verso l’au\[B-]rora
\[G]Elev\[D]iamo a \[A]Lui-l’anima-\[B-]nostra
\[G]Non sa\[D]remo de\[A]lu\[B-]si
\[G]Non sa\[D]remo de\[A]lu\[B-]si –\[A7]i
\endverse




%%%%% STROFA
\beginverse		%Oppure \beginverse* se non si vuole il numero di fianco
%\memorize 		% <<< DECOMMENTA se si vuole utilizzarne la funzione
%\chordsoff		% <<< DECOMMENTA se vuoi una strofa senza accordi
^Egli v^iene vegl^iamo in at^tesa
^Ricor^dando la ^sua pa^rola
^Rives^tiamo la ^forza di ^Dio
^Per re^sistere al ^ma^le
^Per re^sistere al ^ma^le-^e
\endverse




%%%%% STROFA
\beginverse		%Oppure \beginverse* se non si vuole il numero di fianco
%\memorize 		% <<< DECOMMENTA se si vuole utilizzarne la funzione
%\chordsoff		% <<< DECOMMENTA se vuoi una strofa senza accordi
^Egli v^iene and^iamogli in^contro
^Ritor^nando sui ^retti sen^tieri
^Mostre^rà la ^sua miseri^cordia
^Ci da^rà la sua ^gra^zia
^Ci da^rà la sua ^gra^zia-^a.
\endverse




%%%%% STROFA
\beginverse		%Oppure \beginverse* se non si vuole il numero di fianco
%\memorize 		% <<< DECOMMENTA se si vuole utilizzarne la funzione
%\chordsoff		% <<< DECOMMENTA se vuoi una strofa senza accordi
^Egli v^iene è il ^Dio fe^dele
^Che ci ^chiama alla ^sua comu^nione
^Il Si^gnore sa^rà-il nostro ^bene
^Noi la ^terra fe^con^da
^Noi la ^terra fe^con^da-^a.
\endverse





\endsong
%------------------------------------------------------------
%			FINE CANZONE
%------------------------------------------------------------




%-------------------------------------------------------------
%			INIZIO	CANZONE
%-------------------------------------------------------------


%titolo: 	Camminiamo sulla strada
%autore: 	Varnavà, spiritual
%tonalita: 	Re


%%%%%% TITOLO E IMPOSTAZONI
\beginsong{Camminiamo sulla strada}[ititle={O when the saints}, by={Canto Spiritual tradizionale, Testo di S. Varnavà}] 	% <<< MODIFICA TITOLO E AUTORE
\transpose{-2} 						% <<< TRASPOSIZIONE #TONI (0 nullo)
\momenti{Ingresso; Santi}							% <<< INSERISCI MOMENTI	
% momenti vanno separati da ; e vanno scelti tra:
% Ingresso; Atto penitenziale; Acclamazione al Vangelo; Dopo il Vangelo; Offertorio; Comunione; Ringraziamento; Fine; Santi; Pasqua; Avvento; Natale; Quaresima; Canti Mariani; Battesimo; Prima Comunione; Cresima; Matrimonio; Meditazione;
\ifchorded
	%\textnote{Tonalità originale }	% <<< EV COMMENTI (tonalità originale/migliore)
\fi


%%%%%% INTRODUZIONE
\ifchorded
\vspace*{\versesep}
\musicnote{
\begin{minipage}{0.48\textwidth}
\textbf{Intro}
\hfill 
%( \eighthnote \, 80)   % <<  MODIFICA IL TEMPO
% Metronomo: \eighthnote (ottavo) \quarternote (quarto) \halfnote (due quarti)
\end{minipage}
} 	
\vspace*{-\versesep}
\beginverse*

\nolyrics

%---- Prima riga -----------------------------
\vspace*{-\versesep}
\[E] \[A]  \[E] \[B] \[E]	 % \[*D] per indicare le pennate, \rep{2} le ripetizioni

%---- Ogni riga successiva -------------------
%\vspace*{-\versesep}
%\[G] \[C]  \[D]	

%---- Ev Indicazioni -------------------------			
\textnote{\textit{(come la seconda parte della strofa)} }	

\endverse
\fi





%%%%% STROFA
\beginverse		%Oppure \beginverse* se non si vuole il numero di fianco
\memorize 		% <<< DECOMMENTA se si vuole utilizzarne la funzione
%\chordsoff		% <<< DECOMMENTA se vuoi una strofa senza accordi

Cammi\[E]niamo \[7]sulla \[A]strada
che han per\[E]corso i santi \[B7]tuoi
tutti \[E]ci ri\[7]trove\[A]remo
dove e\[E]terno \[B7]splende il \[E]sol.
\endverse





%%%%% RITORNELLO
\beginchorus
\textnote{\textbf{Rit.}}
E \[E]quando in ciel dei santi tuoi
la grande schiera arrive\[B7]rà
o Si\[E]gnor co\[7]me vor\[A]rei \[A-]
che ci \[E]fosse un \[B7]posto per \[E]me.


E \[(E)]quando il sol si spegnerà
e quando il sol si spegne\[B7]rà
o Si\[E]gnor co\[7]me vor\[A]rei \[A-]
che ci \[E]fosse un \[B7]posto per \[E]me.
\endchorus




%%%%% STROFA
\beginverse		%Oppure \beginverse* se non si vuole il numero di fianco
%\memorize 		% <<< DECOMMENTA se si vuole utilizzarne la funzione
%\chordsoff		% <<< DECOMMENTA se vuoi una strofa senza accordi
C'è chi ^dice ^che la ^vita
sia tris^tezza sia do^lor
ma io ^so che vi^ene il ^giorno
in cui ^tutto ^cambie^rà.
\endverse


%%%%% RITORNELLO
\beginchorus
\textnote{\textbf{Rit.}}
E \[E]quando in ciel risuonerà
la tromba che tutti chiame\[B7]rà
o Si\[E]gnor co\[7]me vor\[A]rei \[A-]
che ci \[E]fosse un \[B7]posto per \[E]me.

Il \[(E)]giorno che la terra e il ciel
a nuova vita risorge\[B7]ran
o Si\[E]gnor co\[7]me vor\[A]rei \[A-]
che ci \[E]fosse un \[B7]posto per \[E]me.
\endchorus
\endsong


%-------------------------------------------------------------
%			INIZIO	CANZONE
%-------------------------------------------------------------


%titolo: 	Cantate al Signore
%autore: 	Fallormi
%tonalita: 	Sol 



%%%%%% TITOLO E IMPOSTAZONI
\beginsong{Cantate al Signore}[by={Fallormi}] 	% <<< MODIFICA TITOLO E AUTORE
\transpose{-3} 						% <<< TRASPOSIZIONE #TONI (0 nullo)
\momenti{Ingresso}							% <<< INSERISCI MOMENTI	
% momenti vanno separati da ; e vanno scelti tra:
% Ingresso; Atto penitenziale; Acclamazione al Vangelo; Dopo il Vangelo; Offertorio; Comunione; Ringraziamento; Fine; Santi; Pasqua; Avvento; Natale; Quaresima; Canti Mariani; Battesimo; Prima Comunione; Cresima; Matrimonio; Meditazione; Spezzare del pane;
\ifchorded
	%\textnote{Tonalità migliore }	% <<< EV COMMENTI (tonalità originale/migliore)
\fi


%%%%%% INTRODUZIONE
\ifchorded
\vspace*{\versesep}
\textnote{Intro: \qquad \qquad  }%(\eighthnote 116) % <<  MODIFICA IL TEMPO
% Metronomo: \eighthnote (ottavo) \quarternote (quarto) \halfnote (due quarti)
\vspace*{-\versesep}
\beginverse*

\nolyrics

%---- Prima riga -----------------------------
\vspace*{-\versesep}
\[G]\[D]\[C]\[D] \rep{2}	 % \[*D] per indicare le pennate, \rep{2} le ripetizioni

%---- Ogni riga successiva -------------------
%\vspace*{-\versesep}
%\[G] \[C]  \[D]	

%---- Ev Indicazioni -------------------------			
\textnote{\textit{(Oppure tutto il ritornello)} }	

\endverse
\fi



%%%%% RITORNELLO
\beginchorus
\textnote{\textbf{Rit.}}
Can\[G]tate al Si\[D]gnore un \[C]canto \[G]nuovo,
\[C]perché ha com\[G]piuto pro\[A-7]di|\[D4]gi.
\[(D*)]Ha |\[G]manife\[D]stato la \[C]sua sal\[G]vezza,
\[C]su tutti i |\[B-7/G]popo\[E-7*]li la |\[A-7]sua bon\[G]tà.
\endchorus


%%%%%% EV. INTERMEZZO
\beginverse*
\vspace*{1.3\versesep}
{
	\nolyrics
	\textnote{Intermezzo strumentale}
	
	\ifchorded

	%---- Prima riga -----------------------------
	\vspace*{-\versesep}
	\[G]\[D]\[C]\[D] \rep{2}


	\fi
	%---- Ev Indicazioni -------------------------			
	%\textnote{\textit{(ripetizione della strofa)}} 
	 
}
\vspace*{\versesep}
\endverse

\beginverse
\memorize
\[G] Egli \[C]si è ricor\[G]dato
\[(G)] della \[E-7]sua |\[C]fedel\[D]tà.
\[C] I con\[D]fini \[B-7]della \[E-7]terra 
\[C]hanno ve\[G]duto 
la sal\[A-7]vezza \[D7]del Si\[G]gnor.
\endverse


\beginverse
^ Esul^tiamo di ^gioia 
^ accla^miamo ^al Si^gnor. 
^ Con un ^suono ^melo^dioso: 
^cantiamo in^sieme
lode e ^gloria al ^nostro ^Re.
\endverse




\beginverse
^ Frema il ^mare e la ^terra, 
^ il Si^gno^re ver^rà! 
^ Con giu^dizio ^di giu^stizia, 
^con retti^tudine 
nel ^mondo ^porte\[E&]ra-a-\[F]a. \quad  \[F]
\endverse

%%%%% RITORNELLO
\beginchorus
\textnote{\textbf{Rit.}}
\transpose{3}
Can\[G]tate al Si\[D]gnore un \[C]canto \[G]nuovo,
\[C]perché ha com\[G]piuto pro\[A-7]di|\[D4]gi.
\[(D*)]Ha |\[G]manife\[D]stato la \[C]sua sal\[G]vezza,
\textnote{\textit{rallentando}}
\[C]su tutti i |\[B-7/G]popo\[E-7*]li 
la |\[A-7]sua \[D]bon\[G]tà.
\endchorus


\endsong
%------------------------------------------------------------
%			FINE CANZONE
%------------------------------------------------------------





%-------------------------------------------------------------
%			INIZIO	CANZONE
%-------------------------------------------------------------


%titolo: 	Cantico dei redenti
%autore: 	Marani
%tonalita: 	Sol 



%%%%%% TITOLO E IMPOSTAZONI
\beginsong{Cantico dei redenti}[ititle={Il Signore è la mia salvezza}, by={A. Marani}] 	% <<< MODIFICA TITOLO E AUTORE
\transpose{0} 						% <<< TRASPOSIZIONE #TONI (0 nullo)
\momenti{Ingresso; Congedo; Comunione; Salmi}							% <<< INSERISCI MOMENTI	
% momenti vanno separati da ; e vanno scelti tra:
% Ingresso; Atto penitenziale; Acclamazione al Vangelo; Dopo il Vangelo; Offertorio; Comunione; Ringraziamento; Fine; Santi; Pasqua; Avvento; Natale; Quaresima; Canti Mariani; Battesimo; Prima Comunione; Cresima; Matrimonio; Meditazione; Spezzare del pane;
\ifchorded
	%\textnote{Tonalità migliore }	% <<< EV COMMENTI (tonalità originale/migliore)
\fi




%%%%%% INTRODUZIONE
\ifchorded
\vspace*{\versesep}
\musicnote{
\begin{minipage}{0.48\textwidth}
\textbf{Intro}
\hfill 
%( \eighthnote \, 80)   % <<  MODIFICA IL TEMPO
% Metronomo: \eighthnote (ottavo) \quarternote (quarto) \halfnote (due quarti)
\end{minipage}
} 	
\vspace*{-\versesep}
\beginverse*


\nolyrics

%---- Prima riga -----------------------------
\vspace*{-\versesep}
\[E-] \[D] \[E-]	 % \[*D] per indicare le pennate, \rep{2} le ripetizioni

%---- Ogni riga successiva -------------------
%\vspace*{-\versesep}
%\[G] \[C]  \[D]	

%---- Ev Indicazioni -------------------------			
%\textnote{\textit{(Oppure tutta la strofa)} }	

\endverse
\fi



%%%%% RITORNELLO
\beginchorus
\textnote{\textbf{Rit.}}
Il Si\[E-]gnore è la \[D]mia sal\[E-]vezza
e con \[G]lui non \[D]temo \[E-]più 
perché ho nel \[A-]cuore \[B7]la cer\[E-]tezza  
la sal\[C]vezza è \[D]qui con \[E-]me.
\endchorus



%%%%% STROFA
\beginverse		%Oppure \beginverse* se non si vuole il numero di fianco
\memorize 		% <<< DECOMMENTA se si vuole utilizzarne la funzione
%\chordsoff		% <<< DECOMMENTA se vuoi una strofa senza accordi
Ti \[E-]lodo Si\[D]gnore per\[C]ché 
un giorno \[E-]eri lon\[D7]tano da \[G]me, \[D]
\[D] ora invece sei tor\[E-]nato
e mi \[C]hai pre\[D]so con \[E-]te.
\endverse

\beginverse
%\chordsoff
Ber^rete con ^gioia alle ^fonti
alle ^fonti ^della sal^vez^za 
^ e quel giorno voi di^rete:
lo^date il Signore,
invo^cate il suo ^nome.
\endverse

\beginverse
%\chordsoff
F^ate co^noscere ai ^popoli
tutto ^quello che ^lui ha com^piu^to
^ e ricordino per ^sempre
ri^cordino sempre
che il ^suo nome è ^grande.
\endverse

\beginverse
%\chordsoff
Can^tate a chi ha ^fatto gran^dezze
e sia ^fatto sa^pere nel ^mon^do;
^ grida forte la tua ^gioia, 
abi^tante di Sion,
perché ^grande con te è il Si^gnore.
\endverse
\endsong


%-------------------------------------------------------------
%			INIZIO	CANZONE
%-------------------------------------------------------------


%titolo: 	Cantico delle creature
%autore: 	Varnavà, Mancinoni
%tonalita: 	Mi-



%%%%%% TITOLO E IMPOSTAZONI
\beginsong{Cantico delle creature}[by={Per la terra e le tue creature — S. Varnavà, R. Mancinoni}] 	% <<< MODIFICA TITOLO E AUTORE
\transpose{0} 						% <<< TRASPOSIZIONE #TONI (0 nullo)
%\preferflats  %SE VOGLIO FORZARE i bemolle come alterazioni
%\prefersharps %SE VOGLIO FORZARE i # come alterazioni
\momenti{}							% <<< INSERISCI MOMENTI	
% momenti vanno separati da ; e vanno scelti tra:
% Ingresso; Atto penitenziale; Acclamazione al Vangelo; Dopo il Vangelo; Offertorio; Comunione; Ringraziamento; Fine; Santi; Pasqua; Avvento; Natale; Quaresima; Canti Mariani; Battesimo; Prima Comunione; Cresima; Matrimonio; Meditazione; Spezzare del pane;
\ifchorded
	%\textnote{Tonalità migliore }	% <<< EV COMMENTI (tonalità originale/migliore)
\fi




%%%%%% INTRODUZIONE
\ifchorded
\vspace*{\versesep}
\musicnote{
\begin{minipage}{0.48\textwidth}
\textbf{Intro}
\hfill 
%( \eighthnote \, 80)   % <<  MODIFICA IL TEMPO
% Metronomo: \eighthnote (ottavo) \quarternote (quarto) \halfnote (due quarti)
\end{minipage}
} 	
\vspace*{-\versesep}
\beginverse*


\nolyrics

%---- Prima riga -----------------------------
\vspace*{-\versesep}
\[B-] \[E-] \[B-]	 % \[*D] per indicare le pennate, \rep{2} le ripetizioni

%---- Ogni riga successiva -------------------
%\vspace*{-\versesep}
%\[G] \[C]  \[D]	

%---- Ev Indicazioni -------------------------			
%\textnote{\textit{(Oppure tutta la strofa)} }	

\endverse
\fi






%%%%% STROFA
\beginverse		%Oppure \beginverse* se non si vuole il numero di fianco
\memorize 		% <<< DECOMMENTA se si vuole utilizzarne la funzione
%\chordsoff		% <<< DECOMMENTA se vuoi una strofa senza accordi
\[B-]Laudato \[E-]sii mi Si\[B-]gnore,
\[A]per frate \[B-]sole, \[F#]sora \[B-]luna,
\[B-]frate vento, il \[E-]cielo, \[B-]le stelle,
\[A]per sora \[B-]acqua, \[F#]frate \[B-]focu.
\endverse



%%%%% RITORNELLO
\beginchorus
\textnote{\textbf{Rit.}}
\[G]Lau\[A]dato
\[D*]sii \[F#*]mi Si\[B-]gnore,
\[E-]per la \[B-]terra e \[F#]le tue crea\[B-]ture
\[G]Lau\[A]dato
\[D*]sii \[F#*]mi Si\[B-]gnore,
\[E-]per la \[B-]terra e \[F#]le tue crea\[B-]ture
\endchorus




%%%%% STROFA
\beginverse		%Oppure \beginverse* se non si vuole il numero di fianco
%\memorize 		% <<< DECOMMENTA se si vuole utilizzarne la funzione
%\chordsoff		% <<< DECOMMENTA se vuoi una strofa senza accordi
\[B-]Laudato \[E-]sii, mi Si\[B-]gnore,
\[A]quello che \[B-]porta \[F#]la tua \[B-]pace,
\[B-]e saprà \[E-]perdo\[B-]nare,
\[A]per il tuo a\[B-]more \[F#]saprà a\[B-]mare.
\endverse




%%%%% RITORNELLO
\beginchorus
\textnote{\textbf{Rit.}}
\[G]Lau\[A]dato
\[D*]sii \[F#*]mi Si\[B-]gnore,
\[E-]per la \[B-]terra e \[F#]le tue crea\[B-]ture
\[G]Lau\[A]dato
\[D*]sii \[F#*]mi Si\[B-]gnore,
\[E-]per la \[B-]terra e \[F#]le tue crea\[B-]ture
\endchorus




%%%%% STROFA
\beginverse		%Oppure \beginverse* se non si vuole il numero di fianco
%\memorize 		% <<< DECOMMENTA se si vuole utilizzarne la funzione
%\chordsoff		% <<< DECOMMENTA se vuoi una strofa senza accordi
\[B-]Laudato \[E-]sii, mi Si\[B-]gnore,
\[A]per sora \[B-]morte \[F#]corpo\[B-]rale,
\[B-]dalla quale \[E-]homo vi\[B-]vente
\[A]non potrà mai, \[F#]mai scap\[A]pare.
\endverse




%%%%% RITORNELLO
\beginchorus
\textnote{\textbf{Rit.}}
\[G]Lau\[A]dato
\[D*]sii \[F#*]mi Si\[B-]gnore,
\[E-]per la \[B-]terra e \[F#]le tue crea\[B-]ture
\[G]Lau\[A]dato
\[D*]sii \[F#*]mi Si\[B-]gnore,
\[E-]per la \[B-]terra e \[F#]le tue crea\[B-]ture
\endchorus





%%%%% STROFA
\beginverse		%Oppure \beginverse* se non si vuole il numero di fianco
%\memorize 		% <<< DECOMMENTA se si vuole utilizzarne la funzione
%\chordsoff		% <<< DECOMMENTA se vuoi una strofa senza accordi
\[B-]Laudate \[E-]e bene\[A]dite,
\[A]ringrazi\[B-]ate \[F#]e ser\[B-]vite,
\[B-]il Signore \[E-]con humil\[B-]tate,
\[A]ringra\[B-]ziate \[F#]e ser\[B-]vite.
\endverse




%%%%% RITORNELLO
\beginchorus
\textnote{\textbf{Rit.}}
\[G]Lau\[A]dato
\[D*]sii \[F#*]mi Si\[B-]gnore,
\[E-]per la \[B-]terra e \[F#]le tue crea\[B-]ture
\[G]Lau\[A]dato
\[D*]sii \[F#*]mi Si\[B-]gnore,
\[E-]per la \[B-]terra e \[F#]le tue crea\[B-]ture
\endchorus




\endsong
%------------------------------------------------------------
%			FINE CANZONE
%------------------------------------------------------------



%-------------------------------------------------------------
%			INIZIO	CANZONE
%-------------------------------------------------------------


%titolo: 	Cantico di liberazione
%autore: 	
%tonalita: 	La- 



%%%%%% TITOLO E IMPOSTAZONI
\beginsong{Cantico di liberazione}[by={Canto della Veglia Pasquale}] 	% <<< MODIFICA TITOLO E AUTORE
\transpose{0} 						% <<< TRASPOSIZIONE #TONI (0 nullo)
\momenti{Pasqua}							% <<< INSERISCI MOMENTI	
% momenti vanno separati da ; e vanno scelti tra:
% Ingresso; Atto penitenziale; Acclamazione al Vangelo; Dopo il Vangelo; Offertorio; Comunione; Ringraziamento; Fine; Santi; Pasqua; Avvento; Natale; Quaresima; Canti Mariani; Battesimo; Prima Comunione; Cresima; Matrimonio; Meditazione;
\ifchorded
	%\textnote{Tonalità originale }	% <<< EV COMMENTI (tonalità originale/migliore)
\fi




%%%%%% INTRODUZIONE
\ifchorded
\vspace*{\versesep}
\musicnote{
\begin{minipage}{0.48\textwidth}
\textbf{Intro}
\hfill 
%( \eighthnote \, 80)   % <<  MODIFICA IL TEMPO
% Metronomo: \eighthnote (ottavo) \quarternote (quarto) \halfnote (due quarti)
\end{minipage}
} 	
\vspace*{-\versesep}
\beginverse*


\nolyrics

%---- Prima riga -----------------------------
\vspace*{-\versesep}
\[A-] \[G] \[A-] \[G] \rep{2}	 % \[*D] per indicare le pennate, \rep{2} le ripetizioni

%---- Ogni riga successiva -------------------
%\vspace*{-\versesep}
%\[G] \[C]  \[D]	

%---- Ev Indicazioni -------------------------			
%\textnote{\textit{(Oppure tutta la strofa)} }	

\endverse
\fi




%%%%% STROFA
\beginverse		%Oppure \beginverse* se non si vuole il numero di fianco
\memorize 		% <<< DECOMMENTA se si vuole utilizzarne la funzione
%\chordsoff		% <<< DECOMMENTA se vuoi una strofa senza accordi

\[A-]Voglio can\[G]tare in o\[E-]nore di \[A-]Dio
\[D-]perché mi\[C]rabile \[A-]è la sua \[D-]gloria
\[G-]amore, \[C]forza e mio \[A-]can\[A-]to è il Si\[D-]gnore
\[G-]solo a lui \[A]devo la \[D-]mia sal\[B&]vezza:
\[A-]lo esalte\[F]rò, \[A-]è 
il \[G]Dio di mio \[A-]pa\[G]dre  \qquad \[A-] \[G] 

 


\endverse








%%%%% STROFA
\beginverse		%Oppure \beginverse* se non si vuole il numero di fianco
%\memorize 		% <<< DECOMMENTA se si vuole utilizzarne la funzione
%\chordsoff		% <<< DECOMMENTA se vuoi una strofa senza accordi



^Stettero im^mobili le ^acque di^vise
^per riscat^tare il tuo ^popolo, o ^Dio
^poi l'ira ^tua volò ^so^pra il mar ^Rosso,
^carri ed e^sercito ^di Fara^one,
^fior di guerri^e^ri 
som^mersero l'^on^de \qquad  ^ ^ 



\endverse



%%%%%% EV. INTERMEZZO
\beginverse*
\vspace*{1.3\versesep}
{
	\nolyrics
	\textnote{Intermezzo strumentale}
	
	%---- Ev Indicazioni -------------------------			
	\textnote{\textit{[ripetizione dell'intera strofa]}} 
	 
}
\vspace*{\versesep}
\endverse



%%%%% STROFA
\beginverse		%Oppure \beginverse* se non si vuole il numero di fianco
%\memorize 		% <<< DECOMMENTA se si vuole utilizzarne la funzione
%\chordsoff		% <<< DECOMMENTA se vuoi una strofa senza accordi


^Disse il ne^mico: io l'^insegui^rò,
^raggiunge^rò la mia ^preda Isra^ele,
^sguaine^rò la mia ^spa^da ro^vente,
^divide^rò il bot^tino dei ^vinti,
^la mia ^ma^no 
li s^termine^rà ^ \qquad  ^ ^ 

\endverse





%%%%% STROFA
\beginverse		%Oppure \beginverse* se non si vuole il numero di fianco
%\memorize 		% <<< DECOMMENTA se si vuole utilizzarne la funzione
%\chordsoff		% <<< DECOMMENTA se vuoi una strofa senza accordi



^Ma l'ira ^tua soffiò ^sopra il mar ^Rosso,
^acque im^mense copr^iron le schi^ere
^si river^sarono ^ad ^un tuo ^gesto
^e ricop^rirono ^carri e guerri^eri
^che come pi^e^tre 
rag^giunsero il ^fon^do \qquad  ^ ^ 

\endverse




%%%%%% EV. INTERMEZZO
\beginverse*
\vspace*{1.3\versesep}
{
	\nolyrics
	\textnote{Intermezzo strumentale}
	
	%---- Ev Indicazioni -------------------------			
	\textnote{\textit{[ripetizione dell'intera strofa]}} 
	 
}
\vspace*{\versesep}
\endverse


%%%%% STROFA
\beginverse		%Oppure \beginverse* se non si vuole il numero di fianco
%\memorize 		% <<< DECOMMENTA se si vuole utilizzarne la funzione
%\chordsoff		% <<< DECOMMENTA se vuoi una strofa senza accordi


^Chi è come ^te fra gli ^dei, Si^gnore,
^chi è come ^te maes^toso e po^tente
^che ope^rasti un pro^di^gio gran^dioso
^la tua ^destra sten^desti o ^Dio
^il mare apr^is^ti 
a sal^vare i tuoi ^ser^vi \qquad  ^ ^


\endverse





%%%%% STROFA
\beginverse		%Oppure \beginverse* se non si vuole il numero di fianco
%\memorize 		% <<< DECOMMENTA se si vuole utilizzarne la funzione
%\chordsoff		% <<< DECOMMENTA se vuoi una strofa senza accordi

^Questo tuo ^popolo, che ^hai riscat^tato,
^ora lo gu^idi tu ^solo be^nigno
con ^forza e a^more lo ^stai ^condu^cendo
^alla tua ^santa di^mora di^vina
^che le tue ^ma^ni, 
Si^gnore, han fon^da^to. 

\endverse

%%%%%% EV. INTERMEZZO
\beginverse*
\vspace*{1.3\versesep}
{
	\nolyrics
	\textnote{Chiusura strumentale}
	
	\ifchorded

	%---- Prima riga -----------------------------
	\vspace*{-\versesep}
	\[A-] \[G] \[A-] \[G]  
	%---- Ogni riga successiva -------------------
	\vspace*{-\versesep}
	\[A-] \[G*]  \textit{[sospeso...]}


	\fi
	%---- Ev Indicazioni -------------------------			
	%\textnote{\textit{(ripetizione della strofa)}} 
	 
}
\vspace*{\versesep}
\endverse


\endsong
%------------------------------------------------------------
%			FINE CANZONE
%------------------------------------------------------------


%-------------------------------------------------------------
%			INIZIO	CANZONE
%-------------------------------------------------------------


%titolo: 	Canto degli umili
%autore: 	D. Machetta
%tonalita: 	Re



%%%%%% TITOLO E IMPOSTAZONI
\beginsong{Canto degli umili}[by={Cantico di Anna — D. Machetta}] 	% <<< MODIFICA TITOLO E AUTORE
\transpose{0} 						% <<< TRASPOSIZIONE #TONI (0 nullo)
%\preferflats  %SE VOGLIO FORZARE i bemolle come alterazioni
%\prefersharps %SE VOGLIO FORZARE i # come alterazioni
\momenti{}							% <<< INSERISCI MOMENTI	
% momenti vanno separati da ; e vanno scelti tra:
% Ingresso; Atto penitenziale; Acclamazione al Vangelo; Dopo il Vangelo; Offertorio; Comunione; Ringraziamento; Fine; Santi; Pasqua; Avvento; Natale; Quaresima; Canti Mariani; Battesimo; Prima Comunione; Cresima; Matrimonio; Meditazione; Spezzare del pane;
\ifchorded
	%\textnote{Tonalità migliore }	% <<< EV COMMENTI (tonalità originale/migliore)
\fi




%%%%%% INTRODUZIONE
\ifchorded
\vspace*{\versesep}
\musicnote{
\begin{minipage}{0.48\textwidth}
\textbf{Intro}
\hfill 
%( \eighthnote \, 80)   % <<  MODIFICA IL TEMPO
% Metronomo: \eighthnote (ottavo) \quarternote (quarto) \halfnote (due quarti)
\end{minipage}
} 	
\vspace*{-\versesep}
\beginverse*


\nolyrics

%---- Prima riga -----------------------------
\vspace*{-\versesep}
\[D] \[G] \[D]	 % \[*D] per indicare le pennate, \rep{2} le ripetizioni

%---- Ogni riga successiva -------------------
%\vspace*{-\versesep}
%\[G] \[C]  \[D]	

%---- Ev Indicazioni -------------------------			
%\textnote{\textit{(Oppure tutta la strofa)} }	

\endverse
\fi









%%%%% STROFA
\beginverse		%Oppure \beginverse* se non si vuole il numero di fianco
%\memorize 		% <<< DECOMMENTA se si vuole utilizzarne la funzione
%\chordsoff		% <<< DECOMMENTA se vuoi una strofa senza accordi

\[D]L'arco dei \[G]forti s'è spez\[D]zato,
\[B-]gli umili si \[F#-]vestono \[B-]della tua \[D]forza.
\[G]Grande è il nostro \[A7]Dio!

\endverse




%%%%% RITORNELLO
\beginchorus
\textnote{\textbf{Rit.}}

\[D]Non potrò ta\[F#-]cere, mio \[B-]Signore,
i bene\[G]fici del tuo a\[A7]mo\[D]re.

\endchorus



%%%%% STROFA
\beginverse		%Oppure \beginverse* se non si vuole il numero di fianco
%\memorize 		% <<< DECOMMENTA se si vuole utilizzarne la funzione
%\chordsoff		% <<< DECOMMENTA se vuoi una strofa senza accordi

\[D]Dio solleva il \[G]misero dal \[D]fango,
\[B-]libera il \[F#-]povero \[B-]dall'ingius\[D]tizia.
\[G]Grande è il nostro \[A7]Dio!

\endverse




%%%%% STROFA
\beginverse		%Oppure \beginverse* se non si vuole il numero di fianco
%\memorize 		% <<< DECOMMENTA se si vuole utilizzarne la funzione
%\chordsoff		% <<< DECOMMENTA se vuoi una strofa senza accordi


\[D]Dio tiene i \[G]cardini del \[D]mondo,
\[B-]veglia sui \[F#-]giusti, \[B-]guida i loro \[D]passi.
\[G]Grande è il nostro \[A7]Dio!

\endverse



\endsong
%------------------------------------------------------------
%			FINE CANZONE
%------------------------------------------------------------




%-------------------------------------------------------------
%			INIZIO	CANZONE
%-------------------------------------------------------------


%titolo: 	Santo Ricci
%autore: 	Daniele Ricci
%tonalita: 	Sol 



%%%%%% TITOLO E IMPOSTAZONI
\beginsong{Canto per te Gesù}[by={G. Boretti, J. Akepsimas, A. Fant, M. Deflorian}] 	% <<< MODIFICA TITOLO E AUTORE
\transpose{0} 						% <<< TRASPOSIZIONE #TONI (0 nullo)
%\preferflats  %SE VOGLIO FORZARE i bemolle come alterazioni
%\prefersharps %SE VOGLIO FORZARE i # come alterazioni
\momenti{Ingresso; Comunione }							% <<< INSERISCI MOMENTI	
% momenti vanno separati da ; e vanno scelti tra:
% Ingresso; Atto penitenziale; Acclamazione al Vangelo; Dopo il Vangelo; Offertorio; Comunione; Ringraziamento; Fine; Santi; Pasqua; Avvento; Natale; Quaresima; Canti Mariani; Battesimo; Prima Comunione; Cresima; Matrimonio; Meditazione; Spezzare del pane;
\ifchorded
	%\textnote{Tonalità migliore }	% <<< EV COMMENTI (tonalità originale/migliore)
\fi




%%%%%% INTRODUZIONE
\ifchorded
\vspace*{\versesep}
\musicnote{
\begin{minipage}{0.48\textwidth}
\textbf{Intro}
\hfill 
%( \eighthnote \, 80)   % <<  MODIFICA IL TEMPO
% Metronomo: \eighthnote (ottavo) \quarternote (quarto) \halfnote (due quarti)
\end{minipage}
} 	
\vspace*{-\versesep}
\beginverse*


\nolyrics

%---- Prima riga -----------------------------
\vspace*{-\versesep}
\[F] \[C] \[F]	 % \[*D] per indicare le pennate, \rep{2} le ripetizioni

%---- Ogni riga successiva -------------------
%\vspace*{-\versesep}
%\[G] \[C]  \[D]	

%---- Ev Indicazioni -------------------------			
%\textnote{\textit{(Oppure tutta la strofa)} }	

\endverse
\fi


%%%%% RITORNELLO
\beginchorus
\textnote{\textbf{Rit.}}

\[F]Canto per \[D-]Te, Gesù,
\[B&]canto per \[F]Te!
\[G-]Oggi mi \[F]rendi fe\[G]li\[C]ce.
\[F]Canto per \[D-]Te, Gesù,
\[B&]canto per \[F]Te.
\[G-]Vieni e \[F*]vivi \[C*]con \[F]me.

\endchorus




%%%%% STROFA
\beginverse		%Oppure \beginverse* se non si vuole il numero di fianco
\memorize 		% <<< DECOMMENTA se si vuole utilizzarne la funzione
%\chordsoff		% <<< DECOMMENTA se vuoi una strofa senza accordi

Hai \[F]scelto la mia \[D-]casa
per \[B&]abitar con \[C]me:
è \[G-]piccola, Si\[D-]gnore,
ma \[B&]io vivrò con \[C]Te.

\endverse







%%%%% STROFA
\beginverse		%Oppure \beginverse* se non si vuole il numero di fianco
%\memorize 		% <<< DECOMMENTA se si vuole utilizzarne la funzione
%\chordsoff		% <<< DECOMMENTA se vuoi una strofa senza accordi


Nel ^Pane che man^giamo,
Ge^sù, Tu vieni a ^noi:
e un ^giorno su nel ^cielo
sa^rem fratelli ^tuoi.

\endverse

%%%%% STROFA
\beginverse		%Oppure \beginverse* se non si vuole il numero di fianco
%\memorize 		% <<< DECOMMENTA se si vuole utilizzarne la funzione
\chordsoff		% <<< DECOMMENTA se vuoi una strofa senza accordi


Mi parli con amore
di grandi verità:
ti ascolterò, Signore,
con gioia e fedeltà.


\endverse


%%%%% STROFA
\beginverse		%Oppure \beginverse* se non si vuole il numero di fianco
%\memorize 		% <<< DECOMMENTA se si vuole utilizzarne la funzione
\chordsoff		% <<< DECOMMENTA se vuoi una strofa senza accordi



Gesù, Tu sei risorto
e vivi in mezzo a noi:
fedele nostro amico,
ci rendi amici tuoi.


\endverse

%%%%% STROFA
\beginverse		%Oppure \beginverse* se non si vuole il numero di fianco
%\memorize 		% <<< DECOMMENTA se si vuole utilizzarne la funzione
%\chordsoff		% <<< DECOMMENTA se vuoi una strofa senza accordi


I ^bimbi che abbrac^ciavi
ti a^mavano, Ge^sù,
a ^Te noi promett^iamo
di a^marti sempre ^più.

\endverse


%%%%% STROFA
\beginverse		%Oppure \beginverse* se non si vuole il numero di fianco
%\memorize 		% <<< DECOMMENTA se si vuole utilizzarne la funzione
\chordsoff		% <<< DECOMMENTA se vuoi una strofa senza accordi


Passavi tra la gente,
chiedevi la bontà:
con Te l'amore vero
più facile sarà.

\endverse


%%%%% STROFA
\beginverse		%Oppure \beginverse* se non si vuole il numero di fianco
%\memorize 		% <<< DECOMMENTA se si vuole utilizzarne la funzione
\chordsoff		% <<< DECOMMENTA se vuoi una strofa senza accordi


La strada spesso è buia:
il Sole chi sarà?
Amici non temete:
Gesù vi guiderà.

\endverse


%%%%% STROFA
\beginverse		%Oppure \beginverse* se non si vuole il numero di fianco
%\memorize 		% <<< DECOMMENTA se si vuole utilizzarne la funzione
\chordsoff		% <<< DECOMMENTA se vuoi una strofa senza accordi



Quest'oggi ho tanti amici
che pregano con me:
ma il mondo è ancor più grande
io l'amerò con Te!

\endverse


%%%%% STROFA
\beginverse		%Oppure \beginverse* se non si vuole il numero di fianco
%\memorize 		% <<< DECOMMENTA se si vuole utilizzarne la funzione
\chordsoff		% <<< DECOMMENTA se vuoi una strofa senza accordi


La gioia del mio cuore
a tutti canterò
Perché Gesù mi ha detto:
"Per sempre ti amerò!"

\endverse

%%%%% STROFA
\beginverse		%Oppure \beginverse* se non si vuole il numero di fianco
%\memorize 		% <<< DECOMMENTA se si vuole utilizzarne la funzione
%\chordsoff		% <<< DECOMMENTA se vuoi una strofa senza accordi

Lo ^Spirito che ^doni
la ^terra cambie^rà:
Si^gnore, noi spe^riamo
nel ^Regno che ver^rà.

\endverse


\endsong
%------------------------------------------------------------
%			FINE CANZONE
%------------------------------------------------------------


%-------------------------------------------------------------
%			INIZIO	CANZONE
%-------------------------------------------------------------


%titolo:    Come canto d'amore
%autore: 	P. Sequeri
%tonalita: 	Re 



%%%%%% TITOLO E IMPOSTAZONI
\beginsong{Come canto d'amore}[by={P. Sequeri}] 	% <<< MODIFICA TITOLO E AUTORE
\transpose{0} 						% <<< TRASPOSIZIONE #TONI (0 nullo)
\momenti{Ingresso; Cresima; Ringraziamento; Fine}							% <<< INSERISCI MOMENTI	
% momenti vanno separati da ; e vanno scelti tra:
% Ingresso; Atto penitenziale; Acclamazione al Vangelo; Dopo il Vangelo; Offertorio; Comunione; Ringraziamento; Fine; Santi; Pasqua; Avvento; Natale; Quaresima; Canti Mariani; Battesimo; Prima Comunione; Cresima; Matrimonio; Meditazione; Spezzare del pane;
\ifchorded
	%\textnote{Tonalità originale }	% <<< EV COMMENTI (tonalità originale/migliore)
\fi




%%%%%% INTRODUZIONE
\ifchorded
\vspace*{\versesep}
\musicnote{
\begin{minipage}{0.48\textwidth}
\textbf{Intro}
\hfill 
%( \eighthnote \, 80)   % <<  MODIFICA IL TEMPO
% Metronomo: \eighthnote (ottavo) \quarternote (quarto) \halfnote (due quarti)
\end{minipage}
} 	
\vspace*{-\versesep}
\beginverse*

\nolyrics

%---- Prima riga -----------------------------
\vspace*{-\versesep}
 \[D]  \[E-] \[D] \[E-]	 % \[*D] per indicare le pennate, \rep{2} le ripetizioni

%---- Ogni riga successiva -------------------
%\vspace*{-\versesep}
%\[G] \[C]  \[D]	

%---- Ev Indicazioni -------------------------			
%\textnote{\textit{(Oppure tutta la strofa)} }	

\endverse
\fi




%%%%% STROFA
\beginverse		%Oppure \beginverse* se non si vuole il numero di fianco
\memorize 		% <<< DECOMMENTA se si vuole utilizzarne la funzione
%\chordsoff		% <<< DECOMMENTA se vuoi una strofa senza accordi

\[D]Con il mio canto, \[E-]dolce Signore, 
\[F#-]voglio danzare \[G]questa mia \[D]gioia,
\[G]voglio destare \[E-]tutte le cose,
\[E7]un mondo nuovo \[A]voglio cantare.

\endverse

%%%%% STROFA
\beginverse*		%Oppure \beginverse* se non si vuole il numero di fianco
%\memorize 		% <<< DECOMMENTA se si vuole utilizzarne la funzione
%\chordsoff		% <<< DECOMMENTA se vuoi una strofa senza accordi

^Con il mio canto, ^dolce Signore, 
^voglio riempire ^lunghi si^lenzi,
^voglio abitare s^guardi di pace,
^il tuo perdono ^voglio cantare.

\endverse





%%%%% RITORNELLO
\beginchorus
\textnote{\textbf{Rit.}}

\[D]Tu \[E-]sei \[(E-)]per \[F#-]me, 
\[G]co\[D]me un \[G]can\[E-]to \[(E-)]d'a\[E7]mo\[A]re.
\[D]Res\[E-]ta \[(E-)]con \[F#-]noi, 
\[G]fi\[D]no al \[G]nuo\[E-]vo \[(E-)]mat\[E7]ti\[A]no.

\endchorus



%%%%% STROFA
\beginverse		%Oppure \beginverse* se non si vuole il numero di fianco
%\memorize 		% <<< DECOMMENTA se si vuole utilizzarne la funzione
%\chordsoff		% <<< DECOMMENTA se vuoi una strofa senza accordi

^Con il mio canto, ^dolce Signore, 
^voglio plasmare ^gesti d'a^more,
^voglio arrivare ^oltre la morte,
^la tua speranza ^voglio cantare.

\endverse

%%%%% STROFA
\beginverse*		%Oppure \beginverse* se non si vuole il numero di fianco
%\memorize 		% <<< DECOMMENTA se si vuole utilizzarne la funzione
%\chordsoff		% <<< DECOMMENTA se vuoi una strofa senza accordi

^Con il mio canto, ^dolce Signore, 
^voglio gettare ^semi di ^luce,
^voglio sognare ^cose mai viste,
^la tua bellezza ^voglio cantare.

\endverse



%%%%% RITORNELLO
\beginchorus
\textnote{\textbf{Rit.}}

\[D]Tu \[E-]sei \[(E-)]per \[F#-]me, 
\[G]co\[D]me un \[G]can\[E-]to  \[(E-)]d'a\[E7]mo\[A]re.
\[D]Res\[E-]ta \[(E-)]con \[F#-]noi, 
\[G]fi\[D]no al \[G]nuo\[E-]vo \[(E-)]mat\[E7]ti\[A]no.

\endchorus

%%%%% STROFA
\beginverse*		%Oppure \beginverse* se non si vuole il numero di fianco
%\memorize 		% <<< DECOMMENTA se si vuole utilizzarne la funzione
%\chordsoff		% <<< DECOMMENTA se vuoi una strofa senza accordi
\textnote{\textit{[cambio di tempo: molto più lento]}}
\[D*] Se tu mi ascolti, \[E-*] dolce Signore,
\[F#-*] questo mio canto \[G*] sarà una \[D*]vita,
\[G*] e sarà bello \[E-*]  vivere insieme,
\[E7*] finché la vita un \[A*]canto sarà.

\endverse


%%%%% STROFA
\beginverse*		%Oppure \beginverse* se non si vuole il numero di fianco
%\memorize 		% <<< DECOMMENTA se si vuole utilizzarne la funzione
%\chordsoff		% <<< DECOMMENTA se vuoi una strofa senza accordi
\vspace*{\versesep}
\textnote{\textit{[dall'inizio con il tempo più veloce]}}
\vspace*{-\versesep}
Con il mio canto...
\endverse


\endsong
%------------------------------------------------------------
%			FINE CANZONE
%------------------------------------------------------------



%-------------------------------------------------------------
%			INIZIO	CANZONE
%-------------------------------------------------------------


%titolo: 	Come fuoco vivo
%autore: 	Gen Verde, Gen Rosso
%tonalita: 	Do



%%%%%% TITOLO E IMPOSTAZONI
\beginsong{Come fuoco vivo}[by={Gen\ Verde, Gen\ Rosso}] 	% <<< MODIFICA TITOLO E AUTORE
\transpose{0} 						% <<< TRASPOSIZIONE #TONI (0 nullo)
\momenti{Comunione; Cresima}							% <<< INSERISCI MOMENTI	
% momenti vanno separati da ; e vanno scelti tra:
% Ingresso; Atto penitenziale; Acclamazione al Vangelo; Dopo il Vangelo; Offertorio; Comunione; Ringraziamento; Fine; Santi; Pasqua; Avvento; Natale; Quaresima; Canti Mariani; Battesimo; Prima Comunione; Cresima; Matrimonio; Meditazione; Spezzare del pane;
\ifchorded
	%\textnote{Tonalità migliore }	% <<< EV COMMENTI (tonalità originale/migliore)
\fi




%%%%%% INTRODUZIONE
\ifchorded
\vspace*{\versesep}
\musicnote{
\begin{minipage}{0.48\textwidth}
\textbf{Intro}
\hfill 
%( \eighthnote \, 80)   % <<  MODIFICA IL TEMPO
% Metronomo: \eighthnote (ottavo) \quarternote (quarto) \halfnote (due quarti)
\end{minipage}
} 	
\vspace*{-\versesep}
\beginverse*


\nolyrics

%---- Prima riga -----------------------------
\vspace*{-\versesep}
\[C] \[C] \[D-*]\[C] \[C] % \[*D] per indicare le pennate, \rep{2} le ripetizioni

%---- Ogni riga successiva -------------------
\vspace*{-\versesep}
  \[G*] \[A-7] \[A-7]

%---- Ogni riga successiva -------------------
\vspace*{-\versesep}
 \[E-7*] \[F] \[F] \quad \[G]\[G]

%---- Ev Indicazioni -------------------------			
%\textnote{\textit{(Oppure tutta la strofa)} }	

\endverse
\fi






%%%%% RITORNELLO
\beginchorus
\textnote{\textbf{Rit.}}
Come \[C]fuoco \[G]vivo si ac\[A-]cende in \[A-]noi
un'im\[D-7]mensa \[G]felici\[C]tà \[C]
che mai \[F]più nes\[G]suno ci \[C]toglie\[F]rà
\[D-7]perché tu \[D-7]sei ritor\[G4]nato. \[G]
Chi po\[C]trà ta\[G]cere, da \[A-]ora in \[A-]poi,
che sei \[D-]tu in cam\[G]mino con \[C]noi, \[C]
che la \[F]morte è \[G]vinta per \[C]sempre,
\[F]che \[D-7]ci hai rido\[D-7]nato la \[G4]vita? \[G]
\endchorus



%%%%% STROFA
\beginverse		%Oppure \beginverse* se non si vuole il numero di fianco
\memorize 		% <<< DECOMMENTA se si vuole utilizzarne la funzione
%\chordsoff		% <<< DECOMMENTA se vuoi una strofa senza accordi
\[A-] Spezzi il \[A-]pane da\[F]vanti a \[C]noi \[C]
mentre il \[C]sole è al tra\[G4]monto: \[G]
\[G-]o\[G-]ra gli \[A]occhi ti \[D-]vedono, \[F] sei \[F]tu! 
\[G4]Resta con \[G]noi.
\endverse




%%%%% STROFA
\beginverse		%Oppure \beginverse* se non si vuole il numero di fianco
%\memorize 		% <<< DECOMMENTA se si vuole utilizzarne la funzione
%\chordsoff		% <<< DECOMMENTA se vuoi una strofa senza accordi
^ E per ^sempre ti ^mostre^rai ^
in quel ^gesto d'a^more: ^
^ma^ni che ^ancora ^spezzano ^ pa^ne d'^eterni^tà.
\endverse



%%%%%% EV. INTERMEZZO
\beginverse*
\vspace*{1.3\versesep}
{
	
	\textnote{\textbf{Finale} \textit{[humming]}}
	
	\ifchorded

	%---- Prima riga -----------------------------
	%\vspace*{-\versesep}
	\[C]Uhmmm... \[C] \[D-*]\[C] \[C] % \[*D] per indicare le pennate, \rep{2} le ripetizioni

	%---- Ogni riga successiva -------------------
	\vspace*{-\versesep}
	\nolyrics  \[G*] \[A-7] \[A-7] 

	%---- Ogni riga successiva -------------------
	\vspace*{-\versesep}
	  \[E-7*] \[F] \[F] \quad \[G] \quad \[C*]

	\else 
	\chordsoff Uhmmm...
	\fi
	%---- Ev Indicazioni -------------------------			
	%\textnote{\textit{(ripetizione della strofa)}} 
	 
}
\vspace*{\versesep}
\endverse



\endsong
%------------------------------------------------------------
%			FINE CANZONE
%------------------------------------------------------------



%titolo{Come l'aurora verrai}
%autore{Gen Verde}
%album{Cerco il tuo volto}
%tonalita{Re-}
%famiglia{Liturgica}
%gruppo{}
%momenti{Avvento}
%identificatore{come_l_aurora_verrai}
%data_revisione{2011_12_31}
%trascrittore{Francesco Endrici - Manuel Toniato}
\beginsong{Come l'aurora verrai}[by={Gen\ Verde}]
\beginverse
\[D-]Come l'au\[A-7/C#]rora ver\[D-]rai
le tenebre in \[A-]luce cambie\[B&]rai
tu per \[F/A]noi, Si\[B&]gno\[C]re.
\[D-]Come la pi\[A-7/C#]oggia ca\[D-]drai
sui nostri de\[A-]serti scende\[B&]rai
scorre\[F/A]{rà l'a}\[B&]{mo}\[C]re
\endverse

\beginchorus
\[B&]Tutti i nostri senti\[C]eri percorrer\[D-]{ai,} \[A-] \[D-] 
\[B&]tutti i figli dis\[C]persi raccoglier\[D-]{ai,} \[A-] \[D-] 
\[B&]chiamerai da ogni \[C]terra
il tuo \[A]popo\[B&]lo,
\[G-]in eterno ti a\[G-7]vremo con \[G-/C]{noi.} \[C] 
\endchorus

\beginverse
\chordsoff
Re di giustizia sarai,
le spade in aratri forgerai:
ci darai la pace.
Lupo ed agnello vedrai
insieme sui prati dove mai
tornerà la notte.
\endverse

\beginverse
\chordsoff
Dio di salvezza tu sei
e come una stella sorgerai
su di noi per sempre.
E chi non vede, vedrà,
chi ha chiusi gli orecchi sentirà,
canterà di gioia.
\endverse
\endsong



%-------------------------------------------------------------
%			INIZIO	CANZONE
%-------------------------------------------------------------


%titolo: 	Come Maria
%autore: 	Gen\ Rosso
%tonalita: 	La-



%%%%%% TITOLO E IMPOSTAZONI
\beginsong{Come Maria}[by={Gen\ Rosso}]	% <<< MODIFICA TITOLO E AUTORE
\transpose{0} 						% <<< TRASPOSIZIONE #TONI (0 nullo)
\momenti{Canti Mariani; Offertorio}							% <<< INSERISCI MOMENTI	
% momenti vanno separati da ; e vanno scelti tra:
% Ingresso; Atto penitenziale; Acclamazione al Vangelo; Dopo il Vangelo; Offertorio; Comunione; Ringraziamento; Fine; Santi; Pasqua; Avvento; Natale; Quaresima; Canti Mariani; Battesimo; Prima Comunione; Cresima; Matrimonio; Meditazione; Spezzare del pane;
\ifchorded
	%\textnote{Tonalità migliore }	% <<< EV COMMENTI (tonalità originale/migliore)
\fi


%%%%%% INTRODUZIONE
\ifchorded
\vspace*{\versesep}
\textnote{Intro: \qquad \qquad  }%(\eighthnote 116) % <<  MODIFICA IL TEMPO
% Metronomo: \eighthnote (ottavo) \quarternote (quarto) \halfnote (due quarti)
\vspace*{-\versesep}
\beginverse*

\nolyrics

%---- Prima riga -----------------------------
\vspace*{-\versesep}
\[A-] \[C] \[E-] \[A-]	 % \[*D] per indicare le pennate, \rep{2} le ripetizioni

%---- Ogni riga successiva -------------------
%\vspace*{-\versesep}
%\[G] \[C]  \[D]	

%---- Ev Indicazioni -------------------------			
%\textnote{\textit{(Oppure tutta la strofa)} }	

\endverse
\fi



%%%%% STROFA
\beginverse		%Oppure \beginverse* se non si vuole il numero di fianco
\memorize 		% <<< DECOMMENTA se si vuole utilizzarne la funzione
%\chordsoff		% <<< DECOMMENTA se vuoi una strofa senza accordi
\[A-] Vogliamo vivere Si\[C]gnore, 
\[A-] offrendo a te la nostra \[E-]vita, 
\[F] Con questo pane e questo \[D]vino 
\[A-] accetta quello che noi \[E]siamo. 
\[A-] Vogliamo vivere Si\[C]gnore, 
\[A-] abbandonati alla tua \[E-]voce, 
\[F] staccati dalle cose \[D]vane, 
\[A-] fissati nella vita \[E4]vera. \[E]
\endverse


%%%%% RITORNELLO
\beginchorus
\textnote{\textbf{Rit.}}
Vo\[A]gliamo \[D]vivere \[E] come Ma\[F#-]ria 
\[D] l'irraggiun\[E]gibile, \[D] la madre a\[E]mata, 
\[D] che vince il \[E]mondo con l'a\[C#-]more 
\[D ]e offrire s\[B-]empre \brk la Tua \[C#]vita che \[D]viene dal \[A]cielo.
\endchorus




%%%%% STROFA
\beginverse		%Oppure \beginverse* se non si vuole il numero di fianco
%\memorize 		% <<< DECOMMENTA se si vuole utilizzarne la funzione
%\chordsoff		% <<< DECOMMENTA se vuoi una strofa senza accordi
^ Accetta dalle nostre ^mani 
^ come un'offerta a te gra^dita 
^ i desideri di ogni ^cuore, 
^ le ansie della nostra ^vita. 
^ Vogliamo vivere Si^gnore, 
^ accesi dalle Tue Pa^role, 
^ per riportare in ogni ^uomo 
^ la fiamma viva del Tuo a^more. ^
\endverse




%%%%%% EV. FINALE
\beginchorus %oppure \beginverse*
\vspace*{1.3\versesep}
\textnote{Finale} %<<< EV. INDICAZIONI
\[D] e offrire \[B-]sempre la Tua \[C#]vita che \[D]viene dal \[A]cielo.
\endchorus




\endsong
%------------------------------------------------------------
%			FINE CANZONE
%------------------------------------------------------------


%-------------------------------------------------------------
%			INIZIO	CANZONE
%-------------------------------------------------------------


%titolo: 	Come Te
%autore: 	Gen Rosso
%tonalita: 	Do



%%%%%% TITOLO E IMPOSTAZONI
\beginsong{Come Te}[by={Gen Rosso}] 	% <<< MODIFICA TITOLO E AUTORE
\transpose{0} 						% <<< TRASPOSIZIONE #TONI (0 nullo)
\momenti{Comunione; Ringraziamento; Natale; Avvento}							% <<< INSERISCI MOMENTI	
% momenti vanno separati da ; e vanno scelti tra:
% Ingresso; Atto penitenziale; Acclamazione al Vangelo; Dopo il Vangelo; Offertorio; Comunione; Ringraziamento; Fine; Santi; Pasqua; Avvento; Natale; Quaresima; Canti Mariani; Battesimo; Prima Comunione; Cresima; Matrimonio; Meditazione;
\ifchorded
	%\textnote{Tonalità originale }	% <<< EV COMMENTI (tonalità originale/migliore)
\fi


%%%%%% INTRODUZIONE
\ifchorded
\vspace*{\versesep}
\textnote{Intro: \qquad \qquad  }%(\eighthnote 116) % << MODIFICA IL TEMPO
% Metronomo: \eighthnote (ottavo) \quarternote (quarto) \halfnote (due quarti)
\vspace*{-\versesep}
\beginverse*

\nolyrics

%---- Prima riga -----------------------------
\vspace*{-\versesep}
\[C] \[G]  \[C]	 \[G] % \[*D] per indicare le pennate, \rep{2} le ripetizioni

%---- Ogni riga successiva -------------------
%\vspace*{-\versesep}
%\[G] \[C]  \[D]	

%---- Ev Indicazioni -------------------------			
%\textnote{\textit{(Oppure tutta la strofa)} }	

\endverse
\fi




%%%%% STROFA
\beginverse		%Oppure \beginverse* se non si vuole il numero di fianco
\memorize 		% <<< DECOMMENTA se si vuole utilizzarne la funzione
%\chordsoff		& <<< DECOMMENTA se vuoi una strofa senza accordi

\[C]Come \[G]Te, che sei \[C]sceso dal \[G]cielo \[C] \[G]
ad inse\[D-]gnarci l’a\[F]more di \[G]Dio
\[D-7] e hai preso su di \[A-]Te
la \[(*G)]nostra \[C]povera e \[F]fragile \[D-]umani\[G]tà.

\endverse




\beginverse*

^Come ^Te, che non ^ti sei te^nuto ^  ^
come se^greto l’a^more di ^Dio,
^ ma sei venuto ^qui
a ^rinno^vare la ^vita dell’^umani^tà.

\endverse




\beginverse*

\[G]Io non mi tirerò ind\[F]ietro
io non avrò più p\[C]aura
di dare tutto di \[G]me.  \[*F] \[*G]

\endverse


%%%%% RITORNELLO
\beginchorus
\textnote{\textbf{Rit.}}

\[C]Per am\[F]ore dell’\[C]uomo,\[F]
d’\[D-]ogni u\[F]omo come \[G]me
\[C]mi son \[F]fatto si\[C]lenzio\[F]
\[D-] per diven\[F]tare come \[G]Te.
\[C] Per \[G]amore \[A-]tuo 
\[G] mi farò \[F]servo d’ogni \[G]uomo che \[C]vive\[G]
\[A-] servo \[C]d’ogni u\[F]omo
\[C] per a\[G]mo\[F]re. \[G]

\endchorus






%%%%%% EV. INTERMEZZO
\beginverse*
\vspace*{1.3\versesep}
{
	\nolyrics
	\ifchorded

	\textnote{Intermezzo }

	%---- Prima riga -----------------------------
	\vspace*{-\versesep}
	\[C] \[G]  \[C]	 \[G]


	\fi
	%---- Ev Indicazioni -------------------------			
	%\textnote{\textit{(ripetizione della strofa)}} 
	 
}
\vspace*{\versesep}
\endverse




%%%%% STROFA
\beginverse

^Come ^Te che hai la^sciato le ^stelle ^ ^
per farti ^proprio come ^uno di ^noi,
^ senza tenere ^niente
hai ^dato ^anche la ^vita, hai pa^gato per ^noi.

\endverse




\beginverse*

\[G]Davanti a questo mis\[F]tero
come potrò ricam\[C]biare,
che cosa mai potrò f\[G]are?  \[*F]\[*G]

\endverse









\endsong
%------------------------------------------------------------
%			FINE CANZONE
%------------------------------------------------------------

%-------------------------------------------------------------
%			INIZIO	CANZONE
%-------------------------------------------------------------


%titolo: 	Come ti ama Dio
%autore: 	Anonimo
%tonalita:  DO



%%%%%% TITOLO E IMPOSTAZONI
\beginsong{Come ti ama Dio}[by={Anonimo}] 	% <<< MODIFICA TITOLO E AUTORE
\transpose{0} 						% <<< TRASPOSIZIONE #TONI (0 nullo)
%\preferflats  %SE VOGLIO FORZARE i bemolle come alterazioni
%\prefersharps %SE VOGLIO FORZARE i # come alterazioni
\momenti{Matrimonio}							% <<< INSERISCI MOMENTI	
% momenti vanno separati da ; e vanno scelti tra:
% Ingresso; Atto penitenziale; Acclamazione al Vangelo; Dopo il Vangelo; Offertorio; Comunione; Ringraziamento; Fine; Santi; Pasqua; Avvento; Natale; Quaresima; Canti Mariani; Battesimo; Prima Comunione; Cresima; Matrimonio; Meditazione; Spezzare del pane;
\ifchorded
	%\textnote{$\bigstar$ Tonalità migliore }	% <<< EV COMMENTI (tonalità originale\migliore)
\fi


%%%%%% INTRODUZIONE
\ifchorded
\vspace*{\versesep}
\musicnote{
\begin{minipage}{0.48\textwidth}
\textbf{Intro}
\hfill 
%( \eighthnote \, 80)   % <<  MODIFICA IL TEMPO
% Metronomo: \eighthnote (ottavo) \quarternote (quarto) \halfnote (due quarti)
\end{minipage}
} 	
\vspace*{-\versesep}
\beginverse*

\nolyrics

%---- Prima riga -----------------------------
\vspace*{-\versesep}
\[C] \[A-] \[F]	 \[G] \rep{2} % \[*D] per indicare le pennate, \rep{2} le ripetizioni

%---- Ogni riga successiva -------------------
%\vspace*{-\versesep}
%\[G] \[C]  \[D]	

%---- Ev Indicazioni -------------------------			
%\textnote{\textit{[oppure tutta la strofa]} }	

\endverse
\fi




%%%%% STROFA
\beginverse		%Oppure \beginverse* se non si vuole il numero di fianco
\memorize 		% <<< DECOMMENTA se si vuole utilizzarne la funzione
%\chordsoff		% <<< DECOMMENTA se vuoi una strofa senza accordi

\[C]Io vorrei sa\[A-]perti amare \[F]come Dio
\[G]che ti prende per \[C]mano ma ti \[A-]lascia anche \[F]andare.
\[G]Vorrei saperti \[C]amare senza \[A-]farti mai \[F]domande,
\[G]felice perchè \[C]esisti e co\[A-]sì io posso \[F]darti \brk il \[G]meglio di \[C]me.

\endverse




%%%%% RITORNELLO
\beginchorus
\textnote{\textbf{Rit.}}

Con la \[G]forza del \[A-]mare, \brk l'e\[F]ternità dei \[C]giorni,
la \[G]gioia dei \[A-]voli, \brk la \[F]pace della \[C]sera,
l'im\[G]mensità del \[A-]cielo: \brk \[F]come ti ama \[C]Dio.

\endchorus



%%%%% STROFA
\beginverse		%Oppure \beginverse* se non si vuole il numero di fianco
%\memorize 		% <<< DECOMMENTA se si vuole utilizzarne la funzione
%\chordsoff		% <<< DECOMMENTA se vuoi una strofa senza accordi

\[C]Io vorrei sa\[A-]perti amare \brk \[F]come ti ama \[G]Dio
che ti cono\[C]sce e ti \[A-]accetta \brk come \[F]sei.
\[G]Tenerti fra le \[C]mani \brk come \[A-]voli nell'\[F]azzurro,
\[G]felice perchè \[C]esisti e \[A-]così io posso \[F]darti \brk il \[G]meglio di \[C]me.
 

\endverse



%%%%% RITORNELLO
\beginchorus
\textnote{\textbf{Rit.}}

Con la \[G]forza del \[A-]mare, \brk l'e\[F]ternità dei \[C]giorni,
la \[G]gioia dei \[A-]voli, \brk la \[F]pace della \[C]sera,
l'im\[G]mensità del \[A-]cielo: \brk \[F]come ti ama \[C]Dio.

\endchorus



%%%%% STROFA
\beginverse		%Oppure \beginverse* se non si vuole il numero di fianco
%\memorize 		% <<< DECOMMENTA se si vuole utilizzarne la funzione
%\chordsoff		% <<< DECOMMENTA se vuoi una strofa senza accordi

\[C]Io vorrei sa\[A-]perti amare \brk \[F]come ti ama \[G]Dio
che ti fa mi\[C]gliore con l'a\[A-]more \brk che ti \[F]dona.
\[G]Seguirti fra la \[C]gente \brk con la \[A-]gioia che hai \[F]dentro,
\[G]felice perchè \[C]esisti e \[A-]così io posso \[F]darti \brk il \[G]meglio di \[C]me.
 

\endverse


%%%%% RITORNELLO
\beginchorus
\textnote{\textbf{Rit.}}

Con la \[G]forza del \[A-]mare, \brk l'e\[F]ternità dei \[C]giorni,
la \[G]gioia dei \[A-]voli,\brk  la \[F]pace della \[C]sera,
l'im\[G]mensità del \[A-]cielo: \brk \[F]come ti ama \[C]Dio.

\endchorus



\endsong
%------------------------------------------------------------
%			FINE CANZONE
%------------------------------------------------------------



%-------------------------------------------------------------
%			INIZIO	CANZONE
%-------------------------------------------------------------


%titolo: 	Come tu mi vuoi
%autore: 	Daniele Ricci
%tonalita: 	Sol 



%%%%%% TITOLO E IMPOSTAZONI
\beginsong{Come tu mi vuoi}[by={D. Branca}] 	% <<< MODIFICA TITOLO E AUTORE
\transpose{0} 						% <<< TRASPOSIZIONE #TONI (0 nullo)
%\preferflats  %SE VOGLIO FORZARE i bemolle come alterazioni
%\prefersharps %SE VOGLIO FORZARE i # come alterazioni
\momenti{}							% <<< INSERISCI MOMENTI	
% momenti vanno separati da ; e vanno scelti tra:
% Ingresso; Atto penitenziale; Acclamazione al Vangelo; Dopo il Vangelo; Offertorio; Comunione; Ringraziamento; Fine; Santi; Pasqua; Avvento; Natale; Quaresima; Canti Mariani; Battesimo; Prima Comunione; Cresima; Matrimonio; Meditazione; Spezzare del pane;
\ifchorded
	%\textnote{$\bigstar$ Tonalità migliore }	% <<< EV COMMENTI (tonalità originale\migliore)
\fi


%%%%%% INTRODUZIONE
\ifchorded
\vspace*{\versesep}
\musicnote{
\begin{minipage}{0.48\textwidth}
\textbf{Intro}
\hfill 
%( \eighthnote \, 80)   % <<  MODIFICA IL TEMPO
% Metronomo: \eighthnote (ottavo) \quarternote (quarto) \halfnote (due quarti)
\end{minipage}
} 	
\vspace*{-\versesep}
\beginverse*

\nolyrics

%---- Prima riga -----------------------------
\vspace*{-\versesep}
 \[G] \[C] \[D7] \[G]	 % \[*D] per indicare le pennate, \rep{2} le ripetizioni

%---- Ogni riga successiva -------------------
\vspace*{-\versesep}
\[E-] \[A-] \[D4] \[G] 	

%---- Ev Indicazioni -------------------------			
%\textnote{\textit{[oppure tutta la strofa]} }	

\endverse
\fi




%%%%% STROFA
\beginverse		%Oppure \beginverse* se non si vuole il numero di fianco
\memorize 		% <<< DECOMMENTA se si vuole utilizzarne la funzione
%\chordsoff		% <<< DECOMMENTA se vuoi una strofa senza accordi

\[G]Eccomi Signor, vengo a te mio \[A-7]Re,
\[E-]che si compia in me la tua \[G]volon\[D]tà.
\[G]Eccomi Signor, vengo a te mio \[A-7]Dio,
\[E-]plasma il cuore mio \[B-7]e di te vivrò.
\[G]Se tu lo \[C]vuoi Signo\[D]re manda \[E-]me
\[A-]e il tuo nome \[B-]annun\[G]ce\[C]rò.

\endverse




%%%%% RITORNELLO
\beginchorus
\textnote{\textbf{Rit.}}

Come tu mi \[G]vuoi io sa\[D]rò,
dove tu mi \[E-]vuoi io an\[B-7]drò.
Questa \[C]vita io voglio dona\[B-7]rla a \[E-]te
per dar \[F]gloria al tuo nome \[C]mio \[D4]Re.
\[D]Come tu mi \[C]vuoi io sa\[D]rò,
\[B7]dove tu mi \[E-]vuoi io an\[B-7]drò.
Se mi gu\[C]ida il tuo amore pa\[B7]ura non \[E-]ho,   
per se\[A-]mpre io sa\[D]rò  come tu mi vuoi.

\endchorus



%%%%%% EV. INTERMEZZO
\beginverse*
\vspace*{1.3\versesep}
{
	\nolyrics
	\textnote{Intermezzo strumentale}
	
	\ifchorded

	%---- Prima riga -----------------------------
	\vspace*{-\versesep}
	\[G] \[C] \[D7] \[G]




	\fi
	%---- Ev Indicazioni -------------------------			
	%\textnote{\textit{(ripetizione della strofa)}} 
	 
}
\vspace*{\versesep}
\endverse




%%%%% STROFA
\beginverse		%Oppure \beginverse* se non si vuole il numero di fianco
%\memorize 		% <<< DECOMMENTA se si vuole utilizzarne la funzione
%\chordsoff		% <<< DECOMMENTA se vuoi una strofa senza accordi

\[G]Eccomi Signor, vengo a te mio \[A-7]Re,
\[E-]che si compia in me la tua \[G]volon\[D]tà.
\[G]Eccomi Signor, vengo a te mio \[A-7]Dio,
\[E-]plasma il cuore mio \[B-7]e di te vivrò..
\[G]Tra le tue \[C]mani mai \[D]più vacil\[E-]lerò
\[A-]e strumento \[B-]tuo \[G]sa\[C]rò.

\endverse







%%%%% RITORNELLO
\beginchorus
\textnote{\textbf{Rit.}}

Come tu mi \[G]vuoi io sa\[D]rò,
dove tu mi \[E-]vuoi io an\[B-7]drò.
Questa \[C]vita io voglio dona\[B-7]rla a \[E-]te
per dar \[F]gloria al tuo nome \[C]mio \[D4]Re.
\[D]Come tu mi \[C]vuoi io sa\[D]rò,
\[B7]dove tu mi \[E-]vuoi io an\[B-7]drò.
Se mi gu\[C]ida il tuo amore pa\[B7]ura non \[E-]ho,   
per se\[A-]mpre io sa\[D]rò  come tu mi vuoi.

\endchorus




%%%%%% EV. FINALE

\beginchorus %oppure \beginverse*
\vspace*{1.3\versesep}
\textnote{\textbf{Finale }} %<<< EV. INDICAZIONI
come tu mi vuoi\[C] \[D].\echo{Io sarò}  \rep{4} 

\endchorus  %oppure \endverse






%%%%%% EV. CHIUSURA SOLO STRUMENTALE
\ifchorded
\beginchorus %oppure \beginverse*
\vspace*{1.3\versesep}
\textnote{Chiusura } %<<< EV. INDICAZIONI

\[G*]

\endchorus  %oppure \endverse
\fi


\endsong
%------------------------------------------------------------
%			FINE CANZONE
%------------------------------------------------------------




%-------------------------------------------------------------
%			INIZIO	CANZONE
%-------------------------------------------------------------


%titolo: 	Come un fiume
%autore: 	Paci, Preti
%tonalita: 	Do 



%%%%%% TITOLO E IMPOSTAZONI
\beginsong{Come un fiume}[by={Paci, Preti}] 	% <<< MODIFICA TITOLO E AUTORE
\transpose{0} 						% <<< TRASPOSIZIONE #TONI (0 nullo)
\momenti{Comunione; Avvento}							% <<< INSERISCI MOMENTI	
% momenti vanno separati da ; e vanno scelti tra:
% Ingresso; Atto penitenziale; Acclamazione al Vangelo; Dopo il Vangelo; Offertorio; Comunione; Ringraziamento; Fine; Santi; Pasqua; Avvento; Natale; Quaresima; Canti Mariani; Battesimo; Prima Comunione; Cresima; Matrimonio; Meditazione;
\ifchorded
	%\textnote{Tonalità originale }	% <<< EV COMMENTI (tonalità originale/migliore)
\fi

%%%%%% INTRODUZIONE
\ifchorded
\vspace*{\versesep}
\textnote{Intro: \qquad \qquad  }%(\eighthnote 116) % << MODIFICA IL TEMPO
% Metronomo: \eighthnote (ottavo) \quarternote (quarto) \halfnote (due quarti)
\vspace*{-\versesep}
\beginverse*

\nolyrics

%---- Prima riga -----------------------------
\vspace*{-\versesep}
\[C] \[G]  \[C]	 % \[*D] per indicare le pennate, \rep{2} le ripetizioni

%---- Ogni riga successiva -------------------
%\vspace*{-\versesep}
%\[G] \[C]  \[D]	

%---- Ev Indicazioni -------------------------			
%\textnote{\textit{(Oppure tutta la strofa)} }	

\endverse
\fi

%%%%% RITORNELLO
\beginchorus
\textnote{\textbf{Rit.}}

Come un \[C]fiume in piena che \brk la sabbia \[G]non può arrestare
come l'\[C7]onda che dal mare si di\[F]stende sulla riva
ti pre\[F-]ghiamo Padre \brk che così si \[C]sciolga il nostro amore
e l'a\[D]more dove ar\[D7]riva sciolga il \[G]dubbio e la \[G7]paura. 

\endchorus

%%%%% STROFA
\beginverse		%Oppure \beginverse* se non si vuole il numero di fianco
\memorize 		% <<< DECOMMENTA se si vuole utilizzarne la funzione
%\chordsoff		& <<< DECOMMENTA se vuoi una strofa senza accordi

Come un \[C]pesce che risale a nuoto \[G]fino alla sorgente
va a sco\[C7]prire dove nasce e si di\ch{F}{f}{f}{ff}onde la sua vita
ti pre\[F-]ghiamo Padre che noi risa\[C]liamo la corrente
fino ad \[G]arrivare alla vita \[F]nell'a\[C]more.  \[(G7)]

\endverse

%%%%% RITORNELLO
\beginchorus
\textnote{\textbf{Rit.}}

Come un \[C]fiume in piena che \brk la sabbia \[G]non può arrestare
come l'\[C7]onda che dal mare si di\[F]stende sulla riva
ti pre\[F-]ghiamo Padre \brk che così si \[C]sciolga il nostro amore
e l'a\[D]more dove ar\[D7]riva sciolga il \[G]dubbio e la \[G7]paura. 

\endchorus

%%%%% STROFA
\beginverse		%Oppure \beginverse* se non si vuole il numero di fianco
%\memorize 		% <<< DECOMMENTA se si vuole utilizzarne la funzione
%\chordsoff		& <<< DECOMMENTA se vuoi una strofa senza accordi

Come l'^erba che germoglia cresce ^senza far rumore
ama il ^giorno della pioggia si addor^menta sotto il sole
ti pre^ghiamo Padre che così in un ^giorno di silenzio
anche in ^noi germogli questa vita ^nell'a^more.  \[A7]

\endverse

\transpose{2}

%%%%% RITORNELLO
\beginchorus
\textnote{\textbf{Rit.}}

Come un \[C]fiume in piena che \brk la sabbia \[G]non può arrestare
come l'\[C7]onda che dal mare si di\[F]stende sulla riva
ti pre\[F-]ghiamo Padre \brk che così si \[C]sciolga il nostro amore
e l'a\[D]more dove ar\[D7]riva sciolga il \[G]dubbio e la \[G7]paura. 

\endchorus

%%%%% STROFA
\beginverse		%Oppure \beginverse* se non si vuole il numero di fianco
%\memorize 		% <<< DECOMMENTA se si vuole utilizzarne la funzione
%\chordsoff		& <<< DECOMMENTA se vuoi una strofa senza accordi

Come un ^albero che affonda le ra^dici nella terra
e su ^quella terra un uomo costru^isce la sua casa
ti pre^ghiamo Padre buono di por^tarci alla tua casa
dove ^vivere una vita piena ^nell'a^more. 
\endverse

\endsong
%------------------------------------------------------------
%			FINE CANZONE
%------------------------------------------------------------

%titolo{Come un prodigio}
%autore{Vezzani}
%album{}
%tonalita{Do}
%famiglia{Liturgica}
%gruppo{}
%momenti{}
%identificatore{come_un_prodigio}
%data_revisione{2017_04_26}
%trascrittore{Francesco Endrici}
\beginsong{Come un prodigio}[by={Vezzani}]
%\transpose{5}
\ifchorded
\beginverse*
\vspace*{-0.8\versesep}
{\nolyrics Intro: \[G]\[D/F#] \rep{3}}
\vspace*{-\versesep}
\endverse
\fi
\beginverse
\memorize
Signore \[E-7]tu mi scruti e co\[C9]nosci, 
sai quando \[G]seggo e quando \[D]mi alzo 
riesci a \[E-7]vedere i miei \[C9]pensieri,
sai quando \[G]io cammino \[D]e quando riposo.
\[F]Ti sono note tutte le mie \[C]vie,
la mia pa\[G]rola non è ancora sulla lingua
\[F]e tu Signore già la co\[C]nosci tut\[D]ta.
\endverse
\beginchorus
\[E-7]Sei tu che mi hai \[C9]creato
e mi hai tessuto nel \[G]seno di mia \[D]madre.
\[E-7]Tu mi hai fatto come un pro\[C9]digio,
le tue opere \[G]sono stupende
e per \[D]questo io ti lodo.
\endchorus
\ifchorded
\beginverse*
\vspace*{-\versesep}
{\nolyrics \[E-7]\[C9]\[G]\[D]}
\endverse
\fi
\beginverse
Di fronte e alle ^spalle tu mi circ^ondi 
poni su ^me la tua ^mano.
La tua sag^gezza stupenda per ^me
è troppo ^alta e io non la ^comprendo.
^Che sia al cielo o agli inferi ci ^sei,
non si può ^mai fuggire dalla tua presenza, 
^ovunque la tua mano guide^rà la ^mia.
\endverse
\beginchorus
\[E-7]Sei tu che mi hai \[C9]creato
e mi hai tessuto nel \[G]seno di mia \[D]madre.
\[E-7]Tu mi hai fatto come un pro\[C9]digio,
le tue opere \[G]sono stupende
e per \[D]questo io ti lodo.
\endchorus
\ifchorded
\beginverse*
\vspace*{-\versesep}
{\nolyrics \[E-7]\[C9]\[G]\[D]}
\endverse
\fi
\beginverse
E nel se^greto tu mi hai for^mato
mi hai intes^suto dalla ^terra.
Neanche le ^ossa ti eran na^scoste
ancora in^forme mi hanno ^visto i tuoi occhi.
^I miei giorni erano fis^sati
quando an^cora non ne esisteva uno
e ^tutto quanto era scritto ^nel tuo ^libro.
\endverse
\beginchorus
\[E-7]Sei tu che mi hai \[C9]creato
e mi hai tessuto nel \[G]seno di mia \[D]madre.
\[E-7]Tu mi hai fatto come un pro\[C9]digio,
le tue opere \[G]sono stupende
e per \[D]questo io ti lodo.
\endchorus
\ifchorded
\beginverse*
\vspace*{-\versesep}
{\nolyrics \[E-7]\[C9]\[G]\[D]}
\endverse
\fi
\beginchorus
\[E-7]Sei tu che mi hai \[C9]creato
e mi hai tessuto nel \[G]seno di mia \[D]madre.
\[E-7]Tu mi hai fatto come un pro\[C9]digio,
le tue opere \[G]sono stupende
e per \[D]questo, per questo ti \[C9]lodo
\endchorus
\endsong


%-------------------------------------------------------------
%			INIZIO	CANZONE
%-------------------------------------------------------------


%titolo: 	Conte faremo cose grandi
%autore: 	Meregalli
%tonalita: 	Fa



%%%%%% TITOLO E IMPOSTAZONI
\beginsong{Con Te faremo cose grandi}[by={Meregalli}]	% <<< MODIFICA TITOLO E AUTORE
\transpose{0} 						% <<< TRASPOSIZIONE #TONI (0 nullo)
\momenti{Ingresso; Avvento	}							% <<< INSERISCI MOMENTI	
% momenti vanno separati da ; e vanno scelti tra:
% Ingresso; Atto penitenziale; Acclamazione al Vangelo; Dopo il Vangelo; Offertorio; Comunione; Ringraziamento; Fine; Santi; Pasqua; Avvento; Natale; Quaresima; Canti Mariani; Battesimo; Prima Comunione; Cresima; Matrimonio; Meditazione; Spezzare del pane;
\ifchorded
	%\textnote{Tonalità migliore }	% <<< EV COMMENTI (tonalità originale/migliore)
\fi


%%%%%% INTRODUZIONE
\ifchorded
\vspace*{\versesep}
\textnote{Intro: \qquad \qquad  }%(\eighthnote 116) % <<  MODIFICA IL TEMPO
% Metronomo: \eighthnote (ottavo) \quarternote (quarto) \halfnote (due quarti)
\vspace*{-\versesep}
\beginverse*

\nolyrics

%---- Prima riga -----------------------------
\vspace*{-\versesep}
\[F] \[C] \[F]		 % \[*D] per indicare le pennate, \rep{2} le ripetizioni

%---- Ogni riga successiva -------------------
%\vspace*{-\versesep}
%\[G] \[C]  \[D]	

%---- Ev Indicazioni -------------------------			
%\textnote{\textit{(Oppure tutta la strofa)} }	

\endverse
\fi



%%%%% RITORNELLO
\beginchorus
\textnote{\textbf{Rit.}}
Con \[F]Te fa\[C]remo cose \[D-]gran\[G]di
il cam\[B&]mino che per\[C]correremo in\[A-]sie\[C]me
\[C7]di \[F]Te si \[C]riempiranno \[D-]sguar\[G]di
la spe\[B&]ranza che ri\[C]splenderà nei \[A-]vol\[C]ti.
\endchorus


\beginchorus
Tu la \[B&]luce che ri\[C]schiara,
Tu la \[B&]voce che ci \[C]chiama,
Tu la \[B&]gioia che dà \[C]vita ai nostri \[A-]so\[C]gni.
\endchorus


\beginverse
\memorize
\[F]Parlaci Si\[C]gnore come \[B&]sai,
sei pre\[C]sente nel mi\[C7]stero in mezzo a \[F]noi. \[C]
\[F]Chiamaci col \[C]nome che vor\[B&]rai
e sia \[C]fatto il tuo di\[C7]segno su di \[F]noi. \[C]
\endverse

\beginchorus
Tu la \[B&]luce che ri\[C]schiara,
Tu la \[B&]voce che ci \[C]chiama,
Tu la \[B&]gioia che dà \[C]vita ai nostri \[A-]so\[C]gni.
\endchorus

%%%%% RITORNELLO
\beginchorus
\textnote{\textbf{Rit.}}
Con \[F]Te fa\[C]remo cose \[D-]gran\[G]di
il cam\[B&]mino che per\[C]correremo in\[A-]sie\[C]me
\[C7]di \[F]Te si \[C]riempiranno \[D-]sguar\[G]di
la spe\[B&]ranza che ri\[C]splenderà nei \[A-]vol\[C]ti.
\endchorus


\beginchorus
Tu l'a\[B&]more che dà \[C]vita,
Tu il sor\[B&]riso che ci al\[C]lieta,
Tu la \[B&]forza che ra\[C]duna i nostri \[A-]gior\[C]ni.
\endchorus

\beginverse
^Guidaci Si^gnore dove ^sai
da chi ^soffre chi è più ^piccolo di ^noi ^
stru^menti di quel ^regno che Tu ^fai,
di quel ^regno che ora ^vive in mezzo a ^noi. ^
\endverse


\beginchorus
Tu l'a\[B&]more che dà \[C]vita,
Tu il sor\[B&]riso che ci al\[C]lieta,
Tu la \[B&]forza che ra\[C]duna i nostri \[A-]gior\[C]ni.
\endchorus

%%%%% RITORNELLO
\beginchorus
\textnote{\textbf{Rit.}}
Con \[F]Te fa\[C]remo cose \[D-]gran\[G]di
il cam\[B&]mino che per\[C]correremo in\[A-]sie\[C]me
\[C7]di \[F]Te si \[C]riempiranno \[D-]sguar\[G]di
la spe\[B&]ranza che ri\[C]splenderà nei \[A-]vol\[C]ti.
\endchorus

\beginchorus %oppure \beginverse*
\vspace*{1.3\versesep}
\textnote{Finale } %<<< EV. INDICAZIONI
Tu l'am\[B&]ore che dà \[C]vita,
Tu il sor\[B&]riso che ci al\[C]lieta,
Tu la \[B&]forza che ra\[C]duna i nostri \[B&]gio-o-\[B&]o-or\[F]ni.
\endchorus



\endsong
%------------------------------------------------------------
%			FINE CANZONE
%------------------------------------------------------------





%-------------------------------------------------------------
%			INIZIO	CANZONE
%-------------------------------------------------------------


%titolo: 	Coraggio sono io
%autore: 	E. Bertoglio, A. Testa, C. Pastori
%tonalita: 	Re 



%%%%%% TITOLO E IMPOSTAZONI
\beginsong{Coraggio sono io}[by={E. Bertoglio, A. Testa, C. Pastori}] 	
\transpose{0} 						% <<< TRASPOSIZIONE #TONI (0 nullo)
\momenti{Comunione; Ringraziamento; Congedo}	% <<< INSERISCI MOMENTI	
% momenti vanno separati da ; e vanno scelti tra:
% Ingresso; Atto penitenziale; Acclamazione al Vangelo; Dopo il Vangelo; Offertorio; Comunione; Ringraziamento; Fine; Santi; Pasqua; Avvento; Natale; Quaresima; Canti Mariani; Battesimo; Prima Comunione; Cresima; Matrimonio; Meditazione; Spezzare del pane;
\ifchorded
	%\textnote{Tonalità originale }	% <<< EV COMMENTI (tonalità originale/migliore)
\fi


%%%%%% INTRODUZIONE
\ifchorded
\vspace*{\versesep}
\musicnote{
\begin{minipage}{0.48\textwidth}
Intro
\hfill 
(\quarternote \, 80)
\end{minipage}
} 
% Metronomo: \eighthnote (ottavo)  (quarto) \halfnote (due quarti)
\vspace*{-\versesep}
\beginverse*

\nolyrics

%---- Prima riga -----------------------------
\vspace*{-\versesep}
 \[D]   \[D] 	 % \[*D] per indicare le pennate, \rep{2} le ripetizioni

%---- Ogni riga successiva -------------------
%\vspace*{-\versesep}
%\[G] \[C]  \[D]	

%---- Ev Indicazioni -------------------------			
%\textnote{\textit{(Oppure tutta la strofa)} }	

\endverse
\fi




%%%%% STROFA
\beginverse		%Oppure \beginverse* se non si vuole il numero di fianco
\memorize 		% <<< DECOMMENTA se si vuole utilizzarne la funzione
%\chordsoff		% <<< DECOMMENTA se vuoi una strofa senza accordi

\[D]Strade vuote e silen\[D7+]ziose,
vie de\[E-]serte e sconosci\[B-]ute
è una v\[G]ita che ora scorre senza di \[A]te.
Mi ri\[D]trovo dentro a un \[D7+]mare
di incer\[E-]tezze e turba\[B-]menti,
la fa\[G]tica di un cammino senza di \[A]te.

\[D]Ma \[A]tu,
mano a\[G]mica di ogni u\[D]omo,
pre\[E-]senza che sosti\[E7]ene an\[A]cora.
\[D]Ma \[A]tu,
che ora g\[G]uidi il mio cam\[D]mino,
il\[G]lumina la via da\[E7]vanti a \[A]me.

\endverse







%%%%% RITORNELLO
\beginchorus
\textnote{\textbf{Rit.}}

\[D]No, non a\[G]vere pa\[A7]ura,
se nel \[D]buio il tuo c\[G]uore
un \[B-]giorno perde\[C]ra\[A]i.
\[D]Io v\[G]errò da \[A7]te,
come un \[D]padre \[G]ti di\[A*]rò:
"coraggio, \[G]sono \[D]io!".

\endchorus



%%%%% STROFA
\beginverse		%Oppure \beginverse* se non si vuole il numero di fianco
%\memorize 		% <<< DECOMMENTA se si vuole utilizzarne la funzione
%\chordsoff		% <<< DECOMMENTA se vuoi una strofa senza accordi

Il res^piro di una ^vita
è ^silenzio di de^serto
fatto ^di parole vuote senza di ^te.
Dove ^sono le risp^oste
alle m^ille e più do^mande,
il mio ^cuore non sa amare senza di ^te.

^Ma ^tu,
luce ^nella notte ^buia,
mi ai^uterai a rag^giunger la ^meta.
^Ma ^tu,
fonte ^viva della ^fede,
sa^rai per sempre qui vi^cino a ^me.

\endverse




%%%%% RITORNELLO
\beginchorus
\textnote{\textbf{Rit.}}

\[D]No, non a\[G]vere pa\[A7]ura,
se nel \[D]buio il tuo c\[G]uore
un \[B-]giorno perde\[C]ra\[A]i.
\[D]Io v\[G]errò da \[A7]te,
come un \[D]padre \[G]ti di\[A*]rò:
"coraggio, \[G]sono \[D]io!".

\endchorus


%%%%%% EV. FINALE

\beginchorus %oppure \beginverse*
\vspace*{1.3\versesep}
\textnote{\textbf{Finale} \textit{(rallentando)}} %<<< EV. INDICAZIONI

\[(D)]"Coraggio, \[G]sono \[D]io!".
\[D*]"Coraggio, \[G*]sono \[D]io!".

\endchorus  %oppure \endverse


\endsong
%------------------------------------------------------------
%			FINE CANZONE
%------------------------------------------------------------



%-------------------------------------------------------------
%			INIZIO	CANZONE
%-------------------------------------------------------------


%titolo: 	Credo
%autore: 	Gen Rosso, Gen Verde
%tonalita: 	Re 



%%%%%% TITOLO E IMPOSTAZONI
\beginsong{Credo}[by={Gen\ Rosso, Gen\ Verde}]	% <<< MODIFICA TITOLO E AUTORE
\transpose{0} 						% <<< TRASPOSIZIONE #TONI (0 nullo)
\momenti{}							% <<< INSERISCI MOMENTI	
% momenti vanno separati da ; e vanno scelti tra:
% Ingresso; Atto penitenziale; Acclamazione al Vangelo; Dopo il Vangelo; Offertorio; Comunione; Ringraziamento; Fine; Santi; Pasqua; Avvento; Natale; Quaresima; Canti Mariani; Battesimo; Prima Comunione; Cresima; Matrimonio; Meditazione; Spezzare del pane;
\ifchorded
	%\textnote{Tonalità migliore }	% <<< EV COMMENTI (tonalità originale/migliore)
\fi


%%%%%% INTRODUZIONE
\ifchorded
\vspace*{\versesep}
\textnote{Intro: \qquad \qquad  }%(\eighthnote 116) % <<  MODIFICA IL TEMPO
% Metronomo: \eighthnote (ottavo) \quarternote (quarto) \halfnote (due quarti)
\vspace*{-\versesep}
\beginverse*

\nolyrics

%---- Prima riga -----------------------------
\vspace*{-\versesep}
\[E-] \[E-] \[B-] \[A]	 % \[*D] per indicare le pennate, \rep{2} le ripetizioni

%---- Ogni riga successiva -------------------
%\vspace*{-\versesep}
%\[G] \[C]  \[D]	

%---- Ev Indicazioni -------------------------			
%\textnote{\textit{(Oppure tutta la strofa)} }	

\endverse
\fi


\beginchorus
\[D]Credo, \[G7]credo, \[A*]  \[G]\[E-]cre\[D]do.
\[D]Credo, \[G7]credo, \[A*]  \[G]\[E-]cre\[D]do.
\endchorus

\beginverse*
\[D]Io credo in Dio \[G]Padre onnipotente,
crea\[D]tore del cielo e della ter\[G]ra. \[A4] 
\endverse

\beginchorus
\[D]Credo, \[G7]credo, \[A*]{}  \[G]\[E-]cre\[D]do.
\[D]Credo, \[G7]credo, \[A*]{}  \[G]\[E-]cre\[D]do.
\endchorus

\beginverse*
\[D]Credo in Gesù Cristo,
suo \[G]unico Figlio, nostro Signore,
conce\[D]pito di Spirito Santo
e \[G]nato da Maria Vergine.
Pa\[D]tì sotto Ponzio Pilato,
fu croci\[E-]fisso, morì e fu sepolto;
discese a\[B-]gli inferi. Cre\[G]do. \[D/F#] 
Il \[E-7]terzo giorno risuscitò da morte;
sa\[D]lì al cielo; siede alla destra di \[G]Dio 
\[C]Pa\[G]dre onnipo\[C]ten\[G]te
e di \[B-]là verrà a giudicare,
a \[A4]giudicare i vivi e i \[G]morti.
\endverse

\beginchorus
\[D]Credo, \[G7]credo, \[A*] \[G]\[E-7]cre\[D]do.
\[D]Credo, \[G7]credo, \[A*] \[G]\[E-7]cre\[D]do.
\endchorus

\beginverse*
\[D]Io credo nello \[G]Spirito Santo, \[D]credo,
la \[G]santa Chiesa Cattolica, \[D]credo,
la comunione dei \[G]santi, 
la remissione dei peccati, \[D]credo,
la risurre\[G]zione della carne,
la vita eterna.
\endverse

\beginchorus
\[E]Amen. \[A7]Amen. \[B*] \[A]\[F#-7]A\[C#-7]men. \[B] 
\[E]Amen. \[A7]Amen. \[B*] \[A]\[F#-]A\[E]men.
\endchorus


\endsong
%------------------------------------------------------------
%			FINE CANZONE
%------------------------------------------------------------



%-------------------------------------------------------------
%			INIZIO	CANZONE
%-------------------------------------------------------------


%titolo: 	Credo in unum Deum
%autore: 	M. Balduzzi, C. Casucci
%tonalita: 	Sol 



%%%%%% TITOLO E IMPOSTAZONI
\beginsong{Credo in unum Deum}[by={M. Balduzzi, C. Casucci}] 	% <<< MODIFICA TITOLO E AUTORE
\transpose{0} 						% <<< TRASPOSIZIONE #TONI (0 nullo)
\preferflats  %SE VOGLIO FORZARE i bemolle come alterazioni
%\prefersharps %SE VOGLIO FORZARE i # come alterazioni
\momenti{Meditazione}							% <<< INSERISCI MOMENTI	
% momenti vanno separati da ; e vanno scelti tra:
% Ingresso; Atto penitenziale; Acclamazione al Vangelo; Dopo il Vangelo; Offertorio; Comunione; Ringraziamento; Fine; Santi; Pasqua; Avvento; Natale; Quaresima; Canti Mariani; Battesimo; Prima Comunione; Cresima; Matrimonio; Meditazione; Spezzare del pane;
\ifchorded
	%\textnote{Tonalità migliore }	% <<< EV COMMENTI (tonalità originale/migliore)
\fi



%%%%%% INTRODUZIONE
\ifchorded
\vspace*{\versesep}
\musicnote{
\begin{minipage}{0.48\textwidth}
\textbf{Intro}
\hfill 
%( \eighthnote \, 80)   % <<  MODIFICA IL TEMPO
% Metronomo: \eighthnote (ottavo) \quarternote (quarto) \halfnote (due quarti)
\end{minipage}
} 	
\vspace*{-\versesep}
\beginverse*

\nolyrics

%---- Prima riga -----------------------------
\vspace*{-\versesep}
\[E-] \quad \[C] 	 % \[*D] per indicare le pennate, \rep{2} le ripetizioni

%---- Ogni riga successiva -------------------
\vspace*{-\versesep}
\[G] \qquad \[D]	

%---- Ev Indicazioni -------------------------			
%\textnote{\textit{(Oppure tutta la strofa)} }	

\endverse
\fi


 

%%%%% STROFA
\beginverse		%Oppure \beginverse* se non si vuole il numero di fianco
\memorize 		% <<< DECOMMENTA se si vuole utilizzarne la funzione
%\chordsoff		% <<< DECOMMENTA se vuoi una strofa senza accordi

\[E-]Credo in unum \[C]Deum,
\[G]Patrem omnipo\[D]tentem.
\[E-]Credo in unum \[C]Deum,
\[G]factorem coeli et \[D]terrae.
\[E-]Visibilium om\[C]nium,
\[G]et invisi\[D]bilium.
\[A-]Credo in unum \[C]Deum, \[D]A-a-a-\[E-]men!          

\endverse



\transpose{2}
%\preferflats 
%\prefersharps 

%%%%% STROFA
\beginverse		%Oppure \beginverse* se non si vuole il numero di fianco
%\memorize 		% <<< DECOMMENTA se si vuole utilizzarne la funzione
%\chordsoff		% <<< DECOMMENTA se vuoi una strofa senza accordi

^Credo in unum ^Deum,
^Dominum Jesum ^Christum.
^Credo in unum ^Deum,
^Filium Dei uni^genitum.
^Et ex Patre na^tum,
^ante omnia sae^cula.
^Credo in unum ^Deum, ^A-a-a-^men!      

\endverse





\transpose{2}
%\preferflats 
\prefersharps 
%%%%% STROFA
\beginverse		%Oppure \beginverse* se non si vuole il numero di fianco
%\memorize 		% <<< DECOMMENTA se si vuole utilizzarne la funzione
%\chordsoff		% <<< DECOMMENTA se vuoi una strofa senza accordi

^Credo in unum ^Deum,
^Spiritum sanc^tum.
^Credo in unum ^Deum,
^Dominum et vivifi^cantem.
^Qui       ex         Pa^tre,
^filioque proce^dit.
^Credo in unum ^Deum, ^A-a-a-^men! 

\endverse

\transpose{-3}
%\preferflats 
\prefersharps 
%%%%% STROFA
\beginverse		%Oppure \beginverse* se non si vuole il numero di fianco
%\memorize 		% <<< DECOMMENTA se si vuole utilizzarne la funzione
%\chordsoff		% <<< DECOMMENTA se vuoi una strofa senza accordi
^Credo in unum ^Deum,
^Patrem omnipo^tentem.
^Credo in unum ^Deum,
^Dominum Jesum ^Christum.
^Credo in unum ^Deum,
^Spiritum sanc^tum.
^Credo in unum ^Deum, ^A-a-a-^men!

\endverse

%%%%%% EV. FINALE

\beginchorus %oppure \beginverse*
\vspace*{1.3\versesep}
\textnote{\textbf{Finale} } %<<< EV. INDICAZIONI
\transpose{-1}
%\preferflats 
\prefersharps 
A-a-a\[C-]men!  A-a-a\[C#]men! 
A-a-a-\[G#]men!

\endchorus  %oppure \endverse



\endsong
%------------------------------------------------------------
%			FINE CANZONE
%------------------------------------------------------------



%-------------------------------------------------------------
%			INIZIO	CANZONE
%-------------------------------------------------------------


%titolo: 	Custodiscimi
%autore: 	C. Neuf
%tonalita: 	Mi-



%%%%%% TITOLO E IMPOSTAZONI
\beginsong{Custodiscimi}[by={Chemin\ Neuf}]
\transpose{0} 						% <<< TRASPOSIZIONE #TONI (0 nullo)
%\preferflats  %SE VOGLIO FORZARE i bemolle come alterazioni
%\prefersharps %SE VOGLIO FORZARE i # come alterazioni
\momenti{Ringraziamento}							% <<< INSERISCI MOMENTI	
% momenti vanno separati da ; e vanno scelti tra:
% Ingresso; Atto penitenziale; Acclamazione al Vangelo; Dopo il Vangelo; Offertorio; Comunione; Ringraziamento; Fine; Santi; Pasqua; Avvento; Natale; Quaresima; Canti Mariani; Battesimo; Prima Comunione; Cresima; Matrimonio; Meditazione; Spezzare del pane;
\ifchorded
	%\textnote{Tonalità migliore }	% <<< EV COMMENTI (tonalità originale/migliore)
\fi







%%%%%% INTRODUZIONE
\ifchorded
\vspace*{\versesep}
\musicnote{
\begin{minipage}{0.48\textwidth}
\textbf{Intro}
\hfill 
%( \eighthnote \, 80)   % <<  MODIFICA IL TEMPO
% Metronomo: \eighthnote (ottavo) \quarternote (quarto) \halfnote (due quarti)
\end{minipage}
} 	
\vspace*{-\versesep}
\beginverse*
\nolyrics

%---- Prima riga -----------------------------
\vspace*{-\versesep}
\[E-] \[E-] \[A-*] \[B7*] \[E-]	 % \[*D] per indicare le pennate, \rep{2} le ripetizioni

%---- Ogni riga successiva -------------------
%\vspace*{-\versesep}
%\[G] \[C]  \[D]	

%---- Ev Indicazioni -------------------------			
%\textnote{\textit{(Oppure tutta la strofa)} }	

\endverse
\fi

\beginverse
\[E-]Ho detto a Dio, \[A-*]senza \[B7*]di \[E-]te,
alcun \[D]bene non \[G]ho, custo\[A-]disci\[B7]mi.
\[E-]Magnifica è la \[A-*]mia e\[B7*]redi\[E-]tà,
bene\[D]detto sei \[G]Tu, sempre \[A-*]sei \[B-7*]con \[E-]me.
\endverse

\beginchorus
\textnote{\textbf{Rit.}}
Custo\[E-*]di\[A-*]sci\[D]mi, mia \[G]forza sei \[A-]Tu,
Custo\[E-*]di\[A-*]sci\[D]mi, mia \[C]gioia Ge\[E-]sù. \rep{2}
\endchorus

\beginverse
\chordsoff
Ti pongo sempre innanzi a me
al sicuro sarò, mai vacillerò!
Via, verità e vita sei;
Mio Dio credo che Tu mi guiderai.
\endverse



\endsong






%DDD
%-------------------------------------------------------------
%			INIZIO	CANZONE
%-------------------------------------------------------------


%titolo: 	Dall'aurora al tramonto
%autore: 	Casucci, Balduzzi
%tonalita: 	Si- 



%%%%%% TITOLO E IMPOSTAZONI
\beginsong{Dall'aurora al tramonto}[by={Casucci, Balduzzi}] 	% <<< MODIFICA TITOLO E AUTORE
\transpose{-2} 						% <<< TRASPOSIZIONE #TONI (0 nullo)
\momenti{Comunione; Meditazione; Ringraziamento}							% <<< INSERISCI MOMENTI	
% momenti vanno separati da ; e vanno scelti tra:
% Ingresso; Atto penitenziale; Acclamazione al Vangelo; Dopo il Vangelo; Offertorio; Comunione; Ringraziamento; Fine; Santi; Pasqua; Avvento; Natale; Quaresima; Canti Mariani; Battesimo; Prima Comunione; Cresima; Matrimonio; Meditazione;
\ifchorded
	%\textnote{Tonalità originale }	% <<< EV COMMENTI (tonalità originale/migliore)
\fi





%%%%%% INTRODUZIONE
\ifchorded
\vspace*{\versesep}
\textnote{Intro: \qquad \qquad  }%(\eighthnote 116) % << MODIFICA IL TEMPO
% Metronomo: \eighthnote (ottavo) \quarternote (quarto) \halfnote (due quarti)
\vspace*{-\versesep}
\beginverse*

\nolyrics

%---- Prima riga -----------------------------
\vspace*{-\versesep}
\[C#-]\[E]\[A]\[B] \rep{2}	 % \[*D] per indicare le pennate, \rep{2} le ripetizioni

%---- Ogni riga successiva -------------------
%\vspace*{-\versesep}
%\[G] \[C]  \[D]	

%---- Ev Indicazioni -------------------------			
\textnote{\textit{(Oppure tutto il ritornello)} }	

\endverse
\fi




\ifchorded
\beginverse*
\vspace*{-0.8\versesep}
{ }
\vspace*{-\versesep}
\endverse
\fi


\beginchorus
\memorize
\[C#-]Dall'au\[E]rora io \[F#-]cerco \[B]te,
\[C#-]fino al tra\[E]monto ti \[F#-]chia\[B]mo.
\[C#-]Ha sete \[G#-]solo di \[A]te l'\[B]anima \[C#-]mia
come \[G#-]terra de\[A]ser\[B]ta. \rep{2}
\endchorus




\beginverse
^Non mi ferme^rò un ^solo i^stante
^sempre cante^rò ^la tua ^lode.
^Perché sei il mio ^Dio, ^il mio ri^paro
^mi protegge^rai ^all'ombra delle tue \[B4]ali.
\endverse



\beginverse
^Non mi ferme^rò un ^solo i^stante,
^io racconte^rò ^le tue ^opere
^perché sei il mio ^Dio, ^unico ^bene.
^Nulla mai po^trà ^la notte contro di \[B4]me.
\endverse
\beginchorus
^Dall'au^rora io ^cerco ^te,
^fino al tra^monto ti ^chia^mo.
^Ha sete ^solo di ^te l'^anima ^mia
come ^terra de^ser^ta.

\endchorus


%%%%%% EV. FINALE

\beginchorus %oppure \beginverse*
\vspace*{1.3\versesep}
\textnote{Finale } %<<< EV. INDICAZIONI
\[C#-]Ha sete \[G#-]solo di \[A]te l'\[B]anima \[E]mia
come \[A]terra de\[B]se-\[B]e-r\[E]ta.
\endchorus  %oppure \endverse



\endsong


%titolo{Danza la vita}
%autore{}
%album{}
%tonalita{Re}
%famiglia{Liturgica}
%gruppo{}
%momenti{Congedo}
%identificatore{danza_la_vita}
%data_revisione{2012_11_05}
%trascrittore{Francesco Endrici}
\beginsong{Danza la vita}[]
\beginverse
\[D]Canta con la \[G]voce e con il \[D]cuore, \[G]
\[D]con la bocca e \[G]con la vita, \[D] \[G]
\[D]canta senza \[G]stonature, \[D] \[G]
la \[D]verità \[G] del \[D]cuore. \[G]
\[D]Canta come \[G]cantano i viandanti 
\echo{\[D]canta come \[G]cantano i viandanti}
non \[D]solo per riem\[G]pire il tempo, 
\echo{non \[D]solo per \[G]riempire il tempo,}
\[D]Ma per soste\[G]nere lo sforzo 
\echo{\[D]Ma per soste\[G]nere lo sforzo.}
\[D]Canta \[G] e cam\[D]mina \[G]
\[D]Canta \[G] e cam\[D]mina \[G]
Se \[A]poi, credi non possa ba\[B-]stare
segui il \[E]tempo, stai \[G]pron\[A]to e
\endverse

\beginchorus
\[D]Danza la \[G]vita, al \[A]ritmo dello \[D]Spirito. 
\qquad \quad \echo{Spirito che riempi i nostri }
\[B-]Danza, \[G]danza al \[A]ritmo che c'è in \[D]te. 
\echo{cuor, danza assieme a noi. Danza}
\[G]Spirito \[A]che \[D]riempi i nostri 
\echo{la vita al ritmo dello Spirito}
\[B-]cuor. \[G]Danza assieme a \[A]no\[D]i. 
\echo{Danza, danza al ritmo che c'è in te.}
\endchorus
\beginverse
\chordsoff
Cam^mina sulle ^orme del Si^gnore, ^
non ^solo con i ^piedi ^ma ^ \brk ^usa soprat^tutto il cuore.^^
^Ama ^ chi è con ^te. ^
Cam^mina con lo ^zaino sulle spalle 
\echo{Cam^mina con lo ^zaino sulle spalle}
^la fatica a^iuta a crescere 
\echo{^la fatica a^iuta a crescere}
^nella con^divisione 
\echo{^nella con^divisione.}
^Canta ^ e cam^mina, ^
^canta  ^ e cam^mina. ^
Se ^poi, credi non possa ba^stare
segui il ^tempo, stai ^pron^to e
\endverse
\beginchorus
\chordsoff 
Rit. 
\endchorus
\endsong
%-------------------------------------------------------------
%			INIZIO	CANZONE
%-------------------------------------------------------------


%titolo: 	Del tuo Spirito, Signore
%autore: 	Gen Verde, Gen Rosso
%tonalita: 	Re



%%%%%% TITOLO E IMPOSTAZONI
\beginsong{Del tuo Spirito, Signore}[by={Gen\ Verde, Gen\ Rosso}]	% <<< MODIFICA TITOLO E AUTORE
\transpose{0} 						% <<< TRASPOSIZIONE #TONI (0 nullo)
\momenti{Cresima; Salmi}							% <<< INSERISCI MOMENTI	
% momenti vanno separati da ; e vanno scelti tra:
% Ingresso; Atto penitenziale; Salmi; Acclamazione al Vangelo; Dopo il Vangelo; Offertorio; Comunione; Ringraziamento; Fine; Santi; Pasqua; Avvento; Natale; Quaresima; Canti Mariani; Battesimo; Prima Comunione; Cresima; Matrimonio; Meditazione; Spezzare del pane;
\ifchorded
	%\textnote{Tonalità migliore }	% <<< EV COMMENTI (tonalità originale/migliore)
\fi


%%%%%% INTRODUZIONE
\ifchorded
\vspace*{\versesep}
\musicnote{
\begin{minipage}{0.48\textwidth}
\textbf{Intro}
\hfill 
%( \eighthnote \, 80)   % <<  MODIFICA IL TEMPO
% Metronomo: \eighthnote (ottavo) \quarternote (quarto) \halfnote (due quarti)
\end{minipage}
} 	
\vspace*{-\versesep}
\beginverse*


\nolyrics

%---- Prima riga -----------------------------
\vspace*{-\versesep}
\[A*] \[B-] \[F#-*]  \[G]	 % \[*D] per indicare le pennate, \rep{2} le ripetizioni

%---- Ogni riga successiva -------------------
\vspace*{-\versesep}
\[D] \[E-*]  \[D]	

%---- Ev Indicazioni -------------------------			
\textnote{\textit{[come la seconda parte del ritornello]} }	

\endverse
\fi






\beginchorus
|\[D*] Del tuo |\[G]Spiri\[D*]to, Si|\[G*]gno\[D]re,
\[A*]è |\[B-]piena la \[F#-*]ter|\[G]ra, è |\[D]piena la \[E-*]ter|\[D]ra. 
\endchorus



\musicnote{\textit{[dolce, arpeggiato]}}

\beginverse
\memorize
|\[C]Benedici il Si|\[B&]gnore, \mbar{2}{4}\[D-]anima \mbar{4}{4}\[A-*]mi\[B&]a,
Si\[C*]gnore, \mbar{3}{4}\[F*]Dio,\[C] tu sei |\[G*]gran\[C]de!
|\[C]Sono immense, splen|\[B&]denti
\mbar{2}{4}\[D-]tutte le tue \mbar{3}{4}\[B&*]ope\[F]re e |\[G-]tutte le crea\[A*]tu\mbar{4}{4}\[D]re.
\endverse



\beginverse
%\chordsoff
^Se tu togli il tuo ^soffio ^muore ogni ^co^sa
e ^si dis^sol^ve nella ^ter^ra.
^Il tuo spirito ^scende:
^tutto si ri^cre^a e ^tutto si rin^no^va.
\endverse


\beginverse
%\chordsoff
^La tua gloria, Si^gnore, ^resti per ^sem^pre.
Gio^isci, ^Di^o, del cre^a^to.
^Questo semplice ^canto
^salga a te Si^gno^re, sei ^tu la nostra ^gio^ia.
\endverse


\endsong


%-------------------------------------------------------------
%			INIZIO	CANZONE
%-------------------------------------------------------------


%titolo: 	Dio in me
%autore: 	SERMIG
%tonalita: 	Fa 



%%%%%% TITOLO E IMPOSTAZONI
\beginsong{Dio in me}[by={Sermig}] 	% <<< MODIFICA TITOLO E AUTORE
\transpose{0} 						% <<< TRASPOSIZIONE #TONI (0 nullo)
%\preferflats  %SE VOGLIO FORZARE i bemolle come alterazioni
%\prefersharps %SE VOGLIO FORZARE i # come alterazioni
\momenti{}							% <<< INSERISCI MOMENTI	
% momenti vanno separati da ; e vanno scelti tra:
% Ingresso; Atto penitenziale; Acclamazione al Vangelo; Dopo il Vangelo; Offertorio; Comunione; Ringraziamento; Fine; Santi; Pasqua; Avvento; Natale; Quaresima; Canti Mariani; Battesimo; Prima Comunione; Cresima; Matrimonio; Meditazione; Spezzare del pane;
\ifchorded
	%\textnote{Tonalità migliore }	% <<< EV COMMENTI (tonalità originale\migliore)
\fi


%%%%%% INTRODUZIONE
\ifchorded
\vspace*{\versesep}
\musicnote{
\begin{minipage}{0.48\textwidth}
\textbf{Intro}
\hfill 
%( \eighthnote \, 80)   % <<  MODIFICA IL TEMPO
% Metronomo: \eighthnote (ottavo) \quarternote (quarto) \halfnote (due quarti)
\end{minipage}
} 	
\vspace*{-\versesep}
\beginverse*

\nolyrics

%---- Prima riga -----------------------------
\vspace*{-\versesep}
\[F] \[A-] \[G4] \[G]

%---- Ogni riga successiva -------------------
\vspace*{-\versesep}
\[F] \[A-] \[G4] \[G]	

%---- Ev Indicazioni -------------------------			
%\textnote{\textit{(Oppure tutta la strofa)} }	

\endverse
\fi




%%%%% STROFA
\beginverse		%Oppure \beginverse* se non si vuole il numero di fianco
\memorize 		% <<< DECOMMENTA se si vuole utilizzarne la funzione
%\chordsoff		% <<< DECOMMENTA se vuoi una strofa senza accordi
Sei \[C]qui total\[F]mente \[C]D\[G]i\[A-]o 
dentro m\[G]e 
sei \[C]qui total\[F]mente \[C]u\[G]o\[A-]mo 
dentro \[G]me
\[C]E \[F]vuoi \[G]che \[C]io \[G]vi\[A-]va 
per \[G]te 
\[C]Sil\[F]e-\[G]e-\[C]nzio \[G]pre\[A-]ga 
con \[G]me con \[F]me 

\endverse






%%%%%% EV. INTERMEZZO
\beginverse*
\vspace*{1.3\versesep}
{
	\nolyrics
	\textnote{Intermezzo strumentale}
	
	\ifchorded

	%---- Prima riga -----------------------------
	\vspace*{-\versesep}
	\[F] \[A-] \[G4] \[G]

	%---- Ogni riga successiva -------------------
	\vspace*{-\versesep}
	\[F] \[A-] \[G4] \[G]


	\fi
	%---- Ev Indicazioni -------------------------			
	%\textnote{\textit{(ripetizione della strofa)}} 
	 
}
\vspace*{\versesep}
\endverse


%%%%% STROFA
\beginverse		%Oppure \beginverse* se non si vuole il numero di fianco
%\memorize 		% <<< DECOMMENTA se si vuole utilizzarne la funzione
%\chordsoff		% <<< DECOMMENTA se vuoi una strofa senza accordi

Per \[C]me ti sei \[F]fatto \[C]u\[G]o\[A-]mo 
come \[G]me 
La \[C]Croce tre\[F]menda \[C]più \[G]non \[A-]è 
dopo \[G]che 
\[C]Tu \[F]l’hai \[G]re\[C]sa \[G]bene\[A-]det\[G]ta 
\[C]Sil\[F]e-\[G]e-\[C]nzio \[G]pre\[A-]ga 
con \[G]me
\[C]A\[F]de-\[G]e-\[C]sso \[G]incontr\[A-]ando \[G]me 
\[C]non \[F]tro\[G]ve\[C]re\[G]te \[A-]me 
ma \[G]Dio
 in \[F]me  \[A-] \[G4] \[G]  
 in \[F]me \[A-] \[G4] \[G]


\endverse



%%%%%% EV. CHIUSURA SOLO STRUMENTALE
\ifchorded
\beginchorus %oppure \beginverse*
\vspace*{1.3\versesep}
\textnote{Chiusura strumentale } %<<< EV. INDICAZIONI

\nolyrics
%---- Prima riga -----------------------------
\vspace*{-\versesep}
\[F] \[A-] \[G4] \[G]

%---- Ogni riga successiva -------------------
\vspace*{-\versesep}
\[F] \[A-] \[G4] \[G]

%---- Ev Indicazioni -------------------------			
%\textnote{\textit{(Oppure tutta la strofa)} }	

\endchorus  %oppure \endverse
\fi


\endsong
%------------------------------------------------------------
%			FINE CANZONE
%------------------------------------------------------------




%-------------------------------------------------------------
%			INIZIO	CANZONE
%-------------------------------------------------------------


%titolo: 	Dove vita è davvero
%autore: 	E. Sarini
%tonalita: 	Sol 



%%%%%% TITOLO E IMPOSTAZONI
\beginsong{Dove vita è davvero}[by={E. Sarini}] 
\transpose{0} 						% <<< TRASPOSIZIONE #TONI (0 nullo)
\momenti{Comunione; Fine}							% <<< INSERISCI MOMENTI	
% momenti vanno separati da ; e vanno scelti tra:
% Ingresso; Atto penitenziale; Acclamazione al Vangelo; Dopo il Vangelo; Offertorio; Comunione; Ringraziamento; Fine; Santi; Pasqua; Avvento; Natale; Quaresima; Canti Mariani; Battesimo; Prima Comunione; Cresima; Matrimonio; Meditazione; Spezzare del pane;
\ifchorded
	%\textnote{Tonalità originale }	% <<< EV COMMENTI (tonalità originale/migliore)
\fi

%%%%%% INTRODUZIONE
\ifchorded
\vspace*{\versesep}
\musicnote{
\begin{minipage}{0.48\textwidth}
\textbf{Intro}
\hfill 
%( \eighthnote \, 80)   % <<  MODIFICA IL TEMPO
% Metronomo: \eighthnote (ottavo) \quarternote (quarto) \halfnote (due quarti)
\end{minipage}
} 	
\vspace*{-\versesep}
\beginverse*


\nolyrics

%---- Prima riga -----------------------------
\vspace*{-\versesep}
\[C] \[G] \[C]	 % \[*D] per indicare le pennate, \rep{2} le ripetizioni

%---- Ogni riga successiva -------------------
%\vspace*{-\versesep}
%\[G] \[C]  \[D]	

%---- Ev Indicazioni -------------------------			
\textnote{\textit{[oppure la prima riga]} }	

\endverse
\fi




%%%%% STROFA
\beginverse		%Oppure \beginverse* se non si vuole il numero di fianco
\memorize 		% <<< DECOMMENTA se si vuole utilizzarne la funzione
%\chordsoff		% <<< DECOMMENTA se vuoi una strofa senza accordi

\[C]Cerchi un sor\[G]riso negli \[A-]occhi degli u\[C]omini
\[F]sogni avven\[C]ture che 
il \[D7]tempo porta con \[G]sé
\[C]danzi da \[G]sempre la \[E]gioia di \[A-]vivere
\[F] hai conosci\[C]uto l'uomo 
\[F] che ti ha par\[C]lato di un te\[F]so\[G]ro.

\endverse




%%%%% RITORNELLO
\beginchorus
\textnote{\textbf{Rit.}}

\[C] E quel te\[F]soro sai co\[C]s'è  \quad \[F*]\quad \[G*]
\[C] è la tua \[F]vita nell'a\[G]more 
\[F] è la gioia \[G]di chi annuncia 
l'u\[C*]omo che \[G*]torne\[F*]rà \[F]
e allora \[E-]sciogli i tuoi \[A-]piedi e va'
\[E-] tendi le \[A-]mani e va’ 
dove \[G#]vita è dav\[G7]vero.

\endchorus



%%%%% STROFA
\beginverse		%Oppure \beginverse* se non si vuole il numero di fianco
%\memorize 		% <<< DECOMMENTA se si vuole utilizzarne la funzione
%\chordsoff		% <<< DECOMMENTA se vuoi una strofa senza accordi

^Vivi nel ^mondo la ^storia degli u^omini
^apri il tuo ^cuore a chi 
nel ^mondo ha chiesto di ^te
^chiedi emozi^oni che ^corrano li^bere
^ ed hai cre^duto all'uomo 
^ che ti ha par^lato di un te^so^ro.

\endverse




%%%%% STROFA
\beginverse		%Oppure \beginverse* se non si vuole il numero di fianco
%\memorize 		% <<< DECOMMENTA se si vuole utilizzarne la funzione
%\chordsoff		% <<< DECOMMENTA se vuoi una strofa senza accordi

^Canti la ^pace nei ^gesti degli ^uomini
^offri spe^ranza a chi 
da ^tempo domanda un per^ché.
^Vivi l'at^tesa del ^giorno che ^libera
^ ed hai a^mato l'uomo 
^ che ti ha par^lato di un te^so^ro.

\endverse





\endsong
%------------------------------------------------------------
%			FINE CANZONE
%------------------------------------------------------------



%EEE
%titolo{È bello lodarti}
%autore{Gen Verde}
%album{È bello lodarti}
%tonalita{Sol}
%famiglia{Liturgica}
%gruppo{}
%momenti{Ingresso}
%identificatore{e_bello_lodarti}
%data_revisione{2011_12_31}
%trascrittore{Francesco Endrici}
\beginsong{È bello lodarti}[by={Gen\ Verde}]
\beginchorus
\[G] È \[D]bello can\[C]tare il tuo a\[G]more,
\[A-] è \[7]bello lo\[D]dare il tuo nome.
\[G] È \[B4]bello can\[C]tare il tuo amore,
è \[G]bello lo\[D]darti, Si\[C]gnore,
è \[G]bello can\[D]tare a \[C]te!
\endchorus

\beginverse
\[E-]Tu che sei l'amore infi\[B-6]nito
che nep\[C]pure il cielo può contenere,
ti \[A-]sei fatto \[7]uomo, \[D6]Tu sei venuto qui
ad \[B7]abitare in mezzo a \[C]noi, allora\dots \[G] 
\endverse

\beginverse
^Tu che conti tutte le ^stelle
e le ^chiami una ad una per nome,
da ^mille sen^tieri ^ci hai radunati qui,
^ci hai chiamati figli ^tuoi, allora\dots ^
\endverse
\endsong


%-------------------------------------------------------------
%			INIZIO	CANZONE
%-------------------------------------------------------------


%titolo: 	E camminava con loro
%autore: 	???
%tonalita: 	Fa 



%%%%%% TITOLO E IMPOSTAZONI
\beginsong{E camminava con loro}[by={}] 	% <<< MODIFICA TITOLO E AUTORE
\transpose{0} 						% <<< TRASPOSIZIONE #TONI (0 nullo)
%\preferflats  %SE VOGLIO FORZARE i bemolle come alterazioni
%\prefersharps %SE VOGLIO FORZARE i # come alterazioni
\momenti{}							% <<< INSERISCI MOMENTI	
% momenti vanno separati da ; e vanno scelti tra:
% Ingresso; Atto penitenziale; Acclamazione al Vangelo; Dopo il Vangelo; Offertorio; Comunione; Ringraziamento; Fine; Santi; Pasqua; Avvento; Natale; Quaresima; Canti Mariani; Battesimo; Prima Comunione; Cresima; Matrimonio; Meditazione; Spezzare del pane;
\ifchorded
	%\textnote{Tonalità migliore }	% <<< EV COMMENTI (tonalità originale\migliore)
\fi


%%%%%% INTRODUZIONE
\ifchorded
\vspace*{\versesep}
\musicnote{
\begin{minipage}{0.48\textwidth}
\textbf{Intro}
\hfill 
%( \eighthnote \, 80)   % <<  MODIFICA IL TEMPO
% Metronomo: \eighthnote (ottavo) \quarternote (quarto) \halfnote (due quarti)
\end{minipage}
} 	
\vspace*{-\versesep}
\beginverse*


\nolyrics

%---- Prima riga -----------------------------
\vspace*{-\versesep}
\[F]\[C]\[D-] % \[*D] per indicare le pennate, \rep{2} le ripetizioni

%---- Ogni riga successiva -------------------
\vspace*{-\versesep}
\[G7]\[C7]		

%---- Ev Indicazioni -------------------------			
%\textnote{\textit{(Oppure tutta la strofa)} }	

\endverse
\fi




%%%%% STROFA
\beginverse		%Oppure \beginverse* se non si vuole il numero di fianco
\memorize 		% <<< DECOMMENTA se si vuole utilizzarne la funzione
%\chordsoff		% <<< DECOMMENTA se vuoi una strofa senza accordi

\[F]Ecco il tempo nuovo in c\[C]ui
il Padre mio \[D-]darà
la Grazia ad ogni \[G7]uo\[C7]mo.
\[F]Già lo Spirito che è in \[C]me
annuncia libe\[D-]rtà,
ai poveri la \[B&-]gio\[C7]ia.
\endverse
\beginverse*	
\[D-]Croce fu e si \[C]spense in noi
la Pa\[B&]rola che dava \[C]luce agli occhi.
\[D-]"Stolti e tardi nel \[C]credere";
la sua \[SI&]voce scaldò il \[G7]cuo\[C]re.
\endverse


%%%%% RITORNELLO
\beginchorus
\textnote{\textbf{Rit.}}

\[F]Come un volto \[C]amico, \[D-]verità \[A-]inattesa:
\[B&]è Parola \[F]eterna, \[B&]Pane \[C]vi\[C7]vo.
\[F]Corre nuovo i\[C]l passo, \[D-]carico d'\[A-]annuncio:
\[B&]è risorto, \[F]vive \[B&]e cam-\[F]mi-\[C]na con \[F]noi.

\endchorus




%%%%%% EV. INTERMEZZO
\beginverse*
\vspace*{1.3\versesep}
{
	\nolyrics
	\textnote{Intermezzo strumentale}
	
	\ifchorded

	%---- Prima riga -----------------------------
	\vspace*{-\versesep}
	\[F]\[C]\[D-]
	%---- Ogni riga successiva -------------------
	\vspace*{-\versesep}
    \[G7]\[C7]


	\fi
	%---- Ev Indicazioni -------------------------			
	%\textnote{\textit{(ripetizione della strofa)}} 
	 
}
\vspace*{\versesep}
\endverse


%%%%% STROFA
\beginverse		%Oppure \beginverse* se non si vuole il numero di fianco
\memorize 		% <<< DECOMMENTA se si vuole utilizzarne la funzione
%\chordsoff		% <<< DECOMMENTA se vuoi una strofa senza accordi
\[F]Questo pane che vi \[C]do
è il corpo mio per \[D-]voi;
sia fatto in mia me\[G7]mor\[C7]ia.
\[F]Questo calice sa\[C]rà,
nel sangue mio per \[D-]voi,
un'alleanza \[B&-]nuo\[C7]va.
\endverse
\beginverse*	
\[D-]Croce fu e fug\[C]gimmo noi,
rinne\[B&]gando chi era am\[C]ore eterno.
\[D-]"Resta qui, si fa \[C]sera ormai";
e di\[B&]vise ancora il \[G7]pa\[C]ne.
\endverse




%%%%% STROFA
\beginverse		%Oppure \beginverse* se non si vuole il numero di fianco
%\memorize 		% <<< DECOMMENTA se si vuole utilizzarne la funzione
%\chordsoff		% <<< DECOMMENTA se vuoi una strofa senza accordi
\[F]"Io vi mando ad annun\[C]ciar
la pace a chi non \[D-]sa
che il Regno si avv\[G7]ici\[C7]na.
\[F]Chi vi accoglie, in veri\[C]tà,
accoglie anche \[D-]me
e chi mi ha \[B&-]manda\[C7]to".
\endverse
\beginverse*	
\[D-]Croce fu ed in\[C]creduli
fummo so\[B&]rdi a chi lo di\[C]ceva vivo.
\[D-]"Era Lui per la \[C]via con noi";
ripa\[B&]rtimmo senza \[G7]indu\[C]gio.
\endverse





%%%%% RITORNELLO
\beginchorus
\textnote{\textbf{Rit.}}
\[F]Come un volto \[C]amico, \[D-]verità \[A-]inattesa:
\[B&]è Parola \[F]eterna, \[B&]Pane \[C]vi\[C7]vo.
\[F]Corre nuovo i\[C]l passo, \[D-]carico d'\[A-]annuncio:
\[B&]è risorto, \[F]vive \[B&]e 
\endchorus


%%%%%% EV. FINALE

\beginchorus %oppure \beginverse*
\vspace*{1.3\versesep}
\textnote{\textbf{Finale} \textit{(rallentando)}} %<<< EV. INDICAZIONI
 cam-\[F]mi-\[C]na con \[F]noi. \[F*]
\endchorus  %oppure \endverse







\endsong
%------------------------------------------------------------
%			FINE CANZONE
%------------------------------------------------------------




%-------------------------------------------------------------
%			INIZIO	CANZONE
%-------------------------------------------------------------


%titolo: 	E la gioia
%autore: 	
%tonalita: 	Do



%%%%%% TITOLO E IMPOSTAZONI
\beginsong{E la gioia}[by={Autore sconosciuto}] 	% <<< MODIFICA TITOLO E AUTORE
\transpose{0} 						% <<< TRASPOSIZIONE #TONI (0 nullo)
%\preferflats  %SE VOGLIO FORZARE i bemolle come alterazioni
%\prefersharps %SE VOGLIO FORZARE i # come alterazioni
\momenti{Matrimonio; Congedo; Comunione; Ringraziamento}				 			% <<< INSERISCI MOMENTI	
% momenti vanno separati da ; e vanno scelti tra:
% Ingresso; Atto penitenziale; Acclamazione al Vangelo; 
% Dopo il Vangelo; Offertorio; Comunione; Ringraziamento; 
% Fine; Santi; Pasqua; Avvento; Natale; Quaresima; Canti Mariani; 
% Battesimo; Prima Comunione; Cresima; Matrimonio; Meditazione; Spezzare del pane;
\ifchorded
	%\textnote{Tonalità migliore }	% <<< EV COMMENTI (tonalità originale/migliore)
\fi

%%%%%% INTRODUZIONE
\ifchorded
\vspace*{\versesep}
\musicnote{
\begin{minipage}{0.48\textwidth}
\textbf{Intro}
\hfill 
%( \eighthnote \, 80)   % <<  MODIFICA IL TEMPO
% Metronomo: \eighthnote (ottavo) \quarternote (quarto) \halfnote (due quarti)
\end{minipage}
} 	
\vspace*{-\versesep}
\beginverse*


\nolyrics

%---- Prima riga -----------------------------
\vspace*{-\versesep}
\[C] \[G] \[D] 	\[E-] % \[*D] per indicare le pennate, \rep{2} le ripetizioni

%---- Ogni riga successiva -------------------
\vspace*{-\versesep}
\[C] \[G] \[D]   \[D]	

%---- Ev Indicazioni -------------------------			
%\textnote{\textit{(Oppure tutta la strofa)} }	

\endverse
\fi




%%%%% STROFA
\beginverse		%Oppure \beginverse* se non si vuole il numero di fianco
\memorize 		% <<< DECOMMENTA se si vuole utilizzarne la funzione
%\chordsoff		% <<< DECOMMENTA se vuoi una strofa senza accordi


\[C] Voglio cantare con \[G]voi \brk la  \[D]gioia di essere \[E-]qui
\[C] per dare una ris\[G]posta \brk a \[D]chi ci ha chia\[E-]mati.
\[C] E insieme ci ha vo\[G]luti \brk per \[D]dare spe\[E-]ranza
a \[C]tutti quelli \[C]che sono intorno a \[D]noi.

\endverse


%%%%% RITORNELLO
\beginchorus
\textnote{\textbf{Rit.}}

E la \[C]gioia entra \[E-]dentro 
e scom\[D]bussola o\[E-]gni cosa 
\[C]perché va \[G]dritta al \[D]cuor. \[D]
Vuole \[C]esser di \[E-]tutti 
\[D]non te\[E-]niamola per \[C]noi
ma but\[G]tiamola \[D]fuori! \[D]

\endchorus


%%%%% STROFA
\beginverse		%Oppure \beginverse* se non si vuole il numero di fianco
%\memorize 		% <<< DECOMMENTA se si vuole utilizzarne la funzione
%\chordsoff		% <<< DECOMMENTA se vuoi una strofa senza accordi

^ Hai lasciato la ^tua casa, le ^tue comodi^tà,
^ rischiando un po-o’ di ^te, le ^tue sicu^rezze.
^ Per cerca-are qualche ^cosa
che ^dia fuoco al ^cuore,
e tro^vare così il ^senso della ^vita.

\endverse



%%%%% RITORNELLO
\beginchorus
\textnote{\textbf{Rit.}}

E la \[C]gioia entra \[E-]dentro 
e scom\[D]bussola o\[E-]gni cosa 
\[C]perché va \[G]dritta al \[D]cuor. \[D]
Vuole \[C]esser di \[E-]tutti 
\[D]non te\[E-]niamola per \[C]noi
ma but\[G]tiamola \[D]fuori! \[D]

\endchorus




%%%%% STROFA
\beginverse		%Oppure \beginverse* se non si vuole il numero di fianco
%\memorize 		% <<< DECOMMENTA se si vuole utilizzarne la funzione
%\chordsoff		% <<< DECOMMENTA se vuoi una strofa senza accordi

^ Non opporre resis^tenza \brk alla ^voce che ^chiama,
^ lasciati invece an^dare, \brk ^senza te^mere,
^ vedrai che arrive^rai \brk a pro^vare cos’è l’a^more
che ^prende ogni ^cosa e poi tutto ^dà.

\endverse

%%%%% RITORNELLO
\beginchorus
\textnote{\textbf{Rit.}}

E la \[C]gioia entra \[E-]dentro 
e scom\[D]bussola o\[E-]gni cosa 
\[C]perché va \[G]dritta al \[D]cuor. \[D]
Vuole \[C]esser di \[E-]tutti 
\[D]non te\[E-]niamola per \[C]noi
ma but\[G]tiamola \[D]fuori! \[D]  \rep{3}

\endchorus




\endsong
%------------------------------------------------------------
%			FINE CANZONE
%------------------------------------------------------------


%titolo{È più bello insieme}
%autore{Gen Verde}
%album{Accordi}
%tonalita{Re}
%famiglia{Liturgica}
%gruppo{}
%momenti{Congedo}
%identificatore{e_piu_bello_insieme}
%data_revisione{2011_12_31}
%trascrittore{Francesco Endrici}
\beginsong{È più bello insieme}[by={Gen\ Verde},ititle={Insieme è più bello}]
\ifchorded
\beginverse*
\vspace*{-0.8\versesep}
{\nolyrics \[D]\[A]\[G]\[G]\[D]\[A]\[G]}
\vspace*{-\versesep}
\endverse
\fi
\beginverse
\memorize
\[D] Dietro i volti \[A]sconosciuti \[G]
della gente \[D4]che mi \[D]sfiora, \[G]
quanta vita, \[D]quante attese \[E-]di felici\[A4]tà, \[A]\[D]
quanti atti\[A]mi vissuti, \[G]
mondi da sco\[D4]prire an\[D]cora, \[G]
splendidi uni\[D]versi accanto a \[A4]me. \[A7]
\endverse
\beginchorus
\[G]È più \[A]bello insieme
è un \[B-]dono grande l'\[7]altra gente,
\[G]è più \[A]bello insieme. \[B-7 (G)] \rep{2}
\endchorus
\ifchorded
\beginverse*
\vspace*{-\versesep}
{\nolyrics \[D]\[A]\[G]\[G]\[D]\[A]\[G]\[G]}
\endverse
\fi
\beginverse
^ E raccolgo ^nel mio cuore ^
la speranza ed ^il do^lore ^
il silenzio, il ^pianto della ^gente attorno a ^me. ^^
In quel pianto, in ^quel sorriso ^
è il mio pianto, il ^mio sor^riso ^
chi mi vive ac^canto è un altro ^me. ^
\endverse
\ifchorded
\beginverse*
\vspace*{-\versesep}
{\nolyrics \[D]\[A]\[G]\[G]\[D]\[A]\[G]\[G]\[B]\[E]}
\endverse
\fi
\beginverse
\transpose{2}
^ Fra le case e i ^grattacieli, ^
fra le antenne ^lassù in ^alto, ^
così traspa^rente il cielo ^non l'ho visto ^mai. ^^
E la luce ^getta veli ^
di colore ^sull'a^sfalto ^
ora che can^tate assieme a ^me.
\endverse
\beginchorus
\transpose{2}
\[G]È più \[A]bello insieme
è un \[B-]dono grande l'\[7]altra gente,
\[G]è più \[A]bello insieme. \[B-7 (G)] \rep{2}
\endchorus
\ifchorded
\beginverse*
\vspace*{-\versesep}
{\nolyrics \[D]\[A]\[G]\[G]\[D]\[A]\[G]\[G]}
\endverse
\fi
\endsong


%-------------------------------------------------------------
%			INIZIO	CANZONE
%-------------------------------------------------------------


%titolo: 	E sei rimasto qui
%autore: 	Gen\ Rosso
%tonalita: 	Fa 



%%%%%% TITOLO E IMPOSTAZONI
\beginsong{E sei rimasto qui}[by={Gen\ Rosso}]
\transpose{0} 						% <<< TRASPOSIZIONE #TONI (0 nullo)
%\preferflats  %SE VOGLIO FORZARE i bemolle come alterazioni
%\prefersharps %SE VOGLIO FORZARE i # come alterazioni
\momenti{Comunione; Cresima}							% <<< INSERISCI MOMENTI	
% momenti vanno separati da ; e vanno scelti tra:
% Ingresso; Atto penitenziale; Acclamazione al Vangelo; Dopo il Vangelo; Offertorio; Comunione; Ringraziamento; Fine; Santi; Pasqua; Avvento; Natale; Quaresima; Canti Mariani; Battesimo; Prima Comunione; Cresima; Matrimonio; Meditazione; Spezzare del pane;
\ifchorded
	%\textnote{Tonalità migliore }	% <<< EV COMMENTI (tonalità originale/migliore)
\fi



%%%%%% INTRODUZIONE
\ifchorded
\vspace*{\versesep}
\musicnote{
\begin{minipage}{0.48\textwidth}
\textbf{Intro}
\hfill 
%( \eighthnote \, 80)   % <<  MODIFICA IL TEMPO
% Metronomo: \eighthnote (ottavo) \quarternote (quarto) \halfnote (due quarti)
\end{minipage}
} 	
\vspace*{-\versesep}
\beginverse*


\nolyrics

%---- Prima riga -----------------------------
\vspace*{-\versesep}
 \[F] \[C] \[F] 	 % \[*D] per indicare le pennate, \rep{2} le ripetizioni

%---- Ogni riga successiva -------------------
\vspace*{-\versesep}
\[F] \[C] \[B&]	

%---- Ev Indicazioni -------------------------			
%\textnote{\textit{(Oppure tutta la strofa)} }	

\endverse
\fi



%%%%% STROFA
\beginverse		%Oppure \beginverse* se non si vuole il numero di fianco
\memorize 		% <<< DECOMMENTA se si vuole utilizzarne la funzione
%\chordsoff		% <<< DECOMMENTA se vuoi una strofa senza accordi
\[F] Perché la sete d'infi\[B&/F]nito? \brk \[G-] Perché la fame d'immor\[D-]tali\[C]tà?
\[F] Sei Tu che hai messo dentro \[B&/F]l'uomo  \brk \[G-] il desiderio dell'e\[F/C]terni\[C]tà!
Ma \[G-]Tu sapevi \[F/A]che quel vuoto \[B&]lo colmavi \[F/A]Tu,
per \[G-]questo sei ve\[F/A]nuto in mezzo a \[C]noi.
\endverse




%%%%% RITORNELLO
\beginchorus
\textnote{\textbf{Rit.}}
E \[F]sei rimasto qui, \[B&/F]visibile mistero.
E \[F]sei rimasto qui, \[D-]cuore del mondo in\[C]tero.
E \[B&]rimarrai con noi fin\[A-]ché quest'uni\[D-]verso gire\[G]rà.
Sal\[F]vezza dell'u\[C]mani\[F]tà. \[C] 
\endchorus





%%%%% STROFA
\beginverse		%Oppure \beginverse* se non si vuole il numero di fianco
%\memorize 		% <<< DECOMMENTA se si vuole utilizzarne la funzione
%\chordsoff		% <<< DECOMMENTA se vuoi una strofa senza accordi
\[D] Si apre il cielo del fu\[G/D]turo,   \brk \[E-] il muro della morte or\[B-]mai non \[A]c'è.
\[D]Tu, Pane vivo, ci fai \[G/B]Uno:\brk \[E-]  richiami tutti i figli at\[D]torno a \[A]Te.
E \[E-]doni il tuo \[D/F#]Spirito che \[G]lascia dentro \[D/F#]noi
il \[E-]germe della  \[D/F#]sua immortali\[A]tà.
\endverse






%%%%% RITORNELLO
\beginchorus
\textnote{\textbf{Rit.}}
\[D]Sei rimasto qui, \[G/D]visibile mistero.
\[D]Sei rimasto qui, \[B-]cuore del mondo in\[A]tero.
E \[G]rimarrai con noi fin\[F#-]ché quest'uni\[B-]verso gire\[E]rà.
Sal\[D]vezza dell'u\[A]mani\[D]tà. \[C] 
\endchorus




%%%%% STROFA
\beginverse		%Oppure \beginverse* se non si vuole il numero di fianco
%\memorize 		% <<< DECOMMENTA se si vuole utilizzarne la funzione
%\chordsoff		% <<< DECOMMENTA se vuoi una strofa senza accordi
^ Presenza vera nel mi^stero, \brk ^  ma più reale di ogni ^realtà, ^
^ da te ogni cosa prende ^vita  \brk ^ e tutto un giorno a te ri^torne^rà.
Var^cando l'infi^nito tutti ^troveremo in ^Te
un ^Sole immenso ^di felici\[G-]tà. \[A-]\[B&]\[C]
\endverse




%%%%% STROFA
\beginverse*		%Oppure \beginverse* se non si vuole il numero di fianco
%\memorize 		% <<< DECOMMENTA se si vuole utilizzarne la funzione
%\chordsoff		% <<< DECOMMENTA se vuoi una strofa senza accordi
\[F/A]Noi,  \[B&/D]trasformati in \[C]Te, sa\[F]remo il seme \[B&]che
fa\[G-]rà fiorire l'\[F/A]universo \[B&]nella Trini\[C]tà.
\[F/A]Noi,  \[B&/D]trasformati in \[C]Te, sa\[F]remo il seme \[B&]che
fa\[G-]rà fiorire \[F/A]tutto l'uni\[B&]verso insieme a \[C]Te.
\endverse


%%%%% RITORNELLO
\beginchorus
\textnote{\textbf{Rit.}}
E \[G]sei rimasto qui, vi\[C]sibile mistero.
\[G]Sei rimasto qui, \[E-]cuore del mondo in\[D]tero.
E \[C]rimarrai con noi \brk fin\[B-]ché quest'uni\[E-]verso gire\[A]rà.  \[D] 
\[G]Sei rimasto qui, vi\[C]sibile mistero.
\[G]Sei rimasto qui, \[E-]cuore del mondo in\[D]tero.
E \[C]rimarrai con noi fin\[B-]ché quest'uni\[E-]verso gire\[A]rà.
\[G/D]Ieri oggi e sempre. \[A-] \[G/B] 
\[C]Sal\[C7+/D]vezza dell'u\[D]manità. \[G] 

\endchorus



%%%%%% EV. CHIUSURA SOLO STRUMENTALE
\ifchorded
\beginchorus %oppure \beginverse*
\vspace*{1.3\versesep}
\textnote{Chiusura strumentale} %<<< EV. INDICAZIONI

\[D] \[G] \[G/B] \[D] \[G] \[C] \[G] 

\endchorus  %oppure \endverse
\fi


\endsong
%------------------------------------------------------------
%			FINE CANZONE
%------------------------------------------------------------




%-------------------------------------------------------------
%			INIZIO	CANZONE
%-------------------------------------------------------------


%titolo: 	E sono solo un uomo
%autore: 	Sequeri
%tonalita: 	Re



%%%%%% TITOLO E IMPOSTAZONI
\beginsong{E sono solo un uomo}[by={Sequeri}]	% <<< MODIFICA TITOLO E AUTORE
\transpose{0} 						% <<< TRASPOSIZIONE #TONI (0 nullo)
\momenti{Comunione}							% <<< INSERISCI MOMENTI	
% momenti vanno separati da ; e vanno scelti tra:
% Ingresso; Atto penitenziale; Acclamazione al Vangelo; Dopo il Vangelo; Offertorio; Comunione; Ringraziamento; Fine; Santi; Pasqua; Avvento; Natale; Quaresima; Canti Mariani; Battesimo; Prima Comunione; Cresima; Matrimonio; Meditazione; Spezzare del pane;
\ifchorded
	%\textnote{Tonalità migliore }	% <<< EV COMMENTI (tonalità originale/migliore)
\fi


%%%%%% INTRODUZIONE
\ifchorded
\vspace*{\versesep}
\textnote{Intro: \qquad \qquad  }%(\eighthnote 116) % <<  MODIFICA IL TEMPO
% Metronomo: \eighthnote (ottavo) \quarternote (quarto) \halfnote (due quarti)
\vspace*{-\versesep}
\beginverse*

\nolyrics

%---- Prima riga -----------------------------
\vspace*{-\versesep}
\[D] \[F#-] \[G] \[D]	 % \[*D] per indicare le pennate, \rep{2} le ripetizioni

%---- Ogni riga successiva -------------------
%\vspace*{-\versesep}
%\[G] \[C]  \[D]	

%---- Ev Indicazioni -------------------------			
%\textnote{\textit{(Oppure tutta la strofa)} }	

\endverse
\fi



%%%%% STROFA
\beginverse		%Oppure \beginverse* se non si vuole il numero di fianco
\memorize 		% <<< DECOMMENTA se si vuole utilizzarne la funzione
%\chordsoff		% <<< DECOMMENTA se vuoi una strofa senza accordi
\[D]Io lo so Si\[F#-]gnore che \[G]vengo da lon\[D]tano
\[D]prima nel pen\[F#-]siero e \[G]poi nella tua \[A7]mano
\[D]io mi rendo \[A]conto che \[G]Tu sei la mia \[D]vita
e \[G]non mi sembra \[E-]vero di pre\[E7]garti co\[A4/7]sì.
\endverse

\beginverse*
^Padre di ogni ^uomo e ^non ti ho visto ^mai.
^Spirito di ^Vita e ^nacqui da una ^donna
^Figlio mio fra^tello e ^sono solo un ^uomo
ep^pure io ca^pisco che ^Tu sei veri\[A7]tà.
\endverse



%%%%% RITORNELLO
\beginchorus
E im\[D]parerò a guar\[G]dare tutto il \[A]mondo \[D7]
con gli \[G]occhi traspa\[A]renti di un bam\[D]bino \[D7]
e in\[G]segnerò a chia\[A]marti Padre \[D]nostro \[B7]
ad \[E-]ogni figlio \[E7 ]che diventa \[A7 ]uomo. \[A7 ] \ifchorded
E im\[D]parerò a guar\[G]dare tutto il \[A]mondo \[D7]
con gli \[G]occhi traspa\[A]renti di un bam\[D]bino \[D7]
e in\[G]segnerò a chia\[A]marti Padre \[D]nostro \[B7]
ad \[E-]ogni figlio \[A7]che diventa \[D]uomo. \[G]  \[D]

\else \rep{2}
\fi
\endchorus



%%%%% STROFA
\beginverse		%Oppure \beginverse* se non si vuole il numero di fianco
%\memorize 		% <<< DECOMMENTA se si vuole utilizzarne la funzione
%\chordsoff		% <<< DECOMMENTA se vuoi una strofa senza accordi
^Io lo so Si^gnore che ^Tu mi sei vi^cino,
^luce alla mia ^mente ^guida al mio cam^mino,
^mano che sor^regge ^sguardo che per^dona
e ^non mi sembra ^vero che Tu e^sista co^sì.
\endverse

\beginverse*
^Dove nasce a^more ^Tu sei la sor^gente.
^Dove c'è una ^croce ^Tu sei la spe^ranza
^dove il tempo ha ^fine ^Tu sei vita e^terna
e ^so che posso ^sempre con^tare su di \[A7]Te.
\endverse



%%%%% RITORNELLO
\beginchorus
E ac\[D]coglierò la \[G]vita come un \[A]dono \[D7]
e a\[G]vrò il coraggio \[A]di morire an\[D]ch'io \[D7]
e in\[G]contro a Te ver\[A]rò col mio fra\[D]tello \[B7]
che \[E-]non si sente a\[E7 ]mato da nes\[A7]suno. \[A7] \ifchorded 
E ac\[D]coglierò la \[G]vita come un \[A]dono \[D7]
e a\[G]vrò il coraggio \[A]di morire an\[D]ch'io \[D7]
e in\[G]contro a Te ver\[A]rò col mio fra\[D]tello \[B7]
che \[E-]non si sente a\[A7]mato da nes\[D]suno.   \[G]  \[D*]
\else \rep{2}
\fi 
\endchorus


\endsong
%------------------------------------------------------------
%			FINE CANZONE
%------------------------------------------------------------




%-------------------------------------------------------------
%			INIZIO	CANZONE
%-------------------------------------------------------------


%titolo: 	Ecco il tuo post
%autore: 	Giombini
%tonalita: 	Do



%%%%%% TITOLO E IMPOSTAZONI
\beginsong{Ecco il tuo posto}[by={M. Giombini}] 	% <<< MODIFICA TITOLO E AUTORE
\transpose{0} 						% <<< TRASPOSIZIONE #TONI (0 nullo)
\momenti{Offertorio; Pasqua}							% <<< INSERISCI MOMENTI	
% momenti vanno separati da ; e vanno scelti tra:
% Ingresso; Atto penitenziale; Acclamazione al Vangelo; Dopo il Vangelo; Offertorio; Comunione; Ringraziamento; Fine; Santi; Pasqua; Avvento; Natale; Quaresima; Canti Mariani; Battesimo; Prima Comunione; Cresima; Matrimonio; Meditazione; Spezzare del pane;
\ifchorded
	%\textnote{Tonalità migliore }	% <<< EV COMMENTI (tonalità originale/migliore)
\fi


%%%%%% INTRODUZIONE
\ifchorded
\vspace*{\versesep}
\musicnote{
\begin{minipage}{0.48\textwidth}
\textbf{Intro}
\hfill 
%( \eighthnote \, 80)   % <<  MODIFICA IL TEMPO
% Metronomo: \eighthnote (ottavo) \quarternote (quarto) \halfnote (due quarti)
\end{minipage}
} 	
\vspace*{-\versesep}
\beginverse*
\nolyrics

%---- Prima riga -----------------------------
\vspace*{-\versesep}
\[C] \[G] \[F] \[C] % \[*D] per indicare le pennate, \rep{2} le ripetizioni


%---- Ev Indicazioni -------------------------			
\textnote{\textit{(oppure tutta la strofa)} }	

\endverse
\fi



%%%%% STROFA
\beginverse		%Oppure \beginverse* se non si vuole il numero di fianco
\memorize 		% <<< DECOMMENTA se si vuole utilizzarne la funzione
\[C]Ecco il tuo \[G]posto, \[F]vie\[C]ni, 
\[C]vieni a se\[F]derti fra \[G]noi
\[D-]e ti rac\[G]conte\[C]re\[A-]mo 
\[F]la nostra \[G]sto\[C]ria.
\endverse


%%%%% RITORNELLO
\beginchorus
\textnote{\textbf{Rit.}}
\[F]Quanto a\[C]more nel \[E]semi\[A-]nare,
\[F]quanta spe\[C]ranza nell'\[G]aspet\[C]tare,
\[A-]quanta fa\[E-]tica nel mi\[F]etere il \[C]grano
\[F]e vendem\[G]mia\[C]re.
\[F]e vendem\[G]mia\[C]re.
\endchorus



%%%%% STROFA
\beginverse		%Oppure \beginverse* se non si vuole il numero di fianco
%\memorize 		% <<< DECOMMENTA se si vuole utilizzarne la funzione
^Accanto al ^fuoco, vi^e^ni, 
v^ieni a scald^arti con ^noi:
^tutti di^vide^re^mo 
^pane e ^vi^no!
\endverse



%%%%% STROFA
\beginverse		%Oppure \beginverse* se non si vuole il numero di fianco
%\memorize 		% <<< DECOMMENTA se si vuole utilizzarne la funzione
^Ti senti^rai più ^fo^rte, 
^vieni, ri^mani con ^noi:
^uniti at^tende^re^mo 
^ogni do^ma^ni!
\endverse



\endsong
%------------------------------------------------------------
%			FINE CANZONE
%------------------------------------------------------------
%-------------------------------------------------------------
%			INIZIO	CANZONE
%-------------------------------------------------------------


%titolo: 	Ecco l'uomo
%autore: 	P. Sequeri
%tonalita:  Mi- 



%%%%%% TITOLO E IMPOSTAZONI
\beginsong{Ecco l'uomo}[by={P. Sequeri}]  % <<< MODIFICA TITOLO E AUTORE
\transpose{0} 						% <<< TRASPOSIZIONE #TONI (0 nullo)
%\preferflats  %SE VOGLIO FORZARE i bemolle come alterazioni
%\prefersharps %SE VOGLIO FORZARE i # come alterazioni
\momenti{}							% <<< INSERISCI MOMENTI	
% momenti vanno separati da ; e vanno scelti tra:
% Ingresso; Atto penitenziale; Acclamazione al Vangelo; Dopo il Vangelo; Offertorio; Comunione; Ringraziamento; Fine; Santi; Pasqua; Avvento; Natale; Quaresima; Canti Mariani; Battesimo; Prima Comunione; Cresima; Matrimonio; Meditazione; Spezzare del pane;
\ifchorded
	%\textnote{Tonalità migliore }	% <<< EV COMMENTI (tonalità originale/migliore)
\fi

\beginverse
\[E-7]Nella memoria di \[A-7]questa Passione
\[D7]noi ti chiediamo per\[G]dono, Si\[B7]gnore
\[E-7]per ogni volta che \[A-7]abbiamo lasciato
\[D7]il tuo fratello so\ch{B}{f}{f}{ff}rire da \[B7]solo.
\endverse

\beginchorus
\textnote{\textbf{Rit.}}
\[E-]Noi ti pre\[A-]ghiamo \[D7]Uomo della \[G]Croce;
\[E-]Figlio e Fratel\[F#-dim]lo, \brk \[B7/9]noi speriamo in \[E-]te! \rep{2}
\endchorus

\beginverse
\chordsoff
Nella memoria di questa tua Morte
noi ti chiediamo coraggio, Signore
per ogni volta che il dono d'amore
ci chiederà di soffrire da soli.
\endverse

\beginverse
\chordsoff
Nella memoria dell'Ultima Cena
noi spezzeremo di nuovo il tuo Pane
ed ogni volta il tuo Corpo donato
sarà la nostra speranza di vita.
\endverse
\endsong


%-------------------------------------------------------------
%			INIZIO	CANZONE
%-------------------------------------------------------------


%titolo: 	Ecco quel che abbiamo
%autore: 	Gen Verde
%tonalita: 	Do



%%%%%% TITOLO E IMPOSTAZONI
\beginsong{Ecco quel che abbiamo}[by={Gen\ Verde}]	% <<< MODIFICA TITOLO E AUTORE
\transpose{0} 						% <<< TRASPOSIZIONE #TONI (0 nullo)
\momenti{Offertorio}							% <<< INSERISCI MOMENTI	
% momenti vanno separati da ; e vanno scelti tra:
% Ingresso; Atto penitenziale; Acclamazione al Vangelo; Dopo il Vangelo; Offertorio; Comunione; Ringraziamento; Fine; Santi; Pasqua; Avvento; Natale; Quaresima; Canti Mariani; Battesimo; Prima Comunione; Cresima; Matrimonio; Meditazione; Spezzare del pane;
\ifchorded
	%\textnote{Tonalità migliore }	% <<< EV COMMENTI (tonalità originale/migliore)
\fi




%%%%%% INTRODUZIONE
\ifchorded
\vspace*{\versesep}
\musicnote{
\begin{minipage}{0.48\textwidth}
\textbf{Intro}
\hfill 
%( \eighthnote \, 80)   % <<  MODIFICA IL TEMPO
% Metronomo: \eighthnote (ottavo) \quarternote (quarto) \halfnote (due quarti)
\end{minipage}
} 	
\vspace*{-\versesep}
\beginverse*

\nolyrics

%---- Prima riga -----------------------------
\vspace*{-\versesep}
\[C] \[E-] \[F] \[C]	 % \[*D] per indicare le pennate, \rep{2} le ripetizioni

%---- Ogni riga successiva -------------------
%\vspace*{-\versesep}
%\[G] \[C]  \[D]	

%---- Ev Indicazioni -------------------------			
%\textnote{\textit{(Oppure tutta la strofa)} }	

\endverse
\fi




%%%%% STROFA
\beginverse		%Oppure \beginverse* se non si vuole il numero di fianco
\memorize 		% <<< DECOMMENTA se si vuole utilizzarne la funzione
%\chordsoff		% <<< DECOMMENTA se vuoi una strofa senza accordi

\[C]Ecco quel che ab\[C7+/E-]biamo, 
 nulla \[F]ci appartiene, or\[C]mai.
Ecco i \[A-]frutti della \[E-]terra, che Tu \[F]moltipliche\[G]rai. \[G7]
\[C]Ecco queste \[C7+/E-]mani, puoi u\[F]sarle, se lo \[C]vuoi, 
per di\[A-]videre nel \[E-]mondo il pane 
 \[F*]che Tu hai \[G7*]dato a \[C]noi. \[C]
\endverse





%%%%% RITORNELLO
\beginchorus
\textnote{\textbf{Rit.}}
\[A-] Solo una goccia hai messo \[E-]fra le mani \[E-7]mie,
solo una goccia che tu \[A]ora chiedi a \[A7]me,
\textit{\[D-7](ah.....\[G]ah..... \[E-]ah.....\[A]ah)}
una \[D-7]goccia che in mano a \[F7]te,
una \[D-7]pioggia divente\[E-7]rà 
e la \[F]terra feconde\[C]rà.
\endchorus





%%%%% STROFA
\beginverse		%Oppure \beginverse* se non si vuole il numero di fianco
%\memorize 		% <<< DECOMMENTA se si vuole utilizzarne la funzione
%\chordsoff		% <<< DECOMMENTA se vuoi una strofa senza accordi
^Sulle strade, il ^vento 
da lon^tano porte^rà
il pro^fumo del fru^mento, che ^tutti avvolge^rà. ^
^E sarà l'a^more che il rac^colto sparti^rà
e il mi^racolo del ^pane 
in terra ^si ri^pete^rà. ^
\endverse


%%%%% RITORNELLO
\beginchorus
\textnote{\textbf{Rit.}}
\[A-] Le nostre gocce, pioggia \[E-]fra le mani \[E-7]tue,
saranno linfa di una \[A]nuova civil\[A7]tà
\textit{\[D-7](ah.....\[G]ah..... \[E-]ah.....\[A]ah)}
e la \[D-7]terra prepare\[F7]rà 
la \[D-7]festa del pane \[E-7]che 
ogni \[F]uomo condivide\[C]rà.
\endchorus




\endsong
%------------------------------------------------------------
%			FINE CANZONE
%------------------------------------------------------------





%titolo: 	Eccomi qui
%autore: 	Diliberto Monti
%tonalita: 	Do 



%%%%%% TITOLO E IMPOSTAZONI
\beginsong{Eccomi qui}[by={D. Monti}] 	% <<< MODIFICA TITOLO E AUTORE
\transpose{0} 						% <<< TRASPOSIZIONE #TONI (0 nullo)
\momenti{Ingresso; Dopo il Vangelo}							% <<< INSERISCI MOMENTI	
% momenti vanno separati da ; e vanno scelti tra:
% Ingresso; Atto penitenziale; Acclamazione al Vangelo; Dopo il Vangelo; Offertorio; Comunione; Ringraziamento; Fine; Santi; Pasqua; Avvento; Natale; Quaresima; Canti Mariani; Battesimo; Prima Comunione; Cresima; Matrimonio; Meditazione;
\ifchorded
	%\textnote{Tonalità originale }	% <<< EV COMMENTI (tonalità originale/migliore)
\fi




%%%%%% INTRODUZIONE
\ifchorded
\vspace*{\versesep}
\musicnote{
\begin{minipage}{0.48\textwidth}
\textbf{Intro}
\hfill 
%( \eighthnote \, 80)   % <<  MODIFICA IL TEMPO
% Metronomo: \eighthnote (ottavo) \quarternote (quarto) \halfnote (due quarti)
\end{minipage}
} 	
\vspace*{-\versesep}
\beginverse*
\nolyrics

%---- Prima riga -----------------------------
\vspace*{-\versesep}
\[C*] \[E7*] \[A-] 	 % \[*D] per indicare le pennate, \rep{2} le ripetizioni

%---- Ogni riga successiva -------------------
\vspace*{-\versesep}
\[F] \[G] \[G] 

%---- Ev Indicazioni -------------------------			
%\textnote{\textit{(Oppure tutta la strofa)} }	

\endverse
\fi


%%%%% RITORNELLO
\beginchorus
\textnote{\textbf{Rit.}}

\[C*] E\[E7*]ccomi \[A-]qui, \brk  \[F] di nuovo a \[G]te Signore, 
\[C*] e\[E7*]ccomi \[A-]qui: \brk  \[F] accetta \[G]la mia vita;
\[C] non dire \[G]no  \brk a \[F]chi si affi da a \[C]te,
\[A-] mi accoglie\[D-]rai \brk \[F] per sempre \[G]nel tuo amore.\[F]\[C]

\endchorus



%%%%% STROFA
\beginverse		%Oppure \beginverse* se non si vuole il numero di fianco
\memorize 		% <<< DECOMMENTA se si vuole utilizzarne la funzione
%\chordsoff		& <<< DECOMMENTA se vuoi una strofa senza accordi

Q\[C]uando hai scelto di \[G]vivere quag\[A-]giù, \[A-7]
q\[F]uando hai voluto che \[G]fossimo figli t\[C]uoi, \[C7]
\[F] ti sei do\[C]nato ad \[D-]una come \[A-]noi 
\[A-7] e hai cammi\[D-]nato sulle st\[F]rade dell'\[G]uomo.

\endverse

%%%%% STROFA
\beginverse		%Oppure \beginverse* se non si vuole il numero di fianco
%\memorize 		% <<< DECOMMENTA se si vuole utilizzarne la funzione
%\chordsoff		% <<< DECOMMENTA se vuoi una strofa senza accordi

P^rima che il Padre ti ^richiamasse a ^sé ^
p^rima del buio che il tuo g^rido spezze^rà ^
^ tu hai pro^messo di ^non lasciarci p^iù
^ di accompag^narci sulle st^rade del ^mondo.

\endverse

%%%%% STROFA
\beginverse		%Oppure \beginverse* se non si vuole il numero di fianco
%\memorize 		% <<< DECOMMENTA se si vuole utilizzarne la funzione
%\chordsoff		% <<< DECOMMENTA se vuoi una strofa senza accordi

^Ora ti prego con^ducimi con ^te ^
^nella fatica di ser^vir la veri^tà ^
^ sarò vi^cino a ^chi ti invoche^rà
^ e mi guide^rai sulle st^rade dell'^uomo.

\endverse

\endsong
%------------------------------------------------------------
%			FINE CANZONE
%------------------------------------------------------------

%-------------------------------------------------------------
%			INIZIO	CANZONE
%-------------------------------------------------------------


%titolo: 	Ed essi si ameranno
%autore: 	Meregalli
%tonalita: 	Re / Do -2 / La -5 / Sol -7 /



%%%%%% TITOLO E IMPOSTAZONI
\beginsong{Ed essi si ameranno}[by={Meregalli}]
\transpose{0} % <<< TRASPOSIZIONE #TONI + - (0 nullo)
%\preferflats %SE VOGLIO FORZARE i bemolle come alterazioni
%\prefersharps %SE VOGLIO FORZARE i # come alterazioni						
\momenti{Fine; Pasqua; Matrimonio; Santi; Comunione;}			% <<< INSERISCI MOMENTI	
% momenti vanno separati da ; e vanno scelti tra:
% Ingresso; Atto penitenziale; Acclamazione al Vangelo; Dopo il Vangelo; Offertorio; Comunione; Ringraziamento; Fine; Santi; Pasqua; Avvento; Natale; Quaresima; Canti Mariani; Battesimo; Prima Comunione; Cresima; Matrimonio; Meditazione;
\ifchorded
	%\textnote{Tonalità originale }	% <<< EV COMMENTI (tonalità originale/migliore)
\fi



%%%%%% INTRODUZIONE
\ifchorded
\vspace*{\versesep}
\textnote{Intro: \qquad \qquad  }%(\eighthnote 116) % << MODIFICA IL TEMPO
% Metronomo: \eighthnote (ottavo) \quarternote (quarto) \halfnote (due quarti)
\vspace*{-\versesep}
\beginverse*

\nolyrics

%---- Prima riga -----------------------------
\vspace*{-\versesep}
\[D] 

%---- Ogni riga successiva -------------------
%\vspace*{-\versesep}
%\[A-] \[E-] \[F] \[C] \[D-] \[A-] \[F] \[*G] \[A-] 

%---- Ev Indicazioni -------------------------			
%\textnote{\textit{(Oppure tutta la strofa)} }	

\endverse
\fi


%%%%% STROFA
\beginverse 	%Oppure \beginverse* se non si vuole il numero di fianco
\memorize 		% <<< DECOMMENTA se si vuole utilizzarne la funzione
%\chordsoff		% <<< DECOMMENTA se vuoi una strofa senza accordi

\[D]Vai a dire alla terra 
di sve\[F#]gliare dal sonno le genti;
dì alla \[D7]folgore, al tuono e alla \[B7]voce 
di inon\[E-]dare di luce la \[A]notte;
dì alle \[E-]nuvole bianche del \[A]cielo 
di var\[D]care la soglia del \[A4]tem\[A]po.

\endverse

\beginverse*	%Oppure \beginverse* se non si vuole il numero di fianco
%\memorize 		% <<< DECOMMENTA se si vuole utilizzarne la funzione
%\chordsoff		% <<< DECOMMENTA se vuoi una strofa senza accordi

^Vai a dire alla terra 
di tre^mare al passo tonante
dei messa^ggeri di pace; pro^clama 
la mia ^legge d’amore alle ^genti;
dì che i ^vecchi delitti ho scor^dati... 
e tra ^voi non sia odio né ^guer^ra.

\endverse





%%%%% RITORNELLO
\beginchorus
\textnote{\textbf{Rit.}}
\[G]È finito questo vecchio mondo:
il \[F]cielo antico è lace\[G]rato.  \rep{2}
\endchorus

\beginchorus
\[F] Il mio popolo \[C]si radune\[G]rà. 
\[F] Il mio popolo \[C]si radune\[G]rà.
\[F] Il mio popolo \[C]si radune\[A]rà. \[A]
\endchorus








%%%%% STROFA
\beginverse 	%Oppure \beginverse* se non si vuole il numero di fianco
%\memorize 		% <<< DECOMMENTA se si vuole utilizzarne la funzione
%\chordsoff		% <<< DECOMMENTA se vuoi una strofa senza accordi


^Vai a dire alla terra 
che il Si^gnore l’ha amata da sempre,
che il suo ^servo reietto da ^molti 
si è addos^sato il peccato di ^tutti
e innal^zato ha patito la ^croce 
e, se^polto, ha rivisto la ^luce^.

\endverse

\beginverse*	%Oppure \beginverse* se non si vuole il numero di fianco
%\memorize 		% <<< DECOMMENTA se si vuole utilizzarne la funzione
%\chordsoff		% <<< DECOMMENTA se vuoi una strofa senza accordi

^Vai a dire alle genti 
di invi^tare i fratelli alla mensa;
ogni ^popolo che è sulla ^terra 
la mia ^legge proclami ed os^servi;
messag^geri di pace so^lerti 
testi^moni di pace fe^de^li.

\endverse








%%%%% RITORNELLO
\beginchorus
\textnote{\textbf{Rit.}}
\[G]Ecco che nasce il nuovo mondo:
il \[F]vecchio è termi\[G]nato.   \rep{2}
\endchorus

\beginchorus
\[F] Il mio popolo \[C]si radune\[G]rà. 
\[F] Il mio popolo \[C]si radune\[G]rà.
\[F] Il mio popolo \[C]si radune\[A]rà. \[A]
\endchorus





%%%%%% EV. FINALE
\beginchorus
\textnote{\textbf{Finale.}}
\textnote{\textit{Cambia il tempo, crescendo ad ogni ripetizione.}}
\[D]Ed il giorno an\[*A]co\[*G]ra \[*D]è spun\[*G]tato \[*A]nuo\[*D]vo,
\[D]uomini di \[*A]pa\[*G]ce ed \[*D]essi \[*G]si ame\[*A]ran\[*D]no.   \rep{3}
\endchorus



\endsong
%------------------------------------------------------------
%			FINE CANZONE
%------------------------------------------------------------
%-------------------------------------------------------------
%			INIZIO	CANZONE
%-------------------------------------------------------------


%titolo: 	Emmanuel
%autore: 	M. Mammoli
%tonalita: 	Mi 



%%%%%% TITOLO E IMPOSTAZONI
\beginsong{Emmanuel }[by={Inno Giubileo 2000 — M. Mammoli}] 	% <<< MODIFICA TITOLO E AUTORE
\transpose{-2} 						% <<< TRASPOSIZIONE #TONI (0 nullo)
\momenti{Comunione; Fine}							% <<< INSERISCI MOMENTI	
% momenti vanno separati da ; e vanno scelti tra:
% Ingresso; Atto penitenziale; Acclamazione al Vangelo; Dopo il Vangelo; Offertorio; Comunione; Ringraziamento; Fine; Santi; Pasqua; Avvento; Natale; Quaresima; Canti Mariani; Battesimo; Prima Comunione; Cresima; Matrimonio; Meditazione; Spezzare del pane;
\ifchorded
	\textnote{$\bigstar$ Tonalità migliore }	% <<< EV COMMENTI (tonalità originale/migliore)
\fi




%%%%%% INTRODUZIONE
\ifchorded
\vspace*{\versesep}
\musicnote{
\begin{minipage}{0.48\textwidth}
\textbf{Intro}
\hfill 
%( \eighthnote \, 80)   % <<  MODIFICA IL TEMPO
% Metronomo: \eighthnote (ottavo) \quarternote (quarto) \halfnote (due quarti)
\end{minipage}
} 	
\vspace*{-\versesep}
\beginverse*

\nolyrics

%---- Prima riga -----------------------------
\vspace*{-\versesep}
\[E] \[B] \[E]	\[B] % \[*D] per indicare le pennate, \rep{2} le ripetizioni

%---- Ogni riga successiva -------------------
%\vspace*{-\versesep}
%\[G] \[C]  \[D]	

%---- Ev Indicazioni -------------------------			
%\textnote{\textit{(Oppure tutta la strofa)} }	

\endverse
\fi




%%%%% STROFA
\beginverse		%Oppure \beginverse* se non si vuole il numero di fianco
\memorize 		% <<< DECOMMENTA se si vuole utilizzarne la funzione
%\chordsoff		% <<< DECOMMENTA se vuoi una strofa senza accordi
\[E]Dall'orizzonte una grande luce
v\[B]iaggia nella storia
e \[A]lungo gli anni ha vinto il buio
fa\[B]cendosi Memoria,
e il\[E]luminando la nostra vita
ch\[B]iaro ci rivela
che \[A]non si vive se non si cerca
\[F#-]la Veri\[B]tà...
\endverse

\beginverse*	%Oppure \beginverse* se non si vuole il numero di fianco
...a-a-\[E]ah ...a-a-\[B]ah.. \quad \[A] \quad \[B]
\endverse

%%%%% STROFA
\beginverse*		%Oppure \beginverse* se non si vuole il numero di fianco
%\memorize 		% <<< DECOMMENTA se si vuole utilizzarne la funzione
%\chordsoff		% <<< DECOMMENTA se vuoi una strofa senza accordi
Da ^mille strade arriviamo a Roma
sui ^passi della fede,
senti^amo l'eco della Parola
^che risuona ancora
da ^queste mura, da questo cielo
^per il mondo intero:
è v^ivo oggi, è l'Uomo Vero
^Cristo tra ^noi...
\endverse


%%%%% RITORNELLO
\beginchorus
\textnote{\textbf{Rit.}}

Siamo \[G#-]qui
\[A] sotto la stessa luce
\[F#-] sotto la sua croce
\[D] cantando ad \[B]una voce.
\[E]È l'Emmanu\[B]el,
l'Emmanu\[A]el, Emmanu\[B]el.
\echo{Cantando ad una voce}
\[C#-]È l'Emmanu\[B]el, Emmanu\[A]el. \quad\[B]

\endchorus



%%%%% STROFA
\beginverse		%Oppure \beginverse* se non si vuole il numero di fianco
%\memorize 		% <<< DECOMMENTA se si vuole utilizzarne la funzione
\chordsoff		% <<< DECOMMENTA se vuoi una strofa senza accordi

Dalla città di chi ha versato
il sangue per amore
ed ha cambiato il vecchio mondo
vogliamo ripartire.
Seguendo Cristo, insieme a Pietro,
rinasce in noi la fede,
Parola viva che ci rinnova
e cresce in noi.

\endverse

%%%%% STROFA
\beginverse		%Oppure \beginverse* se non si vuole il numero di fianco
%\memorize 		% <<< DECOMMENTA se si vuole utilizzarne la funzione
\chordsoff		% <<< DECOMMENTA se vuoi una strofa senza accordi
Un grande dono che Dio ci ha fatto
è il Cristo suo Figlio,
l'umanità è rinnovata,
è in lui salvata.
È vero uomo, è vero Dio,
è il Pane della Vita,
che ad ogni uomo ai suoi fratelli
ridonerà.
\endverse

%%%%% STROFA
\beginverse		%Oppure \beginverse* se non si vuole il numero di fianco
%\memorize 		% <<< DECOMMENTA se si vuole utilizzarne la funzione
\chordsoff		% <<< DECOMMENTA se vuoi una strofa senza accordi
La morte è uccisa,
la vita ha vinto,
è Pasqua in tutto il mondo,
un vento soffia in ogni uomo
lo Spirito fecondo,
che porta avanti nella storia
la Chiesa sua sposa,
sotto lo sguardo di Maria,
comunità.
\endverse

%%%%% STROFA
\beginverse		%Oppure \beginverse* se non si vuole il numero di fianco
%\memorize 		% <<< DECOMMENTA se si vuole utilizzarne la funzione
\chordsoff		% <<< DECOMMENTA se vuoi una strofa senza accordi
Noi debitori del passato
di secoli di storia,
di vite date per amore,.
di santi che han creduto,
di uomini che ad alta quota
insegnano a volare,
di chi la storia sa cambiare,
come Gesù. 
\endverse

%%%%% STROFA
\beginverse		%Oppure \beginverse* se non si vuole il numero di fianco
%\memorize 		% <<< DECOMMENTA se si vuole utilizzarne la funzione
\chordsoff		% <<< DECOMMENTA se vuoi una strofa senza accordi
È giunta un'era di primavera,
è tempo di cambiare.
È oggi il tempo sempre nuovo
per ricominciare, per dare svolte, parole nuove
e convertire il cuore,
per dire al mondo, ad ogni uomo:
Signore Gesù. 
\endverse



%%%%% RITORNELLO
\ifchorded
\beginchorus
\textnote{\textbf{Rit.}}

Siamo \[G#-]qui
\[A] sotto la stessa luce
\[F#-] sotto la sua croce
\[D] cantando ad \[B]una voce.
\[E]È l'Emmanu\[B]el,
l'Emmanu\[A]el, Emmanu\[B]el.
\echo{Cantando ad una voce}
\[C#-]È l'Emmanu\[B]el, Emmanu\[A]el. \quad  \[B] \quad \[E*]

\endchorus
\fi


\endsong
%------------------------------------------------------------
%			FINE CANZONE
%------------------------------------------------------------


%FFF
%-------------------------------------------------------------
%			INIZIO	CANZONE
%-------------------------------------------------------------


%titolo: 	Santo Ricci
%autore: 	Daniele Ricci
%tonalita: 	Sol 



%%%%%% TITOLO E IMPOSTAZONI
\beginsong{Fammi conoscere}[by={P. Ruaro}] 	% <<< MODIFICA TITOLO E AUTORE
\transpose{-2} 						% <<< TRASPOSIZIONE #TONI (0 nullo)
\momenti{Ingresso}							% <<< INSERISCI MOMENTI	
% momenti vanno separati da ; e vanno scelti tra:
% Ingresso; Atto penitenziale; Acclamazione al Vangelo; Dopo il Vangelo; Offertorio; Comunione; Ringraziamento; Fine; Santi; Pasqua; Avvento; Natale; Quaresima; Canti Mariani; Battesimo; Prima Comunione; Cresima; Matrimonio; Meditazione; Spezzare del pane;
\ifchorded
	\textnote{$\bigstar$ Tonalità migliore }	% <<< EV COMMENTI (tonalità originale/migliore)
\fi


%%%%%% INTRODUZIONE
\ifchorded
\vspace*{\versesep}
\musicnote{
\begin{minipage}{0.48\textwidth}
\textbf{Intro}
\hfill 
%( \eighthnote \, 80)   % <<  MODIFICA IL TEMPO
% Metronomo: \eighthnote (ottavo) \quarternote (quarto) \halfnote (due quarti)
\end{minipage}
} 	
\vspace*{-\versesep}
\beginverse*


\nolyrics

%---- Prima riga -----------------------------
\vspace*{-\versesep}
\[E] \[F#-] \[A] \[E] % \[*D] per indicare le pennate, \rep{2} le ripetizioni

%---- Ogni riga successiva -------------------
%\vspace*{-\versesep}
%\[G] \[C]  \[D]	

%---- Ev Indicazioni -------------------------			
%\textnote{\textit{(Oppure tutta la strofa)} }	

\endverse
\fi




%%%%% RITORNELLO
\beginchorus
\textnote{\textbf{Rit.}}

\[E]Fammi co\[F#-]noscere la \[Gdim/(A)]tua volon\[G#-]tà,
\[A]parla, Ti as\[E]colto, Si\[F#-]gno\[B]re !
\[A] La mia fe\[B]licità è \[G#]fare il tuo vo\[C#-7]lere:
\[A*]porte\[B7*]rò con \[E*]me la \[A*]tua Pa\[B7]ro\[E]la. \[E*] \quad \[D*]


\endchorus


\preferflats
%%%%% STROFA
\beginverse		%Oppure \beginverse* se non si vuole il numero di fianco
\memorize 		% <<< DECOMMENTA se si vuole utilizzarne la funzione
%\chordsoff		% <<< DECOMMENTA se vuoi una strofa senza accordi
\vspace*{1.3\versesep}
\musicnote{ \textit{(dolce, arpeggiato)}} %<<< EV. INDICAZIONI
\[G]Lampada ai miei \[D]passi 
\[C]è la tua Pa\[D]rola
\[E-]luce sul \[C]mio cam\[G]mi\[D]no.
\[G]Ogni gi\[D]orno \[C]la mia volon\[D]tà
\[C]trova una guida in \[G]te. \[B]\[7]

\endverse







%%%%% STROFA
\beginverse		%Oppure \beginverse* se non si vuole il numero di fianco
%\memorize 		% <<< DECOMMENTA se si vuole utilizzarne la funzione
%\chordsoff		% <<< DECOMMENTA se vuoi una strofa senza accordi

^Porterò con ^me 
i ^tuoi insegna^menti:
^danno al mio ^cuore gi^o^ia!
^La tua Pa^rola è f^onte di ^luce:
^dona saggezza ai ^sempli^ci. ^

\endverse





%%%%% STROFA
\beginverse		%Oppure \beginverse* se non si vuole il numero di fianco
%\memorize 		% <<< DECOMMENTA se si vuole utilizzarne la funzione
%\chordsoff		% <<< DECOMMENTA se vuoi una strofa senza accordi

^La mia ^bocca 
im^pari la tua ^lode
^sempre ti ^renda gr^az^ie.
^Ogni mo^mento ^canti la tua ^lode
^la mia speranza è in ^te. ^ ^ 

\endverse









\endsong
%------------------------------------------------------------
%			FINE CANZONE
%------------------------------------------------------------

%++++++++++++++++++++++++++++++++++++++++++++++++++++++++++++
%			CANZONE TRASPOSTA
%++++++++++++++++++++++++++++++++++++++++++++++++++++++++++++
\ifchorded
%decremento contatore per avere stesso numero
\addtocounter{songnum}{-1} 
\beginsong{Fammi conoscere}[by={P. Ruaro}] 	% <<< MODIFICA TITOLO E AUTORE
\transpose{0} 						% <<< TRASPOSIZIONE #TONI + - (0 nullo)
%\preferflats  %SE VOGLIO FORZARE i bemolle come alterazioni
%\prefersharps %SE VOGLIO FORZARE i # come alterazioni
\ifchorded
	\textnote{$\lozenge$ Tonalità originale}	% <<< EV COMMENTI (tonalità originale/migliore)
\fi



%%%%%% INTRODUZIONE
\ifchorded
\vspace*{\versesep}
\musicnote{
\begin{minipage}{0.48\textwidth}
\textbf{Intro}
\hfill 
%( \eighthnote \, 80)   % <<  MODIFICA IL TEMPO
% Metronomo: \eighthnote (ottavo) \quarternote (quarto) \halfnote (due quarti)
\end{minipage}
} 	
\vspace*{-\versesep}
\beginverse*


\nolyrics

%---- Prima riga -----------------------------
\vspace*{-\versesep}
\[E] \[F#-] \[A] \[E] % \[*D] per indicare le pennate, \rep{2} le ripetizioni

%---- Ogni riga successiva -------------------
%\vspace*{-\versesep}
%\[G] \[C]  \[D]	

%---- Ev Indicazioni -------------------------			
%\textnote{\textit{(Oppure tutta la strofa)} }	

\endverse
\fi




%%%%% RITORNELLO
\beginchorus
\textnote{\textbf{Rit.}}

\[E]Fammi co\[F#-]noscere la \[Gdim/(A)]tua volon\[G#-]tà,
\[A]parla, Ti as\[E]colto, Si\[F#-]gno\[B]re !
\[A] La mia fe\[B]licità è \[G#]fare il tuo vo\[C#-7]lere:
\[A*]porte\[B7*]rò con \[E*]me la \[A*]tua Pa\[B7]ro\[E]la. \[E*] \quad \[D*]


\endchorus


%%%%% STROFA
\beginverse		%Oppure \beginverse* se non si vuole il numero di fianco
\memorize 		% <<< DECOMMENTA se si vuole utilizzarne la funzione
%\chordsoff		% <<< DECOMMENTA se vuoi una strofa senza accordi
\vspace*{1.3\versesep}
\musicnote{ \textit{[dolce, arpeggiato]}} %<<< EV. INDICAZIONI
\[G]Lampada ai miei \[D]passi 
\[C]è la tua Pa\[D]rola
\[E-]luce sul \[C]mio cam\[G]mi\[D]no.
\[G]Ogni gi\[D]orno \[C]la mia volon\[D]tà
\[C]trova una guida in \[G]te. \[B]\[7]

\endverse








%%%%% STROFA
\beginverse		%Oppure \beginverse* se non si vuole il numero di fianco
%\memorize 		% <<< DECOMMENTA se si vuole utilizzarne la funzione
%\chordsoff		% <<< DECOMMENTA se vuoi una strofa senza accordi

^Porterò con ^me 
i ^tuoi insegna^menti:
^danno al mio ^cuore gi^o^ia!
^La tua Pa^rola è f^onte di ^luce:
^dona saggezza ai ^sempli^ci. ^

\endverse





%%%%% STROFA
\beginverse		%Oppure \beginverse* se non si vuole il numero di fianco
%\memorize 		% <<< DECOMMENTA se si vuole utilizzarne la funzione
%\chordsoff		% <<< DECOMMENTA se vuoi una strofa senza accordi

^La mia ^bocca 
im^pari la tua ^lode
^sempre ti ^renda gr^az^ie.
^Ogni mo^mento ^canti la tua ^lode
^la mia speranza è in ^te. ^ ^ 

\endverse









\endsong
\fi
%++++++++++++++++++++++++++++++++++++++++++++++++++++++++++++
%			FINE CANZONE TRASPOSTA
%++++++++++++++++++++++++++++++++++++++++++++++++++++++++++++


%-------------------------------------------------------------
%			INIZIO	CANZONE
%-------------------------------------------------------------


%titolo: 	Fate questo in memoria di me
%autore: 	Burgio
%tonalita: 	Do



%%%%%% TITOLO E IMPOSTAZONI
\beginsong{Fate questo in memoria di me}[by={Burgio}] 	% <<< MODIFICA TITOLO E AUTORE
\transpose{0} 						% <<< TRASPOSIZIONE #TONI (0 nullo)
\momenti{Offertorio}							% <<< INSERISCI MOMENTI	
% momenti vanno separati da ; e vanno scelti tra:
% Ingresso; Atto penitenziale; Acclamazione al Vangelo; Dopo il Vangelo; Offertorio; Comunione; Ringraziamento; Fine; Santi; Pasqua; Avvento; Natale; Quaresima; Canti Mariani; Battesimo; Prima Comunione; Cresima; Matrimonio; Meditazione; Spezzare del pane;
\ifchorded
	%\textnote{Tonalità migliore }	% <<< EV COMMENTI (tonalità originale/migliore)
\fi


%%%%%% INTRODUZIONE
\ifchorded
\vspace*{\versesep}
\textnote{Intro: \qquad \qquad  }%(\eighthnote 116) % <<  MODIFICA IL TEMPO
% Metronomo: \eighthnote (ottavo) \quarternote (quarto) \halfnote (due quarti)
\vspace*{-\versesep}
\beginverse*

\nolyrics

%---- Prima riga -----------------------------
\vspace*{-\versesep}
\[C]\[G] \[F] \[C] % \[*D] per indicare le pennate, \rep{2} le ripetizioni

%---- Ogni riga successiva -------------------
%\vspace*{-\versesep}
%\[F] \[G]  \[C]	

%---- Ev Indicazioni -------------------------			
%\textnote{\textit{(Oppure tutta la strofa)} }	

\endverse
\fi



%%%%% STROFA
\beginverse		%Oppure \beginverse* se non si vuole il numero di fianco
\memorize 		% <<< DECOMMENTA se si vuole utilizzarne la funzione
\[C]Quando nell’\[G]ultima \[F]cena, Si\[E-]gnore,
\[F] spezzando il \[E-]pane \[F*] ti \[E*]desti a \[A-]noi,
\[G-7]ecco ap\[C7]rimmo i nostri \[F4]occhi, \[E]
ve\[A-]demmo il \[D-]Tuo immenso A\[C]more,
cre\[A]demmo alla tua \[D-]voce che di\[G*]ce\[F*]va: \[G]
\endverse


%%%%% RITORNELLO
\beginchorus
\textnote{\textbf{Rit.}}
Questo è il \[C]corpo \[F*] che è \[G*]dato per \[C]voi
questo \[C]calice  \[F] è la nuova alle\[G]anza \[E]
nel mio \[A-]sangue \[D-] ch’è versato \brk per \[C*]v\[E*]o\[A-]i
fate \[D-*]que\[C*]sto \[F*] in me\[G*]moria di \[F]me. \[C]
\endchorus




%%%%% STROFA
\beginverse	%Oppure \beginverse* se non si vuole il numero di fianco
%\memorize 		% <<< DECOMMENTA se si vuole utilizzarne la funzione
^Quando nell’^ultima ^cena, Si^gnore,
^ versando il ^vino, ^ ti ^desti a ^noi,
^ecco sve^lasti il gran Mis^tero, ^
il ^dono di un’^Alleanza ^nuova,
per ^sempre stabi^lita con ^n^o^i

\endverse



%%%%% STROFA
\beginverse		%Oppure \beginverse* se non si vuole il numero di fianco
%\memorize 		% <<< DECOMMENTA se si vuole utilizzarne la funzione
^Ora anche ^noi, Tuoi ^figli a^mati,
^ saremo ^dono ^ per ^ogni u^omo,
^prendici e ^guida i nostri ^passi, ^
do^vunque il tuo ^Spirito ci ^porti,
sa^remo la tua ^voce che ^di^ce:^
\endverse


%%%%% RITORNELLO
\beginchorus
\vspace*{1.3\versesep}
\textnote{Finale }
Non te\[C]mete \[F*] sarò \[G*]sempre con \[C]voi
e por\[C]tate \[F] il Vangelo nel \[G]mondo \[E]
ogni \[A-]uomo \[D-] riconosca il \brk mio a\[C*]m\[E*]o\[A-]re
fate \[D-*]que\[C*]sto \[F*] in me\[G*]moria di \[F]me. \[C*]
\endchorus





\endsong
%------------------------------------------------------------
%			FINE CANZONE
%------------------------------------------------------------
%-------------------------------------------------------------
%			INIZIO	CANZONE
%-------------------------------------------------------------


%titolo: 	Francesco vai
%autore: 	M. C. Bizzeti
%tonalita: 	Mi-



%%%%%% TITOLO E IMPOSTAZONI
\beginsong{Francesco vai}[by={M. C. Bizzeti}]
\transpose{0} 						% <<< TRASPOSIZIONE #TONI (0 nullo)
%\preferflats  %SE VOGLIO FORZARE i bemolle come alterazioni
%\prefersharps %SE VOGLIO FORZARE i # come alterazioni
\momenti{}							% <<< INSERISCI MOMENTI	
% momenti vanno separati da ; e vanno scelti tra:
% Ingresso; Atto penitenziale; Acclamazione al Vangelo; Dopo il Vangelo; Offertorio; Comunione; Ringraziamento; Fine; Santi; Pasqua; Avvento; Natale; Quaresima; Canti Mariani; Battesimo; Prima Comunione; Cresima; Matrimonio; Meditazione; Spezzare del pane;
\ifchorded
	%\textnote{Tonalità migliore }	% <<< EV COMMENTI (tonalità originale/migliore)
\fi




%%%%%% INTRODUZIONE
\ifchorded
\vspace*{\versesep}
\textnote{Intro: \qquad \qquad  }%(\eighthnote 116) % <<  MODIFICA IL TEMPO
% Metronomo: \eighthnote (ottavo) \quarternote (quarto) \halfnote (due quarti)
\vspace*{-\versesep}
\beginverse*

\nolyrics

%---- Prima riga -----------------------------
\vspace*{-\versesep}
\[E-] \[D] \[E-]	 % \[*D] per indicare le pennate, \rep{2} le ripetizioni

%---- Ogni riga successiva -------------------
%\vspace*{-\versesep}
%\[G] \[C]  \[D]	

%---- Ev Indicazioni -------------------------			
%\textnote{\textit{(Oppure tutta la strofa)} }	

\endverse
\fi





%%%%% STROFA
\beginverse		%Oppure \beginverse* se non si vuole il numero di fianco
\memorize 		% <<< DECOMMENTA se si vuole utilizzarne la funzione
%\chordsoff		% <<< DECOMMENTA se vuoi una strofa senza accordi
\[E-]Quello che io vivo non mi \[D]basta \[E-]più,
tutto quel che avevo non mi \[D]serve \[E-]più:
io cerche\[B]rò quello che dav\[E-]vero vale,
e non più il \[A-]servo, ma il pa\[C]drone segui\[B]rò!
\endverse




%%%%% RITORNELLO
\beginchorus
\textnote{\textbf{Rit.}}
Francesco, \[E-]vai, ri\[D]para la mia \[E-]casa!
Fran\[D]cesco, \[E-]vai, non \[D]vedi che è in ro\[G]vina?
E non te\[A-]mere: \[C]io sarò con \[G*]te do\[B*]vunque an\[E-]drai.
\[D]Francesco, \[E-]vai!
\endchorus



%%%%% STROFA
\beginverse		%Oppure \beginverse* se non si vuole il numero di fianco
%\memorize 		% <<< DECOMMENTA se si vuole utilizzarne la funzione
%\chordsoff		% <<< DECOMMENTA se vuoi una strofa senza accordi
Nel ^buio e nel silenzio ti ho cer^cato, ^Dio;
dal fondo della notte ho alzato il ^grido ^mio
e gride^rò finché non a^vrò risposta
per co^noscere la ^tua volon^tà.
\endverse


%%%%% STROFA
\beginverse		%Oppure \beginverse* se non si vuole il numero di fianco
%\memorize 		% <<< DECOMMENTA se si vuole utilizzarne la funzione
\chordsoff		% <<< DECOMMENTA se vuoi una strofa senza accordi
Al^tissimo Signore, cosa ^vuoi da ^me?
Tutto quel che avevo l'ho do^nato a ^te.
Ti segui^rò nella gioia e ^nel dolore
e della ^vita mia una ^lode a te fa^rò.
\endverse

%%%%% STROFA
\beginverse		%Oppure \beginverse* se non si vuole il numero di fianco
%\memorize 		% <<< DECOMMENTA se si vuole utilizzarne la funzione
\chordsoff		% <<< DECOMMENTA se vuoi una strofa senza accordi
^Quello che cercavo l'ho tro^vato ^qui:
ora ho riscoperto nel mio ^dirti ^sì
la liber^tà di essere ^figlio tuo,
fratello e ^sposo di Ma^donna pover^tà.
\endverse



\endsong
%------------------------------------------------------------
%			FINE CANZONE
%------------------------------------------------------------



%-------------------------------------------------------------
%			INIZIO	CANZONE
%-------------------------------------------------------------


%titolo: 	Fratello Sole, sorella Luna
%autore: 	J. M. Benjamin, R. Ortolani
%tonalita: 	Do



%%%%%% TITOLO E IMPOSTAZONI
\beginsong{Fratello Sole, sorella Luna}[by={J. M. Benjamin, R. Ortolani}] 	% <<< MODIFICA TITOLO E AUTORE
\transpose{0} 						% <<< TRASPOSIZIONE #TONI (0 nullo)
\momenti{Comunione; Ringraziamento; Meditazione}							% <<< INSERISCI MOMENTI	
% momenti vanno separati da ; e vanno scelti tra:
% Ingresso; Atto penitenziale; Acclamazione al Vangelo; Dopo il Vangelo; Offertorio; Comunione; Ringraziamento; Fine; Santi; Pasqua; Avvento; Natale; Quaresima; Canti Mariani; Battesimo; Prima Comunione; Cresima; Matrimonio; Meditazione; Spezzare del pane;
\ifchorded
	%\textnote{Tonalità migliore }	% <<< EV COMMENTI (tonalità originale/migliore)
\fi



%%%%%% INTRODUZIONE
\ifchorded
\vspace*{\versesep}
\musicnote{
\begin{minipage}{0.48\textwidth}
\textbf{Intro}
\hfill 
%( \eighthnote \, 80)   % <<  MODIFICA IL TEMPO
% Metronomo: \eighthnote (ottavo) \quarternote (quarto) \halfnote (due quarti)
\end{minipage}
} 	
\vspace*{-\versesep}
\beginverse*


\nolyrics

%---- Prima riga -----------------------------
\vspace*{-\versesep}
\[C]\[A-*] \[F*] \[E-] % \[*D] per indicare le pennate, \rep{2} le ripetizioni

%---- Ogni riga successiva -------------------
\vspace*{-\versesep}
\[F] \[G]  \[C]	

%---- Ev Indicazioni -------------------------			
%\textnote{\textit{(Oppure tutta la strofa)} }	

\endverse
\fi



%%%%% STROFA
\beginverse*		%Oppure \beginverse* se non si vuole il numero di fianco
\memorize 		% <<< DECOMMENTA se si vuole utilizzarne la funzione
\[C]Dol\[A-*]ce \[F*]sen\[E-]tire \brk \[F]come \[G]nel mio \[C]cuore,
\[A-]o\[E-*]ra u\[F*]mil\[E-]men\[A-]te, \brk \[D-]sta nas\[7]cendo a\[G]more.
\endverse


%%%%% STROFA
\beginverse*		%Oppure \beginverse* se non si vuole il numero di fianco
%\memorize 		% <<< DECOMMENTA se si vuole utilizzarne la funzione
%\vspace*{-\versesep}
^Dol^ce ^ca^pire \brk ^che non ^son piu’ ^solo
^ma ^che ^son ^par^te \brk ^di una im\[G]mensa \[C]vita,
\endverse





%%%%% STROFA
\beginverse*	%Oppure \beginverse* se non si vuole il numero di fianco
%\memorize 		% <<< DECOMMENTA se si vuole utilizzarne la funzione
\[A-] che \[F-]gene\[C]rosa \brk ris\[F6]plende in\[G9]torno a \[C]me:
\[A-] do\[F-]no di \[C]Lui \brk del \[F]suo im\[F6/G]menso a\[C]more.

\endverse



%%%%% STROFA
\beginverse*		%Oppure \beginverse* se non si vuole il numero di fianco
%\memorize 		% <<< DECOMMENTA se si vuole utilizzarne la funzione
^Ci ha ^da^to il ^cielo \brk ^e le ^chiare ^stelle
^fra^tel^lo ^so^le \brk ^e so^rella ^luna;
\endverse


%%%%% STROFA
\beginverse*		%Oppure \beginverse* se non si vuole il numero di fianco
%\memorize 		% <<< DECOMMENTA se si vuole utilizzarne la funzione
%\vspace*{-\versesep}
^la ^ma^dre ^terra \brk con ^frutti, ^prati e ^fiori
^il ^fuo^co, il ^ven^to, \brk ^l’aria e ^l’acqua ^pura
\[A-]fon\[E-*]te \[F*]di \[E-]vi\[A-]ta, \brk \[D-]per le \[G]sue crea\[C]ture
\endverse



%%%%% STROFA
\beginverse*		%Oppure \beginverse* se non si vuole il numero di fianco
%\memorize 		% <<< DECOMMENTA se si vuole utilizzarne la funzione
\[A-] do\[F-]no di \[C]Lui \brk del \[F6]suo im\[G9]menso a\[C]more
\[A-] do\[F-]no di \[C]Lui \brk del \[F]suo im\[F6/G]menso a\[C]more.
\endverse



\endsong
%------------------------------------------------------------
%			FINE CANZONE
%------------------------------------------------------------
%titolo{Frutto della nostra terra}
%autore{Buttazzo}
%-------------------------------------------------------------
%			INIZIO	CANZONE
%-------------------------------------------------------------


%titolo: 	Frutto della nostra terra
%autore: 	Buttazzo
%tonalita: 	Sol 

%%%%%% TITOLO E IMPOSTAZONI
\beginsong{Frutto della nostra terra}[by={F. Buttazzo}]
\transpose{0} 						% <<< TRASPOSIZIONE #TONI (0 nullo)
\momenti{Offertorio}							% <<< INSERISCI MOMENTI	
% momenti vanno separati da ; e vanno scelti tra:
% Ingresso; Atto penitenziale; Acclamazione al Vangelo; Dopo il Vangelo; Offertorio; Comunione; Ringraziamento; Fine; Santi; Pasqua; Avvento; Natale; Quaresima; Canti Mariani; Battesimo; Prima Comunione; Cresima; Matrimonio; Meditazione;
\ifchorded
	%\textnote{Tonalità originale }	% <<< EV COMMENTI (tonalità originale/migliore)
\fi





%%%%%% INTRODUZIONE
\ifchorded
\vspace*{\versesep}
\textnote{Intro: \qquad \qquad  }%(\eighthnote 116) % << MODIFICA IL TEMPO
% Metronomo: \eighthnote (ottavo) \quarternote (quarto) \halfnote (due quarti)
\vspace*{-\versesep}
\beginverse*

\nolyrics

%---- Prima riga -----------------------------
\vspace*{-\versesep}
\[G] \[D]  \[C]	 \[D] \[D]% \[*D] per indicare le pennate, \rep{2} le ripetizioni

%---- Ogni riga successiva -------------------
%\vspace*{-\versesep}
%\[G] \[C]  \[D]	

%---- Ev Indicazioni -------------------------			
\textnote{\textit{(melodia con pianoforte o clarinetto) }}	

\endverse
\fi



%%%%% STROFA
\beginverse		%Oppure \beginverse* se non si vuole il numero di fianco
\memorize 		% <<< DECOMMENTA se si vuole utilizzarne la funzione
%\chordsoff		& <<< DECOMMENTA se vuoi una strofa senza accordi
\[G]Frutto della nostra \[C]terra 
\[G]del lavoro di ogni \[D]uomo
\[E-]pane della nostra \[B-]vita, 
cibo \[C]della quotidiani\[D]tà. \[D]
\[G]Tu che lo prendevi un \[C]giorno, 
 \[G]lo spezzavi per i \[D]tuoi
\[E-]oggi vieni in questo \[B-]pane, 
 cibo \[C]vero dell'umani\[D]tà.
\endverse




%%%%% RITORNELLO
\beginchorus
\textnote{\textbf{Rit.}}
E sarò \[G]pane, e sarò \[D]vino,
nella mia \[E-]vita, nelle tue \[B-]mani.
Ti accoglie\[C]rò dentro di \[D]me,
farò di \[E-]me un'offerta \[C]viva,
un sacri\[A-]ficio \[D] gradito a \[G]Te.
\endchorus





%%%%%% EV. INTERMEZZO
\beginverse*
\vspace*{1.3\versesep}
{
	\nolyrics
	\textnote{Intermezzo strumentale}
	
	\ifchorded

	%---- Prima riga -----------------------------
	\vspace*{-\versesep}
	\[C]  \[D]	 \[C]  

	\fi
	%---- Ev Indicazioni -------------------------			
	%\textnote{\textit{(ripetizione della strofa)}} 
	 
}
\vspace*{\versesep}
\endverse



%%%%% STROFA
\beginverse		%Oppure \beginverse* se non si vuole il numero di fianco
%\memorize 		% <<< DECOMMENTA se si vuole utilizzarne la funzione
%\chordsoff		% <<< DECOMMENTA se vuoi una strofa senza accordi
%\chordsoff
^Frutto della nostra ^terra
 ^del lavoro di ogni ^uomo
^vino delle nostre ^vigne 
 sulla ^mensa dei fratelli ^tuoi.  ^
^Tu che lo prendevi un ^giorno, 
 ^lo bevevi con i ^tuoi
^oggi vieni in questo ^vino 
 e ti ^doni per la vita ^mia.
\endverse





%%%%% RITORNELLO
\beginchorus
\textnote{\textbf{Rit.}}
E sarò \[G]pane, e sarò \[D]vino,
nella mia \[E-]vita, nelle tue \[B-]mani.
Ti accoglie\[C]rò dentro di \[D]me,
farò di \[E-]me un'offerta \[C]viva,
un sacri\[A-]ficio \[D] gradito a \[E-]Te,
\[C] un sacri\[A-]ficio \[D] gradito a \[G]Te.
\endchorus





%%%%%% EV. INTERMEZZO
\beginverse*
\vspace*{1.3\versesep}
{
	\nolyrics
	\musicnote{Chiusura strumentale:}
	
	\ifchorded

	%---- Prima riga -----------------------------
	\vspace*{-\versesep}
	\[C] \[G]  \[C]	 \[G*] 


	\fi
	%---- Ev Indicazioni -------------------------			
	%\textnote{\textit{(ripetizione della strofa)}} 
	 
}
\vspace*{\versesep}
\endverse


\endsong
%------------------------------------------------------------
%			FINE CANZONE
%------------------------------------------------------------



%GGG
%-------------------------------------------------------------
%			INIZIO	CANZONE
%-------------------------------------------------------------


%titolo: 	Giovane donna
%autore: 	Scaglianti, Bancolini
%tonalita: 	Re 



%%%%%% TITOLO E IMPOSTAZONI
\beginsong{Giovane donna}[by={Scaglianti, Bancolini}] 	% <<< MODIFICA TITOLO E AUTORE
\transpose{0} 						% <<< TRASPOSIZIONE #TONI (0 nullo)
\momenti{Canti Mariani; }							% <<< INSERISCI MOMENTI	
% momenti vanno separati da ; e vanno scelti tra:
% Ingresso; Atto penitenziale; Acclamazione al Vangelo; Dopo il Vangelo; Offertorio; Comunione; Ringraziamento; Fine; Santi; Pasqua; Avvento; Natale; Quaresima; Canti Mariani; Battesimo; Prima Comunione; Cresima; Matrimonio; Meditazione; Spezzare del pane;
\ifchorded
	%\textnote{Tonalità migliore }	% <<< EV COMMENTI (tonalità originale/migliore)
\fi


%%%%%% INTRODUZIONE
\ifchorded
\vspace*{\versesep}
\textnote{Intro: \qquad \qquad  (\eighthnote 104)} % <<  MODIFICA IL TEMPO
% Metronomo: \eighthnote (ottavo) \quarternote (quarto) \halfnote (due quarti)
\vspace*{-\versesep}
\beginverse*

\nolyrics

%---- Prima riga -----------------------------
\vspace*{-\versesep}
\[D] \[E-]\[G]  \[D]	 % \[*D] per indicare le pennate, \rep{2} le ripetizioni

%---- Ogni riga successiva -------------------
%\vspace*{-\versesep}
%\[G] \[C]  \[D]	

%---- Ev Indicazioni -------------------------			
%\textnote{\textit{(Oppure tutta la strofa)} }	

\endverse
\fi





%%%%% STROFA
\beginverse		%Oppure \beginverse* se non si vuole il numero di fianco
\memorize 		% <<< DECOMMENTA se si vuole utilizzarne la funzione
%\chordsoff		% <<< DECOMMENTA se vuoi una strofa senza accordi
\[D]Giovane \[E-]donna, at\[G]tesa dell'umani\[D]tà,
un desi\[E-]derio d'a\[G]more e pura liber\[D]tà.
Il Dio lon\[F#-]tano è \[G]qui, vicino a \[A]te,
\[D]voce e si\[F#-]lenzio, an\[G]nuncio di novi\[A]tà. \[A4/3]
\endverse

\beginchorus
\[D]\[F#7]A\[B-]ve, Ma\[G]\[G-]ri\[D]a. 
 \[D]\[F#7]A\[B-]ve, Ma\[E-]\[A7]ri\[D]a.
\endchorus

\beginverse
%\chordsoff
^Dio t'ha pre^scelta qual ^madre piena di bel^lezza, 
ed il suo a^more ti av^volgerà con la sua ^ombra.
Grembo per ^Dio ve^nuto sulla ^terra,
^tu sarai ^madre di un ^uomo nuo^vo. ^
\endverse

\beginverse
%\chordsoff
^Ecco l'an^cella che ^vive della tua Pa^rola
libero il ^cuore per^ché l'amore trovi ^casa.
Ora l'at^tesa è ^densa di pre^ghiera,
^e l'uomo ^nuovo è ^qui, in mezzo a ^noi. ^
\endverse
\endsong
%------------------------------------------------------------
%			FINE CANZONE
%------------------------------------------------------------


%-------------------------------------------------------------
%			INIZIO	CANZONE
%-------------------------------------------------------------


%titolo: 	Acqua siamo noi
%autore: 	Cento
%tonalita: 	Re



%%%%%% TITOLO E IMPOSTAZONI
\beginsong{Giullare dei campi}[by={Canto Salesiano dedicato a Don Bosco — P. Pignatelli}] 	% <<< MODIFICA TITOLO E AUTORE
\transpose{0} 						% <<< TRASPOSIZIONE #TONI (0 nullo)
\momenti{Congedo; Santi}							% <<< INSERISCI MOMENTI	
% momenti vanno separati da ; e vanno scelti tra:
% Ingresso; Atto penitenziale; Acclamazione al Vangelo; Dopo il Vangelo; Offertorio; Comunione; Ringraziamento; Fine; Santi; Pasqua; Avvento; Natale; Quaresima; Canti Mariani; Battesimo; Prima Comunione; Cresima; Matrimonio; Meditazione;
\ifchorded
	%\textnote{Tonalità originale }	% <<< EV COMMENTI (tonalità originale/migliore)
\fi



%%%%%% INTRODUZIONE
\ifchorded
\vspace*{\versesep}
\musicnote{
\begin{minipage}{0.48\textwidth}
\textbf{Intro}
\hfill 
%( \eighthnote \, 80)   % <<  MODIFICA IL TEMPO
% Metronomo: \eighthnote (ottavo) \quarternote (quarto) \halfnote (due quarti)
\end{minipage}
} 	
\vspace*{-\versesep}
\beginverse*
\nolyrics

%---- Prima riga -----------------------------
\vspace*{-\versesep}
\[C] \[A-]  \[F]	\[C] % \[*D] per indicare le pennate, \rep{2} le ripetizioni

%---- Ogni riga successiva -------------------
%\vspace*{-\versesep}
%\[G] \[C]  \[D]	

%---- Ev Indicazioni -------------------------			
%\textnote{\textit{(Come la prima riga)} }	

\endverse
\fi



%%%%% STROFA
\beginverse		%Oppure \beginverse* se non si vuole il numero di fianco
\memorize 		% <<< DECOMMENTA se si vuole utilizzarne la funzione
%\chordsoff		& <<< DECOMMENTA se vuoi una strofa senza accordi

Cal\[C]zoni colore del \[A-]prato, 
un ginocchio amma\[F]ccato 
per un salto in \[C]più, 
due pi\[G7]ante un filo ti\[A-]rato,
la mela sul \[F]naso e gli amici \[G7]giù. 
Un \[D-]pezzo di pane e una \[G7]fetta di cielo, 
sa\[C]pore di festa e \[A-]tu: 
Gio\[F]vanni dei Becchi giul\[C]lare dei campi 
re\[G7]galo alla gioven\[C]tù.

\endverse



%%%%% RITORNELLO
\textnote{\textbf{Rit.}}
\beginchorus

Siete tutti \[F]ladri ragazzi mi\[G7]ei, 
non ho più il mio \[C]cuore ce l’avete \[A-]voi! 
Ma non m’inte\[F]ressa da quest’oggi in \[G7]poi 
ogni mio res\[F]piro sarà per \[C]voi. \rep{2}

\endchorus



%%%%% STROFA
\beginverse
La ^veste color della st^rada 
forse un pò consu^mata,
qualche acciacco in ^più, 
nei ^prati intorno a Val^docco 
ti chiama don ^Bosco la tua gioven^tù. 
La ^vecchia tettoia e una ^piccola stanza
fra ^spiagge infinite in ^cuor, 
un ^fischio per Corso Re^gina, uno sguardo 
pro^fondo sentono l’a^more!

\endverse



%%%%% STROFA
\beginverse
%\chordsoff
Un ^eco color della ^storia, 
tesoro dei ^campi 
che oggi non è ^più, 
il ^vecchio pilone del ^sogno, 
il ragazzo sul ^filo non esiste ^più. 
L’an^tica fontana del ^grande cortile 
non ^getta più acqua e ^tu... 
as^petti qualcuno che ^ancora racconti 
l’a^more alla gioven^tù. 
\endverse




\endsong
%------------------------------------------------------------
%			FINE CANZONE
%------------------------------------------------------------
%titolo{Grandi cose}
%autore{Gen Rosso}
%album{Se siamo uniti}
%tonalita{Do}
%famiglia{Liturgica}
%gruppo{}
%momenti{Ingresso}
%identificatore{grandi_cose}
%data_revisione{2011_12_31}
%trascrittore{Francesco Endrici}
\beginsong{Grandi cose}[by={Gen\ Rosso}]
\beginchorus
\[C]Grandi \[G]cose ha fatto \[D-]il Si\[A-]gnore per noi,
\[C]ha fatto \[F]germogliare i \[C]fiori tra le \[G4]rocce. \[G]
\[C]Grandi \[G]cose ha fatto \[D-]il Si\[A-]gnore per noi,
\[C]ci ha ripor\[F]tati liberi \[C]alla nostra \[G4]terra.
Ed \[A-]ora possiamo can\[D-]tare, possiamo gri\[E-]dare
l'amore che \[F]Dio ha versato su \[G4]noi. \[G]
\endchorus
\beginverse*
\[C]Tu che \[G]sai strap\[D-]pare dalla \[A-]morte, \[C]
hai solle\[F]vato il nostro \[C]viso dalla \[G]polvere.
\[C]Tu che \[G]hai sen\[D-]tito il nostro \[A-]pianto, \[C]
nel nostro \[F]cuore hai messo un \[C]seme di \[G]felicità.
\endverse
\endsong


%HHH
%-------------------------------------------------------------
%			INIZIO	CANZONE
%-------------------------------------------------------------


%titolo: 	Holy is the Lord
%autore: 	Chris Tomlin
%tonalita: 	Sol 



%%%%%% TITOLO E IMPOSTAZONI
\beginsong{Holy is the Lord}[by={C. Tomlin, L. Giglio}] 	% <<< MODIFICA TITOLO E AUTORE
\transpose{0} 						% <<< TRASPOSIZIONE #TONI (0 nullo)
\momenti{Natale}							% <<< INSERISCI MOMENTI
\ifchorded
	\textnote{Tonalità originale }
\fi


%%%%%% INTRODUZIONE
\ifchorded
\vspace*{\versesep}
\textnote{Intro: \qquad \qquad  }%(\eighthnote 116) % << MODIFICA IL TEMPO
\vspace*{-\versesep}
\beginverse*

\nolyrics

%---- Prima riga -----------------------------
\vspace*{-\versesep}
\[G] \[C]  \[D]	 \rep{2}

%---- Ogni riga successiva -------------------
%\vspace*{-\versesep}
%\[G] \[C]  \[D]	

%---- Ev Indicazioni -------------------------			
%\textnote{\textit{(Oppure tutta la strofa)} }	

\endverse
\fi



%%%%% STROFA
\beginverse		%Oppure \beginverse* se non si vuole il numero di fianco
%\memorize 		% <<< DECOMMENTA se si vuole utilizzarne la funzione
%\chordsoff		& <<< DECOMMENTA se vuoi una strofa senza accordi

We \[G]stand and \[C]lift up our \[D]hands,
for the \[E-]joy of the \[C]Lord is our str\[D]enght.
We \[G]bow down and \[C]worship Him \[D]now,
how \[E-]great how \[C]awesome is \[D]He.

\endverse




%%%%% RITORNELLO
\beginchorus
Holy is the \[G]Lord
\[C]God al\[D]mighty.
The \[E-]Earth is \[C]filled
with His \[D]glory. \rep{2} \ifchorded \quad \qquad \nolyrics \[G] \[C]  \[D]	\rep{2} \fi

\endchorus








%%%%%% FINALE

\beginchorus
\vspace*{1.3\versesep}
\textnote{Finale \textit{(rallentando)}}
The \[E-]Earth is \[C]filled
with His \[D]glory.
The \[E-]Earth is \[C]filled
with His \[D]glo-o-ory.
\endchorus



\endsong
%------------------------------------------------------------
%			FINE CANZONE
%------------------------------------------------------------






%III
%titolo{I cieli narrano}
%autore{Frisina}
%album{Benedici il Signore}
%tonalita{Re}
%famiglia{Liturgica}
%gruppo{}
%momenti{Ingresso;Salmi}
%identificatore{i_cieli_narrano}
%data_revisione{2011_12_31}
%trascrittore{Francesco Endrici - Manuel Toniato}
\beginsong{I cieli narrano}[by={Frisina}]

\beginchorus
I \[D]cieli \[G]narrano la \[D]gloria di \[A]Dio
e il \[B-]firma\[G]mento annunzia \[D]l'opera \[A]sua.
Al\[B-]lelu\[E-]ia al\[A]lelu\[D]ia \brk al\[B-]lelu\[G]ia alle\[E-7]\[A]lu\[D]ia.
\endchorus

\beginverse
Il \[D]giorno al \[G]giorno ne a\ch{D}{f}{f}{ff}ida il mes\[A]saggio
la \[B-]notte alla \[G]notte ne tras\[D]mette no\[A]tizia
non \[B-]è un lin\[G]guaggio non \[A]sono pa\[D]role 
di \[B-]cui non si \[E]oda il \[A]suo\[A7]no.
\endverse

\beginverse
\chordsoff
Là pose una tenda per il sole che sorge
è come uno sposo dalla stanza nuziale
esulta come un prode che corre 
con gioia la sua strada.
\endverse

\beginverse
\chordsoff
Lui sorge dall'ultimo estremo del cielo
e la sua corsa l'altro estremo raggiunge.
Nessuna delle creature potrà 
mai sottrarsi al suo calore. 
\endverse

\beginverse
\chordsoff
La legge di Dio rinfranca l'anima
la testimonianza del Signore è verace.
Gioisce il cuore ai suoi giusti precetti 
che danno la luce agli occhi.
\endverse
\endsong



%titolo{Il canto dell'amore}
%autore{Russo}
%album{}
%tonalita{Mi}
%famiglia{Liturgica}
%gruppo{}
%momenti{}
%identificatore{il_canto_dell_amore}
%data_revisione{2011_12_31}
%trascrittore{Francesco Endrici}
\beginsong{Il canto dell'amore}[by={Russo}]
\beginverse*
Se dovrai at\[E]traversare il de\[C#-7]serto
non te\[A]mere io sarò con \[E]te
se do\[E]vrai camminare nel \[C#-7]fuoco
la sua \[A]fiamma non ti bruce\[E]rà
segui\[B]rai la mia \[A]luce nella \[E]notte \[E]
senti\[F#-]rai la mia \[B]forza nel cam\[C#-]mino \[C#-]
io s\[D]ono il tuo Dio, \[A] il Signo\[E]re. \[C#-7]\[A]\[E]
\endverse
\beginverse*
Sono ^io che ti ho fatto e plas^mato
ti ho chi^amato per no^me
io da ^sempre ti ho cono^sciuto
e ti ho ^dato il mio amo^re
perché ^tu sei pre^zioso ai miei ^occhi ^
vali ^più del più ^grande dei te^sori ^
io sa^rò con te ^ dovunque an^drai. ^^^
\endverse
\ifchorded
\beginverse*
\vspace*{-\versesep}
{\nolyrics \[B]\[A]\[E]\[E]\[D]\[A]\[B]\[B]}
\endverse
\fi
\beginverse*
Non pen^sare alle cose di ^ieri
cose ^nuove fioriscono ^già
apri^rò nel deserto sen^tieri
darò ^acqua nell'aridi^tà
perché ^tu sei pre^zioso ai miei ^occhi ^
vali ^più del più ^grande dei te^sori ^
io sa^rò con te ^ dovunque an^drai ^^^
perché \[B]tu sei pre\[A]zioso ai miei \[E]occhi \[E]
vali \[F#-]più del più \[B]grande dei te\[C#-]sori \[C#-]
io sa\[D]rò con te \[A] dovunque an\[E]drai. \[C#-7]\[A]\[E]
\endverse
\beginverse*
\[E] Io ti sa\[C#-7]rò accanto \[A]sarò con \[E]te 
\[E] per tutto il \[C#-7]tuo viaggio \[A]sarò con \[E]te. 
\[E] Io ti sa\[C#-7]rò accanto \[A]sarò con \[E]te 
\[E] per tutto il \[C#-7]tuo viaggio \[A]sarò con \[E]te. 
\endverse
\endsong


%-------------------------------------------------------------
%			INIZIO	CANZONE
%-------------------------------------------------------------


%titolo: 	Il cuore nuovo
%autore: 	D. MAchetta
%tonalita: 	Do-



%%%%%% TITOLO E IMPOSTAZONI
\beginsong{Il cuore nuovo}[by={D. Machetta}] 	% <<< MODIFICA TITOLO E AUTORE
%#ALIAS Io verrò a salvarvi
\transpose{2} 						% <<< TRASPOSIZIONE #TONI (0 nullo)
\preferflats  %SE VOGLIO FORZARE i bemolle come alterazioni
%\prefersharps %SE VOGLIO FORZARE i # come alterazioni
\momenti{Atto penitenziale; }							% <<< INSERISCI MOMENTI	
% momenti vanno separati da ; e vanno scelti tra:
% Ingresso; Atto penitenziale; Acclamazione al Vangelo; Dopo il Vangelo; Offertorio; Comunione; Ringraziamento; Fine; Santi; Pasqua; Avvento; Natale; Quaresima; Canti Mariani; Battesimo; Prima Comunione; Cresima; Matrimonio; Meditazione; Spezzare del pane;
\ifchorded
	%\textnote{Tonalità migliore }	% <<< EV COMMENTI (tonalità originale\migliore)
\fi


%%%%%% INTRODUZIONE
\ifchorded
\vspace*{\versesep}
\musicnote{
\begin{minipage}{0.48\textwidth}
\textbf{Intro}
\hfill 
%( \eighthnote \, 80)   % <<  MODIFICA IL TEMPO
% Metronomo: \eighthnote (ottavo) \quarternote (quarto) \halfnote (due quarti)
\end{minipage}
} 	
\vspace*{-\versesep}
\beginverse*
\nolyrics

%---- Prima riga -----------------------------
\vspace*{-\versesep}
\[C-] \[G-] \[C-]	 % \[*D] per indicare le pennate, \rep{2} le ripetizioni

%---- Ogni riga successiva -------------------
%\vspace*{-\versesep}
%\[G] \[C]  \[D]	

%---- Ev Indicazioni -------------------------			
%\textnote{\textit{(Oppure tutta la strofa)} }	

\endverse
\fi




%%%%% STROFA
\beginverse		%Oppure \beginverse* se non si vuole il numero di fianco
\memorize 		% <<< DECOMMENTA se si vuole utilizzarne la funzione
%\chordsoff		% <<< DECOMMENTA se vuoi una strofa senza accordi

\[C-]Io verrò a salvarvi tra le \[G-]genti,
vi condur\[A&]rò nella vostra di\[G-]mora.
\[C-]Spargerò su voi torrenti d'\[G-]acque:
da ogni \[A&]colpa sarete la\[C]vati.

\endverse




%%%%% RITORNELLO
\beginchorus
\textnote{\textbf{Rit.}}

\[E&]Dio ci da\[F]rà un cuore \[B&]nuovo,
\[E&]porrà in \[F]noi uno spirito \[G]nuovo.

\endchorus



%%%%% STROFA
\beginverse		%Oppure \beginverse* se non si vuole il numero di fianco
%\memorize 		% <<< DECOMMENTA se si vuole utilizzarne la funzione
%\chordsoff		% <<< DECOMMENTA se vuoi una strofa senza accordi

\[C-]Voglio liberarvi dai \[G-]peccati,
abbatte\[A&]rò ogni falso di\[G-]o.
\[C-]Tolgo il vostro cuore di pi\[G-]etra
per rega\[A&]larvi un cuore di \[C]carne.

\endverse



%%%%% STROFA
\beginverse		%Oppure \beginverse* se non si vuole il numero di fianco
%\memorize 		% <<< DECOMMENTA se si vuole utilizzarne la funzione
%\chordsoff		% <<< DECOMMENTA se vuoi una strofa senza accordi

\[C-]Voi osserverete la mia \[G-]legge
e abite\[A&]rete la terra dei \[G-]padri.
\[C-]Voi sarete il popolo fe\[G-]dele
e io sa\[A&]rò il vostro Dio per \[C]sempre.

\endverse




\endsong
%------------------------------------------------------------
%			FINE CANZONE
%------------------------------------------------------------



%titolo{Il disegno}
%autore{Marani}
%album{}
%tonalita{La-}
%famiglia{Liturgica}
%gruppo{}
%momenti{Comunione}
%identificatore{il_disegno}
%data_revisione{2011_12_31}
%trascrittore{Francesco Endrici}
\beginsong{Il disegno}[by={Marani}]
\beginverse
Nel \[A-]mare del si\[F]lenzio una \[G]voce si al\[C]zò, \[E7]
da una \[A-]notte senza con\[F]fini una \[G]luce bril\[C]lò, \[E7]
dove non \[A-]c'era niente quel \[E7]giorno.
\endverse
\beginchorus
A\[A-]vevi scritto \[D-]già il mio \[G]nome lassù nel \[C]cielo \[E7]
a\[A-]vevi scritto \[D-]già la mia \[G]vita insieme a \[C]te, \[E7]
avevi \[A-]scritto già di \[E7]me.
\endchorus
\beginverse
%\chordsoff
E ^quando la tua ^mente fece ^splendere le ^stelle ^
e ^quando le tue ^mani model^larono la ^ter^ra, 
dove non c'^era niente quel ^giorno.
\endverse
\beginverse
%\chordsoff
E ^quando hai calco^lato la pro^fondità del ^cie^lo
e ^quando hai colo^rato ogni ^fiore della ^ter^ra,
dove non ^c'era niente quel ^giorno.
\endverse
\beginverse
%\chordsoff
E ^quando hai dise^gnato le ^nubi e le mon^ta^gne,
e ^quando hai dise^gnato il cam^mino di ogni ^uo^mo,
l'avevi ^fatto anche per ^me.
\endverse
\beginchorus
Se \[A-]ieri non sa\[D-]pevo oggi \[G]ho incontrato \[C]te \[E7]
e \[A-]la mia liber\[D-]tà è il tuo di\[G]segno su di \[C]me \[E7]
non cerche\[A-]rò più niente per\[E7]ché Tu mi salve\[A-]rai.
\endchorus
\endsong


%-------------------------------------------------------------
%			INIZIO	CANZONE
%-------------------------------------------------------------


%titolo: 	Il mattino di Pasqua
%autore: 	Sequeri
%tonalita: 	Do



%%%%%% TITOLO E IMPOSTAZONI
\beginsong{Il mattino di Pasqua}[by={Sequeri}] 	% <<< MODIFICA TITOLO E AUTORE
\transpose{-2} 						% <<< TRASPOSIZIONE #TONI (0 nullo)
\momenti{Pasqua}							% <<< INSERISCI MOMENTI	
% momenti vanno separati da ; e vanno scelti tra:
% Ingresso; Atto penitenziale; Acclamazione al Vangelo; Dopo il Vangelo; Offertorio; Comunione; Ringraziamento; Fine; Santi; Pasqua; Avvento; Natale; Quaresima; Canti Mariani; Battesimo; Prima Comunione; Cresima; Matrimonio; Meditazione; Spezzare del pane;
\ifchorded
	%\textnote{Tonalità migliore }	% <<< EV COMMENTI (tonalità originale/migliore)
\fi


%%%%%% INTRODUZIONE
\ifchorded
\vspace*{\versesep}
\textnote{Intro: \qquad \qquad  }%(\eighthnote 116) % <<  MODIFICA IL TEMPO
% Metronomo: \eighthnote (ottavo) \quarternote (quarto) \halfnote (due quarti)
\vspace*{-\versesep}
\beginverse*

\nolyrics

%---- Prima riga -----------------------------
\vspace*{-\versesep}
\[D] \[F#-] \[B-]\[E]  % \[*D] per indicare le pennate, \rep{2} le ripetizioni

%---- Ogni riga successiva -------------------
\vspace*{-\versesep}
 \[A*] \[E*] \[A]

%---- Ev Indicazioni -------------------------			
%\textnote{\textit{(Oppure tutta la strofa)} }	

\endverse
\fi



%%%%% RITORNELLO
\beginchorus
\textnote{\textbf{Rit.}}
\[D] II Signore è ri\[F#-]sorto: cantate con \[B-]noi!
egli ha vinto la \[E]morte, 
al\[A*]le\[E*]lu\[A]ia! \quad \[A]
\[D] Allelu-u\[F#-]ia! Allelu-u\[B-]ia!
Allelu-u\[E]ia!
Al\[A*]le\[E*]lu\[A]ia! \quad \[A]
\endchorus



%%%%% STROFA
\beginverse		%Oppure \beginverse* se non si vuole il numero di fianco
\memorize 		% <<< DECOMMENTA se si vuole utilizzarne la funzione
%\chordsoff		% <<< DECOMMENTA se vuoi una strofa senza accordi
\textnote{\textit{Cambia la velocità}}
\[D] II mattino di \[F#-]Pasqua, \brk nel ricordo di \[B-]lui,
siamo andate al se\[E]polcro: \brk non era più \[A7]là! \quad \[A7]
\[D] Senza nulla spe\[F#-]rare, \brk con il cuore sos\[B-]peso,
siamo andati al se\[E]polcro: \brk non era più \[A7]là! \quad \[A7]
\endverse






%%%%% STROFA
\beginverse		%Oppure \beginverse* se non si vuole il numero di fianco
%\memorize 		% <<< DECOMMENTA se si vuole utilizzarne la funzione
%\chordsoff		% <<< DECOMMENTA se vuoi una strofa senza accordi
^ Sulla strada di ^casa \brk parlavamo di ^lui
e l’abbiamo incon^trato: \brk ha mangiato con ^noi! \quad ^
^ Sulle rive del ^lago \brk pensavamo a quei ^giorni
e l’abbiamo incon^trato: \brk ha mangiato con ^noi! \quad ^
\endverse




%%%%% STROFA
\beginverse		%Oppure \beginverse* se non si vuole il numero di fianco
%\memorize 		% <<< DECOMMENTA se si vuole utilizzarne la funzione
\chordsoff		% <<< DECOMMENTA se vuoi una strofa senza accordi
Oggi ancora fratelli, \brk ricordando quei giorni,
ascoltiamo la voce \brk del Signore tra noi!
E, spezzando il suo pane \brk con la gioia nel cuore
noi cantiamo alla vita \brk nell’attesa di lui!
\endverse



\endsong
%------------------------------------------------------------
%			FINE CANZONE
%------------------------------------------------------------



%-------------------------------------------------------------
%			INIZIO	CANZONE
%-------------------------------------------------------------


%titolo: 	Il pane del cammino
%autore: 	Sequeri
%tonalita: 	Do



%%%%%% TITOLO E IMPOSTAZONI
\beginsong{Il pane del cammino}[by={Sequeri}] 	% <<< MODIFICA TITOLO E AUTORE
\transpose{0} 						% <<< TRASPOSIZIONE #TONI (0 nullo)
\momenti{}							% <<< INSERISCI MOMENTI	
% momenti vanno separati da ; e vanno scelti tra:
% Ingresso; Atto penitenziale; Acclamazione al Vangelo; Dopo il Vangelo; Offertorio; Comunione; Ringraziamento; Fine; Santi; Pasqua; Avvento; Natale; Quaresima; Canti Mariani; Battesimo; Prima Comunione; Cresima; Matrimonio; Meditazione; Spezzare del pane;
\ifchorded
	%\textnote{Tonalità migliore }	% <<< EV COMMENTI (tonalità originale/migliore)
\fi


%%%%%% INTRODUZIONE
\ifchorded
\vspace*{\versesep}
\textnote{Intro: \qquad \qquad  (\eighthnote 76)} % <<  MODIFICA IL TEMPO
% Metronomo: \eighthnote (ottavo) \quarternote (quarto) \halfnote (due quarti)
\vspace*{-\versesep}
\beginverse*

\nolyrics

%---- Prima riga -----------------------------
\vspace*{-\versesep}
\[C] \[G] \[C]  % \[*D] per indicare le pennate, \rep{2} le ripetizioni

%---- Ogni riga successiva -------------------
%\vspace*{-\versesep}
%\[A*] \[E*] \[A]

%---- Ev Indicazioni -------------------------			
%\textnote{\textit{(Oppure tutta la strofa)} }	

\endverse
\fi



%%%%% RITORNELLO
\beginchorus
\textnote{\textbf{Rit.}}
\[C]Il tuo \[G7]popolo in cam\[A-7]mi\[C]no
\[F]cerca in \[D7]te la \[C4]gui\[G7]da.
\[C]Sulla \[G7]strada verso il \[A-7]re\[C]gno
\[F]sei so\[D7]stegno col tuo \[C]cor\[G7]po:
\[E-]resta \[A7]sempre con \[D-]noi, 
\[F]o Si\[C*]gno-\[G*]o-\[C]re!
\endchorus



%%%%% STROFA
\beginverse		%Oppure \beginverse* se non si vuole il numero di fianco
\memorize 		% <<< DECOMMENTA se si vuole utilizzarne la funzione
%\chordsoff		% <<< DECOMMENTA se vuoi una strofa senza accordi
\[E]Se dovrai attraversare il de\[C#-]serto
non te\[A]mere io sarò con \[E]te
se do\[E]vrai camminare nel \[C#-]fuoco
la sua \[A]fiamma non ti bruce\[E]rà
segui\[B]rai la mia luce nella notte
sentirai la mia forza nel cammino
io sono il tuo Dio, il Signore.

Sono io che ti ho fatto e plasmato
ti ho chiamato per nome
io da sempre ti ho conosciuto
e ti ho dato il mio amore
perché tu sei prezioso ai miei occhi
vali più del più grande dei tesori
io sarò con te dovunque andrai.

Non pensare alle cose di ieri
cose nuove fioriscono già
aprirò nel deserto sentieri
darò acqua nell'aridità
perché tu sei prezioso ai miei occhi
vali più del più grande dei tesori
io sarò con te dovunque andrai
perché tu sei prezioso ai miei occhi
vali più del più grande dei tesori
io sarò con te dovunque andrai.

Io ti sarò accanto sarò con te
per tutto il tuo viaggio sarò con te
io ti sarò accanto sarò con te
per tutto il tuo viaggio sarò con te
\endverse






%%%%% STROFA
\beginverse		%Oppure \beginverse* se non si vuole il numero di fianco
%\memorize 		% <<< DECOMMENTA se si vuole utilizzarne la funzione
\chordsoff		% <<< DECOMMENTA se vuoi una strofa senza accordi
È il vino, Gesù, che ci disseta
e sveglia in noi l'ardore di seguirti.
Se la gioia cede il passo alla stanchezza,
la tua voce fa rinascere freschezza.
\endverse




%%%%% STROFA
\beginverse		%Oppure \beginverse* se non si vuole il numero di fianco
%\memorize 		% <<< DECOMMENTA se si vuole utilizzarne la funzione
\chordsoff		% <<< DECOMMENTA se vuoi una strofa senza accordi
È il tuo Corpo, Gesù, che ci fa Chiesa,
fratelli sulle strade della vita.
Se il rancore toglie luce all’amicizia,
dal tuo cuore nasce giovane il perdono.
\endverse



%%%%% STROFA
\beginverse		%Oppure \beginverse* se non si vuole il numero di fianco
%\memorize 		% <<< DECOMMENTA se si vuole utilizzarne la funzione
\chordsoff		% <<< DECOMMENTA se vuoi una strofa senza accordi
È il tuo Sangue, Gesù, il segno eterno
dell’unico linguaggio dell’amore.
Se il donarsi come te richiede fede,
nel tuo Spirito sfidiamo l’incertezza.
\endverse



%%%%% STROFA
\beginverse		%Oppure \beginverse* se non si vuole il numero di fianco
%\memorize 		% <<< DECOMMENTA se si vuole utilizzarne la funzione
\chordsoff		% <<< DECOMMENTA se vuoi una strofa senza accordi
È il tuo Dono, Gesù, la vera fonte
del gesto coraggioso di chi annuncia.
Se la Chiesa non è aperta ad ogni uomo,
il tuo fuoco le rivela la missione.
\endverse



\endsong
%------------------------------------------------------------
%			FINE CANZONE
%------------------------------------------------------------



%-------------------------------------------------------------
%			INIZIO	CANZONE
%-------------------------------------------------------------


%titolo: 	Il Signore ci ha salvato
%autore: 	L. Capello, A. Roncari
%tonalita: 	RE-



%%%%%% TITOLO E IMPOSTAZONI
\beginsong{Il Signore ci ha salvato}[by={L. Capello, A. Roncari}] 	% <<< MODIFICA TITOLO E AUTORE
\transpose{0} 						% <<< TRASPOSIZIONE #TONI (0 nullo)
%\preferflats  %SE VOGLIO FORZARE i bemolle come alterazioni
%\prefersharps %SE VOGLIO FORZARE i # come alterazioni
\momenti{Quaresima}							% <<< INSERISCI MOMENTI	
% momenti vanno separati da ; e vanno scelti tra:
% Ingresso; Atto penitenziale; Acclamazione al Vangelo; Dopo il Vangelo; Offertorio; Comunione; Ringraziamento; Fine; Santi; Pasqua; Avvento; Natale; Quaresima; Canti Mariani; Battesimo; Prima Comunione; Cresima; Matrimonio; Meditazione; Spezzare del pane;
\ifchorded
	%\textnote{Tonalità migliore }	% <<< EV COMMENTI (tonalità originale\migliore)
\fi


%%%%%% INTRODUZIONE
\ifchorded
\vspace*{\versesep}
\musicnote{
\begin{minipage}{0.48\textwidth}
\textbf{Intro}
\hfill 
%( \eighthnote \, 80)   % <<  MODIFICA IL TEMPO
% Metronomo: \eighthnote (ottavo) \quarternote (quarto) \halfnote (due quarti)
\end{minipage}
} 	
\vspace*{-\versesep}
\beginverse*

\nolyrics

%---- Prima riga -----------------------------
\vspace*{-\versesep}
\[D-] \[G-] \[D-]	 % \[*D] per indicare le pennate, \rep{2} le ripetizioni

%---- Ogni riga successiva -------------------
%\vspace*{-\versesep}
%\[G] \[C]  \[D]	

%---- Ev Indicazioni -------------------------			
%\textnote{\textit{(Oppure tutta la strofa)} }	

\endverse
\fi




%%%%% STROFA
\beginverse		%Oppure \beginverse* se non si vuole il numero di fianco
\memorize 		% <<< DECOMMENTA se si vuole utilizzarne la funzione
%\chordsoff		% <<< DECOMMENTA se vuoi una strofa senza accordi

Il Sign\[D-]ore ci ha sal\[G-]vato dai ne\[D-]mici
nel pass\[G-]aggio \[C7]del Mar \[F]Rosso:
\[G-]l'acqua che ha travolto gli Egi\[D-]ziani
fu per \[G-]noi \[A7]la sal\[D-]vezza.\[D7]
\endverse




%%%%% RITORNELLO
\beginchorus
\textnote{\textbf{Rit.}}

"Se cono\[G-]scessi il \[C]dono di \[F]Dio
e chi è co\[G-]lui che ti chi\[A7]ede da \[D-]bere,
lo preghe\[B&]resti tu \[C]stesso di \[F]darti
quell'acqua \[G-]viva che ti salve\[A7]rà".

\endchorus



%%%%% STROFA
\beginverse		%Oppure \beginverse* se non si vuole il numero di fianco
%\memorize 		% <<< DECOMMENTA se si vuole utilizzarne la funzione
%\chordsoff		% <<< DECOMMENTA se vuoi una strofa senza accordi

Era\[D-]vamo pro\[G-]strati nel de\[D-]serto,
consu\[G-]mati \[C7]dalla se\[F]te:
qu\[G-]ando fu percossa la \[D-]roccia,
zampi\[G-]llò \[A7]una so\[D-]rgente.\[D7]

\endverse




%%%%% STROFA
\beginverse		%Oppure \beginverse* se non si vuole il numero di fianco
%\memorize 		% <<< DECOMMENTA se si vuole utilizzarne la funzione
%\chordsoff		% <<< DECOMMENTA se vuoi una strofa senza accordi
\chordsoff
Dal\[D-]le mura del te\[G-]mpio di \[D-]Dio
sgorga \[G-]un fi\[C7]ume d'acqua \[F]viva:
\[G-]tutto quel che l'acqua \[D-]toccherà
na\[G-]scerà \[A7]a nuova vi\[D-]ta.\[D7]
\endverse

%%%%% STROFA
\beginverse		%Oppure \beginverse* se non si vuole il numero di fianco
%\memorize 		% <<< DECOMMENTA se si vuole utilizzarne la funzione
%\chordsoff		% <<< DECOMMENTA se vuoi una strofa senza accordi
\chordsoff
Ve\[D-]nga a me chi ha se\[G-]te e chi mi \[D-]cerca,
si di\[G-]sseti co\[C7]lui che in me cr\[F]ede:
\[G-]fiumi d'acqua viva scorre\[D-]ranno
dal mio \[G-]cuo\[A7]re tra\[D-]fitto.\[D7]
\endverse

%%%%% STROFA
\beginverse		%Oppure \beginverse* se non si vuole il numero di fianco
%\memorize 		% <<< DECOMMENTA se si vuole utilizzarne la funzione
%\chordsoff		% <<< DECOMMENTA se vuoi una strofa senza accordi
\chordsoff
Sul\[D-]la croce il Fi\[G-]glio di \[D-]Dio
fu tra\[G-]fitto \[C7]da una la\[F]ncia:
dal cuore dell'Agnello imm\[D-]olato
sca\[G-]turà sa\[A7]ngue ed a\[D-]cqua.\[D7]
\endverse


%%%%% STROFA
\beginverse		%Oppure \beginverse* se non si vuole il numero di fianco
%\memorize 		% <<< DECOMMENTA se si vuole utilizzarne la funzione
%\chordsoff		% <<< DECOMMENTA se vuoi una strofa senza accordi
\chordsoff
Chi \[D-]berrà l'acqu\[G-]a viva che io \[D-]dono
non a\[G-]vrà mai più \[C7]sete in e\[F]terno:
\[G-]in lui diventerà una sor\[D-]gente
za\[G-]mpilla\[A7]nte per se\[D-]mpre.\[D7]
\endverse




\endsong
%------------------------------------------------------------
%			FINE CANZONE
%------------------------------------------------------------




%-------------------------------------------------------------
%			INIZIO	CANZONE
%-------------------------------------------------------------


%titolo: 	Il Signore è la luce
%autore: 	Giombini
%tonalita: 	Fa



%%%%%% TITOLO E IMPOSTAZONI
\beginsong{Il Signore è la luce}[by={M. Giombini}] 	% <<< MODIFICA TITOLO E AUTORE
\transpose{0} 						% <<< TRASPOSIZIONE #TONI (0 nullo)
\momenti{Acclamazione al Vangelo; Quaresima}							% <<< INSERISCI MOMENTI	
% momenti vanno separati da ; e vanno scelti tra:
% Ingresso; Atto penitenziale; Acclamazione al Vangelo; Dopo il Vangelo; Offertorio; Comunione; Ringraziamento; Fine; Santi; Pasqua; Avvento; Natale; Quaresima; Canti Mariani; Battesimo; Prima Comunione; Cresima; Matrimonio; Meditazione; Spezzare del pane;
\ifchorded
	%\textnote{Tonalità migliore }	% <<< EV COMMENTI (tonalità originale/migliore)
\fi

%%%%%% INTRODUZIONE
\ifchorded
\vspace*{\versesep}
\musicnote{
\begin{minipage}{0.48\textwidth}
\textbf{Intro}
\hfill 
%( \eighthnote \, 80)   % <<  MODIFICA IL TEMPO
% Metronomo: \eighthnote (ottavo) \quarternote (quarto) \halfnote (due quarti)
\end{minipage}
} 	
\vspace*{-\versesep}
\beginverse*

\nolyrics

%---- Prima riga -----------------------------
\vspace*{-\versesep}
\[F] \[C] \[B&]\[F]  % \[*D] per indicare le pennate, \rep{2} le ripetizioni


%---- Ev Indicazioni -------------------------			
%\textnote{\textit{(Oppure tutta la strofa)} }	

\endverse
\fi




%%%%% STROFA
\beginverse		%Oppure \beginverse* se non si vuole il numero di fianco
\memorize 		% <<< DECOMMENTA se si vuole utilizzarne la funzione
%\chordsoff		% <<< DECOMMENTA se vuoi una strofa senza accordi
\[F]Il Signore è la \[C]luce che \[B&]vince la \[F]notte.
\endverse



%%%%% RITORNELLO
\beginchorus
\textnote{\textbf{Rit.}}
\[F]Gloria! \[A-]Glo\[D-]ria! Can\[G-]tiamo al Si\[C]gno\[F]re!
\[F]Gloria! \[A-]Glo\[D-]ria! Can\[G-]tiamo al Si\[C]gno\[F]re!
\endchorus


%%%%% STROFA
\beginverse		%Oppure \beginverse* se non si vuole il numero di fianco
%\memorize 		% <<< DECOMMENTA se si vuole utilizzarne la funzione
%\chordsoff		% <<< DECOMMENTA se vuoi una strofa senza accordi
^Il Signore è la ^vita che ^vince la ^morte.
\endverse




%%%%% STROFA
\beginverse		%Oppure \beginverse* se non si vuole il numero di fianco
%\memorize 		% <<< DECOMMENTA se si vuole utilizzarne la funzione
\chordsoff		% <<< DECOMMENTA se vuoi una strofa senza accordi
Il Signore è la grazia che vince il peccato.
\endverse




%%%%% STROFA
\beginverse		%Oppure \beginverse* se non si vuole il numero di fianco
%\memorize 		% <<< DECOMMENTA se si vuole utilizzarne la funzione
\chordsoff		% <<< DECOMMENTA se vuoi una strofa senza accordi
Il Signore è la gioia che vince l’angoscia.
\endverse


%%%%% STROFA
\beginverse		%Oppure \beginverse* se non si vuole il numero di fianco
%\memorize 		% <<< DECOMMENTA se si vuole utilizzarne la funzione
\chordsoff		% <<< DECOMMENTA se vuoi una strofa senza accordi
Il Signore è la pace che vince la guerra.
\endverse




%%%%% STROFA
\beginverse		%Oppure \beginverse* se non si vuole il numero di fianco
%\memorize 		% <<< DECOMMENTA se si vuole utilizzarne la funzione
\chordsoff		% <<< DECOMMENTA se vuoi una strofa senza accordi
Il Signore è la pace che arde nei cuori.
\endverse


%%%%% STROFA
\beginverse		%Oppure \beginverse* se non si vuole il numero di fianco
%\memorize 		% <<< DECOMMENTA se si vuole utilizzarne la funzione
\chordsoff		% <<< DECOMMENTA se vuoi una strofa senza accordi
Il Signore è il sereno che vince la pioggia.
\endverse

%%%%% STROFA
\beginverse		%Oppure \beginverse* se non si vuole il numero di fianco
%\memorize 		% <<< DECOMMENTA se si vuole utilizzarne la funzione
\chordsoff		% <<< DECOMMENTA se vuoi una strofa senza accordi
Il Signore è l'amore che vince il peccato.
\endverse

%%%%% STROFA
\beginverse		%Oppure \beginverse* se non si vuole il numero di fianco
%\memorize 		% <<< DECOMMENTA se si vuole utilizzarne la funzione
\chordsoff		% <<< DECOMMENTA se vuoi una strofa senza accordi
Il Signore è speranza di un nuovo futuro.
\endverse



\endsong
%------------------------------------------------------------
%			FINE CANZONE
%------------------------------------------------------------



%-------------------------------------------------------------
%			INIZIO	CANZONE
%-------------------------------------------------------------


%titolo: 	Invochiamo la tua presenza
%autore: 	A. Napolitano, D. Bruno
%tonalita:  Re-



%%%%%% TITOLO E IMPOSTAZONI
\beginsong{Invochiamo la tua presenza}[by={A. Napolitano, D. Bruno}] 	% <<< MODIFICA TITOLO E AUTORE
\transpose{-1} 						% <<< TRASPOSIZIONE #TONI (0 nullo)
%\preferflats  %SE VOGLIO FORZARE i bemolle come alterazioni
%\prefersharps %SE VOGLIO FORZARE i # come alterazioni
\momenti{Cresima; Ringraziamento; Meditazione}							% <<< INSERISCI MOMENTI	
% momenti vanno separati da ; e vanno scelti tra:
% Ingresso; Atto penitenziale; Acclamazione al Vangelo; Dopo il Vangelo; Offertorio; Comunione; Ringraziamento; Fine; Santi; Pasqua; Avvento; Natale; Quaresima; Canti Mariani; Battesimo; Prima Comunione; Cresima; Matrimonio; Meditazione; Spezzare del pane;
\ifchorded
	%\textnote{Tonalità migliore }	% <<< EV COMMENTI (tonalità originale\migliore)
\fi

%%%%%% INTRODUZIONE
\ifchorded
\vspace*{\versesep}
\musicnote{
\begin{minipage}{0.48\textwidth}
\textbf{Intro}
\hfill 
(\quarternote  \, 60) 
%( \eighthnote \, 80)   % <<  MODIFICA IL TEMPO
% Metronomo: \eighthnote (ottavo) \quarternote (quarto) \halfnote (due quarti)
\end{minipage}
} 	
\vspace*{-\versesep}
\beginverse*


\nolyrics

%---- Prima riga -----------------------------
\vspace*{-\versesep}
\[D-] \[B&]  \[F] \[C]	 \rep{2} % \[*D] per indicare le pennate, \rep{2} le ripetizioni

%---- Ogni riga successiva -------------------
%\vspace*{-\versesep}
%\[G] \[C]  \[D]	

%---- Ev Indicazioni -------------------------			
%\textnote{\textit{(Oppure tutta la strofa)} }	

\endverse
\fi




%%%%% STROFA
\beginverse		%Oppure \beginverse* se non si vuole il numero di fianco
\memorize 		% <<< DECOMMENTA se si vuole utilizzarne la funzione
%\chordsoff		% <<< DECOMMENTA se vuoi una strofa senza accordi
\[D-]Invochiamo la \[B&]tua presenza,   \brk \[F]vieni Si\[C]gnor
\[D-]Invochiamo la \[B&]tua presenza,   \brk \[F]scendi su di \[C]noi.
\[G-]Vieni consola\[D-]tore,   \brk dona \[B&]pace ed um\[C]iltà.
\[G-]Acqua viva dà \[D-]amore,   \brk questo \[B&]cuore apriamo a \[A]te. \[A]
\endverse




%%%%% RITORNELLO
\beginchorus
\textnote{\textbf{Rit.}}
\[D-]Vieni spirito, \[B&]vieni spirito,  \brk \[F]scendi su di no\[C]i
\[D-]Vieni spirito, \[B&]vieni spirito,  \brk \[F]scendi su di no\[C]i.
\[B&]Vieni su \[C]noi \[A]Maranath\[D-]à,  \brk \[B&]vieni su \[C]noi spi-ri\textit{(-to.)}
\[D-]Vieni spirito, \[B&]vieni spirito, \brk  \[F]scendi su di no\[C]i
\[D-]Vieni spirito, \[B&]vieni spirito, \brk  \[F]scendi su di no\[C]i.
Scendi su di no\[D-]i  
\endchorus



%%%%%% EV. INTERMEZZO
\beginverse*
\vspace*{1.3\versesep}
{
	\nolyrics
	\textnote{Intermezzo strumentale}
	
	\ifchorded

	%---- Prima riga -----------------------------
	\vspace*{-\versesep}
     \[B&]  \[F] \[C]	 
	%---- Ogni riga successiva -------------------
	\vspace*{-\versesep}
	\[D-]  \[B&]  \[F] \[C]


	\fi
	%---- Ev Indicazioni -------------------------			
	%\textnote{\textit{(ripetizione della strofa)}} 
	 
}
\vspace*{\versesep}
\endverse


%%%%% STROFA
\beginverse		%Oppure \beginverse* se non si vuole il numero di fianco
%\memorize 		% <<< DECOMMENTA se si vuole utilizzarne la funzione
%\chordsoff		% <<< DECOMMENTA se vuoi una strofa senza accordi

\[D-]Invochiamo la \[B&]tua presenza, \brk \[F]vieni Si\[C]gnor
\[D-]Invochiamo la \[B&]tua presenza, \brk  \[F]scendi su di \[C]noi.
\[G-]Vieni luce dei \[D-]cuori, \brk  dona \[B&]forza e fe\[C]deltà.
\[G-]Fuoco eterno d'\[D-]amore,  \brk questa \[B&]vita offriamo a \[A]te. \[A]

\endverse



%%%%% RITORNELLO
\beginchorus
\textnote{\textbf{Rit.}}
\[D-]Vieni spirito, \[B&]vieni spirito,  \brk \[F]scendi su di no\[C]i
\[D-]Vieni spirito, \[B&]vieni spirito,  \brk \[F]scendi su di no\[C]i.
\[B&]Vieni su \[C]noi \[A]Maranath\[D-]à,  \brk \[B&]vieni su \[C]noi spi-ri\textit{(-to.)}
\[D-]Vieni spirito, \[B&]vieni spirito, \brk  \[F]scendi su di no\[C]i
\[D-]Vieni spirito, \[B&]vieni spirito, \brk  \[F]scendi su di no\[C]i.
Scendi su di \echo{noi}  
\endchorus
%%%%%INTERLUDIO
\beginverse*		
\musicnote{\textit{(dolce, arpeggiato)}}
\[D-]Vieni spirito, \[B&]vieni spirito, \brk  \[F]scendi su di no\[C]i
\[D-]Vieni spirito, \[B&]vieni spirito, \brk  \[F]scendi su di no\[C]i. \[C]
\endverse
\beginchorus
\textnote{(cambio di tonalità)}
\transpose{3}
\[D-]Vieni spirito, \[B&]vieni spirito,  \brk \[F]scendi su di no\[C]i
\[D-]Vieni spirito, \[B&]vieni spirito,  \brk \[F]scendi su di no\[C]i.
\[B&]Vieni su \[C]noi \[A]Maranath\[D-]à,  \brk \[B&]vieni su \[C]noi spi-ri\textit{(-to.)}
\[D-]Vieni spirito, \[B&]vieni spirito, \brk  \[F]scendi su di no\[C]i
\[D-]Vieni spirito, \[B&]vieni spirito, \brk  \[F]scendi su di no\[C]i.
Scendi su di no\[D-]i  
\endchorus



%%%%%% EV. CHIUSURA SOLO STRUMENTALE
\beginverse*
\vspace*{1.3\versesep}
{   
    \ifchorded
	\nolyrics
	\textnote{Chiusura strumentale}
	\transpose{3}
	

	%---- Prima riga -----------------------------
	\vspace*{-\versesep}
     \[B&]  \[F] \[C]	 
	%---- Ogni riga successiva -------------------
	\vspace*{-\versesep}
	\[D-]  \[B&]  \[F] \[C]


	\fi
	%---- Ev Indicazioni -------------------------			
	%\textnote{\textit{(ripetizione della strofa)}} 
	 
}
\vspace*{\versesep}
\endverse



\endsong
%------------------------------------------------------------
%			FINE CANZONE
%------------------------------------------------------------



%-------------------------------------------------------------
%			INIZIO	CANZONE
%-------------------------------------------------------------


%titolo: 	Santo Ricci
%autore: 	Daniele Ricci
%tonalita: 	Sol 



%%%%%% TITOLO E IMPOSTAZONI
\beginsong{Io credo in te}[by={Sermig}] 	% <<< MODIFICA TITOLO E AUTORE
\transpose{0} 						% <<< TRASPOSIZIONE #TONI (0 nullo)
%\preferflats  %SE VOGLIO FORZARE i bemolle come alterazioni
%\prefersharps %SE VOGLIO FORZARE i # come alterazioni
\momenti{Congedo; Pasqua}							% <<< INSERISCI MOMENTI	
% momenti vanno separati da ; e vanno scelti tra:
% Ingresso; Atto penitenziale; Acclamazione al Vangelo; Dopo il Vangelo; Offertorio; Comunione; Ringraziamento; Fine; Santi; Pasqua; Avvento; Natale; Quaresima; Canti Mariani; Battesimo; Prima Comunione; Cresima; Matrimonio; Meditazione; Spezzare del pane;
\ifchorded
	%\textnote{$\bigstar$ Tonalità migliore }	% <<< EV COMMENTI (tonalità originale\migliore)
\fi


%%%%%% INTRODUZIONE
\ifchorded
\vspace*{\versesep}
\musicnote{
\begin{minipage}{0.48\textwidth}
\textbf{Intro}
\hfill 
%( \eighthnote \, 80)   % <<  MODIFICA IL TEMPO
% Metronomo: \eighthnote (ottavo) \quarternote (quarto) \halfnote (due quarti)
\end{minipage}
} 	
\vspace*{-\versesep}
\beginverse*

\nolyrics

%---- Prima riga -----------------------------
\vspace*{-\versesep}
\[A-] \[F] \[C]	\[G]  \[A-]% \[*D] per indicare le pennate, \rep{2} le ripetizioni

%---- Ogni riga successiva -------------------
%\vspace*{-\versesep}
%\[G] \[C]  \[D]	

%---- Ev Indicazioni -------------------------			
%\textnote{\textit{[oppure tutta la strofa]} }	

\endverse
\fi




%%%%% STROFA
\beginverse		%Oppure \beginverse* se non si vuole il numero di fianco
\memorize 		% <<< DECOMMENTA se si vuole utilizzarne la funzione
%\chordsoff		% <<< DECOMMENTA se vuoi una strofa senza accordi

\[A-]Per \[F]molti Tu sei \[C]storia, 
la \[G]pagina di un \[A-]libro 
un \[F]ruolo tea\[C]tra\[G]le. 
\[A-]Le tue \[F]urla, il tuo dol\[C]ore 
\[G]rivivono ogni \[A-]giorno 
ma \[F]non nel nostro \[C]cuo\[G]re.

\endverse




%%%%% RITORNELLO
\beginchorus
\textnote{\textbf{Rit.}}

\[A-]Io \[F]credo in \[C]Te \brk \echo{tu \[G]sei il figlio di \[A-]Dio} 
io \[F]credo in \[C]Te \brk \echo{ Tu \[G]sei risorto e \[A-]vivo} 
Io \[F]credo in \[C]Te \brk \echo{chi \[G]ha l'amore nei suoi \[A-]occhi ti \[F]riconos\[C]cerà.\[G]} \rep{2}

\endchorus



%%%%% STROFA
\beginverse		%Oppure \beginverse* se non si vuole il numero di fianco
%\memorize 		% <<< DECOMMENTA se si vuole utilizzarne la funzione
%\chordsoff		% <<< DECOMMENTA se vuoi una strofa senza accordi

^Senti l'^urlo di chi ^soffre, 
ogni ^sua lacrima ^Tu vedi, 
anche ^quelle ^non ver^sate. 
^Tu as^colti il mio si^lenzio 
^Tu conosci ogni ^stella 
Tu ^sei il Dio ^che ri^sorge 

\endverse


%%%%% RITORNELLO
\beginchorus
\textnote{\textbf{Rit.}}

\[A-]Io \[F]credo in \[C]Te \brk \echo{tu \[G]sei il figlio di \[A-]Dio} 
io \[F]credo in \[C]Te \brk \echo{ Tu \[G]sei risorto e \[A-]vivo} 
Io \[F]credo in \[C]Te \brk \echo{chi \[G]ha l'amore nei suoi \[A-]occhi ti \[F]riconos\[C]cerà.\[G]} \rep{2}

\endchorus





%%%%% BRIDGE
\beginverse*		%Oppure \beginverse* se non si vuole il numero di fianco
%\memorize 		% <<< DECOMMENTA se si vuole utilizzarne la funzione
%\chordsoff		% <<< DECOMMENTA se vuoi una strofa senza accordi
\vspace*{1.3\versesep}
\textnote{\textbf{Bridge}}

\[F]Grazie a Te io posso \[A-]credere in \[G]me         
\[F]Tu mi hai creato per \[G]amare. 

\endverse

\textnote{\textit{(si alza la tonalità)}}


\transpose{2}
%%%%% RITORNELLO
\beginchorus
\textnote{\textbf{Rit.}}

\[A-]Io \[F]credo in \[C]Te \brk \echo{tu \[G]sei il figlio di \[A-]Dio} 
io \[F]credo in \[C]Te \brk \echo{ Tu \[G]sei risorto e \[A-]vivo} 
Io \[F]credo in \[C]Te \brk \echo{chi \[G]ha l'amore nei suoi \[A-]occhi ti \[F]riconos\[C]cerà.\[G]} \rep{2}

\endchorus




%%%%%% EV. FINALE

\beginchorus %oppure \beginverse*
\vspace*{1.3\versesep}
\textnote{\textbf{Finale}} %<<< EV. INDICAZIONI

\[A-]Io \[F]credo in te… \normalfont\textit{(sospeso...)}

\endchorus  %oppure \endverse



\endsong
%------------------------------------------------------------
%			FINE CANZONE
%------------------------------------------------------------




%-------------------------------------------------------------
%			INIZIO	CANZONE
%-------------------------------------------------------------


%titolo: 	Io vedo la Tua luce
%autore: 	Pierangelo Sequeri
%tonalita: 	Do



%%%%%% TITOLO E IMPOSTAZONI
\beginsong{Io vedo la Tua luce}[by={Sybolum 79 — P. Sequeri}] 	% <<< MODIFICA TITOLO E AUTORE
\transpose{0} 						% <<< TRASPOSIZIONE #TONI (0 nullo)
\momenti{Comunione; Ringraziamento; Avvento; Cresima}							% <<< INSERISCI MOMENTI	
% momenti vanno separati da ; e vanno scelti tra:
% Ingresso; Atto penitenziale; Acclamazione al Vangelo; Dopo il Vangelo; Offertorio; Comunione; Ringraziamento; Fine; Santi; Pasqua; Avvento; Natale; Quaresima; Canti Mariani; Battesimo; Prima Comunione; Cresima; Matrimonio; Meditazione; Spezzare del pane;
\ifchorded
	%\textnote{Tonalità migliore }	% <<< EV COMMENTI (tonalità originale/migliore)
\fi

%%%%%% INTRODUZIONE
\ifchorded
\vspace*{\versesep}
\musicnote{
\begin{minipage}{0.48\textwidth}
\textbf{Intro}
\hfill 
%( \eighthnote \, 80)   % <<  MODIFICA IL TEMPO
% Metronomo: \eighthnote (ottavo) \quarternote (quarto) \halfnote (due quarti)
\end{minipage}
} 	
\vspace*{-\versesep}
\beginverse*

\nolyrics

%---- Prima riga -----------------------------
\vspace*{-\versesep}
\[C] \[G] \[C] \[G]	 % \[*D] per indicare le pennate, \rep{2} le ripetizioni

%---- Ogni riga successiva -------------------
%\vspace*{-\versesep}
%\[G] \[C]  \[D]	

%---- Ev Indicazioni -------------------------			
%\textnote{\textit{(Oppure tutta la strofa)} }	

\endverse
\fi

%%%%% STROFA
\beginverse		%Oppure \beginverse* se non si vuole il numero di fianco
\memorize 		% <<< DECOMMENTA se si vuole utilizzarne la funzione
%\chordsoff		% <<< DECOMMENTA se vuoi una strofa senza accordi

\[C]Tu sei prima \[G]d’ogni cosa, \[C]prima d’ogni \[G]tempo,
d’ogni \[C]mio pensiero: \[G]prima \[C]della \[G]vita.

\endverse

%%%%% STROFA
\beginverse*	%Oppure \beginverse* se non si vuole il numero di fianco
%\memorize 		% <<< DECOMMENTA se si vuole utilizzarne la funzione
\chordsoff		% <<< DECOMMENTA se vuoi una strofa senza accordi

Una voce udimmo che gridava nel deserto
preparate la venuta del Signore.

\endverse

%%%%% STROFA
\beginverse*		%Oppure \beginverse* se non si vuole il numero di fianco
%\memorize 		% <<< DECOMMENTA se si vuole utilizzarne la funzione
\chordsoff		% <<< DECOMMENTA se vuoi una strofa senza accordi

Tu sei la Parola eterna della quale vivo
che mi pronunciò soltanto per amore.

\endverse

%%%%% STROFA
\beginverse*	%Oppure \beginverse* se non si vuole il numero di fianco
%\memorize 		% <<< DECOMMENTA se si vuole utilizzarne la funzione
\chordsoff		% <<< DECOMMENTA se vuoi una strofa senza accordi

E ti abbiamo udito predicare sulle strade
della nostra incomprensione senza fine. 

\endverse

%%%%% RITORNELLO
\beginchorus
\textnote{\textbf{Rit.}}

\[C]Io \[E7]ora so chi \[A-7]sei \[C7]
\[F]io \[D-]sento la tua \[E-]voce \[E-]
\[F]io \[A-]vedo la tua \[E-]luce \[C]
\[D-]io so che tu sei \[G]qui. \[G7]
\[C]E \[E7]sulla tua pa\[A-7]rola \[C7]
\[F]io \[D-]credo nell’a\[E-]more \[E-]
\[F]Io \[A-]vivo nella \[E-]pace \[C]
\[F]io \[G]so che torne\[C]rai.  \[G] \[C] \[G]

\endchorus

%%%%% STROFA
\beginverse		%Oppure \beginverse* se non si vuole il numero di fianco
%\memorize 		% <<< DECOMMENTA se si vuole utilizzarne la funzione
%\chordsoff		% <<< DECOMMENTA se vuoi una strofa senza accordi

^Tu sei l’appa^rire dell’imm^ensa tene^rezza
di un ^Amore che nes^suno ha visto ^mai.^

\endverse

%%%%% STROFA
\beginverse*		%Oppure \beginverse* se non si vuole il numero di fianco
%\memorize 		% <<< DECOMMENTA se si vuole utilizzarne la funzione
\chordsoff		% <<< DECOMMENTA se vuoi una strofa senza accordi

Ci fu dato il lieto annuncio della tua venuta
noi abbiamo visto un uomo come noi.

\endverse

%%%%% STROFA
\beginverse*		%Oppure \beginverse* se non si vuole il numero di fianco
%\memorize 		% <<< DECOMMENTA se si vuole utilizzarne la funzione
\chordsoff		% <<< DECOMMENTA se vuoi una strofa senza accordi

Tu sei verità che non tramonta,sei la vita 
che non muore,sei la via di un mondo nuovo.

\endverse

%%%%% STROFA
\beginverse*		%Oppure \beginverse* se non si vuole il numero di fianco
%\memorize 		% <<< DECOMMENTA se si vuole utilizzarne la funzione
\chordsoff		% <<< DECOMMENTA se vuoi una strofa senza accordi

E ti abbiamo visto stabilire la tua tenda
tra la nostra indifferenza d’ogni giorno.

\endverse


%%%%%% EV. CHIUSURA SOLO STRUMENTALE
\ifchorded
\beginchorus %oppure \beginverse*
\vspace*{1.3\versesep}
\textnote{Chiusura } %<<< EV. INDICAZIONI

\[C] \[G*] \textnote{[sospeso...]}

\endchorus  %oppure \endverse
\fi

\endsong
%------------------------------------------------------------
%			FINE CANZONE
%------------------------------------------------------------


% %++++++++++++++++++++++++++++++++++++++++++++++++++++++++++++
% %			CANZONE TRASPOSTA
% %++++++++++++++++++++++++++++++++++++++++++++++++++++++++++++
% \ifchorded
% %decremento contatore per avere stesso numero
% \addtocounter{songnum}{-1} 
% \beginsong{Io vedo la Tua Luce}[by={Pierangelo Sequeri}] 	% <<< COPIA TITOLO E AUTORE
% \transpose{+2} 						% <<< TRASPOSIZIONE #TONI + - (0 nullo)
% %\preferflats  %SE VOGLIO FORZARE i bemolle come alterazioni
% %\prefersharps %SE VOGLIO FORZARE i # come alterazioni
% \ifchorded
% 	%\textnote{Tonalità originale}	% <<< EV COMMENTI (tonalità originale/migliore)
% \fi


% %%%%%% INTRODUZIONE
% \ifchorded
% \vspace*{\versesep}
% \textnote{Intro: \qquad \qquad  }%(\eighthnote 116) % <<  MODIFICA IL TEMPO
% % Metronomo: \eighthnote (ottavo) \quarternote (quarto) \halfnote (due quarti)
% \vspace*{-\versesep}
% \beginverse*

% \nolyrics

% %---- Prima riga -----------------------------
% \vspace*{-\versesep}
% \[C] \[G] \[C] \[G]	 % \[*D] per indicare le pennate, \rep{2} le ripetizioni

% %---- Ogni riga successiva -------------------
% %\vspace*{-\versesep}
% %\[G] \[C]  \[D]	

% %---- Ev Indicazioni -------------------------			
% %\textnote{\textit{(Oppure tutta la strofa)} }	

% \endverse
% \fi

% %%%%% STROFA
% \beginverse		%Oppure \beginverse* se non si vuole il numero di fianco
% \memorize 		% <<< DECOMMENTA se si vuole utilizzarne la funzione
% %\chordsoff		% <<< DECOMMENTA se vuoi una strofa senza accordi

% \[C]Tu sei prima \[G]d’ogni cosa, \[C]prima d’ogni \[G]tempo,
% d’ogni \[C]mio pensiero: \[G]prima \[C]della \[G]vita.

% \endverse

% %%%%% STROFA
% \beginverse		%Oppure \beginverse* se non si vuole il numero di fianco
% %\memorize 		% <<< DECOMMENTA se si vuole utilizzarne la funzione
% %\chordsoff		% <<< DECOMMENTA se vuoi una strofa senza accordi

% Una voce udimmo che gridava nel deserto
% preparate la venuta del Signore.

% \endverse

% %%%%% STROFA
% \beginverse		%Oppure \beginverse* se non si vuole il numero di fianco
% %\memorize 		% <<< DECOMMENTA se si vuole utilizzarne la funzione
% %\chordsoff		% <<< DECOMMENTA se vuoi una strofa senza accordi

% Tu sei la Parola eterna della quale vivo
% che mi pronunciò soltanto per amore.

% \endverse

% %%%%% STROFA
% \beginverse		%Oppure \beginverse* se non si vuole il numero di fianco
% %\memorize 		% <<< DECOMMENTA se si vuole utilizzarne la funzione
% %\chordsoff		% <<< DECOMMENTA se vuoi una strofa senza accordi

% E ti abbiamo udito predicare sulle strade
% della nostra incomprensione senza fine. 

% \endverse

% %%%%% RITORNELLO
% \beginchorus
% \textnote{\textbf{Rit.}}

% \[C]Io \[E7]ora so chi \[A-7]sei
% \[C7]io \[F]sento \[D-]la tua \[E-]voce
% \[F]io \[A-]vedo la tua \[E-]luce \[C]
% \[D-]io \[G]so che tu sei \[G7]qui.
% \[C]E \[E7]sulla tua pa\[A-7]rola \[C7]
% \[F]io \[D-]credo nell’a\[E-]more
% \[F]Io \[A-]vivo nella \[E-]pace \[C]
% \[F]io \[G]so che torne\[C]rai.

% \endchorus

% %%%%% STROFA
% \beginverse		%Oppure \beginverse* se non si vuole il numero di fianco
% %\memorize 		% <<< DECOMMENTA se si vuole utilizzarne la funzione
% %\chordsoff		% <<< DECOMMENTA se vuoi una strofa senza accordi

% Tu sei l’apparire dell’immensa tenerezza
% di un Amore che nessuno ha visto mai.

% \endverse

% %%%%% STROFA
% \beginverse		%Oppure \beginverse* se non si vuole il numero di fianco
% %\memorize 		% <<< DECOMMENTA se si vuole utilizzarne la funzione
% %\chordsoff		% <<< DECOMMENTA se vuoi una strofa senza accordi

% Ci fu dato il lieto annuncio della tua venuta
% noi abbiamo visto un uomo come noi.

% \endverse

% %%%%% STROFA
% \beginverse		%Oppure \beginverse* se non si vuole il numero di fianco
% %\memorize 		% <<< DECOMMENTA se si vuole utilizzarne la funzione
% %\chordsoff		% <<< DECOMMENTA se vuoi una strofa senza accordi

% Tu sei verità che non tramonta,sei la vita 
% che non muore,sei la via di un mondo nuovo.

% \endverse

% %%%%% STROFA
% \beginverse		%Oppure \beginverse* se non si vuole il numero di fianco
% %\memorize 		% <<< DECOMMENTA se si vuole utilizzarne la funzione
% %\chordsoff		% <<< DECOMMENTA se vuoi una strofa senza accordi

% E ti abbiamo visto stabilire la tua tenda
% tra la nostra indifferenza d’ogni giorno.

% \endverse

% \endsong


% \fi
% %++++++++++++++++++++++++++++++++++++++++++++++++++++++++++++
% %			FINE CANZONE TRASPOSTA
% %++++++++++++++++++++++++++++++++++++++++++++++++++++++++++++
%-------------------------------------------------------------
%			INIZIO	CANZONE
%-------------------------------------------------------------


%titolo: 	Isaia 11
%autore: 	C. Rossi, S. Carocci
%tonalita: 	Do 
%youtube: https://www.youtube.com/watch?v=jf9DEnsybLc&feature=youtu.be


%%%%%% TITOLO E IMPOSTAZONI
\beginsong{Isaia 11}[by={C. Rossi, S. Carocci}] 	% <<< MODIFICA TITOLO E AUTORE
\transpose{0} 						% <<< TRASPOSIZIONE #TONI (0 nullo)
%\preferflats  %SE VOGLIO FORZARE i bemolle come alterazioni
%\prefersharps %SE VOGLIO FORZARE i # come alterazioni
\momenti{}							% <<< INSERISCI MOMENTI	
% momenti vanno separati da ; e vanno scelti tra:
% Ingresso; Atto penitenziale; Acclamazione al Vangelo; Dopo il Vangelo; Offertorio; Comunione; Ringraziamento; Fine; Santi; Pasqua; Avvento; Natale; Quaresima; Canti Mariani; Battesimo; Prima Comunione; Cresima; Matrimonio; Meditazione; Spezzare del pane;
\ifchorded
	%\textnote{Tonalità migliore }	% <<< EV COMMENTI (tonalità originale/migliore)
\fi


%%%%%% INTRODUZIONE
\ifchorded
\vspace*{\versesep}
\textnote{Intro: \qquad \qquad  }%(\eighthnote 116) % <<  MODIFICA IL TEMPO
% Metronomo: \eighthnote (ottavo) \quarternote (quarto) \halfnote (due quarti)
\vspace*{-\versesep}
\beginverse*

\nolyrics

%---- Prima riga -----------------------------
\vspace*{-\versesep}
\[C] \[G] \[A-7] \[G]	 % \[*D] per indicare le pennate, \rep{2} le ripetizioni

%---- Ogni riga successiva -------------------
\vspace*{-\versesep}
\[F]  \[F] \[E-7]  \[E7]	

%---- Ev Indicazioni -------------------------			
%\textnote{\textit{(Oppure tutta la strofa)} }	

\endverse
\fi



%%%%% RITORNELLO
\beginchorus
\textnote{\textbf{Rit.}}

\[C]Ed un vir\[G]gulto dal \[A-7]tronco di \[G]Iesse
do\[F]mani germo-oglie\[E-7]rà. \[E7]
\[C]Un ramo\[G]scello dal\[A-7]le sue ra\[G]dici
a ves\[F]sillo si \[E-]eleve\[A-]rà. 

\endchorus



%%%%% STROFA
\beginverse		%Oppure \beginverse* se non si vuole il numero di fianco
\memorize 		% <<< DECOMMENTA se si vuole utilizzarne la funzione
%\chordsoff		% <<< DECOMMENTA se vuoi una strofa senza accordi

\[D-7]Su lui sapienza, intel\[E-7]letto, consiglio,
for\[D-7]tezza e timor del Si\[E-7]gnor. \[E7]
\[D-7]La sua parola sa\[E-7]rà come verga
e dal \[D-7]male ci libere\[E-7]rà.\[E7] 

\endverse



%%%%% STROFA
\beginverse		%Oppure \beginverse* se non si vuole il numero di fianco
%\memorize 		% <<< DECOMMENTA se si vuole utilizzarne la funzione
%\chordsoff		% <<< DECOMMENTA se vuoi una strofa senza accordi

^L'agnello e il lupo ins^ieme staranno
e ^accanto al capretto viv^ran. ^
^Pascoleranno con ^l'orsa e il leone
un fan^ciullo li guide^rà.^

\endverse


%%%%% STROFA
\beginverse		%Oppure \beginverse* se non si vuole il numero di fianco
%\memorize 		% <<< DECOMMENTA se si vuole utilizzarne la funzione
%\chordsoff		% <<< DECOMMENTA se vuoi una strofa senza accordi

^Ed in quel giorno di ^nuovo il Signore
la ^mano su lui stende^rà. ^
^Come vessillo il ger^moglio di Iesse
sui ^popoli si eleve^rà.^

\endverse





%%%%%% EV. CHIUSURA SOLO STRUMENTALE
\ifchorded
\beginchorus %oppure \beginverse*
\vspace*{1.3\versesep}
\textnote{Chiusura } %<<< EV. INDICAZIONI

\[A-]

\endchorus  %oppure \endverse
\fi


\endsong
%------------------------------------------------------------
%			FINE CANZONE
%------------------------------------------------------------



%JJJ
%-------------------------------------------------------------
%			INIZIO	CANZONE
%-------------------------------------------------------------


%titolo: 	Jesus Christ you are my life
%autore: 	Frisina
%tonalita: 	Re 



%%%%%% TITOLO E IMPOSTAZONI
\beginsong{Jesus Christ you are my life}[by={M. Frisina}] 	% <<< MODIFICA TITOLO E AUTORE
\transpose{0} 						% <<< TRASPOSIZIONE #TONI (0 nullo)
\momenti{Fine; Ringraziamento}							% <<< INSERISCI MOMENTI	
% momenti vanno separati da ; e vanno scelti tra:
% Ingresso; Atto penitenziale; Acclamazione al Vangelo; Dopo il Vangelo; Offertorio; Comunione; Ringraziamento; Fine; Santi; Pasqua; Avvento; Natale; Quaresima; Canti Mariani; Battesimo; Prima Comunione; Cresima; Matrimonio; Meditazione; Spezzare del pane;
\ifchorded
	%\textnote{Tonalità originale }	% <<< EV COMMENTI (tonalità originale/migliore)
\fi

%%%%%% INTRODUZIONE
\ifchorded
\vspace*{\versesep}
\musicnote{
\begin{minipage}{0.48\textwidth}
\textbf{Intro}
\hfill 
(\quarternote \, 72)
%( \eighthnote \, 80)   % <<  MODIFICA IL TEMPO
% Metronomo: \eighthnote (ottavo) \quarternote (quarto) \halfnote (due quarti)
\end{minipage}
} 	
\vspace*{-\versesep}
\beginverse*


\nolyrics

%---- Prima riga -----------------------------
\vspace*{-\versesep}
\[D] \[A] \[B-] \[D]	 % \[*D] per indicare le pennate, \rep{2} le ripetizioni

%---- Ogni riga successiva -------------------
\vspace*{-\versesep}
\[G] \[D]  \[A] \[D]

%---- Ev Indicazioni -------------------------			
\textnote{\textit{[come la seconda metà del ritornello]} }	

\endverse
\fi




%%%%% STROFA
\beginverse*		%Oppure \beginverse* se non si vuole il numero di fianco
%\memorize 		% <<< DECOMMENTA se si vuole utilizzarne la funzione
%\chordsoff		% <<< DECOMMENTA se vuoi una strofa senza accordi

\[D]Cristo \[A]vive \[G]in mezzo a \[D]noi,
\[E-]allel\[B-]uja, alle\[E-]lu\[A]ja.
\[D]Cristo \[A]vive \[B-]in mezzo a \[D]noi,
\[G]in mezzo a \[D]noi, al\[A]lelu\[D]ja.

\endverse




%%%%% RITORNELLO
\beginchorus
\textnote{\textbf{Rit.}}

\[D]Jesus Chr\[A]ist \[G]you are my \[D]life,
\[E-]allel\[B-]uja, alle\[E-]lu\[A]ja.
\[D]Jesus Ch\[A]rist \[B-]you are my l\[D]ife,
\[G]you are my li\[D]fe, al\[A]lelu\[D]ja.

\endchorus





%%%%% STROFA
\beginverse		%Oppure \beginverse* se non si vuole il numero di fianco
\memorize 		% <<< DECOMMENTA se si vuole utilizzarne la funzione
%\chordsoff		% <<< DECOMMENTA se vuoi una strofa senza accordi

\[F#]Tu sei \[B-]via, \[F#]sei veri\[B-]tà, 
\[G]Tu sei la n\[D]ostra \[E-]v\[(D*)]i\[A]ta,
\[F#]cammi\[B-]nando \[G]insieme a \[D]Te 
\[G]vivremo in \[D]Te, per \[G]sem\[A]pre.

\endverse




%%%%% RITORNELLO
\beginchorus
\textnote{\textbf{Rit.}}

\[D]Jesus Chr\[A]ist \[G]you are my \[D]life,
\[E-]allel\[B-]uja, alle\[E-]lu\[A]ja.
\[D]Jesus Ch\[A]rist \[B-]you are my l\[D]ife,
\[G]you are my li\[D]fe, al\[A]lelu\[D]ja.

\endchorus

%%%%% STROFA
\beginverse		%Oppure \beginverse* se non si vuole il numero di fianco
%\memorize 		% <<< DECOMMENTA se si vuole utilizzarne la funzione
%\chordsoff		% <<< DECOMMENTA se vuoi una strofa senza accordi

^Ci rac^cogli ^nell'uni^tà, 
^riuniti ^nell'a^m^o^re,
^nella ^gioia ^dinanzi a ^Te 
^cantando ^la Tua ^glo^ria.

\endverse


%%%%% RITORNELLO
\beginchorus
\textnote{\textbf{Rit.}}

\[D]Jesus Chr\[A]ist \[G]you are my \[D]life,
\[E-]allel\[B-]uja, alle\[E-]lu\[A]ja.
\[D]Jesus Ch\[A]rist \[B-]you are my l\[D]ife,
\[G]you are my li\[D]fe, al\[A]lelu\[D]ja.

\endchorus


%%%%% STROFA
\beginverse		%Oppure \beginverse* se non si vuole il numero di fianco
%\memorize 		% <<< DECOMMENTA se si vuole utilizzarne la funzione
%\chordsoff		% <<< DECOMMENTA se vuoi una strofa senza accordi

^Nella ^gioia ^cammine^rem, 
^portando il ^Tuo Van^g^e^lo,
^testi^moni ^di cari^tà, 
^figli di ^Dio nel ^mon^do.

\endverse


%%%%% RITORNELLO
\beginchorus
\textnote{\textbf{Rit.}}

\[D]Jesus Chr\[A]ist \[G]you are my \[D]life,
\[E-]allel\[B-]uja, alle\[E-]lu\[A]ja.
\[D]Jesus Ch\[A]rist \[B-]you are my l\[D]ife,
\[G]you are my li\[D]fe, al\[A]lelu\[D]ja. \rep{3}

\endchorus





\endsong
%------------------------------------------------------------
%			FINE CANZONE
%------------------------------------------------------------

%KKK
%LLL
%-------------------------------------------------------------
%			INIZIO	CANZONE
%-------------------------------------------------------------


%titolo: 	La stella polare (Tu al centro del mio cuore)
%autore: 	Gen Verde
%tonalita: 	Sol 



%%%%%% TITOLO E IMPOSTAZONI
\beginsong{La stella polare}[ititle={Tu al centro del mio cuore }, by={ Tu al centro del mio cuore — Gen Verde}] 	% <<< MODIFICA TITOLO E AUTORE
\transpose{0} 						% <<< TRASPOSIZIONE #TONI (0 nullo)
\momenti{Ringraziamento}							% <<< INSERISCI MOMENTI	
% momenti vanno separati da ; e vanno scelti tra:
% Ingresso; Atto penitenziale; Acclamazione al Vangelo; Dopo il Vangelo; Offertorio; Comunione; Ringraziamento; Fine; Santi; Pasqua; Avvento; Natale; Quaresima; Canti Mariani; Battesimo; Prima Comunione; Cresima; Matrimonio; Meditazione; Spezzare del pane;
\ifchorded
	%\textnote{Tonalità originale }	% <<< EV COMMENTI (tonalità originale/migliore)
\fi

%%%%%% INTRODUZIONE
\ifchorded
\vspace*{\versesep}
\musicnote{
\begin{minipage}{0.48\textwidth}
\textbf{Intro}
\hfill 
%( \eighthnote \, 80)   % <<  MODIFICA IL TEMPO
% Metronomo: \eighthnote (ottavo) \quarternote (quarto) \halfnote (due quarti)
\end{minipage}
} 	
\vspace*{-\versesep}
\beginverse*

\nolyrics

%---- Prima riga -----------------------------
\vspace*{-\versesep}
\[E-] \[G] \[B-] \[G]	 % \[*D] per indicare le pennate, \rep{2} le ripetizioni

%---- Ogni riga successiva -------------------
%\vspace*{-\versesep}
%\[G] \[C]  \[D]	

%---- Ev Indicazioni -------------------------			
%\textnote{\textit{(Oppure tutta la strofa)} }	

\endverse
\fi

%%%%% STROFA
\beginverse		%Oppure \beginverse* se non si vuole il numero di fianco
\memorize 		% <<< DECOMMENTA se si vuole utilizzarne la funzione
%\chordsoff		% <<< DECOMMENTA se vuoi una strofa senza accordi

\[E-]Ho bisogno di incontrarti nel mio \[G]cuore
\[B-]di trovare Te di stare insieme a \[C]Te
\[A-]unico riferimento del mio an\[E-]dare 
\[C]unica ragione \[D]Tu, \[B-]unico sostegno \[E-]Tu
\[C]al centro del mio cuore \[D]ci sei solo \[G]Tu 

\endverse

%%%%% STROFA
\beginverse*		%Oppure \beginverse* se non si vuole il numero di fianco
\memorize 		% <<< DECOMMENTA se si vuole utilizzarne la funzione
%\chordsoff		% <<< DECOMMENTA se vuoi una strofa senza accordi

\[E-]Anche il cielo gira intorno e non ha \[G]pace
\[B-]ma c'è un punto fermo è quella stella \[C]là
\[A-]la stella polare è fissa ed è la \[E-]sola 
\[C]la stella polare \[D]Tu , \[B-]la stella sicura \[E-]Tu
\[C]al centro del mio cuore \[D]ci sei solo \[G]Tu

\endverse

%%%%% RITORNELLO
\beginchorus
\textnote{\textbf{Rit.}}

Tutto \[B-]ruota intorno a \[C]Te , in funzione di \[B-]Te-\[E-]e
e poi \[B-]non importa il \[C]come, il dove , il \[D]se.

\endchorus

%%%%% STROFA
\beginverse		%Oppure \beginverse* se non si vuole il numero di fianco
%\memorize 		% <<< DECOMMENTA se si vuole utilizzarne la funzione
%\chordsoff		% <<< DECOMMENTA se vuoi una strofa senza accordi

\[E-]Che Tu splenda sempre al centro del mio \[G]cuore
\[B-]il significato allora sarai \[C]Tu
\[A-]quello che farò sarà soltanto \[E-]amore
\[C]unica ragione \[D]Tu , \[B-]la stella polare \[E-]Tu
\[C]al centro del mio cuore \[D]ci sei solo \[G]Tu.

\endverse

%%%%% RITORNELLO
\beginchorus
\textnote{\textbf{Rit.}}

Tutto \[B-]ruota intorno a \[C]Te , in funzione di \[B-]Te-\[E-]e
e poi \[B-]non importa il \[C]come, il dove , il \[D]se.

\endchorus

%%%%%% EV. FINALE

\beginverse* %oppure \beginchorus
\vspace*{1.3\versesep}
\textnote{\textbf{Finale} \textit{(humming)}} %<<< EV. INDICAZIONI

\[E-]Uhmmm... \[G] \[B-] \[G]

\endverse  %oppure \endchorus




\endsong
%------------------------------------------------------------
%			FINE CANZONE
%------------------------------------------------------------
%-------------------------------------------------------------
%			INIZIO	CANZONE
%-------------------------------------------------------------


%titolo: 	Laudato sii, Signore mio
%autore: 	Cento
%tonalita: 	Mi 



%%%%%% TITOLO E IMPOSTAZONI
\beginsong{Laudato sii, Signore mio}[by={Il canto della creazione — G. Cento}]	% <<< MODIFICA TITOLO E AUTORE
\transpose{-2} 						% <<< TRASPOSIZIONE #TONI (0 nullo)
\momenti{Santi}							% <<< INSERISCI MOMENTI	
% momenti vanno separati da ; e vanno scelti tra:
% Ingresso; Atto penitenziale; Acclamazione al Vangelo; Dopo il Vangelo; Offertorio; Comunione; Ringraziamento; Fine; Santi; Pasqua; Avvento; Natale; Quaresima; Canti Mariani; Battesimo; Prima Comunione; Cresima; Matrimonio; Meditazione; Spezzare del pane;
\ifchorded
	%\textnote{Tonalità migliore }	% <<< EV COMMENTI (tonalità originale/migliore)
\fi


%%%%%% INTRODUZIONE
\ifchorded
\vspace*{\versesep}
\musicnote{
\begin{minipage}{0.48\textwidth}
\textbf{Intro}
\hfill 
%( \eighthnote \, 80)   % <<  MODIFICA IL TEMPO
% Metronomo: \eighthnote (ottavo) \quarternote (quarto) \halfnote (due quarti)
\end{minipage}
} 	
\vspace*{-\versesep}
\beginverse*

\nolyrics

%---- Prima riga -----------------------------
\vspace*{-\versesep}
\[E]	 % \[*D] per indicare le pennate, \rep{2} le ripetizioni

%---- Ogni riga successiva -------------------
%\vspace*{-\versesep}
%\[G] \[C]  \[D]	

%---- Ev Indicazioni -------------------------			
%\textnote{\textit{(Oppure tutta la strofa)} }	

\endverse
\fi


\beginchorus
\[E]Laudato sii Signore \[F#-]mio 
\[B] Laudato sii Signore \[C#-]mio 
\[A] Laudato sii Signore \[B]mio 
\[A] Laudato \[F#-*]sii Si\[B*]gnore \[E]mio.
\endchorus
\beginverse
\memorize
\[E] Per il sole d'ogni \[F#-]giorno, 
\[B] che riscalda e dona \[C#-]vita. 
\[A] Egli illumina il cam\[B]mino
\[A] di chi \[F#-*]cerca te Si\[B7]gnore. 
\[E] Per la luna e per le \[F#-]stelle, 
\[B] io le sento mie so\[C#-]relle 
\[A] le hai formate su nel \[B]cielo 
\[A] e le \[F#-]doni a chi è nel \[B]buio. 
\endverse
\beginverse

^ Per la nostra madre ^terra,  
^ che ci dona fiori ed ^erba, 
^ su di lei noi fati^chiamo,  
^ per il ^pane d'ogni ^giorno. 
^ Per chi soffre con co^raggio,  
^ e perdona nel tuo a^more, 
^ Tu gli dai la pace ^tua,  
^ alla ^sera della ^vita.
\endverse
\beginverse

^ Per la morte che è di ^tutti,  
^ io la sento ogni i^stante, 
^ ma se vivo nel tuo a^more, 
^ dona un ^senso alla mia ^vita. 
^ Per l'amore che è nel ^mondo,  
^ tra una donna e l'uomo ^suo, 
^ per la vita dei bam^bini  
^ che il mio ^mondo fanno ^nuovo.
\endverse
\beginverse

^ Io ti canto mio Si^gnore  
^ e con me la crea^zione 
^ ti ringrazia umil^mente  
^ perché ^tu sei il Si^gnore. ^\rep{2}
\endverse


\endsong
%------------------------------------------------------------
%			FINE CANZONE
%------------------------------------------------------------

%titolo{Le tue meraviglie}
%autore{Casucci, Balduzzi}
%album{Verbum Panis}
%tonalita{La-}
%famiglia{Liturgica}
%gruppo{}
%momenti{Congedo;Natale}
%identificatore{le_tue_meraviglie}
%data_revisione{2011_12_31}
%trascrittore{Francesco Endrici - Manuel Toniato}
\beginsong{Le tue meraviglie}[by={Casucci, Balduzzi}]

\ifchorded
\beginverse*
\vspace*{-0.8\versesep}
{\nolyrics \[A-] \[E-] \[F] \[C] \[D-] \[A-] \[F] \[G] 
\[A-] \[E-] \[F] \[C] \[D-] \[A-] \[F] \[G] \[A-] }
\vspace*{-\versesep}
\endverse
\fi

\beginchorus
Ora \[F]lascia, o Si\[G]gnore, che io \[E-]vada in pa\[A-]ce,
perché ho \[D-]visto le tue \[C]mera\[B&]vi\[G]glie.
Il tuo \[F]popolo in \[G]festa per le \[E-]strade corre\[A-]rà
a por\[D-]tare le tue \[C]mera\[B&]vi\[G]glie!
\endchorus

\beginverse
\[A-]La tua pre\[E-]senza ha riem\[F]pito d'a\[C]more
\[A-]le nostre \[E-]vite, le \[F]nostre gior\[C]nate.
\[B&]In te una sola \[F]anima, \[G-]un solo cuore \[F]siamo noi:
\[B&]con te la luce ri\[F]splende, \brk \[G-]splende più chiara che \[C]mai!
\endverse

\beginverse
\chordsoff
La tua presenza ha inondato d'amore
le nostre vite, le nostre giornate,
fra la tua gente resterai, \brk per sempre vivo in mezzo a noi
fino ai confini del tempo: così ci accompagnerai.
\endverse

\beginchorus
Ora \[F]lascia, o Si\[G]gnore, che io \[E-]vada in pa\[A-]ce,
perché ho \[D-]visto le tue \[C]mera\[B&]vi\[G]glie.
Il tuo \[F]popolo in \[G]festa per le \[E-]strade corre\[A-]rà
a por\[D-]tare le tue \[C]mera\[B&]vi\[G]glie!
Ora \[F]lascia, o Si\[G]gnore, che io \[E-]vada in pa\[A-]ce,
perché ho \[F]visto le \[G]tue mera\[E-]vi\[A-]glie.
E il tuo \[F]popolo in \[G]festa per le \[E-]strade corre\[A-]rà
a por\[F]tare le \[G]tue mera\[F]vi\[C]glie!
\endchorus



%%%%%% FINALE
\ifchorded
\beginverse*
\vspace*{-0.5\versesep}
{
	\nolyrics
	\textbf{Finale:} \quad	
	\[A-] \[E-] \[F] \[C] \[D-] \[A-] \[F] \[G] 
	\[A-] \[E-] \[F] \[C] \[D-] \[A-] \[F] \[G] \[A-]
	 
}
\vspace*{\versesep}
\endverse
\fi





\endsong
%------------------------------------------------------------
%			FINE CANZONE
%------------------------------------------------------------
%-------------------------------------------------------------
%			INIZIO	CANZONE
%-------------------------------------------------------------


%titolo: 	Lo Spirito di Cristo
%autore: 	
%tonalita: 	Mi



%%%%%% TITOLO E IMPOSTAZONI
\beginsong{Lo Spirito di Cristo}[by={}] 	% <<< MODIFICA TITOLO E AUTORE
\transpose{0} 						% <<< TRASPOSIZIONE #TONI (0 nullo)
\momenti{Ingresso; Cresima}							% <<< INSERISCI MOMENTI	
% momenti vanno separati da ; e vanno scelti tra:
% Ingresso; Atto penitenziale; Acclamazione al Vangelo; Dopo il Vangelo; Offertorio; Comunione; Ringraziamento; Fine; Santi; Pasqua; Avvento; Natale; Quaresima; Canti Mariani; Battesimo; Prima Comunione; Cresima; Matrimonio; Meditazione; Spezzare del pane;
\ifchorded
	%\textnote{Tonalità originale }	% <<< EV COMMENTI (tonalità originale/migliore)
\fi

%%%%%% INTRODUZIONE
\ifchorded
\vspace*{\versesep}
\textnote{Intro: \qquad \qquad  }%(\eighthnote 116) % <<  MODIFICA IL TEMPO
% Metronomo: \eighthnote (ottavo) \quarternote (quarto) \halfnote (due quarti)
\vspace*{-\versesep}
\beginverse*

\nolyrics

%---- Prima riga -----------------------------
\vspace*{-\versesep}
\[E] \[B] \[A] \[E]	 % \[*D] per indicare le pennate, \rep{2} le ripetizioni

%---- Ogni riga successiva -------------------
%\vspace*{-\versesep}
%\[G] \[C]  \[D]	

%---- Ev Indicazioni -------------------------			
%\textnote{\textit{(Oppure tutta la strofa)} }	

\endverse
\fi

%%%%% RITORNELLO
\beginchorus
\textnote{\textbf{Rit.}}

\[E]Lo \[B]Spirito di \[A]Cristo 
\[E]fa fio\[B]rire il de\[C#-]serto, 
\[G#-] torna la \[C#-]vita, 
\[A] noi diven\[B]tiamo testi\[E]moni di \[B]luce.

\endchorus

%%%%% STROFA
\beginverse		%Oppure \beginverse* se non si vuole il numero di fianco
\memorize 		% <<< DECOMMENTA se si vuole utilizzarne la funzione
%\chordsoff		% <<< DECOMMENTA se vuoi una strofa senza accordi

\[E] Non abbiamo rice\[B]vuto 
\[A] uno spirito di \[B]schiavitù,
\[E] ma uno spirito di a\[B]more, 
\[A] uno spirito di \[B]pace,
\[A] nel quale gri\[B]diamo 
\[A] abbà \[E]Padre,
\[A] abbà \[C#-]Pa\[B]dre.

\endverse

%%%%% STROFA
\beginverse		%Oppure \beginverse* se non si vuole il numero di fianco
%\memorize 		% <<< DECOMMENTA se si vuole utilizzarne la funzione
%\chordsoff		% <<< DECOMMENTA se vuoi una strofa senza accordi

^ Lo Spirito ^che 
^ Cristo ri^suscitò
^ darà vita ai nostri ^corpi, 
^ corpi mor^tali,  
^ e li rende^rà 
^ strumenti di sal^vezza, 
^ strumenti di sal^vez^za.

\endverse

%%%%% STROFA
\beginverse		%Oppure \beginverse* se non si vuole il numero di fianco
%\memorize 		% <<< DECOMMENTA se si vuole utilizzarne la funzione
\chordsoff		% <<< DECOMMENTA se vuoi una strofa senza accordi

Sono venuto a portare
il fuoco sulla terra
e come desidero
che divampi nel mondo 
e porti amore
ed entusiasmo in tutti i cuori.

\endverse

\endsong
%------------------------------------------------------------
%			FINE CANZONE
%------------------------------------------------------------

<<<<<<< HEAD
%++++++++++++++++++++++++++++++++++++++++++++++++++++++++++++
%			CANZONE TRASPOSTA
%++++++++++++++++++++++++++++++++++++++++++++++++++++++++++++
\ifchorded
%decremento contatore per avere stesso numero
\addtocounter{songnum}{-1} 
\beginsong{Lo Spirito di Cristo}[by={}] 	% <<< COPIA TITOLO E AUTORE
\transpose{3} 						% <<< TRASPOSIZIONE #TONI + - (0 nullo)
%\preferflats  %SE VOGLIO FORZARE i bemolle come alterazioni
%\prefersharps %SE VOGLIO FORZARE i # come alterazioni
\ifchorded
	%\textnote{Tonalità originale}	% <<< EV COMMENTI (tonalità originale/migliore)
\fi

%%%%%% INTRODUZIONE
\ifchorded
\vspace*{\versesep}
\textnote{Intro: \qquad \qquad  }%(\eighthnote 116) % <<  MODIFICA IL TEMPO
% Metronomo: \eighthnote (ottavo) \quarternote (quarto) \halfnote (due quarti)
\vspace*{-\versesep}
\beginverse*

\nolyrics

%---- Prima riga -----------------------------
\vspace*{-\versesep}
\[E] \[B] \[A] \[E]	 % \[*D] per indicare le pennate, \rep{2} le ripetizioni

%---- Ogni riga successiva -------------------
%\vspace*{-\versesep}
%\[G] \[C]  \[D]	

%---- Ev Indicazioni -------------------------			
%\textnote{\textit{(Oppure tutta la strofa)} }	

\endverse
\fi

%%%%% RITORNELLO
\beginchorus
\textnote{\textbf{Rit.}}

\[E]Lo \[B]Spirito di \[A]Cristo \[E]fa fio\[B]rire il de\[C#-]serto, \[G#-]
torna la \[C#-]vita, \[A] noi diven\[B]tiamo testi\[E]moni di \[B]luce.

\endchorus

%%%%% STROFA
\beginverse		%Oppure \beginverse* se non si vuole il numero di fianco
\memorize 		% <<< DECOMMENTA se si vuole utilizzarne la funzione
%\chordsoff		% <<< DECOMMENTA se vuoi una strofa senza accordi

\[E] Non abbiamo rice\[B]vuto
\[A] uno spirito di \[B]schiavitù,
\[E] ma uno spirito di a\[B]more,
\[A] uno spirito di \[B]pace,\[A] nel quale gri\[B]diamo
\[A] abbà \[E]Padre,\[A] abbà \[C#-]Pa\[B]dre.

\endverse

%%%%% STROFA
\beginverse		%Oppure \beginverse* se non si vuole il numero di fianco
%\memorize 		% <<< DECOMMENTA se si vuole utilizzarne la funzione
%\chordsoff		% <<< DECOMMENTA se vuoi una strofa senza accordi

Lo Spirito che Cristo
risuscitò
darà vita ai nostri corpi,
corpi mortali, e li renderà
strumenti di salvezza, strumenti di salvezza.

\endverse

%%%%% STROFA
\beginverse		%Oppure \beginverse* se non si vuole il numero di fianco
%\memorize 		% <<< DECOMMENTA se si vuole utilizzarne la funzione
%\chordsoff		% <<< DECOMMENTA se vuoi una strofa senza accordi

Sono venuto a portare
il fuoco sulla terra
e come desidero
che divampi nel mondo e porti amore
ed entusiasmo in tutti i cuori.

\endverse

\endsong

\fi
%++++++++++++++++++++++++++++++++++++++++++++++++++++++++++++
%			FINE CANZONE TRASPOSTA
%++++++++++++++++++++++++++++++++++++++++++++++++++++++++++++
=======
% %++++++++++++++++++++++++++++++++++++++++++++++++++++++++++++
% %			CANZONE TRASPOSTA
% %++++++++++++++++++++++++++++++++++++++++++++++++++++++++++++
% \ifchorded
% %decremento contatore per avere stesso numero
% \addtocounter{songnum}{-1} 
% \beginsong{Lo Spirito di Cristo}[by={}] 	% <<< COPIA TITOLO E AUTORE
% \transpose{3} 						% <<< TRASPOSIZIONE #TONI + - (0 nullo)
% %\preferflats  %SE VOGLIO FORZARE i bemolle come alterazioni
% %\prefersharps %SE VOGLIO FORZARE i # come alterazioni
% \ifchorded
% 	%\textnote{Tonalità originale}	% <<< EV COMMENTI (tonalità originale/migliore)
% \fi

% %%%%%% INTRODUZIONE
% \ifchorded
% \vspace*{\versesep}
% \textnote{Intro: \qquad \qquad  }%(\eighthnote 116) % <<  MODIFICA IL TEMPO
% % Metronomo: \eighthnote (ottavo) \quarternote (quarto) \halfnote (due quarti)
% \vspace*{-\versesep}
% \beginverse*

% \nolyrics

% %---- Prima riga -----------------------------
% \vspace*{-\versesep}
% \[E] \[B] \[A] \[E]	 % \[*D] per indicare le pennate, \rep{2} le ripetizioni

% %---- Ogni riga successiva -------------------
% %\vspace*{-\versesep}
% %\[G] \[C]  \[D]	

% %---- Ev Indicazioni -------------------------			
% %\textnote{\textit{(Oppure tutta la strofa)} }	

% \endverse
% \fi

% %%%%% RITORNELLO
% \beginchorus
% \textnote{\textbf{Rit.}}

% \[E]Lo \[B]Spirito di \[A]Cristo 
% \[E]fa fio\[B]rire il de\[C#-]serto, 
% \[G#-] torna la \[C#-]vita, 
% \[A] noi diven\[B]tiamo testi\[E]moni di \[B]luce.

% \endchorus

% %%%%% STROFA
% \beginverse		%Oppure \beginverse* se non si vuole il numero di fianco
% \memorize 		% <<< DECOMMENTA se si vuole utilizzarne la funzione
% %\chordsoff		% <<< DECOMMENTA se vuoi una strofa senza accordi

% \[E] Non abbiamo rice\[B]vuto 
% \[A] uno spirito di \[B]schiavitù,
% \[E] ma uno spirito di a\[B]more, 
% \[A] uno spirito di \[B]pace,
% \[A] nel quale gri\[B]diamo 
% \[A] abbà \[E]Padre,
% \[A] abbà \[C#-]Pa\[B]dre.

% \endverse

% %%%%% STROFA
% \beginverse		%Oppure \beginverse* se non si vuole il numero di fianco
% %\memorize 		% <<< DECOMMENTA se si vuole utilizzarne la funzione
% %\chordsoff		% <<< DECOMMENTA se vuoi una strofa senza accordi

% ^ Lo Spirito ^che 
% ^ Cristo ri^suscitò
% ^ darà vita ai nostri ^corpi, 
% ^ corpi mor^tali,  
% ^ e li rende^rà 
% ^ strumenti di sal^vezza, 
% ^ strumenti di sal^vez^za.

% \endverse

% %%%%% STROFA
% \beginverse		%Oppure \beginverse* se non si vuole il numero di fianco
% %\memorize 		% <<< DECOMMENTA se si vuole utilizzarne la funzione
% \chordsoff		% <<< DECOMMENTA se vuoi una strofa senza accordi

% Sono venuto a portare
% il fuoco sulla terra
% e come desidero
% che divampi nel mondo 
% e porti amore
% ed entusiasmo in tutti i cuori.

% \endverse

% \endsong

% \fi
% %++++++++++++++++++++++++++++++++++++++++++++++++++++++++++++
% %			FINE CANZONE TRASPOSTA
% %++++++++++++++++++++++++++++++++++++++++++++++++++++++++++++
>>>>>>> 1cf7e891cb70141482a1e55d331c97dc8203b0ef

%-------------------------------------------------------------
%			INIZIO	CANZONE
%-------------------------------------------------------------


%titolo: 	Lodate Dio
%autore: 	Gen Rosso
%tonalita: 	Sol 



%%%%%% TITOLO E IMPOSTAZONI
\beginsong{Lodate Dio}[by={Gen\ Rosso}]	% <<< MODIFICA TITOLO E AUTORE
\transpose{0} 						% <<< TRASPOSIZIONE #TONI (0 nullo)
%\prefersharps
\momenti{Santi}							% <<< INSERISCI MOMENTI	
% momenti vanno separati da ; e vanno scelti tra:
% Ingresso; Atto penitenziale; Acclamazione al Vangelo; Dopo il Vangelo; Offertorio; Comunione; Ringraziamento; Fine; Santi; Pasqua; Avvento; Natale; Quaresima; Canti Mariani; Battesimo; Prima Comunione; Cresima; Matrimonio; Meditazione; Spezzare del pane;
\ifchorded
	%\textnote{Tonalità migliore }	% <<< EV COMMENTI (tonalità originale/migliore)
\fi


%%%%%% INTRODUZIONE
\ifchorded
\vspace*{\versesep}
\musicnote{
\begin{minipage}{0.48\textwidth}
\textbf{Intro}
\hfill 
%( \eighthnote \, 80)   % <<  MODIFICA IL TEMPO
% Metronomo: \eighthnote (ottavo) \quarternote (quarto) \halfnote (due quarti)
\end{minipage}
} 	
\vspace*{-\versesep}
\beginverse*

\nolyrics

%---- Prima riga -----------------------------
\vspace*{-\versesep}
\[G] \[C] \[D]	 % \[*D] per indicare le pennate, \rep{2} le ripetizioni

%---- Ogni riga successiva -------------------
%\vspace*{-\versesep}
%\[G] \[C]  \[D]	

%---- Ev Indicazioni -------------------------			
\textnote{\textit{(oppure tutta la sequenza strumentale)} }	

\endverse
\fi




\beginchorus
\[G] Lodate \[C]Di\[G]o \[F]cieli im\[C]mensi ed \[G]infiniti.
\[(G)] Lodate \[C]Di\[G]o \[D-7]cori eterni \[A-]d'Angeli.\[C] 
Lodate \[D7]Dio \[G]Santi \[A-7]del Suo \[D4]Re\[D]gno.
Lo\[G]datelo \[A-7]uomi\[D]ni, \[E-]Dio vi \[D]ama.
Lo\[G]datelo \[A-7]uomi\[D]ni, \[C]Dio è con \[G]voi.
\endchorus



%%%% STRUMENTALE
\ifchorded
\beginverse*
%\vspace*{-\versesep}
\musicnote{\textit{(strumentale di accompagnamento)}}
{\nolyrics \[G] \[C] \[G] \[F] \[C] \[G] 
\[G] \[C] \[G] \[F] \[D-] \[A-] 
\[C] \[D] \[G] \[A-] \[D] 
\[G] \[C] \[G] \[F] \[C] \[G] }
\endverse
\fi




\beginverse
\chordsoff
\textnote{\textit{[parlato]}}
Ti ringraziamo, Dio nostro Padre
perché sei Amore.
Tu ci hai fatto dono della vita
e ci hai creati per essere figli tuoi.
Ti ringraziamo perché ci fai partecipi
della tua opera creatrice
dandoci un mondo da plasmare
con le nostre mani.
\endverse



\beginverse
\chordsoff
\textnote{\textit{[parlato]}}
Ti ringraziamo, Signore Gesù,
perché per amore nostro
sei venuto in questo mondo
per riscattarci dal ogni male
e riportarci al Padre.
Ti ringraziamo perché sei rimasto
in mezzo a noi per sempre
per far di tutti gli uomini una sola famiglia,
un corpo solo in te.
\endverse




\beginverse
\chordsoff
\textnote{\textit{[parlato]}}
Ti ringraziamo, Spirito d'amore,
perché rinnovi la faccia della terra.
Tu dai luce e conforto ad ogni cuore.
Ti ringraziamo perché con i tuoi santi doni
ci dai la forza di avanzare nel nostro cammino
per giungere uniti alla gioia della tua casa.
\endverse


\endsong
%------------------------------------------------------------
%			FINE CANZONE
%------------------------------------------------------------

%-------------------------------------------------------------
%			INIZIO	CANZONE
%-------------------------------------------------------------


%titolo: 	Lodate il Signore
%autore: 	Daniele Ricci
%tonalita: 	Sol 



%%%%%% TITOLO E IMPOSTAZONI
\beginsong{Lodate il Signore}[by={RnS}] 	% <<< MODIFICA TITOLO E AUTORE
\transpose{0} 						% <<< TRASPOSIZIONE #TONI (0 nullo)
%\preferflats  %SE VOGLIO FORZARE i bemolle come alterazioni
%\prefersharps %SE VOGLIO FORZARE i # come alterazioni
\momenti{}							% <<< INSERISCI MOMENTI	
% momenti vanno separati da ; e vanno scelti tra:
% Ingresso; Atto penitenziale; Acclamazione al Vangelo; Dopo il Vangelo; Offertorio; Comunione; Ringraziamento; Fine; Santi; Pasqua; Avvento; Natale; Quaresima; Canti Mariani; Battesimo; Prima Comunione; Cresima; Matrimonio; Meditazione; Spezzare del pane;
\ifchorded
	%\textnote{Tonalità migliore }	% <<< EV COMMENTI (tonalità originale/migliore)
\fi


%%%%%% INTRODUZIONE
\ifchorded
\vspace*{\versesep}
\musicnote{
\begin{minipage}{0.48\textwidth}
\textbf{Intro}
\hfill 
%( \eighthnote \, 80)   % <<  MODIFICA IL TEMPO
% Metronomo: \eighthnote (ottavo) \quarternote (quarto) \halfnote (due quarti)
\end{minipage}
} 	
\vspace*{-\versesep}
\beginverse*

\nolyrics

%---- Prima riga -----------------------------
\vspace*{-\versesep}
\[A] \[E] \[F#-] \[A]


%---- Ogni riga successiva -------------------
\vspace*{-\versesep}
\[D] \[A] \[B-] \[E]

%---- Ogni riga successiva -------------------
\vspace*{-\versesep}
\[A]



%---- Ev Indicazioni -------------------------			
\textnote{\textit{[come il ritornello]} }	

\endverse
\fi




%%%%% STROFA
\beginverse		%Oppure \beginverse* se non si vuole il numero di fianco
\memorize 		% <<< DECOMMENTA se si vuole utilizzarne la funzione
%\chordsoff		% <<< DECOMMENTA se vuoi una strofa senza accordi

Lo\[A]date il Si\[D]gnore nei c\[E]ieli,
lo\[A]date \[D]angeli \[E]suoi.
Lo\[C]date voi \[F]tutte sue schiere 
la \[E]Sua mae\[E]stà.
\endverse
\beginverse*	
Lo^date da ^tutta la ^terra,
lo^date ^popoli s^uoi.
Lo^date ^giovani e vecchi 
la ^Sua bon^tà.
\endverse

%%%%% RITORNELLO
\beginchorus
\textnote{\textbf{Rit.}}

Cantate al Si\[A]gnore, \[E] 
un cantico \[F#-]nuovo. \[A]
Sia onore al \[D]Re, 
sia gloria al \[A]Dio,
che siede sul \[B-]tro\[E7]no.
Risuoni la \[A]lode, \[E]
la benedi\[F#-]zione al Si\[C#-]gnor,
che era e che \[D]è, che sempre sa\[E]rà.
Allelu\[A]ja! \[E] Allelu\[F#-]ja! \[A]
\endchorus

%%%%%% EV. INTERMEZZO
\beginverse*
\vspace*{1.3\versesep}
{
	\nolyrics
	\textnote{\textbf{Rit.}}
    \textnote{Intermezzo strumentale}
	
	\ifchorded

	%---- Prima riga -----------------------------
	\vspace*{-\versesep}
	\[D] \[A] \[B-] \[E] 

    
	%---- Ogni riga successiva -------------------
	\vspace*{-\versesep}
   \[A]

	\fi
	%---- Ev Indicazioni -------------------------			
	%\textnote{\textit{(ripetizione della strofa)}} 
	 
}
\vspace*{\versesep}
\endverse

%%%%% STROFA
\beginverse		%Oppure \beginverse* se non si vuole il numero di fianco
\memorize 		% <<< DECOMMENTA se si vuole utilizzarne la funzione
%\chordsoff		% <<< DECOMMENTA se vuoi una strofa senza accordi

Gio^iscano ^nel crea^tore,
e^sultino i ^figli di ^Sion.
^danzino, ^facciano festa 
al ^loro ^Dio.

\endverse
\beginverse*	

^Lodino il ^nome del ^Padre,
con ^timpani in^neggino a ^Lui.
C^ieli e ^terra si prostrino 
al ^Re dei ^re. 

\endverse



%%%%% RITORNELLO
\beginchorus
\textnote{\textbf{Rit.}}

Cantate al Si\[A]gnore, \[E] 
un cantico \[F#-]nuovo. \[A]
Sia onore al \[D]Re, 
sia gloria al \[A]Dio,
che siede sul \[B-]tro\[E7]no.
Risuoni la \[A]lode, \[E]
la benedi\[F#-]zione al Si\[C#-]gnor,
che era e che \[D]è, che sempre sa\[E]rà.
Allelu\[A]ja! \[E] Allelu\[F#-]ja! \[A]
\endchorus



%%%%%% EV. INTERMEZZO
\beginverse*
\vspace*{1.3\versesep}
{
	\nolyrics
    \textnote{Intermezzo strumentale}
	
	\ifchorded

	%---- Prima riga -----------------------------
	\vspace*{-\versesep}
	\[D] \[A] \[B-] \[E] 
	\fi
	%---- Ev Indicazioni -------------------------			
	%\textnote{\textit{(ripetizione della strofa)}} 
	 
}
\vspace*{\versesep}
\endverse

%%%%% RITORNELLO
\beginchorus
\vspace*{-\versesep}
Risuoni la \[A]lode, \[E]
la benedi\[F#-]zione al Si\[C#-]gnor,
che era e che \[D]è, che sempre sa\[E]rà.
\textnote{\textit{[si alza la tonalità]}}
Allelu\[A]ja! \[F]
\endchorus



\transpose{1}
\preferflats 



%%%%% RITORNELLO
\beginchorus
\textnote{\textbf{Rit.}}

Cantate al Si\[A]gnore, \[E]
un cantico \[F#-]nuovo. \[A]
Sia onore al \[D]Re, 
sia gloria al \[A]Dio,
che siede sul \[B-]tro\[E7]no.
Risuoni la \[A]lode, \[E]
la benedi\[F#-]zione al Si\[C#-]gnor,
che era e che \[D]è, che sempre sa\[E]rà.
\endchorus




%%%%%% EV. CHIUSURA SOLO STRUMENTALE
\ifchorded
\beginchorus %oppure \beginverse*
\vspace*{1.3\versesep}
\textnote{Chiusura } %<<< EV. INDICAZIONI

\[F]A-\[G4]\[G]a-\[A]alleluja! \[A]Alleluja!   \rep{5} \quad \[A*]
\endchorus  %oppure \endverse
\fi


\endsong
%------------------------------------------------------------
%			FINE CANZONE
%------------------------------------------------------------




%-------------------------------------------------------------
%			INIZIO	CANZONE
%-------------------------------------------------------------


%titolo: 	Lode a te o Cristo
%autore: 	Gen Verde
%tonalita: 	F#-



%%%%%% TITOLO E IMPOSTAZONI
\beginsong{Lode a te o Cristo}[by={Gen Verde}] 	% <<< MODIFICA TITOLO E AUTORE
\transpose{0} 						% <<< TRASPOSIZIONE #TONI (0 nullo)
%\preferflats  %SE VOGLIO FORZARE i bemolle come alterazioni
%\prefersharps %SE VOGLIO FORZARE i # come alterazioni
\momenti{Acclamazione al Vangelo; Quaresima}							% <<< INSERISCI MOMENTI	
% momenti vanno separati da ; e vanno scelti tra:
% Ingresso; Atto penitenziale; Acclamazione al Vangelo; Dopo il Vangelo; Offertorio; Comunione; Ringraziamento; Fine; Santi; Pasqua; Avvento; Natale; Quaresima; Canti Mariani; Battesimo; Prima Comunione; Cresima; Matrimonio; Meditazione; Spezzare del pane;
\ifchorded
	%\textnote{Tonalità migliore }	% <<< EV COMMENTI (tonalità originale/migliore)
\fi

%%%%%% INTRODUZIONE
\ifchorded
\vspace*{\versesep}
\musicnote{
\begin{minipage}{0.48\textwidth}
\textbf{Intro}
\hfill 
%( \eighthnote \, 80)   % <<  MODIFICA IL TEMPO
% Metronomo: \eighthnote (ottavo) \quarternote (quarto) \halfnote (due quarti)
\end{minipage}
} 	
\vspace*{-\versesep}
\beginverse*

\nolyrics

%---- Prima riga -----------------------------
\vspace*{-\versesep}
\[F#-] \[E] \[D] \[F#-]	 % \[*D] per indicare le pennate, \rep{2} le ripetizioni

%---- Ogni riga successiva -------------------
%\vspace*{-\versesep}
%\[G] \[C]  \[D]	

%---- Ev Indicazioni -------------------------			
%\textnote{\textit{(Oppure tutta la strofa)} }	

\endverse
\fi




%%%%% RITORNELLO
\beginchorus
\textnote{\textbf{Rit.}}


\[F#-]Lode a \[E]te, o \[D]Cri\[F#-]sto,
\[B-]Re di e\[D]ter\[E*]na \[F#-]gloria.  \rep{2}

\endchorus



%%%%%% EV. INTERMEZZO
\beginverse*
\vspace*{1.3\versesep}
{
	\nolyrics
	\textnote{Intermezzo strumentale}
	
	\ifchorded

	%---- Prima riga -----------------------------
	\vspace*{-\versesep}
	\[B-] \[E]  \[F#-]	 \rep{2}




	\fi
	%---- Ev Indicazioni -------------------------			
	%\textnote{\textit{(ripetizione della strofa)}} 
	 
}
\vspace*{\versesep}
\endverse



%%%%% STROFA
\beginverse		%Oppure \beginverse* se non si vuole il numero di fianco
\memorize 		% <<< DECOMMENTA se si vuole utilizzarne la funzione
%\chordsoff		% <<< DECOMMENTA se vuoi una strofa senza accordi

Si\[F#-]gnore, tu \[E]sei vera\[D]men\[F#-]te 
\[F#-]il Salva\[E]tore del \[D]mon\[E]do,
\[F#-]dammi dell’\[E]acqua \[D]vi\[C#-]va 
per\[D7]ché non \[E]abbia più \[F#-]sete.

\endverse






%%%%% STROFA
\beginverse		%Oppure \beginverse* se non si vuole il numero di fianco
%\memorize 		% <<< DECOMMENTA se si vuole utilizzarne la funzione
%\chordsoff		% <<< DECOMMENTA se vuoi una strofa senza accordi

Chi ^beve ^di quest’^acq^ua 
^avrà di ^nuovo ^se^te,
ma chi ^beve dell’^acqua che ^io gli da^rò
^non av^rà mai più ^sete.

\endverse







%%%%%% EV. INTERMEZZO
\beginverse*
\vspace*{1.3\versesep}
{
	\nolyrics
	\musicnote{Finale strumentale}
	
	\ifchorded

	%---- Prima riga -----------------------------
	\vspace*{-\versesep}
	\[F#-] \[E] \[D] \[F#-]	 % \[*D] per indicare le pennate, \rep{2} le ripetizioni


	%---- Ogni riga successiva -------------------
	\vspace*{-\versesep}
	\[F#-] \[E] \[D] \[C#-]	 % \[*D] per indicare le pennate, \rep{2} le ripetizioni

	%---- Ogni riga successiva -------------------
	\vspace*{-\versesep}
	\[F#-] \[E] \[D] \[C#-]	 % \[*D] per indicare le pennate, \rep{2} le ripetizioni

	%---- Prima riga -----------------------------
	\vspace*{-\versesep}
	\[D] \[E] \[F#-*]	 % \[*D] per indicare le pennate, \rep{2} le ripetizioni


	\fi
	%---- Ev Indicazioni -------------------------			
	%\textnote{\textit{(ripetizione della strofa)}} 
	 
}
\vspace*{\versesep}
\endverse

\endsong
%------------------------------------------------------------
%			FINE CANZONE
%------------------------------------------------------------


%-------------------------------------------------------------
%			INIZIO	CANZONE
%-------------------------------------------------------------


%titolo: 	Benedetto il nome del Signore
%autore: 	Matt Redman
%tonalita: 	Sol > La 



%%%%%% TITOLO E IMPOSTAZONI
\beginsong{Lode al nome Tuo}[by={Blessed be your name — M. Redman, B. Redman}] 	% <<< MODIFICA TITOLO E AUTORE
\transpose{0} 						% <<< TRASPOSIZIONE #TONI (0 nullo)
%\preferflats  %SE VOGLIO FORZARE i bemolle come alterazioni
%\prefersharps %SE VOGLIO FORZARE i # come alterazioni
\momenti{Meditazione; Ringraziamento; Fine}							% <<< INSERISCI MOMENTI	
% momenti vanno separati da ; e vanno scelti tra:
% Ingresso; Atto penitenziale; Acclamazione al Vangelo; Dopo il Vangelo; Offertorio; Comunione; Ringraziamento; Fine; Santi; Pasqua; Avvento; Natale; Quaresima; Canti Mariani; Battesimo; Prima Comunione; Cresima; Matrimonio; Meditazione; Spezzare del pane;
\ifchorded
	%\textnote{Tonalità migliore }	% <<< EV COMMENTI (tonalità originale/migliore)
\fi


%%%%%% INTRODUZIONE
\ifchorded
\vspace*{\versesep}
\musicnote{
\begin{minipage}{0.48\textwidth}
\textbf{Intro}
\hfill 
%( \eighthnote \, 80)   % <<  MODIFICA IL TEMPO
% Metronomo: \eighthnote (ottavo) \quarternote (quarto) \halfnote (due quarti)
\end{minipage}
} 	
\vspace*{-\versesep}
\beginverse*

\nolyrics

%---- Prima riga -----------------------------
\vspace*{-\versesep}
\[G] \[D] \[E-] \[C] 	 % \[*D] per indicare le pennate, \rep{2} le ripetizioni

%---- Ogni riga successiva -------------------
\vspace*{-\versesep}
\[G] \[D] \[E-] \[C] \[C]	

%---- Ev Indicazioni -------------------------			
%\textnote{\textit{(Oppure tutta la strofa)} }	

\endverse
\fi




%%%%% STROFA
\beginverse		%Oppure \beginverse* se non si vuole il numero di fianco
\memorize 		% <<< DECOMMENTA se si vuole utilizzarne la funzione
%\chordsoff		% <<< DECOMMENTA se vuoi una strofa senza accordi

\[G] Lode al \[D]nome Tuo, \brk dalle \[E-]terre più \[C]floride.
Dove \[G]tutto sembra \[D]vivere,  
lode al \[C]nome Tuo.

\endverse
\beginverse*

^ Lode al ^nome Tuo,  \brk  dalle ^terre più ^aride.
Dove ^tutto sembra ^sterile, 
lode al ^nome Tuo.


\endverse
\beginverse*	

^ Tornerò a lo^darti sempre 
^ per ogni dono ^Tuo.
^ E quando scende^rà la notte 
\[E-] sempre io di\[C]rò:

\endverse





%%%%% RITORNELLO
\beginchorus
\textnote{\textbf{Rit.}}

Benedetto il \[G]nome del Si\[D]gnor,
lode al nome \[E-]Tu-u-\[C]o!
Benedetto il \[G]nome del Si\[D]gnor,
il glorioso \[E-]nome di Ge\[C]sù. 	

\endchorus





%%%%% STROFA
\beginverse		%Oppure \beginverse* se non si vuole il numero di fianco
%\memorize 		% <<< DECOMMENTA se si vuole utilizzarne la funzione
%\chordsoff		% <<< DECOMMENTA se vuoi una strofa senza accordi

^ Lode al ^nome Tuo, \brk quando il ^sole splende ^su di me.
Quando ^tutto è incan^tevole,
lode al ^nome Tuo.

\endverse
\beginverse*	

^ Lode al ^nome Tuo, \brk quando ^io sto da^vanti a Te.
Con il ^cuore triste e ^fragile,
lode al ^nome Tuo.

\endverse
\beginverse*	

^ Tornerò a lo^darti sempre 
^ per ogni dono ^Tuo.
^ E quando scende^rà la notte 
\[E-] sempre io di\[C]rò:

\endverse



%%%%% RITORNELLO
\beginchorus
\textnote{\textbf{Rit.}}

Benedetto il \[G]nome del Si\[D]gnor,
lode al nome \[E-]Tu-u-\[C]o!
Benedetto il \[G]nome del Si\[D]gnor,
il glorioso \[E-]nome di Ge\[C]sù. 	

\endchorus



%%%%% BRIDGE
\beginverse*		%Oppure \beginverse* se non si vuole il numero di fianco
%\memorize 		% <<< DECOMMENTA se si vuole utilizzarne la funzione
%\chordsoff		% <<< DECOMMENTA se vuoi una strofa senza accordi
\vspace*{1.3\versesep}
\textnote{\textbf{Bridge}} %<<< EV. INDICAZIONI


Tu ^doni e porti ^via.
Tu ^doni e porti ^via.
Ma ^sempre sceglie^rò
di \[E-]benedire \[C]te!  

\endverse
\beginverse*	

^ Tornerò a lo^darti sempre 
^ per ogni dono ^Tuo.
^ E quando scende^rà la notte 
\[E-] sempre io di\[C]rò:

\endverse


%%%%% RITORNELLO
\beginchorus
\textnote{\textbf{Rit.}}

Benedetto il \[G]nome del Si\[D]gnor,
lode al nome \[E-]Tu-u-\[C]o! \rep{3}
Benedetto il \[G]nome del Si\[D]gnor,
il glorioso \[E-]nome di Ge\[C]sù. 	

\endchorus



%%%%% BRIDGE
\beginverse*		%Oppure \beginverse* se non si vuole il numero di fianco
%\memorize 		% <<< DECOMMENTA se si vuole utilizzarne la funzione
%\chordsoff		% <<< DECOMMENTA se vuoi una strofa senza accordi

Tu ^doni e porti ^via.
Tu ^doni e porti ^via.
Ma ^sempre sceglie^rò
di \[E-]benedire \[C]te!  \rep{2}

\endverse





%%%%%% EV. INTERMEZZO
\beginverse*
\vspace*{1.3\versesep}
{
	\nolyrics
	\musicnote{Chiusura strumentale}
	
	\ifchorded

	%---- Prima riga -----------------------------
	\vspace*{-\versesep}
	\[G] \[D] \[E-] \[C] 

	%---- Ogni riga successiva -------------------
	\vspace*{-\versesep}
	\[G] \[D] \[C*]  \textit{[sospeso...]}


	\fi
	%---- Ev Indicazioni -------------------------			
	%\musicnote{\textit{sospeso}} 
	 
}
\vspace*{\versesep}
\endverse


\endsong
%------------------------------------------------------------
%			FINE CANZONE
%------------------------------------------------------------



%titolo{Luce}
%autore{Comunità del Cenacolo}
%album{Dio dell'amore}
%tonalita{Sol}
%famiglia{Liturgica}
%gruppo{}
%momenti{}
%identificatore{luce}
%data_revisione{2011_12_31}
%trascrittore{Francesco Endrici - Manuel Toniato}
\beginsong{Luce}[by={Comunità\ del\ Cenacolo}]

\beginverse
\[G]C'è il segreto della liber\[D]tà \brk quella \[E-]vera nasce \[C]dentro di \[D]te.
È \[G]come risvegliarsi un mat\[D]tino con il \[E-]sole \brk dopo un \[C]lungo in\[D]verno
\[G] nel soffrire \[D] mio Signore \[E-] \brk ho incontrato \[C]Te Dio a\[D]more
\[G] nel perdono \[D] nel gioire \[E-] \brk ho capito \[C]che sei \[D]luce per me.
\endverse

\beginchorus
\[G] Signore sono \[D]qui per dirti ancora \[E-]sì \[C]lu\[D]ce
\[G] fammi scoppiare \[D]di gioia di vive\[E-]re \[C]lu\[D]ce.
\[G] Fammi strumento \[D]per portare attorno a \[E-]me \[C]lu\[D]ce
\[G] e chi è vicino a \[D]me sappia che tutto in \[E-]te è \[C]lu\[D]ce.
\endchorus

\beginverse
\chordsoff
Voglio ringraziarti Signore \brk per la Vita che mi hai ridonato
so che sei nell'amore degli amici \brk che ora ho incontrato
nel soffrire mio Signore \brk ho incontrato Te Dio amore
nel perdono nel gioire ho capito \brk che sei luce per me.
\endverse

\beginchorus
\chordsoff
Signore sono qui per dirti ancora sì luce\ldots
\endchorus

\beginverse*
\[E-] E con le lacrime agli \[C]occhi \brk e le mie mani al\[G]zate verso te Ge\[D]sù
\[E-] con la speranza nel \[C]cuore \brk e la tua luce in \[D4]me paura non ho \[D]più.
\endverse
\endsong




%-------------------------------------------------------------
%			INIZIO	CANZONE
%-------------------------------------------------------------


%titolo: 	Luce dei miei passi (XdiQua)
%autore: 	N. Cermenati, A. Motti, E. Porro, M. Violato, S. Tremolada
%tonalita: 	Re-



%%%%%% TITOLO E IMPOSTAZONI
\beginsong{Luce dei miei passi - XdiQua}[by={N. Cermenati, A. Motti, E. Porro, M. Violato, S. Tremolada}] 	% <<< MODIFICA TITOLO E AUTORE
\transpose{-2} 						% <<< TRASPOSIZIONE #TONI (0 nullo)
%\preferflats  %SE VOGLIO FORZARE i bemolle come alterazioni
%\prefersharps %SE VOGLIO FORZARE i # come alterazioni
\momenti{Ringraziamento; Congedo; Comunione}							% <<< INSERISCI MOMENTI	
% momenti vanno separati da ; e vanno scelti tra:
% Ingresso; Atto penitenziale; Acclamazione al Vangelo; Dopo il Vangelo; Offertorio; Comunione; Ringraziamento; Fine; Santi; Pasqua; Avvento; Natale; Quaresima; Canti Mariani; Battesimo; Prima Comunione; Cresima; Matrimonio; Meditazione; Spezzare del pane;
\ifchorded
	%\textnote{Tonalità migliore }	% <<< EV COMMENTI (tonalità originale/migliore)
\fi


%%%%%% INTRODUZIONE
\ifchorded
\vspace*{\versesep}
\musicnote{
\begin{minipage}{0.48\textwidth}
\textbf{Intro}
\hfill 
%( \eighthnote \, 80)   % <<  MODIFICA IL TEMPO
% Metronomo: \eighthnote (ottavo) \quarternote (quarto) \halfnote (due quarti)
\end{minipage}
} 	
\vspace*{-\versesep}
\beginverse*
\nolyrics

%---- Prima riga -----------------------------
\vspace*{-\versesep}
\[E] \[A] \[E] \[B]	 % \[*D] per indicare le pennate, \rep{2} le ripetizioni

%---- Ogni riga successiva -------------------
\vspace*{-\versesep}
\[E] \[A] \[B] \[B]

%---- Ev Indicazioni -------------------------			
%\textnote{\textit{(Oppure tutta la strofa)} }	

\endverse
\fi




%%%%% STROFA
\beginverse		%Oppure \beginverse* se non si vuole il numero di fianco
\memorize 		% <<< DECOMMENTA se si vuole utilizzarne la funzione
%\chordsoff		% <<< DECOMMENTA se vuoi una strofa senza accordi

\[E]Nel cammino del\[A]la mia vita \brk \[E]cerco quel che \[B4]sono, \[B]
\[E]ogni passo in \[A]questo mondo \brk \[E]sogno la mia liber\[B]tà.
\[E]Ma la strada a \[A]volte è buia, \brk \[E]perdo la fi\[B4]ducia. \[B]
\[E]Ho bisogno di un \[A]po' di luce, \brk il \[E]sole chi sa\[B]rà?

\endverse
\beginverse*		

Ma se guardo il \[F#-]volto tuo, \brk io l'Amore \[A]vedo in Te.
Guardo la tua \[F#-]croce e 
Tu speranza \[A]sei.. per \[B]me. \[B]

\endverse


%%%%% RITORNELLO
\beginchorus
\textnote{\textbf{Rit.}}

Sono \[E]qui, Signore, sono \[A]qui! \[A]
\[F#-]Luce dei miei passi è la \[A]Tua pa\[B]rola!
\echo{Sono} \[E]qui, Signore, sono \[A]qui! \[A]
\[F#-]Lungo la mia strada ogni \[A]giorno \brk tu sei \[B]guida per \[E]me!
\endchorus







%%%%%% EV. INTERMEZZO
\beginverse*
\vspace*{1.3\versesep}
{
	\nolyrics
	\textnote{Intermezzo strumentale}
	
	\ifchorded

	%---- Prima riga -----------------------------
	\vspace*{-\versesep}
	\[A] \[E]  \[B]	 

	%---- Ogni riga successiva -------------------
	\vspace*{-\versesep}
	\[E] \[A]  \[B]	 \[B]


	\fi
	%---- Ev Indicazioni -------------------------			
	%\textnote{\textit{(ripetizione della strofa)}} 
	 
}
\vspace*{\versesep}
\endverse



%%%%% STROFA
\beginverse		%Oppure \beginverse* se non si vuole il numero di fianco
\memorize 		% <<< DECOMMENTA se si vuole utilizzarne la funzione
%\chordsoff		% <<< DECOMMENTA se vuoi una strofa senza accordi

^Nel cammino del^la mia vita \brk in^contro il mio fra^tello.^
^Condivido la ^mia strada, \brk la ^meta è PerDi^Quà!
^Ma la strada a ^volte è dura, \brk ^perdo la spe^ranza.^
^Cerco solo un ^punto fermo, \brk in ^chi lo trove^rò?

\endverse
\beginverse*		

Ma se guardo il \[F#-]volto tuo, \brk io l'Amore \[A]vedo in Te.
Guardo la tua \[F#-]croce e 
Tu speranza \[A]sei.. per \[B]me. \[B]

\endverse



%%%%% RITORNELLO
\beginchorus
\textnote{\textbf{Rit.}}

Sono \[E]qui, Signore, sono \[A]qui! \[A]
\[F#-]Luce dei miei passi è la \[A]Tua pa\[B]rola!
\echo{Sono} \[E]qui, Signore, sono \[A]qui! \[A]
\[F#-]Lungo la mia strada ogni \[A]giorno \brk tu sei \[B]guida per \[E]me! \[B] 
\endchorus






%%%%% BRIDGE
\beginverse*		%Oppure \beginverse* se non si vuole il numero di fianco
%\memorize 		% <<< DECOMMENTA se si vuole utilizzarne la funzione
%\chordsoff		% <<< DECOMMENTA se vuoi una strofa senza accordi

Ma se guardo il \[F#-]volto tuo, \brk io l'Amore \[A]vedo in Te.
Guardo la tua \[F#-]croce e 
Tu speranza \[A]sei.. per \[B]me. \[C#]

\endverse




%%%%% RITORNELLO
\beginchorus

\textnote{\textbf{Rit.}}
\textnote{\textit{(si alza la tonalità)}}
\transpose{2}
Sono \[E]qui, Signore, sono \[A]qui! \[A]
\[F#-]Luce dei miei passi è la \[A]Tua pa\[B]rola!
\echo{Sono} \[E]qui, Signore, sono \[A]qui! \[A]
\[F#-]Lungo la mia strada ogni \[A]giorno \brk tu sei \[B]guida per \[E]me! \[A]
\endchorus




%%%%%% EV. FINALE

\beginchorus %oppure \beginverse*
\vspace*{1.3\versesep}
\textnote{\textbf{Finale} \textit{(rallentando)}} %<<< EV. INDICAZIONI

\[F#*] Tu sei \[C#*]guida per \[B]me!  \[B]
\[F#*] Tu sei \[C#*]guida per \[B]me! \[B] \[F#*]
\endchorus  %oppure \endverse


\endsong
%------------------------------------------------------------
%			FINE CANZONE
%------------------------------------------------------------



%-------------------------------------------------------------
%			INIZIO	CANZONE
%-------------------------------------------------------------


%titolo: 	Luce del mondo
%autore: 	F. Pesarese, RnS
%tonalita: 	Fa 



%%%%%% TITOLO E IMPOSTAZONI
\beginsong{Luce del mondo}[by={F. Pesarese, RnS}] 	% <<< MODIFICA TITOLO E AUTORE
\transpose{0} 						% <<< TRASPOSIZIONE #TONI (0 nullo)
%\preferflats  %SE VOGLIO FORZARE i bemolle come alterazioni
%\prefersharps %SE VOGLIO FORZARE i # come alterazioni
\momenti{}							% <<< INSERISCI MOMENTI	
% momenti vanno separati da ; e vanno scelti tra:
% Ingresso; Atto penitenziale; Acclamazione al Vangelo; Dopo il Vangelo; Offertorio; Comunione; Ringraziamento; Fine; Santi; Pasqua; Avvento; Natale; Quaresima; Canti Mariani; Battesimo; Prima Comunione; Cresima; Matrimonio; Meditazione; Spezzare del pane;
\ifchorded
	%\textnote{Tonalità migliore }	% <<< EV COMMENTI (tonalità originale/migliore)
\fi


%%%%%% INTRODUZIONE
\ifchorded
\vspace*{\versesep}
\textnote{Intro: \qquad \qquad  }%(\eighthnote 116) % <<  MODIFICA IL TEMPO
% Metronomo: \eighthnote (ottavo) \quarternote (quarto) \halfnote (due quarti)
\vspace*{-\versesep}
\beginverse*

\nolyrics

%---- Prima riga -----------------------------
\vspace*{-\versesep}
\[F] \[B&]  \[F] \[B&]  \[C*]	 % \[*D] per indicare le pennate, \rep{2} le ripetizioni

%---- Ogni riga successiva -------------------
%\vspace*{-\versesep}
%\[G] \[C]  \[D]	

%---- Ev Indicazioni -------------------------			
%\textnote{\textit{(Oppure tutta la strofa)} }	

\endverse
\fi

\newchords{verse}
\newchords{chorus}


%%%%% STROFA
\beginverse		%Oppure \beginverse* se non si vuole il numero di fianco
\memorize[verse]% <<< DECOMMENTA se si vuole utilizzarne la funzione
%\chordsoff		% <<< DECOMMENTA se vuoi una strofa senza accordi

\[F]Luce del mondo sei, Signor,
\[F] il tuo Amore ci illumina
e le \[D-]tenebre che avvolgono il nostro cuor
\[B&]con la tua Luce sva\[C*]nisco\[F]no.
\[F]Luce del mondo sei, Signor,
\[F] il tuo Amore ci illumina
e le \[D-]tenebre che avvolgono il nostro cuor
\[B&]con la tua Luce sva\[C*]nisco\[D-]no,
\[B&]con la tua Luce sva\[C*]nisco\[F]no.
\endverse




%%%%% RITORNELLO
\beginchorus
\textnote{\textbf{Rit.}}
\memorize[chorus]
\[F]Luce del mondo sei,
luce che illumi\[A-]na,
\[D-]luce del mondo sei.
\[B&]Il tuo Amore, Si\[C]gnor,
\[A]mai si spegne\[D-]rà,
perché \[B&*]luce del \[C*]mondo sei,
\[B&*]luce che il\[G-*]lumi\[C]na. \[C]

\endchorus





%%%%% STROFA
\beginverse		%Oppure \beginverse* se non si vuole il numero di fianco
\replay[verse]		% <<< DECOMMENTA se si vuole utilizzarne la funzione
%\chordsoff		% <<< DECOMMENTA se vuoi una strofa senza accordi

^Gioia del mondo sei, Signor,
^ il tuo Amore ci fa cantar,
le tris^tezze che sempre ci opprimono,
^con la tua gioia sva^nisco^no.
^Gioia del mondo sei, Signor,
^ il tuo Amore ci fa cantar,
le tris^tezze che sempre ci opprimono,
^con la tua gioia sva^nisco^no,
^con la tua gioia sva^nisco^no.

\endverse




%%%%% RITORNELLO
\beginchorus
\textnote{\textbf{Rit.}}
\replay[chorus]

^Gioia del mondo sei,
gioia che fa can^tar,
^gioia del mondo sei!
^Il tuo amore, Si^gnor,
^mai si spegne^rà,
perché ^gioia del ^mondo sei,
\vspace*{\versesep}
\textnote{Si alza di tonalità}
^gioia che ^fa can^ta-\[D]re.
\endchorus


\transpose{2}


%%%%% STROFA
\beginverse		%Oppure \beginverse* se non si vuole il numero di fianco
\replay[verse]		% <<< DECOMMENTA se si vuole utilizzarne la funzione
%\chordsoff		% <<< DECOMMENTA se vuoi una strofa senza accordi


^Forza del mondo sei, Signor,
^ il tuo amore ci libera,
le ca^tene che ci legano,
^con la tua forza sva^nisco^no.
^Forza del mondo sei, Signor,
^ il tuo amore ci libera,
le ca^tene che ci legano,
^con la tua forza sva^nisco^no,
^con la tua forza sva^nisco^no.

\endverse




%%%%% RITORNELLO
\beginchorus
\textnote{\textbf{Rit.}}
\replay[chorus]


^Forza del mondo sei,
forza che ^libera,
^forza del mondo sei!
^Il tuo amore, Si^gnor,
^mai si spegne^rà,
perché ^forza del ^mondo sei,
^forza che ^libe^ra. ^
\endchorus

%%%%%% EV. INTERMEZZO
\beginverse*
\vspace*{1.3\versesep}
{
	\nolyrics
	\textnote{Intermezzo strumentale}
	
	\ifchorded

	%---- Prima riga -----------------------------
	\vspace*{-\versesep}
	\[F] \[F] 
    
	%---- Ogni riga successiva -------------------
	\vspace*{-\versesep}
	\[F] \[A-] 

	%---- Ogni riga successiva -------------------
	\vspace*{-\versesep}
	\[D-] \[D-]
 

    %---- Ogni riga successiva -------------------
	\vspace*{-\versesep}
	\[B&] \[C*] \[D-] \[B&] \[C*] \[F] 

	\fi
	%---- Ev Indicazioni -------------------------			
	%\textnote{\textit{(ripetizione della strofa)}} 
	 
}
\vspace*{\versesep}
\endverse



%%%%% STROFA
\beginverse		%Oppure \beginverse* se non si vuole il numero di fianco
%\memorize[verse]% <<< DECOMMENTA se si vuole utilizzarne la funzione
%\chordsoff		% <<< DECOMMENTA se vuoi una strofa senza accordi

\[F]Luce del mondo sei, Signor,
\[F] il tuo Amore ci illumina
e le \[D-]tenebre che avvolgono il nostro cuor
\[B&]con la tua Luce sva\[C*]nisco\[F]no.
\[F]Luce del mondo sei, Signor,
\[F] il tuo Amore ci illumina
e le \[D-]tenebre che avvolgono il nostro cuor
\[B&]con la tua Luce sva\[C*]nisco\[D-]no,
\[B&]con la tua Luce sva\[C*]nisco\[F]no.
\endverse




%%%%% RITORNELLO
\beginchorus
\textnote{\textbf{Rit.}}
%\memorize[chorus]

\[F]Luce del mondo sei,
gioia che fa can\[A-]tar,
\[D-]forza che libera.
\[B&]Il tuo Amore, Si\[C]gnor,
\[A]mai si spegne\[D-]rà,
perché \[B&*]luce del \[C*]mondo sei,
\[B&*]luce che il\[G-*]lumi\[C]na. \[C]

\endchorus

%%%%%% EV. FINALE

\beginchorus %oppure \beginverse*
\vspace*{1.3\versesep}
\textnote{Finale \textit{(rallentando)}} %<<< EV. INDICAZIONI

\[C/A-]Luce che illumi\[F]na. \[F*]

\endchorus  %oppure \endverse




\endsong
%------------------------------------------------------------
%			FINE CANZONE
%------------------------------------------------------------



%-------------------------------------------------------------
%			INIZIO	CANZONE
%-------------------------------------------------------------


%titolo: 	Luce di verità
%autore: 	Becchimanzi, Scordari, Giordano
%tonalita: 	Mi



%%%%%% TITOLO E IMPOSTAZONI
\beginsong{Luce di verità}[by={G. Becchimanzi, G. Scordari, C. Giordano}] 	% <<< MODIFICA TITOLO E AUTORE
\transpose{0} 						% <<< TRASPOSIZIONE #TONI (0 nullo)
\momenti{Ingresso; Comunione; Dopo il Vangelo; Cresima}							% <<< INSERISCI MOMENTI	
% momenti vanno separati da ; e vanno scelti tra:
% Ingresso; Atto penitenziale; Acclamazione al Vangelo; Dopo il Vangelo; Offertorio; Comunione; Ringraziamento; Fine; Santi; Pasqua; Avvento; Natale; Quaresima; Canti Mariani; Battesimo; Prima Comunione; Cresima; Matrimonio; Meditazione; Spezzare del pane;
\ifchorded
	%\textnote{Tonalità originale }	% <<< EV COMMENTI (tonalità originale/migliore)
\fi

%%%%%% INTRODUZIONE
\ifchorded
\vspace*{\versesep}
\musicnote{
\begin{minipage}{0.48\textwidth}
\textbf{Intro}
\hfill 
%( \eighthnote \, 80)   % <<  MODIFICA IL TEMPO
% Metronomo: \eighthnote (ottavo) \quarternote (quarto) \halfnote (due quarti)
\end{minipage}
} 	
\vspace*{-\versesep}
\beginverse*

\nolyrics

%---- Prima riga -----------------------------
\vspace*{-\versesep}
\[E] \[A] 	\[E] \[G#-]  % \[*D] per indicare le pennate, \rep{2} le ripetizioni

%---- Ogni riga successiva -------------------
\vspace*{-\versesep}
\[A] \[E] \[(C#-7*)] \[F#7] \[A] \[B]

%---- Ev Indicazioni -------------------------			
%\textnote{\textit{(Oppure tutta la strofa)} }	

\endverse
\fi


%%%%% RITORNELLO
\beginchorus
\textnote{\textbf{Rit.}}
\[E]Luce di veri\[A]tà, \[E]fiamma di cari\[G#-]tà,
\[A]vincolo di uni\[E]tà, \[(C#-7*)]Spirito \[F#7]Santo A\[A]mo\[B]re.
\[E]Dona la liber\[A]tà, \[E]dona la santi\[G#-]tà,
\[A]fa' dell'umani\[E]tà \[(C#-7*)]il tuo \[D]canto di \[A]lo\[B]de.
\endchorus

\musicnote{\textit{(dolce, leggero)}}

%%%%% STROFA
\beginverse		%Oppure \beginverse* se non si vuole il numero di fianco
\memorize 		% <<< DECOMMENTA se si vuole utilizzarne la funzione
%\chordsoff		% <<< DECOMMENTA se vuoi una strofa senza accordi
\[C#-] Ci poni come \[B]luce sopra un \[E]mon\[A]te:
\[F#-] in noi l'umani\[E]tà vedrà il tuo \[B4]vol\[B]to. 
\[A] Ti testimonie\[B]remo fra le \[E]gen\[A]ti:
\[F#-] in noi l'umani\[E]tà vedrà il tuo \[B4]volto,  
\[B]\textbf{Spirito, vieni!}
\endverse

%%%%% STROFA
\beginverse		%Oppure \beginverse* se non si vuole il numero di fianco
%\memorize 		% <<< DECOMMENTA se si vuole utilizzarne la funzione
%\chordsoff		% <<< DECOMMENTA se vuoi una strofa senza accordi
^ Cammini accanto a ^noi lungo la ^stra^da,
^ si realizzi in ^noi la tua missi^one.^
^ Attingeremo ^forza dal tuo ^cuor^e,
^ si realizzi in ^noi la tua missi^one, 
^\textbf{Spirito, vieni!}
\endverse

%%%%% STROFA
\beginverse		%Oppure \beginverse* se non si vuole il numero di fianco
%\memorize 		% <<< DECOMMENTA se si vuole utilizzarne la funzione
%\chordsoff		% <<< DECOMMENTA se vuoi una strofa senza accordi
^ Come sigillo ^posto sul tuo ^cuo^re,
^ ci custodisci, ^Dio, nel tuo a^more.^
^ Hai dato la tua ^vita per sal^var^ci,
^ ci custodisci, ^Dio, nel tuo a^more, 
^\textbf{Spirito, vieni!}
\endverse

%%%%% STROFA
\beginverse		%Oppure \beginverse* se non si vuole il numero di fianco
%\memorize 		% <<< DECOMMENTA se si vuole utilizzarne la funzione
%\chordsoff		% <<< DECOMMENTA se vuoi una strofa senza accordi
^ Dissiperai le ^tenebre del ^ma^le,
^ esulterà in ^te la tua crea^zione.^
^ Vivremo al tuo cos^petto in e^te^rno,
^ esulterà in ^te la tua crea^zione, 
^\textbf{Spirito, vieni!}
\endverse

%%%%% STROFA
\beginverse		%Oppure \beginverse* se non si vuole il numero di fianco
%\memorize 		% <<< DECOMMENTA se si vuole utilizzarne la funzione
%\chordsoff		% <<< DECOMMENTA se vuoi una strofa senza accordi
^ Vergine del si^lenzio e della ^fe^de
^ l'Eterno ha posto in ^te la sua di^mora.^
^ Il tuo "sì" ri^suonerà per ^sem^pre:
^ l'Eterno ha posto in ^te la sua di^mora, 
^\textbf{Spirito, vieni!}
\endverse

%%%%% STROFA
\beginverse		%Oppure \beginverse* se non si vuole il numero di fianco
%\memorize 		% <<< DECOMMENTA se si vuole utilizzarne la funzione
%\chordsoff		% <<< DECOMMENTA se vuoi una strofa senza accordi
^ Tu nella Santa ^Casa accogli il ^do^no,
^ sei tu la porta ^che ci apre il C^ielo^
^ Con te la Chiesa ^canta la sua ^lo^de,
^ sei tu la porta ^che ci apre il Ci^elo, 
^\textbf{Spirito, vieni!}
\endverse

%%%%% STROFA
\beginverse		%Oppure \beginverse* se non si vuole il numero di fianco
%\memorize 		% <<< DECOMMENTA se si vuole utilizzarne la funzione
%\chordsoff		% <<< DECOMMENTA se vuoi una strofa senza accordi
^ Tu nella brezza ^parli al nostro ^cuo^re:
^ ascolteremo, ^Dio, la tua pa^rola;^
^ ci chiami a condi^videre il tuo a^mo^re:
^ ascolteremo, ^Dio, la tua pa^rola, 
^\textbf{Spirito, vieni!}
\endverse

\ifchorded
%%%%% RITORNELLO
\beginchorus
\textnote{\textbf{Finale }}
\[E]Luce di veri\[A]tà, \[E]fiamma di cari\[G#-]tà,
\[A]vincolo di uni\[E]tà, \[(C#-7*)]Spirito \[F#7]Santo A\[A]mo\[B]re.
\[E]Dona la liber\[A]tà, \[E]dona la santi\[G#-]tà,
\[A]fa' dell'umani\[E]tà \[(C#-7*)]il tuo \[D]canto di \[A]lo-\[B]o\[E]de. \[E*]
\endchorus
\fi

\endsong
%------------------------------------------------------------
%			FINE CANZONE
%------------------------------------------------------------





%MMM
%-------------------------------------------------------------
%			INIZIO	CANZONE
%-------------------------------------------------------------


%titolo: 	Madonna nera
%autore: 	Bagniewski
%tonalita: 	Sol



%%%%%% TITOLO E IMPOSTAZONI
\beginsong{Madonna nera}[by={Madonna di Czestochowa - J. Bagniewski}] 	% <<< MODIFICA TITOLO E AUTORE
\transpose{0} 						% <<< TRASPOSIZIONE #TONI (0 nullo)
\momenti{Canti Mariani}							% <<< INSERISCI MOMENTI	
% momenti vanno separati da ; e vanno scelti tra:
% Ingresso; Atto penitenziale; Acclamazione al Vangelo; Dopo il Vangelo; Offertorio; Comunione; Ringraziamento; Fine; Santi; Pasqua; Avvento; Natale; Quaresima; Canti Mariani; Battesimo; Prima Comunione; Cresima; Matrimonio; Meditazione; Spezzare del pane;
\ifchorded
	%\textnote{Tonalità migliore }	% <<< EV COMMENTI (tonalità originale/migliore)
\fi


%%%%%% INTRODUZIONE
\ifchorded
\vspace*{\versesep}
\musicnote{
\begin{minipage}{0.48\textwidth}
\textbf{Intro}
\hfill 
%( \eighthnote \, 80)   % <<  MODIFICA IL TEMPO
% Metronomo: \eighthnote (ottavo) \quarternote (quarto) \halfnote (due quarti)
\end{minipage}
} 	
\vspace*{-\versesep}
\beginverse*

\nolyrics

%---- Prima riga -----------------------------
\vspace*{-\versesep}
\[G]  % \[*D] per indicare le pennate, \rep{2} le ripetizioni


%---- Ev Indicazioni -------------------------			
%\textnote{\textit{(Oppure tutta la strofa)} }	

\endverse
\fi



%%%%% STROFA
\beginverse		%Oppure \beginverse* se non si vuole il numero di fianco
\memorize 		% <<< DECOMMENTA se si vuole utilizzarne la funzione
C'è una \[G]terra silen\[G7]ziosa dove o\[C]gnuno vuol tor\[E]nare
una \[A-]terra un dolce \[A7]volto
con due \[D]segni di vio\[D7]lenza
Sguardo in\[G]tenso e premu\[G7]roso
che ti \[C]chiede di affi\[A-]dare
la tua \[D]vita e il tuo \[D7]mondo in \[C*]mano a \[G]Lei. 
\endverse


%%%%% RITORNELLO
\beginchorus
\textnote{\textbf{Rit.}}
Ma\[G]donna, Madonna \[C]Nera, 
è \[D]dolce esser tuo \[G]figlio! \[D7]
Oh \[G]lascia, Madonna \[C]Nera, 
ch'io \[D]viva vicino a \[G]Te.
\endchorus



%%%%% STROFA
\beginverse		%Oppure \beginverse* se non si vuole il numero di fianco
%\memorize 		% <<< DECOMMENTA se si vuole utilizzarne la funzione
Lei ti ^calma e rasse^rena, Lei ti ^libera dal ^male,
perché ^sempre ha un cuore ^grande \brk per cia^scuno dei suoi ^figli.
Lei t'il^lumina il cam^mino \brk se le ^offri un po' d'a^more,
se ogni ^giorno parle^rai a ^Lei co^sì.
\endverse



%%%%% STROFA
\beginverse		%Oppure \beginverse* se non si vuole il numero di fianco
%\memorize 		% <<< DECOMMENTA se si vuole utilizzarne la funzione
Questo ^mondo in sub^buglio \brk cosa all'^uomo potrà of^frire?
Solo il ^volto di una ^Madre \brk pace ^vera può do^nare.
Nel suo ^sguardo noi cer^chiamo \brk quel sor^riso del Si^gnore
che ri^desta un po' di ^bene in ^fondo al ^cuor.
\endverse



\endsong
%------------------------------------------------------------
%			FINE CANZONE
%------------------------------------------------------------
%titolo{Magnificat}
%autore{Gragnani, Casini, Ciardella}
%album{}
%tonalita{Sol}
%famiglia{Liturgica}
%gruppo{}
%momenti{Magnificat}
%identificatore{magnificat_dio_ha_fatto}
%data_revisione{2011_12_31}
%trascrittore{Francesco Endrici}
\beginsong{Magnificat}[by={Gragnani, Casini, Ciardella}]
\beginverse
\[G]Dio \[C7+] ha fatto in \[B-7]me cose \[E-7]grandi.
\[C7+]Lui \[A-7] che guarda l'\[D]umile \[G7+]servo
e di\[C7+]sperde i su\[A-6]perbi
nell'or\[B7]goglio del \[E-]cuore. \[E]
\endverse
\beginchorus
\[A-]\[D7]L'anima \[G7+]mia \[C7+] esulta in \[A-]Dio \[B7]
mio salva\[E-]tore. \[E]
\[A-]\[D7]L'anima \[G7+]mia \[C7+] esulta in \[A-]Dio \[B7]
mio salva\[E-]tore. \[C]
La sua sal\[A-]vezza \[D]cante\[G]rò.
\endchorus
\beginverse
%\chordsoff
^Lui, ^ Onnipo^tente e ^santo.
^Lui ^ abbatte i ^grandi dai ^troni 
e sol^leva dal ^fango il suo ^umile ^servo. ^
\endverse
\beginverse
%\chordsoff
^Lui ^ miseri^cordia infi^nita,
^Lui ^ che rende ^povero il ^ricco 
e ri^colma di ^beni chi si af^fida al suo a^more. ^
\endverse
\beginverse
%\chordsoff
^Lui, ^ Amore ^sempre fe^dele.
^Lui ^ guida il suo ^servo Isra^ele 
e ri^corda il suo ^patto stabi^lito per ^sempre. ^
\endverse
\endsong


%-------------------------------------------------------------
%			INIZIO	CANZONE
%-------------------------------------------------------------


%titolo: 	Mani
%autore: 	Colombo
%tonalita: 	Do 



%%%%%% TITOLO E IMPOSTAZONI
\beginsong{Mani}[by={F. Colombo}] 	% <<< MODIFICA TITOLO E AUTORE
\transpose{-2} 						% <<< TRASPOSIZIONE #TONI (0 nullo)
\momenti{Comunione; Fine}							% <<< INSERISCI MOMENTI	
% momenti vanno separati da ; e vanno scelti tra:
% Ingresso; Atto penitenziale; Acclamazione al Vangelo; Dopo il Vangelo; Offertorio; Comunione; Ringraziamento; Fine; Santi; Pasqua; Avvento; Natale; Quaresima; Canti Mariani; Battesimo; Prima Comunione; Cresima; Matrimonio; Meditazione; Spezzare del pane;
\ifchorded
	%\textnote{Tonalità originale }	% <<< EV COMMENTI (tonalità originale/migliore)
\fi

%%%%%% INTRODUZIONE
\ifchorded
\vspace*{\versesep}
\textnote{Intro: \qquad \qquad  }%(\eighthnote 116) % <<  MODIFICA IL TEMPO
% Metronomo: \eighthnote (ottavo) \quarternote (quarto) \halfnote (due quarti)
\vspace*{-\versesep}
\beginverse*

\nolyrics

%---- Prima riga -----------------------------
\vspace*{-\versesep}
\[D] \[G] \[A] \[D]	 % \[*D] per indicare le pennate, \rep{2} le ripetizioni

%---- Ogni riga successiva -------------------
%\vspace*{-\versesep}
%\[G] \[C]  \[D]	

%---- Ev Indicazioni -------------------------			
%\textnote{\textit{(Oppure tutta la strofa)} }	

\endverse
\fi

%%%%% STROFA
\beginverse		%Oppure \beginverse* se non si vuole il numero di fianco
\memorize 		% <<< DECOMMENTA se si vuole utilizzarne la funzione
%\chordsoff		% <<< DECOMMENTA se vuoi una strofa senza accordi

Vor\[D]rei che le pa\[G]role mu\[A]tassero in pre\[D]ghiera
e \[G]rivederti o \[B-]Padre che \[G]dipingevi il \[A]cielo.
Sa\[D]pessi quante \[G]volte guar\[A]dando questo \[D]mondo
vor\[G]rei che tu tor\[B-]nassi a \[G]ritoc\[A]carne il \[D]cuore.
\vspace{\versesep}
Vor\[B-]rei che le mie \[F#-]mani a\[G]vessero la \[D]forza
per \[G]sostenere chi \[E]non può cammi\[A]nare.
Vor\[B-]rei che questo \[F#-]cuore che e\[G]splode in senti\[D]menti
\[G]diventasse \[B-]culla per \[G]chi non ha più \[A]madre.

\endverse

%%%%% RITORNELLO
\beginchorus
\textnote{\textbf{Rit.}}

\[D]Mani, prendi queste mie \[A]mani,
fanne vita, fanne a\[G]more
braccia aperte per ri\[B-]ceve\[A]re chi è solo.
\[D]Cuore, prendi questo mio \[A]cuore,
fa' che si spalanchi al \[G]mondo
germogliando per quegli \[B-]occhi
che non \[A]sanno pianger \[G]più. \[A]

\endchorus

%%%%% STROFA
\beginverse		%Oppure \beginverse* se non si vuole il numero di fianco
%\memorize 		% <<< DECOMMENTA se si vuole utilizzarne la funzione
%\chordsoff		% <<< DECOMMENTA se vuoi una strofa senza accordi

Sei ^tu lo spazio ^che de^sidero da ^sempre,
so ^che mi stringe^rai e ^mi terrai la ^mano.
^Fa' che le mie ^strade si ^perdano nel ^buio
ed ^io cammini ^dove cam^mine^resti ^Tu.
\vspace{\versesep}
Tu ^soffio della ^vita ^prendi la mia giovi^nezza
con ^le contraddi^zioni e le falsi^tà.
Stru^mento fa' che ^sia per an^nunciare il ^Regno
a ^chi per queste ^vie Tu ^chiami Be^ati.

\endverse

%%%%% STROFA
\beginverse		%Oppure \beginverse* se non si vuole il numero di fianco
%\memorize 		% <<< DECOMMENTA se si vuole utilizzarne la funzione
%\chordsoff		% <<< DECOMMENTA se vuoi una strofa senza accordi

Noi ^giovani di un ^mondo che can^cella i senti^menti
e in^scatola le ^forze nell'a^sfalto di cit^tà.
\vspace{\versesep}
Siamo \[B-]stanchi di guar\[F#-]dare 
siamo s\[G]tanchi di gri\[D]dare
ci hai chia\[G]mati siamo \[B-]Tuoi cam\[G]mineremo in\[A]sieme.

\endverse



%%%%% RITORNELLO
\beginchorus
\textnote{\textbf{Rit.}}

\[D]Mani, prendi queste mie \[A]mani,
fanne vita, fanne a\[G]more
braccia aperte per ri\[B-]ceve\[A]re chi è solo.
\[D]Cuore, prendi questo mio \[A]cuore,
fa' che si spalanchi al \[G]mondo
germogliando per quegli \[B-]occhi
che non \[A]sanno pianger\dots

\vspace{\versesep}

\[D]Mani, prendi queste nostre \[A]mani,
fanne vita, fanne a\[G]more
braccia aperte per ri\[B-]ceve\[A]re chi è solo.
\[D]Cuori, prendi questi nostri \[A]cuori,
 a che siano testi\[G]moni
che tu chiami ogni \[B-]uomo 
a far \[A]festa con \[G]Dio. \[A]  \[D*]

\endchorus
\endsong
%------------------------------------------------------------
%			FINE CANZONE
%------------------------------------------------------------



















%++++++++++++++++++++++++++++++++++++++++++++++++++++++++++++
%			CANZONE TRASPOSTA
%++++++++++++++++++++++++++++++++++++++++++++++++++++++++++++
% \ifchorded
% %decremento contatore per avere stesso numero
% \addtocounter{songnum}{-1} 
% \beginsong{Mani}[by={Colombo}] 	% <<< COPIA TITOLO E AUTORE
% 					% <<< TRASPOSIZIONE #TONI + - (0 nullo)
% %\preferflats  %SE VOGLIO FORZARE i bemolle come alterazioni
% %\prefersharps %SE VOGLIO FORZARE i # come alterazioni
% \ifchorded
% 	%\textnote{Tonalità originale}	% <<< EV COMMENTI (tonalità originale/migliore)
% \fi


% %%%%%% INTRODUZIONE
% \ifchorded
% \vspace*{\versesep}
% \textnote{Intro: \qquad \qquad  }%(\eighthnote 116) % <<  MODIFICA IL TEMPO
% % Metronomo: \eighthnote (ottavo) \quarternote (quarto) \halfnote (due quarti)
% \vspace*{-\versesep}
% \beginverse*

% \nolyrics

% %---- Prima riga -----------------------------
% \vspace*{-\versesep}
% \[D] \[G] \[A] \[D]	 % \[*D] per indicare le pennate, \rep{2} le ripetizioni

% %---- Ogni riga successiva -------------------
% %\vspace*{-\versesep}
% %\[G] \[C]  \[D]	

% %---- Ev Indicazioni -------------------------			
% %\textnote{\textit{(Oppure tutta la strofa)} }	

% \endverse
% \fi

% %%%%% STROFA
% \beginverse		%Oppure \beginverse* se non si vuole il numero di fianco
% \memorize 		% <<< DECOMMENTA se si vuole utilizzarne la funzione
% %\chordsoff		% <<< DECOMMENTA se vuoi una strofa senza accordi

% Vor\[D]rei che le pa\[G]role mu\[A]tassero in pre\[D]ghiera
% e \[G]rivederti o \[B-]Padre che \[G]dipingevi il \[A]cielo.
% Sa\[D]pessi quante \[G]volte guar\[A]dando questo \[D]mondo
% vor\[G]rei che tu tor\[B-]nassi a \[G]ritoc\[A]carne il \[D]cuore.
% \vspace{\versesep}
% Vor\[B-]rei che le mie \[F#-]mani a\[G]vessero la \[D]forza
% per \[G]sostenere chi \[E]non può cammi\[A]nare.
% Vor\[B-]rei che questo \[F#-]cuore che e\[G]splode in senti\[D]menti
% \[G]diventasse \[B-]culla per \[G]chi non ha più \[A]madre.

% \endverse

% %%%%% RITORNELLO
% \beginchorus
% \textnote{\textbf{Rit.}}

% \[D]Mani, prendi queste mie \[A]mani,
% fanne vita, fanne a\[G]more
% braccia aperte per ri\[B-]ceve\[A]re chi è solo.
% \[D]Cuore, prendi questo mio \[A]cuore,
% fa' che si spalanchi al \[G]mondo
% germogliando per quegli \[B-]occhi
% che non \[A]sanno pianger \[G]più. \[A]

% \endchorus

% %%%%% STROFA
% \beginverse		%Oppure \beginverse* se non si vuole il numero di fianco
% %\memorize 		% <<< DECOMMENTA se si vuole utilizzarne la funzione
% %\chordsoff		% <<< DECOMMENTA se vuoi una strofa senza accordi

% Sei ^tu lo spazio ^che de^sidero da ^sempre,
% so ^che mi stringe^rai e ^mi terrai la ^mano.
% ^Fa' che le mie ^strade si ^perdano nel ^buio
% ed ^io cammini ^dove cam^mine^resti ^Tu.
% \vspace{\versesep}
% Tu ^soffio della ^vita ^prendi la mia giovi^nezza
% con ^le contraddi^zioni e le falsi^tà.
% Stru^mento fa' che ^sia per an^nunciare il ^Regno
% a ^chi per queste ^vie Tu ^chiami Be^ati.

% \endverse

% %%%%% STROFA
% \beginverse		%Oppure \beginverse* se non si vuole il numero di fianco
% %\memorize 		% <<< DECOMMENTA se si vuole utilizzarne la funzione
% %\chordsoff		% <<< DECOMMENTA se vuoi una strofa senza accordi

% Noi ^giovani di un ^mondo che can^cella i senti^menti
% e in^scatola le ^forze nell'a^sfalto di cit^tà.
% \vspace{\versesep}
% Siamo \[B-]stanchi di guar\[F#-]dare 
% siamo s\[G]tanchi di gri\[D]dare
% ci hai chia\[G]mati siamo \[B-]Tuoi cam\[G]mineremo in\[A]sieme.

% \endverse




% %%%%% RITORNELLO
% \beginchorus
% \textnote{\textbf{Rit.}}

% \[D]Mani, prendi queste mie \[A]mani,
% fanne vita, fanne a\[G]more
% braccia aperte per ri\[B-]ceve\[A]re chi è solo.
% \[D]Cuore, prendi questo mio \[A]cuore,
% fa' che si spalanchi al \[G]mondo
% germogliando per quegli \[B-]occhi
% che non \[A]sanno pianger\dots

% \vspace{\versesep}

% \[D]Mani, prendi queste nostre \[A]mani,
% fanne vita, fanne a\[G]more
% braccia aperte per ri\[B-]ceve\[A]re chi è solo.
% \[D]Cuori, prendi questi nostri \[A]cuori,
%  a che siano testi\[G]moni
% che tu chiami ogni \[B-]uomo 
% a far \[A]festa con \[G]Dio. \[A]  \[D*]

% \endchorus

% \endsong

% \fi
%++++++++++++++++++++++++++++++++++++++++++++++++++++++++++++
%			FINE CANZONE TRASPOSTA
%++++++++++++++++++++++++++++++++++++++++++++++++++++++++++++
%-------------------------------------------------------------
%			INIZIO	CANZONE
%-------------------------------------------------------------


%titolo: 	Mani che si stringono
%autore: 	M. Gioia
%tonalita: 	Do 



%%%%%% TITOLO E IMPOSTAZONI
\beginsong{Mani che si stringono}[by={M. Gioia}] 	% <<< MODIFICA TITOLO E AUTORE
\transpose{0} 						% <<< TRASPOSIZIONE #TONI (0 nullo)
%\preferflats  %SE VOGLIO FORZARE i bemolle come alterazioni
%\prefersharps %SE VOGLIO FORZARE i # come alterazioni
\momenti{}							% <<< INSERISCI MOMENTI	
% momenti vanno separati da ; e vanno scelti tra:
% Ingresso; Atto penitenziale; Acclamazione al Vangelo; Dopo il Vangelo; Offertorio; Comunione; Ringraziamento; Fine; Santi; Pasqua; Avvento; Natale; Quaresima; Canti Mariani; Battesimo; Prima Comunione; Cresima; Matrimonio; Meditazione; Spezzare del pane;
\ifchorded
	%\textnote{Tonalità migliore }	% <<< EV COMMENTI (tonalità originale/migliore)
\fi


%%%%%% INTRODUZIONE
\ifchorded
\vspace*{\versesep}
\textnote{Intro: \qquad \qquad  }%(\eighthnote 116) % <<  MODIFICA IL TEMPO
% Metronomo: \eighthnote (ottavo) \quarternote (quarto) \halfnote (due quarti)
\vspace*{-\versesep}
\beginverse*

\nolyrics

%---- Prima riga -----------------------------
\vspace*{-\versesep}
\[C] \[F] \[C]	 % \[*D] per indicare le pennate, \rep{2} le ripetizioni

%---- Ogni riga successiva -------------------
%\vspace*{-\versesep}
%\[G] \[C]  \[D]	

%---- Ev Indicazioni -------------------------			
%\textnote{\textit{(Oppure tutta la strofa)} }	

\endverse
\fi




%%%%% STROFA
\beginverse		%Oppure \beginverse* se non si vuole il numero di fianco
\memorize 		% <<< DECOMMENTA se si vuole utilizzarne la funzione
%\chordsoff		% <<< DECOMMENTA se vuoi una strofa senza accordi

\[C]Il sole scende \[F] \brk è quasi notte or\[C]mai, \[C7]
dai re\[F]stiamo ancora insieme un \[G]po',  \brk  meno buio sa\[C]rà. \[C7]
La pa\[F]rola del Si\[G]gnore  \brk come \[E-]luce in mezzo a \[A-]noi,
ci ris\[D7]chiara,\[F] e ci unisce a \[G4]Lui. \[G]


\endverse




%%%%% RITORNELLO
\beginchorus
\textnote{\textbf{Rit.}}

\[C]Mani \[F]che si stringono \[C]forte,
\[D-7]in un cerchio di \[C]sguardi 
\[F]che s'incrocia\[G4]no, \[G]
come un ab\[A-]braccio stret\[D-]to-o,
per sen\[G]tire che la \[C7+]chiesa vive,
che \[F7+]vive dentro \[B&]noi, insieme a \[G4]noi. \[G]
E so-no \[C]mani \[F4]che si tendono in \[C]alto,
\[D-7]che si aprono \[C]grandi 
\[F]per raccoglie\[G4]re \[G]
quella \[A-]forza immensa-a \[D-7]
che il Si\[G]gnore mette in \[C7+]fondo al cuore,
annun\[F7+]ciare a tutti \[B&]che Dio ci \[G4]ama. \[G]


\endchorus



%%%%% STROFA
\beginverse		%Oppure \beginverse* se non si vuole il numero di fianco
%\memorize 		% <<< DECOMMENTA se si vuole utilizzarne la funzione
%\chordsoff		% <<< DECOMMENTA se vuoi una strofa senza accordi

^Noi veglieremo ^con le lampa^de, ^
aspet^tando nella notte, ^ \brk  finché giorno sa^rà. ^
E la ^voce del Si^gnore, \brk  all'improv^viso giunge^rà,
saremo ^pronti,  \brk ^ saremo amici s^uoi. ^


\endverse



%%%%%% EV. FINALE

\beginchorus %oppure \beginverse*
\vspace*{1.3\versesep}
\textnote{Finale } %<<< EV. INDICAZIONI

...Dio ci \[F]ama! \[F] \[C]

\endchorus  %oppure \endverse





\endsong
%------------------------------------------------------------
%			FINE CANZONE
%------------------------------------------------------------



%-------------------------------------------------------------
%			INIZIO	CANZONE
%-------------------------------------------------------------


%titolo: 	Maranathà
%autore: 	Buttazzo, Ricci
%tonalita: 	Sim



%%%%%% TITOLO E IMPOSTAZONI
\beginsong{Maranathà}[by={Buttazzo, Ricci}] 	% <<< MODIFICA TITOLO E AUTORE
\transpose{0} 						% <<< TRASPOSIZIONE #TONI (0 nullo)
\momenti{Ingresso; Avvento; Canti Mariani; Fine}							% <<< INSERISCI MOMENTI	
% momenti vanno separati da ; e vanno scelti tra:
% Ingresso; Atto penitenziale; Acclamazione al Vangelo; Dopo il Vangelo; Offertorio; Comunione; Ringraziamento; Fine; Santi; Pasqua; Avvento; Natale; Quaresima; Canti Mariani; Battesimo; Prima Comunione; Cresima; Matrimonio; Meditazione;
\ifchorded
	%\textnote{Tonalità originale }	% <<< EV COMMENTI (tonalità originale/migliore)
\fi




%%%%%% INTRODUZIONE
\ifchorded
\vspace*{\versesep}
\textnote{Intro: \qquad \qquad  }%(\eighthnote 116) % << MODIFICA IL TEMPO
% Metronomo: \eighthnote (ottavo) \quarternote (quarto) \halfnote (due quarti)
\vspace*{-\versesep}
\beginverse*

\nolyrics

%---- Prima riga -----------------------------
\vspace*{-\versesep}
\[(D)] \[B-]  \[F#-] \[G] \[A] \[D] % \[*D] per indicare le pennate, \rep{2} le ripetizioni

%---- Ogni riga successiva -------------------
%\vspace*{-\versesep}
%\[G] \[C]  \[D]	

%---- Ev Indicazioni -------------------------			
\textnote{\textit{(Come mezzo ritornello)} }	

\endverse
\fi



%%%%% RITORNELLO
\textnote{\textbf{Rit.}}
\beginchorus
\[D]Marana\[B-]thà, vieni Si\[F#-]gnor!
Verso te, Ge\[G]sù, le mani \[A]noi levi\[D]am.
\[D]Marana\[B-]thà, vieni Si\[F#-]gnor!
Prendici con \[G]te e salva\[A]ci Si\[D]gnor.
\endchorus


%%%%% STROFA
\beginverse		%Oppure \beginverse* se non si vuole il numero di fianco
\memorize 		% <<< DECOMMENTA se si vuole utilizzarne la funzione
%\chordsoff		% <<< DECOMMENTA se vuoi una strofa senza accordi

\[D]Guardo verso le mon\[A]tagne, 
\echo{Guardo verso le mon\[D]tagne}
\[(D)]donde mi verrà il soc\[A]corso, 
\echo{donde mi verrà il soc\[D]corso}
\[(D)]il soccorso vien da \[A]Dio, 
\echo{il soccorso vien da \[D]Dio}
\[(D)]che ha creato il mondo in\[E-]tero. 
\echo{che ha creato il mondo in\[D]tero}
\endverse



%%%%% STROFE
\beginverse	
\chordsoff
Sorgi con il tuo Amore, la Tua luce splenderà,
ogni ombra svanirà, la tua Gloria apparirà.	
\endverse
\beginverse
\chordsoff	
Santo è nostro Signor, il peccato Egli levò,
dalla morte ci salvò, e la vita a noi donò.
\endverse
\beginverse
\chordsoff	
Mio Signor son peccatore, a Te apro il mio cuore,
fa’ di me quello che vuoi e per sempre in Te vivrò.
\endverse
\beginverse
\chordsoff	
La Parola giungerà sino ad ogni estremità,
testimoni noi sarem della tua verità.
\endverse
\beginverse
\chordsoff	
Tu sei la mia libertà, solo in Te potrò sperar,
ho fiducia in te Signor, la mia vita cambierai.	
\endverse
\beginverse
\chordsoff	
Mi consegno a te Signor, vieni dentro il mio cuor,
ti ricevo o Salvator, tu sei il mio liberator.
\endverse
\beginverse
\chordsoff	
Benedicici o Signor, sii custode ai nostri cuor,
giorno e notte veglierai, e con noi sempre sarai.
\endverse
\beginverse
\chordsoff	
Ringraziamo te o Signor, a te Padre Creator,
allo Spirito d’Amor, vieni presto, o Signor.
\endverse



\endsong
%-------------------------------------------------------------
%			INIZIO	CANZONE
%-------------------------------------------------------------


%titolo: 	Mi hai tenuto per mano
%autore: 	Meregalli
%tonalita: 	 



%%%%%% TITOLO E IMPOSTAZONI
\beginsong{Mi hai tenuto per mano}[by={G. Meregalli}] 	% <<< MODIFICA TITOLO E AUTORE
\transpose{0} 						% <<< TRASPOSIZIONE #TONI (0 nullo)
\momenti{Comunione; Cresima; Prima Comunione}							% <<< INSERISCI MOMENTI	
% momenti vanno separati da ; e vanno scelti tra:
% Ingresso; Atto penitenziale; Acclamazione al Vangelo; Dopo il Vangelo; Offertorio; Comunione; Ringraziamento; Fine; Santi; Pasqua; Avvento; Natale; Quaresima; Canti Mariani; Battesimo; Prima Comunione; Cresima; Matrimonio; Meditazione; Spezzare del pane;
\ifchorded
	%\textnote{Tonalità originale }	% <<< EV COMMENTI (tonalità originale/migliore)
\fi

%%%%%% INTRODUZIONE
\ifchorded
\vspace*{\versesep}
\musicnote{
\begin{minipage}{0.48\textwidth}
\textbf{Intro}
\hfill 
%( \eighthnote \, 80)   % <<  MODIFICA IL TEMPO
% Metronomo: \eighthnote (ottavo) \quarternote (quarto) \halfnote (due quarti)
\end{minipage}
} 	
\vspace*{-\versesep}
\beginverse*
\nolyrics

%---- Prima riga -----------------------------
\vspace*{-\versesep}
\[C] \rep{4} % \[*D] per indicare le pennate, \rep{2} le ripetizioni

%---- Ogni riga successiva -------------------
%\vspace*{-\versesep}
%\[G] \[C]  \[D]	

%---- Ev Indicazioni -------------------------			
%\textnote{\textit{(Oppure tutta la strofa)} }	

\endverse
\fi




%%%%% STROFA
\beginverse		%Oppure \beginverse* se non si vuole il numero di fianco
%\memorize 		% <<< DECOMMENTA se si vuole utilizzarne la funzione
%\chordsoff		% <<< DECOMMENTA se vuoi una strofa senza accordi

\textnote{\textbf{Mistero dell'esistenza}}

\[C]Gli occhi videro il mondo
durante giorni che non ricordo.
Il \[G]tempo seguì i miei passi
per un mistero che non co\[C]nosco.
Un \[A-]uomo tese le mani,
raccolse il bimbo che aveva in \[G]dono
e \[F]buona vidi mia \[G]madre:
il mio res\[A-]piro fu pianto e \[E]sogno. \[E]

\vspace*{\versesep}

Il \[C]tempo indurì le \[G]mani,
ogni mio \[A-]sogno si fece \[E]terra.
La \[F]strada a chi non ha \[G]ali,
il primo \[A-]passo fu sfida e \[E]rabbia;
pa\[C]role a nutrir la \[G]voce
e luci \[A-]senza veder la \[E]strada:
sul \[F]ciglio stavo se\[G]duto
e fui bam\[A-]bino tra stelle e \[E]sabbia.

\endverse



%%%%%% EV. INTERMEZZO
\ifchorded
\beginverse*
\vspace*{1.3\versesep}
{
	\nolyrics
	\textnote{\textit{(il tempo cambia)}}
	

	%---- Prima riga -----------------------------
	\vspace*{-\versesep}
	\[E] \[E]  \[E]

	%---- Ev Indicazioni -------------------------			
	%\textnote{\textit{(ripetizione della strofa)}} 
	 
}
\endverse
\fi



%%%%% RITORNELLO
\beginchorus
%\textnote{\textbf{Rit.}}
\memorize


\[C]Io non ho più pa\[G]ura,
\[F]vedo un torrente di \[G]volti
che scorre tra gli \[C]argini
di una stessa avven\[G]tura:
\[F] chi è vicino alla \[C]fonte,
\[F] chi già lascia la \[C]valle
\[F] chi ha percorso \[B&]tempo
la grande pia\[C]nura... \[C]  \[C] \[C] 


\endchorus
\beginchorus


Da\[F]vanti al mio \[C]nome 
\[B&] ti sei fer\[C]mato
e dal \[F]primo res\[C]piro
\[B&] mi hai te\[C]nuto per \[F]ma\[C]no,
\[B&] mi hai te\[C]nuto per \[F]ma\[C]no, 
\[B&] mi hai te\[C]nuto per \[F]ma\[F]no, \[F] \[F]


\endchorus




%%%%%% EV. INTERMEZZO
\beginverse*
\vspace*{1.3\versesep}
{
	\nolyrics
	%\textnote{Intermezzo strumentale}
	
	\ifchorded

	%---- Prima riga -----------------------------
	\vspace*{-\versesep}
	\[C] \rep{4}


	\fi
	%---- Ev Indicazioni -------------------------			
	%\textnote{\textit{(ripetizione della strofa)}} 
	 
}
\vspace*{\versesep}
\endverse



%%%%% STROFA
\beginverse		%Oppure \beginverse* se non si vuole il numero di fianco
%\memorize 		% <<< DECOMMENTA se si vuole utilizzarne la funzione
\chordsoff		% <<< DECOMMENTA se vuoi una strofa senza accordi

\textnote{\textbf{Mistero dell’Incarnazione}}
 
Fu ^quando mi seppi dono
che dire grazie fu dire molto.
A ^tanti chiesi una meta:
ebbi il consiglio di stare ^pronto.
Te^metti la verità
quando il coraggio si fece or^goglio.
Qual^cuno mi disse:^cerca 
con l’umil^tà di chi accoglie un ^dono. ^

\vspace*{\versesep}

La ^storia si è fatta ^casa;
il tempo, at^tesa di una sal^vezza.
L’a^more di chi ha cre^ato
si è fatto im^pronta di ogni spe^ranza. 
La ^carne si è fatta ^segno,
custode e ^tempio di tene^rezza.
Il ^Padre si è fatto ^Figlio
e fu bam^bino fra stelle e ^sabbia.
\endverse




%%%%% RITORNELLO
\beginchorus
%\textnote{\textbf{Rit.}}
\chordsoff

^Io non ho più pa^ura,
^vedo la luce e ri^torna la pace,
la str^ada si è fatta si^cura;
^ ora vedo la f^onte
^ e la grande pia^nura:
^ la potenza crea^trice si è fatta crea^tura. ^ ^ ^


\endchorus
\beginchorus


Da\[F]vanti al mio \[C]volto 
\[B&] ti sei chi\[C]nato
e  fra\[F]tello nel v\[C]iaggio
\[B&] mi hai te\[C]nuto per \[F]ma\[C]no,
\[B&] mi hai te\[C]nuto per \[F]ma\[C]no, 
\[B&] mi hai te\[C]nuto per \[F]ma\[F]no, \[F] \[F]

\endchorus




%%%%%% EV. INTERMEZZO
\beginverse*
\vspace*{1.3\versesep}
{
	\nolyrics
	%\textnote{Intermezzo strumentale}
	
	\ifchorded

	%---- Prima riga -----------------------------
	\vspace*{-\versesep}
	\[C] \rep{4}


	\fi
	%---- Ev Indicazioni -------------------------			
	%\textnote{\textit{(ripetizione della strofa)}} 
	 
}
\vspace*{\versesep}
\endverse






%%%%% STROFA
\beginverse		%Oppure \beginverse* se non si vuole il numero di fianco
%\memorize 		% <<< DECOMMENTA se si vuole utilizzarne la funzione
\chordsoff		% <<< DECOMMENTA se vuoi una strofa senza accordi

\textnote{\textbf{Mistero della morte}}

Ho visto semi d’amore unire mani,
destini e corpi,
portare consolazione
a chi conobbe dolore e morte;
ho visto colmare abissi
con un abbraccio riconciliante,
ho visto una croce infissa
dove il perdono si fece sangue.

\vspace*{\versesep}

La strada si è fatta folla,
il monte, pietra dell’agonia.
Il cielo si è fatto tenda
della promessa e della memoria.
La voce si è fatta grido,
le mani preda della follia.
Il corpo si è fatto pane
per tutti gli uomini senza storia.

\endverse



%%%%% RITORNELLO
\beginchorus
%\textnote{\textbf{Rit.}}
\chordsoff

^Io non ho più pa^ura,
^vedo l’amore che ^scende la valle
e con^sola la grande pia^nura;
^ è sospinto alla ^foce 
^ sopra un legno cru^dele:
^ condivide la ^sorte di ogni crea^tura. ^ ^ ^


\endchorus
\beginchorus


Da\[F]vanti a una c\[C]roce, 
\[B&] ti sei pie\[C]gato,
le tue \[F]mani nel l\[C]egno,
\[B&] mi han te\[C]nuto per \[F]ma\[C]no,
\[B&] mi han te\[C]nuto per \[F]ma\[C]no, 
\[B&] mi han te\[C]nuto per \[F]ma\[F]no, \[F] \[F]

\endchorus




%%%%%% EV. INTERMEZZO
\beginverse*
\vspace*{1.3\versesep}
{
	\nolyrics
	%\textnote{Intermezzo strumentale}
	
	\ifchorded

	%---- Prima riga -----------------------------
	\vspace*{-\versesep}
	\[C] \rep{4}


	\fi
	%---- Ev Indicazioni -------------------------			
	%\textnote{\textit{(ripetizione della strofa)}} 
	 
}
\vspace*{\versesep}
\endverse






%%%%% STROFA
\beginverse		%Oppure \beginverse* se non si vuole il numero di fianco
%\memorize 		% <<< DECOMMENTA se si vuole utilizzarne la funzione
\chordsoff		% <<< DECOMMENTA se vuoi una strofa senza accordi

\textnote{\textbf{Mistero della risurrezione}}


La storia si fece luce durante un’ora di poesia.
L’amore divenne forza che,
dolce, vince ogni resistenza.
La terra si fece altare di imprevedibile liturgia.
La morte si fece figlia riconciliata dell’esistenza.

\vspace*{\versesep}

Fu allora che vidi il cielo
sacra dimora dell’infinito;
fu allora che amai la terra
forma possibile dell’assoluto;
feconda di una speranza
che lega volti, cuori e destini,
accolgo la verità
fino a che tutto non sia compiuto.

\endverse



%%%%% RITORNELLO
\beginchorus
%\textnote{\textbf{Rit.}}
\chordsoff

^Io non ho più pa^ura,
^vedo una grande dis^tesa di pace
acc^ogliere l’acqua più ^pura:
^ chi è vicino alla ^fonte, 
^ chi già lascia la ^valle,
^ chi è già parte di ^Dio nella grande dis^tesa. ^ ^ ^



\endchorus
\beginchorus


Da\[F]vanti al mio \[C]nome 
\[B&] ti sei fer\[C]mato
e dal \[F]primo res\[C]piro
\[B&] mi hai te\[C]nuto per \[F]ma\[C]no,
\[B&] mi hai te\[C]nuto per \[F]ma\[C]no, 
\[B&] mi hai te\[C]nuto per \[F]mano...





\endchorus







%%%%%% EV. FINALE
\ifchorded
\beginchorus %oppure \beginverse*
\vspace*{1.3\versesep}
\textnote{Finale strumentale} %<<< EV. INDICAZIONI
\nolyrics

\[C] \[B&] \[C] \[B&] \[C] 

%---- Ogni riga successiva -------------------
\vspace*{-\versesep}
\[F] \[F] \[F] \[F*]

\endchorus  %oppure \endverse
\fi



\endsong
%------------------------------------------------------------
%			FINE CANZONE
%------------------------------------------------------------




%NNN
%-------------------------------------------------------------
%			INIZIO	CANZONE
%-------------------------------------------------------------


%titolo: 	Nella tua parola
%autore: 	
%tonalita: 	Re

%%%%%% TITOLO E IMPOSTAZONI
\beginsong{Nella tua parola}[by={P. Sequeri}] 	% <<< MODIFICA TITOLO E AUTORE
\transpose{0} 						% <<< TRASPOSIZIONE #TONI (0 nullo)
\momenti{Acclamazione al Vangelo; Dopo il Vangelo; Quaresima}							% <<< INSERISCI MOMENTI	
% momenti vanno separati da ; e vanno scelti tra:
% Ingresso; Atto penitenziale; Acclamazione al Vangelo; Dopo il Vangelo; Offertorio; Comunione; Ringraziamento; Fine; Santi; Pasqua; Avvento; Natale; Quaresima; Canti Mariani; Battesimo; Prima Comunione; Cresima; Matrimonio; Meditazione;
\ifchorded
	%\textnote{Tonalità originale }	% <<< EV COMMENTI (tonalità originale/migliore)
\fi

%%%%%% INTRODUZIONE
\ifchorded
\vspace*{\versesep}
\musicnote{
\begin{minipage}{0.48\textwidth}
\textbf{Intro}
\hfill 
%( \eighthnote \, 80)   % <<  MODIFICA IL TEMPO
% Metronomo: \eighthnote (ottavo) \quarternote (quarto) \halfnote (due quarti)
\end{minipage}
} 	
\vspace*{-\versesep}
\beginverse*

\nolyrics

%---- Prima riga -----------------------------
\vspace*{-\versesep}
\[D] \[A] \[D]	 % \[*D] per indicare le pennate, \rep{2} le ripetizioni

%---- Ogni riga successiva -------------------
%\vspace*{-\versesep}
%\[G] \[C]  \[D]	

%---- Ev Indicazioni -------------------------			
%\textnote{\textit{(Oppure tutta la strofa)} }	

\endverse
\fi

%%%%% RITORNELLO
\beginchorus
\textnote{\textbf{Rit.}}

\[D]Nella tua pa\[A/Si&dim]rola noi
\[B-]camminiamo ins\[E-]ieme a te,
\[G]ti preghiamo \[F#]resta con \[B-]noi. \rep{2}

\endchorus

%%%%% STROFA
\beginverse*		%Oppure \beginverse* se non si vuole il numero di fianco
%\memorize 		% <<< DECOMMENTA se si vuole utilizzarne la funzione
%\chordsoff		% <<< DECOMMENTA se vuoi una strofa senza accordi

\[B-]Luce dei miei passi, 
\[E-]guida al mio cam\[B-]mino
\[E-]è la tua pa\[B-]ro-o-\[F#*]la. \[B-*] \[F#]

\endverse

\endsong
%------------------------------------------------------------
%			FINE CANZONE
%------------------------------------------------------------
%-------------------------------------------------------------
%			INIZIO	CANZONE
%-------------------------------------------------------------


%titolo: 	Niente vale di più
%autore: 	
%tonalita: 	Re

%%%%%% TITOLO E IMPOSTAZONI
\beginsong{Niente vale di più}[by={}] 	% <<< MODIFICA TITOLO E AUTORE
\transpose{0} 						% <<< TRASPOSIZIONE #TONI (0 nullo)
\momenti{Comunione; Fine; Prima Comunione}							% <<< INSERISCI MOMENTI	
% momenti vanno separati da ; e vanno scelti tra:
% Ingresso; Atto penitenziale; Acclamazione al Vangelo; Dopo il Vangelo; Offertorio; Comunione; Ringraziamento; Fine; Santi; Pasqua; Avvento; Natale; Quaresima; Canti Mariani; Battesimo; Prima Comunione; Cresima; Matrimonio; Meditazione;
\ifchorded
	%\textnote{Tonalità originale }	% <<< EV COMMENTI (tonalità originale/migliore)
\fi

%%%%%% INTRODUZIONE
\ifchorded
\vspace*{\versesep}
\textnote{Intro: \qquad \qquad  }%(\eighthnote 116) % << MODIFICA IL TEMPO
% Metronomo: \eighthnote (ottavo) \quarternote (quarto) \halfnote (due quarti)
\vspace*{-\versesep}
\beginverse*

\nolyrics

%---- Prima riga -----------------------------
\vspace*{-\versesep}
\[D] \[A] \[D]	 % \[*D] per indicare le pennate, \rep{2} le ripetizioni

%---- Ogni riga successiva -------------------
%\vspace*{-\versesep}
%\[G] \[C]  \[D]	

%---- Ev Indicazioni -------------------------			
%\textnote{\textit{(Oppure tutta la strofa)} }	

\endverse
\fi

%%%%% STROFA
\beginverse		%Oppure \beginverse* se non si vuole il numero di fianco
\memorize 		% <<< DECOMMENTA se si vuole utilizzarne la funzione
%\chordsoff		& <<< DECOMMENTA se vuoi una strofa senza accordi

Non \[D]so cosa voglio rag\[A]giungere
non \[B-]so quali stelle rac\[F#-]cogliere
pe\[G]rò ho una \[A]gioia da \[D]vive\[B-]re
dai \[E-]dammi la mano cam\[A]mina con \[A7]me.
Io \[D]credo in un mondo fan\[A]tastico
che u\[B-]nisce il sorriso degli \[F#-]uomini
non \[G]dirmi che è un \[A]sogno impos\[D]sibi\[B-]le
se as\[C]colti il tuo cuore al\[A]lora sapr\[A7]ai:

\endverse


%%%%% RITORNELLO
\beginchorus
\textnote{\textbf{Rit.}}

Che \[D]niente è più bello di una \[A]vita vissuta,
di una \[B-]pace donata di un a\[G]more fe\[A]dele,
di un fra\[G]tello che \[A]crede.
Che \[D]niente è più grande di una \[A]voce che chiama
il tuo \[B-]nome nel mondo di una \[G]vita che an\[A]nuncia, 
la pa\[G]rola che \[A]salva. 
Ti \[G]svelo un segreto: se \[D]cerchi un amico 
il Si\[Bb]gnore ti \[C]sta amando \[D]già 
il Si\[Bb]gnore ti \[C]sta amando \[D]già. \[A]
\endchorus

%%%%% STROFA
\beginverse		%Oppure \beginverse* se non si vuole il numero di fianco
%\memorize 		% <<< DECOMMENTA se si vuole utilizzarne la funzione
%\chordsoff		% <<< DECOMMENTA se vuoi una strofa senza accordi

Io ^chiedo il coraggio di ^vivere 
fra^tello alle voci che at^tendono 
spe^ranze che ^volano ^libe^re 
più ^alte del sole ^raggiungono ^Te. 
Non ^so quali volti co^noscerò 
e ^quante illusioni attra^verserai
se un ^giorno si ^leverà i^nuti^le 
as^colta il tuo cuore al^lora sapr^ai:

\endverse

%%%%% STROFA
\beginverse		%Oppure \beginverse* se non si vuole il numero di fianco
%\memorize 		% <<< DECOMMENTA se si vuole utilizzarne la funzione
%\chordsoff		% <<< DECOMMENTA se vuoi una strofa senza accordi

Per ^ogni momento che ^tu mi dai
do^mando la forza di ^credere 
nel ^gesto d'a^more che ^libe^ra 
e ^questo mio canto pregh^iera sa^rà. 
Se ^scopri con gioia la ^verità 
racch^iusa negli occhi degli ^uomini 
se ^cerchi un te^soro per ^vive^re 
as^colta il tuo cuore al^lora sapr^ai:

\endverse

%%%%%% EV. CHIUSURA SOLO STRUMENTALE
\ifchorded
\beginchorus %oppure \beginverse*
\vspace*{1.3\versesep}
\textnote{Chiusura } %<<< EV. INDICAZIONI

\[D*]

\endchorus  %oppure \endverse
\fi



\endsong
%------------------------------------------------------------
%			FINE CANZONE
%------------------------------------------------------------

%-------------------------------------------------------------
%			INIZIO	CANZONE
%-------------------------------------------------------------


%titolo: 	Noi saremo il pane
%autore: 	Fusco
%tonalita: 	Do 



%%%%%% TITOLO E IMPOSTAZONI
\beginsong{Noi saremo il pane}[by={M. G. Fusco}] 	% <<< MODIFICA TITOLO E AUTORE
\transpose{0} 						% <<< TRASPOSIZIONE #TONI (0 nullo)
\momenti{Offertorio; Prima Comunione; Spezzare del Pane}							% <<< INSERISCI MOMENTI	
% momenti vanno separati da ; e vanno scelti tra:
% Ingresso; Atto penitenziale; Acclamazione al Vangelo; Dopo il Vangelo; Offertorio; Comunione; Ringraziamento; Fine; Spezzare del Pane; Santi; Pasqua; Avvento; Natale; Quaresima; Canti Mariani; Battesimo; Prima Comunione; Cresima; Matrimonio; Meditazione;
\ifchorded
	%\textnote{Tonalità originale }	% <<< EV COMMENTI (tonalità originale/migliore)
\fi


%%%%%% INTRODUZIONE
\ifchorded
\vspace*{\versesep}
\textnote{Intro: \qquad \qquad  }%(\eighthnote 116) % << MODIFICA IL TEMPO
% Metronomo: \eighthnote (ottavo) \quarternote (quarto) \halfnote (due quarti)
\vspace*{-\versesep}
\beginverse*

\nolyrics

%---- Prima riga -----------------------------
\vspace*{-\versesep}
\[C] \[F]  \[C]	 \rep{2} % \[*D] per indicare le pennate, \rep{2} le ripetizioni

%---- Ogni riga successiva -------------------
%\vspace*{-\versesep}
%\[G] \[C]  \[D]	

%---- Ev Indicazioni -------------------------			
%\textnote{\textit{(Oppure tutta la strofa)} }	

\endverse
\fi




%%%%% STROFA
\beginverse		%Oppure \beginverse* se non si vuole il numero di fianco
\memorize 		% <<< DECOMMENTA se si vuole utilizzarne la funzione
%\chordsoff		& <<< DECOMMENTA se vuoi una strofa senza accordi

Un \[C]chicco da \[F]solo che \[C]fa?
Non fa un \[F]campo di grano né un \[G]pane!
Un \[C]chicco da \[F]solo non po\[C]trà
esser la \[F]gioia di chi ha \[G]fame!
\[D-7]Ma uniti in\[E-]sieme
tanti \[D-7]chicchi un solo \[G]pane!

\endverse




%%%%% RITORNELLO
\beginchorus
\textnote{\textbf{Rit.}}

\[C]Noi saremo il \[E-]pane, \[F]noi sarem l'a\[E-]more
\[F]noi sarem la \[G]gioia per un \[A-]mondo che ha
fame d'infi\[D-]ni\[G]to!
\[C]Noi saremo il \[E-]pane, \[F]noi sarem l'a\[E-]more
\[F]noi sarem la \[G]gioia per un \[A-]mondo che ha
\[G]fame di \[C]Te!

\endchorus



%%%%%% EV. INTERMEZZO
\beginverse*
\vspace*{1.3\versesep}
{
	\nolyrics
	\textnote{Breve intermezzo }
	
	\ifchorded

	%---- Prima riga -----------------------------
	\vspace*{-\versesep}
	\[C] \[F]  \[C]	 

	\fi
	%---- Ev Indicazioni -------------------------			
	%\textnote{\textit{(ripetizione della strofa)}} 
	 
}
\vspace*{\versesep}
\endverse




%%%%% STROFA
\beginverse		%Oppure \beginverse* se non si vuole il numero di fianco
%\memorize 		% <<< DECOMMENTA se si vuole utilizzarne la funzione
\chordsoff		% <<< DECOMMENTA se vuoi una strofa senza accordi

Un ^acino ^solo che ^fa?
Non è ^uva che matura sui ^colli!
Un ^uomo ^solo non po^trà
essere ^"segno" dell'a^more
^ma noi invi^tati
tutti in^sieme Chiesa ^viva!

\endverse






\endsong
%------------------------------------------------------------
%			FINE CANZONE
%------------------------------------------------------------


%-------------------------------------------------------------
%			INIZIO	CANZONE
%-------------------------------------------------------------


%titolo: 	Noi veglieremo
%autore: 	Machetta
%tonalita: 	Re 



%%%%%% TITOLO E IMPOSTAZONI
\beginsong{Noi veglieremo}[by={D. Machetta}]	% <<< MODIFICA TITOLO E AUTORE
\transpose{-2} 						% <<< TRASPOSIZIONE #TONI (0 nullo)
\momenti{Avvento; Congedo}							% <<< INSERISCI MOMENTI	
% momenti vanno separati da ; e vanno scelti tra:
% Ingresso; Atto penitenziale; Acclamazione al Vangelo; Dopo il Vangelo; Offertorio; Comunione; Ringraziamento; Fine; Santi; Pasqua; Avvento; Natale; Quaresima; Canti Mariani; Battesimo; Prima Comunione; Cresima; Matrimonio; Meditazione; Spezzare del pane;
\ifchorded
	%\textnote{Tonalità migliore }	% <<< EV COMMENTI (tonalità originale/migliore)
\fi


%%%%%% INTRODUZIONE
\ifchorded
\vspace*{\versesep}
\musicnote{
\begin{minipage}{0.48\textwidth}
\textbf{Intro}
\hfill 
%( \eighthnote \, 80)   % <<  MODIFICA IL TEMPO
% Metronomo: \eighthnote (ottavo) \quarternote (quarto) \halfnote (due quarti)
\end{minipage}
} 	
\vspace*{-\versesep}
\beginverse*

\nolyrics

%---- Prima riga -----------------------------
\vspace*{-\versesep}
\[D*] \[G] \[E-*]\[A*] \[D]	 % \[*D] per indicare le pennate, \rep{2} le ripetizioni

%---- Ogni riga successiva -------------------
%\vspace*{-\versesep}
%\[G] \[C]  \[D]	

%---- Ev Indicazioni -------------------------			
%\textnote{\textit{(Oppure tutta la strofa)} }	

\endverse
\fi







%%%%% RITORNELLO
\beginchorus
\textnote{\textbf{Rit.}}
Nella \[D*]notte, o \[G]Dio, \[E-7*] noi \[A*]veglie\[D]remo,
con le \[B-]lampade, vestiti a \[F#-]festa: \[B]  
presto \[E-*]arri\[G]verai  e \[A*]sarà \[D]giorno.
\endchorus




%%%%% STROFA
\beginverse		%Oppure \beginverse* se non si vuole il numero di fianco
\memorize 		% <<< DECOMMENTA se si vuole utilizzarne la funzione
%\chordsoff		% <<< DECOMMENTA se vuoi una strofa senza accordi
\[B-*]Ralle\[E-]gratevi in at\[A]tesa del Si\[D7+]gnore:
improv\[B-]visa giungerà la sua \[E-7]voce. \[A9]
Quando \[G-]Lui verrà, sarete \[D7+]pronti
e vi \[E-]chiamerà “\[G]amici” per \[F#]sempre. \[A7]  
\endverse



%%%%% STROFA
\beginverse		%Oppure \beginverse* se non si vuole il numero di fianco
%\memorize 		% <<< DECOMMENTA se si vuole utilizzarne la funzione
%\chordsoff		% <<< DECOMMENTA se vuoi una strofa senza accordi
^Raccogli^ete per il ^giorno della ^vita
dove ^tutto sarà giovane in e^terno. ^
Quando ^lui verrà sarete ^pronti
e vi ^chiamerà “^amici” per ^sempre. ^
\endverse

\endsong
%------------------------------------------------------------
%			FINE CANZONE
%------------------------------------------------------------


%-------------------------------------------------------------
%			INIZIO	CANZONE
%-------------------------------------------------------------


%titolo: 	Non di solo pane
%autore: 	Tuttiattavola — Oratori Bresso e Lainate
%tonalita: 	Mi 



%%%%%% TITOLO E IMPOSTAZONI
\beginsong{Non di solo pane}[by={Tuttiattavola — Oratori Bresso e Lainate}] 	% <<< MODIFICA TITOLO E AUTORE
\transpose{-2} 						% <<< TRASPOSIZIONE #TONI (0 nullo)
%\preferflats  %SE VOGLIO FORZARE i bemolle come alterazioni
%\prefersharps %SE VOGLIO FORZARE i # come alterazioni
\momenti{Meditazione; Comunione; Ringraziamento}							% <<< INSERISCI MOMENTI	
% momenti vanno separati da ; e vanno scelti tra:
% Ingresso; Atto penitenziale; Acclamazione al Vangelo; Dopo il Vangelo; Offertorio; Comunione; 
% Ringraziamento; Fine; Santi; Pasqua; Avvento; Natale; Quaresima; Canti Mariani; Battesimo; 
% Prima Comunione; Cresima; Matrimonio; Meditazione; Spezzare del pane;
\ifchorded
	\textnote{$\bigstar$ Tonalità migliore }	% <<< EV COMMENTI (tonalità originale/migliore)
\fi


%%%%%% INTRODUZIONE
\ifchorded
\vspace*{\versesep}
\musicnote{
\begin{minipage}{0.48\textwidth}
\textbf{Intro}
\hfill 
%( \eighthnote \, 80)   % <<  MODIFICA IL TEMPO
% Metronomo: \eighthnote (ottavo) \quarternote (quarto) \halfnote (due quarti)
\end{minipage}
} 	
\vspace*{-\versesep}
\beginverse*

\nolyrics

%---- Prima riga -----------------------------
\vspace*{-\versesep}
\[C#-] \[A] \[E] \[B] \rep{2}	 % \[*D] per indicare le pennate, \rep{2} le ripetizioni

%---- Ogni riga successiva -------------------
%\vspace*{-\versesep}
%\[G] \[C]  \[D]	

%---- Ev Indicazioni -------------------------			
%\textnote{\textit{(Oppure tutta la strofa)} }	

\endverse
\fi




%%%%% STROFA
\beginverse		%Oppure \beginverse* se non si vuole il numero di fianco
\memorize 		% <<< DECOMMENTA se si vuole utilizzarne la funzione
%\chordsoff		% <<< DECOMMENTA se vuoi una strofa senza accordi

Non di solo \[C#-]pane ogni giorno vi\[A]vrai,
ma di ogni Pa\[E]rola che il Padre ti dona  \brk ti nutri\[B]rai.
Fedele cus\[C#-]tode del gesto d'a\[A]more,
nel pane spez\[E]zato memoria preziosa \brk  conserve\[B]rai.

\endverse
\beginverse*		%Oppure \beginverse* se non si vuole il numero di fianco
%\memorize 		% <<< DECOMMENTA se si vuole utilizzarne la funzione
%\chordsoff		% <<< DECOMMENTA se vuoi una strofa senza accordi

Sulla nostra ^terra che lui ci do^nò
c'è posto per ^tutti se apri il tuo cuore \brk all'umani^tà.
Per vivere in^sieme ci vuole co^raggio
un tenero s^guardo di misericordia e fraterni^tà.

\endverse



%%%%% RITORNELLO
\beginchorus
\textnote{\textbf{Rit.}}

Tu \[E]sai Signore,
co\[B]me saziare
quel \[A]vuoto che sento dentro \[C#-]me.
Perchè o\[A]gnuno sa dare
la \[E]parte migliore di \[B]sé
insieme a \[B]te.

\endchorus


%%%%%% EV. INTERMEZZO
\beginverse*
\vspace*{1.3\versesep}
{
	\nolyrics
	\textnote{Intermezzo strumentale}
	
	\ifchorded

	%---- Prima riga -----------------------------
	\vspace*{-\versesep}
	\[C#-] \[A] \[E] \[B]


	\fi
	%---- Ev Indicazioni -------------------------			
	%\textnote{\textit{(ripetizione della strofa)}} 
	 
}
\vspace*{\versesep}
\endverse



%%%%% STROFA
\beginverse		%Oppure \beginverse* se non si vuole il numero di fianco
%\memorize 		% <<< DECOMMENTA se si vuole utilizzarne la funzione
%\chordsoff		% <<< DECOMMENTA se vuoi una strofa senza accordi

Non di solo ^pane \brk ogni giorno viv^rai, 
avrai sete di ^pace guardando la croce \brk con umil^tà.
Alle tue mille do^mande \brk troverai le ris^poste,
se camminerai ac^canto a un fratello \brk in difficol^tà.

\endverse
\beginverse*		%Oppure \beginverse* se non si vuole il numero di fianco
%\memorize 		% <<< DECOMMENTA se si vuole utilizzarne la funzione
%\chordsoff		% <<< DECOMMENTA se vuoi una strofa senza accordi
Spalanca il ^cuore, ascolta il Si^gnore.
Tieni aperte le ^porte in costante ricerca \brk della veri^tà.
Coltiva la ^Fede, che luce sa^rà,
perchè siamo in ^viaggio insieme ad un Altro \brk che ci guida ^già.

\endverse



%%%%% RITORNELLO
\beginchorus
\textnote{\textbf{Rit.}}

Tu \[E]sai Signore,
co\[B]me saziare
quel \[A]vuoto che sento dentro \[C#-]me.
Perchè o\[A]gnuno sa dare
la \[E]parte migliore di \[B]sé
insieme a \[B]te.

\endchorus



%%%%% BRIDGE
\beginverse*		%Oppure \beginverse* se non si vuole il numero di fianco
%\memorize 		% <<< DECOMMENTA se si vuole utilizzarne la funzione
%\chordsoff		% <<< DECOMMENTA se vuoi una strofa senza accordi
\vspace*{1.3\versesep}
\textnote{\textbf{Bridge}}

Non è \[E]facile,
quando \[B]niente va,
aspet\[A]tare che un sorriso 
torni ad \[C#-]esser pane per la vita.
Ar\[E]rendersi
non a\[B]iuterà
prendi \[A]posto alla mia tavo\[B*]la:
chi ama gioia guste\[C#-]rà!
\endverse


%%%%%% EV. INTERMEZZO
\beginverse*
\vspace*{1.3\versesep}
{
	\nolyrics
	\textnote{Intermezzo strumentale}
	
	\ifchorded

	%---- Prima riga -----------------------------
	\vspace*{-\versesep}
	 \[A] \[E] \[B]


	\fi
	%---- Ev Indicazioni -------------------------			
	%\textnote{\textit{(ripetizione della strofa)}} 
	 
}
\vspace*{\versesep}
\endverse


%%%%% RITORNELLO
\beginchorus
\textnote{\textbf{Rit.}}

Tu \[E]sai Signore,
co\[B]me saziare
quel \[A]vuoto che sento dentro \[C#-]me.
Perchè o\[A]gnuno sa dare
la \[E]parte migliore di \[B]sé
insieme a \[B]te. \rep{2} 

\endchorus



%%%%%% EV. CHIUSURA SOLO STRUMENTALE
\ifchorded
\beginchorus %oppure \beginverse*
\vspace*{1.3\versesep}
\textnote{Chiusura } %<<< EV. INDICAZIONI

\[E*]

\endchorus  %oppure \endverse
\fi


\endsong
%------------------------------------------------------------
%			FINE CANZONE
%------------------------------------------------------------


%++++++++++++++++++++++++++++++++++++++++++++++++++++++++++++
%			CANZONE TRASPOSTA
%++++++++++++++++++++++++++++++++++++++++++++++++++++++++++++
\ifchorded
%decremento contatore per avere stesso numero
\addtocounter{songnum}{-1} 
\beginsong{Non di solo pane}[by={Tuttiattavola — Oratori Bresso e Lainate}] 	% <<< COPIA TITOLO E AUTORE
\transpose{0} 						% <<< TRASPOSIZIONE #TONI + - (0 nullo)
%\preferflats  %SE VOGLIO FORZARE i bemolle come alterazioni
%\prefersharps %SE VOGLIO FORZARE i # come alterazioni
\ifchorded
	\textnote{$\lozenge$ Tonalità originale}	% <<< EV COMMENTI (tonalità originale/migliore)
\fi

%%%%%% INTRODUZIONE
\ifchorded
\vspace*{\versesep}
\musicnote{
\begin{minipage}{0.48\textwidth}
\textbf{Intro}
\hfill 
%( \eighthnote \, 80)   % <<  MODIFICA IL TEMPO
% Metronomo: \eighthnote (ottavo) \quarternote (quarto) \halfnote (due quarti)
\end{minipage}
} 	
\vspace*{-\versesep}
\beginverse*

\nolyrics

%---- Prima riga -----------------------------
\vspace*{-\versesep}
\[C#-] \[A] \[E] \[B] \rep{2}	 % \[*D] per indicare le pennate, \rep{2} le ripetizioni

%---- Ogni riga successiva -------------------
%\vspace*{-\versesep}
%\[G] \[C]  \[D]	

%---- Ev Indicazioni -------------------------			
%\textnote{\textit{(Oppure tutta la strofa)} }	

\endverse
\fi




%%%%% STROFA
\beginverse		%Oppure \beginverse* se non si vuole il numero di fianco
\memorize 		% <<< DECOMMENTA se si vuole utilizzarne la funzione
%\chordsoff		% <<< DECOMMENTA se vuoi una strofa senza accordi

Non di solo \[C#-]pane ogni giorno vi\[A]vrai,
ma di ogni Pa\[E]rola che il Padre ti dona  \brk ti nutri\[B]rai.
Fedele cus\[C#-]tode del gesto d'a\[A]more,
nel pane spez\[E]zato memoria preziosa \brk  conserve\[B]rai.

\endverse
\beginverse*		%Oppure \beginverse* se non si vuole il numero di fianco
%\memorize 		% <<< DECOMMENTA se si vuole utilizzarne la funzione
%\chordsoff		% <<< DECOMMENTA se vuoi una strofa senza accordi

Sulla nostra ^terra che lui ci do^nò
c'è posto per ^tutti se apri il tuo cuore \brk all'umani^tà.
Per vivere in^sieme ci vuole co^raggio
un tenero s^guardo di misericordia e fraterni^tà.

\endverse



%%%%% RITORNELLO
\beginchorus
\textnote{\textbf{Rit.}}

Tu \[E]sai Signore,
co\[B]me saziare
quel \[A]vuoto che sento dentro \[C#-]me.
Perchè o\[A]gnuno sa dare
la \[E]parte migliore di \[B]sé
insieme a \[B]te.

\endchorus


%%%%%% EV. INTERMEZZO
\beginverse*
\vspace*{1.3\versesep}
{
	\nolyrics
	\textnote{Intermezzo strumentale}
	
	\ifchorded

	%---- Prima riga -----------------------------
	\vspace*{-\versesep}
	\[C#-] \[A] \[E] \[B]


	\fi
	%---- Ev Indicazioni -------------------------			
	%\textnote{\textit{(ripetizione della strofa)}} 
	 
}
\vspace*{\versesep}
\endverse



%%%%% STROFA
\beginverse		%Oppure \beginverse* se non si vuole il numero di fianco
%\memorize 		% <<< DECOMMENTA se si vuole utilizzarne la funzione
%\chordsoff		% <<< DECOMMENTA se vuoi una strofa senza accordi

Non di solo ^pane \brk ogni giorno viv^rai, 
avrai sete di ^pace guardando la croce \brk con umil^tà.
Alle tue mille do^mande \brk troverai le ris^poste,
se camminerai ac^canto a un fratello \brk in difficol^tà.

\endverse
\beginverse*		%Oppure \beginverse* se non si vuole il numero di fianco
%\memorize 		% <<< DECOMMENTA se si vuole utilizzarne la funzione
%\chordsoff		% <<< DECOMMENTA se vuoi una strofa senza accordi
Spalanca il ^cuore, ascolta il Si^gnore.
Tieni aperte le ^porte in costante ricerca \brk della veri^tà.
Coltiva la ^Fede, che luce sa^rà,
perchè siamo in ^viaggio insieme ad un Altro \brk che ci guida ^già.

\endverse



%%%%% RITORNELLO
\beginchorus
\textnote{\textbf{Rit.}}

Tu \[E]sai Signore,
co\[B]me saziare
quel \[A]vuoto che sento dentro \[C#-]me.
Perchè o\[A]gnuno sa dare
la \[E]parte migliore di \[B]sé
insieme a \[B]te.

\endchorus



%%%%% BRIDGE
\beginverse*		%Oppure \beginverse* se non si vuole il numero di fianco
%\memorize 		% <<< DECOMMENTA se si vuole utilizzarne la funzione
%\chordsoff		% <<< DECOMMENTA se vuoi una strofa senza accordi
\vspace*{1.3\versesep}
\textnote{\textbf{Bridge}}

Non è \[E]facile,
quando \[B]niente va,
aspet\[A]tare che un sorriso 
torni ad \[C#-]esser pane per la vita.
Ar\[E]rendersi
non a\[B]iuterà
prendi \[A]posto alla mia tavo\[B*]la:
chi ama gioia guste\[C#-]rà!
\endverse


%%%%%% EV. INTERMEZZO
\beginverse*
\vspace*{1.3\versesep}
{
	\nolyrics
	\textnote{Intermezzo strumentale}
	
	\ifchorded

	%---- Prima riga -----------------------------
	\vspace*{-\versesep}
	 \[A] \[E] \[B]


	\fi
	%---- Ev Indicazioni -------------------------			
	%\textnote{\textit{(ripetizione della strofa)}} 
	 
}
\vspace*{\versesep}
\endverse


%%%%% RITORNELLO
\beginchorus
\textnote{\textbf{Rit.}}

Tu \[E]sai Signore,
co\[B]me saziare
quel \[A]vuoto che sento dentro \[C#-]me.
Perchè o\[A]gnuno sa dare
la \[E]parte migliore di \[B]sé
insieme a \[B]te. \rep{2} 

\endchorus



%%%%%% EV. CHIUSURA SOLO STRUMENTALE
\ifchorded
\beginchorus %oppure \beginverse*
\vspace*{1.3\versesep}
\textnote{Chiusura } %<<< EV. INDICAZIONI

\[E*]

\endchorus  %oppure \endverse
\fi


\endsong

\fi
%++++++++++++++++++++++++++++++++++++++++++++++++++++++++++++
%			FINE CANZONE TRASPOSTA
%++++++++++++++++++++++++++++++++++++++++++++++++++++++++++++



%-------------------------------------------------------------
%			INIZIO	CANZONE
%-------------------------------------------------------------


%titolo: 	Nulla è impossibile a Dio
%autore: 	C. Burgio
%tonalita: 	Re>Fa



%%%%%% TITOLO E IMPOSTAZONI
\beginsong{Nulla è impossibile a Dio}[by={C. Burgio}] 	% <<< MODIFICA TITOLO E AUTORE
\transpose{3} 						% <<< TRASPOSIZIONE #TONI (0 nullo)
\momenti{Ingresso; Prima Comunione}							% <<< INSERISCI MOMENTI	
% momenti vanno separati da ; e vanno scelti tra:
% Ingresso; Atto penitenziale; Acclamazione al Vangelo; Dopo il Vangelo; Offertorio; Comunione; Ringraziamento; Fine; Santi; Pasqua; Avvento; Natale; Quaresima; Canti Mariani; Battesimo; Prima Comunione; Cresima; Matrimonio; Meditazione; Spezzare del pane;
\ifchorded
	\textnote{Tonalità migliore}	% <<< EV COMMENTI (tonalità originale/migliore)
\fi


%%%%%% INTRODUZIONE
\ifchorded
\vspace*{\versesep}
\textnote{Intro: \qquad \qquad  }%(\eighthnote 116) % <<  MODIFICA IL TEMPO
% Metronomo: \eighthnote (ottavo) \quarternote (quarto) \halfnote (due quarti)
\vspace*{-\versesep}
\beginverse*

\nolyrics

%---- Prima riga -----------------------------
\vspace*{-\versesep}
\[D] \[A] \[G] \[A]	 % \[*D] per indicare le pennate, \rep{2} le ripetizioni

%---- Ogni riga successiva -------------------
%\vspace*{-\versesep}
%\[G] \[C]  \[D]	

%---- Ev Indicazioni -------------------------			
%\textnote{\textit{(Oppure tutta la strofa)} }	

\endverse
\fi




%%%%% STROFA
\beginverse		%Oppure \beginverse* se non si vuole il numero di fianco
\memorize 		% <<< DECOMMENTA se si vuole utilizzarne la funzione
%\chordsoff		% <<< DECOMMENTA se vuoi una strofa senza accordi
\[D]Quando Dio ci chia\[A]mò 
ed il \[G]tempo ci do\[A]nò,
come un \[D]padre fidu\[A]cioso 
nel suo \[G]cuore ci por\[A]tò.

\[B-]Egli fece di \[F#-]noi 
una s\[G]toria un solo \[A]popolo;
\[B-]forte, lui, ci gui\[F#-]dò 
sulle \[G]strade che con\[E-]ducono 
alla \[F#]libertà.

\endverse




%%%%% RITORNELLO
\beginchorus
\textnote{\textbf{Rit.}}

\[D]Ecco il grande mis\[A]tero 
dai \[B-]secoli annun\[F#-]ciato:
“\[G]Nulla è impos\[D]sibile a \[A]Dio”. \[A7]
\[D]Nasce nuova spe\[A7]ranza 
si \[B-]compie ormai la pro\[F#-]messa:
“\[G]Nulla è impos\[D]sibile a \[A]Di\[D]o”.

\endchorus



%%%%% STROFA
\beginverse		%Oppure \beginverse* se non si vuole il numero di fianco
%\memorize 		% <<< DECOMMENTA se si vuole utilizzarne la funzione
%\chordsoff		% <<< DECOMMENTA se vuoi una strofa senza accordi

^Quando venne tra ^noi, 
come ^figlio e “Dio con ^noi”, 
fatto ^uomo in Ma^ria, 
la sal^vezza ci por^tò. 

^Noi credemmo in ^Lui, 
vivo ^segno della ^Verità; 
^imparammo da ^Lui 
che l’^Amore non ha ^prezzo,
non pos^siede mai. 

\endverse



%%%%% STROFA
\beginverse		%Oppure \beginverse* se non si vuole il numero di fianco
%\memorize 		% <<< DECOMMENTA se si vuole utilizzarne la funzione
%\chordsoff		% <<< DECOMMENTA se vuoi una strofa senza accordi

^Quando poi ci las^ciò 
e dal ^padre rito^rnò, 
il Si^gnore dalla ^croce 
il suo ^spirito do^nò. 

^Nuova vita per ^noi 
questa ^grazia che ci il^lumina, 
^è memoria tra ^noi 
dell’A^more che ci ac^coglie
e non ci ^lascia mai.

\endverse




\endsong
%------------------------------------------------------------
%			FINE CANZONE
%------------------------------------------------------------

%++++++++++++++++++++++++++++++++++++++++++++++++++++++++++++
%			CANZONE TRASPOSTA
%++++++++++++++++++++++++++++++++++++++++++++++++++++++++++++
\ifchorded
%decremento contatore per avere stesso numero
\addtocounter{songnum}{-1} 
\beginsong{Nulla è impossibile a Dio}[by={C. Burgio}] 	% <<< MODIFICA TITOLO E AUTORE 
\transpose{0} 						% <<< TRASPOSIZIONE #TONI + - (0 nullo)
%\preferflats  %SE VOGLIO FORZARE i bemolle come alterazioni
%\prefersharps %SE VOGLIO FORZARE i # come alterazioni
\ifchorded
	\textnote{Tonalità originale}	% <<< EV COMMENTI (tonalità originale/migliore)
\fi


%%%%%% INTRODUZIONE
\ifchorded
\vspace*{\versesep}
\textnote{Intro: \qquad \qquad  }%(\eighthnote 116) % <<  MODIFICA IL TEMPO
% Metronomo: \eighthnote (ottavo) \quarternote (quarto) \halfnote (due quarti)
\vspace*{-\versesep}
\beginverse*

\nolyrics

%---- Prima riga -----------------------------
\vspace*{-\versesep}
\[D] \[A] \[G] \[A]	 % \[*D] per indicare le pennate, \rep{2} le ripetizioni

%---- Ogni riga successiva -------------------
%\vspace*{-\versesep}
%\[G] \[C]  \[D]	

%---- Ev Indicazioni -------------------------			
%\textnote{\textit{(Oppure tutta la strofa)} }	

\endverse
\fi




%%%%% STROFA
\beginverse		%Oppure \beginverse* se non si vuole il numero di fianco
\memorize 		% <<< DECOMMENTA se si vuole utilizzarne la funzione
%\chordsoff		% <<< DECOMMENTA se vuoi una strofa senza accordi
\[D]Quando Dio ci chia\[A]mò 
ed il \[G]tempo ci do\[A]nò,
come un \[D]padre fidu\[A]cioso 
nel suo \[G]cuore ci por\[A]tò.

\[B-]Egli fece di \[F#-]noi 
una s\[G]toria un solo \[A]popolo;
\[B-]forte, lui, ci gui\[F#-]dò 
sulle \[G]strade che con\[E-]ducono 
alla \[F#]libertà.

\endverse




%%%%% RITORNELLO
\beginchorus
\textnote{\textbf{Rit.}}

\[D]Ecco il grande mis\[A]tero 
dai \[B-]secoli annun\[F#-]ciato:
“\[G]Nulla è impos\[D]sibile a \[A]Dio”. \[A7]
\[D]Nasce nuova spe\[A7]ranza 
si \[B-]compie ormai la pro\[F#-]messa:
“\[G]Nulla è impos\[D]sibile a \[A]Di\[D]o”.

\endchorus



%%%%% STROFA
\beginverse		%Oppure \beginverse* se non si vuole il numero di fianco
%\memorize 		% <<< DECOMMENTA se si vuole utilizzarne la funzione
%\chordsoff		% <<< DECOMMENTA se vuoi una strofa senza accordi

^Quando venne tra ^noi, 
come ^figlio e “Dio con ^noi”, 
fatto ^uomo in Ma^ria, 
la sal^vezza ci por^tò. 

^Noi credemmo in ^Lui, 
vivo ^segno della ^Verità; 
^imparammo da ^Lui 
che l’^Amore non ha ^prezzo,
non pos^siede mai. 

\endverse



%%%%% STROFA
\beginverse		%Oppure \beginverse* se non si vuole il numero di fianco
%\memorize 		% <<< DECOMMENTA se si vuole utilizzarne la funzione
%\chordsoff		% <<< DECOMMENTA se vuoi una strofa senza accordi

^Quando poi ci las^ciò 
e dal ^padre rito^rnò, 
il Si^gnore dalla ^croce 
il suo ^spirito do^nò. 

^Nuova vita per ^noi 
questa ^grazia che ci il^lumina, 
^è memoria tra ^noi 
dell’A^more che ci ac^coglie
e non ci ^lascia mai.

\endverse




\endsong


\fi
%++++++++++++++++++++++++++++++++++++++++++++++++++++++++++++
%			FINE CANZONE TRASPOSTA
%++++++++++++++++++++++++++++++++++++++++++++++++++++++++++++

%OOO
%titolo{Ogni mia parola}
%autore{Gen Verde}
%album{È bello lodarti}
%tonalita{Do}
%famiglia{Liturgica}
%gruppo{}
%momenti{Acclamazione al Vangelo}
%identificatore{ogni_mia_parola}
%data_revisione{2011_12_31}
%trascrittore{Francesco Endrici}
\beginsong{Ogni mia parola}[by={Gen\ Verde}]
\beginverse*
\[C]Come la \[G]pioggia e la \[C]ne\[G]ve
\[C]scendono \[F]giù dal \[G]cielo
e \[A-]non vi ri\[G]tornano \[F]senza irri\[G]gare
e \[F]far germo\[G]gliare la \[F]ter\[G]ra,
\[C]così ogni mia Pa\[G]rola non ri\[C]tornerà a \[G]me
\[C]senza operare \[F]quanto de\[G]sidero,
\[A-]senza aver compiuto \[E-]ciò per cui l'a\[F]vevo man\[C]data.
\[F]Ogni mia Pa\[G]rola, \[F]ogni mia Pa\[G]rola \[C]
\endverse
\endsong


%-------------------------------------------------------------
%			INIZIO	CANZONE
%-------------------------------------------------------------


%titolo: 	Oltre la memoria (Symbolum 80)
%autore: 	Sequeri
%tonalita: 	Sol 



%%%%%% TITOLO E IMPOSTAZONI
\beginsong{Oltre la memoria }[ititle={Symbolum 80}, by={Symbolum 80 — P. Sequeri}] 	% <<< MODIFICA TITOLO E AUTORE
\transpose{0} 						% <<< TRASPOSIZIONE #TONI (0 nullo)
\momenti{Comunione}							% <<< INSERISCI MOMENTI	
% momenti vanno separati da ; e vanno scelti tra:
% Ingresso; Atto penitenziale; Acclamazione al Vangelo; Dopo il Vangelo; Offertorio; Comunione; Ringraziamento; Fine; Santi; Pasqua; Avvento; Natale; Quaresima; Canti Mariani; Battesimo; Prima Comunione; Cresima; Matrimonio; Meditazione;
\ifchorded
	\textnote{Tonalità migliore}	% <<< EV COMMENTI (tonalità originale/migliore)
\fi





%%%%%% INTRODUZIONE
\ifchorded
\vspace*{\versesep}
\textnote{Intro: \qquad \qquad  (\eighthnote 136)}% % << MODIFICA IL TEMPO
% Metronomo: \eighthnote (ottavo) \quarternote (quarto) \halfnote (due quarti)
\vspace*{-\versesep}
\beginverse*

\nolyrics

%---- Prima riga -----------------------------
\vspace*{-\versesep}
\[D-] \[D-] \[G-]  \[D-]	 % \[*D] per indicare le pennate, \rep{2} le ripetizioni

%---- Ogni riga successiva -------------------
%\vspace*{-\versesep}
%\[G] \[C]  \[G]	

%---- Ev Indicazioni -------------------------			
%\textnote{\textit{(Oppure tutta la strofa)} }	

\endverse
\fi



%%%%% STROFA
\beginverse
\memorize


\[D-]Oltre la me\[7]moria del \[G-]tempo che ho vis\[D-]suto,
\[D-]oltre la spe\[A-]ranza 
che \[G-]serve al mio do\[A]mani, \quad \[A]
\[D-]oltre il desi\[7]derio di \[G-]vivere il pre\[D-]sente,
an\[D-]ch’io, confesso, ho chi\[A-]esto 
che \[G-]cosa è veri\[A]tà. 

\vspace{1.5\versesep}


\[D]E \[A]tu \[B-]come un desi\[F#-]derio 
\[G]che non \[A]ha me\[B-]morie, \[E]Padre bu\[A]ono, 
\[D]come una spe\[A]ranza c\[G]he non ha con\[F#-]fini,
\[G]come un \[A]tempo e\[B-]terno 
\[E7]sei per \[A]me.

\endverse


%%%%% RITORNELLO
\beginchorus
\textnote{\textbf{Rit.}}

\[F]Io \[C]so \[D-]quanto amore ch\[A-]iede 
\[B&]questa \[C]lunga at\[F]te\[(D-)]sa 
\[G]del tuo giorno, \[C]Dio; 
\[F]luce in ogni \[C]cosa \[D-]io non vedo an\[A-]cora:
\[B&]ma la \[C7]tua pa\[F]ro\[(D-)]la 
\[G]mi risch\[7]iare\[A]rà!

\endchorus




%%%%% STROFA
\beginverse

^Quando le pa^role non ^bastano all’a^more,
^quando il mio fra^tello 
do^manda più del ^pane, \quad ^
^quando l’illu^sione pro^mette un mondo ^nuovo,
anch’^io rimango in^certo 
nel ^mezzo del cam^mino.

\vspace{1.5\versesep}

^E ^tu ^Figlio tanto a^mato,
^veri^tà dell’^uomo, ^mio Si^gnore,
^come la pro^messa ^di un perdono e^terno,
^liber^tà infi^nita 
^sei per ^me.


\endverse


%%%%% RITORNELLO
\beginchorus
\textnote{\textbf{Rit.}}

\[F]Io \[C]so \[D-]quanto amore ch\[A-]iede 
\[B&]questa \[C]lunga at\[F]te\[(D-)]sa 
\[G]del tuo giorno, \[C]Dio; 
\[F]luce in ogni \[C]cosa \[D-]io non vedo an\[A-]cora:
\[B&]ma la \[C7]tua pa\[F]ro\[(D-)]la 
\[G]mi risch\[7]iare\[A]rà!

\endchorus




%%%%% STROFA
\beginverse


^Chiedo alla mia ^mente co^raggio di cer^care,
^chiedo alle mie ^mani 
la ^forza di do^nare, \quad ^
^chiedo al cuore in^certo pas^sione per la ^vita,
e ^chiedo a te fra^tello 
di ^credere con ^me.

\vspace{1.5\versesep}

^E ^tu, ^forza della ^vita,
^Spiri^to d’a^more, ^dolce Id^dio,
^grembo d’ogni ^cosa, ^tenerezza im^mensa,
^veri^tà del ^mondo 
^sei per ^me.


\endverse


%%%%% RITORNELLO
\beginchorus
\textnote{\textbf{Rit.}}

\[F]Io \[C]so \[D-]quanto amore ch\[A-]iede 
\[B&]questa \[C]lunga at\[F]te\[(D-)]sa 
\[G]del tuo giorno, \[C]Dio; 
\[F]luce in ogni \[C]cosa \[D-]io non vedo an\[A-]cora:
\[B&]ma la \[C7]tua pa\[F]ro\[(D-)]la 

\endchorus




%%%%%% EV. FINALE
\beginchorus %oppure \beginverse*
\vspace*{1.3\versesep}
\textnote{Finale \textit{(rallentando)}} %<<< EV. INDICAZIONI

\[E]mi risch\[A]iare\[D]rà!

\endchorus  %oppure \endverse






\endsong
%------------------------------------------------------------
%			FINE CANZONE
%------------------------------------------------------------




%++++++++++++++++++++++++++++++++++++++++++++++++++++++++++++
%			CANZONE TRASPOSTA
%++++++++++++++++++++++++++++++++++++++++++++++++++++++++++++
\ifchorded
%decremento contatore per avere stesso numero
\addtocounter{songnum}{-1} 
\beginsong{Oltre la memoria}[ititle={Ma la tua parola}, by={Symbolum 80 — P. Sequeri}] 	% <<< MODIFICA TITOLO E AUTORE
\transpose{2} 						% <<< TRASPOSIZIONE #TONI (0 nullo)
%\preferflats %SE VOGLIO FORZARE i bemolle come alterazioni
\prefersharps %SE VOGLIO FORZARE i # come alterazioni
\ifchorded
	\textnote{Tonalità originale}	% <<< EV COMMENTI (tonalità originale/migliore)
\fi



%%%%%% INTRODUZIONE
\ifchorded
\vspace*{\versesep}
\textnote{Intro: \qquad \qquad  (\eighthnote 136)}% % << MODIFICA IL TEMPO
% Metronomo: \eighthnote (ottavo) \quarternote (quarto) \halfnote (due quarti)
\vspace*{-\versesep}
\beginverse*

\nolyrics

%---- Prima riga -----------------------------
\vspace*{-\versesep}
\[D-] \[D-] \[G-]  \[D-]	 % \[*D] per indicare le pennate, \rep{2} le ripetizioni

%---- Ogni riga successiva -------------------
%\vspace*{-\versesep}
%\[G] \[C]  \[G]	

%---- Ev Indicazioni -------------------------			
%\textnote{\textit{(Oppure tutta la strofa)} }	

\endverse
\fi



%%%%% STROFA
\beginverse
\memorize


\[D-]Oltre la me\[7]moria del \[G-]tempo che ho vis\[D-]suto,
\[D-]oltre la spe\[A-]ranza 
che \[G-]serve al mio do\[A]mani, \quad \[A]
\[D-]oltre il desi\[7]derio di \[G-]vivere il pre\[D-]sente,
an\[D-]ch’io, confesso, ho chi\[A-]esto 
che \[G-]cosa è veri\[A]tà. 

\vspace{1.5\versesep}

\[D]E \[A]tu \[B-]come un desi\[F#-]derio 
\[G]che non \[A]ha me\[B-]morie, \[E]Padre bu\[A]ono, 
\[D]come una spe\[A]ranza c\[G]he non ha con\[F#-]fini,
\[G]come un \[A]tempo e\[B-]terno 
\[E7]sei per \[A]me.

\endverse


%%%%% RITORNELLO
\beginchorus
\textnote{\textbf{Rit.}}

\[F]Io \[C]so \[D-]quanto amore ch\[A-]iede 
\[B&]questa \[C]lunga at\[F]te\[(D-)]sa 
\[G]del tuo giorno, \[C]Dio; 
\[F]luce in ogni \[C]cosa \[D-]io non vedo an\[A-]cora:
\[B&]ma la \[C7]tua pa\[F]ro\[(D-)]la 
\[G]mi risch\[7]iare\[A]rà!

\endchorus



%%%%% STROFA
\beginverse

^Quando le pa^role non ^bastano all’a^more,
^quando il mio fra^tello 
do^manda più del ^pane, \quad ^
^quando l’illu^sione pro^mette un mondo ^nuovo,
anch’^io rimango in^certo 
nel ^mezzo del cam^mino.

\vspace{1.5\versesep}

^E ^tu ^Figlio tanto a^mato,
^veri^tà dell’^uomo, ^mio Si^gnore,
^come la pro^messa ^di un perdono e^terno,
^liber^tà infi^nita 
^sei per ^me.


\endverse


%%%%% RITORNELLO
\beginchorus
\textnote{\textbf{Rit.}}

\[F]Io \[C]so \[D-]quanto amore ch\[A-]iede 
\[B&]questa \[C]lunga at\[F]te\[(D-)]sa 
\[G]del tuo giorno, \[C]Dio; 
\[F]luce in ogni \[C]cosa \[D-]io non vedo an\[A-]cora:
\[B&]ma la \[C7]tua pa\[F]ro\[(D-)]la 
\[G]mi risch\[7]iare\[A]rà!

\endchorus




%%%%% STROFA
\beginverse


^Chiedo alla mia ^mente co^raggio di cer^care,
^chiedo alle mie ^mani 
la ^forza di do^nare, \quad ^
^chiedo al cuore in^certo pas^sione per la ^vita,
e ^chiedo a te fra^tello 
di ^credere con ^me.

\vspace{1.5\versesep}

^E ^tu, ^forza della ^vita,
^Spiri^to d’a^more, ^dolce Id^dio,
^grembo d’ogni ^cosa, ^tenerezza im^mensa,
^veri^tà del ^mondo 
^sei per ^me.


\endverse



%%%%% RITORNELLO
\beginchorus
\textnote{\textbf{Rit.}}

\[F]Io \[C]so \[D-]quanto amore ch\[A-]iede 
\[B&]questa \[C]lunga at\[F]te\[(D-)]sa 
\[G]del tuo giorno, \[C]Dio; 
\[F]luce in ogni \[C]cosa \[D-]io non vedo an\[A-]cora:
\[B&]ma la \[C7]tua pa\[F]ro\[(D-)]la 

\endchorus



%%%%%% EV. FINALE
\beginchorus %oppure \beginverse*
\vspace*{1.3\versesep}
\textnote{Finale \textit{(rallentando)}} %<<< EV. INDICAZIONI

\[E]mi risch\[A]iare\[D]rà!

\endchorus  %oppure \endverse





\endsong

\fi
%++++++++++++++++++++++++++++++++++++++++++++++++++++++++++++
%			FINE CANZONE TRASPOSTA
%++++++++++++++++++++++++++++++++++++++++++++++++++++++++++++

%titolo{Ora è tempo di gioia}
%autore{Gen Rosso}
%album{Se siamo uniti}
%tonalita{Re}
%famiglia{Liturgica}
%gruppo{}
%momenti{}
%identificatore{ora_e_tempo_di_gioia}
%data_revisione{2014_09_30}
%trascrittore{Francesco Endrici}
\beginsong{Ora è tempo di gioia}[by={Gen\ Rosso}]
\beginverse
L'\[D]eco \[E-]torna d'an\[D]tiche \[G]val\[A]li
\[D]la sua \[E-]voce \[D7+]non porta \[C7+]più,
\[B-]ricordo \[F#-]di som\[G]messe \[E7]lacri\[A]me
\[D]di e\[E-7]si\[D]li in terre \[A4]lonta\[D]ne.
\endverse
\beginchorus
\[G]Ora è \[D]tempo di \[C]gio\[D]ia, \brk \[G]non \[A-7]ve \[G]ne ac\[C]cor\[D4]ge\[D]te
\[G]ecco \[D]faccio una \[A-]cosa \[E-]nuova
\[B7]nel de\[C7+]serto una \[B-]strada apri\[E-]rò.
\endchorus
\beginverse
%\chordsoff
^Come l'^onda che ^sulla ^sab^bia
^copre le ^orme e ^poi passa e ^va,
^così nel ^tempo ^si can^cella^no
^le ombre ^scure ^del lungo in^ver^no.
\endverse
\beginverse
%\chordsoff
^Tra i sen^tieri dei ^boschi il ^ven^to
^con i ^rami ^ricompor^rà
^nuove armo^nie ^che tra^sforma^no
^i la^menti ^in canti di ^fe^sta.
\endverse
\endsong
%PPP
%-------------------------------------------------------------
%			INIZIO	CANZONE
%-------------------------------------------------------------


%titolo: 	Pace sia, pace a voi
%autore: 	Gen Verde, Gen Rosso
%tonalita: 	Mi 



%%%%%% TITOLO E IMPOSTAZONI
\beginsong{Pace sia, pace a voi}[by={Gen Verde, Gen Rosso}] 	% <<< MODIFICA TITOLO E AUTORE
\transpose{0} 						% <<< TRASPOSIZIONE #TONI (0 nullo)
\momenti{Offertorio}							% <<< INSERISCI MOMENTI	
% momenti vanno separati da ; e vanno scelti tra:
% Ingresso; Atto penitenziale; Acclamazione al Vangelo; Dopo il Vangelo; Offertorio; Comunione; Ringraziamento; Fine; Santi; Pasqua; Avvento; Natale; Quaresima; Canti Mariani; Battesimo; Prima Comunione; Cresima; Matrimonio; Meditazione;
\ifchorded
	%\textnote{Tonalità originale }	% <<< EV COMMENTI (tonalità originale/migliore)
\fi

%%%%%% INTRODUZIONE
\ifchorded
\vspace*{\versesep}
\musicnote{
\begin{minipage}{0.48\textwidth}
\textbf{Intro}
\hfill 
%( \eighthnote \, 80)   % <<  MODIFICA IL TEMPO
% Metronomo: \eighthnote (ottavo) \quarternote (quarto) \halfnote (due quarti)
\end{minipage}
} 	
\vspace*{-\versesep}
\beginverse*

\nolyrics

%---- Prima riga -----------------------------
\vspace*{-\versesep}
\[E] \[A] \[E] \[B] \[E] \[A] \[E]	 % \[*D] per indicare le pennate, \rep{2} le ripetizioni

%---- Ogni riga successiva -------------------
%\vspace*{-\versesep}
%\[G] \[C]  \[D]	

%---- Ev Indicazioni -------------------------			
%\textnote{\textit{(Oppure tutta la strofa)} }	

\endverse
\fi

%%%%% RITORNELLO
\beginchorus
\textnote{\textbf{Rit.}}

“Pace \[E]sia, pace a voi”: la tua \[A]pace sarà
sulla \[C#-]terra com'è nei \[B]cieli.
“Pace \[E]sia, pace a voi”: la tua \[A]pace sarà
gioia \[G]nei nostri \[D]occhi, nei \[A]cuo\[B]ri.
“Pace \[E]sia, pace a voi”: la tua \[A]pace sarà
luce \[C#-]limpida nei pen\[B]sieri.
“Pace \[E]sia, pace a voi”: la tua \[A]pace sarà
una \[E]casa per \[B]tutti. \[E]\[A]\[E]

\endchorus

%%%%% STROFA
\beginverse		%Oppure \beginverse* se non si vuole il numero di fianco
\memorize 		% <<< DECOMMENTA se si vuole utilizzarne la funzione
%\chordsoff		% <<< DECOMMENTA se vuoi una strofa senza accordi

“\[A]Pace a \[E]voi”: sia il tuo \[B]dono vi\[C#-]sibile.
“\[A]Pace a \[E]voi”: la tua e\[B]redi\[C#-]tà.
“\[A]Pace a \[E]voi”: come un \[B]canto all'u\[C#-]nisono
che \[D]sale dalle nostre cit\[B]tà.

\endverse

%%%%% STROFA
\beginverse		%Oppure \beginverse* se non si vuole il numero di fianco
%\memorize 		% <<< DECOMMENTA se si vuole utilizzarne la funzione
%\chordsoff		% <<< DECOMMENTA se vuoi una strofa senza accordi

“^Pace a ^voi”: sia un im^pronta nei ^secoli.
“^Pace a ^voi”: segno d'^uni^tà.
“^Pace a ^voi”: sia l'ab^braccio tra i ^popoli,
la ^tua promessa all'umani^tà.

\endverse

\endsong
%------------------------------------------------------------
%			FINE CANZONE
%------------------------------------------------------------

%titolo{Pane del cielo}
%autore{Gen Rosso}
%album{Dove tu sei}
%tonalita{Do}
%famiglia{Liturgica}
%gruppo{}
%momenti{Comunione}
%identificatore{pane_del_cielo}
%data_revisione{2014_09_30}
%trascrittore{Francesco Endrici}
\beginsong{Pane del cielo}[by={Gen\ Rosso}]
\beginchorus
\[C]Pane del \[E-]cielo, \[F]sei Tu Ge\[C]sù,
\[A-]via d'a\[D-]more: \[F]Tu ci fai come \[C]Te. \rep{2}
\endchorus
\beginverse
\[F]No, non è ri\[D-]masta fredda la \[G]terra;
\[E-]Tu sei ri\[F]masto con \[C]noi \[F] per nutrirci di \[C]Te.
\[A-]Pane di \[G]vita, \[A-] ed infiam\[G]mare col tuo a\[E]more
\[G]tutta l'u\[F]mani\[C]tà.
\endverse
\beginverse
%\chordsoff
^Sì, il cielo è ^qui su questa ^terra;
^Tu sei ri^masto con ^noi ^ ma ci porti con ^Te
^nella tua ^casa ^ dove vi^vremo insieme a ^Te
^tutta l'e^terni^tà.
\endverse
\beginverse
%\chordsoff
^No, la morte non ^può farci pa^ura;
^Tu sei ri^masto con ^noi, ^ e chi vive di ^Te
^vive per ^sempre. ^
Sei Dio con ^noi, sei Dio per ^noi,
^Dio in ^mezzo a ^noi.
\endverse
\endsong
%-------------------------------------------------------------
%			INIZIO	CANZONE
%-------------------------------------------------------------


%titolo: 	Pane di vita nuova
%autore: 	Frisina
%tonalita: 	Re 



%%%%%% TITOLO E IMPOSTAZONI
\beginsong{Pane di vita nuova}[by={M. Frisina}] 	% <<< MODIFICA TITOLO E AUTORE
\transpose{0} 						% <<< TRASPOSIZIONE #TONI (0 nullo)
\momenti{Comunione;  Pasqua; Spezzare del Pane;}							% <<< INSERISCI MOMENTI	
% momenti vanno separati da ; e vanno scelti tra:
% Ingresso; Atto penitenziale; Acclamazione al Vangelo; Dopo il Vangelo; Offertorio; Comunione; Ringraziamento; Fine; Santi; Pasqua; Avvento; Natale; Quaresima; Canti Mariani; Battesimo; Prima Comunione; Cresima; Matrimonio; Meditazione; Spezzare del pane;
\ifchorded
	%\textnote{Tonalità originale }	% <<< EV COMMENTI (tonalità originale/migliore)
\fi

%%%%%% INTRODUZIONE
\ifchorded
\vspace*{\versesep}
\musicnote{
\begin{minipage}{0.48\textwidth}
\textbf{Intro}
\hfill 
%( \eighthnote \, 80)   % <<  MODIFICA IL TEMPO
% Metronomo: \eighthnote (ottavo) \quarternote (quarto) \halfnote (due quarti)
\end{minipage}
} 	
\vspace*{-\versesep}
\beginverse*

\nolyrics

%---- Prima riga -----------------------------
\vspace*{-\versesep}
\[D] \[G] \[A] \[D]	 % \[*D] per indicare le pennate, \rep{2} le ripetizioni

%---- Ogni riga successiva -------------------
%\vspace*{-\versesep}
%\[G] \[C]  \[D]	

%---- Ev Indicazioni -------------------------			
%\textnote{\textit{(Oppure tutta la strofa)} }	

\endverse
\fi


%%%%% STROFA
\beginverse		%Oppure \beginverse* se non si vuole il numero di fianco
\memorize 		% <<< DECOMMENTA se si vuole utilizzarne la funzione
%\chordsoff		% <<< DECOMMENTA se vuoi una strofa senza accordi
\[D]Pane \[G]di vita \[A]nuo\[D]va, 
\[G]vero \[D]cibo dato agli \[E-]uomi\[A]ni,
\[G]nutri\[D]mento \[E-]che sostiene il \[A]mondo, 
\[B-*]do\[G]no  \[E-]splendido  di \[A]gra\[D]zia.
\endverse




%%%%% STROFA
\beginverse*		%Oppure \beginverse* se non si vuole il numero di fianco
%\memorize 		% <<< DECOMMENTA se si vuole utilizzarne la funzione
%\chordsoff		% <<< DECOMMENTA se vuoi una strofa senza accordi
^Tu sei ^sublime ^frut^to 
^di quell'^albero di ^vi^ta
^che A^damo ^non potè toc^care:
^Ora ^è in  ^Cristo a noi do^na^to.
\endverse







%%%%% RITORNELLO
\beginchorus
\textnote{\textbf{Rit.}}
\[G]Pane \[D]della \[E-]vi\[A]ta, 
\[D]sangue \[F#-]di sal\[G]vez\[A]za,
\[G]vero \[D]corpo, \[E-]vera be\[B-]vanda,
\[E-]cibo di \[B-]grazia per il \[A]mon\[D]do.
\endchorus



%%%%% STROFA
\beginverse		%Oppure \beginverse* se non si vuole il numero di fianco
%\memorize 		% <<< DECOMMENTA se si vuole utilizzarne la funzione
%\chordsoff		% <<< DECOMMENTA se vuoi una strofa senza accordi
^Sei l'A^gnello immo^la^to
^nel cui ^Sangue è la sal^vez^za,
^memo^riale ^della vera ^Pasqua
^del^la ^nuova Alle^an^za.
\endverse



%%%%% STROFA
\beginverse*		%Oppure \beginverse* se non si vuole il numero di fianco
%\memorize 		% <<< DECOMMENTA se si vuole utilizzarne la funzione
%\chordsoff		% <<< DECOMMENTA se vuoi una strofa senza accordi
^Manna ^che nel de^ser^to
^nutri il ^popolo in cam^mi^no,
^sei so^stegno e ^forza nella ^prova
^per ^la ^Chiesa in mezzo al ^mon^do.
\endverse


%%%%% STROFA
\beginverse		%Oppure \beginverse* se non si vuole il numero di fianco
%\memorize 		% <<< DECOMMENTA se si vuole utilizzarne la funzione
\chordsoff		% <<< DECOMMENTA se vuoi una strofa senza accordi
^Vino ^che ci dà ^gio^ia,
^che ri^scalda il nostro ^cuo^re,
^sei per ^noi ^il prezioso ^frutto
^del^la ^vigna del Si^gno^re.
\endverse

%%%%% STROFA
\beginverse*		%Oppure \beginverse* se non si vuole il numero di fianco
%\memorize 		% <<< DECOMMENTA se si vuole utilizzarne la funzione
\chordsoff		% <<< DECOMMENTA se vuoi una strofa senza accordi
^Dalla ^vite ai ^tral^ci
^scorre ^la vitale ^lin^fa
^che ci ^dona ^la vita di^vina,
^scor^re il ^sangue dell'a^mo^re.
\endverse



%%%%% STROFA
\beginverse
\chordsoff
^Al ban^chetto ci in^vi^ti
^che per ^noi hai prepa^ra^to,
^doni all'^uomo ^la tua Sa^pienza,
^do^ni il ^Verbo della ^vi^ta.
\endverse

%%%%% STROFA
\beginverse*		%Oppure \beginverse* se non si vuole il numero di fianco
%\memorize 		% <<< DECOMMENTA se si vuole utilizzarne la funzione
\chordsoff		% <<< DECOMMENTA se vuoi una strofa senza accordi
^Segno ^d'amore e^ter^no
^pegno ^di sublimi ^noz^ze,
^comu^nione ^nell'unico ^corpo
^che ^in ^Cristo noi for^mia^mo.
\endverse



\beginverse
\chordsoff
^Nel tuo ^Sangue è la ^vi^ta
^ed il ^fuoco dello ^Spiri^to,
^la sua ^fiamma in^cendia il nostro ^cuore 
^e ^pu^rifica il ^mon^do.
\endverse
\beginverse*
\chordsoff
^Nel pro^digio dei ^pa^ni
^tu sfa^masti ogni ^uo^mo,
^nel tuo a^more il ^povero è nu^trito
^e ^ri^ceve la tua ^vi^ta.
\endverse




\beginverse
\chordsoff
Sacerdote eterno
Tu sei vittima ed altare,
offri al Padre tutto l'universo,
sacrificio dell'amore.
\endverse
\beginverse*
\chordsoff
Il tuo Corpo è tempio
della lode della Chiesa,
dal costato tu l'hai generata,
nel tuo Sangue l'hai redenta.
\endverse






\beginverse
\chordsoff
Vero Corpo di Cristo
tratto da Maria Vergine,
dal tuo fianco doni a noi la grazia, 
per mandarci tra le genti.
\endverse
\beginverse*
\chordsoff
Dai confini del mondo,
da ogni tempo e ogni luogo
il creato a te renda grazie,
per l'eternità ti adori.
\endverse






\beginverse
\chordsoff
A te Padre la lode,
che donasti il Redentore,
e al Santo Spirito di vita 
sia per sempre onore e gloria. 
\endverse



%%%%%% EV. FINALE

\beginchorus %oppure \beginverse*
\vspace*{1.3\versesep}
\textnote{\textbf{Finale} } %<<< EV. INDICAZIONI

\[B-]\[A]A\[D]men.

\endchorus  %oppure \endverse




\endsong
%------------------------------------------------------------
%			FINE CANZONE
%------------------------------------------------------------



%-------------------------------------------------------------
%			INIZIO	CANZONE
%-------------------------------------------------------------


%titolo: 	Parole di vita
%autore: 	Pierangelo Sequeri
%tonalita: 	Do 



%%%%%% TITOLO E IMPOSTAZONI
\beginsong{Parole di vita}[by={P. Sequeri}] 	% <<< MODIFICA TITOLO E AUTORE
\transpose{0} 						% <<< TRASPOSIZIONE #TONI (0 nullo)
\momenti{Dopo il Vangelo}							% <<< INSERISCI MOMENTI	
% momenti vanno separati da ; e vanno scelti tra:
% Ingresso; Atto penitenziale; Acclamazione al Vangelo; Dopo il Vangelo; Offertorio; Comunione; Ringraziamento; Fine; Santi; Pasqua; Avvento; Natale; Quaresima; Canti Mariani; Battesimo; Prima Comunione; Cresima; Matrimonio; Meditazione; Spezzare del pane;
\ifchorded
	%\textnote{Tonalità migliore }	% <<< EV COMMENTI (tonalità originale/migliore)
\fi

%%%%%% INTRODUZIONE
\ifchorded
\vspace*{\versesep}
\textnote{Intro: \qquad \qquad  }%(\eighthnote 116) % <<  MODIFICA IL TEMPO
% Metronomo: \eighthnote (ottavo) \quarternote (quarto) \halfnote (due quarti)
\vspace*{-\versesep}
\beginverse*

\nolyrics

%---- Prima riga -----------------------------
\vspace*{-\versesep}
\[C] \[G] \[C]	 % \[*D] per indicare le pennate, \rep{2} le ripetizioni

%---- Ogni riga successiva -------------------
%\vspace*{-\versesep}
%\[G] \[C]  \[D]	

%---- Ev Indicazioni -------------------------			
%\textnote{\textit{(Oppure tutta la strofa)} }	

\endverse
\fi

%%%%% STROFA
\beginverse*		%Oppure \beginverse* se non si vuole il numero di fianco
%\memorize 		% <<< DECOMMENTA se si vuole utilizzarne la funzione
%\chordsoff		% <<< DECOMMENTA se vuoi una strofa senza accordi

Pa\[C]role di \[G]vita abbi\[A-]amo ascolt\[E-]ato 
e \[C]gesti d'a\[F]more ve\[D7]demmo tra \[G]noi.

La \[C]nostra spe\[G]ranza è un \[A-]pane spezz\[E-]ato 
la \[C]nostra cert\[F]ezza l'a\[G]more di \[C]Dio.

\endverse

\endsong
%------------------------------------------------------------
%			FINE CANZONE
%------------------------------------------------------------
%-------------------------------------------------------------
%			INIZIO	CANZONE
%-------------------------------------------------------------


%titolo: 	Popoli tutti
%autore: 	Zschech
%tonalita: 	La



%%%%%% TITOLO E IMPOSTAZONI
\beginsong{Popoli tutti acclamate}[by={Zschech}]	
\transpose{0} 						% <<< TRASPOSIZIONE #TONI (0 nullo)
\momenti{Ingresso; Comunione; Fine}							% <<< INSERISCI MOMENTI	
% momenti vanno separati da ; e vanno scelti tra:
% Ingresso; Atto penitenziale; Acclamazione al Vangelo; Dopo il Vangelo; Offertorio; Comunione; Ringraziamento; Fine; Santi; Pasqua; Avvento; Natale; Quaresima; Canti Mariani; Battesimo; Prima Comunione; Cresima; Matrimonio; Meditazione; Spezzare del pane;
\ifchorded
	%\textnote{Tonalità originale }	% <<< EV COMMENTI (tonalità originale/migliore)
\fi


%%%%%% INTRODUZIONE
\ifchorded
\vspace*{\versesep}
\textnote{Intro: \qquad \qquad  }%(\eighthnote 116) % <<  MODIFICA IL TEMPO
% Metronomo: \eighthnote (ottavo) \quarternote (quarto) \halfnote (due quarti)
\vspace*{-\versesep}
\beginverse*

\nolyrics

%---- Prima riga -----------------------------
\vspace*{-\versesep}
\[A] \[E] \[A]	 % \[*D] per indicare le pennate, \rep{2} le ripetizioni

%---- Ogni riga successiva -------------------
%\vspace*{-\versesep}
%\[G] \[C]  \[D]	

%---- Ev Indicazioni -------------------------			
%\textnote{\textit{(Oppure tutta la strofa)} }	

\endverse
\fi




%%%%% STROFA
\beginverse		%Oppure \beginverse* se non si vuole il numero di fianco
\memorize 		% <<< DECOMMENTA se si vuole utilizzarne la funzione
%\chordsoff		% <<< DECOMMENTA se vuoi una strofa senza accordi
\[A] Mio Dio, \[E] Signore, 
\[F#-]nulla è \[E]pari a \[D]te.
Ora e per \[A]sempre, \[D]voglio lo\[A]dare
\[F#-7]il Tuo grande \[G]amor \[D]per \[E4]noi. \[E] 
\[A]  Mia roccia \[E] Tu sei, 
\[F#-]pace e con\[E]forto mi \[D]dai.
Con tutto il \[A]cuore \[D]e le mie \[A]forze,
\[F#-7]sempre io Ti a\[G]do\[D]re\[E4]rò. \[E] 
\endverse


%%%%% RITORNELLO
\beginchorus
\textnote{\textbf{Rit.}}
\[A]Popoli \[F#-]tutti accla\[D]mate al Si\[E]gnore.
\[A]Gloria e po\[F#-]tenza can\[D7+]tiamo al \[E]re.
\[F#-]Mari e monti si \[D]prostrino a te,
al tuo \[E]nome, \[F#-]o Si\[E]gnore.
\[A]Canto di \[F#-]gioia per \[D]quello che \[E]fai,
per \[A]sempre Si\[F#-]gnore con \[D7+]te reste\[E]rò.
\[F#-]Non c'è promessa non \[D]c'è fedel\[E7]tà che in \[A]te.
\endchorus

%%%%% STROFA
\beginverse		%Oppure \beginverse* se non si vuole il numero di fianco
%\memorize 		% <<< DECOMMENTA se si vuole utilizzarne la funzione
%\chordsoff		% <<< DECOMMENTA se vuoi una strofa senza accordi
^ Tu luce ^ d'amore, 
^Spirito ^di Santi^tà
entra nei ^cuori di-^questi tuoi ^figli
chia^mati ad annun^cia^re il ^Re. ^
^ Tu forza ^ d'amore 
^nuova spe^ranza ci ^dai
in questo ^giorno a ^te consa^crato
^gioia immensa ^can^te^rò. ^
\endverse




\endsong
%------------------------------------------------------------
%			FINE CANZONE
%------------------------------------------------------------




%-------------------------------------------------------------
%			INIZIO	CANZONE
%-------------------------------------------------------------


%titolo: 	Prendi questo pane
%autore: 	
%tonalita: 	Do


%%%%%% TITOLO E IMPOSTAZONI
\beginsong{Prendi questo pane}[by={Caidate}] 	% <<< MODIFICA TITOLO E AUTORE
\transpose{0} 						% <<< TRASPOSIZIONE #TONI (0 nullo)
%\preferflats  %SE VOGLIO FORZARE i bemolle come alterazioni
%\prefersharps %SE VOGLIO FORZARE i # come alterazioni
\momenti{Offertorio}							% <<< INSERISCI MOMENTI	
% momenti vanno separati da ; e vanno scelti tra:
% Ingresso; Atto penitenziale; Acclamazione al Vangelo; Dopo il Vangelo; Offertorio; Comunione; Ringraziamento; Fine; Santi; Pasqua; Avvento; Natale; Quaresima; Canti Mariani; Battesimo; Prima Comunione; Cresima; Matrimonio; Meditazione; Spezzare del pane;
\ifchorded
	%\textnote{Tonalità migliore }	% <<< EV COMMENTI (tonalità originale/migliore)
\fi


%%%%%% INTRODUZIONE
\ifchorded
\vspace*{\versesep}
\textnote{Intro: \qquad \qquad  }%(\eighthnote 116) % <<  MODIFICA IL TEMPO
% Metronomo: \eighthnote (ottavo) \quarternote (quarto) \halfnote (due quarti)
\vspace*{-\versesep}
\beginverse*

\nolyrics

%---- Prima riga -----------------------------
\vspace*{-\versesep}
\[C] \[E-] \[A-] \[C]	 % \[*D] per indicare le pennate, \rep{2} le ripetizioni

%---- Ogni riga successiva -------------------
%\vspace*{-\versesep}
%\[G] \[C]  \[D]	

%---- Ev Indicazioni -------------------------			
%\textnote{\textit{(Oppure tutta la strofa)} }	

\endverse
\fi

%%%%% STROFA
\beginverse		%Oppure \beginverse* se non si vuole il numero di fianco
\memorize 		% <<< DECOMMENTA se si vuole utilizzarne la funzione
%\chordsoff		% <<< DECOMMENTA se vuoi una strofa senza accordi

\[C]Prendi questo \[E-]pane e questo \[A-]vino \[C]
\[F]il mio corpo e sangue dò per \[G]voi
\[C]non siete più \[E]servi, ma \[A-]miei a\[C]mici
\[C7]ricor\[F]date e poi \[G]fate come \[C]me.

\endverse

%%%%% STROFA
\beginverse		%Oppure \beginverse* se non si vuole il numero di fianco
\memorize 		% <<< DECOMMENTA se si vuole utilizzarne la funzione
%\chordsoff		% <<< DECOMMENTA se vuoi una strofa senza accordi

^Mi sarà dis^cepolo e fra^tello ^
^chi per un amico mori^rà
^non c'è amore al ^mondo \brk che ^sia più ^grande
^ricor^date e poi ^fate come ^me.

\endverse

%%%%% STROFA
\beginverse		%Oppure \beginverse* se non si vuole il numero di fianco
\memorize 		% <<< DECOMMENTA se si vuole utilizzarne la funzione
%\chordsoff		% <<< DECOMMENTA se vuoi una strofa senza accordi

^Voi che siete ^sempre stati in^sieme ^
^voi credete sempre nel suo a^more
^gireremo il ^mondo e la ^sua Pa^rola
^ogni ^uomo con ^noi conosce^rà.

\endverse

%%%%%% EV. FINALE

\beginverse*
\vspace*{1.3\versesep}
\textnote{Finale \textit{}} %<<< EV. INDICAZIONI

\[C]Gireremo \[E]il mondo e la \[A-]sua Pa\[C]rola
\[C7]ogni \[F]uomo con \[G]noi conosce\[C*]rà.

\endverse

\endsong
%------------------------------------------------------------
%			FINE CANZONE
%------------------------------------------------------------
%-------------------------------------------------------------
%			INIZIO	CANZONE
%-------------------------------------------------------------


%titolo: 	Proteggimi o Dio
%autore: 	Gallotta
%tonalita: 	Re 



%%%%%% TITOLO E IMPOSTAZONI
\beginsong{Proteggimi o Dio}[by={A. Gallotta}] 	% <<< MODIFICA TITOLO E AUTORE
\transpose{0} 						% <<< TRASPOSIZIONE #TONI (0 nullo)
\momenti{Ingresso}							% <<< INSERISCI MOMENTI	
% momenti vanno separati da ; e vanno scelti tra:
% Ingresso; Atto penitenziale; Acclamazione al Vangelo; Dopo il Vangelo; Offertorio; Comunione; Ringraziamento; Fine; Santi; Pasqua; Avvento; Natale; Quaresima; Canti Mariani; Battesimo; Prima Comunione; Cresima; Matrimonio; Meditazione; Spezzare del pane;
\ifchorded
	%\textnote{Tonalità originale }	% <<< EV COMMENTI (tonalità originale/migliore)
\fi



%%%%%% INTRODUZIONE
\ifchorded
\vspace*{\versesep}
\musicnote{
\begin{minipage}{0.48\textwidth}
\textbf{Intro}
\hfill 
%( \eighthnote \, 80)   % <<  MODIFICA IL TEMPO
% Metronomo: \eighthnote (ottavo) \quarternote (quarto) \halfnote (due quarti)
\end{minipage}
} 	
\vspace*{-\versesep}
\beginverse*

\nolyrics

%---- Prima riga -----------------------------
\vspace*{-\versesep}
\[D]  \[D]	 % \[*D] per indicare le pennate, \rep{2} le ripetizioni

%---- Ogni riga successiva -------------------
%\vspace*{-\versesep}
%\[G] \[C]  \[D]	

%---- Ev Indicazioni -------------------------			
%\textnote{\textit{(Oppure tutta la strofa)} }	

\endverse
\fi



%%%%% STROFA
\beginverse		%Oppure \beginverse* se non si vuole il numero di fianco
\memorize 		% <<< DECOMMENTA se si vuole utilizzarne la funzione
%\chordsoff		% <<< DECOMMENTA se vuoi una strofa senza accordi
\[D]Proteggimi, o \[A]Dio: in \[G*]te io \[E*]mi \[A]rifugio.
\[B-]Ho detto a \[D]lui: “Sei \[G*]tu il mi\[E-*]o Signo\[A]re,
\[D]senza di \[A]te \[G]non ho bene al\[A]cuno”.
\[B-]Nelle tue \[D]mani, Si\[G*]gnore, è \[E*]la mia \[A]vita!
\endverse




%%%%% RITORNELLO
\beginchorus
\textnote{\textbf{Rit.}}
\[F]Tu mi \[B&]indicherai il sen\[C]tiero della \[F]vita
\[B&]gioia piena nella \[F]tua presenza,
\[G-]dolcezza senza \[C]fine.
\[F]Tu mi indiche\[B&]rai il sen\[C]tiero della \[F]vita
\[B&]gioia \[C]piena nella \[D-]tua presenza,
\[G-]dolcezza senza \[A]fine.
\endchorus


%%%%% STROFA
\beginverse		%Oppure \beginverse* se non si vuole il numero di fianco
%\memorize 		% <<< DECOMMENTA se si vuole utilizzarne la funzione
%\chordsoff		% <<< DECOMMENTA se vuoi una strofa senza accordi

^Benedico ^Dio che ^m'ha da^to con^siglio;
^anche di ^notte il mio ^cuore m'^istru^isce.
^Innanzi a ^me ^sempre il Si^gnore,
^sta alla mia ^destra, non ^posso ^vacil^lare.

\endverse


%%%%% RITORNELLO
\beginchorus
\textnote{\textbf{Rit.}}
\[F]Tu mi \[B&]indicherai il sen\[C]tiero della \[F]vita
\[B&]gioia piena nella \[F]tua presenza,
\[G-]dolcezza senza \[C]fine.
\[F]Tu mi indiche\[B&]rai il sen\[C]tiero della \[F]vita
\[B&]gioia \[C]piena nella \[D-]tua presenza,
\[G-]dolcezza senza \[A]fine.
\endchorus


%%%%% STROFA
\beginverse		%Oppure \beginverse* se non si vuole il numero di fianco
%\memorize 		% <<< DECOMMENTA se si vuole utilizzarne la funzione
%\chordsoff		% <<< DECOMMENTA se vuoi una strofa senza accordi

^Mia eredi^tà, mio ^calice^ è il Si^gnore,
^per me la ^sorte è su ^luoghi ^deli^ziosi.
^Lieto e se^reno ^è il cuore ^mio,
^luce e spe^ranza ai miei ^passi ^tu da^rai.

\endverse

%%%%% RITORNELLO
\beginchorus
\textnote{\textbf{Rit.}}
\[F]Tu mi \[B&]indicherai il sen\[C]tiero della \[F]vita
\[B&]gioia piena nella \[F]tua presenza,
\[G-]dolcezza senza \[C]fine.
\[F]Tu mi indiche\[B&]rai il sen\[C]tiero della \[F]vita
\[B&]gioia \[C]piena nella \[D-]tua presenza,
\[G-]dolcezza senza \[A]fi-\[A]i-\[D]ne.
\endchorus

\endsong








%QQQ
%-------------------------------------------------------------
%			INIZIO	CANZONE
%-------------------------------------------------------------


%titolo: 	Quale gioia è star con te
%autore: 	D. Branca, L. Ciancio
%tonalita: 	Sol 



%%%%%% TITOLO E IMPOSTAZONI
\beginsong{Quale gioia è star con te}[by={D. Branca, L. Ciancio}]	% <<< MODIFICA TITOLO E AUTORE
\transpose{0} 						% <<< TRASPOSIZIONE #TONI (0 nullo)
%\preferflats  %SE VOGLIO FORZARE i bemolle come alterazioni
%\prefersharps %SE VOGLIO FORZARE i # come alterazioni
\momenti{}							% <<< INSERISCI MOMENTI	
% momenti vanno separati da ; e vanno scelti tra:
% Ingresso; Atto penitenziale; Acclamazione al Vangelo; Dopo il Vangelo; Offertorio; Comunione; Ringraziamento; Fine; Santi; Pasqua; Avvento; Natale; Quaresima; Canti Mariani; Battesimo; Prima Comunione; Cresima; Matrimonio; Meditazione; Spezzare del pane;
\ifchorded
	%\textnote{Tonalità migliore }	% <<< EV COMMENTI (tonalità originale/migliore)
\fi


%%%%%% INTRODUZIONE
\ifchorded
\vspace*{\versesep}
\musicnote{
\begin{minipage}{0.48\textwidth}
\textbf{Intro}
\hfill 
%( \eighthnote \, 80)   % <<  MODIFICA IL TEMPO
% Metronomo: \eighthnote (ottavo) \quarternote (quarto) \halfnote (due quarti)
\end{minipage}
} 	
\vspace*{-\versesep}
\beginverse*

\nolyrics

%---- Prima riga -----------------------------
\vspace*{-\versesep}
\[D] \[A] \[E-] \[B-] % \[*D] per indicare le pennate, \rep{2} le ripetizioni

%---- Ogni riga successiva -------------------
\vspace*{-\versesep}
 \[D] \[A] \[G] \[A] \rep{2}


%---- Ev Indicazioni -------------------------			
%\textnote{\textit{(Oppure tutta la strofa)} }	

\endverse
\fi



%%%%% STROFA
\beginverse		%Oppure \beginverse* se non si vuole il numero di fianco
\memorize 		% <<< DECOMMENTA se si vuole utilizzarne la funzione
%\chordsoff		% <<< DECOMMENTA se vuoi una strofa senza accordi
\[D4]Ogni volta che \[D]ti cerco, \brk \[E-] ogni volta che ti invoco,
\[D]Sempre mi ac\[B-]cogli Si\[A4]gnor. \[A] 
\[B-7]Grandi sono i tuoi prodigi,\brk \[G2] Tu sei buono verso tutti, 
\[D/F#]Santo Tu \[G]regni tra \[A4]noi. \[A] 
\endverse


%%%%% RITORNELLO
\beginchorus
\textnote{\textbf{Rit.}}
Qua-le \[D]gioia è star con \[A]Te Gesù \[E-]vivo e vi\[B-]cino,
\[D]bello è dar \[A]lode a Te, \[G]Tu sei il Si\[A]gnor.
Quale \[D]dono è aver cre\[A]duto in Te \brk \[E-]che non mi abban\[B-]doni,
\[D]Io per sempre a\[A]biterò 
\[G]la Tua \[A]casa, mio \[D]Re.
\endchorus





%%%%%% EV. INTERMEZZO
\beginverse*
\vspace*{1.3\versesep}
{
	\nolyrics
	\textnote{Intermezzo strumentale}
	
	\ifchorded

	%---- Prima riga -----------------------------
    \vspace*{-\versesep}
     \[A] \[E-] \[B-] % \[*D] per indicare le pennate, \rep{2} le ripetizioni

    %---- Ogni riga successiva -------------------
    \vspace*{-\versesep}
    \[D] \[A] \[G] \[A] 


	\fi
	%---- Ev Indicazioni -------------------------			
	%\textnote{\textit{(ripetizione della strofa)}} 
	 
}
\vspace*{\versesep}
\endverse





%%%%% STROFA
\beginverse		%Oppure \beginverse* se non si vuole il numero di fianco
%\memorize 		% <<< DECOMMENTA se si vuole utilizzarne la funzione
%\chordsoff		% <<< DECOMMENTA se vuoi una strofa senza accordi
^Hai guarito il ^mio dolore, \brk ^ hai cambiato questo cuore,
^oggi ri^nasco, Si^gnor. ^
^Grandi sono i tuoi prodigi, \brk ^ Tu sei buono verso tutti, 
^santo Tu ^regni tra ^noi. ^
\endverse


%%%%% RITORNELLO
\beginchorus
\textnote{\textbf{Rit.}}
Qua-le \[D]gioia è star con \[A]Te Gesù \[E-]vivo e vi\[B-]cino,
\[D]bello è dar \[A]lode a Te, \[G]Tu sei il Si\[A]gnor.
Quale \[D]dono è aver cre\[A]duto in Te \brk \[E-]che non mi abban\[B-]doni,
\[D]Io per sempre a\[A]biterò 
\[G]la Tua \[A]casa, mio \[D]Re.
\endchorus





%%%%%% EV. INTERMEZZO
\beginverse*
\vspace*{1.3\versesep}
{
	\nolyrics
	\textnote{Intermezzo strumentale}
	\textnote{(aumento di tonalità)}
	
	\ifchorded

	%---- Prima riga -----------------------------
    \vspace*{-\versesep}
    \[A] \[E-] \[B-] % \[*D] per indicare le pennate, \rep{2} le ripetizioni

    %---- Ogni riga successiva -------------------
    \vspace*{-\versesep}
    \[D] \[A] \[G] \[A]  \quad \[E]


	\fi
	%---- Ev Indicazioni -------------------------			
	%\textnote{\textit{(ripetizione della strofa)}} 
	 
}
\vspace*{\versesep}
\endverse



%%%%% STROFA
\beginverse		%Oppure \beginverse* se non si vuole il numero di fianco
%\memorize 		% <<< DECOMMENTA se si vuole utilizzarne la funzione
%\chordsoff		% <<< DECOMMENTA se vuoi una strofa senza accordi
\[E]Hai salvato la mia vita, \[F#-7]hai aperto la mia bocca,
\[E/G#]Canto per \[C#-]Te, mio Si\[B4]gnor.\[B] 
\[C#-7]Grandi sono i tuoi prodigi, \brk \[A]tu sei buono verso tutti,
\[E]santo Tu \[A]regni tra \[B]noi. \[B] 
\endverse




%%%%% RITORNELLO

\beginchorus
\textnote{\textbf{Rit.}}
Quale \[E]gioia è star con \[B]Te Gesù \[F#-]vivo e \[C#-]vicino,
\[E]Bello è dar \[B7]lode a Te, \[A]Tu sei il Si\[B7]gnor.
Quale \[E]dono è aver creduto in \[B]Te \brk \[F#-]che non mi abban\[C#-]doni.
\[E]Io per sempre a\[B7]biterò 
\[A]la Tua \[B]casa, \brk mio \[E4/C#]Re.
\endchorus

%%%%% BRIDGE
\beginverse*		%Oppure \beginverse* se non si vuole il numero di fianco
%\memorize 		% <<< DECOMMENTA se si vuole utilizzarne la funzione
%\chordsoff		% <<< DECOMMENTA se vuoi una strofa senza accordi
\vspace*{1.3\versesep}
\textnote{\textbf{Bridge}}
Ti lode\[B4/G#]rò \echo{ti loderò}, \brk Ti adore\[C#4]rò \echo{ti adorerò}
Ti cante\[A/B]rò, cante\[B&/C]remo.
\endverse




%%%%% RITORNELLO
\prefersharps
\beginchorus
\textnote{\textbf{Rit.}}
Quale \[F]gioia è star con \[C7]Te Gesù \[G-]vivo e \[D-]vicino,
\[F]bello è dar \[C7]lode a Te, \[B&]Tu sei il Si\[C7]gnor.
Quale \[F]dono è aver cre\[C7]duto in Te \brk \[G-]che non mi abban\[D-]doni.
\[F]Io per sempre a\[C7]biterò 
\[B&]la Tua \[C]casa....  \rep{3}

\endchorus

%%%%%% EV. FINALE

\beginchorus %oppure \beginverse*
\vspace*{1.3\versesep}
\textnote{\textbf{Finale }} %<<< EV. INDICAZIONI
...mio \[F4]Re. \[F]
\[C4]la Tua casa, mio \[F4]Re. \[F] 
\[C4]Tu sei il Signor... \quad \[F*]mio \[F*]Re!
\endchorus  %oppure \endverse








\endsong
%------------------------------------------------------------
%			FINE CANZONE
%------------------------------------------------------------





%RRR
%-------------------------------------------------------------
%			INIZIO	CANZONE
%-------------------------------------------------------------


%titolo: 	Re dei re
%autore: 	E. Munda, G. Pretto, L. Christille
%tonalita: 	Mi- 



%%%%%% TITOLO E IMPOSTAZONI
\beginsong{Re dei re}[by={E. Munda, G. Pretto, L. Christille}] 	% <<< MODIFICA TITOLO E AUTORE
\transpose{0} 						% <<< TRASPOSIZIONE #TONI (0 nullo)
%\preferflats  %SE VOGLIO FORZARE i bemolle come alterazioni
%\prefersharps %SE VOGLIO FORZARE i # come alterazioni
\momenti{Meditazione; Ringraziamento; Comunione; Fine}							% <<< INSERISCI MOMENTI	
% momenti vanno separati da ; e vanno scelti tra:
% Ingresso; Atto penitenziale; Acclamazione al Vangelo; Dopo il Vangelo; Offertorio; Comunione; Ringraziamento; Fine; Santi; Pasqua; Avvento; Natale; Quaresima; Canti Mariani; Battesimo; Prima Comunione; Cresima; Matrimonio; Meditazione; Spezzare del pane;
\ifchorded
	%\textnote{Tonalità migliore }	% <<< EV COMMENTI (tonalità originale/migliore)
\fi


%%%%%% INTRODUZIONE
\ifchorded
\vspace*{\versesep}
\textnote{Intro: \qquad \qquad  (\quarternote 106)} % <<  MODIFICA IL TEMPO
% Metronomo: \eighthnote (ottavo) \quarternote (quarto) \halfnote (due quarti)
\vspace*{-\versesep}
\beginverse*

\nolyrics

%---- Prima riga -----------------------------
\vspace*{-\versesep}
\[E-] \[C] \[G] \[D]	 \rep{2}% \[*D] per indicare le pennate, \rep{2} le ripetizioni

%---- Ogni riga successiva -------------------
%\vspace*{-\versesep}
%\[G] \[C]  \[D]	

%---- Ev Indicazioni -------------------------			
%\textnote{\textit{(Oppure tutta la strofa)} }	

\endverse
\fi




%%%%% STROFA
\beginverse		%Oppure \beginverse* se non si vuole il numero di fianco
\memorize 		% <<< DECOMMENTA se si vuole utilizzarne la funzione
%\chordsoff		% <<< DECOMMENTA se vuoi una strofa senza accordi

\[E-]Hai solle\[C]vato \brk i nostri \[G]volti dalla \[D]polvere, 
\[E-] le nostre \[C]colpe hai por\[G]tato su di \[D]te. 
\[E-] Signore \[C]ti sei fatto \brk \[G]uomo in tutto \[D]come noi 
\[E-]per \[C]a-\[G]mo-\[D]re, 

\endverse
\beginverse*

^Figlio dell’altissimo, ^povero tra i poveri, 
^vieni a dimorare tra ^noi. 
^Dio dell’impossibile, ^Re di tutti i secoli 
^vieni nella tua mae^stà. 

\endverse



%%%%% RITORNELLO
\beginchorus
\textnote{\textbf{Rit.}}

^Re dei ^re 
i ^popoli ti acclamano, \brk i ^cieli ti proclamano 
^Re dei ^re 

^luce degli uomini, \brk ^regna con il tuo amore tra \[E-]no-\[C]o-\[G]o\[D]i 
\[E-]Oo\[C]oh \[G]\[D] \brk \[E-]Oo\[C]oh \[G]\[D]  \brk \[E-]Oo\[C]oh \[G]\[D]  
\[E-]Oh... \[E-] \[E-*]

\endchorus


%%%%%% EV. INTERMEZZO
\beginverse*
\vspace*{1.3\versesep}
{
	\nolyrics
	\textnote{Intermezzo musicale}
	
	\ifchorded

    %---- Prima riga -----------------------------
    \vspace*{-\versesep}
    \[E-] \[C] \[G] \[D]	 \rep{2}% \[*D] per indicare le pennate, \rep{2} le ripetizioni



	\fi
	%---- Ev Indicazioni -------------------------			
	%\textnote{\textit{(ripetizione della strofa)}} 
	 
}
\vspace*{\versesep}
\endverse


%%%%% STROFA
\beginverse		%Oppure \beginverse* se non si vuole il numero di fianco
%\memorize 		% <<< DECOMMENTA se si vuole utilizzarne la funzione
%\chordsoff		% <<< DECOMMENTA se vuoi una strofa senza accordi

^ Ci hai riscat^tati \brk dalla s^tretta delle ^tenebre, 
^ perché po^tessimo glo^rificare ^te. 
^ Hai river^sato in noi \brk la ^vita del tuo ^Spirito 
^per ^a-^mo-^re, 

\endverse
\beginverse*

^Figlio dell’altissimo, ^povero tra i poveri, 
^vieni a dimorare tra ^noi. 
^Dio dell’impossibile, ^Re di tutti i secoli 
^vieni nella tua mae^stà. 

\endverse



%%%%% RITORNELLO
\beginchorus
\textnote{\textbf{Rit.}}

^Re dei ^re 
i ^popoli ti acclamano, \brk i ^cieli ti proclamano 
^Re dei ^re 

^luce degli uomini, \brk ^regna con il tuo amore tra \[E-]no-\[C]o-\[G]o\[D]i 
\[E-]Oo\[C]oh \[G]\[D] \brk \[E-]Oo\[C]oh \[G]\[D]  \brk \[E-]Oo\[C]oh \[G]\[D]  
\[E-]Oh... \[E-] \[E-] \[E-]

\endchorus




%%%%% BRIDGE
\beginverse*		%Oppure \beginverse* se non si vuole il numero di fianco
%\memorize 		% <<< DECOMMENTA se si vuole utilizzarne la funzione
%\chordsoff		% <<< DECOMMENTA se vuoi una strofa senza accordi
\vspace*{1.3\versesep}
\textnote{Bridge} %<<< EV. INDICAZIONI

^Tua ^è la ^glo^ria per ^se-^em^pre, ^ 
^Tua ^è la ^glo^ria per ^se-^em^pre, ^ 
\endverse
\beginverse*
\vspace*{-\versesep}
^glo^ria, ^glo^ria, ^glo^ria, ^glo^ria.  

\endverse
\beginverse*

^Figlio dell’altissimo, ^povero tra i poveri, 
^vieni a dimorare tra ^noi. 
^Dio dell’impossibile, ^Re di tutti i secoli 
^vieni nella tua mae^stà. 

\endverse




%%%%% RITORNELLO
\beginchorus
\textnote{\textbf{Rit.}}

^Re dei ^re 
i ^popoli ti acclamano, \brk i ^cieli ti proclamano 
^Re dei ^re 

^luce degli uomini, \brk ^regna con il tuo amore tra \[E-]no-\[C]o-\[G]o\[D]i 
\[E-]Oo\[C]oh \[G]\[D] \brk \[E-]Oo\[C]oh \[G]\[D]  \brk \[E-]Oo\[C]oh \[G]\[D]  
\[E-]Oh... \[E-] \[E-] \[E-] \[E-*]

\endchorus




\endsong
%------------------------------------------------------------
%			FINE CANZONE
%------------------------------------------------------------



%++++++++++++++++++++++++++++++++++++++++++++++++++++++++++++
%			CANZONE TRASPOSTA
%++++++++++++++++++++++++++++++++++++++++++++++++++++++++++++
\ifchorded
%decremento contatore per avere stesso numero
\addtocounter{songnum}{-1} 
\beginsong{Re dei re}[by={E. Munda, G. Pretto, L. Christille}] 	% <<< MODIFICA TITOLO E AUTORE
\transpose{-2} 							% <<< TRASPOSIZIONE #TONI + - (0 nullo)
\preferflats  %SE VOGLIO FORZARE i bemolle come alterazioni
%\prefersharps %SE VOGLIO FORZARE i # come alterazioni
\ifchorded
	\textnote{Con aumento di tonalità}	% <<< EV COMMENTI (tonalità originale/migliore)
\fi




%%%%%% INTRODUZIONE
\ifchorded
\vspace*{\versesep}
\textnote{Intro: \qquad \qquad  }%(\eighthnote 116) % <<  MODIFICA IL TEMPO
% Metronomo: \eighthnote (ottavo) \quarternote (quarto) \halfnote (due quarti)
\vspace*{-\versesep}
\beginverse*

\nolyrics

%---- Prima riga -----------------------------
\vspace*{-\versesep}
\[E-] \[C] \[G] \[D]	 \rep{2}% \[*D] per indicare le pennate, \rep{2} le ripetizioni

%---- Ogni riga successiva -------------------
%\vspace*{-\versesep}
%\[G] \[C]  \[D]	

%---- Ev Indicazioni -------------------------			
%\textnote{\textit{(Oppure tutta la strofa)} }	

\endverse
\fi




%%%%% STROFA
\beginverse		%Oppure \beginverse* se non si vuole il numero di fianco
\memorize 		% <<< DECOMMENTA se si vuole utilizzarne la funzione
%\chordsoff		% <<< DECOMMENTA se vuoi una strofa senza accordi

\[E-]Hai solle\[C]vato \brk i nostri \[G]volti dalla \[D]polvere, 
\[E-] le nostre \[C]colpe hai por\[G]tato su di \[D]te. 
\[E-] Signore \[C]ti sei fatto \brk \[G]uomo in tutto \[D]come noi 
\[E-]per \[C]a-\[G]mo-\[D]re, 

\endverse
\beginverse*

^Figlio dell’altissimo, ^povero tra i poveri, 
^vieni a dimorare tra ^noi. 
^Dio dell’impossibile, ^Re di tutti i secoli 
^vieni nella tua mae^stà. 

\endverse




%%%%% RITORNELLO
\beginchorus
\textnote{\textbf{Rit.}}

^Re dei ^re 
i ^popoli ti acclamano, \brk i ^cieli ti proclamano 
^Re dei ^re 

^luce degli uomini, \brk ^regna con il tuo amore tra \[E-]no-\[C]o-\[G]o\[D]i 
\[E-]Oo\[C]oh \[G]\[D] \brk \[E-]Oo\[C]oh \[G]\[D]  \brk \[E-]Oo\[C]oh \[G]\[D]  
\[E-]Oh... \[E-] \[E-] \[E-]

\endchorus


\transpose{2}
%%%%%% EV. INTERMEZZO
\beginverse*
\vspace*{1.3\versesep}
{
	\nolyrics
	\textnote{Intermezzo con aumento di tonalità}
	
	\ifchorded

    %---- Prima riga -----------------------------
    \vspace*{-\versesep}
    \[E-] \[C] \[G] \[D]	 \rep{2}% \[*D] per indicare le pennate, \rep{2} le ripetizioni



	\fi
	%---- Ev Indicazioni -------------------------			
	%\textnote{\textit{(ripetizione della strofa)}} 
	 
}
\vspace*{\versesep}
\endverse

%%%%% STROFA
\beginverse		%Oppure \beginverse* se non si vuole il numero di fianco
%\memorize 		% <<< DECOMMENTA se si vuole utilizzarne la funzione
%\chordsoff		% <<< DECOMMENTA se vuoi una strofa senza accordi

^ Ci hai riscat^tati \brk dalla s^tretta delle ^tenebre, 
^ perché po^tessimo glo^rificare ^te. 
^ Hai river^sato in noi \brk la ^vita del tuo ^Spirito 
^per ^a-^mo-^re, 

\endverse
\beginverse*

^Figlio dell’altissimo, ^povero tra i poveri, 
^vieni a dimorare tra ^noi. 
^Dio dell’impossibile, ^Re di tutti i secoli 
^vieni nella tua mae^stà. 

\endverse



%%%%% RITORNELLO
\beginchorus
\textnote{\textbf{Rit.}}

^Re dei ^re 
i ^popoli ti acclamano, \brk i ^cieli ti proclamano 
^Re dei ^re 

^luce degli uomini, \brk ^regna con il tuo amore tra \[E-]no-\[C]o-\[G]o\[D]i 
\[E-]Oo\[C]oh \[G]\[D] \brk \[E-]Oo\[C]oh \[G]\[D]  \brk \[E-]Oo\[C]oh \[G]\[D]  
\[E-]Oh... \[E-] \[E-] \[E-]

\endchorus




%%%%% BRIDGE
\beginverse*		%Oppure \beginverse* se non si vuole il numero di fianco
%\memorize 		% <<< DECOMMENTA se si vuole utilizzarne la funzione
%\chordsoff		% <<< DECOMMENTA se vuoi una strofa senza accordi
\vspace*{1.3\versesep}
\textnote{Bridge} %<<< EV. INDICAZIONI

^Tua ^è la ^glo^ria per ^se-^em^pre, ^ 
^Tua ^è la ^glo^ria per ^se-^em^pre, ^ 
\endverse
\beginverse*
\vspace*{-\versesep}
^glo^ria, ^glo^ria, ^glo^ria, ^glo^ria.  

\endverse
\beginverse*

^Figlio dell’altissimo, ^povero tra i poveri, 
^vieni a dimorare tra ^noi. 
^Dio dell’impossibile, ^Re di tutti i secoli 
^vieni nella tua mae^stà. 

\endverse


%%%%% RITORNELLO
\beginchorus
\textnote{\textbf{Rit.}}

^Re dei ^re 
i ^popoli ti acclamano, \brk i ^cieli ti proclamano 
^Re dei ^re 

^luce degli uomini, \brk ^regna con il tuo amore tra \[E-]no-\[C]o-\[G]o\[D]i 
\[E-]Oo\[C]oh \[G]\[D] \brk \[E-]Oo\[C]oh \[G]\[D]  \brk \[E-]Oo\[C]oh \[G]\[D]  
\[E-]Oh... \[E-] \[E-] \[E-] \[E-*]

\endchorus


\endsong
\fi
%++++++++++++++++++++++++++++++++++++++++++++++++++++++++++++
%			FINE CANZONE TRASPOSTA
%++++++++++++++++++++++++++++++++++++++++++++++++++++++++++++
%titolo{Re di gloria}
%autore{Marranzino, De Luca}
%album{Cantiamo con gioia}
%tonalita{Sol}
%famiglia{Liturgica}
%gruppo{}
%momenti{Comunione}
%identificatore{re_di_gloria}
%data_revisione{2011_12_31}
%trascrittore{Francesco Endrici - Manuel Toniato}
\beginsong{Re di gloria}[by={Marranzino, De\ Luca}]

\ifchorded
\beginverse*
\vspace*{-0.8\versesep}
{\nolyrics \[A-7] \[C] \[D] \[G] }
\vspace*{-\versesep}
\endverse
\fi
\beginverse
\[G]Ho incontrato Te Gesù \brk e ogni \[D]cosa in me è cambiata
\[A-]tutta la mia \[A-7]vita ora ti \[D]appar\[A-7]tie\[D]ne
\[G]tutto il mio passato io lo a\ch{C}{f}{f}{ff}i\[B7]do a \[E-]Te \[E-7] 
Ge\[A-]sù Re di \[C7+]gloria mio Si\[D4]gnor\[D7].
\endverse

\beginverse
\chordsoff
Tutto in Te riposa, la mia mente il mio cuore
trovo pace in Te Signor,Tu mi dai la gioia
voglio stare insieme a Te,non lasciarti mai
Gesù Re di gloria mio Signor.
\endverse

\beginchorus
\[D7]Dal Tuo a\[G]more chi \[C7+]mi separe\[D4]rà \[D] 
sulla \[A-7]croce hai \[C7+]dato la \[D4]vita per \[D]me
u\[D7]na co\[G]rona di \[C7+]gloria mi da\[D4]rai \[D] 
quando un \[A-7]gior\[C]no \[D]ti ve\[G]drò.
\endchorus

\beginverse
\chordsoff
Tutto in Te riposa,la mia mente il mio cuore
trovo pace in Te Signor, Tu mi dai la gioia vera
voglio stare insieme a Te, non lasciarti mai
Gesù Re di gloria mio Signor.
\endverse

\beginchorus
\[D7]Dal Tuo a\[G]more chi \[C7+]mi separe\[D4]rà \[D] 
sulla \[A-7]croce hai \[C7+]dato la vita per \[D]me
u\[D7]na co\[G]rona di \[C7+]gloria mi da\[D4]rai \[D] 
quando un \[A-7]gior\[C]no \[D]ti ve\[E&]drò. \[(Sol)]
\[E&7]Dal Tuo a\[Ab]more chi \[D&7+]mi separe\[E&]rà \ldots
\endchorus
\ifchorded
\beginverse*
\vspace*{-\versesep}
{\nolyrics \[B&-7] \[D&7+] \[E&]  \[E&7] \[Ab] \[D&7+] \[E&4] \[E&] }
\endverse
\fi
\beginverse*\bfseries
Io ti a\[B&-7]spet\[D&7+]to \[E&]mio Si\[F-]gnor \[Ab] 
io ti a\[B&-7]spet\[Ab]to \[E&]mio \[C]Si\[F-]gnor \[Ab] 
io ti a\[B&-7]spet\[Ab]to \[E&]mio \[Ab4]Re! \[Ab] 
\endverse
\textnote{Nel caso non si cambi tonalità, riportiamo il finale in Sol maggiore:}
\beginverse*\bfseries
\[D7]Dal Tuo a\[G]more chi \[C7+]mi separe\[D]rà \ldots
\endverse
\ifchorded
\beginverse*
\vspace*{-\versesep}
{\nolyrics \[A-7]  \[C7+]  \[D]  \[D7]  \[G]  \[C7+]  \[D4]  \[D] }
\endverse
\fi
\beginverse*\bfseries
Io ti a\[A-7]spet\[C7+]to \[D]mio Si\[E-]gnor \[G] 
io ti a\[A-7]spet\[G]to \[D]mio \[B]Si\[E-]gnor \[G] 
io ti a\[A-7]spet\[G]to \[D]mio \[G4]Re! \[G] 
\endverse

\endsong



%-------------------------------------------------------------
%			INIZIO	CANZONE
%-------------------------------------------------------------


%titolo: 	Resta accanto a me
%autore: 	Gen Verde
%tonalita: 	Mi 



%%%%%% TITOLO E IMPOSTAZONI
\beginsong{Resta accanto a me}[by={Gen Verde}] 	% <<< MODIFICA TITOLO E AUTORE
\transpose{0} 						% <<< TRASPOSIZIONE #TONI (0 nullo)
\momenti{Ringraziamento; Fine}							% <<< INSERISCI MOMENTI	
% momenti vanno separati da ; e vanno scelti tra:
% Ingresso; Atto penitenziale; Acclamazione al Vangelo; Dopo il Vangelo; Offertorio; Comunione; Ringraziamento; Fine; Santi; Pasqua; Avvento; Natale; Quaresima; Canti Mariani; Battesimo; Prima Comunione; Cresima; Matrimonio; Meditazione;
\ifchorded
	%\textnote{Tonalità originale }	% <<< EV COMMENTI (tonalità originale/migliore)
\fi

%%%%%% INTRODUZIONE
\ifchorded
\vspace*{\versesep}
\textnote{Intro: \qquad \qquad  }%(\eighthnote 116) % << MODIFICA IL TEMPO
% Metronomo: \eighthnote (ottavo) \quarternote (quarto) \halfnote (due quarti)
\vspace*{-\versesep}
\beginverse*

\nolyrics

%---- Prima riga -----------------------------
\vspace*{-\versesep}
\[A] \[E]  \[B]	 % \[*D] per indicare le pennate, \rep{2} le ripetizioni

%---- Ogni riga successiva -------------------
%\vspace*{-\versesep}
%\[G] \[C]  \[D]	

%---- Ev Indicazioni -------------------------			
%\textnote{\textit{(Oppure tutta la strofa)} }	

\endverse
\fi

%%%%% RITORNELLO
\beginchorus
\textnote{\textbf{Rit.}}

\[E]Ora \[B]vado \[A]sulla mia \[E]strada
\[F#-]con l'a\[G#-]more \[A]tuo che mi \[B]guida.
\[E]O Si\[B]gnore, o\[A]vunque io \[E]vada,
\[A]resta ac\[B]canto a \[E]me.
\[E]Io ti \[B]prego, \[A]stammi vi\[E]cino
\[F#-]ogni \[G#-]passo \[A]del mio cam\[B]mino,
\[E]ogni \[B]notte, \[A]ogni mat\[E]tino,
\[A]resta ac\[B]canto a \[E]me.

\endchorus

%%%%% STROFA
\beginverse		%Oppure \beginverse* se non si vuole il numero di fianco
%\memorize 		% <<< DECOMMENTA se si vuole utilizzarne la funzione
%\chordsoff		& <<< DECOMMENTA se vuoi una strofa senza accordi

\[B]Il tuo sguardo \[A]puro sia luce per \[C#-]me
\[B]e la tua Pa\[A]rola sia voce per \[E]me.
\[A]Che io trovi il \[B]senso del mio andare
\[C#-]so\[B]lo in \[E]te, nel \[B]tuo fedele amare il mio per\[E]ché.

\endverse

%%%%% RITORNELLO
\beginchorus
\textnote{\textbf{Rit.}}

\[E]Ora \[B]vado \[A]sulla mia \[E]strada
\[F#-]con l'a\[G#-]more \[A]tuo che mi \[B]guida.
\[E]O Si\[B]gnore, o\[A]vunque io \[E]vada,
\[A]resta ac\[B]canto a \[E]me.
\[E]Io ti \[B]prego, \[A]stammi vi\[E]cino
\[F#-]ogni \[G#-]passo \[A]del mio cam\[B]mino,
\[E]ogni \[B]notte, \[A]ogni mat\[E]tino,
\[A]resta ac\[B]canto a \[E]me.

\endchorus

%%%%% STROFA
\beginverse		%Oppure \beginverse* se non si vuole il numero di fianco
%\memorize 		% <<< DECOMMENTA se si vuole utilizzarne la funzione
%\chordsoff		& <<< DECOMMENTA se vuoi una strofa senza accordi

^Fa' che chi mi ^guarda non veda che ^te.
^Fa' che chi mi a^scolta non senta che ^te,
^e chi pensa a ^me, fa' che nel cuore
^pen^si a ^te e ^trovi quell'amore \brk che hai dato a ^me.
\endverse

%%%%% RITORNELLO
\beginchorus
\textnote{\textbf{Rit.}}

\[E]Ora \[B]vado \[A]sulla mia \[E]strada
\[F#-]con l'a\[G#-]more \[A]tuo che mi \[B]guida.
\[E]O Si\[B]gnore, o\[A]vunque io \[E]vada,
\[A]resta ac\[B]canto a \[E]me.
\[E]Io ti \[B]prego, \[A]stammi vi\[E]cino
\[F#-]ogni \[G#-]passo \[A]del mio cam\[B]mino,
\[E]ogni \[B]notte, \[A]ogni mat\[E]tino,
\[A]resta ac\[B]canto a \[E]me.

%%%%%% EV. FINALE

\beginchorus %oppure \beginverse*
\vspace*{1.3\versesep}
%\textnote{Finale \textit{(rallentando)}} %<<< EV. INDICAZIONI

\[E]Ora \[B]vado \[A]sulla mia \[E]strada
\[A]resta ac\[B]canto a \[E]me.

\endchorus  %oppure \endverse




\endsong
%------------------------------------------------------------
%			FINE CANZONE
%------------------------------------------------------------



%-------------------------------------------------------------
%			INIZIO	CANZONE
%-------------------------------------------------------------


%titolo: 	Resta qui con noi
%autore: 	Gen Rosso
%tonalita: 	Re 



%%%%%% TITOLO E IMPOSTAZONI
\beginsong{Resta qui con noi}[by={Gen Rosso}] 	% <<< MODIFICA TITOLO E AUTORE
\transpose{0} 						% <<< TRASPOSIZIONE #TONI (0 nullo)
\momenti{Congedo}							% <<< INSERISCI MOMENTI	
% momenti vanno separati da ; e vanno scelti tra:
% Ingresso; Atto penitenziale; Acclamazione al Vangelo; Dopo il Vangelo; Offertorio; Comunione; Ringraziamento; Fine; Santi; Pasqua; Avvento; Natale; Quaresima; Canti Mariani; Battesimo; Prima Comunione; Cresima; Matrimonio; Meditazione;
\ifchorded
	%\textnote{Tonalità originale }	% <<< EV COMMENTI (tonalità originale/migliore)
\fi


%%%%%% INTRODUZIONE
\ifchorded
\vspace*{\versesep}
\musicnote{
\begin{minipage}{0.48\textwidth}
\textbf{Intro}
\hfill 
%( \eighthnote \, 80)   % <<  MODIFICA IL TEMPO
% Metronomo: \eighthnote (ottavo) \quarternote (quarto) \halfnote (due quarti)
\end{minipage}
} 	
\vspace*{-\versesep}
\beginverse*

\nolyrics

%---- Prima riga -----------------------------
\vspace*{-\versesep}
\[D] \[G] \[D] \[G] % \[*D] per indicare le pennate, \rep{2} le ripetizioni

%---- Ogni riga successiva -------------------
%\vspace*{-\versesep}
%\[G] \[C]  \[D]	

%---- Ev Indicazioni -------------------------			
%\textnote{\textit{(Oppure tutta la strofa)} }	

\endverse
\fi


%%%%% STROFA
\beginverse
\memorize
Le \[D]ombre si distendono, \[G]scende ormai la sera,
\[D]e s'allontanano dietro i \[E-7]monti
i riflessi di un \[B-]giorno che non \[B-7]finirà,
di un \[E7]giorno che ora \[*G]correrà \[*A]sempre, \[D]
Perché sap\[F#-7]piamo che \[G]una nuova \[E-7]vita
da qui è par\[D]tita e mai \[G]più si ferme\[A4]rà. \[A]
\endverse


%%%%% RITORNELLO
\beginchorus
\textnote{\textbf{Rit.}}
\[D]Resta qui con \[F#-]noi, il \[G]sole scende \[D]già,
\[E-7]resta qui con \[A]noi, Si\[G]gnore è \[A]sera or\[D]mai.
\[D]Resta qui con \[F#-7]noi, il \[G]sole scende \[D]già,
\[E-7]se tu sei fra \[A]noi, la \[G]notte \[A]non ver\[D]rà.
\endchorus




%%%%% STROFA
\beginverse
%\chordsoff
S'al^larga verso il mare ^il tuo cerchio d'onda,
^che il vento spingerà fino a ^quando
giungerà ai con^fini di ogni ^cuore,
alle ^porte dell'a^more ^vero. ^
Come una ^fiamma che ^dove passa ^brucia
così il tuo a^more tutto il ^mondo invade^rà. ^
\endverse



%%%%% STROFA
\beginverse
%\chordsoff
Da^vanti a noi l'umanità ^lotta, soffre e spera
^come una terra che nell'ar^sura
chiede l'acqua da un ^cielo senza ^nuvole,
ma che ^sempre le può ^dare ^vita. ^
Con te sa^remo sor^gente d'acqua ^pura,
con te fra ^noi il de^serto fiori^rà. ^
\endverse


\endsong
%------------------------------------------------------------
%			FINE CANZONE
%------------------------------------------------------------




%-------------------------------------------------------------
%			INIZIO	CANZONE
%-------------------------------------------------------------


%titolo: 	Resurrezione
%autore: 	Gen Rosso
%tonalita: 	Re e Do 



%%%%%% TITOLO E IMPOSTAZONI
\beginsong{Resurrezione}[by={Gen Rosso}] 	% <<< MODIFICA TITOLO E AUTORE
\transpose{-2} 						% <<< TRASPOSIZIONE #TONI (0 nullo)
\momenti{Pasqua}							% <<< INSERISCI MOMENTI	
% momenti vanno separati da ; e vanno scelti tra:
% Ingresso; Atto penitenziale; Acclamazione al Vangelo; Dopo il Vangelo; Offertorio; Comunione; Ringraziamento; Fine; Santi; Pasqua; Avvento; Natale; Quaresima; Canti Mariani; Battesimo; Prima Comunione; Cresima; Matrimonio; Meditazione;
\ifchorded
	\textnote{$\bigstar$ Tonalità migliore per le bambine }	% <<< EV COMMENTI (tonalità originale/migliore)
\fi


%%%%%% INTRODUZIONE
\ifchorded
\vspace*{\versesep}
\musicnote{
\begin{minipage}{0.48\textwidth}
\textbf{Intro}
\hfill 
%( \eighthnote \, 80)   % <<  MODIFICA IL TEMPO
% Metronomo: \eighthnote (ottavo) \quarternote (quarto) \halfnote (due quarti)
\end{minipage}
} 	
\vspace*{-\versesep}
\beginverse*
\nolyrics

%---- Prima riga -----------------------------
\vspace*{-\versesep}
\[D] \[G] \[D]	 \[G] % \[*D] per indicare le pennate, \rep{2} le ripetizioni

%---- Ogni riga successiva -------------------
%\vspace*{-\versesep}
%\[G] \[C]  \[D]	

%---- Ev Indicazioni -------------------------			
%\textnote{\textit{(Oppure tutta la strofa)} }	

\endverse
\fi




%%%%% STROFA
\beginverse		%Oppure \beginverse* se non si vuole il numero di fianco
\memorize 		% <<< DECOMMENTA se si vuole utilizzarne la funzione
%\chordsoff		& <<< DECOMMENTA se vuoi una strofa senza accordi

Che \[D]gioia ci hai \[G]dato, Si\[D]gnore del \[G]cielo
Si\[D]gnore del \[G]grande uni\[A]verso!
Che \[D]gioia ci hai \[G]dato, ves\[D]tito di \[G]luce
ves\[D]tito di \[A]gloria infi\[B-]ni\[G]ta,
ves\[D]tito di \[A]gloria infi\[G]n\[A*]i\[D]ta!

\endverse




%%%%%% EV. INTERMEZZO
\beginverse*
\vspace*{1.3\versesep}
{
	\nolyrics
	\textnote{Intermezzo strumentale}
	
	\ifchorded

	%---- Prima riga -----------------------------
	\vspace*{-\versesep}
	 \[G]  \[D]	 \[G] 


	\fi
	%---- Ev Indicazioni -------------------------			
	%\textnote{\textit{(ripetizione della strofa)}} 
	 
}
\vspace*{\versesep}
\endverse


%%%%% STROFA
\beginverse		%Oppure \beginverse* se non si vuole il numero di fianco
%\memorize 		% <<< DECOMMENTA se si vuole utilizzarne la funzione
%\chordsoff		& <<< DECOMMENTA se vuoi una strofa senza accordi

Ve^derti ri^sorto, ve^derti Si^gnore,
il ^cuore sta ^per impaz^zire!
Tu ^sei ritor^nato, Tu ^sei qui tra ^noi
e a^desso Ti a^vremo per ^sem^pre,
e a^desso Ti a^vremo per ^se^m^pre.

\endverse




%%%%%% EV. INTERMEZZO
\beginverse*
\vspace*{1.3\versesep}
{
	\nolyrics
	\textnote{Intermezzo strumentale}
	
	\ifchorded

	%---- Prima riga -----------------------------
	\vspace*{-\versesep}
	\[G]  \[D]	 \[G] 


	\fi
	%---- Ev Indicazioni -------------------------			
	%\textnote{\textit{(ripetizione della strofa)}} 
	 
}
\vspace*{\versesep}
\endverse



%%%%% STROFA
\beginverse		%Oppure \beginverse* se non si vuole il numero di fianco
%\memorize 		% <<< DECOMMENTA se si vuole utilizzarne la funzione
%\chordsoff		& <<< DECOMMENTA se vuoi una strofa senza accordi

^ Chi cercate, ^donne, quag^giù,
chi cercate, ^donne, quag^giù?
Quello che era ^morto non è ^qui:
è ri^sorto, sì! come a^veva detto ^anche a voi,
^voi gridate a ^tutti che^
^è risorto ^Lui,
a ^tutti che^
^è risor^to ^Lui!

\endverse




%%%%%% EV. INTERMEZZO
\beginverse*
\vspace*{1.3\versesep}
{
	\nolyrics
	\textnote{Intermezzo strumentale}
	
	\ifchorded

	%---- Prima riga -----------------------------
	\vspace*{-\versesep}
	 \[G]  \[D]	 \[G] 


	\fi
	%---- Ev Indicazioni -------------------------			
	%\textnote{\textit{(ripetizione della strofa)}} 
	 
}
\vspace*{\versesep}
\endverse




%%%%% STROFA
\beginverse		%Oppure \beginverse* se non si vuole il numero di fianco
%\memorize 		% <<< DECOMMENTA se si vuole utilizzarne la funzione
%\chordsoff		& <<< DECOMMENTA se vuoi una strofa senza accordi

^ Tu hai vinto il ^mondo, Ge^sù,
Tu hai vinto il ^mondo, Ge^sù,
liberiamo ^la felici^tà!
E la ^morte, no, non e^siste più, l’hai ^vinta Tu
e ^hai salvato ^tutti noi, ^
^uomini con ^Te,
^tutti noi, ^
^uomini ^con ^Te.

\endverse



%%%%%% EV. FINALE

\beginverse*
\vspace*{1.3\versesep}
\textnote{\textbf{Finale}} %<<< EV. INDICAZIONI

\[G]Uomini con te, uomini con te - e
\echo{\[D]Che gioia ci hai \[G]dato 
ti a\[D]vremo per \[G]sem\[D]pre - \[D]e!} \[D*]

\endverse




\endsong
%------------------------------------------------------------
%			FINE CANZONE
%------------------------------------------------------------

%++++++++++++++++++++++++++++++++++++++++++++++++++++++++++++
%			CANZONE TRASPOSTA
%++++++++++++++++++++++++++++++++++++++++++++++++++++++++++++
\ifchorded
%decremento contatore per avere stesso numero
\addtocounter{songnum}{-1} 
\beginsong{Resurrezione}[by={Gen Rosso}] 	% <<< COPIA TITOLO E AUTORE
\transpose{0} 						% <<< TRASPOSIZIONE #TONI + - (0 nullo)
%\preferflats  %SE VOGLIO FORZARE i bemolle come alterazioni
%\prefersharps %SE VOGLIO FORZARE i # come alterazioni
\ifchorded
	\textnote{$\lozenge$ Tonalità originale}	% <<< EV COMMENTI (tonalità originale/migliore)
\fi


%%%%%% INTRODUZIONE
\ifchorded
\vspace*{\versesep}
\musicnote{
\begin{minipage}{0.48\textwidth}
\textbf{Intro}
\hfill 
%( \eighthnote \, 80)   % <<  MODIFICA IL TEMPO
% Metronomo: \eighthnote (ottavo) \quarternote (quarto) \halfnote (due quarti)
\end{minipage}
} 	
\vspace*{-\versesep}
\beginverse*

\nolyrics

%---- Prima riga -----------------------------
\vspace*{-\versesep}
\[D] \[G] \[D]	 \[G] % \[*D] per indicare le pennate, \rep{2} le ripetizioni

%---- Ogni riga successiva -------------------
%\vspace*{-\versesep}
%\[G] \[C]  \[D]	

%---- Ev Indicazioni -------------------------			
%\textnote{\textit{(Oppure tutta la strofa)} }	

\endverse
\fi




%%%%% STROFA
\beginverse		%Oppure \beginverse* se non si vuole il numero di fianco
\memorize 		% <<< DECOMMENTA se si vuole utilizzarne la funzione
%\chordsoff		& <<< DECOMMENTA se vuoi una strofa senza accordi

Che \[D]gioia ci hai \[G]dato, Si\[D]gnore del \[G]cielo
Si\[D]gnore del \[G]grande uni\[A]verso!
Che \[D]gioia ci hai \[G]dato, ves\[D]tito di \[G]luce
ves\[D]tito di \[A]gloria infi\[B-]ni\[G]ta,
ves\[D]tito di \[A]gloria infi\[G]n\[A*]i\[D]ta!

\endverse




%%%%%% EV. INTERMEZZO
\beginverse*
\vspace*{1.3\versesep}
{
	\nolyrics
	\textnote{Intermezzo strumentale}
	
	\ifchorded

	%---- Prima riga -----------------------------
	\vspace*{-\versesep}
	 \[G]  \[D]	 \[G] 


	\fi
	%---- Ev Indicazioni -------------------------			
	%\textnote{\textit{(ripetizione della strofa)}} 
	 
}
\vspace*{\versesep}
\endverse


%%%%% STROFA
\beginverse		%Oppure \beginverse* se non si vuole il numero di fianco
%\memorize 		% <<< DECOMMENTA se si vuole utilizzarne la funzione
%\chordsoff		& <<< DECOMMENTA se vuoi una strofa senza accordi

Ve^derti ri^sorto, ve^derti Si^gnore,
il ^cuore sta ^per impaz^zire!
Tu ^sei ritor^nato, Tu ^sei qui tra ^noi
e a^desso Ti a^vremo per ^sem^pre,
e a^desso Ti a^vremo per ^se^m^pre.

\endverse




%%%%%% EV. INTERMEZZO
\beginverse*
\vspace*{1.3\versesep}
{
	\nolyrics
	\textnote{Intermezzo strumentale}
	
	\ifchorded

	%---- Prima riga -----------------------------
	\vspace*{-\versesep}
	\[G]  \[D]	 \[G] 


	\fi
	%---- Ev Indicazioni -------------------------			
	%\textnote{\textit{(ripetizione della strofa)}} 
	 
}
\vspace*{\versesep}
\endverse



%%%%% STROFA
\beginverse		%Oppure \beginverse* se non si vuole il numero di fianco
%\memorize 		% <<< DECOMMENTA se si vuole utilizzarne la funzione
%\chordsoff		& <<< DECOMMENTA se vuoi una strofa senza accordi

^ Chi cercate, ^donne, quag^giù,
chi cercate, ^donne, quag^giù?
Quello che era ^morto non è ^qui:
è ri^sorto, sì! come a^veva detto ^anche a voi,
^voi gridate a ^tutti che^
^è risorto ^Lui,
a ^tutti che^
^è risor^to ^Lui!

\endverse




%%%%%% EV. INTERMEZZO
\beginverse*
\vspace*{1.3\versesep}
{
	\nolyrics
	\textnote{Intermezzo strumentale}
	
	\ifchorded

	%---- Prima riga -----------------------------
	\vspace*{-\versesep}
	 \[G]  \[D]	 \[G] 


	\fi
	%---- Ev Indicazioni -------------------------			
	%\textnote{\textit{(ripetizione della strofa)}} 
	 
}
\vspace*{\versesep}
\endverse




%%%%% STROFA
\beginverse		%Oppure \beginverse* se non si vuole il numero di fianco
%\memorize 		% <<< DECOMMENTA se si vuole utilizzarne la funzione
%\chordsoff		& <<< DECOMMENTA se vuoi una strofa senza accordi

^ Tu hai vinto il ^mondo, Ge^sù,
Tu hai vinto il ^mondo, Ge^sù,
liberiamo ^la felici^tà!
E la ^morte, no, non e^siste più, l’hai ^vinta Tu
e ^hai salvato ^tutti noi, ^
^uomini con ^Te,
^tutti noi, ^
^uomini ^con ^Te.

\endverse



%%%%%% EV. FINALE

\beginverse*
\vspace*{1.3\versesep}
\textnote{\textbf{Finale}} %<<< EV. INDICAZIONI

\[G]Uomini con te, uomini con te - e
\echo{\[D]Che gioia ci hai \[G]dato 
ti a\[D]vremo per \[G]sem\[D]pre - \[D]e!} \[D*]

\endverse




\endsong

\fi
%++++++++++++++++++++++++++++++++++++++++++++++++++++++++++++
%			FINE CANZONE TRASPOSTA
%++++++++++++++++++++++++++++++++++++++++++++++++++++++++++++


%SSS
%-------------------------------------------------------------
%			INIZIO	CANZONE
%-------------------------------------------------------------


%titolo: 	Saldo è il mio cuore
%autore: 	M. Frisina
%tonalita:  Re 
%youtube: https://www.youtube.com/watch?v=cG_-iab1xcI&feature=youtu.be



%%%%%% TITOLO E IMPOSTAZONI
\beginsong{Saldo è il mio cuore}[by={M. Frisina}] 	% <<< MODIFICA TITOLO E AUTORE
\transpose{0} 						% <<< TRASPOSIZIONE #TONI (0 nullo)
%\preferflats  %SE VOGLIO FORZARE i bemolle come alterazioni
%\prefersharps %SE VOGLIO FORZARE i # come alterazioni
\momenti{}							% <<< INSERISCI MOMENTI	
% momenti vanno separati da ; e vanno scelti tra:
% Ingresso; Atto penitenziale; Acclamazione al Vangelo; Dopo il Vangelo; Offertorio; Comunione; Ringraziamento; Fine; Santi; Pasqua; Avvento; Natale; Quaresima; Canti Mariani; Battesimo; Prima Comunione; Cresima; Matrimonio; Meditazione; Spezzare del pane;
\ifchorded
	%\textnote{Tonalità migliore }	% <<< EV COMMENTI (tonalità originale/migliore)
\fi


%%%%%% INTRODUZIONE
\ifchorded
\vspace*{\versesep}
\musicnote{
\begin{minipage}{0.48\textwidth}
\textbf{Intro}
\hfill 
%( \eighthnote \, 80)   % <<  MODIFICA IL TEMPO
% Metronomo: \eighthnote (ottavo) \quarternote (quarto) \halfnote (due quarti)
\end{minipage}
} 	
\vspace*{-\versesep}
\beginverse*

\nolyrics

%---- Prima riga -----------------------------
\vspace*{-\versesep}
\[D] \[G] \quad \[D]\[A]\[D]	 % \[*D] per indicare le pennate, \rep{2} le ripetizioni

%---- Ogni riga successiva -------------------
%\vspace*{-\versesep}
%\[G] \[C]  \[D]	

%---- Ev Indicazioni -------------------------			
%\textnote{\textit{(Oppure tutta la strofa)} }	

\endverse
\fi




%%%%% STROFA
\beginverse		%Oppure \beginverse* se non si vuole il numero di fianco
\memorize 		% <<< DECOMMENTA se si vuole utilizzarne la funzione
%\chordsoff		% <<< DECOMMENTA se vuoi una strofa senza accordi


\[D]Saldo è il mio \[G]cuore, Dio \[D]\[A]mi\[D]o.
A \[D]te cante\[G]rà l'anima \[A]\[G]mi\[A]a.
De\[D]statevi \[G]arpa e \[D]\[A]ce\[B-]tra,
\[G]vo\[A]glio sve\[B-]glia\[E-]re l'au\[D]\[A]ro\[D]ra. \quad \[A]

\endverse
\beginverse	
\textbf{A ^te la mia ^lode tra le ^^gen^ti,
per^chè fino ai ^cieli \brk è il tuo ^^amo^re.
^Sorgi ed in^nalzati, o ^^Di^o,
^splen^da sul ^mondo ^la tua ^^glo^ria.\quad ^
}
\endverse
\beginverse

Con ^te noi fa^remo cose ^^gran^di.
Con ^te noi con^vertiremo il ^^mon^do.
Tu ^sei nostra ^luce e con^^for^to,
^for^za, ri^fugio, ^o Si^^gno^re. \quad ^

\endverse
\beginverse
\textbf{^Per te noi an^dremo per il ^^mon^do,
^inni cante^remo alla tua ^^glo^ria.
^Donaci la ^grazia, Si^^gno^re,
^an^nunce^remo ^il tuo a^^mo^re. \quad \[D*]
}
\endverse





\endsong
%------------------------------------------------------------
%			FINE CANZONE
%------------------------------------------------------------



%-------------------------------------------------------------
%			INIZIO	CANZONE
%-------------------------------------------------------------


%titolo: 	Salmo 8
%autore: 	Meregalli
%tonalita: 	Mi-



%%%%%% TITOLO E IMPOSTAZONI
\beginsong{Salmo 8}[by={G. Meregalli}]	% <<< MODIFICA TITOLO E AUTORE
\transpose{-2} 						% <<< TRASPOSIZIONE #TONI (0 nullo)
\preferflats 
\momenti{Ringraziamento; Salmi}							% <<< INSERISCI MOMENTI	
% momenti vanno separati da ; e vanno scelti tra:
% Ingresso; Atto penitenziale; Acclamazione al Vangelo; Dopo il Vangelo; Offertorio; Comunione; Ringraziamento; Fine; Santi; Pasqua; Avvento; Natale; Quaresima; Canti Mariani; Battesimo; Prima Comunione; Cresima; Matrimonio; Meditazione; Spezzare del pane;
\ifchorded
	\textnote{$\bigstar$ Tonalità migliore }	% <<< EV COMMENTI (tonalità originale/migliore)
\fi


%%%%%% INTRODUZIONE
\ifchorded
\vspace*{\versesep}
\musicnote{
\begin{minipage}{0.48\textwidth}
\textbf{Intro}
\hfill 
( \eighthnote \, 76)   % <<  MODIFICA IL TEMPO
% Metronomo: \eighthnote (ottavo) \quarternote (quarto) \halfnote (due quarti)
\end{minipage}
} 	
\vspace*{-\versesep}
\beginverse*
\nolyrics

%---- Prima riga -----------------------------
\vspace*{-\versesep}
\[E-] \[E-/C] \[A-] \[E-]	 % \[*D] per indicare le pennate, \rep{2} le ripetizioni

%---- Ogni riga successiva -------------------
%\vspace*{-\versesep}
%\[G] \[C]  \[D]	

%---- Ev Indicazioni -------------------------			
%\textnote{\textit{(Oppure tutta la strofa)} }	

\endverse
\fi




%%%%% RITORNELLO
\beginchorus
\textnote{\textbf{Rit.}}
\[E-]Se guardo il c\[E-]ielo, \brk\[A-]  la luna e le \[E-]stelle,
opere che \[C]Tu  \brk con le \[D]dita hai model\[B-]lato,
\[B7] che cosa \[C]è,  \brk \[D]perché Te ne \[E-7]curi,
\[C7+] che cosa \[A-]è, \brk  per\[D7]ché te ne ri\[G*]cor\[B7*]di
\[E-]l’uo\[9]mo, \[C]l’uo\[A-7]mo,  \brk \[E-]l’uomo?
\endchorus




%%%%% STROFA
\beginverse*		%Oppure \beginverse* se non si vuole il numero di fianco
%\memorize 		% <<< DECOMMENTA se si vuole utilizzarne la funzione
%\chordsoff
\[(E-)]Eppure l’hai \[C]fatto \brk\[D] poco meno degli \[E-]angeli,
di \[(E-)]gloria e di o\[C]nore \brk\[D]  lo hai coro\[E-]nato
gli hai \[C]dato po\[D]tere \brk  sulle \[G]opere delle tue \[B7]mani,
su \[C]tutte le \[D]cose \brk  che \[G]tu avevi cre\[B7]ato:
gli u\[C]ccelli del \[E-]cielo, \brk  i \[C]pesci del \[E-]mare,
le \[C]greggi e gli ar\[E-]menti, \brk  gli ani\[A-*]mali \[B7*]della cam\[E-]pagna.
\endverse






\endsong
%------------------------------------------------------------
%			FINE CANZONE
%------------------------------------------------------------

%++++++++++++++++++++++++++++++++++++++++++++++++++++++++++++
%			CANZONE TRASPOSTA
%++++++++++++++++++++++++++++++++++++++++++++++++++++++++++++
\ifchorded
%decremento contatore per avere stesso numero
\addtocounter{songnum}{-1} 
\beginsong{Salmo 8}[by={Meregalli}]	% <<< COPIA TITOLO E AUTORE
\transpose{0} 						% <<< TRASPOSIZIONE #TONI + - (0 nullo)
%\preferflats  %SE VOGLIO FORZARE i bemolle come alterazioni
%\prefersharps %SE VOGLIO FORZARE i # come alterazioni
\ifchorded
	\textnote{$\lozenge$ Tonalità originale}	% <<< EV COMMENTI (tonalità originale/migliore)
\fi


%%%%%% INTRODUZIONE
\ifchorded
\vspace*{\versesep}
\musicnote{
\begin{minipage}{0.48\textwidth}
\textbf{Intro}
\hfill 
( \eighthnote \, 76)   % <<  MODIFICA IL TEMPO
% Metronomo: \eighthnote (ottavo) \quarternote (quarto) \halfnote (due quarti)
\end{minipage}
} 	
\vspace*{-\versesep}
\beginverse*

\nolyrics

%---- Prima riga -----------------------------
\vspace*{-\versesep}
\[E-] \[E-/C] \[A-] \[E-]	 % \[*D] per indicare le pennate, \rep{2} le ripetizioni

%---- Ogni riga successiva -------------------
%\vspace*{-\versesep}
%\[G] \[C]  \[D]	

%---- Ev Indicazioni -------------------------			
%\textnote{\textit{(Oppure tutta la strofa)} }	

\endverse
\fi




%%%%% RITORNELLO
\beginchorus
\textnote{\textbf{Rit.}}
\[E-]Se guardo il c\[E-]ielo, \brk\[A-]  la luna e le \[E-]stelle,
opere che \[C]Tu  \brk con le \[D]dita hai model\[B-]lato,
\[B7] che cosa \[C]è,  \brk \[D]perché Te ne \[E-7]curi,
\[C7+] che cosa \[A-]è, \brk  per\[D7]ché te ne ri\[G*]cor\[B7*]di
\[E-]l’uo\[9]mo, \[C]l’uo\[A-7]mo,  \brk \[E-]l’uomo?
\endchorus




%%%%% STROFA
\beginverse		%Oppure \beginverse* se non si vuole il numero di fianco
%\memorize 		% <<< DECOMMENTA se si vuole utilizzarne la funzione
%\chordsoff
\[(E-)]Eppure l’hai \[C]fatto \brk\[D] poco meno degli \[E-]angeli,
di \[(E-)]gloria e di o\[C]nore \brk\[D]  lo hai coro\[E-]nato
gli hai \[C]dato po\[D]tere \brk  sulle \[G]opere delle tue \[B7]mani,
su \[C]tutte le \[D]cose \brk  che \[G]tu avevi cre\[B7]ato:
gli u\[C]ccelli del \[E-]cielo, \brk  i \[C]pesci del \[E-]mare,
le \[C]greggi e gli ar\[E-]menti, \brk  gli ani\[A-*]mali \[B7*]della cam\[E-]pagna.
\endverse



\endsong
\fi
%++++++++++++++++++++++++++++++++++++++++++++++++++++++++++++
%			FINE CANZONE TRASPOSTA
%++++++++++++++++++++++++++++++++++++++++++++++++++++++++++++

%-------------------------------------------------------------
%			INIZIO	CANZONE
%-------------------------------------------------------------


%titolo: 	Salve Regina
%autore: 	Gen\ Verde
%tonalita: 	Sol 



%%%%%% TITOLO E IMPOSTAZONI
\beginsong{Salve Regina}[by={Gen\ Verde}] 	% <<< MODIFICA TITOLO E AUTORE
\transpose{0} 						% <<< TRASPOSIZIONE #TONI (0 nullo)
\momenti{Canti Mariani}							% <<< INSERISCI MOMENTI	
% momenti vanno separati da ; e vanno scelti tra:
% Ingresso; Atto penitenziale; Acclamazione al Vangelo; Dopo il Vangelo; Offertorio; Comunione; Ringraziamento; Fine; Santi; Pasqua; Avvento; Natale; Quaresima; Canti Mariani; Battesimo; Prima Comunione; Cresima; Matrimonio; Meditazione; Spezzare del pane;
\ifchorded
	%\textnote{Tonalità migliore }	% <<< EV COMMENTI (tonalità originale/migliore)
\fi


%%%%%% INTRODUZIONE
\ifchorded
\vspace*{\versesep}
\musicnote{
\begin{minipage}{0.48\textwidth}
\textbf{Intro}
\hfill 
%( \eighthnote \, 80)   % <<  MODIFICA IL TEMPO
% Metronomo: \eighthnote (ottavo) \quarternote (quarto) \halfnote (due quarti)
\end{minipage}
} 	
\vspace*{-\versesep}
\beginverse*
\nolyrics

%---- Prima riga -----------------------------
\vspace*{-\versesep}
\[G] \[D] \[G] \[D]	 % \[*D] per indicare le pennate, \rep{2} le ripetizioni

%---- Ogni riga successiva -------------------
%\vspace*{-\versesep}
%\[G] \[C]  \[D]	

%---- Ev Indicazioni -------------------------			
%\textnote{\textit{(Oppure tutta la strofa)} }	

\endverse
\fi

\beginverse*
\[G]Salve Re\[D]gina, \[C]Madre di miseri\[G]cordia
vita dol\[D]cezza speranza \[C]nostra \[G]salve
Salve Re\[D]gina. \[D7] 
\[G]Salve Re\[D]gina, \[C]Madre di miseri\[G]cordia
vita dol\[D]cezza speranza \[C]nostra \[G]salve
Salve Re\[D]gina. \[D7] 
\endverse


\beginverse*
\[G] A te ricor\[A-]riamo \[D]esuli figli di \[G]Eva
A te sospi\[B-]riamo
pian\[C]genti in questa valle di \[D]lacrime.
\[G] Avvocata \[A-]nostra
\[D]volgi a noi gli occhi \[G]tuoi
mostraci dopo questo e\[B-]silio
il \[C]frutto del tuo seno Ge\[D]sù.
\endverse


\beginverse*
\[G]Salve Re\[D]gina, \[C]Madre di miseri\[G]cordia
o Clemente, o \[D]Pia, o dolce \[C]Vergine Ma\[G]ria.
Salve Re\[D]gina. \[D7]
\[G]Salve Re\[D]gina. \[C]Sal\[G]ve. \[C]Sal\[G]ve.
\endverse


\endsong
%------------------------------------------------------------
%			FINE CANZONE
%------------------------------------------------------------




%-------------------------------------------------------------
%			INIZIO	CANZONE
%-------------------------------------------------------------


%titolo: 	Scatenate la gioia!
%autore: 	L. D'Amico
%tonalita: 	Re



%%%%%% TITOLO E IMPOSTAZONI
\beginsong{Scatenate la gioia}[by={L. D'Amico}] 	% <<< MODIFICA TITOLO E AUTORE
\transpose{-2} 						% <<< TRASPOSIZIONE #TONI (0 nullo)
%\preferflats  %SE VOGLIO FORZARE i bemolle come alterazioni
%\prefersharps %SE VOGLIO FORZARE i # come alterazioni
\momenti{Fine; Matrimonio; Ringraziamento}							% <<< INSERISCI MOMENTI	
% momenti vanno separati da ; e vanno scelti tra:
% Ingresso; Atto penitenziale; Acclamazione al Vangelo; Dopo il Vangelo; Offertorio; Comunione; Ringraziamento; Fine; Santi; Pasqua; Avvento; Natale; Quaresima; Canti Mariani; Battesimo; Prima Comunione; Cresima; Matrimonio; Meditazione; Spezzare del pane;
\ifchorded
	%\textnote{Tonalità migliore }	% <<< EV COMMENTI (tonalità originale/migliore)
\fi


%%%%%% INTRODUZIONE
\ifchorded
\vspace*{\versesep}
\textnote{Intro: \qquad \qquad  }%(\eighthnote 116) % <<  MODIFICA IL TEMPO
% Metronomo: \eighthnote (ottavo) \quarternote (quarto) \halfnote (due quarti)
\vspace*{-\versesep}
\beginverse*

\nolyrics

%---- Prima riga -----------------------------
\vspace*{-\versesep}
\[D] \[G] \[D] \[G]	 % \[*D] per indicare le pennate, \rep{2} le ripetizioni

%---- Ogni riga successiva -------------------
\vspace*{-\versesep}
\[D] \[G]  \[D]  \[A] \[A]	

%---- Ev Indicazioni -------------------------			
%\textnote{\textit{(Oppure tutta la strofa)} }	

\endverse
\fi




%%%%% STROFA
\beginverse		%Oppure \beginverse* se non si vuole il numero di fianco
\memorize 		% <<< DECOMMENTA se si vuole utilizzarne la funzione
%\chordsoff		% <<< DECOMMENTA se vuoi una strofa senza accordi

\[D]Uscite dalle case voi, \[G]che siete chiusi dentro
\[D]Venite qui tra noi, \[G]qualcosa sta accadendo.
\[D]Qui non piove, \[G]qui c’è solo il sole
\[D]Fate in fretta, man\[A4]cate solo \[A]voi. 


\endverse

%%%%% STROFA
\beginverse*		%Oppure \beginverse* se non si vuole il numero di fianco
%\memorize 		% <<< DECOMMENTA se si vuole utilizzarne la funzione
%\chordsoff		% <<< DECOMMENTA se vuoi una strofa senza accordi

^Muovi i piedi tu, ^tu che stai ascoltando,
^e le mani alza tu, ^per tenere il tempo.
^Segui il ritmo di ^questa canzone,
t^utti pronti, poss^iamo dare la via. 


\endverse



%%%%% RITORNELLO
\beginchorus
\textnote{\textbf{Rit.}}

\[D]Scate\[A]nate la \[B-]gioia, \[G]oggi \[D]qui si fa \[A]festa!
\[B-]Dai, cantate con \[G]noi. 
\[D]Qui la \[A]festa siamo \[G]noi!  \rep{2}


\endchorus




%%%%%% EV. INTERMEZZO
\beginverse*
\vspace*{1.3\versesep}
{
	\nolyrics
	\textnote{Intermezzo strumentale}
	
	\ifchorded

	%---- Prima riga -----------------------------
	\vspace*{-\versesep}
	\[D] \[G]  \[D] \[G]	 

	%---- Ogni riga successiva -------------------
	%\vspace*{-\versesep}
	%\[G] \[C]  \[D]	


	\fi
	%---- Ev Indicazioni -------------------------			
	%\textnote{\textit{(ripetizione della strofa)}} 
	 
}
\vspace*{\versesep}
\endverse



%%%%% STROFA
\beginverse		%Oppure \beginverse* se non si vuole il numero di fianco
%\memorize 		% <<< DECOMMENTA se si vuole utilizzarne la funzione
%\chordsoff		% <<< DECOMMENTA se vuoi una strofa senza accordi

^Non si sente bene qui, ^qualcuno canta piano,
^fatti trascinare tu, che ^non puoi farne a meno.
^Se qui non canti, ^togli il tuo colore
^all’arcobaleno di ^questa canzone. ^

\endverse




%%%%% STROFA
\beginverse		%Oppure \beginverse* se non si vuole il numero di fianco
%\memorize 		% <<< DECOMMENTA se si vuole utilizzarne la funzione
%\chordsoff		% <<< DECOMMENTA se vuoi una strofa senza accordi

^Siamo in tanti qui, ^a cantare forte
^che la gioia entrerà, ^basta aprire le porte.
^Qui nell’aria si ^sente un buon profumo,
^se ci stai, non ^manca più nessuno! ^ 

\endverse






\endsong
%------------------------------------------------------------
%			FINE CANZONE
%------------------------------------------------------------


 
%-------------------------------------------------------------
%			INIZIO	CANZONE
%-------------------------------------------------------------


%titolo: 	Scelti per ri-creare
%autore: 	M.A. Chiodaroli
%tonalita: 	Do



%%%%%% TITOLO E IMPOSTAZONI
\beginsong{Scelti per ri-creare}[by={M.A. Chiodaroli}] 	% <<< MODIFICA TITOLO E AUTORE
\transpose{0} 						% <<< TRASPOSIZIONE #TONI (0 nullo)
\momenti{Meditazione; Ringraziamento; Comunione; Cresima}			% <<< INSERISCI MOMENTI	
% momenti vanno separati da ; e vanno scelti tra:
% Ingresso; Atto penitenziale; Acclamazione al Vangelo; Dopo il Vangelo; Offertorio; Comunione; Ringraziamento; Fine; Santi; Pasqua; Avvento; Natale; Quaresima; Canti Mariani; Battesimo; Prima Comunione; Cresima; Matrimonio; Meditazione; Spezzare del pane;
\ifchorded
	%\textnote{Tonalità migliore}	% <<< EV COMMENTI (tonalità originale/migliore)
\fi


%%%%%% INTRODUZIONE
\ifchorded
\vspace*{\versesep}
\musicnote{
\begin{minipage}{0.48\textwidth}
\textbf{Intro}
\hfill 
%( \eighthnote \, 80)   % <<  MODIFICA IL TEMPO
% Metronomo: \eighthnote (ottavo) \quarternote (quarto) \halfnote (due quarti)
\end{minipage}
} 	
\vspace*{-\versesep}
\beginverse*

\nolyrics

%---- Prima riga -----------------------------
\vspace*{-\versesep}
\[C] \[A-] \[F]	 \[G] \rep{2} % \[*D] per indicare le pennate, \rep{2} le ripetizioni

%---- Ogni riga successiva -------------------
%\vspace*{-\versesep}
%\[G] \[C]  \[D]	

%---- Ev Indicazioni -------------------------			
%\textnote{\textit{(Oppure tutta la strofa)} }	

\endverse
\fi




%%%%% STROFA
\beginverse		%Oppure \beginverse* se non si vuole il numero di fianco
\memorize 		% <<< DECOMMENTA se si vuole utilizzarne la funzione
%\chordsoff		% <<< DECOMMENTA se vuoi una strofa senza accordi

Ho cer\[C]cato di impa\[G]rare \brk la bel\[F]lezza dell’a\[G]gire
Ho ca\[C]pito che ci v\[G]oglion \brk molti \[F]mesi giorni e \[G]ore
Non ci vu\[C]ole più la \[G]fretta  \brk di chi \[F]corre nel suo \[G]tempo 
devo at\[C]tendere pa\[G]ziente, \brk contin\[F]uando a costr\[G]uire.

\endverse




%%%%% RITORNELLO
\beginchorus
%\textnote{\textbf{Rit.}}

\[C]Tu Sign\[A-]ore hai scelto \[F]me come att\[G]ore,
non per \[C]finge\[A-]re ma per \[F]ri-cre\[G]are
\[C]Tu hai \[A-]scelto noi per \[F]ricostru\[G]ire
con pa\[C]zienza \[A-]la bellezza \[F]che c’è per\[G]chè…
Ci hai scelti \[F]per ri-cre\[G]are con \[C]Te!


\endchorus



%%%%%% EV. INTERMEZZO
\beginverse*
\vspace*{1.3\versesep}
{
	\nolyrics
	\textnote{Intermezzo strumentale}
	
	\ifchorded

	%---- Prima riga -----------------------------
	\vspace*{-\versesep}
	\[C] \[A-] \[F]	 \[G] \rep{2} 


	\fi
	%---- Ev Indicazioni -------------------------			
	%\textnote{\textit{(ripetizione della strofa)}} 
	 
}
\vspace*{\versesep}
\endverse



%%%%% STROFA
\beginverse		%Oppure \beginverse* se non si vuole il numero di fianco
%\memorize 		% <<< DECOMMENTA se si vuole utilizzarne la funzione
%\chordsoff		% <<< DECOMMENTA se vuoi una strofa senza accordi

Ogni ^volta che poi ^provo  \brk a costru^ire con le ^mani
Scopro ^che non c’è crea^zione \brk che non ^porti scritta in ^sé
La pass^ione e poi la ^cura  \brk di chi ^non si è mai stan^cato
E ha pro^vato a model^lare \brk tutto ^ciò che gli hai do^nato



\endverse


%%%%% RITORNELLO
\beginchorus
%\textnote{\textbf{Rit.}}

\[C]Tu Sign\[A-]ore hai scelto \[F]me come att\[G]ore,
non per \[C]finge\[A-]re ma per \[F]ri-cre\[G]are
\[C]Tu hai \[A-]scelto noi per \[F]ricostru\[G]ire
con pa\[C]zienza \[A-]la bellezza \[F]che c’è per\[G]chè…
Ci hai scelti \[F]per ri-cre\[G]are con \[C]Te!


\endchorus


%%%%% STROFA
\beginverse*		%Oppure \beginverse* se non si vuole il numero di fianco
%\memorize 		% <<< DECOMMENTA se si vuole utilizzarne la funzione
%\chordsoff		% <<< DECOMMENTA se vuoi una strofa senza accordi
\vspace*{1.3\versesep}
\textnote{\textbf{Bridge}}
Ho apprez\[F]zato quanto è \[C]bello \brk lavo\[G]rare fianco a f\[A-]ianco 
con chi \[F]sa come il mio \[C]gesto \brk con il \[G]proprio comple\[A-]tare,
\[F]Non è \[C]solo \brk l’amic\[G]izia che ci \[A-]lega,
c’è qual\[F*]cosa di più Grande 
per cui \[G*]noi collaboriamo


\endverse


%%%%% RITORNELLO
\beginchorus
%\textnote{\textbf{Rit.}}

\[C]Tu Sign\[A-]ore hai scelto \[F]noi come att\[G]ori,
non per \[C]finge\[A-]re ma per \[F]ri-cre\[G]are
\[C]Tu hai \[A-]scelto noi per \[F]ricostru\[G]ire
con pa\[C]zienza \[A-]la bellezza \[F]che c’è per\[G]chè…
Ci hai scelti \[F]per ri-cre\[G]are con \[C]Te!	

Ci hai scelti \[F]per ri-cre\[G]are con \[C]Te!
Ci hai scelti \[F]per ri-cre\[G]are con \[C]Te! \[C*]
\endchorus



\endsong
%------------------------------------------------------------
%			FINE CANZONE
%------------------------------------------------------------



%titolo{Se m'accogli}
%autore{Sequeri}
%album{Eppure tu sei qui}
%tonalita{Do}
%famiglia{Liturgica}
%gruppo{}
%momenti{Offertorio}
%identificatore{se_m_accogli}
%data_revisione{2011_12_31}
%trascrittore{Francesco Endrici}
\beginsong{Se m'accogli}[by={Sequeri}]
\beginverse
Tra le \[C]mani non ho \[F]niente \brk spero \[E-]che mi accoglie\[A-]rai
chiedo \[C]solo di re\[F]stare accanto a \[G4]Te. \[G]
Sono \[C]ricco sola\[F]mente dell'a\[E-]more che mi \[A-]dai:
è per \[C]quelli che non l'\[F]hanno avuto \[G4]mai. \[G]
\endverse
\beginchorus
Se m'ac\[C]cogli, mio Si\[E7]gnore, altro \[A-]non Ti chiede\[C7]rò
e per \[F]sempre la Tua \[D-]strada la mia \[E7]strada reste\[A-]rà
nella \[F]gioia e nel do\[C]lore fino a \[F]quando \[D-]Tu vor\[E]rai
con la \[A-]mano nella \[G]Tua cammine\[F]rò. \[C]
\endchorus
\beginverse
%\chordsoff
Io Ti ^prego con il ^cuore, so che ^Tu mi ascolte^rai
rendi ^forte la mia ^fede più che ^mai. ^
Tieni ac^cesa la mia ^luce fino al ^giorno che Tu ^sai, 
con i ^miei fratelli in^contro a Te ver^rò. ^
\endverse
\endsong


%-------------------------------------------------------------
%			INIZIO	CANZONE
%-------------------------------------------------------------


%titolo: 	Segni del tuo amore
%autore: 	Gen Verde, Gen rosso
%tonalita: 	Do

\beginsong{Segni del Tuo amore}[by={Gen\ Verde, Gen\ Rosso}]
\transpose{0} 						% <<< TRASPOSIZIONE #TONI (0 nullo)
\momenti{Offertorio; Prima Comunione}							% <<< INSERISCI MOMENTI	
% momenti vanno separati da ; e vanno scelti tra:
% Ingresso; Atto penitenziale; Acclamazione al Vangelo; Dopo il Vangelo; Offertorio; Comunione; Ringraziamento; Fine; Santi; Pasqua; Avvento; Natale; Quaresima; Canti Mariani; Battesimo; Prima Comunione; Cresima; Matrimonio; Meditazione;
\ifchorded
	%\textnote{Tonalità originale }	% <<< EV COMMENTI (tonalità originale/migliore)
\fi



%%%%%% INTRODUZIONE
\ifchorded
\vspace*{\versesep}
\textnote{Intro: \qquad \qquad  (\quarternote 118)}% % << MODIFICA IL TEMPO
% Metronomo: \eighthnote (ottavo) \quarternote (quarto) \halfnote (due quarti)
\vspace*{-\versesep}
\beginverse*

\nolyrics

%---- Prima riga -----------------------------
\vspace*{-\versesep}
\[C] \[C]  \[C] \[D-]	 % \[*D] per indicare le pennate, \rep{2} le ripetizioni

%---- Ogni riga successiva -------------------
\vspace*{-\versesep}
\[C] \[C]  \[*D-] \[*C] \[D-]	

%---- Ev Indicazioni -------------------------			
%\textnote{\textit{(Oppure tutta la strofa)} }	

\endverse
\fi


%%%%% STROFA
\beginverse
\memorize

\[C]Mille e mille grani nelle 
spighe \[D-]d'o\[C]ro \quad \[*D-] \[*C] \[D-]
\[C]mandano fragranza e danno 
gioia al \[D-]cuo\[C]re, \quad \[*D-] \[*C] \[D-]
\[C]quando, macinati, fanno un 
pane \[D-]so\[C]lo, \quad \[*D-] \[*C] \[D-]
\[C]pane quotidiano, dono tuo, 
Si\[D-]gno\[C]re. \quad \[*D-] \[*C] \[C]

\endverse



%%%%% RITORNELLO
\beginchorus
\textnote{\textbf{Rit.}}

\[G]Ecco il pane e il vino, segni del tuo a\[F]mo\[C]re.
\[G]Ecco questa offerta, accoglila Si\[F]gno\[C]re,
\[F]tu di mille e mille \[G]cuori fai un cuore \[C]solo,
un corpo solo in \[G]te
e il \[F]Figlio tuo verrà, vi\[G]vrà 
ancora in mezzo a \[C]noi.

\endchorus




%%%%%% EV. INTERMEZZO
\beginverse*
\vspace*{1.3\versesep}
{
	\nolyrics
	\textnote{Breve intermezzo strumentale}
	
	\ifchorded

	%---- Prima riga -----------------------------
	\vspace*{-\versesep}
	\[(C))] \[C]  \[C] \[D-]	 % \[*D] per indicare le pennate, \rep{2} le ripetizioni

	%---- Ogni riga successiva -------------------
	\vspace*{-\versesep}
	\[C] \[C]  \[*D-] \[*C] \[D-]	


	\fi
	%---- Ev Indicazioni -------------------------			
	%\textnote{\textit{(ripetizione della strofa)}} 
	 
}
\vspace*{\versesep}
\endverse




%%%%% STROFA
\beginverse

^Mille grappoli maturi 
sotto il ^so^le, \quad ^ ^ ^
^festa della terra donano 
vi^go^re, \quad ^ ^ ^
^quando da ogni perla stilla 
il vino ^nuo^vo, \quad ^ ^ ^
^vino della gioia, dono tuo, 
Si^gno^re. \quad ^ ^ ^

\endverse


%%%%% RITORNELLO
\beginchorus
\textnote{\textbf{Rit.}}

\[G]Ecco il pane e il vino, segni del tuo a\[F]mo\[C]re.
\[G]Ecco questa offerta, accoglila Si\[F]gno\[C]re,
\[F]tu di mille e mille \[G]cuori fai un cuore \[C]solo,
un corpo solo in \[G]te
e il \[F]Figlio tuo verrà, vi\[G]vrà 
ancora in mezzo a \[F]noi.		\quad \[C]  \iflyric \rep{2} \fi
\ifchorded
\vspace*{2\versesep}
\[G]Ecco il pane e il vino, segni del tuo a\[F]mo\[C]re.
\[G]Ecco questa offerta, accoglila Si\[F]gno\[C]re,
\[F]tu di mille e mille \[G]cuori fai un cuore \[C]solo,
un corpo solo in \[G]te
e il \[F]Figlio tuo verrà, vi\[G]vrà 
ancora in mezzo a \[C]noi.
\fi
\endchorus



%%%%%% EV. INTERMEZZO
\beginverse*
\vspace*{1.3\versesep}
{
	\nolyrics
	\musicnote{Finale}
	
	\ifchorded

	%---- Prima riga -----------------------------
	\vspace*{-\versesep}
	\[(C)] \[C]  \[C] \[D-]	 % \[*D] per indicare le pennate, \rep{2} le ripetizioni

	%---- Ogni riga successiva -------------------
	\vspace*{-\versesep}
	\[C] \[C]  \[*D-] \[*C] \[D-]	\quad \[C]


	\fi
	%---- Ev Indicazioni -------------------------			
	%\textnote{\textit{(ripetizione della strofa)}} 
	 
}
\vspace*{\versesep}
\endverse



\endsong
%------------------------------------------------------------
%			FINE CANZONE
%------------------------------------------------------------


%-------------------------------------------------------------
%			INIZIO	CANZONE
%-------------------------------------------------------------


%titolo: 	Sei fuoco e vento
%autore: 	Andrea Testa
%tonalita: 	Do



%%%%%% TITOLO E IMPOSTAZONI
\beginsong{Sei fuoco e vento}[by={A. Testa}] 	% <<< MODIFICA TITOLO E AUTORE
\transpose{0} 						% <<< TRASPOSIZIONE #TONI (0 nullo)
\momenti{Ingresso; Comunione; Cresima}							% <<< INSERISCI MOMENTI	
% momenti vanno separati da ; e vanno scelti tra:
% Ingresso; Atto penitenziale; Acclamazione al Vangelo; Dopo il Vangelo; Offertorio; Comunione; Ringraziamento; Fine; Santi; Pasqua; Avvento; Natale; Quaresima; Canti Mariani; Battesimo; Prima Comunione; Cresima; Matrimonio; Meditazione; Spezzare del pane;
\ifchorded
	%\textnote{Tonalità originale }	% <<< EV COMMENTI (tonalità originale/migliore)
\fi


%%%%%% INTRODUZIONE
\ifchorded
\vspace*{\versesep}
\musicnote{
\begin{minipage}{0.48\textwidth}
\textbf{Intro}
\hfill 
%( \eighthnote \, 80)   % <<  MODIFICA IL TEMPO
% Metronomo: \eighthnote (ottavo) \quarternote (quarto) \halfnote (due quarti)
\end{minipage}
} 	
\vspace*{-\versesep}
\beginverse*

\nolyrics

%---- Prima riga -----------------------------
\vspace*{-\versesep}
\[A-]  % \[*D] per indicare le pennate, \rep{2} le ripetizioni

%---- Ogni riga successiva -------------------
%\vspace*{-\versesep}
%\[G] \[C]  \[D]	

%---- Ev Indicazioni -------------------------			
%\textnote{\textit{(Oppure tutta la strofa)} }	

\endverse
\fi



\beginverse*

\textnote{Recitato}	
\musicnote{\textit{(accompagnamento con gli accordi del ritornello)}}
\chordsoff
\textit{All'improvviso si sentì un rumore dal cielo 
come quando tira un forte vento
e riempì tutta la casa  dove si trovavano.
Allora videro qualcosa di simile \brk a lingue di fuoco
che si separavano e si posavano \brk	 sopra ciascuno di loro...
...e tutti furono pieni del Suo Spirito}
\endverse


%%%%% STROFA
\beginverse		%Oppure \beginverse* se non si vuole il numero di fianco
\memorize 		% <<< DECOMMENTA se si vuole utilizzarne la funzione
%\chordsoff		% <<< DECOMMENTA se vuoi una strofa senza accordi

In un \[A-]mare calmo e immobile, \brk con un \[C]cielo senza nuvole,
non si \[G]riesce a navi\[D-]gare,  \brk prose\[F]guire non si \[A-]può.
Una \[A-]brezza lieve e debole,    \brk poi di\[C]venta un vento a raffiche,
soffia \[G]forte sulle \[D-]barche    \brk e ci \[F]spinge via di \[A-]qua.
\vspace*{\versesep}
Come il \[C]vento da la \[G]forza   \brk per viag\[A-]giare in un o\[E-]ceano
così \[F]Tu ci dai lo \[C]Spirito   \brk che ci \[D-]guiderà da \[G]Te.

\endverse

%%%%% RITORNELLO
\beginchorus
\textnote{\textbf{Rit.} }

Sei come \[C]vento  \brk  che \[F]gonfia le \[C]vele,
sei come \[C]fuoco  \brk che ac\[F]cende l'a\[G]more, 
\[E] sei come l'\[A-]aria  \brk che \[E-]si respira \[F]libera
chiara \[C]luce che \[G]il cammino \[F]indica.	\rep{2}

\endchorus


%%%%%% EV. INTERMEZZO
\beginverse*
\vspace*{1.3\versesep}
{
	\nolyrics
	\textnote{Intermezzo strumentale}
	
	\ifchorded

	%---- Prima riga -----------------------------
	\vspace*{-\versesep}
	\[A-] \[A-]
	\fi
	%---- Ev Indicazioni -------------------------			
	%\textnote{\textit{(ripetizione della strofa)}} 
	 
}
\vspace*{\versesep}
\endverse






%%%%% STROFA
\beginverse		%Oppure \beginverse* se non si vuole il numero di fianco
%\memorize 		% <<< DECOMMENTA se si vuole utilizzarne la funzione
%\chordsoff		% <<< DECOMMENTA se vuoi una strofa senza accordi

Nella ^notte impenetrabile, \brk ogni ^cosa è irraggiungibile,
non puoi ^scegliere la ^strada  \brk se non ^vedi avanti a ^te.
Una ^luce fioca e debole,  \brk  sembra ^sorgere e poi crescere,
come ^fiamma che ^rigenera  \brk e che il^lumina la ^via.
\vspace*{\versesep}
Come il ^fuoco scioglie il ^gelo  \brk e rischi^ara ogni sen^tiero
così ^Tu riscaldi il ^cuore  \brk di chi ^Verbo annunce^rà.

\endverse




%%%%%% EV. CHIUSURA SOLO STRUMENTALE
\ifchorded
\beginchorus %oppure \beginverse*
\vspace*{1.3\versesep}
\textnote{Chiusura } %<<< EV. INDICAZIONI

\[C*]

\endchorus  %oppure \endverse
\fi


\endsong
%------------------------------------------------------------
%			FINE CANZONE
%------------------------------------------------------------
%-------------------------------------------------------------
%			INIZIO	CANZONE
%-------------------------------------------------------------


%titolo: 	Servo per amore
%autore: 	Gen Rosso
%tonalita: 	Si-



%%%%%% TITOLO E IMPOSTAZONI
\beginsong{Servo per amore}[by={Gen\ Rosso}]	% <<< MODIFICA TITOLO E AUTORE
\transpose{0} 						% <<< TRASPOSIZIONE #TONI (0 nullo)
\momenti{Offertorio}							% <<< INSERISCI MOMENTI	
% momenti vanno separati da ; e vanno scelti tra:
% Ingresso; Atto penitenziale; Acclamazione al Vangelo; Dopo il Vangelo; Offertorio; Comunione; Ringraziamento; Fine; Santi; Pasqua; Avvento; Natale; Quaresima; Canti Mariani; Battesimo; Prima Comunione; Cresima; Matrimonio; Meditazione; Spezzare del pane;
\ifchorded
	%\textnote{Tonalità migliore }	% <<< EV COMMENTI (tonalità originale/migliore)
\fi


%%%%%% INTRODUZIONE
\ifchorded
\vspace*{\versesep}
\textnote{Intro: \qquad \qquad  }%(\eighthnote 116) % <<  MODIFICA IL TEMPO
% Metronomo: \eighthnote (ottavo) \quarternote (quarto) \halfnote (due quarti)
\vspace*{-\versesep}
\beginverse*

\nolyrics

%---- Prima riga -----------------------------
\vspace*{-\versesep}
\[B-]	 % \[*D] per indicare le pennate, \rep{2} le ripetizioni

%---- Ogni riga successiva -------------------
%\vspace*{-\versesep}
%\[G] \[C]  \[D]	

%---- Ev Indicazioni -------------------------			
%\textnote{\textit{(Oppure tutta la strofa)} }	

\endverse
\fi





%%%%% STROFA
\beginverse		%Oppure \beginverse* se non si vuole il numero di fianco
\memorize 		% <<< DECOMMENTA se si vuole utilizzarne la funzione
%\chordsoff		% <<< DECOMMENTA se vuoi una strofa senza accordi 
Una \[B-]notte di sudore \brk sulla barca in mezzo al \[D]mare
e mentre il \[A]cielo s'imbianca \[F#-]già
tu guardi \[G]le tue reti \[B-]vuote.
Ma la \[D]voce che ti chiama \brk un altro \[A]mare ti mostre\[D]rà
e sulle \[G]rive di ogni \[D]cuore le tue \[E-*]reti \[G*]gette\[D]rai.
\endverse




%%%%% RITORNELLO
\beginchorus
\textnote{\textbf{Rit.}}
\[B-]Offri la vita \[D]tua
come Ma\[A]ria ai \[F#-]piedi della \[G]croce
\[B-]e sarai \[D]servo di ogni \[A]uomo,
\[D]servo per a\[G]more, 
\[D]sacerdote \[E-*]dell'u\[G*]mani\[D]tà.
\endchorus




%%%%% STROFA
\beginverse		%Oppure \beginverse* se non si vuole il numero di fianco
%\memorize 		% <<< DECOMMENTA se si vuole utilizzarne la funzione
%\chordsoff		% <<< DECOMMENTA se vuoi una strofa senza accordi
Avan^zavi nel silenzio,  \brk fra le lacrime spe^ravi
che il seme ^sparso davanti a ^te 
cadesse ^sulla buona ^terra. 
Ora il ^cuore tuo è in festa \brk perché il ^grano biondeggia or^mai, 
è matu^rato sotto il ^sole,  \brk puoi ri^porlo ^nei gra^nai.
\endverse



\endsong
%------------------------------------------------------------
%			FINE CANZONE
%------------------------------------------------------------
%-------------------------------------------------------------
%			INIZIO	CANZONE
%-------------------------------------------------------------


%titolo: 	Solo per Te
%autore: 	Sermig
%tonalita: 	Re



%%%%%% TITOLO E IMPOSTAZONI
\beginsong{Solo per Te}[by={Sermig}] 	% <<< MODIFICA TITOLO E AUTORE
\transpose{0} 						% <<< TRASPOSIZIONE #TONI (0 nullo)
%\preferflats  %SE VOGLIO FORZARE i bemolle come alterazioni
%\prefersharps %SE VOGLIO FORZARE i # come alterazioni
\momenti{}							% <<< INSERISCI MOMENTI	
% momenti vanno separati da ; e vanno scelti tra:
% Ingresso; Atto penitenziale; Acclamazione al Vangelo; Dopo il Vangelo; Offertorio; Comunione; Ringraziamento; Fine; Santi; Pasqua; Avvento; Natale; Quaresima; Canti Mariani; Battesimo; Prima Comunione; Cresima; Matrimonio; Meditazione; Spezzare del pane;
\ifchorded
	%\textnote{$\bigstar$ Tonalità migliore }	% <<< EV COMMENTI (tonalità originale\migliore)
\fi


%%%%%% INTRODUZIONE
\ifchorded
\vspace*{\versesep}
\musicnote{
\begin{minipage}{0.48\textwidth}
\textbf{Intro}
\hfill 
%( \eighthnote \, 80)   % <<  MODIFICA IL TEMPO
% Metronomo: \eighthnote (ottavo) \quarternote (quarto) \halfnote (due quarti)
\end{minipage}
} 	
\vspace*{-\versesep}
\beginverse*

\nolyrics

%---- Prima riga -----------------------------
\vspace*{-\versesep}
\[D] \[A] \[G] \[D] \[A] \[G] 	 % \[*D] per indicare le pennate, \rep{2} le ripetizioni

%---- Ogni riga successiva -------------------
%\vspace*{-\versesep}
%\[G] \[C]  \[D]	

%---- Ev Indicazioni -------------------------			
%\textnote{\textit{[oppure tutta la strofa]} }	

\endverse
\fi




%%%%% STROFA
\beginverse*		%Oppure \beginverse* se non si vuole il numero di fianco
\memorize 		% <<< DECOMMENTA se si vuole utilizzarne la funzione
%\chordsoff		% <<< DECOMMENTA se vuoi una strofa senza accordi

\[D]Solo l'a\[A]more\[B-] \brk \[A]rende il \[D]sacrificio un \[A]dono\[G] \[A]
\[D]solo l'a\[A]more\[B-]  \brk fa \[A]diven\[D]tare giorno la \[A]notte\[G] \[A]
\[B-]solo l'a\[A]mo\[D]re  \brk \[A]fa del \[B-]tempo una preghie\[A]\[D]ra\[A]
\[B-]solo l'a\[A]mo\[D]re   \brk dà \[A]la cer\[B-]tezza che per t\[A]e
vale la \[G]pena correre \brk  \[D]cercare\[A], senza sost\[D]a.

\endverse



%%%%%% EV. INTERMEZZO
\beginverse*
\vspace*{1.3\versesep}
{
	\nolyrics
	\textnote{Intermezzo strumentale}
	
	\ifchorded

	%---- Prima riga -----------------------------
	\vspace*{-\versesep}
	\[A] \[G] \[A] \[D]

	%---- Ogni riga successiva -------------------
	\vspace*{-\versesep}
	\[A] \[G] \[A]


	\fi
	%---- Ev Indicazioni -------------------------			
	%\textnote{\textit{(ripetizione della strofa)}} 
	 
}
\vspace*{\versesep}
\endverse


%%%%% STROFA
\beginverse*		%Oppure \beginverse* se non si vuole il numero di fianco
%\memorize 		% <<< DECOMMENTA se si vuole utilizzarne la funzione
%\chordsoff		% <<< DECOMMENTA se vuoi una strofa senza accordi

\[D]Solo l'a\[A]more\[B-]  \brk \[A]apre il \[D]cuore alla sag\[A]ge\[G]zza\[A]
\[D]solo l'a\[A]more\[B-]  \brk \[A]rende \[D]liberi e si\[A]cu\[G]ri\[A]
\[B-]solo l'a\[A]mo\[D]re  \brk a\[A]sciuga il \[B-]viso di chi pia\[A]n\[D]ge\[A]
\[B-]solo l'a\[A]mo\[D]re  \brk \[A]dà un \[B-]volto alla bont\[A]à
divide il \[G]pane  \brk senza chie\[D]dere per sé.\[A] \[F#]

\endverse



%%%%% STROFA
\beginverse*		%Oppure \beginverse* se non si vuole il numero di fianco
%\memorize 		% <<< DECOMMENTA se si vuole utilizzarne la funzione
%\chordsoff		% <<< DECOMMENTA se vuoi una strofa senza accordi

\[B-]Solo per te, \[A] \[D]
\[G]solo con te, \[D] \[A] \[F#7]
\[B-]solo in te \[A]che \[D]sei l'amo\[A]re.\[F#7]
\[B-]Solo per te, \[A] \[D]
\[G]solo con te, \[D] \[A] \[F#7]
\[B-]solo in te \[A]che \[D]sei l'amo\[A]re.

\endverse








%%%%%% EV. CHIUSURA SOLO STRUMENTALE
\ifchorded
\beginchorus %oppure \beginverse*
\vspace*{1.3\versesep}
\textnote{Chiusura strumentale } %<<< EV. INDICAZIONI

\nolyrics

	%---- Prima riga -----------------------------
	\vspace*{-\versesep}
	\[D] \[A] \[G] \[A]

	%---- Ogni riga successiva -------------------
	\vspace*{-\versesep}
	\[D] \[A] \[G] \[A]	\rep{2}

	%---- Ogni riga successiva -------------------
	\vspace*{-\versesep}
	\[D*] 	


	%---- Ev Indicazioni -------------------------			
	%\textnote{\textit{(ripetizione della strofa)}} 

\endchorus  %oppure \endverse
\fi


\endsong
%------------------------------------------------------------
%			FINE CANZONE
%------------------------------------------------------------



%-------------------------------------------------------------
%			INIZIO	CANZONE
%-------------------------------------------------------------


%titolo: 	Sono qui a lodarti
%autore: 	T. Hughes
%tonalita: 	Mi > Re 



%%%%%% TITOLO E IMPOSTAZONI
\beginsong{Sono qui a lodarti}[by={T. Hughes}] 	% <<< MODIFICA TITOLO E AUTORE
\transpose{-4} 						% <<< TRASPOSIZIONE #TONI (0 nullo)
%\preferflats  %SE VOGLIO FORZARE i bemolle come alterazioni
%\prefersharps %SE VOGLIO FORZARE i # come alterazioni
\momenti{Ringraziamento; Meditazione}							% <<< INSERISCI MOMENTI	
% momenti vanno separati da ; e vanno scelti tra:
% Ingresso; Atto penitenziale; Acclamazione al Vangelo; Dopo il Vangelo; Offertorio; Comunione; Ringraziamento; Fine; Santi; Pasqua; Avvento; Natale; Quaresima; Canti Mariani; Battesimo; Prima Comunione; Cresima; Matrimonio; Meditazione; Spezzare del pane;
\ifchorded
	\textnote{Tonalità migliore }	% <<< EV COMMENTI (tonalità originale/migliore)
\fi


%%%%%% INTRODUZIONE
\ifchorded
\vspace*{\versesep}
\textnote{Intro: \qquad \qquad (\eighthnote 70)  }%(\eighthnote 116) % <<  MODIFICA IL TEMPO
% Metronomo: \eighthnote (ottavo) \quarternote (quarto) \halfnote (due quarti)
\vspace*{-\versesep}
\beginverse*

\nolyrics

%---- Prima riga -----------------------------
\vspace*{-\versesep}
\[E] \[B] \[F#-] \[A]	 % \[*D] per indicare le pennate, \rep{2} le ripetizioni

%---- Ogni riga successiva -------------------
\vspace*{-\versesep}
\[E] \[B]\[A] \[A]

%---- Ev Indicazioni -------------------------			
%\textnote{\textit{(Oppure tutta la strofa)} }	

\endverse
\fi




%%%%% STROFA
\beginverse		%Oppure \beginverse* se non si vuole il numero di fianco
\memorize 		% <<< DECOMMENTA se si vuole utilizzarne la funzione
%\chordsoff		% <<< DECOMMENTA se vuoi una strofa senza accordi

\[E]Luce del \[B]mondo, nel \[F#-]buio del \[A]cuore
\[E]Vieni ed il\[B]lumina\[A]mi. \[A]
\[E]Tu mia \[B]sola spe\[F#-]ranza di \[A]vita,
\[E]Resta per \[B]sempre con \[A]me.

\endverse




%%%%% RITORNELLO
\beginchorus
\textnote{\textbf{Rit.}}

\[(A)]Sono qui a lo\[E]darti, qui per ado\[B]rarti
qui per dirti \[E]che Tu sei il mio \[A]Dio.
E solo Tu sei \[E]santo, sei meravigl\[B]ioso
Degno e glor\[E]ioso sei per \[A]me.

\endchorus



%%%%% STROFA
\beginverse		%Oppure \beginverse* se non si vuole il numero di fianco
%\memorize 		% <<< DECOMMENTA se si vuole utilizzarne la funzione
%\chordsoff		% <<< DECOMMENTA se vuoi una strofa senza accordi

^Re della ^storia e ^Re nella ^gloria,
^sei sceso in ^terra fra ^noi. ^
^Con umil^tà il Tuo ^trono hai lasc^iato,
^per dimos^trarci il Tuo a^mor.

\endverse



%%%%% RITORNELLO
\beginchorus
\textnote{\textbf{Rit.}}

\[(A)]Sono qui a lo\[E]darti, qui per ado\[B]rarti
qui per dirti \[E]che Tu sei il mio \[A]Dio.
E solo Tu sei \[E]santo, sei meravigl\[B]ioso
Degno e glor\[E]ioso sei per \[A]me.

\endchorus


%%%%% STROFA
\beginverse		%Oppure \beginverse* se non si vuole il numero di fianco
%\memorize 		% <<< DECOMMENTA se si vuole utilizzarne la funzione
%\chordsoff		% <<< DECOMMENTA se vuoi una strofa senza accordi
\vspace*{1.3\versesep}
\textnote{Finale \textit{(a ripetizione)}}

Non \[B]so quanto \[E]è cos\[A]tato a Te
mo\[B]rire in \[E]croce, \[A]lì per me...

\endverse





\endsong
%------------------------------------------------------------
%			FINE CANZONE
%------------------------------------------------------------




%++++++++++++++++++++++++++++++++++++++++++++++++++++++++++++
%			CANZONE TRASPOSTA
%++++++++++++++++++++++++++++++++++++++++++++++++++++++++++++
\ifchorded
%decremento contatore per avere stesso numero
\addtocounter{songnum}{-1} 
\beginsong{Sono qui a lodarti}[by={T. Hughes}]  	% <<< COPIA TITOLO E AUTORE
\transpose{0} 						% <<< TRASPOSIZIONE #TONI + - (0 nullo)
%\preferflats  %SE VOGLIO FORZARE i bemolle come alterazioni
%\prefersharps %SE VOGLIO FORZARE i # come alterazioni
\ifchorded
	\textnote{Tonalità originale}	% <<< EV COMMENTI (tonalità originale/migliore)
\fi



%%%%%% INTRODUZIONE
\ifchorded
\vspace*{\versesep}
\textnote{Intro: \qquad \qquad (\eighthnote 70)  }%(\eighthnote 116) % <<  MODIFICA IL TEMPO
% Metronomo: \eighthnote (ottavo) \quarternote (quarto) \halfnote (due quarti)
\vspace*{-\versesep}
\beginverse*

\nolyrics

%---- Prima riga -----------------------------
\vspace*{-\versesep}
\[E] \[B] \[F#-] \[A]	 % \[*D] per indicare le pennate, \rep{2} le ripetizioni

%---- Ogni riga successiva -------------------
\vspace*{-\versesep}
\[E] \[B]\[A] \[A]

%---- Ev Indicazioni -------------------------			
%\textnote{\textit{(Oppure tutta la strofa)} }	

\endverse
\fi





%%%%% STROFA
\beginverse		%Oppure \beginverse* se non si vuole il numero di fianco
\memorize 		% <<< DECOMMENTA se si vuole utilizzarne la funzione
%\chordsoff		% <<< DECOMMENTA se vuoi una strofa senza accordi

\[E]Luce del \[B]mondo, nel \[F#-]buio del \[A]cuore
\[E]Vieni ed il\[B]lumina\[A]mi. \[A]
\[E]Tu mia \[B]sola spe\[F#-]ranza di \[A]vita,
\[E]Resta per \[B]sempre con \[A]me.

\endverse




%%%%% RITORNELLO
\beginchorus
\textnote{\textbf{Rit.}}

\[(A)]Sono qui a lo\[E]darti, qui per ado\[B]rarti
qui per dirti \[E]che Tu sei il mio \[A]Dio.
E solo Tu sei \[E]santo, sei meravigl\[B]ioso
Degno e glor\[E]ioso sei per \[A]me.

\endchorus



%%%%% STROFA
\beginverse		%Oppure \beginverse* se non si vuole il numero di fianco
%\memorize 		% <<< DECOMMENTA se si vuole utilizzarne la funzione
%\chordsoff		% <<< DECOMMENTA se vuoi una strofa senza accordi

^Re della ^storia e ^Re nella ^gloria,
^sei sceso in ^terra fra ^noi. ^
^Con umil^tà il Tuo ^trono hai lasc^iato,
^per dimos^trarci il Tuo a^mor.

\endverse



%%%%% RITORNELLO
\beginchorus
\textnote{\textbf{Rit.}}

\[(A)]Sono qui a lo\[E]darti, qui per ado\[B]rarti
qui per dirti \[E]che Tu sei il mio \[A]Dio.
E solo Tu sei \[E]santo, sei meravigl\[B]ioso
Degno e glor\[E]ioso sei per \[A]me.

\endchorus


%%%%% STROFA
\beginverse		%Oppure \beginverse* se non si vuole il numero di fianco
%\memorize 		% <<< DECOMMENTA se si vuole utilizzarne la funzione
%\chordsoff		% <<< DECOMMENTA se vuoi una strofa senza accordi
\vspace*{1.3\versesep}
\textnote{Finale \textit{(a ripetizione)}}

Non \[B]so quanto \[E]è cos\[A]tato a Te
mo\[B]rire in \[E]croce, \[A]lì per me...

\endverse



\endsong


\fi
%++++++++++++++++++++++++++++++++++++++++++++++++++++++++++++
%			FINE CANZONE TRASPOSTA
%++++++++++++++++++++++++++++++++++++++++++++++++++++++++++++

%-------------------------------------------------------------
%			INIZIO	CANZONE
%-------------------------------------------------------------


%titolo: 	Spirito Santo, discendi tra noi
%autore: 	
%tonalita: 	Do 



%%%%%% TITOLO E IMPOSTAZONI
\beginsong{Spirito Santo, discendi tra noi}[by={}] 	% <<< MODIFICA TITOLO E AUTORE
\transpose{0} 						% <<< TRASPOSIZIONE #TONI (0 nullo)
%\preferflats  %SE VOGLIO FORZARE i bemolle come alterazioni
%\prefersharps %SE VOGLIO FORZARE i # come alterazioni
\momenti{Cresima; Pasqua}							% <<< INSERISCI MOMENTI	
% momenti vanno separati da ; e vanno scelti tra:
% Ingresso; Atto penitenziale; Acclamazione al Vangelo; Dopo il Vangelo; Offertorio; Comunione; Ringraziamento; Fine; Santi; Pasqua; Avvento; Natale; Quaresima; Canti Mariani; Battesimo; Prima Comunione; Cresima; Matrimonio; Meditazione; Spezzare del pane;
\ifchorded
	%\textnote{Tonalità migliore }	% <<< EV COMMENTI (tonalità originale/migliore)
\fi

%%%%%% INTRODUZIONE
\ifchorded
\vspace*{\versesep}
\musicnote{
\begin{minipage}{0.48\textwidth}
\textbf{Intro}
\hfill 
%( \eighthnote \, 80)   % <<  MODIFICA IL TEMPO
% Metronomo: \eighthnote (ottavo) \quarternote (quarto) \halfnote (due quarti)
\end{minipage}
} 	
\vspace*{-\versesep}
\beginverse*

\nolyrics

%---- Prima riga -----------------------------
\vspace*{-\versesep}
\[C] \[A-] \[F] \[C]	 % \[*D] per indicare le pennate, \rep{2} le ripetizioni

%---- Ogni riga successiva -------------------
%\vspace*{-\versesep}
%\[G] \[C]  \[D]	

%---- Ev Indicazioni -------------------------			
%\textnote{\textit{(Oppure tutta la strofa)} }	

\endverse
\fi








%%%%% RITORNELLO
\beginchorus
\textnote{\textbf{Rit.}}

S\[C]pirito \[A-]Santo, dis\[F]cendi tra \[G]noi:
\[C]la nostra \[F]fede ha bi\[G]sogno di \[E]te;
\[A-]al nostro c\[F]uore in\[A-]segna ad a\[G7]mare
\[C]e la spe\[F]ranza non \[D-*]toglier\[G7*]ci \[C]mai!


\endchorus



%%%%% STROFA
\beginverse		%Oppure \beginverse* se non si vuole il numero di fianco
\memorize 		% <<< DECOMMENTA se si vuole utilizzarne la funzione
%\chordsoff		% <<< DECOMMENTA se vuoi una strofa senza accordi

\[C]Tu sei il \[A-]dono pro\[D-]messo dal \[G7]Padre;
sei \[C]fuoco di a\[A-]more, sor\[D-]gente di \[G7]vita.


\endverse

%%%%% STROFA
\beginverse		%Oppure \beginverse* se non si vuole il numero di fianco
%\memorize 		% <<< DECOMMENTA se si vuole utilizzarne la funzione
\chordsoff		% <<< DECOMMENTA se vuoi una strofa senza accordi

Tu vivi con noi e sei nostra forza:
sostienici sempre nel nostro cammino.


\endverse



%%%%% STROFA
\beginverse		%Oppure \beginverse* se non si vuole il numero di fianco
%\memorize 		% <<< DECOMMENTA se si vuole utilizzarne la funzione
\chordsoff		% <<< DECOMMENTA se vuoi una strofa senza accordi

Tu sei sapienza che vince ogni errore;
di te ci fidiamo e avremo la luce.

\endverse

\endsong
%------------------------------------------------------------
%			FINE CANZONE
%------------------------------------------------------------



%-------------------------------------------------------------
%			INIZIO	CANZONE
%-------------------------------------------------------------


%titolo: 	Su ali d'acquila
%autore: 	Daniele Ricci
%tonalita: 	FA 



%%%%%% TITOLO E IMPOSTAZONI
\beginsong{Su ali d'aquila}[by={J. M. Joncas}] 	% <<< MODIFICA TITOLO E AUTORE
\transpose{0} 						% <<< TRASPOSIZIONE #TONI (0 nullo)
\momenti{Congedo; Matrimonio}							% <<< INSERISCI MOMENTI	
% momenti vanno separati da ; e vanno scelti tra:
% Ingresso; Atto penitenziale; Acclamazione al Vangelo; Dopo il Vangelo; Offertorio; Comunione; Ringraziamento; Fine; Santi; Pasqua; Avvento; Natale; Quaresima; Canti Mariani; Battesimo; Prima Comunione; Cresima; Matrimonio; Meditazione;
\ifchorded
	\textnote{$\bigstar$ Tonalità originale }	% <<< EV COMMENTI (tonalità originale/migliore)
\fi


%%%%%% INTRODUZIONE
\ifchorded
\vspace*{\versesep}
\musicnote{
\begin{minipage}{0.48\textwidth}
\textbf{Intro}
\hfill 
%( \eighthnote \, 80)   % <<  MODIFICA IL TEMPO
% Metronomo: \eighthnote (ottavo) \quarternote (quarto) \halfnote (due quarti)
\end{minipage}
} 	
\vspace*{-\versesep}
\beginverse*

\nolyrics

%---- Prima riga -----------------------------
\vspace*{-\versesep}
\[F7+] \[C7+] \[F7+] \[C7+] 	 % \[*D] per indicare le pennate, \rep{2} le ripetizioni

%---- Ogni riga successiva -------------------
%\vspace*{-\versesep}
%\[G] \[C]  \[D]	

%---- Ev Indicazioni -------------------------			
%\textnote{\textit{(Oppure tutta la strofa)} }	

\endverse
\fi




%%%%% STROFA
\beginverse		%Oppure \beginverse* se non si vuole il numero di fianco
%\memorize 		% <<< DECOMMENTA se si vuole utilizzarne la funzione
%\chordsoff		& <<< DECOMMENTA se vuoi una strofa senza accordi

\[F7+]Tu che abiti al \[C7+]riparo del Signore
\[F7+]e che dimori alla sua \[C7+]ombra
\[E&]di al Signore mio \[C-]Rifugio,
mia \[F-]roccia in cui con\[G]fido.

\endverse




%%%%% RITORNELLO
\beginchorus
\textnote{\textbf{Rit.}}

E ti ri\[C]alzerà, \[7+]ti solleverà
su ali d'\[D-]aquila \[G]ti regge\[7]rà
sulla \[G-7]brezza dell'\[C]alba ti \[F]farà bril\[D-]lar
co\[G*]me il \[A-]sole, \[E-7]  co\[F]sì nelle sue \[G4]\[7]mani vi\[C]vrai.

\endchorus


%%%%% STROFA
\beginverse		%Oppure \beginverse* se non si vuole il numero di fianco
%\memorize 		% <<< DECOMMENTA se si vuole utilizzarne la funzione
\chordsoff		% <<< DECOMMENTA se vuoi una strofa senza accordi
Dal laccio del cacciatore ti libererà
e dalla carestia che ti distrugge
poi ti coprirà con le sue ali
e rifugio troverai.

\endverse



%%%%% STROFA
\beginverse		%Oppure \beginverse* se non si vuole il numero di fianco
%\memorize 		% <<< DECOMMENTA se si vuole utilizzarne la funzione
\chordsoff		% <<< DECOMMENTA se vuoi una strofa senza accordi
Perché ai suoi angeli da dato un comando
di preservarti in tutte le tue vie
ti porteranno sulle loro mani
contro la pietra non inciamperai.

\endverse



%%%%%% EV. FINALE

\beginchorus %oppure \beginverse*
\vspace*{1.3\versesep}
\textnote{\textbf{Finale} } %<<< EV. INDICAZIONI

E ti ri\[C]alzerò, \[7+]ti solleverò
su ali d'\[D-]aquila \[G]ti regge\[7]rò
sulla \[G-7]brezza dell'\[C]alba ti \[F]farò bril\[D-]lar
co\[G]me il \[A-]sole, \[E-7]  co\[F]sì nelle mie \[G4]\[7]mani vi\[C]vrai.


\endchorus  %oppure \endverse




\endsong
%------------------------------------------------------------
%			FINE CANZONE
%------------------------------------------------------------




%++++++++++++++++++++++++++++++++++++++++++++++++++++++++++++
%			CANZONE TRASPOSTA
%++++++++++++++++++++++++++++++++++++++++++++++++++++++++++++
\ifchorded
%decremento contatore per avere stesso numero
\addtocounter{songnum}{-1} 
\beginsong{Su ali d'aquila}[by={J. M. Joncas}]	% <<< COPIA TITOLO E AUTORE
\transpose{-3} 						% <<< TRASPOSIZIONE #TONI + - (0 nullo)
\ifchorded
	\textnote{$\triangle$ Tonalità migliore per le chitarre}	% <<< EV COMMENTI (tonalità originale/migliore)
\fi


%%%%%% INTRODUZIONE
\ifchorded
\vspace*{\versesep}
\musicnote{
\begin{minipage}{0.48\textwidth}
\textbf{Intro}
\hfill 
%( \eighthnote \, 80)   % <<  MODIFICA IL TEMPO
% Metronomo: \eighthnote (ottavo) \quarternote (quarto) \halfnote (due quarti)
\end{minipage}
} 	
\vspace*{-\versesep}
\beginverse*

\nolyrics

%---- Prima riga -----------------------------
\vspace*{-\versesep}
\[F7+] \[C7+] \[F7+] \[C7+] 	 % \[*D] per indicare le pennate, \rep{2} le ripetizioni

%---- Ogni riga successiva -------------------
%\vspace*{-\versesep}
%\[G] \[C]  \[D]	

%---- Ev Indicazioni -------------------------			
%\textnote{\textit{(Oppure tutta la strofa)} }	

\endverse
\fi




%%%%% STROFA
\beginverse		%Oppure \beginverse* se non si vuole il numero di fianco
%\memorize 		% <<< DECOMMENTA se si vuole utilizzarne la funzione
%\chordsoff		& <<< DECOMMENTA se vuoi una strofa senza accordi

\[F7+]Tu che abiti al \[C7+]riparo del Signore
\[F7+]e che dimori alla sua \[C7+]ombra
\[E&]dì al Signore mio \[C-]Rifugio,
mia \[F-]roccia in cui con\[G]fido.

\endverse




%%%%% RITORNELLO
\beginchorus
\textnote{\textbf{Rit.}}

E ti ri\[C]alzerà, \[7+]ti solleverà
su ali d'\[D-]aquila \[G]ti regge\[7]rà
sulla \[G-7]brezza dell'\[C]alba ti \[F]farà bril\[D-]lar
co\[G]me il \[A-]sole, \[E-7]  co\[F]sì nelle sue \[G4]\[7]mani vi\[C]vrai.

\endchorus



%%%%% STROFA
\beginverse		%Oppure \beginverse* se non si vuole il numero di fianco
%\memorize 		% <<< DECOMMENTA se si vuole utilizzarne la funzione
\chordsoff		% <<< DECOMMENTA se vuoi una strofa senza accordi
Dal laccio del cacciatore ti libererà
e dalla carestia che ti distrugge
poi ti coprirà con le sue ali
e rifugio troverai.

\endverse


%%%%% STROFA
\beginverse		%Oppure \beginverse* se non si vuole il numero di fianco
%\memorize 		% <<< DECOMMENTA se si vuole utilizzarne la funzione
\chordsoff		% <<< DECOMMENTA se vuoi una strofa senza accordi
Non devi temer i terrori della notte
né freccia che vola di giorno
mille cadranno al tuo fianco
ma nulla ti colpirà
\endverse


%%%%% STROFA
\beginverse		%Oppure \beginverse* se non si vuole il numero di fianco
%\memorize 		% <<< DECOMMENTA se si vuole utilizzarne la funzione
\chordsoff		% <<< DECOMMENTA se vuoi una strofa senza accordi
Perché ai suoi angeli ha dato un comando
di preservarti in tutte le tue vie
ti porteranno sulle loro mani
contro la pietra non inciamperai.

\endverse



%%%%%% EV. FINALE

\beginchorus %oppure \beginverse*
\vspace*{1.3\versesep}
\textnote{\textbf{Finale} } %<<< EV. INDICAZIONI

E ti ri\[C]alzerò, \[7+]ti solleverò
su ali d'\[D-]aquila \[G]ti regge\[7]rò
sulla \[G-7]brezza dell'\[C]alba ti \[F]farò bril\[D-]lar
co\[G*]me il \[A-]sole, \[E-7]  co\[F]sì nelle mie \[G4]\[7]mani vi\[C]vrai.


\endchorus  %oppure \endverse




\endsong

\fi
%++++++++++++++++++++++++++++++++++++++++++++++++++++++++++++
%			FINE CANZONE TRASPOSTA
%++++++++++++++++++++++++++++++++++++++++++++++++++++++++++++

%-------------------------------------------------------------
%			INIZIO	CANZONE
%-------------------------------------------------------------


%titolo: 	Sulla tua parola
%autore: 	Paci Varnavà
%tonalita: 	La- / Re- 



%%%%%% TITOLO E IMPOSTAZONI
\beginsong{Sulla tua parola (Pietro vai)}[by={Paci, Varnavà}]
%\transpose{5}\preferflats						% <<< TRASPOSIZIONE #TONI (0 nullo)
\momenti{Comunione; Fine}							% <<< INSERISCI MOMENTI	
% momenti vanno separati da ; e vanno scelti tra:
% Ingresso; Atto penitenziale; Acclamazione al Vangelo; Dopo il Vangelo; Offertorio; Comunione; Ringraziamento; Fine; Santi; Pasqua; Avvento; Natale; Quaresima; Canti Mariani; Battesimo; Prima Comunione; Cresima; Matrimonio; Meditazione;
\ifchorded
	\textnote{Tonalità migliore}	% <<< EV COMMENTI (tonalità originale/migliore)
\fi


%%%%%% INTRODUZIONE
\ifchorded
\vspace*{\versesep}
\textnote{Intro: \qquad \qquad  }%(\eighthnote 116) % << MODIFICA IL TEMPO
% Metronomo: \eighthnote (ottavo) \quarternote (quarto) \halfnote (due quarti)
\vspace*{-\versesep}
\beginverse*

\nolyrics

%---- Prima riga -----------------------------
\vspace*{-\versesep}
\[A-] \[D-]  \[A-]	 % \[*D] per indicare le pennate, \rep{2} le ripetizioni

%---- Ogni riga successiva -------------------
%\vspace*{-\versesep}
%\[G] \[C]  \[D]	

%---- Ev Indicazioni -------------------------			
%\textnote{\textit{(Oppure tutta la strofa)} }	

\endverse
\fi




%%%%% STROFA
\beginverse		%Oppure \beginverse* se non si vuole il numero di fianco
\memorize 		% <<< DECOMMENTA se si vuole utilizzarne la funzione
%\chordsoff		& <<< DECOMMENTA se vuoi una strofa senza accordi

Si\[A-]gnore, ho pe\[D-]scato tutto il \[A-]giorno,
le \[F]reti son ri\[G]maste sempre \[C]vuote, \[E]
s'è \[D-]fatto \[E]tardi, a \[A-*]casa \[E*]ora ri\[A-]torno,
Si\[F]gnore, son de\[G]luso e me ne \[A-]vado.
\endverse



\beginverse*
La ^vita con ^me è ^sempre stata \[E]dura
e ^niente mai mi ^dà soddisfa^zione, ^
la ^strada in cui mi ^guidi è ^in^si^cura,
sono ^stanco e ora ^non aspetto \[E]più.
\endverse



%%%%% RITORNELLO
\beginchorus
\textnote{\textbf{Rit.}}
\[A]Pietro, \[E]vai, \[F#-]fidati di \[C#-]me,
\[D]getta ancora in \[A]acqua le tue \[E]reti,
\[F#-]prendi ancora il \[C#-]largo, sulla \[D]mia pa\[A]rola,
\[D]con la mia po\[A]tenza, \[B-]io ti fa\[F#-]rò \[D]
pesca\[E]tore di \[D]uomi\[A]ni.
\endchorus




%%%%% STROFA
\beginverse
Ma^estro, dimmi ^cosa devo ^fare,
in^segnami, Si^gnore, dove an^dare, ^
Ge^sù, dammi la ^forza ^di ^par^tire,
la ^forza di la^sciare le mie ^cose:
\endverse



\beginverse*
que^sta fa^miglia che mi ^son cre\[E]ato,
le ^barche che a fa^tica ho conqui^stato, ^
la ^casa, la mia ^terra, la ^mi^a ^gente,
Si^gnore, dammi ^tu una fede \[E]forte.
\endverse


%%%%% RITORNELLO
\beginchorus
\textnote{\textbf{Rit.}}
\[A]Pietro, \[E]vai, \[F#-]fidati di \[C#-]me,
\[D]la mia Chiesa \[A]su te fonde\[E]rò;
\[F#-]manderò lo \[C#-]Spirito, ti \[D]darà co\[A]raggio,
\[D]donerà la \[A]forza \[B-]dell'a\[F#-]mor \[D]
per il \[E]Regno di \[D]Di\[A]o.
\endchorus

\endsong
%------------------------------------------------------------
%			FINE CANZONE
%------------------------------------------------------------





%++++++++++++++++++++++++++++++++++++++++++++++++++++++++++++
%			CANZONE TRASPOSTA
%++++++++++++++++++++++++++++++++++++++++++++++++++++++++++++
\ifchorded
%decremento contatore per avere stesso numero
\addtocounter{songnum}{-1} 
\beginsong{Sulla tua parola (Pietro vai)}[by={P. Paci, S. Varnavà}]
\transpose{5}\prefersharps						% <<< TRASPOSIZIONE #TONI (0 nullo)
%\preferflats SE VOGLIO FORZARE i bemolle come alterazioni
%\prefersharps SE VOGLIO FORZARE i # come alterazioni
\ifchorded
	\textnote{Tonalità originale}	% <<< EV COMMENTI (tonalità originale/migliore)
\fi



%%%%%% INTRODUZIONE
\ifchorded
\vspace*{\versesep}
\textnote{Intro: \qquad \qquad  }%(\eighthnote 116) % << MODIFICA IL TEMPO
% Metronomo: \eighthnote (ottavo) \quarternote (quarto) \halfnote (due quarti)
\vspace*{-\versesep}
\beginverse*

\nolyrics

%---- Prima riga -----------------------------
\vspace*{-\versesep}
\[A-] \[D-]  \[A-]	 % \[*D] per indicare le pennate, \rep{2} le ripetizioni

%---- Ogni riga successiva -------------------
%\vspace*{-\versesep}
%\[G] \[C]  \[D]	

%---- Ev Indicazioni -------------------------			
%\textnote{\textit{(Oppure tutta la strofa)} }	

\endverse
\fi




%%%%% STROFA
\beginverse		%Oppure \beginverse* se non si vuole il numero di fianco
\memorize 		% <<< DECOMMENTA se si vuole utilizzarne la funzione
%\chordsoff		& <<< DECOMMENTA se vuoi una strofa senza accordi

Si\[A-]gnore, ho pe\[D-]scato tutto il \[A-]giorno,
le \[F]reti son ri\[G]maste sempre \[C]vuote, \[E]
s'è \[D-]fatto \[E]tardi, a \[A-*]casa \[E*]ora ri\[A-]torno,
Si\[F]gnore, son de\[G]luso e me ne \[A-]vado.
\endverse



\beginverse*
La ^vita con ^me è ^sempre stata \[E]dura
e ^niente mai mi ^dà soddisfa^zione, ^
la ^strada in cui mi ^guidi è ^in^si^cura,
sono ^stanco e ora ^non aspetto \[E]più.
\endverse



%%%%% RITORNELLO
\beginchorus
\textnote{\textbf{Rit.}}
\[A]Pietro, \[E]vai, \[F#-]fidati di \[C#-]me,
\[D]getta ancora in \[A]acqua le tue \[E]reti,
\[F#-]prendi ancora il \[C#-]largo, sulla \[D]mia pa\[A]rola,
\[D]con la mia po\[A]tenza, \[B-]io ti fa\[F#-]rò \[D]
pesca\[E]tore di \[D]uomi\[A]ni.
\endchorus




%%%%% STROFA
\beginverse
Ma^estro, dimmi ^cosa devo ^fare,
in^segnami, Si^gnore, dove an^dare, ^
Ge^sù, dammi la ^forza ^di ^par^tire,
la ^forza di la^sciare le mie ^cose:
\endverse



\beginverse*
que^sta fa^miglia che mi ^son cre\[E]ato,
le ^barche che a fa^tica ho conqui^stato, ^
la ^casa, la mia ^terra, la ^mi^a ^gente,
Si^gnore, dammi ^tu una fede \[E]forte.
\endverse


%%%%% RITORNELLO
\beginchorus
\textnote{\textbf{Rit.}}
\[A]Pietro, \[E]vai, \[F#-]fidati di \[C#-]me,
\[D]la mia Chiesa \[A]su te fonde\[E]rò;
\[F#-]manderò lo \[C#-]Spirito, ti \[D]darà co\[A]raggio,
\[D]donerà la \[A]forza \[B-]dell'a\[F#-]mor \[D]
per il \[E]Regno di \[D]Di\[A]o.
\endchorus

\endsong

\fi
%++++++++++++++++++++++++++++++++++++++++++++++++++++++++++++
%			FINE CANZONE TRASPOSTA
%++++++++++++++++++++++++++++++++++++++++++++++++++++++++++++

%TTT
%-------------------------------------------------------------
%			INIZIO	CANZONE
%-------------------------------------------------------------


%titolo: 	Terra buona per te
%autore: 	Diliberto, Arzuffi, Bodega, Ruaro
%tonalita: 	Re



%%%%%% TITOLO E IMPOSTAZONI
\beginsong{Terra buona per te}[by={Diliberto, Arzuffi, Bodega, Ruaro}] 	% <<< MODIFICA TITOLO E AUTORE
\transpose{0} 						% <<< TRASPOSIZIONE #TONI (0 nullo)
\momenti{Comunione}			% <<< INSERISCI MOMENTI	
% momenti vanno separati da ; e vanno scelti tra:
% Ingresso; Atto penitenziale; Acclamazione al Vangelo; Dopo il Vangelo; Offertorio; Comunione; Ringraziamento; Fine; Santi; Pasqua; Avvento; Natale; Quaresima; Canti Mariani; Battesimo; Prima Comunione; Cresima; Matrimonio; Meditazione; Spezzare del pane;
\ifchorded
	%\textnote{Tonalità migliore}	% <<< EV COMMENTI (tonalità originale/migliore)
\fi


%%%%%% INTRODUZIONE
\ifchorded
\vspace*{\versesep}
\textnote{Intro: \qquad \qquad }% (\eighthnote 120)} % <<  MODIFICA IL TEMPO
% Metronomo: \eighthnote (ottavo) \quarternote (quarto) \halfnote (due quarti)
\vspace*{-\versesep}
\beginverse*

\nolyrics

%---- Prima riga -----------------------------
\vspace*{-\versesep}
\[D] \[D] \[A]	 \[B-]  % \[*D] per indicare le pennate, \rep{2} le ripetizioni

%---- Ogni riga successiva -------------------
\vspace*{-\versesep}
\[G] \[D] \[A]	 	

%---- Ev Indicazioni -------------------------			
%\textnote{\textit{(Oppure tutta la strofa)} }	

\endverse
\fi




%%%%% RITORNELLO
\beginchorus
%\textnote{\textbf{Rit.}}

\[D]Vieni, vieni nel mio \[A]campo
\[G]c’è terra \[D]buona per \[G]te. \[A*]
\[D]Vieni e getta ancora il \[A]se\[B-]me
\[G]che cerche\[D]rà la \[A]lu\[B-]ce, 
\[G]germoglie\[D]rà e darà \[A]frut\[(D*)]to.
\endchorus


%%%%%% EV. INTERMEZZO
\beginverse*
\vspace*{1.3\versesep}
{
	\nolyrics
	\textnote{Breve intermezzo}
	
	\ifchorded

	%---- Prima riga -----------------------------
	\vspace*{-\versesep}
	\[D] \[D4] \[D] \rep{2} 


	\fi
	%---- Ev Indicazioni -------------------------			
	%\textnote{\textit{(ripetizione della strofa)}} 
	 
}
\vspace*{\versesep}
\endverse


%%%%% STROFA
\beginverse		%Oppure \beginverse* se non si vuole il numero di fianco
\memorize 		% <<< DECOMMENTA se si vuole utilizzarne la funzione
%\chordsoff		% <<< DECOMMENTA se vuoi una strofa senza accordi

Non \[B-]l’ansia, \[D]non la volon\[G]tà di a\[A]vere \[A]
potranno \[D]mai riem\[G]pire questa \[A]vita.
Ac\[G]colgo la pa\[A]rola che mi in\[F#-]vita \[B-]
a \[E-]vivere per gli \[G*]altri o\[D*]gni mio \[A]giorno,
\[F#*]co\[B-]sì potrò sco\[B-]prire vera\[G]mente,
\[G]moltipli\[D]cati, \[G]i doni \[E]rice\[A]vuti.

\endverse

%%%%% RITORNELLO
\beginchorus
%\textnote{\textbf{Rit.}}

\[D]Vieni, vieni nel mio \[A]campo
\[G]c’è terra \[D]buona per \[G]te. \[A*]
\[D]Vieni e getta ancora il \[A]se\[B-]me
\[G]che cerche\[D]rà la \[A]lu\[B-]ce, 
\[G]germoglie\[D]rà e darà \[A]frut\[(D*)]to.
\endchorus


%%%%%% EV. INTERMEZZO
\beginverse*
\vspace*{1.3\versesep}
{
	\nolyrics
	\textnote{Breve intermezzo}
	
	\ifchorded

	%---- Prima riga -----------------------------
	\vspace*{-\versesep}
	\[D] \[D4] \[D] \rep{2} 


	\fi
	%---- Ev Indicazioni -------------------------			
	%\textnote{\textit{(ripetizione della strofa)}} 
	 
}
\vspace*{\versesep}
\endverse


%%%%% STROFA
\beginverse		%Oppure \beginverse* se non si vuole il numero di fianco
%\memorize 		% <<< DECOMMENTA se si vuole utilizzarne la funzione
%\chordsoff		% <<< DECOMMENTA se vuoi una strofa senza accordi

E ^forte, ^da radici or^mai pro^fonde ^
si allarghe^rà la ^pianta nel ter^reno.
Con^templo nel tuo ^volto l’infi^nito, ^
ma ^so che quoti^diana è ^la fa^tica. 
^Di^fendimi, Si^gnore, e il mio cam^mino
^incontro al ^mondo ^sarà di ^nuovo ^festa.

\endverse



%%%%% RITORNELLO
\beginchorus
%\textnote{\textbf{Rit.}}

\[D]Vieni, vieni nel mio \[A]campo
\[G]c’è terra \[D]buona per \[G]te. \[A*]
\[D]Vieni e getta ancora il \[A]se\[B-]me
\[G]che cerche\[D]rà la \[A]lu\[B-]ce, 
\[G]germoglie\[D]rà e darà \[A]frut\[(D*)]to.
\endchorus


%%%%%% EV. INTERMEZZO
\beginverse*
\vspace*{1.3\versesep}
{
	\nolyrics
	\textnote{Breve intermezzo}
	
	\ifchorded

	%---- Prima riga -----------------------------
	\vspace*{-\versesep}
	\[D] \[D4] \[D] \rep{2} 


	\fi
	%---- Ev Indicazioni -------------------------			
	%\textnote{\textit{(ripetizione della strofa)}} 
	 
}
\vspace*{\versesep}
\endverse



%%%%% STROFA
\beginverse		%Oppure \beginverse* se non si vuole il numero di fianco
%\memorize 		% <<< DECOMMENTA se si vuole utilizzarne la funzione
%\chordsoff		% <<< DECOMMENTA se vuoi una strofa senza accordi

Per ^sempre ^questo tuo res^tarmi ac^canto ^
dilate^rà la ^gioia e la fi^ducia.
Com’^è davvero ^grande la pa^zienza ^
di ^Dio che semi^nando ^non si ^stanca!
^E ^tu, compagno ^nel mio andare in^torno,
^spalanca il ^cielo ^a questa ^voce at^tesa.

\endverse


%%%%% RITORNELLO
\beginchorus
%\textnote{\textbf{Rit.}}

\[D]Vieni, vieni nel mio \[A]campo
\[G]c’è terra \[D]buona per \[G]te. \[A*]
\[D]Vieni e getta ancora il \[A]se\[B-]me
\[G]che cerche\[D]rà la \[A]lu\[B-]ce, 
\[G]germoglie\[D]rà e darà \[A]frut \textit{(to)}
\[D]Vieni, vieni nel mio \[A]campo
\[G]c’è terra \[D]buona per \[G]te. \[A*]
\[D]Vieni e getta ancora il \[A]se\[B-]me
\[G]che cerche\[D]rà la \[A]lu\[B-]ce, 
\[G]germoglie\[D]rà e darà \[A]frut\[(D*)]to.
\endchorus


%%%%%% EV. INTERMEZZO
\beginverse*
\vspace*{1.3\versesep}
{
	\nolyrics
	\textnote{Finale strumentale}
	
	\ifchorded

	%---- Prima riga -----------------------------
	\vspace*{-\versesep}
	\[D] \[D4] \[D] 

	%---- Ogni riga successiva -------------------
	\vspace*{-\versesep}
	\[D] \[D4] \[D*] 	
	\fi
	%---- Ev Indicazioni -------------------------			
	%\textnote{\textit{(ripetizione della strofa)}} 
	 
}
\vspace*{\versesep}
\endverse


\endsong
%------------------------------------------------------------
%			FINE CANZONE
%------------------------------------------------------------



%-------------------------------------------------------------
%			INIZIO	CANZONE
%-------------------------------------------------------------


%titolo: 	Ti chiedo perdono
%autore: 	P. Sequeri
%tonalita: 	Sol 



%%%%%% TITOLO E IMPOSTAZONI
\beginsong{Ti chiedo perdono}[by={P. Sequeri}] 	% <<< MODIFICA TITOLO E AUTORE
\transpose{0} 						% <<< TRASPOSIZIONE #TONI (0 nullo)
\momenti{Atto penitenziale}							% <<< INSERISCI MOMENTI	
% momenti vanno separati da ; e vanno scelti tra:
% Ingresso; Atto penitenziale; Acclamazione al Vangelo; Dopo il Vangelo; Offertorio; Comunione; Ringraziamento; Fine; Santi; Pasqua; Avvento; Natale; Quaresima; Canti Mariani; Battesimo; Prima Comunione; Cresima; Matrimonio; Meditazione; Spezzare del pane;
\ifchorded
	%\textnote{Tonalità originale }	% <<< EV COMMENTI (tonalità originale/migliore)
\fi


%%%%%% INTRODUZIONE
\ifchorded
\vspace*{\versesep}
\musicnote{
\begin{minipage}{0.48\textwidth}
\textbf{Intro}
\hfill 
%( \eighthnote \, 80)   % <<  MODIFICA IL TEMPO
% Metronomo: \eighthnote (ottavo) \quarternote (quarto) \halfnote (due quarti)
\end{minipage}
} 	
\vspace*{-\versesep}
\beginverse*

\nolyrics

%---- Prima riga -----------------------------
\vspace*{-\versesep}
\[B-] \[C#dim/A]	 \[B-] \rep{2} % \[*D] per indicare le pennate, \rep{2} le ripetizioni

%---- Ogni riga successiva -------------------
%\vspace*{-\versesep}
%\[G] \[C]  \[D]	

%---- Ev Indicazioni -------------------------			
%\textnote{\textit{(Oppure tutta la strofa)} }	

\endverse
\fi




%%%%% STROFA
\beginverse*		%Oppure \beginverse* se non si vuole il numero di fianco
%\memorize 		% <<< DECOMMENTA se si vuole utilizzarne la funzione
%\chordsoff		% <<< DECOMMENTA se vuoi una strofa senza accordi


Ti \[B-]chiedo per\[C#dim/A]dono, Padre \[B-]buono, \[D]
per \[G]ogni man\[F#]canza d'a\[B-]more, \[G*] \[F#*]
\[B-]per la mia \[C#dim/A]debole spe\[B-]ranza \[D]
e \[G]per la mia \[F#]fragile \[B-]fede. \[G*] \[A*]

\endverse


%%%%% STROFA
\beginverse*		%Oppure \beginverse* se non si vuole il numero di fianco
%\memorize 		% <<< DECOMMENTA se si vuole utilizzarne la funzione
%\chordsoff		% <<< DECOMMENTA se vuoi una strofa senza accordi

Do\[D]mando a \[G]Te, Si\[D]gnore, \[A]
che il\[G]lumi\[A]ni i miei \[D]passi, \[F#-]
la \[G]forza di \[G-]vivere,
con \[D]tutti i miei fra\[G]telli, \[G-]
nuova\[D]mente fe\[A9]dele al Tuo van\[D*]gelo.\[G*] \[D]

\endverse

\endsong
%------------------------------------------------------------
%			FINE CANZONE
%------------------------------------------------------------


%-------------------------------------------------------------
%			INIZIO	CANZONE
%-------------------------------------------------------------


%titolo: 	Ti do La pace
%autore: 	SERMIG
%tonalita: 	Fa



%%%%%% TITOLO E IMPOSTAZONI
\beginsong{Ti do La pace}[by={Sermig}] 	% <<< MODIFICA TITOLO E AUTORE
\transpose{0} 						% <<< TRASPOSIZIONE #TONI (0 nullo)
%\preferflats  %SE VOGLIO FORZARE i bemolle come alterazioni
%\prefersharps %SE VOGLIO FORZARE i # come alterazioni
\momenti{}							% <<< INSERISCI MOMENTI	
% momenti vanno separati da ; e vanno scelti tra:
% Ingresso; Atto penitenziale; Acclamazione al Vangelo; Dopo il Vangelo; Offertorio; Comunione; Ringraziamento; Fine; Santi; Pasqua; Avvento; Natale; Quaresima; Canti Mariani; Battesimo; Prima Comunione; Cresima; Matrimonio; Meditazione; Spezzare del pane;
\ifchorded
	%\textnote{$\bigstar$ Tonalità migliore }	% <<< EV COMMENTI (tonalità originale\migliore)
\fi


%%%%%% INTRODUZIONE
\ifchorded
\vspace*{\versesep}
\musicnote{
\begin{minipage}{0.48\textwidth}
\textbf{Intro}
\hfill 
%( \eighthnote \, 80)   % <<  MODIFICA IL TEMPO
% Metronomo: \eighthnote (ottavo) \quarternote (quarto) \halfnote (due quarti)
\end{minipage}
} 	
\vspace*{-\versesep}
\beginverse*

\nolyrics

%---- Prima riga -----------------------------
\vspace*{-\versesep}
\[B&] \[C] \[F]	 % \[*D] per indicare le pennate, \rep{2} le ripetizioni

%---- Ogni riga successiva -------------------
%\vspace*{-\versesep}
%\[G] \[C]  \[D]	

%---- Ev Indicazioni -------------------------			
%\textnote{\textit{[oppure tutta la strofa]} }	

\endverse
\fi




%%%%% STROFA
\beginverse		%Oppure \beginverse* se non si vuole il numero di fianco
\memorize 		% <<< DECOMMENTA se si vuole utilizzarne la funzione
%\chordsoff		% <<< DECOMMENTA se vuoi una strofa senza accordi

Ti do la \[B&]pace \[C]perché ci cre\[F]do 
ti do la \[B&]pace \[C]perché la viv\[F]o 
ti do la \[D-]pace \[C]perché la vogli\[B&]o
per t\[G-]e e per tutte le don\[A4]ne 
e tu\[G-]tti gli uomini del \[A]mondo 
ti do la \[B&]pace \[C]perché ci cred\[F]o.

\endverse


%%%%% STROFA
\beginverse		%Oppure \beginverse* se non si vuole il numero di fianco
\memorize 		% <<< DECOMMENTA se si vuole utilizzarne la funzione
%\chordsoff		% <<< DECOMMENTA se vuoi una strofa senza accordi

Ti do la \[B&]pace \[C]perché ci cred\[F]o 
ti do la \[B&]pace \[C]perché la viv\[F]o 
ti do la p\[D-]ace \[C]perché la vo\[B&]glio 
per \[G-]te e per tutte le \[A4]donne
e \[G-]tutti gli uomini del \[A4]mondo


\endverse




%%%%% STROFA
\beginverse		%Oppure \beginverse* se non si vuole il numero di fianco
%\memorize 		% <<< DECOMMENTA se si vuole utilizzarne la funzione
%\chordsoff		% <<< DECOMMENTA se vuoi una strofa senza accordi
Ti do la \[B&]pace \[C]perché io spero ch\[F]e 
la pace \[B&]possa \[C]abitare semp\[F]re 
\[D-]e nel creato e in \[C]tutte le crea\[B&]ture.

\endverse


%%%%% STROFA
\beginverse		%Oppure \beginverse* se non si vuole il numero di fianco
%\memorize 		% <<< DECOMMENTA se si vuole utilizzarne la funzione
%\chordsoff		% <<< DECOMMENTA se vuoi una strofa senza accordi

Ti do la \[B&]pace \[C]perché ci cred\[F]o, 
ti do la \[B&]pace \[C]perché ci cred\[F]o, 
ti do la \[B&]pace, \[C]la voglio anche per \[F]te.  


\endverse



\endsong
%------------------------------------------------------------
%			FINE CANZONE
%------------------------------------------------------------


%-------------------------------------------------------------
%			INIZIO	CANZONE
%-------------------------------------------------------------


%titolo: 	Ti ringrazio mi Signore
%autore: 	P. Sequeri
%tonalita: 	Re



%%%%%% TITOLO E IMPOSTAZONI
\beginsong{Ti ringrazio mio Signore}[by={P. Sequeri}] 	% <<< MODIFICA TITOLO E AUTORE
\transpose{0} 						% <<< TRASPOSIZIONE #TONI (0 nullo)
%\preferflats  %SE VOGLIO FORZARE i bemolle come alterazioni
%\prefersharps %SE VOGLIO FORZARE i # come alterazioni
\momenti{Congedo; Prima Comunione}							% <<< INSERISCI MOMENTI	
% momenti vanno separati da ; e vanno scelti tra:
% Ingresso; Atto penitenziale; Acclamazione al Vangelo; Dopo il Vangelo; Offertorio; 
%Comunione; Ringraziamento; Fine; Santi; Pasqua; Avvento; Natale; Quaresima; Canti Mariani; 
%Battesimo; Prima Comunione; Cresima; Matrimonio; Meditazione; Spezzare del pane;
\ifchorded
	%\textnote{Tonalità migliore }	% <<< EV COMMENTI (tonalità originale/migliore)
\fi

%%%%%% INTRODUZIONE
\ifchorded
\vspace*{\versesep}
\musicnote{
\begin{minipage}{0.48\textwidth}
\textbf{Intro}
\hfill 
( \quarternote \, 120)   % <<  MODIFICA IL TEMPO
% Metronomo: \eighthnote (ottavo) \quarternote (quarto) \halfnote (due quarti)
\end{minipage}
} 	
\vspace*{-\versesep}
\beginverse*

\nolyrics

%---- Prima riga -----------------------------
\vspace*{-\versesep}
\[D] \[G] \[A7]	 \rep{2}% \[*D] per indicare le pennate, \rep{2} le ripetizioni

%---- Ogni riga successiva -------------------
%\vspace*{-\versesep}
%\[G] \[C]  \[D]	

%---- Ev Indicazioni -------------------------			
%\textnote{\textit{(Oppure tutta la strofa)} }	

\endverse
\fi

%%%%% STROFA
\beginverse		%Oppure \beginverse* se non si vuole il numero di fianco
\memorize 		% <<< DECOMMENTA se si vuole utilizzarne la funzione
%\chordsoff		% <<< DECOMMENTA se vuoi una strofa senza accordi

A\[D]matevi l’un l’\[B-]altro
come \[E-]Lui ha amato \[A]noi:
e si\[D]ate per \[B-]sempre suoi a\[G]mi\[D]ci;
e \[E-]quello che fa\[A]rete
al più \[F#]piccolo tra \[B-]voi,
cre\[E-]dete, l’a\[A]vete fatto a \[D*]Lui. \[G*] \[D*]

\endverse

%%%%% RITORNELLO
\beginchorus
\textnote{\textbf{Rit.}}

Ti ring\[D]razio \[B-]mio Sig\[G]nore \brk e \[A7]non ho più pa\[D]ura, \[B-]
per\[G]ché, \[A7]con la mia \[D]mano
\[B-]nella \[G]mano \[A7]degli amici \[D7]miei, \[B-]
cam\[G]mino tra la \[A7]gente della \[D*]mia \[F#*]cit\[B-]tà
e \[G]non mi \[A]sento più \[D7]solo; \[B-]
non \[G]sento la stan\[A7]chezza
e guardo \[D*]dritto \[F#*]avanti a \[B-]me,
per\[G]ché sulla mia \[A7]strada ci sei \[D*]tu. \[G*] \[D*]

\endchorus

%%%%% STROFA
\beginverse		%Oppure \beginverse* se non si vuole il numero di fianco
%\memorize 		% <<< DECOMMENTA se si vuole utilizzarne la funzione
%\chordsoff		% <<< DECOMMENTA se vuoi una strofa senza accordi

Se a^mate vera^mente perdo^natevi tra ^voi,
nel ^cuore di og^nuno ci sia ^pa^ce,
il ^Padre che dai ^cieli vede ^tutti i figli ^suoi
con ^gioia a ^voi perdone^rà. ^ ^

\endverse

%%%%% STROFA
\beginverse		%Oppure \beginverse* se non si vuole il numero di fianco
%\memorize 		% <<< DECOMMENTA se si vuole utilizzarne la funzione
%\chordsoff		% <<< DECOMMENTA se vuoi una strofa senza accordi

Sa^rete suoi a^mici se vi ^amate tra di ^voi
e ^questo è ^tutto il suo Van^ge^lo
l’ a^more non ha ^prezzo, non mi^sura ciò che ^dà
l’a^more con^fini non ne ^ha. ^ ^

\endverse

\endsong
%------------------------------------------------------------
%			FINE CANZONE
%------------------------------------------------------------
%-------------------------------------------------------------
%			INIZIO	CANZONE
%-------------------------------------------------------------


%titolo: 	Ti seguirò
%autore: 	M. Frisina
%tonalita: 	Do



%%%%%% TITOLO E IMPOSTAZONI
\beginsong{Ti seguirò}[by={M. Frisina}] 	% <<< MODIFICA TITOLO E AUTORE
\transpose{0} 						% <<< TRASPOSIZIONE #TONI (0 nullo)
%\preferflats  %SE VOGLIO FORZARE i bemolle come alterazioni
%\prefersharps %SE VOGLIO FORZARE i # come alterazioni
\momenti{Quaresima}							% <<< INSERISCI MOMENTI	
% momenti vanno separati da ; e vanno scelti tra:
% Ingresso; Atto penitenziale; Acclamazione al Vangelo; Dopo il Vangelo; Offertorio; Comunione; Ringraziamento; Fine; Santi; Pasqua; Avvento; Natale; Quaresima; Canti Mariani; Battesimo; Prima Comunione; Cresima; Matrimonio; Meditazione; Spezzare del pane;
\ifchorded
	%\textnote{Tonalità migliore }	% <<< EV COMMENTI (tonalità originale/migliore)
\fi

%%%%%% INTRODUZIONE
\ifchorded
\vspace*{\versesep}
\musicnote{
\begin{minipage}{0.48\textwidth}
\textbf{Intro}
\hfill 
%( \eighthnote \, 80)   % <<  MODIFICA IL TEMPO
% Metronomo: \eighthnote (ottavo) \quarternote (quarto) \halfnote (due quarti)
\end{minipage}
} 	
\vspace*{-\versesep}
\beginverse*

\nolyrics

%---- Prima riga -----------------------------
\vspace*{-\versesep}
\[C] 	 % \[*D] per indicare le pennate, \rep{2} le ripetizioni

%---- Ogni riga successiva -------------------
%\vspace*{-\versesep}
%\[G] \[C]  \[D]	

%---- Ev Indicazioni -------------------------			
\textnote{\textit{(oppure tutto il ritornello)} }	

\endverse
\fi










%%%%% RITORNELLO
\beginchorus
\textnote{\textbf{Rit.}}

\[C]Ti segui\[G]rò, ti \[A-]seguirò, o Si\[F]gnore
\[C]e \[G]nella \[E]tua \[A-]strada \[F]cammine\[C]rò.

\endchorus




%%%%% STROFA
\beginverse		%Oppure \beginverse* se non si vuole il numero di fianco
%\memorize 		% <<< DECOMMENTA se si vuole utilizzarne la funzione
%\chordsoff		% <<< DECOMMENTA se vuoi una strofa senza accordi

\[C]Ti segui\[G]rò nella \[A-]via dell'a\[F]more
\[C]e \[G]done\[E]rò al \[A-]mondo \[F]la vi\[C]ta.

\endverse




%%%%% STROFA
\beginverse		%Oppure \beginverse* se non si vuole il numero di fianco
%\memorize 		% <<< DECOMMENTA se si vuole utilizzarne la funzione
%\chordsoff		% <<< DECOMMENTA se vuoi una strofa senza accordi

\[C]Ti segui\[G]rò nella \[A-]via del do\[F]lore
\[C]e \[G]la \[E]tua \[A-]croce \[F]ci salve\[C]rà.

\endverse





%%%%% STROFA
\beginverse		%Oppure \beginverse* se non si vuole il numero di fianco
%\memorize 		% <<< DECOMMENTA se si vuole utilizzarne la funzione
%\chordsoff		% <<< DECOMMENTA se vuoi una strofa senza accordi

\[C]Ti segui\[G]rò nella \[A-]via della \[F]gioia
\[C]e \[G]la \[E]tua \[A-]luce \[F]ci guide\[C]rà.

\endverse




\endsong
%------------------------------------------------------------
%			FINE CANZONE
%------------------------------------------------------------


%-------------------------------------------------------------
%			INIZIO	CANZONE
%-------------------------------------------------------------


%titolo: 	Tu sarai profeta
%autore: 	M. Frisina
%tonalita: 	Re 



%%%%%% TITOLO E IMPOSTAZONI
\beginsong{Tu sarai profeta}[by={M. Frisina}] 	% <<< MODIFICA TITOLO E AUTORE
\transpose{0} 						% <<< TRASPOSIZIONE #TONI (0 nullo)
%\preferflats  %SE VOGLIO FORZARE i bemolle come alterazioni
%\prefersharps %SE VOGLIO FORZARE i # come alterazioni
\momenti{}							% <<< INSERISCI MOMENTI	
% momenti vanno separati da ; e vanno scelti tra:
% Ingresso; Atto penitenziale; Acclamazione al Vangelo; Dopo il Vangelo; Offertorio; Comunione; Ringraziamento; Fine; Santi; Pasqua; Avvento; Natale; Quaresima; Canti Mariani; Battesimo; Prima Comunione; Cresima; Matrimonio; Meditazione; Spezzare del pane;
\ifchorded
	%\textnote{Tonalità migliore }	% <<< EV COMMENTI (tonalità originale\migliore)
\fi


%%%%%% INTRODUZIONE
\ifchorded
\vspace*{\versesep}
\textnote{Intro: \qquad \qquad  }%(\eighthnote 116) % <<  MODIFICA IL TEMPO
% Metronomo: \eighthnote (ottavo) \quarternote (quarto) \halfnote (due quarti)
\vspace*{-\versesep}
\beginverse*

\nolyrics

%---- Prima riga -----------------------------
\vspace*{-\versesep}
\[D] \[A] \[B-]	 \[F#-] % \[*D] per indicare le pennate, \rep{2} le ripetizioni

%---- Ogni riga successiva -------------------
\vspace*{-\versesep}
\[G] \[D]  \[A] \[A]	

%---- Ev Indicazioni -------------------------			
%\textnote{\textit{(Oppure tutta la strofa)} }	

\endverse
\fi




%%%%% STROFA
\beginverse		%Oppure \beginverse* se non si vuole il numero di fianco
\memorize 		% <<< DECOMMENTA se si vuole utilizzarne la funzione
%\chordsoff		% <<< DECOMMENTA se vuoi una strofa senza accordi

\[D]Una \[A]luce che ri\[B-]schia\[F#-]ra,
\[D]una \[G]lampada che \[D]ar\[A]de,
\[B-*]u\[G*]na \[G]voce che pro\[A]cla\[B-]ma
\[E-]la Pa\[G]rola di sal\[A4]vezza.
\[D]Precur\[A]sore nella \[B-]gio\[F#-]ia,
\[D]precu\[G]rsore nel \[D]dolo\[A]re,
\[B-*]tu \[G*]che \[G]sveli nel \[A]perdo\[B-]no
\[E-]l'annunzio \[G]di miseri\[A]cord\[D]ia.

\endverse




%%%%% RITORNELLO
\beginchorus
\textnote{\textbf{Rit.}}

Tu sa\[G]ra\[D]i pro\[E-]feta di salve\[B-]zza
\[G]fino ai con\[D]fini della \[A4]terra,
porte\[G]ra\[D]i la \[E-]mia Pa\[B-]rola,
\[G]risplende\[D]rai della mia \[A]lu\[D]ce.

\endchorus



%%%%% STROFA
\beginverse		%Oppure \beginverse* se non si vuole il numero di fianco
%\memorize 		% <<< DECOMMENTA se si vuole utilizzarne la funzione
%\chordsoff		% <<< DECOMMENTA se vuoi una strofa senza accordi

\[D]Forte \[A]amico dello \[B-]Spo\[F#-]so,
\[D]che \[G]gioisci alla sua \[D]vo\[A]ce,
\[B-*]tu \[G*]cam\[G]mini per il \[A]mon\[B-]do
\[E-]per pre\[G]cedere il \[A4]Signore.
\[D]Stende\[A]rò la mia \[B-]ma\[F#-]no
\[D]e por\[G]rò sulla tua \[D]boc\[A]ca
\[B-*]la \[G*]po\[G]tente mia Pa\[A]ro\[B-]la
\[E-]che con\[G]vertirà il \[A]mon\[D]do.

\endverse



\endsong
%------------------------------------------------------------
%			FINE CANZONE
%------------------------------------------------------------



%-------------------------------------------------------------
%			INIZIO	CANZONE
%-------------------------------------------------------------


%titolo: 	Symbolum 77 (Tu sei la mia vita)
%autore: 	Pierangelo Sequeri
%tonalita: 	Mi- 



%%%%%% TITOLO E IMPOSTAZONI
\beginsong{Tu sei la mia vita}[ititle={Symbolum 77}, by={Symbolum 77 — P. Sequeri}] 	% <<< MODIFICA TITOLO E AUTORE
\transpose{0} 						% <<< TRASPOSIZIONE #TONI (0 nullo)
%\preferflats  %SE VOGLIO FORZARE i bemolle come alterazioni
%\prefersharps %SE VOGLIO FORZARE i # come alterazioni
\momenti{Comunione}							% <<< INSERISCI MOMENTI	
% momenti vanno separati da ; e vanno scelti tra:
% Ingresso; Atto penitenziale; Acclamazione al Vangelo; Dopo il Vangelo; Offertorio; Comunione; Ringraziamento; Fine; Santi; Pasqua; Avvento; Natale; Quaresima; Canti Mariani; Battesimo; Prima Comunione; Cresima; Matrimonio; Meditazione; Spezzare del pane;
\ifchorded
	%\textnote{Tonalità migliore }	% <<< EV COMMENTI (tonalità originale/migliore)
\fi


%%%%%% INTRODUZIONE
\ifchorded
\vspace*{\versesep}
\musicnote{
\begin{minipage}{0.48\textwidth}
\textbf{Intro}
\hfill 
%( \eighthnote \, 80)   % <<  MODIFICA IL TEMPO
% Metronomo: \eighthnote (ottavo) \quarternote (quarto) \halfnote (due quarti)
\end{minipage}
} 	
\vspace*{-\versesep}
\beginverse*

\nolyrics

%---- Prima riga -----------------------------
\vspace*{-\versesep}
\[E-] \[C] \[D] \[E-]\[E-]	 % \[*D] per indicare le pennate, \rep{2} le ripetizioni

%---- Ogni riga successiva -------------------
%\vspace*{-\versesep}
%\[G] \[C]  \[D]	

%---- Ev Indicazioni -------------------------			
%\textnote{\textit{(Oppure tutta la strofa)} }	

\endverse
\fi

%%%%% STROFA
\beginverse		%Oppure \beginverse* se non si vuole il numero di fianco
\memorize 		% <<< DECOMMENTA se si vuole utilizzarne la funzione
%\chordsoff		% <<< DECOMMENTA se vuoi una strofa senza accordi

\[E-]Tu sei la mia \[C]vita \[D]altro io non \[G]ho.
\[E-]Tu sei la mia \[C]strada, \[D]la mia veri\[B7]tà.
\[A-]Nella tua pa\[D7]rola \[G]io cammine\[E-]rò
\[C]finché avrò res\[D]piro, fino a \[G]quando Tu vorr\[B7]ai:
\[A-]non avrò pa\[D7]ura sai, \[G]se Tu sei con \[E-]me,
\[C]io ti prego \[D]resta con \[E-]me.

\endverse

%%%%% STROFA
\beginverse		%Oppure \beginverse* se non si vuole il numero di fianco
\memorize 		% <<< DECOMMENTA se si vuole utilizzarne la funzione
%\chordsoff		% <<< DECOMMENTA se vuoi una strofa senza accordi

^Credo in Te, Si^gnore, ^nato da Mar^ia,
^Figlio eterno e ^santo, ^uomo come ^noi,
^morto per a^more, ^vivo in mezzo a ^noi:
^una cosa ^sola con il ^Padre e con i ^tuoi,
^fino a quando, ^io lo so, ^Tu ritorner^ai
^per aprirci il ^regno di ^Dio.

\endverse

%%%%% STROFA
\beginverse		%Oppure \beginverse* se non si vuole il numero di fianco
\memorize 		% <<< DECOMMENTA se si vuole utilizzarne la funzione
%\chordsoff		% <<< DECOMMENTA se vuoi una strofa senza accordi

^Tu sei la mia ^forza, ^altro io non ^ho. 
^Tu sei la mia ^pace, ^la mia liber^tà. 
^Niente nella ^vita ^ci separe^rà,
^so che la tua ^mano forte ^non mi lasce^rà. 
^So che da ogni ^male Tu ^mi libere^rai 
^e nel tuo ^perdono viv^rò.

\endverse

%%%%% STROFA
\beginverse		%Oppure \beginverse* se non si vuole il numero di fianco
\memorize 		% <<< DECOMMENTA se si vuole utilizzarne la funzione
%\chordsoff		% <<< DECOMMENTA se vuoi una strofa senza accordi

^Padre della ^vita ^noi crediamo in ^Te. 
^Figlio Salva^tore ^noi speriamo in ^Te. 
^Spirito d'a^more ^vieni in mezzo a ^noi,
^Tu da mille ^strade ci ra^duni in uni^tà 
^e per mille ^strade poi, ^dove Tu vor^rai,
^noi saremo il ^seme di ^Dio.

\endverse

\endsong
%------------------------------------------------------------
%			FINE CANZONE
%------------------------------------------------------------
%UUU
%-------------------------------------------------------------
%			INIZIO	CANZONE
%-------------------------------------------------------------


%titolo: 	Un solo spirito
%autore: 	P. Sequeri
%tonalita: 	Mi 



%%%%%% TITOLO E IMPOSTAZONI
\beginsong{Un solo Spirito}[by={P. Sequeri}] 	% <<< MODIFICA TITOLO E AUTORE
\transpose{-2} 						% <<< TRASPOSIZIONE #TONI (0 nullo)

\momenti{Ingresso; Cresima; Battesimo; }							% <<< INSERISCI MOMENTI	
% momenti vanno separati da ; e vanno scelti tra:
% Ingresso; Atto penitenziale; Acclamazione al Vangelo; Dopo il Vangelo; Offertorio; Comunione; Ringraziamento; Fine; Santi; Pasqua; Avvento; Natale; Quaresima; Canti Mariani; Battesimo; Prima Comunione; Cresima; Matrimonio; Meditazione;
\ifchorded
	\textnote{$\triangle$ Tonalità alternativa }	% <<< EV COMMENTI (tonalità originale/migliore)
\fi


%%%%%% INTRODUZIONE
\ifchorded
\vspace*{\versesep}
\musicnote{
\begin{minipage}{0.48\textwidth}
\textbf{Intro}
\hfill 
%( \eighthnote \, 80)   % <<  MODIFICA IL TEMPO
% Metronomo: \eighthnote (ottavo) \quarternote (quarto) \halfnote (due quarti)
\end{minipage}
} 	
\vspace*{-\versesep}
\beginverse*

\nolyrics

%---- Prima riga -----------------------------
\vspace*{-\versesep}
\[E] \[A] \[F#-] \[B7]	 % \[*D] per indicare le pennate, \rep{2} le ripetizioni

%---- Ogni riga successiva -------------------
%\vspace*{-\versesep}
%\[G] \[C]  \[D]	

%---- Ev Indicazioni -------------------------			
%\textnote{\textit{(Oppure tutta la strofa)} }	

\endverse
\fi


%%%%% RITORNELLO
\beginchorus
\textnote{\textbf{Rit.}}

Un \[E]solo \[A]Spirito, un \[F#-]solo Bat\[B7]tesimo, 
un \[E]solo Si\[G#7]gnore: Ge\[C#]sù!
Nel \[F#-]segno dell'a\[B]more tu \[G#]sei con \[A]noi, 
nel \[F#-]nome tuo vi\[B]viamo fra\[G#]telli: \[C#7]
nel \[F#-]cuore la spe\[B]ranza che \[G#]tu ci \[A]dai, 
la \[F#-]fede che ci u\[B7]nisce can\[E]tiamo!
\endchorus


%%%%% STROFA
\beginverse		%Oppure \beginverse* se non si vuole il numero di fianco
%\memorize 		% <<< DECOMMENTA se si vuole utilizzarne la funzione
%\chordsoff		% <<< DECOMMENTA se vuoi una strofa senza accordi
\preferflats  %SE VOGLIO FORZARE i bemolle come alterazioni
Io \[E-]sono la vite e \[A-]voi siete i t\[E-]ralci miei:
il \[C]tralcio che in me non \[D]vive sfiori\[G]rà; \[B7] 
ma \[E-]se rimanete in \[A-]me il \[C]Padre mio vi da\[G]rà 
la \[F]forza di una \[B&dim/(A)]vita che non \[B7]muore mai. 

\endverse

%%%%% STROFA
\beginverse		%Oppure \beginverse* se non si vuole il numero di fianco
%\memorize 		% <<< DECOMMENTA se si vuole utilizzarne la funzione
\chordsoff		% <<< DECOMMENTA se vuoi una strofa senza accordi

Io sono la vera via e la verità:
amici vi chiamo sempre sto con voi;
chi annuncia al fratello suo la fede nel nome mio
davanti al Padre io lo riconoscerò. 


\endverse

%%%%% STROFA
\beginverse		%Oppure \beginverse* se non si vuole il numero di fianco
%\memorize 		% <<< DECOMMENTA se si vuole utilizzarne la funzione
\chordsoff		% <<< DECOMMENTA se vuoi una strofa senza accordi
 
Lo Spirito Santo in voi parlerà di me;
dovunque c'è un uomo al mondo sono io;
ognuno che crede in me fratello vostro sarà
nel segno del Battesimo rinascerà. 

\endverse





\endsong
%------------------------------------------------------------
%			FINE CANZONE
%------------------------------------------------------------

%++++++++++++++++++++++++++++++++++++++++++++++++++++++++++++
%			CANZONE TRASPOSTA
%++++++++++++++++++++++++++++++++++++++++++++++++++++++++++++
\ifchorded
%decremento contatore per avere stesso numero
\addtocounter{songnum}{-1} 
\beginsong{Un solo Spirito}[by={P. Sequeri}] 	% <<< COPIA TITOLO E AUTORE
\transpose{0} 						% <<< TRASPOSIZIONE #TONI + - (0 nullo)
%\preferflats  %SE VOGLIO FORZARE i bemolle come alterazioni
%\prefersharps %SE VOGLIO FORZARE i # come alterazioni
\ifchorded
	\textnote{$\bigstar$ Tonalità originale}	% <<< EV COMMENTI (tonalità originale/migliore)
\fi



%%%%%% INTRODUZIONE
\ifchorded
\vspace*{\versesep}
\musicnote{
\begin{minipage}{0.48\textwidth}
\textbf{Intro}
\hfill 
%( \eighthnote \, 80)   % <<  MODIFICA IL TEMPO
% Metronomo: \eighthnote (ottavo) \quarternote (quarto) \halfnote (due quarti)
\end{minipage}
} 	
\vspace*{-\versesep}
\beginverse*

\nolyrics

%---- Prima riga -----------------------------
\vspace*{-\versesep}
\[E] \[A] \[F#-] \[B7]	 % \[*D] per indicare le pennate, \rep{2} le ripetizioni

%---- Ogni riga successiva -------------------
%\vspace*{-\versesep}
%\[G] \[C]  \[D]	

%---- Ev Indicazioni -------------------------			
%\textnote{\textit{(Oppure tutta la strofa)} }	

\endverse
\fi


%%%%% RITORNELLO
\beginchorus
\textnote{\textbf{Rit.}}

Un \[E]solo \[A]Spirito, un \[F#-]solo Bat\[B7]tesimo, 
un \[E]solo Si\[G#7]gnore: Ge\[C#]sù!
Nel \[F#-]segno dell'a\[B]more tu \[G#]sei con \[A]noi, 
nel \[F#-]nome tuo vi\[B]viamo fra\[G#]telli: \[C#7]
nel \[F#-]cuore la spe\[B]ranza che \[G#]tu ci \[A]dai, 
la \[F#-]fede che ci u\[B7]nisce can\[E]tiamo!
\endchorus


%%%%% STROFA
\beginverse		%Oppure \beginverse* se non si vuole il numero di fianco
%\memorize 		% <<< DECOMMENTA se si vuole utilizzarne la funzione
%\chordsoff		% <<< DECOMMENTA se vuoi una strofa senza accordi

Io \[E-]sono la vite e \[A-]voi siete i t\[E-]ralci miei:
il \[C]tralcio che in me non \[D]vive sfiori\[G]rà; \[B7] 
ma \[E-]se rimanete in \[A-]me il \[C]Padre mio vi da\[G]rà 
la \[F]forza di una \[B&dim/(A)]vita che non \[B7]muore mai. 

\endverse

%%%%% STROFA
\beginverse		%Oppure \beginverse* se non si vuole il numero di fianco
%\memorize 		% <<< DECOMMENTA se si vuole utilizzarne la funzione
\chordsoff		% <<< DECOMMENTA se vuoi una strofa senza accordi

Io sono la vera via e la verità:
amici vi chiamo sempre sto con voi;
chi annuncia al fratello suo la fede nel nome mio
davanti al Padre io lo riconoscerò. 


\endverse

%%%%% STROFA
\beginverse		%Oppure \beginverse* se non si vuole il numero di fianco
%\memorize 		% <<< DECOMMENTA se si vuole utilizzarne la funzione
\chordsoff		% <<< DECOMMENTA se vuoi una strofa senza accordi
 
Lo Spirito Santo in voi parlerà di me;
dovunque c'è un uomo al mondo sono io;
ognuno che crede in me fratello vostro sarà
nel segno del Battesimo rinascerà. 

\endverse





\endsong
\fi
%++++++++++++++++++++++++++++++++++++++++++++++++++++++++++++
%			FINE CANZONE TRASPOSTA
%++++++++++++++++++++++++++++++++++++++++++++++++++++++++++++
%VVV

%-------------------------------------------------------------
%			INIZIO	CANZONE
%-------------------------------------------------------------


%titolo: 	Veniamo ad incontrarti
%autore: 	
%tonalita: 	La- 



%%%%%% TITOLO E IMPOSTAZONI
\beginsong{Veniamo ad incontrarti}[by={}] 	% <<< MODIFICA TITOLO E AUTORE
\transpose{0} 						% <<< TRASPOSIZIONE #TONI (0 nullo)
%\preferflats  %SE VOGLIO FORZARE i bemolle come alterazioni
%\prefersharps %SE VOGLIO FORZARE i # come alterazioni
\momenti{}							% <<< INSERISCI MOMENTI	
% momenti vanno separati da ; e vanno scelti tra:
% Ingresso; Atto penitenziale; Acclamazione al Vangelo; Dopo il Vangelo; Offertorio; Comunione; Ringraziamento; Fine; Santi; Pasqua; Avvento; Natale; Quaresima; Canti Mariani; Battesimo; Prima Comunione; Cresima; Matrimonio; Meditazione; Spezzare del pane;
\ifchorded
	%\textnote{Tonalità migliore }	% <<< EV COMMENTI (tonalità originale/migliore)
\fi


%%%%%% INTRODUZIONE
\ifchorded
\vspace*{\versesep}
\textnote{Intro: \qquad \qquad  }%(\eighthnote 116) % <<  MODIFICA IL TEMPO
% Metronomo: \eighthnote (ottavo) \quarternote (quarto) \halfnote (due quarti)
\vspace*{-\versesep}
\beginverse*

\nolyrics

%---- Prima riga -----------------------------
\vspace*{-\versesep}
\[A-] \[E] \[A-]	 % \[*D] per indicare le pennate, \rep{2} le ripetizioni

%---- Ogni riga successiva -------------------
%\vspace*{-\versesep}
%\[G] \[C]  \[D]	

%---- Ev Indicazioni -------------------------			
%\textnote{\textit{(Oppure tutta la strofa)} }	

\endverse
\fi

%%%%% STROFA
\beginverse		%Oppure \beginverse* se non si vuole il numero di fianco
\memorize 		% <<< DECOMMENTA se si vuole utilizzarne la funzione
%\chordsoff		% <<< DECOMMENTA se vuoi una strofa senza accordi

\[A-]Noi riceviamo \[E]quel pane che \[A-]Tu
hai prep\[E]arato per ogni \[D-]uomo 
che a Te s’\[E]affida.
V\[A-]eniamo, veniamo ad incont\[D-]rarti,
veni\[G7]amo, veniamo a ricord\[C]are ancor
la \[F]cena che un giorno hai chiesto agli \[D-]uomini
di \[E]fare con \[A-]Te.

\endverse

%%%%% STROFA
\beginverse		%Oppure \beginverse* se non si vuole il numero di fianco
\memorize 		% <<< DECOMMENTA se si vuole utilizzarne la funzione
%\chordsoff		% <<< DECOMMENTA se vuoi una strofa senza accordi

^Noi porteremo, usc^endo da ^qui,
la Tua pre^senza, il Tuo mess^aggio, 
tra l’altra ^gente:
C^erchiamo, negli occhi d’ogni ^uomo,
la ^luce di chi ha conosciuto ^Te,
per^ché noi vogliamo che tu abbia ^ancor
fra^telli fra ^noi.

\endverse

\endsong
%------------------------------------------------------------
%			FINE CANZONE
%------------------------------------------------------------

%-------------------------------------------------------------
%			INIZIO	CANZONE
%-------------------------------------------------------------


%titolo: 	Venimus adorare eum
%autore: 	Lissen
%tonalita: 	Re



%%%%%% TITOLO E IMPOSTAZONI
\beginsong{Venimus adorare eum}[by={Inno della XX GMG, Colonia 2005 — Linssen}]
	% <<< MODIFICA TITOLO E AUTORE
\transpose{0} 						% <<< TRASPOSIZIONE #TONI (0 nullo)
\momenti{Ringraziamento}							% <<< INSERISCI MOMENTI	
% momenti vanno separati da ; e vanno scelti tra:
% Ingresso; Atto penitenziale; Acclamazione al Vangelo; Dopo il Vangelo; Offertorio; Comunione; Ringraziamento; Fine; Santi; Pasqua; Avvento; Natale; Quaresima; Canti Mariani; Battesimo; Prima Comunione; Cresima; Matrimonio; Meditazione; Spezzare del pane;
\ifchorded
	%\textnote{Tonalità migliore }	% <<< EV COMMENTI (tonalità originale/migliore)
\fi


%%%%%% INTRODUZIONE
\ifchorded
\vspace*{\versesep}
\textnote{Intro: \qquad \qquad  }%(\eighthnote 116) % <<  MODIFICA IL TEMPO
% Metronomo: \eighthnote (ottavo) \quarternote (quarto) \halfnote (due quarti)
\vspace*{-\versesep}
\beginverse*

\nolyrics

%---- Prima riga -----------------------------
\vspace*{-\versesep}
\[D] \[G] \[A4] \[A]% \[*D] per indicare le pennate, \rep{2} le ripetizioni

%---- Ogni riga successiva -------------------
%\vspace*{-\versesep}
%\[G] \[C]  \[D]	

%---- Ev Indicazioni -------------------------			
%\textnote{\textit{(Oppure tutta la strofa)} }	

\endverse
\fi




\beginverse
\[D]Chiedi perché par\[G]tire \[E-]dal proprio \[A]regno
\[B-]solo per inse\[G]guire \[E-]una stella e per\[A]ché
\[G]per un Bimbo pi\[E7]egano quelle gi\[D]nocchia da \[B-]Re?
\[E-] Tu la risposta  sai che \[A]è:
\endverse

\beginchorus
Ve\[D]nimus adorare Eum, Em\[C]manuel \[G]Dio con noi
\[D]Venimus adorare \[C]Eum, Em\[A]manuel 
Ve\[D]nimus adorare Eum, Em\[C]manuel \[G]Dio con noi
\[D]Venimus adorare \[C]Eum, Em\[G]manuel 
\endchorus

\chordsoff
\beginverse
Chiedi perché lasciare sui monti il gregge
solo per ascoltare un canto e perché
per un Bimbo piegano quelle ginocchia, perché?
Tu la risposta  sai che è:
\endverse

\beginverse
Ecco da lontano per adorarlo \brk siamo giunti anche noi,
noi, tutti figli suoi, profeti e sacerdoti ormai.
Nel pane e nel vino noi siamo in lui e lui è in noi:
e un canto qui si alza già.
\endverse



\endsong
%------------------------------------------------------------
%			FINE CANZONE
%------------------------------------------------------------







%titolo{Verbum panis}
%autore{Casucci, Balduzzi}
%album{Verbum panis}
%tonalita{Mi-}
%famiglia{Liturgica}
%gruppo{}
%momenti{Comunione}
%identificatore{verbum_panis}
%data_revisione{2011_12_31}
%trascrittore{Francesco Endrici}
\beginsong{Verbum panis}[by={Casucci, Balduzzi}]
\ifchorded
\beginverse*
\vspace*{-0.8\versesep}
{\nolyrics \[E-]\[(6x)]\[\vline]\[C7+]\[D]\[\vline]\[E-]}
\vspace*{-\versesep}
\endverse
\fi
\beginverse
\memorize
\[E-]Prima del \[D]tempo
prima an\[E-]cora che la \[D]terra
comin\[E-]ciasse a vive\[D]re 
\[E-]il Verbo \[D]era presso \[E-]Dio. \[D]\[E-]\[D]
\[E-]Venne nel \[D]mondo
e per \[E-]non abbando\[D]narci
in questo \[E-]viaggio ci la\[D]sciò
\[E-]tutto sé \[D]stesso come \[E-]pane. \[D] \[E-]
\endverse
\beginchorus
Verbum \[E-]caro factum est \[E-]
Verbum \[E-]panis factum est \[E-]
Verbum \[E-]caro factum est \[E-]
Verbum \[E-]panis factum \[C7+]est. \[C7+]\[D4]\[D]
\endchorus
\beginverse
\[G]Qui \[D]spezzi ancora il \[C]pane in mezzo a \[D]noi
e chi\[G]unque mange\[D]rà \[C]non avrà più \[D]fame.
\[G]Qui \[D]vive la tua \[C]chiesa intorno a \[D]te
dove o\[G]gnuno trove\[D]rà \brk \[C]la sua vera \[D]casa. \[E-]
\endverse
\beginchorus
Verbum \[E-]caro factum est \[E-]
Verbum \[E-]panis factum est \[E-]
Verbum \[E-]caro factum est \[E-]
Verbum \[E-]panis 
\endchorus
\beginverse
^Prima del ^tempo
quando l'^universo ^fu creato
^dall'oscuri^tà
^il Verbo ^era presso ^Dio. ^^^
^Venne nel ^mondo
nella ^sua miseri^cordia
Dio ha man^dato il Figlio ^suo
^tutto sé ^stesso come ^pane. ^ ^
\endverse
\beginchorus
Verbum \[E-]caro factum est \[E-]
Verbum \[E-]panis factum est \[E-]
Verbum \[E-]caro factum est \[E-]
Verbum \[E-]panis factum \[C7+]est. \[C7+]\[D4]\[D]
\endchorus
\beginverse
\[G]Qui \[D]spezzi ancora il \[C]pane in mezzo a \[D]noi
e chi\[G]unque mange\[D]rà \[C]non avrà più \[D]fame.
\[G]Qui \[D]vive la tua \[C]chiesa intorno a \[D]te
dove o\[G]gnuno trove\[D]rà \brk \[C]la sua vera \[D]casa. \[E-]
\endverse
\beginchorus
Verbum \[E-]caro factum est \[E-]
Verbum \[E-]panis factum est \[E-]
Verbum \[E-]caro factum est \[E-]
Verbum \[E-]panis factum \[E-]est. 
\endchorus
\endsong


%titolo{Vieni e seguimi}
%autore{Gen Rosso}
%album{Se siamo uniti}
%tonalita{La}
%famiglia{Liturgica}
%gruppo{}
%momenti{Congedo}
%identificatore{vieni_e_seguimi}
%data_revisione{2012_03_03}
%trascrittore{Francesco Endrici}
\beginsong{Vieni e seguimi}[by={Gen\ Rosso}]
\beginverse
Lascia \[A]che il mondo \[B-]vada per la sua \[A]strada.
Lascia \[C#-]che l'uomo ri\[F#-]torni alla sua \[E]casa.
Lascia \[D]che la gente accumuli la sua for\[A]tuna.
\endverse
\beginchorus
Ma \[E]tu, tu \[D]vieni e \[A]seguimi, \[E]tu, \[D] vieni e \[A]seguimi.
\endchorus
\beginverse
Lascia ^che la barca in ^mare spieghi la ^vela.
Lascia ^che trovi af^fetto chi segue il ^cuore.
Lascia ^che dall'albero cadano i frutti ma^turi.
\endverse
\beginchorus
Ma \[E]tu, tu \[D]vieni e \[A]seguimi, \[E]tu, \[D] vieni e \[F#-]seguimi.
\endchorus
\beginverse
E sa\[F#]rai luce per gli \[B]uomini
e sa\[F#]rai sale della \[C#-]terra \[E]
e nel mondo de\[F#]serto aprirai
una \[B]strada nuova. \rep{2}
E per \[F#]questa strada, \[G#-]va', \[F#]va',
e \[B]non voltarti indietro, \[F#]va'
e \[B]non voltarti indietro\[F#]{\dots}
\endverse
\endsong



%-------------------------------------------------------------
%			INIZIO	CANZONE
%-------------------------------------------------------------


%titolo: 	Santo Ricci
%autore: 	Daniele Ricci
%tonalita: 	Sol 



%%%%%% TITOLO E IMPOSTAZONI
\beginsong{Vieni o Spirito}[by={M. C. Bizzeti}] 	% <<< MODIFICA TITOLO E AUTORE
\transpose{0} 						% <<< TRASPOSIZIONE #TONI (0 nullo)
%\preferflats  %SE VOGLIO FORZARE i bemolle come alterazioni
%\prefersharps %SE VOGLIO FORZARE i # come alterazioni
\momenti{}							% <<< INSERISCI MOMENTI	
% momenti vanno separati da ; e vanno scelti tra:
% Ingresso; Atto penitenziale; Acclamazione al Vangelo; Dopo il Vangelo; Offertorio; Comunione; Ringraziamento; Fine; Santi; Pasqua; Avvento; Natale; Quaresima; Canti Mariani; Battesimo; Prima Comunione; Cresima; Matrimonio; Meditazione; Spezzare del pane;
\ifchorded
	%\textnote{$\bigstar$ Tonalità migliore }	% <<< EV COMMENTI (tonalità originale\migliore)
\fi


%%%%%% INTRODUZIONE
\ifchorded
\vspace*{\versesep}
\musicnote{
\begin{minipage}{0.48\textwidth}
\textbf{Intro}
\hfill 
%( \eighthnote \, 80)   % <<  MODIFICA IL TEMPO
% Metronomo: \eighthnote (ottavo) \quarternote (quarto) \halfnote (due quarti)
\end{minipage}
} 	
\vspace*{-\versesep}
\beginverse*

\nolyrics

%---- Prima riga -----------------------------
\vspace*{-\versesep}
\[B-] \[A] \[B-]	 % \[*D] per indicare le pennate, \rep{2} le ripetizioni

%---- Ogni riga successiva -------------------
%\vspace*{-\versesep}
%\[G] \[C]  \[D]	

%---- Ev Indicazioni -------------------------			
%\textnote{\textit{[oppure tutta la strofa]} }	

\endverse
\fi



%%%%% RITORNELLO
\beginchorus
\textnote{\textbf{Rit.}}

\[B-]Vieni o Sprito \[B-/D]Spirito di Dio,	
\[A]vieni o \[F#-]Spirito \[B-]Santo.	
\[B-]Vieni o Spirito soffia su di noi 
\[A]dona ai tuoi \[F#-]figli la \[B-]vita.

\endchorus



%%%%% STROFA
\beginverse		%Oppure \beginverse* se non si vuole il numero di fianco
%\memorize 		% <<< DECOMMENTA se si vuole utilizzarne la funzione
%\chordsoff		% <<< DECOMMENTA se vuoi una strofa senza accordi


\[G]Dona la \[A]luce ai nostri \[B-]occhi,
\[D]dona la \[A]forza ai nostri \[B-]cuori,
\[F#]dona alle menti la sapie\[B-]nza,
\[G]dona il tuo \[A]fuoco d'am\[B-]ore.

\endverse

%%%%% RITORNELLO
\beginchorus
\textnote{\textbf{Rit.}}

\[B-]Vieni o Sprito \[B-/D]Spirito di Dio,	
\[A]vieni o \[F#-]Spirito \[B-]Santo.	
\[B-]Vieni o Spirito soffia su di noi 
\[A]dona ai tuoi \[F#-]figli la \[B-]vita.

\endchorus


%%%%% STROFA
\beginverse		%Oppure \beginverse* se non si vuole il numero di fianco
%\memorize 		% <<< DECOMMENTA se si vuole utilizzarne la funzione
%\chordsoff		% <<< DECOMMENTA se vuoi una strofa senza accordi

\[G]Tu sei per \[A]noi consola\[B-]tore;
\[D]nella \[A]calura sei \[B-]riparo
\[F#]nella fatica sei \[B-]riposo
\[G]nel pianto \[A]sei conf\[B-]orto.

\endverse





%%%%% RITORNELLO
\beginchorus
\textnote{\textbf{Rit.}}

\[B-]Vieni o Sprito \[B-/D]Spirito di Dio,	
\[A]vieni o \[F#-]Spirito \[B-]Santo.	
\[B-]Vieni o Spirito soffia su di noi 
\[A]dona ai tuoi \[F#-]figli la \[B-]vita.

\endchorus


%%%%% STROFA
\beginverse		%Oppure \beginverse* se non si vuole il numero di fianco
%\memorize 		% <<< DECOMMENTA se si vuole utilizzarne la funzione
%\chordsoff		% <<< DECOMMENTA se vuoi una strofa senza accordi

\[G]Dona a \[A]tutti i tuoi \[B-]fedeli
\[D]Che \[A]confidano in \[B-]Te.
\[F#]I tuoi sette Santi \[B-]doni,
\[G]dona la \[A]gioia \[B-]eterna.

\endverse





\endsong
%------------------------------------------------------------
%			FINE CANZONE
%------------------------------------------------------------


%-------------------------------------------------------------
%			INIZIO	CANZONE
%-------------------------------------------------------------


%titolo: 	Vieni qui tra noi
%autore: 	Gen Rosso
%tonalita: 	Sol 



%%%%%% TITOLO E IMPOSTAZONI
\beginsong{Vieni qui tra noi}[by={Gen\ Verde, Gen\ Rosso}] 	% <<< MODIFICA TITOLO E AUTORE
\transpose{0} 						% <<< TRASPOSIZIONE #TONI (0 nullo)
\momenti{}							% <<< INSERISCI MOMENTI	
% momenti vanno separati da ; e vanno scelti tra:
% Ingresso; Atto penitenziale; Acclamazione al Vangelo; Dopo il Vangelo; Offertorio; Comunione; Ringraziamento; Fine; Santi; Pasqua; Avvento; Natale; Quaresima; Canti Mariani; Battesimo; Prima Comunione; Cresima; Matrimonio; Meditazione; Spezzare del pane;
\ifchorded
	%\textnote{Tonalità migliore }	% <<< EV COMMENTI (tonalità originale/migliore)
\fi


%%%%%% INTRODUZIONE
\ifchorded
\vspace*{\versesep}
\textnote{Intro: \qquad \qquad  }%(\eighthnote 116) % <<  MODIFICA IL TEMPO
% Metronomo: \eighthnote (ottavo) \quarternote (quarto) \halfnote (due quarti)
\vspace*{-\versesep}
\beginverse*

\nolyrics

%---- Prima riga -----------------------------
\vspace*{-\versesep}
\[C] \[G] \[D*] \[G]	 % \[*D] per indicare le pennate, \rep{2} le ripetizioni

%---- Ogni riga successiva -------------------
%\vspace*{-\versesep}
%\[G] \[C]  \[D]	

%---- Ev Indicazioni -------------------------			
%\textnote{\textit{(Oppure tutta la strofa)} }	

\endverse
\fi


%%%%% STROFA
\beginverse*
\[C]Vie\[G]ni \[D*]qui tra \[G]noi
come \[C]fiamma che \[E-]scende dal \[D4]cielo.
\[C]Vie\[G]ni \[D*]qui tra \[E-7]noi,
\[A-7]rin\[G]nova il \[C]cuore del \[D4]mondo.
\[C]Vie\[G]ni \[D*]qui tra \[G]noi,
col tuo a\[C]more ri\[E-]schiara la \[D4]terra.
\[C]Vie\[G]ni \[D*]qui \[E-7]noi,
\[A-7]soffio di \[D]liber\[G7]tà. \[C7 G]
\endverse


%%%%% STROFA
\beginverse*
Nel si\[G]lenzio tu \[C]sei pa\[G]ce,
nella notte \[C]lu\[G]ce,
Dio nascosto, \[C]vi\[G]ta,
Dio tu sei, A\[C]mo\[G]re.
Tutto si ri\[C]crea in \[G]te,
tutto \[D]vive in \[E-7]te.
Scalda col tuo \[C]fuo\[G]co \brk terra e \[D]cie\[E-7]lo.
Tu, che sai rac\[C]coglie\[G]re \brk ogni \[D]ge\[B*]mi\[E-]to, 
semina nel \[C]nostro \[G]cuore \brk una spe\[A-7]ranza d'eterni\[C]tà.
\endverse


%%%%% STROFA
\beginverse*
\[C]Vie\[G]ni \[D*]qui tra \[G]noi
come \[C]fiamma che \[E-]scende dal \[D4]cielo.
\[C]Vie\[G]ni \[D*]qui tra \[E-7]noi,
\[A-7]rin\[G]nova il \[C]cuore del \[D4]mondo.
\[C]Vie\[G]ni \[D*]qui tra \[G]noi,
col tuo a\[C]more ri\[E-]schiara la \[D4]terra.
\[C]Vie\[G]ni \[D*]qui \[E-7]noi,
\[C]soffio di \[C-]liber\[A7]tà
amore, \[C]Dio in mezzo a \[G]noi!
\endverse



\endsong
%------------------------------------------------------------
%			FINE CANZONE
%------------------------------------------------------------


%-------------------------------------------------------------
%			INIZIO	CANZONE
%-------------------------------------------------------------


%titolo: 	Vieni Santo Spirito
%autore: 	Gen Verde
%tonalita: 	Do 



%%%%%% TITOLO E IMPOSTAZONI
\beginsong{Vieni Santo Spirito}[by={Gen Verde}] 	% <<< MODIFICA TITOLO E AUTORE
\transpose{0} 						% <<< TRASPOSIZIONE #TONI (0 nullo)
%\preferflats  %SE VOGLIO FORZARE i bemolle come alterazioni
%\prefersharps %SE VOGLIO FORZARE i # come alterazioni
\momenti{Cresima}							% <<< INSERISCI MOMENTI	
% momenti vanno separati da ; e vanno scelti tra:
% Ingresso; Atto penitenziale; Acclamazione al Vangelo; Dopo il Vangelo; Offertorio; Comunione; Ringraziamento; Fine; Santi; Pasqua; Avvento; Natale; Quaresima; Canti Mariani; Battesimo; Prima Comunione; Cresima; Matrimonio; Meditazione; Spezzare del pane;
\ifchorded
	%\textnote{Tonalità migliore }	% <<< EV COMMENTI (tonalità originale/migliore)
\fi


%%%%%% INTRODUZIONE
\ifchorded
\vspace*{\versesep}
\textnote{Intro: \qquad \qquad  }%(\eighthnote 116) % <<  MODIFICA IL TEMPO
% Metronomo: \eighthnote (ottavo) \quarternote (quarto) \halfnote (due quarti)
\vspace*{-\versesep}
\beginverse*

\nolyrics

%---- Prima riga -----------------------------
\vspace*{-\versesep}
\[C] \[A-] \[F] \[G] 	 % \[*D] per indicare le pennate, \rep{2} le ripetizioni

%---- Ogni riga successiva -------------------
%\vspace*{-\versesep}
%\[G] \[C]  \[D]	

%---- Ev Indicazioni -------------------------			
%\textnote{\textit{(Oppure tutta la strofa)} }	

\endverse
\fi










%%%%% RITORNELLO
\beginchorus
\textnote{\textbf{Rit.}}
\[C]Vieni \[G]Santo \[A-]Spirito,  \brk \[F]manda a \[C]noi dal \[G4]cie\[G]lo,
un \[C]ra-a-g\[F]gio di \[C]lu-u-\[G]ce,  \brk un \[F]raggio di \[C]luce.
\endchorus
\beginchorus
\[C]Vieni \[G]Padre dei \[A-]poveri, \brk \[F]vieni da\[C]tore dei \[G4]do-o-\[G]ni,
\[C]lu-u-\[F]ce dei \[C]cuo-o-\[G]ri,  \brk \[F]luce dei \[C]cuori.
\endchorus



%%%%% STROFA
\beginverse		%Oppure \beginverse* se non si vuole il numero di fianco
\memorize 		% <<< DECOMMENTA se si vuole utilizzarne la funzione
%\chordsoff		% <<< DECOMMENTA se vuoi una strofa senza accordi
\[A-]Consola\[F]tore per\[G]fe-e-t\[A-]to, \brk \[F]ospite \[G]dolce dell’\[C]a-a-ni\[E]ma,
\[A-]dolcissimo sol\[G]lievo,  \brk  \[A-]dolcissimo sol\[E7]lievo.
\[A-]Nella fa\[F]tica ri\[G]po-o-\[A-]so, \brk \[F]nel ca\[G]lore ri\[C]pa-a-\[E]ro,
nel \[A-]pianto con\[G]forto,  \brk  \[A-]ne-el pianto con\[E]forto. \[E]
\endverse



%%%%% RITORNELLO
\beginchorus
\textnote{\textbf{Rit.}}
\[C]Vieni \[G]Santo \[A-]Spirito,  \brk \[F]manda a \[C]noi dal \[G4]cie\[G]lo,
un \[C]ra-a-g\[F]gio di \[C]lu-u-\[G]ce,  \brk un \[F]raggio di \[C]luce.
\endchorus


%%%%% STROFA
\beginverse		%Oppure \beginverse* se non si vuole il numero di fianco
%\memorize 		% <<< DECOMMENTA se si vuole utilizzarne la funzione
%\chordsoff		% <<< DECOMMENTA se vuoi una strofa senza accordi

^Luce ^be-e-a^tissi^ma,  \brk in^vadi i ^nostri ^cuo-o-^ri,
^senza la tua forza ^nu-ulla, \brk  ^nulla è nell’^uomo.
^Lava ^ciò che è ^sordi^do, \brk  ^scalda ^ciò che è ^geli^do,
^rialza chi è ca^duto, \brk  ri^alza chi è ca^duto. ^

\endverse



%%%%% RITORNELLO
\beginchorus
\textnote{\textbf{Rit.}}
\[C]Vieni \[G]Padre dei \[A-]poveri, \brk \[F]vieni da\[C]tore dei \[G4]do-o-\[G]ni,
\[C]lu-u-\[F]ce dei \[C]cuo-o-\[G]ri,  \brk \[F]luce dei \[C]cuori.
\endchorus



%%%%%% EV. INTERMEZZO
\beginverse*
\vspace*{1.3\versesep}
{
	\nolyrics
	\textnote{Intermezzo strumentale}
	
	\ifchorded

	%---- Prima riga -----------------------------
	\vspace*{-\versesep}
	\[C] \[A-]  \[F]\[G] \[A] \[F#]


	\fi
	%---- Ev Indicazioni -------------------------			
	\textnote{\textit{(Si alza la tonalità)}} 
	 
}
\vspace*{\versesep}
\endverse


%%%%% STROFA
\beginverse		%Oppure \beginverse* se non si vuole il numero di fianco
%\memorize 		% <<< DECOMMENTA se si vuole utilizzarne la funzione
%\chordsoff		% <<< DECOMMENTA se vuoi una strofa senza accordi
\transpose{2}
^Dona ai ^tuoi fe^de-e-^li,  \brk ^che in ^te con^fida^no;
^i sette santi ^doni,  \brk ^i sette santi ^doni.
^Dona vir^tù e ^premi^o, \brk  ^dona ^morte ^sa-a-n^ta,
^dona eterna ^gioia, \brk  ^dona eterna ^gioia. ^
\endverse



%%%%% RITORNELLO
\beginchorus
\textnote{\textbf{Rit.}}
\transpose{2}
\[C]Vieni \[G]Santo \[A-]Spirito,  \brk \[F]manda a \[C]noi dal \[G4]cie\[G]lo,
un \[C]ra-a-g\[F]gio di \[C]lu-u-\[G]ce,  \brk un \[F]raggio di \[C]luce.
\endchorus
\beginchorus
\transpose{2}
\[C]Vieni \[G]Padre dei \[A-]poveri, \brk \[F]vieni da\[C]tore dei \[G4]do-o-\[G]ni,
\[C]lu-u-\[F]ce dei \[C]cuo-o-\[G]ri,  \brk \[F]luce dei \[C]cuori.
\endchorus



\endsong
%------------------------------------------------------------
%			FINE CANZONE
%------------------------------------------------------------



%-------------------------------------------------------------
%			INIZIO	CANZONE
%-------------------------------------------------------------


%titolo: 	Vieni Spirito d'amore
%autore: 	G. Amadei, K. Arguello
%tonalita: 	Do 



%%%%%% TITOLO E IMPOSTAZONI
\beginsong{Vieni Spirito d'amore}[by={G. Amadei, K. Arguello}] 	% <<< MODIFICA TITOLO E AUTORE
\transpose{0} 						% <<< TRASPOSIZIONE #TONI (0 nullo)
%\preferflats  %SE VOGLIO FORZARE i bemolle come alterazioni
%\prefersharps %SE VOGLIO FORZARE i # come alterazioni
\momenti{Battesimo; Cresima}							% <<< INSERISCI MOMENTI	
% momenti vanno separati da ; e vanno scelti tra:
% Ingresso; Atto penitenziale; Acclamazione al Vangelo; Dopo il Vangelo; Offertorio; Comunione; Ringraziamento; Fine; Santi; Pasqua; Avvento; Natale; Quaresima; Canti Mariani; Battesimo; Prima Comunione; Cresima; Matrimonio; Meditazione; Spezzare del pane;
\ifchorded
	%\textnote{Tonalità migliore }	% <<< EV COMMENTI (tonalità originale/migliore)
\fi


%%%%%% INTRODUZIONE
\ifchorded
\vspace*{\versesep}
\musicnote{
\begin{minipage}{0.48\textwidth}
\textbf{Intro}
\hfill 
%( \eighthnote \, 80)   % <<  MODIFICA IL TEMPO
% Metronomo: \eighthnote (ottavo) \quarternote (quarto) \halfnote (due quarti)
\end{minipage}
} 	
\vspace*{-\versesep}
\beginverse*

\nolyrics

%---- Prima riga -----------------------------
\vspace*{-\versesep}
\[E-] \[A-]  \rep{2}	 % \[*D] per indicare le pennate, \rep{2} le ripetizioni

%---- Ogni riga successiva -------------------
%\vspace*{-\versesep}
%\[G] \[C]  \[D]	

%---- Ev Indicazioni -------------------------			
%\textnote{\textit{(Oppure tutta la strofa)} }	

\endverse
\fi



%%%%% RITORNELLO
\beginchorus
\textnote{\textbf{Rit.}}

\[E-]Vieni, vieni, \[A-]Spirito d'amore,
ad \[E-]insegnar le cose di \[B-]Dio.
\[E-]Vieni, vieni, \[A-]Spirito di pace,
a s\[E-]uggerir le cose che \[B-]Lui...
ha detto a \[E-]noi.

\endchorus


%%%%% STROFA
\beginverse		%Oppure \beginverse* se non si vuole il numero di fianco
\memorize 		% <<< DECOMMENTA se si vuole utilizzarne la funzione
%\chordsoff		% <<< DECOMMENTA se vuoi una strofa senza accordi

\[E-]Noi t'invochiamo, \[A-]Spirito di Cristo,
\[E-]vieni Tu dentro di \[B-]noi.
\[E-]Cambia i nostri occhi, \[A-]fa' che noi vediamo
\[E-]la bontà di Dio per \[B-]noi.

\endverse




%%%%% STROFA
\beginverse		%Oppure \beginverse* se non si vuole il numero di fianco
%\memorize 		% <<< DECOMMENTA se si vuole utilizzarne la funzione
\chordsoff		% <<< DECOMMENTA se vuoi una strofa senza accordi

^Vieni o Spirito ^dai quattro venti
e ^soffia su chi non ha ^vita.
^Vieni o Spirito, ^soffia su di noi
^perché anche noi rivi^viamo.

\endverse



%%%%% STROFA
\beginverse		%Oppure \beginverse* se non si vuole il numero di fianco
%\memorize 		% <<< DECOMMENTA se si vuole utilizzarne la funzione
\chordsoff		% <<< DECOMMENTA se vuoi una strofa senza accordi

^Insegnaci a sperare, in^segnaci ad amare.
Ins^egnaci a lodare Id^dio.
In^segnaci a pregare, in^segnaci la via.
In^segnaci Tu l'uni^tà.

\endverse



\endsong
%------------------------------------------------------------
%			FINE CANZONE
%------------------------------------------------------------



%titolo{Vivere la vita}
%autore{Gen Verde}
%album{È bello lodarti}
%tonalita{Do}
%famiglia{Liturgica}
%gruppo{}
%momenti{Ingresso}
%identificatore{vivere_la_vita}
%data_revisione{2011_12_31}
%trascrittore{Francesco Endrici}
\beginsong{Vivere la vita}[by={Gen\ Verde}]
\beginverse
\[C]Vivere la \[G]vita con le \[D-]gioie 
e coi do\[F]lori di ogni \[A-]giorno, \[G]
è quello che Dio \[C]vuole da te. \[G]
\[C]Vivere la \[G]vita e inabis\[D-]sarsi 
nell'a\[F]more è il tuo de\[A-]stino, \[G]
è quello che Dio \[C]vuole da \[G]te.
\[F]Fare insieme agli \[G]altri la tua \[C]strada verso \[E-]Lui,
\[F]correre con \[G]i fratelli \[C]tuo\[E-]i.
\[F]Scoprirai al\[G]lora il \[G7]cielo \[C]dentro di \[E-]te,
\[F]una scia di \[D-]luce lasce\[G]rai.
\endverse
\beginverse
%\chordsoff
^Vivere la ^vita è l'avven^tura 
più stu^penda dell'a^more, ^ 
è quello che Dio ^vuole da te. ^
^Vivere la ^vita e gene^rare 
ogni mo^mento il para^diso, ^ 
è quello che Dio ^vuole da ^te.
^Vivere per^ché ritorni al ^mondo l'uni^tà,
^perché Dio sta ^nei fratelli ^tuo^i.
^Scoprirai al^lora il ^cielo ^dentro di ^te,
^una scia di ^luce lasce^rai,
\[F]una scia di \[D-]luce lasce\[C]rai.
\endverse
\endsong


%titolo{Vocazione}
%autore{Sequeri}
%album{In cerca d'autore}
%tonalita{Do}
%famiglia{Liturgica}
%gruppo{}
%momenti{Ingresso}
%identificatore{vocazione}
%data_revisione{2011_12_31}
%trascrittore{Francesco Endrici}
\beginsong{Vocazione}[by={Sequeri}]
\beginverse
\[C]Era un giorno \[G]come tanti \[F]altri,
e quel \[G]giorno Lui pas\[C]sò. \[F]\[C]\[G]
\[C]Era un uomo \[G]come tutti gli \[F]altri,
e pas\[G]sando mi chia\[C]mò \[F]\[C]\[E]
\[A-]come lo sa\[E-]pesse che il mio \[F]nome \brk era \[G]proprio quello
\[C]come mai ve\[G]desse proprio \[F]me
nella sua \[G]vita, non lo \[C]so. \[F]\[C]\[G]
\[C]Era un giorno \[G]come tanti \[F]altri
e quel \[G]giorno mi chia\[C]mò. \[F]\[C]\[E]
\endverse
\beginchorus
\[A-]Tu \[E-]Dio, \[F]che conosci il \[G]nome mio
\[A-]fa' \[E-]che \[F]ascoltando \[G]la tua voce
\[C]io ri\[G]cordi dove \[F]porta la mia \[G]strada
\[C]nella \[G]vita, all'in\[F]contro con \[C]Te. \[F]\[C]\[G]
\endchorus
\beginverse
^Era l'alba ^triste e senza ^vita,
e ^qualcuno mi chia^mò ^^^
^era un uomo ^come tanti ^altri,
ma la ^voce, quella ^no. ^^^
^Quante volte un ^uomo
con il ^nome giusto ^mi ha chiamato
^una volta ^sola l'ho sen^tito
pronun^ciare con a^more. ^^^
^Era un uomo ^come nessun ^altro
e quel ^giorno mi chia^mò. ^^^
\endverse
\endsong


%-------------------------------------------------------------
%			INIZIO	CANZONE
%-------------------------------------------------------------


%titolo: 	Voglio esaltare
%autore: 	Giampiero Colombo
%tonalita: 	MIm e REm 



%%%%%% TITOLO E IMPOSTAZONI
\beginsong{Voglio esaltare}[by={Giampiero Colombo}] 	% <<< MODIFICA TITOLO E AUTORE
\transpose{-2} 						% <<< TRASPOSIZIONE #TONI (0 nullo)
\preferflats
\momenti{Ingresso; Avvento}							% <<< INSERISCI MOMENTI	
% momenti vanno separati da ; e vanno scelti tra:
% Ingresso; Atto penitenziale; Acclamazione al Vangelo; Dopo il Vangelo; Offertorio; Comunione; Ringraziamento; Fine; Santi; Pasqua; Avvento; Natale; Quaresima; Canti Mariani; Battesimo; Prima Comunione; Cresima; Matrimonio; Meditazione;
\ifchorded
	\textnote{Tonalità migliore per le bambine }	% <<< EV COMMENTI (tonalità originale/migliore)
\fi


%%%%%% INTRODUZIONE
\ifchorded
\vspace*{\versesep}
\textnote{Intro: \qquad \qquad  }%(\eighthnote 116) % << MODIFICA IL TEMPO
% Metronomo: \eighthnote (ottavo) \quarternote (quarto) \halfnote (due quarti)
\vspace*{-\versesep}
\beginverse*

\nolyrics

%---- Prima riga -----------------------------
\vspace*{-\versesep}
\[E-] \[B-] \[E-] \[B-]  	 % \[*D] per indicare le pennate, \rep{2} le ripetizioni

%---- Ogni riga successiva -------------------
%\vspace*{-\versesep}
%\[G] \[C]  \[D]	

%---- Ev Indicazioni -------------------------			
%\textnote{\textit{(Oppure tutta la strofa)} }	

\endverse
\fi




%%%%% STROFA
\beginverse		%Oppure \beginverse* se non si vuole il numero di fianco
\memorize 		% <<< DECOMMENTA se si vuole utilizzarne la funzione
%\chordsoff		% <<< DECOMMENTA se vuoi una strofa senza accordi

\[E-]Voglio esal\[B-]tare \brk il \[E-]nome del Dio \[B-]nostro: 
\[C]è Lui la mia \[D]liber\[G]tà! \[B7]
\[E-]Ecco il ma\[B-]ttino, \brk \[E-]gioia di sal\[B-]vezza, 
un \[C]canto sta nas\[D]cendo in \[E-]noi.

\endverse




%%%%% RITORNELLO
\beginchorus
\textnote{\textbf{Rit.}}

\[A-]Vieni, o Si\[D7]gnore, \[G7]luce del cam\[C7]mino,
\[A-]fuoco che nel \[B]cuore 
ac\[E-]cen\[D]de il \[G]"sì". \[*E7]
\[A-]Lieto il tuo pas\[D7]saggio, \[G7]ritmi la spe\[C7]ranza,
\[A-]Padre della \[B7]veri\[E-]tà. \[B-] \[E-] \[B-]

\endchorus




%%%%% STROFA
\beginverse		%Oppure \beginverse* se non si vuole il numero di fianco
%\memorize 		% <<< DECOMMENTA se si vuole utilizzarne la funzione
%\chordsoff		% <<< DECOMMENTA se vuoi una strofa senza accordi
^Voglio esal^tare \brk il ^nome del Dio ^nostro
^grande nella ^fedel^tà! ^
^Egli mi ha ^posto \brk ^sull'alto suo ^monte.
^Roccia che non ^crolla ^mai.

\endverse


%%%%% STROFA
\beginverse		%Oppure \beginverse* se non si vuole il numero di fianco
%\memorize 		% <<< DECOMMENTA se si vuole utilizzarne la funzione
\chordsoff		% <<< DECOMMENTA se vuoi una strofa senza accordi

Voglio Annunciare il dono crocifisso
di Cristo, il Dio con noi!
Perchè della morte lui si prende gioco,
Figlio che ci attira a sé!

\endverse


\endsong
%------------------------------------------------------------
%			FINE CANZONE
%------------------------------------------------------------




% %++++++++++++++++++++++++++++++++++++++++++++++++++++++++++++
% %			CANZONE TRASPOSTA
% %++++++++++++++++++++++++++++++++++++++++++++++++++++++++++++
% \ifchorded
% %decremento contatore per avere stesso numero
% \addtocounter{songnum}{-1} 
% \beginsong{Voglio esaltare}[by={Giampiero Colombo}]  	% <<< COPIA TITOLO E AUTORE
% \transpose{0} 						% <<< TRASPOSIZIONE #TONI + - (0 nullo)
% \ifchorded
% 	\textnote{Tonalità alternativa}	% <<< EV COMMENTI (tonalità originale/migliore)
% \fi



% %%%%%% INTRODUZIONE
% \ifchorded
% \vspace*{\versesep}
% \textnote{Intro: \qquad \qquad  }%(\eighthnote 116) % << MODIFICA IL TEMPO
% % Metronomo: \eighthnote (ottavo) \quarternote (quarto) \halfnote (due quarti)
% \vspace*{-\versesep}
% \beginverse*

% \nolyrics

% %---- Prima riga -----------------------------
% \vspace*{-\versesep}
% \[E-] \[B-] \[E-] \[B-]  	 % \[*D] per indicare le pennate, \rep{2} le ripetizioni

% %---- Ogni riga successiva -------------------
% %\vspace*{-\versesep}
% %\[G] \[C]  \[D]	

% %---- Ev Indicazioni -------------------------			
% %\textnote{\textit{(Oppure tutta la strofa)} }	

% \endverse
% \fi




% %%%%% STROFA
% \beginverse		%Oppure \beginverse* se non si vuole il numero di fianco
% \memorize 		% <<< DECOMMENTA se si vuole utilizzarne la funzione
% %\chordsoff		% <<< DECOMMENTA se vuoi una strofa senza accordi

% \[E-]Voglio esal\[B-]tare \brk il \[E-]nome del Dio \[B-]nostro: 
% \[C]è Lui la mia \[D]liber\[G]tà! \[B7]
% \[E-]Ecco il ma\[B-]ttino, \brk \[E-]gioia di sal\[B-]vezza, 
% un \[C]canto sta nas\[D]cendo in \[E-]noi.

% \endverse




% %%%%% RITORNELLO
% \beginchorus
% \textnote{\textbf{Rit.}}

% \[A-]Vieni, o Si\[D7]gnore, \[G7]luce del cam\[C7]mino,
% \[A-]fuoco che nel \[B]cuore 
% ac\[E-]cen\[D]de il \[G]"sì". \[*E7]
% \[A-]Lieto il tuo pas\[D7]saggio, \[G7]ritmi la spe\[C7]ranza,
% \[A-]Padre della \[B7]veri\[E-]tà. \[B-] \[E-] \[B-]

% \endchorus




% %%%%% STROFA
% \beginverse		%Oppure \beginverse* se non si vuole il numero di fianco
% %\memorize 		% <<< DECOMMENTA se si vuole utilizzarne la funzione
% %\chordsoff		% <<< DECOMMENTA se vuoi una strofa senza accordi
% ^Voglio esal^tare \brk il ^nome del Dio ^nostro
% ^grande nella ^fedel^tà! ^
% ^Egli mi ha ^posto \brk ^sull'alto suo ^monte.
% ^Roccia che non ^crolla ^mai.

% \endverse


% %%%%% STROFA
% \beginverse		%Oppure \beginverse* se non si vuole il numero di fianco
% %\memorize 		% <<< DECOMMENTA se si vuole utilizzarne la funzione
% \chordsoff		% <<< DECOMMENTA se vuoi una strofa senza accordi

% Voglio Annunciare il dono crocifisso
% di Cristo, il Dio con noi!
% Perchè della morte lui si prende gioco,
% Figlio che ci attira a sé!

% \endverse


% \endsong

% \fi
% %++++++++++++++++++++++++++++++++++++++++++++++++++++++++++++
% %			FINE CANZONE TRASPOSTA
% %++++++++++++++++++++++++++++++++++++++++++++++++++++++++++++

%-------------------------------------------------------------
%			INIZIO	CANZONE
%-------------------------------------------------------------


%titolo: 	Voi siete di Dio
%autore: 	Balduzzi, Casucci, Savelli 
%tonalita: 	Sol 



%%%%%% TITOLO E IMPOSTAZONI
\beginsong{Voi siete di Dio}[by={Balduzzi, Casucci, Savelli}] 	% <<< MODIFICA TITOLO E AUTORE
\transpose{0} 						% <<< TRASPOSIZIONE #TONI (0 nullo)
\momenti{Ingresso; Comunione; Ringraziamento; Meditazione}							% <<< INSERISCI MOMENTI	
% momenti vanno separati da ; e vanno scelti tra:
% Ingresso; Atto penitenziale; Acclamazione al Vangelo; Dopo il Vangelo; Offertorio; Comunione; Ringraziamento; Fine; Santi; Pasqua; Avvento; Natale; Quaresima; Canti Mariani; Battesimo; Prima Comunione; Cresima; Matrimonio; Meditazione; Spezzare del pane;
\ifchorded
	%\textnote{Tonalità migliore }	% <<< EV COMMENTI (tonalità originale/migliore)
\fi


%%%%%% INTRODUZIONE
\ifchorded
\vspace*{\versesep}
\textnote{Intro: \qquad \qquad  }%(\eighthnote 116) % <<  MODIFICA IL TEMPO
% Metronomo: \eighthnote (ottavo) \quarternote (quarto) \halfnote (due quarti)
\vspace*{-\versesep}
\beginverse*

\nolyrics

%---- Prima riga -----------------------------
\vspace*{-\versesep}
\[G] \[G] \[D] \[G] \rep{2}	 % \[*D] per indicare le pennate, \rep{2} le ripetizioni

%---- Ogni riga successiva -------------------
%\vspace*{-\versesep}
%\[G] \[C]  \[D]	

%---- Ev Indicazioni -------------------------			
\textnote{\textit{(Oppure tutta la strofa)} }	

\endverse
\fi




%%%%% STROFA
\beginverse		%Oppure \beginverse* se non si vuole il numero di fianco
\memorize 		% <<< DECOMMENTA se si vuole utilizzarne la funzione
%\chordsoff		% <<< DECOMMENTA se vuoi una strofa senza accordi

\[G] Tutte le stelle della \[D]not\[G]te 
\[G] le nebulose e le co\[D]me\[E-]te 
\[G] il sole su una ragna\[D]te\[G]la 
\textbf{\[C] è tutto vostro e voi \[G]sie\[D]te di \[G]Dio.} 

\endverse



%%%%% STROFA
\beginverse		%Oppure \beginverse* se non si vuole il numero di fianco
%\memorize 		% <<< DECOMMENTA se si vuole utilizzarne la funzione
%\chordsoff		% <<< DECOMMENTA se vuoi una strofa senza accordi
\transpose{-3} 
^ Tutte le rose della ^vi^ta 
^ il grano, i prati, i fili d’^er^ba 
^il mare, i fiumi, le mon^ta^gne 
\textbf{\[C] è tutto vostro e voi \[E-]sie\[D]te di \[C]Dio.} 

\endverse




%%%%% STROFA
\beginverse		%Oppure \beginverse* se non si vuole il numero di fianco
%\memorize 		% <<< DECOMMENTA se si vuole utilizzarne la funzione
%\chordsoff		% <<< DECOMMENTA se vuoi una strofa senza accordi

^ Tutte le musiche e le ^dan^ze, 
^ i grattacieli, le astro^na^vi 
^ i quadri, i libri, le cul^tu^re
\textbf{\[C] è tutto vostro e voi \[G]sie\[D]te di \[G]Dio.}  


\endverse




%%%%% STROFA
\beginverse		%Oppure \beginverse* se non si vuole il numero di fianco
%\memorize 		% <<< DECOMMENTA se si vuole utilizzarne la funzione
%\chordsoff		% <<< DECOMMENTA se vuoi una strofa senza accordi
\transpose{-3} 
^ Tutte le volte che per^do^no 
^ quando sorrido, quando ^pian^go 
^ quando mi accorgo di chi ^so^no 
\textbf{\[C] è tutto vostro e voi \[E-]sie\[D]te di \[C]Dio.} 

\endverse






%%%%%% EV. FINALE

\beginchorus %oppure \beginverse*
\transpose{-3} 
\[C]E’ tutto nostro e noi \[G]sia\[D]mo di \[G]Dio. 


\endchorus  %oppure \endverse




\endsong
%------------------------------------------------------------
%			FINE CANZONE
%------------------------------------------------------------



%-------------------------------------------------------------
%			INIZIO	CANZONE
%-------------------------------------------------------------


%titolo: 	Voi tutte opere del Signore
%autore: 	Rossi
%tonalita: 	Sol 



%%%%%% TITOLO E IMPOSTAZONI
\beginsong{Voi tutte opere del Signore}[by={Rossi}] 	% <<< MODIFICA TITOLO E AUTORE
\transpose{0} 						% <<< TRASPOSIZIONE #TONI (0 nullo)
\momenti{Ringraziamento}							% <<< INSERISCI MOMENTI	
% momenti vanno separati da ; e vanno scelti tra:
% Ingresso; Atto penitenziale; Acclamazione al Vangelo; Dopo il Vangelo; Offertorio; Comunione; Ringraziamento; Fine; Santi; Pasqua; Avvento; Natale; Quaresima; Canti Mariani; Battesimo; Prima Comunione; Cresima; Matrimonio; Meditazione; Spezzare del pane;
\ifchorded
	%\textnote{Tonalità originale }	% <<< EV COMMENTI (tonalità originale/migliore)
\fi

%%%%%% INTRODUZIONE
\ifchorded
\vspace*{\versesep}
\textnote{Intro: \qquad \qquad  }%(\eighthnote 116) % <<  MODIFICA IL TEMPO
% Metronomo: \eighthnote (ottavo) \quarternote (quarto) \halfnote (due quarti)
\vspace*{-\versesep}
\beginverse*

\nolyrics

%---- Prima riga -----------------------------
\vspace*{-\versesep}
\[D] %\[G] \[B-] \[G]	 % \[*D] per indicare le pennate, \rep{2} le ripetizioni

%---- Ogni riga successiva -------------------
%\vspace*{-\versesep}
%\[G] \[C]  \[D]	

%---- Ev Indicazioni -------------------------			
\textnote{\textit{(Oppure tutta la prima sequenza)} }	

\endverse
\fi

%%%%% STROFA
\beginverse		%Oppure \beginverse* se non si vuole il numero di fianco
\memorize 		% <<< DECOMMENTA se si vuole utilizzarne la funzione
%\chordsoff		% <<< DECOMMENTA se vuoi una strofa senza accordi

\[(A*)]Voi \[D]tut\[(A*)]te \[D]opere \[G]del Si\[D]gnore,  \brk  \[G]bene\[D]dite il Si\[A]gno\[D]re!
\[(A*)]Voi \[D]tut\[(A*)]ti \[D]Angeli \[G]del Si\[D]gnore, \brk  \[G]bene\[D]dite il Si\[A]gno\[D]re!
E \[F#]voi, o \[B-]cieli, \[E]voi, o \[A]acque, \brk  \[D]bene\[A]dite il Si\[E]gno\[A]re!
\[(A*)]Voi \[D]tut\[(A*)]te \[D]opere \[G]del Si\[D]gnore,  \brk  \[G]bene\[D]dite il Si\[A]gno\[D]re!
\endverse

%%%%% STROFA
\beginverse		%Oppure \beginverse* se non si vuole il numero di fianco
%\memorize 		% <<< DECOMMENTA se si vuole utilizzarne la funzione
%\chordsoff		% <<< DECOMMENTA se vuoi una strofa senza accordi

^Voi ^tut^te po^tenze e ^astri del c^ielo,  \brk   ^bene^dite il Si^gno^re!
^Voi ^tut^te p^iogge, ru^giade e ^nevi,  \brk  ^bene^dite il Si^gno^re!
Voi ^sole e ^luna, ^voi, o ^venti,  \brk ^bene^dite il Si^gno^re!
^Voi ^tut^te ^opere ^del Si^gnore,   \brk  ^bene^dite il Si^gno^re!
\endverse


%%%%% STROFA
\beginverse		%Oppure \beginverse* se non si vuole il numero di fianco
%\memorize 		% <<< DECOMMENTA se si vuole utilizzarne la funzione
%\chordsoff		% <<< DECOMMENTA se vuoi una strofa senza accordi

^Voi ^fuo^co e ca^lore, ^freddo e ^caldo,   \brk   ^bene^dite il Si^gno^re!
^voi ^luc^e e ^tenebre, ^ghiaccio e ^freddo,  \brk ^bene^dite il Si^gno^re!
Voi ^notti e ^giorni, ^lampi e ^nubi,  \brk  ^bene^dite il Si^gno^re!
^Voi ^tut^te ^opere ^del Si^gnore,  \brk   ^bene^dite il Si^gno^re!
\endverse


%%%%% STROFA
\beginverse		%Oppure \beginverse* se non si vuole il numero di fianco
%\memorize 		% <<< DECOMMENTA se si vuole utilizzarne la funzione
\chordsoff		% <<< DECOMMENTA se vuoi una strofa senza accordi

La terra tutta lodi il Signore:     benedite il Signore!
Voi tutti viventi lodate il Signore,    benedite il Signore!
Voi monti e colli, mari e fiumi,    benedite il Signore!
Voi tutte opere del Signore,    benedite il Signore!

\endverse


%%%%% STROFA
\beginverse		%Oppure \beginverse* se non si vuole il numero di fianco
%\memorize 		% <<< DECOMMENTA se si vuole utilizzarne la funzione
\chordsoff		% <<< DECOMMENTA se vuoi una strofa senza accordi

Voi tutti pesci e mostri del mare,  benedite il Signore!
Voi tutte belve feroci e armenti,   benedite il Signore!
Voi acque e fonti, voi uccelli,     benedite il Signore!
Voi tutte opere del Signore,    benedite il Signore!

\endverse


%%%%% STROFA
\beginverse		%Oppure \beginverse* se non si vuole il numero di fianco
%\memorize 		% <<< DECOMMENTA se si vuole utilizzarne la funzione
\chordsoff		% <<< DECOMMENTA se vuoi una strofa senza accordi

Voi tutti uomini del Signore,   benedite il Signore!
E voi sacerdoti del Signore,    benedite il Signore!
Voi popolo santo, eletto da Dio,    benedite il Signore!
Voi tutte opere del Signore,    benedite il Signore!
\endverse
\endsong
%------------------------------------------------------------
%			FINE CANZONE
%------------------------------------------------------------
%-------------------------------------------------------------
%			INIZIO	CANZONE
%-------------------------------------------------------------


%titolo: 	Volto dell'Uomo
%autore: 	C. Davide, D. Machetta
%tonalita: 	Mi-



%%%%%% TITOLO E IMPOSTAZONI
\beginsong{Volto dell'Uomo}[by={C. Davide, D. Machetta}] 	% <<< MODIFICA TITOLO E AUTORE
\transpose{0} 						% <<< TRASPOSIZIONE #TONI (0 nullo)
%\preferflats  %SE VOGLIO FORZARE i bemolle come alterazioni
%\prefersharps %SE VOGLIO FORZARE i # come alterazioni
\momenti{Quaresima}							% <<< INSERISCI MOMENTI	
% momenti vanno separati da ; e vanno scelti tra:
% Ingresso; Atto penitenziale; Acclamazione al Vangelo; Dopo il Vangelo; Offertorio; Comunione; Ringraziamento; Fine; Santi; Pasqua; Avvento; Natale; Quaresima; Canti Mariani; Battesimo; Prima Comunione; Cresima; Matrimonio; Meditazione; Spezzare del pane;
\ifchorded
	%\textnote{Tonalità migliore }	% <<< EV COMMENTI (tonalità originale/migliore)
\fi


%%%%%% INTRODUZIONE
\ifchorded
\vspace*{\versesep}
\musicnote{
\begin{minipage}{0.48\textwidth}
\textbf{Intro}
\hfill 
( \halfnote \, 52)   % <<  MODIFICA IL TEMPO
% Metronomo: \eighthnote (ottavo) \quarternote (quarto) \halfnote (due quarti)
\end{minipage}
} 	
\vspace*{-\versesep}
\beginverse*

\nolyrics

%---- Prima riga -----------------------------
\vspace*{-\versesep}
\[E-] \[B-] % \[*D] per indicare le pennate, \rep{2} le ripetizioni

%---- Ogni riga successiva -------------------
\vspace*{-\versesep}
\[D] \[A]  \[B-]	

%---- Ev Indicazioni -------------------------			
%\textnote{\textit{(Oppure tutta la strofa)} }	

\endverse
\fi




%%%%% STROFA
\beginverse		%Oppure \beginverse* se non si vuole il numero di fianco
%\memorize 		% <<< DECOMMENTA se si vuole utilizzarne la funzione
%\chordsoff		% <<< DECOMMENTA se vuoi una strofa senza accordi

\[E-]Volto dell'\[B-]uomo
pene\[D]trato \[A]dal do\[B-]lore,
\[G]volto di \[D]Dio
pene\[E-]trato di \[A]umil\[B]tà,
\[G]scandalo dei \[D]grandi
che con\[A-]fidano nel \[B]mondo,
\[A-]uomo dei do\[E-]lori, pie\[B-7]tà di \[E]noi.

\endverse





%%%%% STROFA
\beginverse		%Oppure \beginverse* se non si vuole il numero di fianco
%\memorize 		% <<< DECOMMENTA se si vuole utilizzarne la funzione
%\chordsoff		% <<< DECOMMENTA se vuoi una strofa senza accordi

\[E-]Volto di \[B-]pace,
di per\[D]dono e \[A]di bon\[B-]tà,
\[G]tu che in si\[D]lenzio
hai \[E-]pagato i \[A]nostri \[B]errori,
\[G]scandalo dei \[D]forti,
di chi ha \[A-]sete di vio\[B]lenza,
\[A-]Cristo Salva\[E-]tore, pie\[B-7]tà di \[E]noi.

\endverse


%%%%% STROFA
\beginverse		%Oppure \beginverse* se non si vuole il numero di fianco
%\memorize 		% <<< DECOMMENTA se si vuole utilizzarne la funzione
%\chordsoff		% <<< DECOMMENTA se vuoi una strofa senza accordi

\[E-]Volto di \[B-]luce,
di vit\[D]toria e \[A]liber\[B-]tà,
\[G]tu hai trac\[D]ciato
i senti\[E-]eri \[A]della \[B]vita;
\[G]spezzi con la \[D]Croce
le bar\[A-]riere della \[B]morte:
\[A-]Figlio di \[E-]Dio, pie\[B-7]tà di \[E]noi.

\endverse



\endsong
%------------------------------------------------------------
%			FINE CANZONE
%------------------------------------------------------------



%WWW
%XXX
%YYY
%ZZZ





%******* END SONGS ENVIRONMENT ******
\setcounter{GlobalSongCounter}{\thesongnum}
\end{songs}


\songchapter{Taizè}
%...............................................................................
%
% ████████╗ █████╗ ██╗███████╗███████╗    
% ╚══██╔══╝██╔══██╗██║╚══███╔╝██╔════╝    
%    ██║   ███████║██║  ███╔╝ █████╗      
%    ██║   ██╔══██║██║ ███╔╝  ██╔══╝      
%    ██║   ██║  ██║██║███████╗███████╗    
%    ╚═╝   ╚═╝  ╚═╝╚═╝╚══════╝╚══════╝ 
% Font ANSI Shadow
%..............................................................................
\begin{songs}{}
\songcolumns{\canzsongcolumsnumber}
\setcounter{songnum}{\theGlobalSongCounter} %set songnum counter, otherwise would be reset

%set the default path inside current folder
\makeatletter
\def\input@path{{Songs/Taize/}}
\makeatother


%***** INSERT SONGS HERE ******

%-------------------------------------------------------------
%			INIZIO	CANZONE
%-------------------------------------------------------------


%titolo: 	Benedicimus te Christe
%autore: 	Taizè
%tonalita: 	Mi--



%%%%%% TITOLO E IMPOSTAZONI
\beginsong{Benedicimus te Christe}[by={Taizè}]	% <<< MODIFICA TITOLO E AUTORE
\transpose{-2} 						% <<< TRASPOSIZIONE #TONI (0 nullo)
\momenti{Ringraziamento;  Meditazione}							% <<< INSERISCI MOMENTI	
% momenti vanno separati da ; e vanno scelti tra:
% Ingresso; Atto penitenziale; Acclamazione al Vangelo; Dopo il Vangelo; Offertorio; Comunione; Ringraziamento; Fine; Santi; Pasqua; Avvento; Natale; Quaresima; Canti Mariani; Battesimo; Prima Comunione; Cresima; Matrimonio; Meditazione;
\ifchorded
	%\textnote{Tonalità originale }	% <<< EV COMMENTI (tonalità originale/migliore)
\fi


%%%%%% INTRODUZIONE
\ifchorded
\vspace*{\versesep}
\textnote{Intro: \qquad \qquad  }%(\eighthnote 116) % << MODIFICA IL TEMPO
% Metronomo: \eighthnote (ottavo) \quarternote (quarto) \halfnote (due quarti)
\vspace*{-\versesep}
\beginverse*

\nolyrics

%---- Prima riga -----------------------------
\vspace*{-\versesep}
\[F#-]

%---- Ogni riga successiva -------------------
%\vspace*{-\versesep}
%\[G] \[C]  \[D]	

%---- Ev Indicazioni -------------------------			
\textnote{\textit{(Oppure tutta la strofa)} }	

\endverse
\fi



%%%%% STROFA
\beginverse*
\memorize 
\[F#-]Adoramus te \[B-*]Chris\[C#]te,
Benedicimus \[F#-*]ti\[E]bi,
\[A*]Quia \[E*]per \[C#]crucem \[D*]tu\[C#]am
\[A*]Rede\[B*]misti \[C#*]mun\[F#-]dum.
\[A*]Quia \[E*]per \[C#*]crucem \[D*]tu\[C#]am
\[A*]Rede\[B*]misti \[C#*]mun\[F#-]dum.
\endverse


\endsong
%------------------------------------------------------------
%			FINE CANZONE
%------------------------------------------------------------
%-------------------------------------------------------------
%			INIZIO	CANZONE
%-------------------------------------------------------------


%titolo: 	Bless the Lord my soul
%autore: 	Taizé
%tonalita: 	Si- 



\beginsong{Bless the lord my soul}[by={Taizé}]
\transpose{0} 						% <<< TRASPOSIZIONE #TONI (0 nullo)
\momenti{Dopo il Vangelo; Meditazione}				% <<< INSERISCI MOMENTI	
\ifchorded
	%\textnote{Tonalità originale }	% <<< EV COMMENTI (tonalità originale/migliore)
\fi
\preferflats

%%%%%% INTRODUZIONE
\ifchorded
\vspace*{\versesep}
\textnote{Intro: \qquad \qquad  }%(\eighthnote 116) % << MODIFICA IL TEMPO
% Metronomo: \eighthnote (ottavo) \quarternote (quarto) \halfnote (due quarti)
\vspace*{-\versesep}
\beginverse*

\nolyrics

%---- Prima riga -----------------------------
\vspace*{-\versesep}
\[B-] \[E]  \[B-]	 % \[*D] per indicare le pennate, \rep{2} le ripetizioni


%---- Ev Indicazioni -------------------------			
\textnote{\textit{(Oppure tutta la strofa)} }	

\endverse
\fi



%%%%% STROFA
\beginverse*
\[B-]Bless the \[E]Lord, my \[B-]soul, \brk and \[G]bless God's \[A]holy \[D]name. \[F#] 
\[B-]Bless the \[E]Lord, my \[B-]soul, \brk who \[G]leads \[A7]me into \[B-]life.
\endverse
\endsong


%-------------------------------------------------------------
%			INIZIO	CANZONE
%-------------------------------------------------------------


%titolo: 	Bonum est confidere
%autore: 	Taizè
%tonalita: 	La-



%%%%%% TITOLO E IMPOSTAZONI
\beginsong{Bonum est confidere}[by={Taizè}]	% <<< MODIFICA TITOLO E AUTORE
%\transpose{-2} 						% <<< TRASPOSIZIONE #TONI (0 nullo)
\momenti{Ringraziamento; Meditazione}							% <<< INSERISCI MOMENTI	
% momenti vanno separati da ; e vanno scelti tra:
% Ingresso; Atto penitenziale; Acclamazione al Vangelo; Dopo il Vangelo; Offertorio; Comunione; Ringraziamento; Fine; Santi; Pasqua; Avvento; Natale; Quaresima; Canti Mariani; Battesimo; Prima Comunione; Cresima; Matrimonio; Meditazione;
\ifchorded
	%\textnote{Tonalità originale }	% <<< EV COMMENTI (tonalità originale/migliore)
\fi


%%%%%% INTRODUZIONE
\ifchorded
\vspace*{\versesep}
\musicnote{
\begin{minipage}{0.48\textwidth}
\textbf{Intro}
\hfill 
%( \eighthnote \, 80)   % <<  MODIFICA IL TEMPO
% Metronomo: \eighthnote (ottavo) \quarternote (quarto) \halfnote (due quarti)
\end{minipage}
} 	
\vspace*{-\versesep}
\beginverse*

\nolyrics

%---- Prima riga -----------------------------
\vspace*{-\versesep}
\[D-]

%---- Ogni riga successiva -------------------
%\vspace*{-\versesep}
%\[G] \[C]  \[D]	

%---- Ev Indicazioni -------------------------			
\textnote{\textit{(oppure tutta la strofa)} }	

\endverse
\fi

\beginverse*
\[D-]Bonum est con\[A]fidere in \[D-]{Do}\[C]{mi}\[F]no
\[G-]Bonum spe\[D-]rare in \[G-]{Do}\[A]{mi}\[D-]no.
\endverse
\endsong


%titolo{Christe Lux mundi}
%autore{Taizé}
%album{}
%tonalita{Re-}
%famiglia{Liturgica}
%gruppo{Canoni_ritornelli}
%momenti{Ritornelli;Taizé}
%identificatore{christe_lux_mundi}
%data_revisione{2011_12_31}
%trascrittore{Francesco Endrici - Manuel Toniato}
\beginsong{Christe Lux mundi}[by={Taizé}]
\beginverse*
\[D-]Christe Lux \[A-]mun\[D-]di, qui \[F]sequi\[G-/B&]tur \[A-]te, 
ha\[F]bebit \[B&]lumen \[C]vi\[G-/B&]t\ae, \[B&]lumen \[G-7]vi\[C]t\ae.
\endverse
\endsong


%-------------------------------------------------------------
%			INIZIO	CANZONE
%-------------------------------------------------------------


%titolo: 	Dona la pace
%autore: 	Taizè
%tonalita: 	La-



%%%%%% TITOLO E IMPOSTAZONI
\beginsong{Dona la pace}[by={Taizè}]	% <<< MODIFICA TITOLO E AUTORE
\transpose{0} 						% <<< TRASPOSIZIONE #TONI (0 nullo)
\momenti{ Meditazione}							% <<< INSERISCI MOMENTI	
% momenti vanno separati da ; e vanno scelti tra:
% Ingresso; Atto penitenziale; Acclamazione al Vangelo; Dopo il Vangelo; Offertorio; Comunione; Ringraziamento; Fine; Santi; Pasqua; Avvento; Natale; Quaresima; Canti Mariani; Battesimo; Prima Comunione; Cresima; Matrimonio; Meditazione;
\ifchorded
	%\textnote{Tonalità originale }	% <<< EV COMMENTI (tonalità originale/migliore)
\fi


%%%%%% INTRODUZIONE
\ifchorded
\vspace*{\versesep}
\textnote{Intro: \qquad \qquad  }%(\eighthnote 116) % << MODIFICA IL TEMPO
% Metronomo: \eighthnote (ottavo) \quarternote (quarto) \halfnote (due quarti)
\vspace*{-\versesep}
\beginverse*

\nolyrics

%---- Prima riga -----------------------------
\vspace*{-\versesep}
\[A-]

%---- Ogni riga successiva -------------------
%\vspace*{-\versesep}
%\[G] \[C]  \[D]	

%---- Ev Indicazioni -------------------------			
\textnote{\textit{(Oppure tutta la strofa)} }	

\endverse
\fi



%%%%% STROFA
\beginverse*
\memorize 


\[A-]Dona la pace, Si\[D-7]gno\[E]re,
\[A-]a chi con\[F]fida in \[E]te.
\[(E)]Dona, \[A-]dona la pace, Si\[D-7]gno\[E]re,
\[C]Do\[D-7]na la  \[E7]pa\[A-]ce.

\endverse


\endsong
%------------------------------------------------------------
%			FINE CANZONE
%------------------------------------------------------------
%-------------------------------------------------------------
%			INIZIO	CANZONE
%-------------------------------------------------------------


%titolo: 	Il Signore è la mia forza
%autore: 	Taizè
%tonalita: 	Re 



%%%%%% TITOLO E IMPOSTAZONI
\beginsong{Il Signore è la mia forza}[by={Taizè}] 	% <<< MODIFICA TITOLO E AUTORE
%\transpose{-3} 						% <<< TRASPOSIZIONE #TONI (0 nullo)
\momenti{Dopo il Vangelo; Meditazione}							% <<< INSERISCI MOMENTI	
% momenti vanno separati da ; e vanno scelti tra:
% Ingresso; Atto penitenziale; Acclamazione al Vangelo; Dopo il Vangelo; Offertorio; Comunione; Ringraziamento; Fine; Santi; Pasqua; Avvento; Natale; Quaresima; Canti Mariani; Battesimo; Prima Comunione; Cresima; Matrimonio; Meditazione;
\ifchorded
	%\textnote{Tonalità originale }	% <<< EV COMMENTI (tonalità originale/migliore)
\fi


%%%%%% INTRODUZIONE
\ifchorded
\vspace*{\versesep}
\textnote{Intro: \qquad \qquad  }%(\eighthnote 116) % << MODIFICA IL TEMPO
% Metronomo: \eighthnote (ottavo) \quarternote (quarto) \halfnote (due quarti)
\vspace*{-\versesep}
\beginverse*

\nolyrics

%---- Prima riga -----------------------------
\vspace*{-\versesep}
\[F]

%---- Ogni riga successiva -------------------
%\vspace*{-\versesep}
%\[G] \[C]  \[D]	

%---- Ev Indicazioni -------------------------			
\textnote{\textit{(oppure tutta la strofa)} }	

\endverse
\fi


\beginverse*
\[D-*]Il \[(C*)]Si\[F]gnore è la mia \[B&*]for\[C]za 
 \[D-*]ed \[(C*)]io \[F]spero in \[C]lui.
Il Si\[B&]gnore è il \[A]Salva\[D-]tor.
In \[C]lui con\[F]fido, non \[B&]ho ti\[C]mor, 
in \[A-]lui con\[D-]fido non \[B&*]ho \[(C*)]ti\[F]mor.
\endverse
\endsong
%------------------------------------------------------------
%			FINE CANZONE
%------------------------------------------------------------



%-------------------------------------------------------------
%			INIZIO	CANZONE
%-------------------------------------------------------------


%titolo: 	In manus tuas Pater
%autore: 	Taizè
%tonalita: 	Re



%%%%%% TITOLO E IMPOSTAZONI
\beginsong{In manus tuas Pater}[by={Taizè}]	% <<< MODIFICA TITOLO E AUTORE
%\transpose{-2} 						% <<< TRASPOSIZIONE #TONI (0 nullo)
\momenti{ Meditazione}							% <<< INSERISCI MOMENTI	
% momenti vanno separati da ; e vanno scelti tra:
% Ingresso; Atto penitenziale; Acclamazione al Vangelo; Dopo il Vangelo; Offertorio; Comunione; Ringraziamento; Fine; Santi; Pasqua; Avvento; Natale; Quaresima; Canti Mariani; Battesimo; Prima Comunione; Cresima; Matrimonio; Meditazione;
\ifchorded
	%\textnote{Tonalità originale }	% <<< EV COMMENTI (tonalità originale\migliore)
\fi


%%%%%% INTRODUZIONE
\ifchorded
\vspace*{\versesep}
\musicnote{
\begin{minipage}{0.48\textwidth}
\textbf{Intro}
\hfill 
%( \eighthnote \, 80)   % <<  MODIFICA IL TEMPO
% Metronomo: \eighthnote (ottavo) \quarternote (quarto) \halfnote (due quarti)
\end{minipage}
} 	
\vspace*{-\versesep}
\beginverse*

\nolyrics

%---- Prima riga -----------------------------
\vspace*{-\versesep}
\[D] \[F#-] \[E-] \[D]

%---- Ogni riga successiva -------------------
%\vspace*{-\versesep}
%\[G] \[C]  \[D]	

%---- Ev Indicazioni -------------------------			
\textnote{\textit{(oppure tutta la strofa)} }	

\endverse
\fi

\beginverse*
In \[D]manus \[F#-]tuas, \[E-]Pat\[D]er 
co\[G]mmendo spiritum \[F#]meum
In \[B-]manus \[F#]tuas \[B-*]P\[A*]at\[D]er 
co\[G]mmendo \[A]spiritum \[D]meum
\endverse
\endsong


%-------------------------------------------------------------
%			INIZIO	CANZONE
%-------------------------------------------------------------

%titolo: Laudate Dominum
%autore: Taizè
%tonalita: Lam 


%%%%%% TITOLO E IMPOSTAZONI
\beginsong{Laudate Dominum}[by={Taizè}] 	% <<< MODIFICA TITOLO E AUTORE
\transpose{0} 						% <<< TRASPOSIZIONE #TONI (0 nullo)
\momenti{Meditazione; Dopo il Vangelo; Acclamazione al Vangelo}							% <<< INSERISCI MOMENTI	
% momenti vanno separati da ; e vanno scelti tra:
% Ingresso; Atto penitenziale; Acclamazione al Vangelo; Dopo il Vangelo; Offertorio; Comunione; Ringraziamento; Fine; Santi; Pasqua; Avvento; Natale; Quaresima; Canti Mariani; Battesimo; Prima Comunione; Cresima; Matrimonio; Meditazione;


%%%%%% INTRODUZIONE
\ifchorded
\vspace*{\versesep}
\musicnote{
\begin{minipage}{0.48\textwidth}
\textbf{Intro}
\hfill 
%( \eighthnote \, 80)   % <<  MODIFICA IL TEMPO
% Metronomo: \eighthnote (ottavo) \quarternote (quarto) \halfnote (due quarti)
\end{minipage}
} 	
\vspace*{-\versesep}
\beginverse*

\nolyrics

%---- Prima riga -----------------------------
\vspace*{-\versesep}
\[A-] 	\[E7]  \rep{2} % \[*D] per indicare le pennate, \rep{2} le ripetizioni

%---- Ogni riga successiva -------------------
%\vspace*{-\versesep}
%\[G] \[C]  \[D]	

%---- Ev Indicazioni -------------------------			
\textnote{\textit{(oppure tutta la strofa)} }	

\endverse
\fi




%%%%% STROFA
\beginverse*
\[A-]Laudate \[E7]Dominum, \[A-]laudate \[G]Dominum, 
\[C]Omnes \[G]gentes, \[A-]Alle\[F*]lu\[E]ia!
\[A-]Laudate \[E7]Dominum, \[A-]laudate \[G]Dominum, 
\[C]Omnes \[G]gentes, \[A-]Alle\[E*]lu\[A-]ia!
\endverse



\endsong
%------------------------------------------------------------
%			FINE CANZONE
%------------------------------------------------------------
%-------------------------------------------------------------
%			INIZIO	CANZONE
%-------------------------------------------------------------

%titolo: Laudate omnes gentes
%autore: Taizè
%tonalita: Re


%%%%%% TITOLO E IMPOSTAZONI
\beginsong{Laudate omnes gentes}[by={Taizè}] 	% <<< MODIFICA TITOLO E AUTORE
\transpose{0} 						% <<< TRASPOSIZIONE #TONI (0 nullo)
\momenti{Meditazione; Ringraziamento}							% <<< INSERISCI MOMENTI	
% momenti vanno separati da ; e vanno scelti tra:
% Ingresso; Atto penitenziale; Acclamazione al Vangelo; Dopo il Vangelo; Offertorio; Comunione; Ringraziamento; Fine; Santi; Pasqua; Avvento; Natale; Quaresima; Canti Mariani; Battesimo; Prima Comunione; Cresima; Matrimonio; Meditazione;


%%%%%% INTRODUZIONE
\ifchorded
\vspace*{\versesep}
\musicnote{
\begin{minipage}{0.48\textwidth}
\textbf{Intro}
\hfill 
%( \eighthnote \, 80)   % <<  MODIFICA IL TEMPO
% Metronomo: \eighthnote (ottavo) \quarternote (quarto) \halfnote (due quarti)
\end{minipage}
} 	
\vspace*{-\versesep}
\beginverse*

\nolyrics

%---- Prima riga -----------------------------
\vspace*{-\versesep}
\[D] 	 % \[*D] per indicare le pennate, \rep{2} le ripetizioni

%---- Ogni riga successiva -------------------
%\vspace*{-\versesep}
%\[G] \[C]  \[D]	

%---- Ev Indicazioni -------------------------			
\textnote{\textit{(oppure tutta la strofa)} }	

\endverse
\fi




%%%%% STROFA
\beginverse*
\[D]Laudate, \[G]omnes \[A]gen\[B-]tes, \quad \[(F#*)]lau\[B-]date \[D]Domi\[A]num.
\[D]Laudate, \[G]omnes \[A]gen\[B-]tes, \quad lau\[E-]date \[A4]Domi\[D]num.
\endverse



\endsong
%------------------------------------------------------------
%			FINE CANZONE
%------------------------------------------------------------
%-------------------------------------------------------------
%			INIZIO	CANZONE
%-------------------------------------------------------------


%titolo: 	Magnificat
%autore: 	Taizè
%tonalita: 	Sol



%%%%%% TITOLO E IMPOSTAZONI
\beginsong{Magnificat}[by={Taizè}]	% <<< MODIFICA TITOLO E AUTORE
\transpose{0} 						% <<< TRASPOSIZIONE #TONI (0 nullo)
\momenti{ Canti Mariani; Meditazione}							% <<< INSERISCI MOMENTI	
% momenti vanno separati da ; e vanno scelti tra:
% Ingresso; Atto penitenziale; Acclamazione al Vangelo; Dopo il Vangelo; Offertorio; Comunione; Ringraziamento; Fine; Santi; Pasqua; Avvento; Natale; Quaresima; Canti Mariani; Battesimo; Prima Comunione; Cresima; Matrimonio; Meditazione;
\ifchorded
	%\textnote{Tonalità originale }	% <<< EV COMMENTI (tonalità originale/migliore)
\fi


%%%%%% INTRODUZIONE
\ifchorded
\vspace*{\versesep}
\textnote{Intro: \qquad \qquad  }%(\eighthnote 116) % << MODIFICA IL TEMPO
% Metronomo: \eighthnote (ottavo) \quarternote (quarto) \halfnote (due quarti)
\vspace*{-\versesep}
\beginverse*

\nolyrics

%---- Prima riga -----------------------------
\vspace*{-\versesep}
\[G]

%---- Ogni riga successiva -------------------
%\vspace*{-\versesep}
%\[G] \[C]  \[D]	

%---- Ev Indicazioni -------------------------			
\textnote{\textit{(Oppure tutta la strofa)} }	

\endverse
\fi



%%%%% STROFA
\beginverse*
\memorize 


\[G]Magnifi\[C]cat, \[D]magnifi\[G]cat, 
\[G]magnifi\[C]cat anima \[D]mea \[G]Dominum.
\[G]Magnifi\[C]cat, \[D7]magnifi\[G]cat, 
\[G]magnifi\[A-]cat anima \[D7]me\[G]a.

\endverse


\endsong
%------------------------------------------------------------
%			FINE CANZONE
%------------------------------------------------------------
%-------------------------------------------------------------
%			INIZIO	CANZONE
%-------------------------------------------------------------

%titolo: Misericordias DOmini
%autore: Taizè
%tonalita: Rem


%%%%%% TITOLO E IMPOSTAZONI
\beginsong{Misericordias Domini}[by={Taizé}] 	% <<< MODIFICA TITOLO E AUTORE
\transpose{0} 						% <<< TRASPOSIZIONE #TONI (0 nullo)
\momenti{Meditazione}							% <<< INSERISCI MOMENTI	
% momenti vanno separati da ; e vanno scelti tra:
% Ingresso; Atto penitenziale; Acclamazione al Vangelo; Dopo il Vangelo; Offertorio; Comunione; Ringraziamento; Fine; Santi; Pasqua; Avvento; Natale; Quaresima; Canti Mariani; Battesimo; Prima Comunione; Cresima; Matrimonio; Meditazione;


%%%%%% INTRODUZIONE
\ifchorded
\vspace*{\versesep}
\textnote{Intro: \qquad \qquad  }%(\eighthnote 116) % << MODIFICA IL TEMPO(\eighthnote 88) 
% Metronomo: \eighthnote (ottavo) \quarternote (quarto) \halfnote (due quarti)
\vspace*{-\versesep}
\beginverse*

\nolyrics

%---- Prima riga -----------------------------
\vspace*{-\versesep}
\[D-] 	 % \[*D] per indicare le pennate, \rep{2} le ripetizioni

%---- Ogni riga successiva -------------------
%\vspace*{-\versesep}
%\[G] \[C]  \[D]	

%---- Ev Indicazioni -------------------------			
\textnote{\textit{(Oppure tutta la strofa)} }	

\endverse
\fi




%%%%% STROFA
\beginverse*
\[D-]Miseri\[A]cordias \[D-]Domi\[C]ni, 
\[F]in ae\[C]ternum \[D-]can\[A]ta\[D-]bo.
\endverse



\endsong
%------------------------------------------------------------
%			FINE CANZONE
%------------------------------------------------------------
%-------------------------------------------------------------
%			INIZIO	CANZONE
%-------------------------------------------------------------


%titolo: 	Nada te turbe
%autore: 	Taizè
%tonalita: 	La-



%%%%%% TITOLO E IMPOSTAZONI
\beginsong{Nada te turbe}[by={Taizè}]	% <<< MODIFICA TITOLO E AUTORE
\transpose{-2} 						% <<< TRASPOSIZIONE #TONI (0 nullo)
\momenti{Ringraziamento; Acclamazione al Vangelo; Meditazione}							% <<< INSERISCI MOMENTI	
% momenti vanno separati da ; e vanno scelti tra:
% Ingresso; Atto penitenziale; Acclamazione al Vangelo; Dopo il Vangelo; Offertorio; Comunione; Ringraziamento; Fine; Santi; Pasqua; Avvento; Natale; Quaresima; Canti Mariani; Battesimo; Prima Comunione; Cresima; Matrimonio; Meditazione;
\ifchorded
	%\textnote{Tonalità originale }	% <<< EV COMMENTI (tonalità originale/migliore)
\fi


%%%%%% INTRODUZIONE
\ifchorded
\vspace*{\versesep}
\textnote{Intro: \qquad \qquad  }%(\eighthnote 116) % << MODIFICA IL TEMPO
% Metronomo: \eighthnote (ottavo) \quarternote (quarto) \halfnote (due quarti)
\vspace*{-\versesep}
\beginverse*

\nolyrics

%---- Prima riga -----------------------------
\vspace*{-\versesep}
\[A-]

%---- Ogni riga successiva -------------------
%\vspace*{-\versesep}
%\[G] \[C]  \[D]	

%---- Ev Indicazioni -------------------------			
\textnote{\textit{(Oppure tutta la strofa)} }	

\endverse
\fi



%%%%% STROFA
\beginverse*
\memorize 
\[A-]Nada te \[D-7]turbe, \[G]nada te e\[C]spante,
\[F]quien a Dios \[D-6]tiene \[E]nada la \[A-]falta.
\[A-]Nada te \[D-7]turbe, \[G]nada te e\[C]spante,
\[F]solo \[D-6]Dios \[E]ba\[A-]sta.
\endverse

%%%%% STROFA
\beginverse*		%Oppure \beginverse* se non si vuole il numero di fianco
%\memorize 		% <<< DECOMMENTA se si vuole utilizzarne la funzione
%\chordsoff		% <<< DECOMMENTA se vuoi una strofa senza accordi


^Niente ti ^turbi, ^niente ti spa^venti:
^chi ha ^Dio, ^niente gli ^manca.
^Niente ti ^turbi, ^niente ti spa^venti:
^solo ^Dio ^bas^ta.

\endverse

\endsong
%------------------------------------------------------------
%			FINE CANZONE
%------------------------------------------------------------
%-------------------------------------------------------------
%			INIZIO	CANZONE
%-------------------------------------------------------------

%titolo: Questa Notte
%autore: Taizè
%tonalita: Lam TRASPOSTA IN Sim


%%%%%% TITOLO E IMPOSTAZONI
\beginsong{Questa Notte}[by={Taizé}] 	% <<< MODIFICA TITOLO E AUTORE
\transpose{2} 						% <<< TRASPOSIZIONE #TONI (0 nullo)
\momenti{Meditazione}							% <<< INSERISCI MOMENTI	
% momenti vanno separati da ; e vanno scelti tra:
% Ingresso; Atto penitenziale; Acclamazione al Vangelo; Dopo il Vangelo; Offertorio; Comunione; Ringraziamento; Fine; Santi; Pasqua; Avvento; Natale; Quaresima; Canti Mariani; Battesimo; Prima Comunione; Cresima; Matrimonio; Meditazione;


%%%%%% INTRODUZIONE
\ifchorded
\vspace*{\versesep}
\textnote{Intro: \qquad \qquad  (\eighthnote 88) }%(\eighthnote 116) % << MODIFICA IL TEMPO(\eighthnote 88) 
% Metronomo: \eighthnote (ottavo) \quarternote (quarto) \halfnote (due quarti)
\vspace*{-\versesep}
\beginverse*

\nolyrics

%---- Prima riga -----------------------------
\vspace*{-\versesep}
\[A-] 	 % \[*D] per indicare le pennate, \rep{2} le ripetizioni

%---- Ogni riga successiva -------------------
%\vspace*{-\versesep}
%\[G] \[C]  \[D]	

%---- Ev Indicazioni -------------------------			
%\textnote{\textit{(Oppure tutta la strofa)} }	

\endverse
\fi




%%%%% STROFA
\beginverse*
Questa \[A-]notte non \[G]è più \[C]notte da\[F]vanti \[E]te:
il \[F]buio come \[A-]luce ri\[E]\[(Am)]splende.
\endverse



\endsong
%------------------------------------------------------------
%			FINE CANZONE
%------------------------------------------------------------
%-------------------------------------------------------------
%			INIZIO	CANZONE
%-------------------------------------------------------------


%titolo: 	Spiritus Jesu Christi
%autore: 	Taizè
%tonalita: 	Re-



%%%%%% TITOLO E IMPOSTAZONI
\beginsong{Spiritus Jesu Christi}[by={Taizè}]	% <<< MODIFICA TITOLO E AUTORE
\transpose{0} 						% <<< TRASPOSIZIONE #TONI (0 nullo)
\momenti{ Meditazione}							% <<< INSERISCI MOMENTI	
% momenti vanno separati da ; e vanno scelti tra:
% Ingresso; Atto penitenziale; Acclamazione al Vangelo; Dopo il Vangelo; Offertorio; Comunione; Ringraziamento; Fine; Santi; Pasqua; Avvento; Natale; Quaresima; Canti Mariani; Battesimo; Prima Comunione; Cresima; Matrimonio; Meditazione;
\ifchorded
	%\textnote{Tonalità originale }	% <<< EV COMMENTI (tonalità originale/migliore)
\fi


%%%%%% INTRODUZIONE
\ifchorded
\vspace*{\versesep}
\musicnote{
\begin{minipage}{0.48\textwidth}
\textbf{Intro}
\hfill 
%( \eighthnote \, 80)   % <<  MODIFICA IL TEMPO
% Metronomo: \eighthnote (ottavo) \quarternote (quarto) \halfnote (due quarti)
\end{minipage}
} 	
\vspace*{-\versesep}
\beginverse*

\nolyrics

%---- Prima riga -----------------------------
\vspace*{-\versesep}
\[D-]

%---- Ogni riga successiva -------------------
%\vspace*{-\versesep}
%\[G] \[C]  \[D]	

%---- Ev Indicazioni -------------------------			
\textnote{\textit{(oppure tutta la strofa)} }	

\endverse
\fi



%%%%% STROFA
\beginverse*
\memorize 

\[B&]Spiritus \[B&*]Je\[A*]su \[D-]Christi, 
Spiritus \[D-*]ca\[C*]ri\[F]tatis, 
\[F*]con\[B&]firmet \[C]cor \[A]tuum; 
\[F*]con\[B&]firmet \[C]cor t\[A]u\[D-]um.

\endverse


\endsong
%------------------------------------------------------------
%			FINE CANZONE
%------------------------------------------------------------
%titolo{Tui amoris ignem}
%autore{Taizé}
%album{}
%tonalita{Mi-}
%famiglia{Liturgica}
%gruppo{Canoni_ritornelli}
%momenti{Ritornelli;Taizé}
%identificatore{tui_amoris_ignem}
%data_revisione{2011_12_31}
%trascrittore{Francesco Endrici - Manuel Toniato}
\beginsong{Tui amoris ignem}[by={Taizé}]
\beginverse*
\[E-]Veni \[B-/D]Sancte \[C]Spi\[A-6]ri\[B]tus \brk \[E-]tui a\[C]moris \[A-]ignem ac\[B]cende. 
\[E-]Veni \[A-]Sancte \[D]Spiri\[G]tus, \[C]Veni \[A-6]San\[E-]cte \[B4]Spi\[B]ri\[E-]tus.
\endverse
\endsong


%titolo{Ubi caritas}
%autore{Taizé}
%album{}
%tonalita{Re}
%famiglia{Liturgica}
%gruppo{Canoni_ritornelli}
%momenti{Ritornelli;Taizé}
%identificatore{ubi_caritas}
%data_revisione{2011_12_31}
%trascrittore{Francesco Endrici - Manuel Toniato}
\beginsong{Ubi caritas}[by={Taizé}]
\beginverse*
\[D]Ubi \[A]cari\[B-]tas et \[G]\[B]a\[E]mo\[A]r
\[D]ubi \[A]cari\[B-]tas \[E-7]Deus  \[A7]ibi \[D]est.
\endverse
\endsong



%-------------------------------------------------------------
%			INIZIO	CANZONE
%-------------------------------------------------------------


%titolo: 	Vieni, spirito creatore
%autore: 	Taizè
%tonalita: 	Si-



%%%%%% TITOLO E IMPOSTAZONI
\beginsong{Vieni, spirito creatore}[by={Taizè}]	% <<< MODIFICA TITOLO E AUTORE
\transpose{0} 						% <<< TRASPOSIZIONE #TONI (0 nullo)
\momenti{ Meditazione}							% <<< INSERISCI MOMENTI	
% momenti vanno separati da ; e vanno scelti tra:
% Ingresso; Atto penitenziale; Acclamazione al Vangelo; Dopo il Vangelo; Offertorio; Comunione; Ringraziamento; Fine; Santi; Pasqua; Avvento; Natale; Quaresima; Canti Mariani; Battesimo; Prima Comunione; Cresima; Matrimonio; Meditazione;
\ifchorded
	%\textnote{Tonalità originale }	% <<< EV COMMENTI (tonalità originale/migliore)
\fi


%%%%%% INTRODUZIONE
\ifchorded
\vspace*{\versesep}
\textnote{Intro: \qquad \qquad  }%(\eighthnote 116) % << MODIFICA IL TEMPO
% Metronomo: \eighthnote (ottavo) \quarternote (quarto) \halfnote (due quarti)
\vspace*{-\versesep}
\beginverse*

\nolyrics

%---- Prima riga -----------------------------
\vspace*{-\versesep}
\[B-]

%---- Ogni riga successiva -------------------
%\vspace*{-\versesep}
%\[G] \[C]  \[D]	

%---- Ev Indicazioni -------------------------			
\textnote{\textit{(Oppure tutta la strofa)} }	

\endverse
\fi



%%%%% STROFA
\beginverse*
\memorize 

\[B-]Vieni Spirito \[E-]crea\[F#]tore,
\[B-]vieni, \[E-]vie\[F#]ni.
\[B-]Vieni Spirito \[E-]crea\[F#]tore,
\[B-]vieni, \[E-]vie\[F#]ni.

\endverse


\endsong
%------------------------------------------------------------
%			FINE CANZONE
%------------------------------------------------------------





%******* END SONGS ENVIRONMENT ******
\setcounter{GlobalSongCounter}{\thesongnum}
\end{songs}


\songchapter{Canti tradizionali di Natale}
%...............................................................................
% ███╗   ██╗ █████╗ ████████╗ █████╗ ██╗     ███████╗
% ████╗  ██║██╔══██╗╚══██╔══╝██╔══██╗██║     ██╔════╝
% ██╔██╗ ██║███████║   ██║   ███████║██║     █████╗  
% ██║╚██╗██║██╔══██║   ██║   ██╔══██║██║     ██╔══╝  
% ██║ ╚████║██║  ██║   ██║   ██║  ██║███████╗███████╗
% ╚═╝  ╚═══╝╚═╝  ╚═╝   ╚═╝   ╚═╝  ╚═╝╚══════╝╚══════╝
% Font: ANSI Shadow                                                                               
%...............................................................................
\begin{songs}{}
\songcolumns{\canzsongcolumsnumber}
\setcounter{songnum}{\theGlobalSongCounter} %set songnum counter, otherwise would be reset

%set the default path inside current folder
\makeatletter
\def\input@path{{Songs/Natale/}}
\makeatother


%***** INSERT SONGS HERE ******

%titolo{Astro del ciel}
%autore{Gruber}
%album{}
%tonalita{La}
%famiglia{Liturgica}
%gruppo{}
%momenti{Natale}
%identificatore{astro_del_ciel}
%data_revisione{2011_12_31}
%trascrittore{Francesco Endrici - Manuel Toniato}
\beginsong{Astro del ciel}[by={Gruber}]
\beginverse
\[A]Astro del ciel, Pargol divin, \[E]mite agnello \[A]redentor!
\[D]Tu che i Vati da \[A]lungi sognar, \brk \[D]tu che angeliche \[A]voci annunziar,
\endverse

\beginchorus
\[E]Luce dona alle \[A]men\[F#-]ti, \brk \[A]pace in\[E]fondi nei \[A]cuor. \rep{2}
\endchorus

\beginverse
%\chordsoff
^Astro del ciel, Pargol divin, ^mite agnello ^redentor,
^tu di stirpe re^gale decor, \brk ^tu virgineo ^mistico fior.
\endverse

\beginverse
%\chordsoff
^Astro del ciel, Pargol divin, ^mite agnello ^redentor
^tu disceso a scon^tare l'error, \brk ^tu sol nato a par^lare d'amor.
\endverse
\endsong




%-------------------------------------------------------------
%			INIZIO	CANZONE
%-------------------------------------------------------------


%titolo: 	Dio è qui
%autore: 	J. Williams
%tonalita: 	Do



%%%%%% TITOLO E IMPOSTAZONI
\beginsong{Dio è qui}[by={Somewhere in my memory — J. Williams}] 	% <<< MODIFICA TITOLO E AUTORE
\transpose{0} 						% <<< TRASPOSIZIONE #TONI (0 nullo)
%\preferflats  %SE VOGLIO FORZARE i bemolle come alterazioni
%\prefersharps %SE VOGLIO FORZARE i # come alterazioni
\momenti{Natale}							% <<< INSERISCI MOMENTI	
% momenti vanno separati da ; e vanno scelti tra:
% Ingresso; Atto penitenziale; Acclamazione al Vangelo; 
%Dopo il Vangelo; Offertorio; Comunione; Ringraziamento; 
%Fine; Santi; Pasqua; Avvento; Natale; Quaresima; Canti Mariani; Battesimo; Prima Comunione; Cresima; Matrimonio; Meditazione; Spezzare del pane;
\ifchorded
	%\textnote{Tonalità migliore }	% <<< EV COMMENTI (tonalità originale/migliore)
\fi




%%%%%% INTRODUZIONE
\ifchorded
\vspace*{\versesep}
\musicnote{
\begin{minipage}{0.48\textwidth}
\textbf{Intro}
\hfill 
%( \eighthnote \, 80)   % <<  MODIFICA IL TEMPO
% Metronomo: \eighthnote (ottavo) \quarternote (quarto) \halfnote (due quarti)
\end{minipage}
} 	
\vspace*{-\versesep}
\beginverse*

\nolyrics

%---- Prima riga -----------------------------
\vspace*{-\versesep}
\[C] \[F] \[C]	 % \[*D] per indicare le pennate, \rep{2} le ripetizioni

%---- Ogni riga successiva -------------------
\vspace*{-\versesep}
\[G] \[C]  \[F] \[G]	

%---- Ev Indicazioni -------------------------			
%\textnote{\textit{(Oppure tutta la strofa)} }	

\endverse
\fi






%%%%% STROFA
\beginverse*		%Oppure \beginverse* se non si vuole il numero di fianco
\memorize 		% <<< DECOMMENTA se si vuole utilizzarne la funzione
%\chordsoff		% <<< DECOMMENTA se vuoi una strofa senza accordi

\[C]Stella della \[F]not\[C]te
\[G]guida il \[C]nostro cam\[F]mi\[G]no
\[C]dentro una \[C7]ca\[F]pan\[E-]na
\[F]dove c’è un \[E-]bimbo che \[D-/F]dor\[G]me.
\endverse
\beginverse*
\[C]Dove il \[D-]cuore \[D]si ride\[G]sta
\[E]e festo\[A-]so \[F]gioi\[G7]rà.
\endverse
\beginverse*
\[C]Una luce \[F]nuo\[C]va
\[G]viene al \[C]mondo a \[F]Betlem\[G]me
\[C]ogni \[C7]uomo at\[F]tra\[C]e
\[F] col suo can\[C]dore
\[F] dona la \[C]vita
\[F] nuova spe\[C]ranza
\[F]Dio \[G]è \[C]qui.

\endverse




\endsong
%------------------------------------------------------------
%			FINE CANZONE
%------------------------------------------------------------


%-------------------------------------------------------------
%			INIZIO	CANZONE
%-------------------------------------------------------------


%titolo: 	Ecco gli angeli cantare
%autore:  C. Wesley
%tonalita: Re 


%%%%%% TITOLO E IMPOSTAZONI
\beginsong{Ecco gli angeli cantare}[by={Hark! The Herald Angels Sing — C. Wesley}] 	% <<< MODIFICA TITOLO E AUTORE
\transpose{0} 						% <<< TRASPOSIZIONE #TONI (0 nullo)
%\preferflats  %SE VOGLIO FORZARE i bemolle come alterazioni
%\prefersharps %SE VOGLIO FORZARE i # come alterazioni
\momenti{Natale}							% <<< INSERISCI MOMENTI	
% momenti vanno separati da ; e vanno scelti tra:
% Ingresso; Atto penitenziale; Acclamazione al Vangelo; Dopo il Vangelo; Offertorio; Comunione; Ringraziamento; Fine; Santi; Pasqua; Avvento; Natale; Quaresima; Canti Mariani; Battesimo; Prima Comunione; Cresima; Matrimonio; Meditazione; Spezzare del pane;
\ifchorded
	%\textnote{$\bigstar$ Tonalità migliore }	% <<< EV COMMENTI (tonalità originale/migliore)
\fi



%%%%%% INTRODUZIONE
\ifchorded
\vspace*{\versesep}
\musicnote{
\begin{minipage}{0.48\textwidth}
\textbf{Intro}
\hfill 
%( \eighthnote \, 80)   % <<  MODIFICA IL TEMPO
% Metronomo: \eighthnote (ottavo) \quarternote (quarto) \halfnote (due quarti)
\end{minipage}
} 	
\vspace*{-\versesep}
\beginverse*

\nolyrics

%---- Prima riga -----------------------------
\vspace*{-\versesep}
\[G]  \[E-]	\[B] \[E-]

%---- Ogni riga successiva -------------------
\vspace*{-\versesep}
\[A] \[D]  \[D*]	\[A*] \[D*]		

%---- Ev Indicazioni -------------------------			
\textnote{\textit{(melodia del ritornello)} }	

\endverse
\fi




%%%%% STROFA
\beginverse		%Oppure \beginverse* se non si vuole il numero di fianco
\memorize 		% <<< DECOMMENTA se si vuole utilizzarne la funzione
%\chordsoff		& <<< DECOMMENTA se vuoi una strofa senza accordi

\[D]Senti l'angelo che \[D]can\[A]ta:
\[D]"Gloria al \[G]nato \[D*]re \[A*]dei \[D*]re!"
\[D]Pace \[B-]vera e vero a\[E-]more
\[A]ha portato al \[A]mon\[(E)]do in\[A]ter.
\[D]Sveglia dunque \[G]le na\[D*]zio\[A*]ni
\[D]alla gioia \[G]del cre\[D*]a\[A*]to
\[G]e con gli \[E-]angeli \[B7]gri\[E-]diam:
\[A]"Cristo è \[D]nato a Be\[A7]tle\[D]hem!"

\endverse




%%%%% RITORNELLO
\beginchorus

\[G]Senti \[E-]l'ange\[E-*]lo \[B7*]can\[E-*]tar:
\[A]"Gloria al \[D]nato \[D*]Re \[A*]dei \[D*]Re!"

\endchorus



%%%%% STROFA
\beginverse		%Oppure \beginverse* se non si vuole il numero di fianco
%\memorize 		% <<< DECOMMENTA se si vuole utilizzarne la funzione
%\chordsoff		& <<< DECOMMENTA se vuoi una strofa senza accordi

^Cristo è l'unico Si^gno^re
^e per ^sempre ^re^gne^rà,
^desi^derio delle ^genti
^vieni presto ^in me^zzo a ^noi.
^Dio velato in ^carne u^ma^na
^hai voluto ^vera^men^te
^abi^tare ^qui con ^noi
^"Salve o ^nostro Emma^nu^el!"

\endverse




%%%%% RITORNELLO
\beginchorus

\[G]Senti \[E-]l'ange\[E-*]lo \[B7*]can\[E-*]tar:
\[A]"Gloria al \[D]nato \[D*]Re \[A*]dei \[D*]Re!"

\endchorus



%%%%% STROFA
\beginverse		%Oppure \beginverse* se non si vuole il numero di fianco
%\memorize 		% <<< DECOMMENTA se si vuole utilizzarne la funzione
%\chordsoff		& <<< DECOMMENTA se vuoi una strofa senza accordi

^Vero Principe di ^pa^ce
^dona al ^mondo ^li^ber^tà,
^vero s^ole di gius^tizia
^sana Tu l'u^ma^ni^tà.
^Tu sei nato ^perchè ^l'uo^mo
^non possa mai ^più mo^ri^re,
^perchè ^l'uomo ^abbia in ^Te
^una sec^onda ^nasci^ta

\endverse




%%%%% RITORNELLO
\beginchorus

\[G]Senti \[E-]l'ange\[E-*]lo \[B7*]can\[E-*]tar:
\[A]"Gloria al \[D]nato \[D*]Re \[A*]dei \[D*]Re!"

\endchorus





\endsong
%------------------------------------------------------------
%			FINE CANZONE
%------------------------------------------------------------




%-------------------------------------------------------------
%			INIZIO	CANZONE
%-------------------------------------------------------------


%titolo: 	Gaudete
%autore: 	
%tonalita: 	Lam 



%%%%%% TITOLO E IMPOSTAZONI
\beginsong{Gaudete!}[by={Canto medievale natalizio}]	% <<< MODIFICA TITOLO E AUTORE
\transpose{0} 						% <<< TRASPOSIZIONE #TONI (0 nullo)
\momenti{Ringraziamento; Natale}							% <<< INSERISCI MOMENTI	
% momenti vanno separati da ; e vanno scelti tra:
% Ingresso; Atto penitenziale; Acclamazione al Vangelo; Dopo il Vangelo; Offertorio; Comunione; Ringraziamento; Fine; Santi; Pasqua; Avvento; Natale; Quaresima; Canti Mariani; Battesimo; Prima Comunione; Cresima; Matrimonio; Meditazione; Spezzare del pane;
\ifchorded
	%\textnote{Tonalità migliore }	% <<< EV COMMENTI (tonalità originale/migliore)
\fi


%%%%%% INTRODUZIONE
\ifchorded
\vspace*{\versesep}
\textnote{Intro: \qquad \qquad  }%(\eighthnote 116) % <<  MODIFICA IL TEMPO
% Metronomo: \eighthnote (ottavo) \quarternote (quarto) \halfnote (due quarti)
\vspace*{-\versesep}
\beginverse*

\nolyrics

%---- Prima riga -----------------------------
\vspace*{-\versesep}
\[A-]\[A-]\[A-] 	 % \[*D] per indicare le pennate, \rep{2} le ripetizioni

%---- Ogni riga successiva -------------------
\vspace*{-\versesep}
\[A-] \[A-] \[E-] \rep{2}

%---- Ev Indicazioni -------------------------			
%\textnote{\textit{(Oppure tutta la strofa)} }	

\endverse
\fi



%%%%% RITORNELLO
\beginchorus
\textnote{\textbf{Rit.}}
\[A-] Gaude\[E-]te, \[A-]gaude\[C]te!
\[G]Christus est \[E-]natus
\[(E-*)]ex \[A-*]Ma\[G]ria \[E-]Virgi\[G]ne.
Gau\[E-*]de\[A-]te!   \rep{2}
\endchorus



%%%%% STROFA
\beginverse		%Oppure \beginverse* se non si vuole il numero di fianco
\memorize 		% <<< DECOMMENTA se si vuole utilizzarne la funzione
%\chordsoff		% <<< DECOMMENTA se vuoi una strofa senza accordi
\[A-7] Tempus adest gratiae,
\[D-]hoc quod optabamus,
carmina laetitiae
\[A-7]devote redamus.
\endverse





%%%%% STROFA
\beginverse		%Oppure \beginverse* se non si vuole il numero di fianco
%\memorize 		% <<< DECOMMENTA se si vuole utilizzarne la funzione
%\chordsoff		% <<< DECOMMENTA se vuoi una strofa senza accordi
^ Deus homo factus est
^natura mirante,
mundus renovatus est
^a Christo regnante.
\endverse




%%%%% STROFA
\beginverse		%Oppure \beginverse* se non si vuole il numero di fianco
%\memorize 		% <<< DECOMMENTA se si vuole utilizzarne la funzione
%\chordsoff		% <<< DECOMMENTA se vuoi una strofa senza accordi
^ Ezechielis porta
^clausa pertransitur
unde Lux est orta
^salus invenitur.
\endverse





%%%%% STROFA
\beginverse		%Oppure \beginverse* se non si vuole il numero di fianco
%\memorize 		% <<< DECOMMENTA se si vuole utilizzarne la funzione
%\chordsoff		% <<< DECOMMENTA se vuoi una strofa senza accordi
^ Ergo nostra contio
^psallat iam in lustro,
benedicat Domino
^salus Regi nostro.
\endverse






\endsong
%------------------------------------------------------------
%			FINE CANZONE
%------------------------------------------------------------




%-------------------------------------------------------------
%			INIZIO	CANZONE
%-------------------------------------------------------------


%titolo: 	Gioia al mondo
%autore: 	G. F. Handel
%tonalita: 	Do 



%%%%%% TITOLO E IMPOSTAZONI
\beginsong{Gioia al mondo}[by={Joy to the world! — G. F. Handel}] 	% <<< MODIFICA TITOLO E AUTORE
\transpose{0} 						% <<< TRASPOSIZIONE #TONI (0 nullo)
\momenti{Natale}					% <<< INSERISCI MOMENTI



%%%%%% INTRODUZIONE
\ifchorded
\vspace*{\versesep}
\textnote{Intro: \qquad \qquad  }%(\eighthnote 116) % << MODIFICA IL TEMPO
% Metronomo: \eighthnote (ottavo) \quarternote (quarto) \halfnote (due quarti)
\vspace*{-\versesep}
\beginverse*

\nolyrics

%---- Prima riga -----------------------------
\vspace*{-\versesep}
\[C] \[C] \[*F] \[*G] \[*C] % \[*D] per indicare le pennate, \rep{2} le ripetizioni

%---- Ogni riga successiva -------------------
\vspace*{-\versesep}
\[F] \[G]  \[C]	

%---- Ev Indicazioni -------------------------			
\textnote{\textit{(Come le prime due righe)} }	

\endverse
\fi




%%%%% STROFA
\beginverse		%Oppure \beginverse* se non si vuole il numero di fianco
\memorize 		% <<< DECOMMENTA se si vuole utilizzarne la funzione
%\chordsoff		& <<< DECOMMENTA se vuoi una strofa senza accordi

\[C]Gioia al mondo il \[*F]Cre\[*G]a\[*C]tor,
è \[F]nato in \[G]mezzo a \[C]noi!
I \[C]cuori allor si aprono
con \[C]gioia alla sua luce
e i \[C]cieli e terra sa\[C]ran 
una \[G]lode all'unico \[G]re,
che go\[F]verna ogni \[C]popolo nei \[*F]se\[*G]co\[*C]li.

\endverse



%%%%%% EV. INTERMEZZO
\ifchorded
\beginverse*
\vspace*{1.3\versesep}
{	

	\nolyrics
	\textnote{Intermezzo strumentale}
	

	%---- Prima riga -----------------------------
	\vspace*{-\versesep}
	\[C] \[C] \[*F] \[*G] \[*C] 

	%---- Ogni riga successiva -------------------
	\vspace*{-\versesep}
	\[F] \[G]  \[C]	
	 
}
\vspace*{\versesep}
\endverse
\fi


%%%%% STROFA
\beginverse		%Oppure \beginverse* se non si vuole il numero di fianco
%\memorize 		% <<< DECOMMENTA se si vuole utilizzarne la funzione
%\chordsoff		% <<< DECOMMENTA se vuoi una strofa senza accordi

^Lui porta a noi la ^ve^ri^tà, 
il ^mondo ^salve^rà! 
La^sciate che ogni uomo 
in^vochi il salvatore 
il ^cielo splende^rà 
e la ^terra esulte^rà 
della ^grazia di^vi - i - na del ^Sal^va^tor!

\endverse



%%%%%% EV. INTERMEZZO
\ifchorded
\beginverse*
\vspace*{1.3\versesep}
{	

	\nolyrics
	\textnote{Intermezzo strumentale}
	

	%---- Prima riga -----------------------------
	\vspace*{-\versesep}
	\[C] \[C] \[*F] \[*G] \[*C] 

	%---- Ogni riga successiva -------------------
	\vspace*{-\versesep}
	\[F] \[G]  \[C]	
	 
}
\vspace*{\versesep}
\endverse
\fi



%%%%% STROFA
\beginverse		%Oppure \beginverse* se non si vuole il numero di fianco
%\memorize 		% <<< DECOMMENTA se si vuole utilizzarne la funzione
%\chordsoff		% <<< DECOMMENTA se vuoi una strofa senza accordi

^Oggi è nato il ^re^den^tor, 
la ^terra e^sulte^rà! 
La^sciate che ogni cuore 
gli ^faccia un po' di posto
il ^cielo splende^rà, 
e la ^terra gioi^rà 
della ^grazia di^vi - i - na del ^Sal^va^tor!

\endverse



%%%%%%FINALE

\beginchorus %oppure \beginverse*
\vspace*{1.3\versesep}
\textnote{Finale \textit{(rallentando)}} %<<< EV. INDICAZIONI

della \[F]grazia di\[C]vi - i - na del \[*F]Sal\[*G]va\[*C]tor!

\endchorus  %oppure \endverse






\endsong
%------------------------------------------------------------
%			FINE CANZONE
%------------------------------------------------------------
%-------------------------------------------------------------
%			INIZIO	CANZONE
%-------------------------------------------------------------


%titolo: 	Gli angeli delle campagne
%autore: 	
%tonalita: 	Do


%%%%%% TITOLO E IMPOSTAZONI
\beginsong{ Gli angeli delle campagne }[by={Les anges dans nos campagnes }] 	% <<< MODIFICA TITOLO E AUTORE
\transpose{0} 						% <<< TRASPOSIZIONE #TONI (0 nullo)
\momenti{Natale}							% <<< INSERISCI MOMENTI	
% momenti vanno separati da ; e vanno scelti tra:
% Ingresso; Atto penitenziale; Acclamazione al Vangelo; Dopo il Vangelo; Offertorio; Comunione; Ringraziamento; Fine; Santi; Pasqua; Avvento; Natale; Quaresima; Canti Mariani; Battesimo; Prima Comunione; Cresima; Matrimonio; Meditazione; Spezzare del pane;
\ifchorded
	\textnote{$\bigstar$ Tonalità singola }	% <<< EV COMMENTI (tonalità originale/migliore)
\fi

%%%%%% INTRODUZIONE
\ifchorded
\vspace*{\versesep}
\musicnote{
\begin{minipage}{0.48\textwidth}
\textbf{Intro}
\hfill 
%( \eighthnote \, 80)   % <<  MODIFICA IL TEMPO
% Metronomo: \eighthnote (ottavo) \quarternote (quarto) \halfnote (due quarti)
\end{minipage}
} 	
\vspace*{-\versesep}
\beginverse*
\nolyrics

%---- Prima riga -----------------------------
\vspace*{-\versesep}
\[C] \[G] \[C]  \rep{2} % \[*D] per indicare le pennate, \rep{2} le ripetizioni

%---- Ogni riga successiva -------------------
%\vspace*{-\versesep}
%\[G] \[C]  \[D]	

%---- Ev Indicazioni -------------------------			
%\textnote{\textit{(Oppure tutta la strofa)} }	

\endverse
\fi

%%%%% STROFA
\beginverse		%Oppure \beginverse* se non si vuole il numero di fianco
\memorize 		% <<< DECOMMENTA se si vuole utilizzarne la funzione
%\chordsoff		% <<< DECOMMENTA se vuoi una strofa senza accordi

\[C]Gli angeli del\[G]le cam\[C]pagne 
\[C]cantano l’inno "\[G]Gloria in ci\[C]el !"
\[C]e l’eco del\[G]le mon\[C]tagne 
\[C]ripete il canto \[G]dei fe\[C]del.
\endverse

%%%%% RITORNELLO
\beginchorus
\textnote{\textbf{Rit.}}

\[C]Glo  \[A]o  o  \[D-]o  \[G]o  o \[C]o \[F]o o \[D]o \[G7]ria         
\[C*]in \[G*]ex \[C*]cel \[F*]sis \[C]De \[G]o
\[C]Glo  \[A]o  o  \[D-]o  \[G]o  o \[C]o \[F]o o \[D]o \[G7]ria         
\[C*]in \[G*]ex \[C*]cel \[F*]sis \[C]De-\[G]e-\[C]o
\endchorus

%%%%% STROFA
\beginverse		%Oppure \beginverse* se non si vuole il numero di fianco
%\memorize 		% <<< DECOMMENTA se si vuole utilizzarne la funzione
%\chordsoff		% <<< DECOMMENTA se vuoi una strofa senza accordi

^O pastori ^che can^tate
^dite il perché di ^tanto o^nor.
^Qual Signor, qua^le pro^feta
^merita questo ^gran splen^dor.

\endverse

%%%%% STROFA
\beginverse		%Oppure \beginverse* se non si vuole il numero di fianco
%\memorize 		% <<< DECOMMENTA se si vuole utilizzarne la funzione
%\chordsoff		% <<< DECOMMENTA se vuoi una strofa senza accordi

^Oggi è nato in ^una s^talla
^nella notturna os^curi^tà.
^Egli è il Verbo, ^s’è incar^nato
^e venne in questa ^pover^tà.

\endverse

\endsong
%------------------------------------------------------------
%			FINE CANZONE
%------------------------------------------------------------






%++++++++++++++++++++++++++++++++++++++++++++++++++++++++++++
%			CANZONE TRASPOSTA
%++++++++++++++++++++++++++++++++++++++++++++++++++++++++++++
\ifchorded
%decremento contatore per avere stesso numero
\addtocounter{songnum}{-1} 
\beginsong{Gli angeli delle campagne}[by={Les anges dans nos campagnes — Autore ignoto}] 	% <<< \transpose{0} 						% <<< TRASPOSIZIONE #TONI + - (0 nullo)
%\preferflats  %SE VOGLIO FORZARE i bemolle come alterazioni
%\prefersharps %SE VOGLIO FORZARE i # come alterazioni
\ifchorded
	\textnote{$\triangle$ Tonalità crescente}	% <<< EV COMMENTI (tonalità originale/migliore)
\fi

%%%%%% INTRODUZIONE
\ifchorded
\vspace*{\versesep}
\musicnote{
\begin{minipage}{0.48\textwidth}
\textbf{Intro}
\hfill 
%( \eighthnote \, 80)   % <<  MODIFICA IL TEMPO
% Metronomo: \eighthnote (ottavo) \quarternote (quarto) \halfnote (due quarti)
\end{minipage}
} 	
\vspace*{-\versesep}
\beginverse*
\nolyrics

%---- Prima riga -----------------------------
\vspace*{-\versesep}
\[C] \[G] \[C]  \rep{2} % \[*D] per indicare le pennate, \rep{2} le ripetizioni

%---- Ogni riga successiva -------------------
%\vspace*{-\versesep}
%\[G] \[C]  \[D]	

%---- Ev Indicazioni -------------------------			
%\textnote{\textit{(Oppure tutta la strofa)} }	

\endverse
\fi

%%%%% STROFA
\beginverse		%Oppure \beginverse* se non si vuole il numero di fianco
\memorize 		% <<< DECOMMENTA se si vuole utilizzarne la funzione
%\chordsoff		% <<< DECOMMENTA se vuoi una strofa senza accordi

\[C]Gli angeli del\[G]le cam\[C]pagne 
\[C]cantano l’inno "\[G]Gloria in ci\[C]el !"
\[C]e l’eco del\[G]le mon\[C]tagne 
\[C]ripete il canto \[G]dei fe\[C]del.
\endverse

%%%%% RITORNELLO
\beginchorus
\textnote{\textbf{Rit.}}

\[C]Glo  \[A]o  o  \[D-]o  \[G]o  o \[C]o \[F]o o \[D]o \[G7]ria         
\[C*]in \[G*]ex \[C*]cel \[F*]sis \[C]De \[G]o
\[C]Glo  \[A]o  o  \[D-]o  \[G]o  o \[C]o \[F]o o \[D]o \[G7]ria         
\[C*]in \[G*]ex \[C*]cel \[F*]sis \[C]De-\[G]e-\[C]o \qquad \[A]
\endchorus


%%%%%% EV. INTERMEZZO
\beginverse*
\vspace*{1.3\versesep}
{
	\nolyrics
	\textnote{Intermezzo strumentale}
	
	\ifchorded

	%---- Prima riga -----------------------------
	\vspace*{-\versesep}
	\[D] \[A]  \[D]	 \rep{2}

	%---- Ogni riga successiva -------------------
	%\vspace*{-\versesep}
	%\[G] \[C]  \[D]	


	\fi
	%---- Ev Indicazioni -------------------------			
	%\textnote{\textit{(ripetizione della strofa)}} 
	 
}
\vspace*{\versesep}
\endverse
%%%%% STROFA
\beginverse		%Oppure \beginverse* se non si vuole il numero di fianco
%\memorize 		% <<< DECOMMENTA se si vuole utilizzarne la funzione
%\chordsoff		% <<< DECOMMENTA se vuoi una strofa senza accordi
\transpose{2}

^O pastori ^che can^tate
^dite il perché di ^tanto o^nor.
^Qual Signor, qua^le pro^feta
^merita questo ^gran splen^dor.

\endverse


%%%%% RITORNELLO
\beginchorus
\textnote{\textbf{Rit.}}
\transpose{2}
\[C]Glo  \[A]o  o  \[D-]o  \[G]o  o \[C]o \[F]o o \[D]o \[G7]ria         
\[C*]in \[G*]ex \[C*]cel \[F*]sis \[C]De \[G]o
\[C]Glo  \[A]o  o  \[D-]o  \[G]o  o \[C]o \[F]o o \[D]o \[G7]ria         
\[C*]in \[G*]ex \[C*]cel \[F*]sis \[C]De-\[G]e-\[C]o \qquad \[A]
\endchorus


%%%%%% EV. INTERMEZZO
\beginverse*
\vspace*{1.3\versesep}
{
	\nolyrics
	\textnote{Intermezzo strumentale}
	
	\ifchorded

	%---- Prima riga -----------------------------
	\vspace*{-\versesep}
	\[E] \[B]  \[E]	 \rep{2}

	%---- Ogni riga successiva -------------------
	%\vspace*{-\versesep}
	%\[G] \[C]  \[D]	


	\fi
	%---- Ev Indicazioni -------------------------			
	%\textnote{\textit{(ripetizione della strofa)}} 
	 
}
\vspace*{\versesep}
\endverse

%%%%% STROFA
\beginverse		%Oppure \beginverse* se non si vuole il numero di fianco
%\memorize 		% <<< DECOMMENTA se si vuole utilizzarne la funzione
%\chordsoff		% <<< DECOMMENTA se vuoi una strofa senza accordi
\transpose{4}
^Oggi è nato in ^una s^talla
^nella notturna os^curi^tà.
^Egli è il Verbo, ^s’è incar^nato
^e venne in questa ^pover^tà.

\endverse

%%%%% RITORNELLO
\beginchorus
\textnote{\textbf{Rit.}}
\transpose{4}
\[C]Glo  \[A]o  o  \[D-]o  \[G]o  o \[C]o \[F]o o \[D]o \[G7]ria         
\[C*]in \[G*]ex \[C*]cel \[F*]sis \[C]De \[G]o
\[C]Glo  \[A]o  o  \[D-]o  \[G]o  o \[C]o \[F]o o \[D]o \[G7]ria         
\[C*]in \[G*]ex \[C*]cel \[F*]sis \[C]De-\[G]e-\[C]o!
\endchorus

\endsong
\fi
%++++++++++++++++++++++++++++++++++++++++++++++++++++++++++++
%			FINE CANZONE TRASPOSTA
%++++++++++++++++++++++++++++++++++++++++++++++++++++++++++++

%-------------------------------------------------------------
%			INIZIO	CANZONE
%-------------------------------------------------------------


%titolo: 	In notte placida
%autore: 	M. Frisina
%tonalita: 	Sol 



%%%%%% TITOLO E IMPOSTAZONI
\beginsong{In notte placida}[by={M. Frisina}] 	% <<< MODIFICA TITOLO E AUTORE
\transpose{0} 						% <<< TRASPOSIZIONE #TONI (0 nullo)
%\preferflats  %SE VOGLIO FORZARE i bemolle come alterazioni
%\prefersharps %SE VOGLIO FORZARE i # come alterazioni
\momenti{Natale}							% <<< INSERISCI MOMENTI	
% momenti vanno separati da ; e vanno scelti tra:
% Ingresso; Atto penitenziale; Acclamazione al Vangelo; Dopo il Vangelo; Offertorio; Comunione; Ringraziamento; Fine; Santi; Pasqua; Avvento; Natale; Quaresima; Canti Mariani; Battesimo; Prima Comunione; Cresima; Matrimonio; Meditazione; Spezzare del pane;
\ifchorded
	%\textnote{Tonalità migliore }	% <<< EV COMMENTI (tonalità originale/migliore)
\fi


%%%%%% INTRODUZIONE
\ifchorded
\vspace*{\versesep}
\textnote{Intro: \qquad \qquad  }%(\eighthnote 116) % <<  MODIFICA IL TEMPO
% Metronomo: \eighthnote (ottavo) \quarternote (quarto) \halfnote (due quarti)
\vspace*{-\versesep}
\beginverse*

\nolyrics

%---- Prima riga -----------------------------
\vspace*{-\versesep}
\[C] \[G] \[C] \rep{2}	 % \[*D] per indicare le pennate, \rep{2} le ripetizioni

%---- Ogni riga successiva -------------------
%\vspace*{-\versesep}
%\[G] \[C]  \[D]	

%---- Ev Indicazioni -------------------------			
%\textnote{\textit{(Oppure tutta la strofa)} }	

\endverse
\fi




%%%%% STROFA
\beginverse		%Oppure \beginverse* se non si vuole il numero di fianco
\memorize 		% <<< DECOMMENTA se si vuole utilizzarne la funzione
%\chordsoff		% <<< DECOMMENTA se vuoi una strofa senza accordi
\[C]In \[G]notte \[C]placida, per \[C]mu\[G]to sen\[C]tier,
dai campi del \[F]ciel è discese l’A\[C]mor,
all’alme fe\[F]deli il Reden\[G]tor!
\endverse
\beginverse*
^Nell’^aura è il ^palpito d’un ^gran^de mis^ter:
del nuovo Isra^el è nato il Si^gnor,
il fiore più ^bello ^dei nostri \[C]fior!
del nuovo Isra\[F]el è nato il Si\[C]gnor,
il fiore più \[F]bello \[G]dei nostri \[C]fior!
\endverse



%%%%% RITORNELLO
\beginchorus
\textnote{\textbf{Rit.}}

\[C-]Cantate, o \[G]popoli, \[C-]gloria all’Al\[G]tissimo
\[C-]l’animo ap\[G]rite a spe\[C-]ranza ed a\[G]mor! \rep{2}

\endchorus



%%%%% STROFA
\beginverse		%Oppure \beginverse* se non si vuole il numero di fianco
%\memorize 		% <<< DECOMMENTA se si vuole utilizzarne la funzione
%\chordsoff		% <<< DECOMMENTA se vuoi una strofa senza accordi

^Se l’^aura è ^gelida, se ^fos^co è il c^iel,
oh, vieni al mio ^cuore, vieni a po^sar,
ti vò col mio a^more riscal^dar.
\endverse
\beginverse*
^Se il ^fieno è ^rigido, se il ^ven^to è cru^del,
un cuore che t’^ama voglio a Te ^dar,
un cuor che Te ^brama, ^Gesù cul\[C]lar.
un cuore che t’\[F]ama voglio a Te \[C]dar,
un cuor che Te \[F]brama, \[G]Gesù cul\[C]lar.
\endverse




\endsong
%------------------------------------------------------------
%			FINE CANZONE
%------------------------------------------------------------
%-------------------------------------------------------------
%			INIZIO	CANZONE
%-------------------------------------------------------------

%titolo: In questa notte splendida
%autore: Sequeri
%tonalita: Mi e Sol 


%%%%%% TITOLO E IMPOSTAZONI
\beginsong{In questa notte splendida}[by={Chieffo}] 	% <<< MODIFICA TITOLO E AUTORE
\transpose{5} 						% <<< TRASPOSIZIONE #TONI (0 nullo)
\momenti{Natale}							% <<< INSERISCI MOMENTI
\ifchorded
	\textnote{Tonalità migliore per le bambine}	% <<< EV COMMENTI (tonalità originale/migliore)
\fi


%%%%%% INTRODUZIONE
\ifchorded
\vspace*{\versesep}
\textnote{Intro: \qquad \qquad  }%(\eighthnote 116) % << MODIFICA IL TEMPO
% Metronomo: \eighthnote (ottavo) \quarternote (quarto) \halfnote (due quarti)
\musicnote{\textit{Dolcissimo, partendo molto piano}}
\vspace*{-\versesep}
\beginverse*

\nolyrics

%---- Prima riga -----------------------------
\vspace*{-\versesep}
\[E] \[A]  \[E]  \[B] 

%---- Ogni riga successiva -------------------
\vspace*{-\versesep}
\[E] \[A]  \[E]  \[B] \[E]



\endverse
\fi






%%%%% STROFA
\beginverse		%Oppure \beginverse* se non si vuole il numero di fianco
\memorize 		% <<< DECOMMENTA se si vuole utilizzarne la funzione
%\chordsoff		% <<< DECOMMENTA se vuoi una strofa senza accordi

In \[E]questa notte \[A]splendida
di \[E]luce e di chia\[B]ror
il \[E]nostro cuore \[A]trepida,
è \[E]nato il \[B]Salva\[E]tor.
Un \[A]bimbo picco\[E]lissimo
le \[A]porte ci apri\[E]rà
del \[E]cielo dell'Al\[A]tissimo
nel\[E]la sua \[B]veri\[E]tà.

\endverse

%%%%%% EV. INTERMEZZO
\beginverse*
\vspace*{1.3\versesep}
{
	\nolyrics
	\textnote{Intermezzo strumentale}
	\musicnote{\textit{(Crescendo di intensità)}} 
	 
	\ifchorded

	%---- Prima riga -----------------------------
	\vspace*{-\versesep}
	\[E] \[A]  \[E]  \[B] 

	%---- Ogni riga successiva -------------------
	\vspace*{-\versesep}
	\[E] \[A]  \[E]  \[B] \[E]

	\fi

}
\vspace*{\versesep}
\endverse

%%%%% STROFA
\beginverse		%Oppure \beginverse* se non si vuole il numero di fianco
%\memorize 		% <<< DECOMMENTA se si vuole utilizzarne la funzione
%\chordsoff		% <<< DECOMMENTA se vuoi una strofa senza accordi

Svegli^atevi dal ^sonno,
corr^ete coi pas^tor,
è ^notte di mi^racoli,
di ^grazia e ^di stu^por.
A^sciuga le tue ^lacrime,
non ^piangere per^chè
Ge^sù nostro ca^rissimo 
è ^nato an^che per ^te.

\endverse

\transpose{3}



%%%%%% EV. INTERMEZZO
\beginverse*
\vspace*{1.3\versesep}
{
	\nolyrics
	\textnote{Intermezzo strumentale}
	\musicnote{\textit{(Più forte, cambia la tonalità)}} 
	 
	\ifchorded

	%---- Prima riga -----------------------------
	\vspace*{-\versesep}
	\[E] \[A]  \[E]  \[B] 

	%---- Ogni riga successiva -------------------
	\vspace*{-\versesep}
	\[E] \[A]  \[E]  \[B] \[E]

	\fi

}
\vspace*{\versesep}
\endverse


%%%%% STROFA
\beginverse		%Oppure \beginverse* se non si vuole il numero di fianco
%\memorize 		% <<< DECOMMENTA se si vuole utilizzarne la funzione
%\chordsoff		% <<< DECOMMENTA se vuoi una strofa senza accordi

In ^questa notte ^limpida
di ^gloria e di splen^dor,
il ^nostro cuore ^trepida
è ^nato il ^Salva^tor.
Ge^sù nostro ca^rissimo
le ^porte ci apri^rà,

\musicnote{\textit{Rallentando}}
il ^figlio dell'Al^tissimo
con ^no\[E]i sem^pr\[B]e sa^r\[E]à.

\endverse





\endsong
%------------------------------------------------------------
%			FINE CANZONE
%------------------------------------------------------------

% %++++++++++++++++++++++++++++++++++++++++++++++++++++++++++++
% %			CANZONE TRASPOSTA
% %++++++++++++++++++++++++++++++++++++++++++++++++++++++++++++
% \ifchorded
% %decremento contatore per avere stesso numero
% \addtocounter{songnum}{-1} 
% \beginsong{In questa notte splendida}[by={Chieffo}] 	% <<< MODIFICA TITOLO E AUTORE
% \transpose{0} 						% <<< TRASPOSIZIONE #TONI + - (0 nullo)
% \ifchorded
% 	\textnote{Tonalità originale}	% <<< EV COMMENTI (tonalità originale/migliore)
% \fi



% %%%%%% INTRODUZIONE
% \ifchorded
% \vspace*{\versesep}
% \textnote{Intro: \qquad \qquad  }%(\eighthnote 116) % << MODIFICA IL TEMPO
% % Metronomo: \eighthnote (ottavo) \quarternote (quarto) \halfnote (due quarti)
% \musicnote{\textit{Dolcissimo, partendo molto piano}}
% \vspace*{-\versesep}
% \beginverse*

% \nolyrics

% %---- Prima riga -----------------------------
% \vspace*{-\versesep}
% \[E] \[A]  \[E]  \[B] 

% %---- Ogni riga successiva -------------------
% \vspace*{-\versesep}
% \[E] \[A]  \[E]  \[B] \[E]



% \endverse
% \fi






% %%%%% STROFA
% \beginverse		%Oppure \beginverse* se non si vuole il numero di fianco
% \memorize 		% <<< DECOMMENTA se si vuole utilizzarne la funzione
% %\chordsoff		% <<< DECOMMENTA se vuoi una strofa senza accordi

% In \[E]questa notte \[A]splendida
% di \[E]luce e di chia\[B]ror
% il \[E]nostro cuore \[A]trepida,
% è \[E]nato il \[B]Salva\[E]tor.
% Un \[A]bimbo picco\[E]lissimo
% le \[A]porte ci apri\[E]rà
% del \[E]cielo dell'Al\[A]tissimo
% nel\[E]la sua \[B]veri\[E]tà.

% \endverse

% %%%%%% EV. INTERMEZZO
% \beginverse*
% \vspace*{1.3\versesep}
% {
% 	\nolyrics
% 	\textnote{Intermezzo strumentale}
% 	\musicnote{\textit{(Crescendo di intensità)}} 
	 
% 	\ifchorded

% 	%---- Prima riga -----------------------------
% 	\vspace*{-\versesep}
% 	\[E] \[A]  \[E]  \[B] 

% 	%---- Ogni riga successiva -------------------
% 	\vspace*{-\versesep}
% 	\[E] \[A]  \[E]  \[B] \[E]

% 	\fi

% }
% \vspace*{\versesep}
% \endverse

% %%%%% STROFA
% \beginverse		%Oppure \beginverse* se non si vuole il numero di fianco
% %\memorize 		% <<< DECOMMENTA se si vuole utilizzarne la funzione
% %\chordsoff		% <<< DECOMMENTA se vuoi una strofa senza accordi

% Svegli^atevi dal ^sonno,
% corr^ete coi pas^tor,
% è ^notte di mi^racoli,
% di ^grazia e ^di stu^por.
% A^sciuga le tue ^lacrime,
% non ^piangere per^chè
% Ge^sù nostro ca^rissimo 
% è ^nato an^che per ^te.

% \endverse

% \transpose{3}



% %%%%%% EV. INTERMEZZO
% \beginverse*
% \vspace*{1.3\versesep}
% {
% 	\nolyrics
% 	\textnote{Intermezzo strumentale}
% 	\musicnote{\textit{(Più forte, cambia la tonalità)}} 
	 
% 	\ifchorded

% 	%---- Prima riga -----------------------------
% 	\vspace*{-\versesep}
% 	\[E] \[A]  \[E]  \[B] 

% 	%---- Ogni riga successiva -------------------
% 	\vspace*{-\versesep}
% 	\[E] \[A]  \[E]  \[B] \[E]

% 	\fi

% }
% \vspace*{\versesep}
% \endverse


% %%%%% STROFA
% \beginverse		%Oppure \beginverse* se non si vuole il numero di fianco
% %\memorize 		% <<< DECOMMENTA se si vuole utilizzarne la funzione
% %\chordsoff		% <<< DECOMMENTA se vuoi una strofa senza accordi

% In ^questa notte ^limpida
% di ^gloria e di splen^dor,
% il ^nostro cuore ^trepida
% è ^nato il ^Salva^tor.
% Ge^sù nostro ca^rissimo
% le ^porte ci apri^rà,

% \musicnote{\textit{Rallentando}}
% il ^figlio dell'Al^tissimo
% con ^no\[E]i sem^pr\[B]e sa^r\[E]à.

% \endverse





% \endsong
% \fi
% %++++++++++++++++++++++++++++++++++++++++++++++++++++++++++++
% %			FINE CANZONE TRASPOSTA
% %++++++++++++++++++++++++++++++++++++++++++++++++++++++++++++
%-------------------------------------------------------------
%			INIZIO	CANZONE
%-------------------------------------------------------------


%titolo: 	Là sulla montagna
%autore: 	Daniele Ricci
%tonalita: 	Sol 



%%%%%% TITOLO E IMPOSTAZONI
\beginsong{Là là sulla montagna}[by={Go Tell It on the Mountain, J. W. Work jr.}] 	% <<< MODIFICA TITOLO E AUTORE
\transpose{0} 						% <<< TRASPOSIZIONE #TONI (0 nullo)
\momenti{Natale}							% <<< INSERISCI MOMENTI	
% momenti vanno separati da ; e vanno scelti tra:
% Ingresso; Atto penitenziale; Acclamazione al Vangelo; Dopo il Vangelo; Offertorio; Comunione; Ringraziamento; Fine; Santi; Pasqua; Avvento; Natale; Quaresima; Canti Mariani; Battesimo; Prima Comunione; Cresima; Matrimonio; Meditazione;
\ifchorded
	%\textnote{Tonalità originale }	% <<< EV COMMENTI (tonalità originale/migliore)
\fi


%%%%%% INTRODUZIONE
\ifchorded
\vspace*{\versesep}
\textnote{Intro: \qquad \qquad  }%(\eighthnote 116) % << MODIFICA IL TEMPO
% Metronomo: \eighthnote (ottavo) \quarternote (quarto) \halfnote (due quarti)
\vspace*{-\versesep}
\beginverse*

\nolyrics

%---- Prima riga -----------------------------
\vspace*{-\versesep}
\[F] \[B&]  \[F]	 % \[*D] per indicare le pennate, \rep{2} le ripetizioni

%---- Ogni riga successiva -------------------
%\vspace*{-\versesep}
%\[G] \[C]  \[D]	

%---- Ev Indicazioni -------------------------			
%\textnote{\textit{(Oppure tutta la strofa)} }	

\endverse
\fi


%%%%% RITORNELLO
\beginchorus
\textnote{\textbf{Rit.}}

\[F]Là, \[B&]là sulla mon\[F]ta\[D-]gna,
\[B&]sulle col\[C7]line \[F*]vai ad \[B&*]annun\[F*]ziar \[B&*]
\[F]che \[B&]il Signore è \[F]na\[D-]to, è \[B&]nato, 
\[C7]nato per \[F*]\[B&*]no\[F]i.

\endchorus

%%%%% STROFA
\beginverse		%Oppure \beginverse* se non si vuole il numero di fianco
\memorize 		% <<< DECOMMENTA se si vuole utilizzarne la funzione
%\chordsoff		& <<< DECOMMENTA se vuoi una strofa senza accordi

Pa\[F]stori \[A-]che re\[D-]state 
sui \[B&]monti a \[B&-]vigi\[F]lar
la \[F]luce \[A-]voi ve\[D-]dete, 
la \[G]stella \[7]di Ge\[C]sù.

\endverse








%%%%% STROFA
\beginverse		%Oppure \beginverse* se non si vuole il numero di fianco
%\memorize 		% <<< DECOMMENTA se si vuole utilizzarne la funzione
%\chordsoff		% <<< DECOMMENTA se vuoi una strofa senza accordi

Se il ^nostro ^canto è im^menso, 
pa^store ^non tre^mar
noi ^Ange^li can^tiamo, 
è ^nato il ^Salva^tor.

\endverse

%%%%% STROFA
\beginverse		%Oppure \beginverse* se non si vuole il numero di fianco
%\memorize 		% <<< DECOMMENTA se si vuole utilizzarne la funzione
\chordsoff		% <<< DECOMMENTA se vuoi una strofa senza accordi

In una mangiatoia,
un bimbo aspetterà
che l’uomo ancor ritrovi, 
la strada dell’amor.

\endverse

\endsong
%------------------------------------------------------------
%			FINE CANZONE
%------------------------------------------------------------



%-------------------------------------------------------------
%			INIZIO	CANZONE
%-------------------------------------------------------------


%titolo: 	Nella notte sbocciò
%autore: 	The first Nowell — Sir D. V. Willcocks, F. Filisetti.
%tonalita: 	Do 



%%%%%% TITOLO E IMPOSTAZONI
\beginsong{Nella notte sbocciò}[by={The first Nowell — Sir D. V. Willcocks\ifacinquevert \iftwocolumns \else , F. Filisetti\fi\fi}] 	% <<< MODIFICA TITOLO E AUTORE
\transpose{0} 						% <<< TRASPOSIZIONE #TONI (0 nullo)
%\preferflats  %SE VOGLIO FORZARE i bemolle come alterazioni
%\prefersharps %SE VOGLIO FORZARE i # come alterazioni
\momenti{Natale}							% <<< INSERISCI MOMENTI	
% momenti vanno separati da ; e vanno scelti tra:
% Ingresso; Atto penitenziale; Acclamazione al Vangelo; Dopo il Vangelo; Offertorio; Comunione; Ringraziamento; Fine; Santi; Pasqua; Avvento; Natale; Quaresima; Canti Mariani; Battesimo; Prima Comunione; Cresima; Matrimonio; Meditazione; Spezzare del pane;
\ifchorded
	%\textnote{Tonalità migliore }	% <<< EV COMMENTI (tonalità originale/migliore)
\fi


%%%%%% INTRODUZIONE
\ifchorded
\vspace*{\versesep}
\musicnote{
\begin{minipage}{0.48\textwidth}
\textbf{Intro}
\hfill 
%( \eighthnote \, 80)   % <<  MODIFICA IL TEMPO
% Metronomo: \eighthnote (ottavo) \quarternote (quarto) \halfnote (due quarti)
\end{minipage}
} 	
\vspace*{-\versesep}
\beginverse*

\nolyrics

%---- Prima riga -----------------------------
\vspace*{-\versesep}
\[C] \[E-] \[F] \[C]


%---- Ev Indicazioni -------------------------			
\textnote{\textit{(oppure tutta la strofa)} }	

\endverse
\fi




%%%%% STROFA
\beginverse		%Oppure \beginverse* se non si vuole il numero di fianco
\memorize 		% <<< DECOMMENTA se si vuole utilizzarne la funzione
%\chordsoff		% <<< DECOMMENTA se vuoi una strofa senza accordi

Nella \[C]notte sboc\[E-]ciò tutto il \[F]Cielo las\[C]sù;
una \[A-]festa \[F*]di \[C]luce \[G*]il \[C]buio \[F*]squar\[C]ciò.
\endverse
\beginverse*
Una ^voce annun^ciò ai pa^stori quag^giù: 
“Oggi è ^nato a ^Be^tlemme ^per ^noi il ^Si^gnor”.
\endverse




%%%%% RITORNELLO
\beginchorus
\textnote{\textbf{Rit.}}

O-o^sanna al Si^gnor che per ^noi si incar^nò, 
che ci ^dona ^la ^pace, ^la ^gioia, ^l’a^mor.

\endchorus



%%%%% STROFA
\beginverse		%Oppure \beginverse* se non si vuole il numero di fianco
%\memorize 		% <<< DECOMMENTA se si vuole utilizzarne la funzione
%\chordsoff		% <<< DECOMMENTA se vuoi una strofa senza accordi

Come ^l’alba si a^prì, come il ^giorno spun^tò
sul de^serto o^riz^zonte l’^at^teso ^Ge^sù.
\endverse
\beginverse*
Come ^miele stil^lò, come ^manna fio^rì:
all’at^tesa ^del ^mondo ^il ^Verbo ^si of^frì.

\endverse



%%%%% STROFA
\beginverse		%Oppure \beginverse* se non si vuole il numero di fianco
%\memorize 		% <<< DECOMMENTA se si vuole utilizzarne la funzione
%\chordsoff		% <<< DECOMMENTA se vuoi una strofa senza accordi
Ogni ^giorno per ^te il Na^tale sa^rà,
se nell’^uomo ^che ^soffre ^ri^trovi ^Ge^sù.
\endverse
\beginverse*
Tra la ^gente che ^va, che cam^mina con ^te,
sulle ^strade ^del ^mondo ^ri^nasce ^Ge^sù. 
\endverse




\endsong
%------------------------------------------------------------
%			FINE CANZONE
%------------------------------------------------------------



%-------------------------------------------------------------
%			INIZIO	CANZONE
%-------------------------------------------------------------

%titolo: Quando Nacque Gesù
%autore: Canto popolare
%tonalita: Re- (e La-)


%%%%%% TITOLO E IMPOSTAZONI
\beginsong{Quando nacque Gesuuuuuuuù}[by={Greensleeves — Canto popolare natalizio}] 	% <<< MODIFICA TITOLO E AUTORE
\transpose{-5} 			% <<< TRASPOSIZIONE #TONI (0 nullo)
\momenti{Natale}		% <<< INSERISCI MOMENTI



%%%%%% INTRODUZIONE
\ifchorded
\vspace*{\versesep}
\textnote{Intro: \qquad \qquad  }%(\eighthnote 116) % << MODIFICA IL TEMPO
\vspace*{-\versesep}
\beginverse*

\nolyrics

%---- Prima riga -----------------------------
\vspace*{-\versesep}
\[D-] 

%---- Ev Indicazioni -------------------------			
\textnote{\textit{(Oppure tutta la strofa)} }	
	 
\endverse
\fi



%%%%% STROFA
\beginverse		%Oppure \beginverse* se non si vuole il numero di fianco
\memorize 		% <<< DECOMMENTA se si vuole utilizzarne la funzione

Un \[D-]bimbo è \[F]nato a Be\[C]tlem\[A-]me,
un bam\[D-]bino è nato per \[A]noi!
Ri\[D-]posa qu\[F]ieto su \[C]paglia e \[A-]fien
e Ma\[D-]ria lo \[A]culla se\[D-]ren.


\endverse




%%%%% RITORNELLO

\beginchorus

\[F]Gloria, gloria, Alle\[C]lu\[A-]ia!
Un bam\[D-]bino è nato per \[A]noi.
\[F]Gloria, gloria, Alle\[C]lu\[A-]ia! 
Oggi è \[D-]nato il \[A]Cristo \[D-]Gesù!

\endchorus





%%%%%% INTERMEZZO

\beginverse*
\vspace*{1.3\versesep}
{
	\textnote{Ev. intermezzo strumentale}
	\textnote{\textit{(ripetizione della strofa)}} 
	 
}
\endverse



%%%%% STROFA
\beginverse		%Oppure \beginverse* se non si vuole il numero di fianco
%\memorize 		% <<< DECOMMENTA se si vuole utilizzarne la funzione

Nel ^cielo gli ^angeli ^can^tano:
"su cor^rete tutti a Bet^lemme,
vi ^trove^rete su ^paglia e ^fien
il Si^gnore il ^Cristo Ge^sù".


\endverse




\endsong
%------------------------------------------------------------
%			FINE CANZONE
%------------------------------------------------------------
%-------------------------------------------------------------
%			INIZIO	CANZONE
%-------------------------------------------------------------

%titolo: Senti l'angelo
%autore:  C. Wesley
%tonalita: Re 


%%%%%% TITOLO E IMPOSTAZONI
\beginsong{Senti l'Angelo}[by={Hark! the Herald Angels Sing — C. Wesley}] 	% <<< MODIFICA TITOLO E AUTORE
\transpose{0} 						% <<< TRASPOSIZIONE #TONI (0 nullo)
\momenti{Natale}							% <<< INSERISCI MOMENTI


%%%%%% INTRODUZIONE
\ifchorded
\vspace*{\versesep}
\musicnote{
\begin{minipage}{0.48\textwidth}
\textbf{Intro}
\hfill 
%( \eighthnote \, 80)   % <<  MODIFICA IL TEMPO
% Metronomo: \eighthnote (ottavo) \quarternote (quarto) \halfnote (due quarti)
\end{minipage}
} 	
\vspace*{-\versesep}
\beginverse*

\nolyrics

%---- Prima riga -----------------------------
\vspace*{-\versesep}
\[G]  \[E-]	\[B] \[E-]

%---- Ogni riga successiva -------------------
\vspace*{-\versesep}
\[A] \[D]  \[D*]	\[A*] \[D*]		

%---- Ev Indicazioni -------------------------			
\textnote{\textit{[Come il ritornello]} }	

\endverse
\fi




%%%%% STROFA
\beginverse		%Oppure \beginverse* se non si vuole il numero di fianco
\memorize 		% <<< DECOMMENTA se si vuole utilizzarne la funzione
%\chordsoff		& <<< DECOMMENTA se vuoi una strofa senza accordi

\[D]Senti l'angelo che \[D]can\[A]ta:
\[D]"Gloria al \[G]nato \[D*]re \[A*]dei \[D*]re!"
\[D]Pace \[B-]vera e vero a\[E-]more
\[A]ha portato al \[A]mon\[(E)]do in\[A]ter.
\[D]Sveglia dunque \[G]le na\[D*]zio\[A*]ni
\[D]alla gioia \[G]del cre\[D*]a\[A*]to
\[G]e con gli \[E-]angeli \[B7]gri\[E-]diam:
\[A]"Cristo è \[D]nato a Be\[A7]tle\[D]hem!"

\endverse




%%%%% RITORNELLO
\beginchorus

\[G]Senti \[E-]l'ange\[E-*]lo \[B7*]can\[E-*]tar:
\[A]"Gloria al \[D]nato \[D*]Re \[A*]dei \[D*]Re!"

\endchorus



%%%%% STROFA
\beginverse		%Oppure \beginverse* se non si vuole il numero di fianco
%\memorize 		% <<< DECOMMENTA se si vuole utilizzarne la funzione
%\chordsoff		& <<< DECOMMENTA se vuoi una strofa senza accordi

^Cristo è l'unico Si^gno^re
^e per ^sempre ^re^gne^rà,
^desi^derio delle ^genti
^vieni presto ^in me^zzo a ^noi.
^Dio velato in ^carne u^ma^na
^hai voluto ^vera^men^te
^abi^tare ^qui con ^noi
^"Salve o ^nostro Emma^nu^el!"

\endverse




%%%%% RITORNELLO
\beginchorus

\[G]Senti \[E-]l'ange\[E-*]lo \[B7*]can\[E-*]tar:
\[A]"Gloria al \[D]nato \[D*]Re \[A*]dei \[D*]Re!"

\endchorus



%%%%% STROFA
\beginverse		%Oppure \beginverse* se non si vuole il numero di fianco
%\memorize 		% <<< DECOMMENTA se si vuole utilizzarne la funzione
%\chordsoff		& <<< DECOMMENTA se vuoi una strofa senza accordi

^Vero Principe di ^pa^ce
^dona al ^mondo ^li^ber^tà,
^vero s^ole di gius^tizia
^sana Tu l'u^ma^ni^tà.
^Tu sei nato ^perchè ^l'uo^mo
^non possa mai ^più mo^ri^re,
^perchè ^l'uomo ^abbia in ^Te
^una sec^onda ^nasci^ta

\endverse




%%%%% RITORNELLO
\beginchorus

\[G]Senti \[E-]l'ange\[E-*]lo \[B7*]can\[E-*]tar:
\[A]"Gloria al \[D]nato \[D*]Re \[A*]dei \[D*]Re!"

\endchorus





\endsong
%------------------------------------------------------------
%			FINE CANZONE
%------------------------------------------------------------
%-------------------------------------------------------------
%			INIZIO	CANZONE
%-------------------------------------------------------------


%titolo: 	Tu Scendi Dalle Stelle
%autore: 	ALfonso de' Liguori
%tonalita: 	Do (abbassata) 



%%%%%% TITOLO E IMPOSTAZONI
\beginsong{Tu scendi dalle stelle}[by={A. De'\ Liguori}]	% <<< MODIFICA TITOLO E AUTORE
\transpose{-2} 						% <<< TRASPOSIZIONE #TONI (0 nullo)
\momenti{Natale}							% <<< INSERISCI MOMENTI	
% momenti vanno separati da ; e vanno scelti tra:
% Ingresso; Atto penitenziale; Acclamazione al Vangelo; Dopo il Vangelo; Offertorio; Comunione; Ringraziamento; Fine; Santi; Pasqua; Avvento; Natale; Quaresima; Canti Mariani; Battesimo; Prima Comunione; Cresima; Matrimonio; Meditazione;
\ifchorded
	%\textnote{Tonalità originale }	% <<< EV COMMENTI (tonalità originale/migliore)
\fi


%%%%%% INTRODUZIONE
\ifchorded
\vspace*{\versesep}
\musicnote{
\begin{minipage}{0.48\textwidth}
\textbf{Intro}
\hfill 
%( \eighthnote \, 80)   % <<  MODIFICA IL TEMPO
% Metronomo: \eighthnote (ottavo) \quarternote (quarto) \halfnote (due quarti)
\end{minipage}
} 	
\vspace*{-\versesep}
\beginverse*

\nolyrics

%---- Prima riga -----------------------------
\vspace*{-\versesep}
\[D] 	 % \[*D] per indicare le pennate, \rep{2} le ripetizioni

%---- Ogni riga successiva -------------------
%\vspace*{-\versesep}
%\[G] \[C]  \[D]	

%---- Ev Indicazioni -------------------------			
%\textnote{\textit{(Oppure tutta la strofa)} }	

\endverse
\fi



%%%%% STROFA
\beginverse
\memorize
Tu \[D]scendi dalle stelle, o Re del cie\[A]lo
e vieni in una \[A]grot\[G]ta 
al \[D]freddo e al \[A]ge\[D]lo,
e \[A]vieni in una \[A]grot\[G]ta 
al \[D]freddo e al \[A]ge\[D]lo.
O Bam\[A]bino, mio Di\[D]vino,
io Ti \[A]vedo qui a tre\[D]mar. O Dio be\[A]ato!
Ah quanto Ti cos\[A]tò \[G]l'a\[D]vermi a\[A]ma\[D]to!
Ah \[A]quanto Ti cos\[A]tò \[G]l'a\[D]vermi a\[A]ma\[D]to!
\endverse



%%%%% STROFA
\beginverse
%\chordsoff
A ^Te che sei del mondo il Creato^re,
mancano panni e ^fuo^co 
o ^mio Si^gno^re,
manc^ano panni e ^fuo^co 
o ^mio Si^gno^re.
Caro e^letto Pargo^letto,
quanto ^questa pover^tà 
più m'inna^mora.
Giacchè Ti fece am^or ^po^vero an^co^ra!
Giac^chè Ti fece am^or ^po^vero an^co^ra!
\endverse



%%%%% STROFA
\beginverse
%\chordsoff
Tu ^lasci il bel gioire del divin se^no,
per giungere a tre^ma^re su ^questo ^fie^no;
per ^giungere a tre^ma^re su ^questo ^fie^no.
Dolce a^more del mio ^cuore, 
dove a^mor ti traspo^rtò! O Gesù ^mio, 
perchè tanto pat^ir, ^^per amor ^mi^o.
Per^chè tanto pat^ir, ^^per amor ^mi^o.
\endverse


\endsong
%------------------------------------------------------------
%			FINE CANZONE
%------------------------------------------------------------
%titolo{Venite fedeli}
%autore{Stefani, Wade}
%album{}
%tonalita{Fa}
%famiglia{Liturgica}
%gruppo{}
%momenti{Natale}
%identificatore{venite_fedeli}
%data_revisione{2014_10_03}
%trascrittore{Francesco Endrici}
\beginsong{Venite fedeli}[by={Stefani, Wade}]
\beginverse
Ve\[F]nite, fe\[C]deli, l'\[F]an\[C]ge\[F]lo \[B&]ci in\[F]vi\[C]ta, 
ve\[D-]ni\[C]te, \[B&]ve\[C]\[D-]ni\[C]te \[F]a Be\[C]\[G7]tlem\[C]me.
\endverse
\beginchorus
\[F]Na\[G-]sce per \[C7]no\[F]i \[G-]Cristo \[D-]Salva\[C]\[B&]to\[C]re.
Ve\[F]ni\[C]te, \[F]a\[C]do\[F]ria\[C]mo, ve\[F]ni\[C]te, \[F]a\[B&]do\[F]ria\[C]mo, 
\[F]ve\[B&]ni\[F]te, \[G-]ado\[C]ria\[D-]mo \[G-]il Si\[F]gno\[C7]re Ge\[F]sù!
\endchorus
\beginverse
%\chordsoff
La ^luce del ^mondo ^bril^la in ^u^na ^grot^ta: 
la ^fe^de ^ci ^^gui^da ^a Be^^tlem^me.
\endverse
\beginverse
%\chordsoff
La ^notte ri^splende, ^tut^to il ^mon^do at^ten^de; 
se^guia^mo i ^pa^^sto^ri ^a Be^^tlem^me.
\endverse
\beginverse
%\chordsoff
Il ^Figlio di ^Dio, ^Re ^dell'^u^ni^ver^so, 
si é ^fat^to ^bam^^bi^no ^a Be^^tlem^me.
\endverse
\beginverse
%\chordsoff
«Sia ^gloria nei ^cieli, ^pa^ce ^sul^la ^ter^ra» 
un ^an^ge^lo an^^nun^cia ^a Be^^tlem^me.
\endverse
\endsong





%******* END SONGS ENVIRONMENT ******
\setcounter{GlobalSongCounter}{\thesongnum}
\end{songs}


\songchapter{Santo}
%...............................................................................
%
% ███████╗ █████╗ ███╗   ██╗████████╗ ██████╗ 
% ██╔════╝██╔══██╗████╗  ██║╚══██╔══╝██╔═══██╗
% ███████╗███████║██╔██╗ ██║   ██║   ██║   ██║
% ╚════██║██╔══██║██║╚██╗██║   ██║   ██║   ██║
% ███████║██║  ██║██║ ╚████║   ██║   ╚██████╔╝
% ╚══════╝╚═╝  ╚═╝╚═╝  ╚═══╝   ╚═╝    ╚═════╝ 
% Font ANSI Shadow
%...............................................................................
\begin{songs}{}
\songcolumns{\canzsongcolumsnumber}
\setcounter{songnum}{\theGlobalSongCounter} %set songnum counter, otherwise would be reset

%set the default path inside current folder
\makeatletter
\def\input@path{{Songs/Santo/}}
\makeatother


%***** INSERT SONGS HERE ******


%-------------------------------------------------------------
%			INIZIO	CANZONE
%-------------------------------------------------------------


%titolo: 	Santo Bonfitto
%autore: 	Michele Bonfitto
%tonalita: 	Sol 



%%%%%% TITOLO E IMPOSTAZONI
\beginsong{Santo Bonfitto}[by={M. Bonfitto}] 	% <<< MODIFICA TITOLO E AUTORE
\transpose{-2} 						% <<< TRASPOSIZIONE #TONI (0 nullo)
%\preferflats  %SE VOGLIO FORZARE i bemolle come alterazioni
%\prefersharps %SE VOGLIO FORZARE i # come alterazioni
\momenti{}							% <<< INSERISCI MOMENTI	
% momenti vanno separati da ; e vanno scelti tra:
% Ingresso; Atto penitenziale; Acclamazione al Vangelo; Dopo il Vangelo; Offertorio; Comunione; Ringraziamento; Fine; Santi; Pasqua; Avvento; Natale; Quaresima; Canti Mariani; Battesimo; Prima Comunione; Cresima; Matrimonio; Meditazione; Spezzare del pane;
\ifchorded
	%\textnote{Tonalità migliore }	% <<< EV COMMENTI (tonalità originale/migliore)
\fi


%%%%%% INTRODUZIONE
\ifchorded
\vspace*{\versesep}
\textnote{Intro: \qquad \qquad  }%(\eighthnote 116) % <<  MODIFICA IL TEMPO
% Metronomo: \eighthnote (ottavo) \quarternote (quarto) \halfnote (due quarti)
\vspace*{-\versesep}
\beginverse*

\nolyrics

%---- Prima riga -----------------------------
\vspace*{-\versesep}
\[G] \[G]	 % \[*D] per indicare le pennate, \rep{2} le ripetizioni

%---- Ogni riga successiva -------------------
%\vspace*{-\versesep}
%\[G] \[C]  \[D]	

%---- Ev Indicazioni -------------------------			
%\textnote{\textit{(Oppure tutta la strofa)} }	

\endverse
\fi

%%%%% STROFA
\beginverse	*	%Oppure \beginverse* se non si vuole il numero di fianco
\memorize 		% <<< DECOMMENTA se si vuole utilizzarne la funzione
%\chordsoff		% <<< DECOMMENTA se vuoi una strofa senza accordi

\[G]San\[D]to, \[E-]santo, \[A-]santo il Si\[D]gnore 
\[G]Dio dell'uni\[D*]\[C*]ver\[D]so. \[(A-*)] \[(D)]

\endverse

%%%%% STROFA
\beginverse*		%Oppure \beginverse* se non si vuole il numero di fianco
\memorize 		% <<< DECOMMENTA se si vuole utilizzarne la funzione
%\chordsoff		% <<< DECOMMENTA se vuoi una strofa senza accordi

I \[G]cieli e la \[C]terra sono \[A-]pieni della tua \[D]gloria.

\endverse

%%%%% RITORNELLO
\beginchorus

O\[G]sanna, o\[E-]sanna, \brk  o\[B-]sa\[E-]nna nell'\[A-]alto dei \[D]cieli. \[D]

\endchorus

%%%%% STROFA
\beginverse*		%Oppure \beginverse* se non si vuole il numero di fianco
\memorize 		% <<< DECOMMENTA se si vuole utilizzarne la funzione
%\chordsoff		% <<< DECOMMENTA se vuoi una strofa senza accordi

Bene^detto colui che ^viene nel ^nome del Si^gnore.

\endverse

%%%%% RITORNELLO
\beginchorus

O\[G]sanna, o\[E-]sanna, \brk o\[B-]sa\[E-]nna nell'\[A-]alto dei \[D]cieli. \[D]

\endchorus

%%%%% STROFA
\beginverse*		%Oppure \beginverse* se non si vuole il numero di fianco
\memorize 		% <<< DECOMMENTA se si vuole utilizzarne la funzione
%\chordsoff		% <<< DECOMMENTA se vuoi una strofa senza accordi

I ^cieli  \quad \echo{Benedetto!}
e la ^terra  \quad \echo{colui che viene!}
sono ^pieni della tua ^gloria. \echo{nel nome del Signore!}


\endverse

%%%%% RITORNELLO
\beginchorus

O\[G]sanna, o\[E-]sanna, \brk o\[B-]sa\[E-]nna nell'\[A-]alto dei \[D]cieli. \[D*]

\endchorus

\endsong
%------------------------------------------------------------
%			FINE CANZONE
%------------------------------------------------------------
%-------------------------------------------------------------
%			INIZIO	CANZONE
%-------------------------------------------------------------


%titolo: 	Santo Campagnolo
%autore: 	
%tonalita: 	Do Re 



%%%%%% TITOLO E IMPOSTAZONI
\beginsong{Santo Campagnolo}[by={}] 	% <<< MODIFICA TITOLO E AUTORE
\transpose{0} 						% <<< TRASPOSIZIONE #TONI (0 nullo)
\momenti{}							% <<< INSERISCI MOMENTI	
% momenti vanno separati da ; e vanno scelti tra:
% Ingresso; Atto penitenziale; Acclamazione al Vangelo; Dopo il Vangelo; Offertorio; Comunione; Ringraziamento; Fine; Santi; Pasqua; Avvento; Natale; Quaresima; Canti Mariani; Battesimo; Prima Comunione; Cresima; Matrimonio; Meditazione;
\ifchorded
	%\textnote{Tonalità originale }	% <<< EV COMMENTI (tonalità originale/migliore)
\fi


%%%%%% INTRODUZIONE
\ifchorded
\vspace*{\versesep}
\textnote{Intro: \qquad \qquad  }%(\eighthnote 116) % << MODIFICA IL TEMPO
% Metronomo: \eighthnote (ottavo) \quarternote (quarto) \halfnote (due quarti)
\vspace*{-\versesep}
\beginverse*

\nolyrics

%---- Prima riga -----------------------------
\vspace*{-\versesep}
\[C] \[A-]  \[D-7]	\[G] % \[*D] per indicare le pennate, \rep{2} le ripetizioni

%---- Ogni riga successiva -------------------
%\vspace*{-\versesep}
%\[G] \[C]  \[D]	

%---- Ev Indicazioni -------------------------			
%\textnote{\textit{(Oppure tutta la strofa)} }	

\endverse
\fi






%%%%% RITORNELLO
\beginchorus


\[C]Santo \[A-]santo \[D-7]santo il Si\[G]gnore 
\chordsoff
Dio, dell'universo,
i cieli e la terra sono pieni della Tua 
gloria!
Osanna nell’alto dei cieli, 
osanna nell’alto dei cieli.
Benedetto colui che viene
nel nome del Signore! 
Osanna nell’alto dei cieli,
osanna nell’alto dei cieli.
\endchorus







\endsong
%------------------------------------------------------------
%			FINE CANZONE
%------------------------------------------------------------




%++++++++++++++++++++++++++++++++++++++++++++++++++++++++++++
%			CANZONE TRASPOSTA
%++++++++++++++++++++++++++++++++++++++++++++++++++++++++++++
\ifchorded
%decremento contatore per avere stesso numero
\addtocounter{songnum}{-1} 
\beginsong{Santo Campagnolo}[by={}] 	% <<< COPIA TITOLO E AUTORE
\transpose{2} 						% <<< TRASPOSIZIONE #TONI + - (0 nullo)
\ifchorded
	\textnote{Tonalità alternativa}	% <<< EV COMMENTI (tonalità originale/migliore)
\fi


%%%%%% INTRODUZIONE
\ifchorded
\vspace*{\versesep}
\textnote{Intro: \qquad \qquad  }%(\eighthnote 116) % << MODIFICA IL TEMPO
% Metronomo: \eighthnote (ottavo) \quarternote (quarto) \halfnote (due quarti)
\vspace*{-\versesep}
\beginverse*

\nolyrics

%---- Prima riga -----------------------------
\vspace*{-\versesep}
\[C] \[A-]  \[D-7]	\[G] % \[*D] per indicare le pennate, \rep{2} le ripetizioni

%---- Ogni riga successiva -------------------
%\vspace*{-\versesep}
%\[G] \[C]  \[D]	

%---- Ev Indicazioni -------------------------			
%\textnote{\textit{(Oppure tutta la strofa)} }	

\endverse
\fi






%%%%% RITORNELLO
\beginchorus


\[C]Santo \[A-]santo \[D-7]santo il Si\[G]gnore 
\chordsoff
Dio, dell'universo,
i cieli e la terra sono pieni della Tua 
gloria!
Osanna nell’alto dei cieli, 
osanna nell’alto dei cieli.
Benedetto colui che viene
nel nome del Signore! 
Osanna nell’alto dei cieli,
osanna nell’alto dei cieli.
\endchorus







\endsong

\fi
%++++++++++++++++++++++++++++++++++++++++++++++++++++++++++++
%			FINE CANZONE TRASPOSTA
%++++++++++++++++++++++++++++++++++++++++++++++++++++++++++++

%-------------------------------------------------------------
%			INIZIO	CANZONE
%-------------------------------------------------------------

%titolo: 	Santo Ricci
%autore: 	Daniele Ricci
%tonalita: 	Sol 


%%%%%% TITOLO E IMPOSTAZONI
\beginsong{Santo Ricci}[by={D. Ricci}] 	% <<< MODIFICA TITOLO E AUTORE
\transpose{0} 						% <<< TRASPOSIZIONE #TONI (0 nullo)
\momenti{Santo}							% <<< INSERISCI MOMENTI




%%%%%% INTRODUZIONE
\ifchorded
\vspace*{\versesep}
\musicnote{
\begin{minipage}{0.48\textwidth}
\textbf{Intro}
\hfill 
%( \eighthnote \, 80)   % <<  MODIFICA IL TEMPO
% Metronomo: \eighthnote (ottavo) \quarternote (quarto) \halfnote (due quarti)
\end{minipage}
} 	
\vspace*{-\versesep}
\beginverse*

	\nolyrics
	\textbf{Intro:} \qquad \qquad % (\eighthnote 116) % << MODIFICA IL TEMPO
	\vspace*{-\versesep}

	%---- Prima riga -----------------------------
	\[G] \[D]  \[A-]	%\textbf{x2}

	%---- Ogni riga successiva -------------------
	\vspace*{-\versesep}
	\[G] \[D]  \[A-]	%\textbf{x2}



\vspace*{-0.3\versesep}
\endverse
\fi



%%%%% STROFA
\beginverse*	%Oppure \beginverse* se non si vuole il numero di fianco
\memorize 		% <<< DECOMMENTA se si vuole utilizzarne la funzione
%\chordsoff		& <<< DECOMMENTA se vuoi una strofa senza accordi

\[G]Santo santo \[D]santo - \[A-]o
\echo{è il Signore Dio dell'universo}
^Santo santo ^santo - ^o 
\echo{è il Signore Dio dell'universo}
^Santo santo ^santo - ^o 
\echo{i cieli e la terra}
\[G]Santo santo \[D]santo - \[A-]o
\echo{sono pieni della tua gloria}
\[G]Osanna osanna \[D]osanna - \[A-]a
\echo{nell'alto dei cieli}


\endverse

\vspace*{-0.3\versesep}
\beginverse*
\[G*]Benedetto colui che vi\[D*]ene
nel \[C*]nome \[D*]del Si\[C*]gno\[D]re
\endverse

\vspace*{-0.3\versesep}
\beginverse*	

^Osanna osanna ^osanna - ^a
\echo{nell'alto dei cieli}
^Osanna osanna ^osanna - ^a
\echo{nell'alto dei cieli}
^Osanna osanna ^osanna - ^a
\echo{nell'alto dei cieli}
\[G]Osanna-a...

\endverse

\endsong
%------------------------------------------------------------
%			FINE CANZONE
%------------------------------------------------------------

%-------------------------------------------------------------
%			INIZIO	CANZONE
%-------------------------------------------------------------


%titolo: 	Santo Gen Verde
%autore: 	Gen Verde
%tonalita: 	Mi 



%%%%%% TITOLO E IMPOSTAZONI
\beginsong{Santo Gen Verde}[by={Gen Verde}] 	% <<< MODIFICA TITOLO E AUTORE
\transpose{0} 						% <<< TRASPOSIZIONE #TONI (0 nullo)
\momenti{Santo}							% <<< INSERISCI MOMENTI	
% momenti vanno separati da ; e vanno scelti tra:
% Ingresso; Atto penitenziale; Acclamazione al Vangelo; Dopo il Vangelo; Offertorio; Comunione; Ringraziamento; Fine; Santi; Pasqua; Avvento; Natale; Quaresima; Canti Mariani; Battesimo; Prima Comunione; Cresima; Matrimonio; Meditazione;
\ifchorded
	%\textnote{Tonalità originale }	% <<< EV COMMENTI (tonalità originale/migliore)
\fi


%%%%%% INTRODUZIONE
\ifchorded
\vspace*{\versesep}
\musicnote{
\begin{minipage}{0.48\textwidth}
\textbf{Intro}
\hfill 
%( \eighthnote \, 80)   % <<  MODIFICA IL TEMPO
% Metronomo: \eighthnote (ottavo) \quarternote (quarto) \halfnote (due quarti)
\end{minipage}
} 	
\vspace*{-\versesep}
\beginverse*

\nolyrics

%---- Prima riga -----------------------------
\vspace*{-\versesep}
\[E] \[A*] \[E] \[A*] 	 % \[*D] per indicare le pennate, \rep{2} le ripetizioni

%---- Ogni riga successiva -------------------
\vspace*{-\versesep}
\[E] \[B] \[E] \rep{2}

%---- Ev Indicazioni -------------------------			
%\textnote{\textit{(Oppure tutta la strofa)} }	

\endverse
\fi

%%%%% STROFA
\beginverse*		%Oppure \beginverse* se non si vuole il numero di fianco
\memorize 		% <<< DECOMMENTA se si vuole utilizzarne la funzione
%\chordsoff		% <<< DECOMMENTA se vuoi una strofa senza accordi
\[E]San\[A*]to, \[E]san\[A*]to, \[E]\[B]san\[E]to,
\[E]san\[A*]to, \[E]san\[A*]to, \[E]\[B]san\[E]to.
Il Si\[A]gnore Dio dell'uni\[E]verso,
il Si\[A]gnore Dio dell'uni\[E]verso,
i \[A]cieli e la \[E]terra \brk sono \[B]pieni della tua \[E]gloria.
\endverse


%%%%% RITORNELLO
\beginchorus
O\[A]sanna o\[E]sanna nell'\[B]alto dei \[E]cieli.
O\[A]sanna o\[E]sanna nell'\[B]alto dei \[E]cieli.
\endchorus

%%%%% STROFA
\beginverse*
^San^to, ^san^to, ^^san^to.
^San^to, ^san^to, ^^san^to.
Bene^detto colui che ^viene nel ^nome del Si^gnore. 
Bene^detto colui che ^viene nel ^nome del Si^gnore. 
\endverse


%%%%% RITORNELLO
\beginchorus
O\[A]sanna o\[E]sanna nell'\[B]alto dei \[E]cieli.
O\[A]sanna o\[E]sanna nell'\[B]alto dei \[E]cieli.
\endchorus



%%%%% STROFA
\beginverse*
^San^to, ^san^to, ^^san^to.
^San^to, ^san^to, ^^san^to.
\endverse

\endsong
%------------------------------------------------------------
%			FINE CANZONE
%------------------------------------------------------------



%-------------------------------------------------------------
%			INIZIO	CANZONE
%-------------------------------------------------------------


%titolo: 	Santo Milan
%autore: 	Gen Verde
%tonalita: 	Sol 



%%%%%% TITOLO E IMPOSTAZONI
\beginsong{Santo Milan}[by={Gen Verde}] 	% <<< MODIFICA TITOLO E AUTORE
\transpose{0} 						% <<< TRASPOSIZIONE #TONI (0 nullo)
\momenti{}							% <<< INSERISCI MOMENTI	
% momenti vanno separati da ; e vanno scelti tra:
% Ingresso; Atto penitenziale; Acclamazione al Vangelo; Dopo il Vangelo; Offertorio; Comunione; Ringraziamento; Fine; Santi; Pasqua; Avvento; Natale; Quaresima; Canti Mariani; Battesimo; Prima Comunione; Cresima; Matrimonio; Meditazione;
\ifchorded
	\textnote{Tonalità originale }	% <<< EV COMMENTI (tonalità originale/migliore)
\fi


%%%%%% INTRODUZIONE
\ifchorded
\vspace*{\versesep}
\textnote{Intro: \qquad \qquad  }%(\eighthnote 116) % << MODIFICA IL TEMPO
% Metronomo: \eighthnote (ottavo) \quarternote (quarto) \halfnote (due quarti)
\vspace*{-\versesep}
\beginverse*

\nolyrics

%---- Prima riga -----------------------------
\vspace*{-\versesep}
\[A] \[E]  \[C#-] \[B]	 % \[*D] per indicare le pennate, \rep{2} le ripetizioni

%---- Ogni riga successiva -------------------
\vspace*{-\versesep}
\[F#-] \[E]  \[A]  \[B]	

%---- Ev Indicazioni -------------------------			
\textnote{\textit{(come le prime due righe)} }	

\endverse
\fi








%%%%% RITORNELLO
\beginchorus

\[A]San\[E]to, \[C#-]San\[B]to,
\[F#-]Santo il Si\[E]gnore, \[A]Dio dell'uni\[B]verso.
\[A]San\[E]to, \[C#-]San\[B]to.
I \[F#-]cieli e la \[E]terra 
sono \[A]pieni della tua \[(F#-)]glo\[E]ria.

\endchorus



%%%%% STROFA
\beginverse*		%Oppure \beginverse* se non si vuole il numero di fianco
%\memorize 		% <<< DECOMMENTA se si vuole utilizzarne la funzione
%\chordsoff		% <<< DECOMMENTA se vuoi una strofa senza accordi

O\[A]sanna nel\[B]l'alto dei \[A]cie\[B]li.
O\[F#-]sanna nell'alto dei \[A]cieli.

\endverse



%%%%% RITORNELLO
\beginchorus

\[A]San\[E]to, \[C#-]San\[B]to,
\[F#-]Santo il Si\[E]gnore, \[A]Dio dell'uni\[B]verso.
\[A]San\[E]to, \[C#-]San\[B]to.
I \[F#-]cieli e la \[E]terra 
sono \[A]pieni della tua \[(F#-)]glo\[E]ria.

\endchorus




%%%%% STROFA
\beginverse*		%Oppure \beginverse* se non si vuole il numero di fianco
%\memorize 		% <<< DECOMMENTA se si vuole utilizzarne la funzione
%\chordsoff		& <<< DECOMMENTA se vuoi una strofa senza accordi

\[B]Benedetto co\[A]lui che viene
nel \[E]nome del Sig\[B]nore.
O\[A]sanna nel\[B]l'alto dei \[A]cie\[B]li.
O\[F#-]sanna nell'alto dei \[A]cieli.

\endverse


%%%%% RITORNELLO
\beginchorus

\[A]San\[E]to, \[C#-]San\[B]to,
\[F#-]Sa-\[A]a-n\[E]to.  \[*E] 

\endchorus











\endsong
%------------------------------------------------------------
%			FINE CANZONE
%------------------------------------------------------------




%++++++++++++++++++++++++++++++++++++++++++++++++++++++++++++
%			CANZONE TRASPOSTA
%++++++++++++++++++++++++++++++++++++++++++++++++++++++++++++
\ifchorded
%decremento contatore per avere stesso numero
\addtocounter{songnum}{-1} 
\beginsong{Santo Milan}[by={Gen Verde}] 	% <<< COPIA TITOLO E AUTORE
\transpose{-2} 						% <<< TRASPOSIZIONE #TONI + - (0 nullo)
\ifchorded
	\textnote{Tonalità più facile per le chitarre}	% <<< EV COMMENTI (tonalità originale/migliore)
\fi


%%%%%% INTRODUZIONE
\ifchorded
\vspace*{\versesep}
\textnote{Intro: \qquad \qquad  }%(\eighthnote 116) % << MODIFICA IL TEMPO
% Metronomo: \eighthnote (ottavo) \quarternote (quarto) \halfnote (due quarti)
\vspace*{-\versesep}
\beginverse*

\nolyrics

%---- Prima riga -----------------------------
\vspace*{-\versesep}
\[A] \[E]  \[C#-] \[B]	 % \[*D] per indicare le pennate, \rep{2} le ripetizioni

%---- Ogni riga successiva -------------------
\vspace*{-\versesep}
\[F#-] \[E]  \[A]  \[B]	

%---- Ev Indicazioni -------------------------			
\textnote{\textit{(come le prime due righe)} }	

\endverse
\fi








%%%%% RITORNELLO
\beginchorus

\[A]San\[E]to, \[C#-]San\[B]to,
\[F#-]Santo il Si\[E]gnore, \[A]Dio dell'uni\[B]verso.
\[A]San\[E]to, \[C#-]San\[B]to.
I \[F#-]cieli e la \[E]terra 
sono \[A]pieni della tua \[(F#-)]glo\[E]ria.

\endchorus



%%%%% STROFA
\beginverse*		%Oppure \beginverse* se non si vuole il numero di fianco
%\memorize 		% <<< DECOMMENTA se si vuole utilizzarne la funzione
%\chordsoff		% <<< DECOMMENTA se vuoi una strofa senza accordi

O\[A]sanna nel\[B]l'alto dei \[A]cie\[B]li.
O\[F#-]sanna nell'alto dei \[A]cieli.

\endverse



%%%%% RITORNELLO
\beginchorus

\[A]San\[E]to, \[C#-]San\[B]to,
\[F#-]Santo il Si\[E]gnore, \[A]Dio dell'uni\[B]verso.
\[A]San\[E]to, \[C#-]San\[B]to.
I \[F#-]cieli e la \[E]terra 
sono \[A]pieni della tua \[(F#-)]glo\[E]ria.

\endchorus




%%%%% STROFA
\beginverse*		%Oppure \beginverse* se non si vuole il numero di fianco
%\memorize 		% <<< DECOMMENTA se si vuole utilizzarne la funzione
%\chordsoff		& <<< DECOMMENTA se vuoi una strofa senza accordi

\[B]Benedetto co\[A]lui che viene
nel \[E]nome del Sig\[B]nore.
O\[A]sanna nel\[B]l'alto dei \[A]cie\[B]li.
O\[F#-]sanna nell'alto dei \[A]cieli.

\endverse


%%%%% RITORNELLO
\beginchorus

\[A]San\[E]to, \[C#-]San\[B]to,
\[F#-]Sa-\[A]a-n\[E]to.  \[*E] 

\endchorus






\endsong

\fi
%++++++++++++++++++++++++++++++++++++++++++++++++++++++++++++
%			FINE CANZONE TRASPOSTA
%++++++++++++++++++++++++++++++++++++++++++++++++++++++++++++

%-------------------------------------------------------------
%			INIZIO	CANZONE
%-------------------------------------------------------------


%titolo: 	Santo (Prendi questo pane)
%autore: 	
%tonalita: 	Do 



%%%%%% TITOLO E IMPOSTAZONI
\beginsong{Santo Caidate}[by={Caidate}] 	% <<< MODIFICA TITOLO E AUTORE
\transpose{0} 						% <<< TRASPOSIZIONE #TONI (0 nullo)
%\preferflats  %SE VOGLIO FORZARE i bemolle come alterazioni
%\prefersharps %SE VOGLIO FORZARE i # come alterazioni
\momenti{}							% <<< INSERISCI MOMENTI	
% momenti vanno separati da ; e vanno scelti tra:
% Ingresso; Atto penitenziale; Acclamazione al Vangelo; Dopo il Vangelo; Offertorio; Comunione; Ringraziamento; Fine; Santi; Pasqua; Avvento; Natale; Quaresima; Canti Mariani; Battesimo; Prima Comunione; Cresima; Matrimonio; Meditazione; Spezzare del pane;
\ifchorded
	%\textnote{Tonalità migliore }	% <<< EV COMMENTI (tonalità originale/migliore)
\fi


%%%%%% INTRODUZIONE
\ifchorded
\vspace*{\versesep}
\textnote{Intro: \qquad \qquad  }%(\eighthnote 116) % <<  MODIFICA IL TEMPO
% Metronomo: \eighthnote (ottavo) \quarternote (quarto) \halfnote (due quarti)
\vspace*{-\versesep}
\beginverse*

\nolyrics

%---- Prima riga -----------------------------
\vspace*{-\versesep}
\[C] \[E-] \[A-] \[C]	 % \[*D] per indicare le pennate, \rep{2} le ripetizioni

%---- Ogni riga successiva -------------------
%\vspace*{-\versesep}
%\[G] \[C]  \[D]	

%---- Ev Indicazioni -------------------------			
%\textnote{\textit{(Oppure tutta la strofa)} }	

\endverse
\fi

%%%%% STROFA
\beginverse		%Oppure \beginverse* se non si vuole il numero di fianco
\memorize 		% <<< DECOMMENTA se si vuole utilizzarne la funzione
%\chordsoff		% <<< DECOMMENTA se vuoi una strofa senza accordi

\[C]Santo, santo, \[E-]santo il Si\[A-]gnore. \[C]
\[F]Il Signore \[D-]Dio dell'uni\[G]verso.
I \[C]cieli e la \[E]terra sono \[A-]pieni \[C]della tua \[F]gloria, O\[G]sa\[C]nna

\endverse

%%%%% STROFA
\beginverse		%Oppure \beginverse* se non si vuole il numero di fianco
\memorize 		% <<< DECOMMENTA se si vuole utilizzarne la funzione
%\chordsoff		% <<< DECOMMENTA se vuoi una strofa senza accordi

^Benedetto ^chi viene nel ^nome ^
^chi viene nel ^nome del Si^gnore.
O^sanna nell'^alto dei ^cieli
^O^sanna nell'^alto dei ^cieli.

\endverse

\endsong
%------------------------------------------------------------
%			FINE CANZONE
%------------------------------------------------------------
%titolo{Santo Zaire}
%autore{}
%album{Guarda al di là}
%tonalita{Mi}
%famiglia{Liturgica}
%gruppo{Santo}
%momenti{Santo}
%identificatore{santo_zaire}
%data_revisione{2014_09_30}
%trascrittore{Francesco Endrici}
\beginsong{Santo Zaire}
\beginverse*
\[E]Santo \[A]Santo O\[E]\[B]san\[E]na
\[E]Santo \[A]Santo O\[E]\[B]san\[E]na \[A]\[E]
\endverse
\beginchorus
O\[E]sanna eh, O\[A]sanna \[E]eh,
O\[A]sanna a \[B]Cristo Si\[E]gnor
O\[E]sanna eh, O\[A]sanna \[E]eh,
O\[A]sanna a \[B]Cristo Si\[E]gnor \[A]\[E]
\endchorus
\beginverse*
%\chordsoff
I ^cieli e la terra, o Si^gnore, sono ^pie^ni di ^te. 
I ^cieli e la terra, o Si^gnore, sono ^pie^ni di ^te. ^^
\endverse
\beginchorus
O\[E]sanna eh, O\[A]sanna \[E]eh,
O\[A]sanna a \[B]Cristo Si\[E]gnor
O\[E]sanna eh, O\[A]sanna \[E]eh,
O\[A]sanna a \[B]Cristo Si\[E]gnor \[A]\[E]
\endchorus
\beginverse*
%\chordsoff
Bene^detto Colui che ^viene nel ^nome ^tuo Si^gnor.
Bene^detto Colui che ^viene nel ^nome ^tuo Si^gnor. ^^
\endverse
\beginchorus
O\[E]sanna eh, O\[A]sanna \[E]eh,
O\[A]sanna a \[B]Cristo Si\[E]gnor
O\[E]sanna eh, O\[A]sanna \[E]eh,
O\[A]sanna a \[B]Cristo Si\[E]gnor \[A]\[E]
\endchorus
\endsong
%-------------------------------------------------------------
%			INIZIO	CANZONE
%-------------------------------------------------------------


%titolo: 	Santo Zappalà
%autore: 	
%tonalita: 	Sol 



%%%%%% TITOLO E IMPOSTAZONI
\beginsong{Santo Zappalà}[by={G. Zappalà, A. Mancuso}] 	% <<< MODIFICA TITOLO E AUTORE
\transpose{0} 						% <<< TRASPOSIZIONE #TONI (0 nullo)
\momenti{Santo}							% <<< INSERISCI MOMENTI	
% momenti vanno separati da ; e vanno scelti tra:
% Ingresso; Atto penitenziale; Acclamazione al Vangelo; Dopo il Vangelo; Offertorio; Comunione; Ringraziamento; Fine; Santi; Pasqua; Avvento; Natale; Quaresima; Canti Mariani; Battesimo; Prima Comunione; Cresima; Matrimonio; Meditazione;
\ifchorded
	%\textnote{$\bigstar$ Tonalità originale }	% <<< EV COMMENTI (tonalità originale\migliore)
\fi


%%%%%% INTRODUZIONE
\ifchorded
\vspace*{\versesep}
\musicnote{
\begin{minipage}{0.48\textwidth}
\textbf{Intro}
\hfill 
%( \eighthnote \, 80)   % <<  MODIFICA IL TEMPO
% Metronomo: \eighthnote (ottavo) \quarternote (quarto) \halfnote (due quarti)
\end{minipage}
} 	
\vspace*{-\versesep}
\beginverse*

\nolyrics

%---- Prima riga -----------------------------
\vspace*{-\versesep}
\[G] \[E-] \[B-] \[D7] 	 % \[*D] per indicare le pennate, \rep{2} le ripetizioni


\endverse
\fi








%%%%% RITORNELLO
\beginchorus

\[G]Santo, \[E-]Santo, \[B-]Santo, \brk il Sign\[C]ore Dio dell’uni\[D]verso.
I \[C]cie\[B-]li e la \[C]ter\[E-]ra \brk sono \[C]pieni \[A7]della tua \[D]gloria.

\endchorus




%%%%% RITORNELLO
\beginchorus

Os\[G]an\[D]na, Os\[G]an\[D]na, \brk Os\[C]anna nell’\[D]alto dei ci\[G]eli. \[C*]\[D*] 
Os\[G]an\[D]na, Os\[G]an\[D]na, \brk Os\[C]anna nell’\[D]alto dei ci\[G]eli.
\endchorus





%%%%% STROFA
\beginverse*		%Oppure \beginverse* se non si vuole il numero di fianco
%\memorize 		% <<< DECOMMENTA se si vuole utilizzarne la funzione
%\chordsoff		% <<< DECOMMENTA se vuoi una strofa senza accordi

Benede\[E-]tto co\[C]lui che \[D]viene nel nome del Si\[G]gnore. \[C*] \[D*]

\endverse


%%%%% RITORNELLO
\beginchorus

Os\[G]an\[D]na, Os\[G]an\[D]na, \brk Os\[C]anna nell’\[D]alto dei ci\[G]eli. \[C*]\[D*] 
Os\[G]an\[D]na, Os\[G]an\[D]na, \brk Os\[C]anna nell’\[D]alto dei ci\[G]eli.
\endchorus








\endsong
%------------------------------------------------------------
%			FINE CANZONE
%------------------------------------------------------------






%******* END SONGS ENVIRONMENT ******
\setcounter{GlobalSongCounter}{\thesongnum}
\end{songs}




% INDICI
%-------------------------------------------------------------------------------



%Serie di comandi per scrivere gli indici corretti a seconda
%delle indicazioni iniziali del documento.

\ifcanzsingole
	\relax
\else
	\iftitleindex
		\ifxetex
		\printindex[alfabetico]
		\else
		\printindex
		\fi
	\else
	\fi
	\ifauthorsindex
	\printindex[autori]
	\else
	\fi
	\iftematicindex
	\printindex[tematico]
	\else
	\fi
	\ifcover
		\relax
	\else
		\colophon
	\fi
\fi



%Fine documento
\end{document}