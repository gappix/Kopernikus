% SETTINGS
%——————————————————————————————————————————————————————
% STILE DOCUMENTO
%-------------------------------------------------------------------------------                                                                             


%% CHORDED Settings
% > A4vert + palatino + titleindex + tematicindex + chorded

\documentclass[a5vert, palatino, titleindex, tematicindex, lyric, cover]{canzoniereonline}

%opzioni formato: singoli, standard (A4), a5vert, a5oriz, a6vert;
%opzioni accordi: lyric, chorded {quelli d Songs}
%opzioni font: palatino, libertine
%opzioni segno minore: "minorsign=quel che vuoi"
%opzioni indici: authorsindex, titleindex, tematicindex

%opzioi copertina: cover e nocover

\def\canzsongcolumsnumber{2} %# coolonne lungo cui disporre le canzoni




% PACCHETTI DA IMPORTARE
%-------------------------------------------------------------------------------                                                                             
\usepackage[T1]{fontenc}
\usepackage[utf8]{inputenc}
\usepackage[italian]{babel}
\usepackage{pdfpages}
\usepackage{hyperref}
\usepackage{wasysym}






% NUOVI COMANDI E VARIABILI GLOBALI
%--------------------------------------------------------------------------------
%Coomando per la suddivisione in capitoli
\renewcommand{\songchapter}{\chapter*}


%Counter globale per tenere traccia di una numerazione progressiva
%Si affianca a un altro counter già utilizzato nella classe CanzoniereOnLine "songnum" 
%che, tuttavia, si riazzera ognivolta viene definito un nuovo ambiente \beginsongs{}
\newcounter{GlobalSongCounter} 

%Ciascun capitolo contiene già tutta la logica di gestione della 
%numerazione progressiva, del DB locale da cui attingere le canzoni
%e la creazione/chiusura dell'ambiente in cui vengono importate 
%tutte le canzoni relative
\addtocounter{GlobalSongCounter}{1} %set starting song counter to 1 (0 otherwise)


%------------
\makeatletter
\newcommand*{\textoverline}[1]{$\overline{\hbox{#1}}\m@th$}
\makeatother
%-----------

%Starting Document
\begin{document}


\begin{songs}{}
\songcolumns{\canzsongcolumsnumber}
\setcounter{songnum}{\theGlobalSongCounter} %set songnum counter, otherwise would be reset


%  *  *  *  *  *  TEST SONG HERE  *  *  *  *  *  *  * ]
%set the default path inside current folder


% ----- LITURGIA ---------
\makeatletter \def\input@path{{Songs/Liturgia/}} \makeatother
%-------------------------------------------------------------
%			INIZIO	CANZONE
%-------------------------------------------------------------


%titolo: 	Nulla è impossibile a Dio
%autore: 	C. Burgio
%tonalita: 	Re>Fa



%%%%%% TITOLO E IMPOSTAZONI
\beginsong{Nulla è impossibile a Dio}[by={C. Burgio}] 	% <<< MODIFICA TITOLO E AUTORE
\transpose{3} 						% <<< TRASPOSIZIONE #TONI (0 nullo)
\momenti{Ingresso; Prima Comunione}							% <<< INSERISCI MOMENTI	
% momenti vanno separati da ; e vanno scelti tra:
% Ingresso; Atto penitenziale; Acclamazione al Vangelo; Dopo il Vangelo; Offertorio; Comunione; Ringraziamento; Fine; Santi; Pasqua; Avvento; Natale; Quaresima; Canti Mariani; Battesimo; Prima Comunione; Cresima; Matrimonio; Meditazione; Spezzare del pane;
\ifchorded
	\textnote{Tonalità migliore}	% <<< EV COMMENTI (tonalità originale/migliore)
\fi


%%%%%% INTRODUZIONE
\ifchorded
\vspace*{\versesep}
\textnote{Intro: \qquad \qquad  }%(\eighthnote 116) % <<  MODIFICA IL TEMPO
% Metronomo: \eighthnote (ottavo) \quarternote (quarto) \halfnote (due quarti)
\vspace*{-\versesep}
\beginverse*

\nolyrics

%---- Prima riga -----------------------------
\vspace*{-\versesep}
\[D] \[A] \[G] \[A]	 % \[*D] per indicare le pennate, \rep{2} le ripetizioni

%---- Ogni riga successiva -------------------
%\vspace*{-\versesep}
%\[G] \[C]  \[D]	

%---- Ev Indicazioni -------------------------			
%\textnote{\textit{(Oppure tutta la strofa)} }	

\endverse
\fi




%%%%% STROFA
\beginverse		%Oppure \beginverse* se non si vuole il numero di fianco
\memorize 		% <<< DECOMMENTA se si vuole utilizzarne la funzione
%\chordsoff		% <<< DECOMMENTA se vuoi una strofa senza accordi
\[D]Quando Dio ci chia\[A]mò 
ed il \[G]tempo ci do\[A]nò,
come un \[D]padre fidu\[A]cioso 
nel suo \[G]cuore ci por\[A]tò.

\[B-]Egli fece di \[F#-]noi 
una s\[G]toria un solo \[A]popolo;
\[B-]forte, lui, ci gui\[F#-]dò 
sulle \[G]strade che con\[E-]ducono 
alla \[F#]libertà.

\endverse




%%%%% RITORNELLO
\beginchorus
\textnote{\textbf{Rit.}}

\[D]Ecco il grande mis\[A]tero 
dai \[B-]secoli annun\[F#-]ciato:
“\[G]Nulla è impos\[D]sibile a \[A]Dio”. \[A7]
\[D]Nasce nuova spe\[A7]ranza 
si \[B-]compie ormai la pro\[F#-]messa:
“\[G]Nulla è impos\[D]sibile a \[A]Di\[D]o”.

\endchorus



%%%%% STROFA
\beginverse		%Oppure \beginverse* se non si vuole il numero di fianco
%\memorize 		% <<< DECOMMENTA se si vuole utilizzarne la funzione
%\chordsoff		% <<< DECOMMENTA se vuoi una strofa senza accordi

^Quando venne tra ^noi, 
come ^figlio e “Dio con ^noi”, 
fatto ^uomo in Ma^ria, 
la sal^vezza ci por^tò. 

^Noi credemmo in ^Lui, 
vivo ^segno della ^Verità; 
^imparammo da ^Lui 
che l’^Amore non ha ^prezzo,
non pos^siede mai. 

\endverse



%%%%% STROFA
\beginverse		%Oppure \beginverse* se non si vuole il numero di fianco
%\memorize 		% <<< DECOMMENTA se si vuole utilizzarne la funzione
%\chordsoff		% <<< DECOMMENTA se vuoi una strofa senza accordi

^Quando poi ci las^ciò 
e dal ^padre rito^rnò, 
il Si^gnore dalla ^croce 
il suo ^spirito do^nò. 

^Nuova vita per ^noi 
questa ^grazia che ci il^lumina, 
^è memoria tra ^noi 
dell’A^more che ci ac^coglie
e non ci ^lascia mai.

\endverse




\endsong
%------------------------------------------------------------
%			FINE CANZONE
%------------------------------------------------------------

%++++++++++++++++++++++++++++++++++++++++++++++++++++++++++++
%			CANZONE TRASPOSTA
%++++++++++++++++++++++++++++++++++++++++++++++++++++++++++++
\ifchorded
%decremento contatore per avere stesso numero
\addtocounter{songnum}{-1} 
\beginsong{Nulla è impossibile a Dio}[by={C. Burgio}] 	% <<< MODIFICA TITOLO E AUTORE 
\transpose{0} 						% <<< TRASPOSIZIONE #TONI + - (0 nullo)
%\preferflats  %SE VOGLIO FORZARE i bemolle come alterazioni
%\prefersharps %SE VOGLIO FORZARE i # come alterazioni
\ifchorded
	\textnote{Tonalità originale}	% <<< EV COMMENTI (tonalità originale/migliore)
\fi


%%%%%% INTRODUZIONE
\ifchorded
\vspace*{\versesep}
\textnote{Intro: \qquad \qquad  }%(\eighthnote 116) % <<  MODIFICA IL TEMPO
% Metronomo: \eighthnote (ottavo) \quarternote (quarto) \halfnote (due quarti)
\vspace*{-\versesep}
\beginverse*

\nolyrics

%---- Prima riga -----------------------------
\vspace*{-\versesep}
\[D] \[A] \[G] \[A]	 % \[*D] per indicare le pennate, \rep{2} le ripetizioni

%---- Ogni riga successiva -------------------
%\vspace*{-\versesep}
%\[G] \[C]  \[D]	

%---- Ev Indicazioni -------------------------			
%\textnote{\textit{(Oppure tutta la strofa)} }	

\endverse
\fi




%%%%% STROFA
\beginverse		%Oppure \beginverse* se non si vuole il numero di fianco
\memorize 		% <<< DECOMMENTA se si vuole utilizzarne la funzione
%\chordsoff		% <<< DECOMMENTA se vuoi una strofa senza accordi
\[D]Quando Dio ci chia\[A]mò 
ed il \[G]tempo ci do\[A]nò,
come un \[D]padre fidu\[A]cioso 
nel suo \[G]cuore ci por\[A]tò.

\[B-]Egli fece di \[F#-]noi 
una s\[G]toria un solo \[A]popolo;
\[B-]forte, lui, ci gui\[F#-]dò 
sulle \[G]strade che con\[E-]ducono 
alla \[F#]libertà.

\endverse




%%%%% RITORNELLO
\beginchorus
\textnote{\textbf{Rit.}}

\[D]Ecco il grande mis\[A]tero 
dai \[B-]secoli annun\[F#-]ciato:
“\[G]Nulla è impos\[D]sibile a \[A]Dio”. \[A7]
\[D]Nasce nuova spe\[A7]ranza 
si \[B-]compie ormai la pro\[F#-]messa:
“\[G]Nulla è impos\[D]sibile a \[A]Di\[D]o”.

\endchorus



%%%%% STROFA
\beginverse		%Oppure \beginverse* se non si vuole il numero di fianco
%\memorize 		% <<< DECOMMENTA se si vuole utilizzarne la funzione
%\chordsoff		% <<< DECOMMENTA se vuoi una strofa senza accordi

^Quando venne tra ^noi, 
come ^figlio e “Dio con ^noi”, 
fatto ^uomo in Ma^ria, 
la sal^vezza ci por^tò. 

^Noi credemmo in ^Lui, 
vivo ^segno della ^Verità; 
^imparammo da ^Lui 
che l’^Amore non ha ^prezzo,
non pos^siede mai. 

\endverse



%%%%% STROFA
\beginverse		%Oppure \beginverse* se non si vuole il numero di fianco
%\memorize 		% <<< DECOMMENTA se si vuole utilizzarne la funzione
%\chordsoff		% <<< DECOMMENTA se vuoi una strofa senza accordi

^Quando poi ci las^ciò 
e dal ^padre rito^rnò, 
il Si^gnore dalla ^croce 
il suo ^spirito do^nò. 

^Nuova vita per ^noi 
questa ^grazia che ci il^lumina, 
^è memoria tra ^noi 
dell’A^more che ci ac^coglie
e non ci ^lascia mai.

\endverse




\endsong


\fi
%++++++++++++++++++++++++++++++++++++++++++++++++++++++++++++
%			FINE CANZONE TRASPOSTA
%++++++++++++++++++++++++++++++++++++++++++++++++++++++++++++


%% ----- GLORIA ---------
\makeatletter \def\input@path{{Songs/Gloria/}} \makeatother
%-------------------------------------------------------------
%			INIZIO	CANZONE
%-------------------------------------------------------------


%titolo: 	Gloria (esme)
%autore: 	Gen Verde
%tonalita: 	FA e RE 



%%%%%% TITOLO E IMPOSTAZONI
\beginsong{Gloria nell'alto dei cieli}[by={Gen Verde, Esme}] 	% <<< MODIFICA TITOLO E AUTORE
\transpose{0} 						% <<< TRASPOSIZIONE #TONI (0 nullo)
\momenti{}							% <<< INSERISCI MOMENTI	
% momenti vanno separati da ; e vanno scelti tra:
% Ingresso; Atto penitenziale; Acclamazione al Vangelo; Dopo il Vangelo; Offertorio; Comunione; Ringraziamento; Fine; Santi; Pasqua; Avvento; Natale; Quaresima; Canti Mariani; Battesimo; Prima Comunione; Cresima; Matrimonio; Meditazione;
\ifchorded
	%\textnote{Tonalità originale }	% <<< EV COMMENTI (tonalità originale/migliore)
\fi




%%%%%% INTRODUZIONE
\ifchorded
\vspace*{\versesep}
\textnote{Intro: \qquad \qquad  }%(\eighthnote 116) % << MODIFICA IL TEMPO
% Metronomo: \eighthnote (ottavo) \quarternote (quarto) \halfnote (due quarti)
\vspace*{-\versesep}
\beginverse*

\nolyrics

%---- Prima riga -----------------------------
\vspace*{-\versesep}
\[F]   \[B&]  \[C]  \[C]	 \rep{2} % \[*D] per indicare le pennate, \rep{2} le ripetizioni

%---- Ogni riga successiva -------------------
%\vspace*{-\versesep}
%\[G] \[C]  \[D]	

%---- Ev Indicazioni -------------------------			
%\textnote{\textit{(Oppure tutta la strofa)} }	

\endverse
\fi





%%%%% RITORNELLO
\beginchorus
%\textnote{\textbf{Rit.}}

\[F]Gloria, \[B&]gloria a \[D-]Dio. \[C]
Gloria, \[F]gloria nel\[B&]l'alto dei \[D-]cie\[C]li.
\[F]Pace in \[B&]terra agli \[D-]uomi\[C]ni
di \[F]buona \[B&]volon\[F]tà. \[B&] 
\[F]Gl\[B&]o\[F]ria!
\endchorus







%%%%% STROFA
\beginverse*		%Oppure \beginverse* se non si vuole il numero di fianco
\memorize 		% <<< DECOMMENTA se si vuole utilizzarne la funzione
%\chordsoff		& <<< DECOMMENTA se vuoi una strofa senza accordi
Noi \[B&]ti lo\[F]diamo, \[G-]ti benedi\[F]ciamo,
ti \[B&]ador\[F]iamo, glo\[E&]rifichiamo \[F]te,
\[B&]ti ren\[F]diamo \[G-7]grazie per la \[F]tua immensa
\[E&]glor\[C4]ia. \[C]

\endverse


%%%%% STROFA
\beginverse*
Si^gnore ^Dio, ^glor^ia!  ^Re del ci^elo, ^glor^ia!
^Dio ^Padre, ^Dio onnipo^tente, ^glor\[C]ia! \[G-] \[E&] \[C]
\endverse



%%%%% RITORNELLO
\beginchorus
%\textnote{\textbf{Rit.}}

\[F]Gloria, \[B&]gloria a \[D-]Dio. \[C]
Gloria, \[F]gloria nel\[B&]l'alto dei \[D-]cie\[C]li.
\[F]Pace in \[B&]terra agli \[D-]uomi\[C]ni
di \[F]buona \[B&]volon\[F]tà. \[B&] 
\[F]Gl\[B&]o\[F]ria!
\endchorus



%%%%% STROFA
\beginverse*
Si\[F]gnore, Figlio uni\[E&]genito, \[B&]Gesù Cri\[F]sto,
Si\[F]gnore, Agnello di \[E&]Dio, \[B&]Figlio del Pad\[F]re.
\[F]Tu che togli i pec\[E&]cati del mondo,
a\[B&]bbi pietà  di no\[F]i;
\[F]tu che togli i pec\[E&]cati del mondo,
a\[B&]ccogli la nostra su\[F]pplica;
\[F]tu che siedi alla \[E&]destra del Padre,
\[B&]abbi pietà  di n\[C4]oi. \[C]
\endverse




%%%%% RITORNELLO
\beginchorus
%\textnote{\textbf{Rit.}}

\[F]Gloria, \[B&]gloria a \[D-]Dio. \[C]
Gloria, \[F]gloria nel\[B&]l'alto dei \[D-]cie\[C]li.
\[F]Pace in \[B&]terra agli \[D-]uomi\[C]ni
di \[F]buona \[B&]volon\[F]tà. \[B&] 
\[F]Gl\[B&]o\[F]ria!
\endchorus


%%%%% STROFA
\beginverse*
Per^chè tu ^solo il ^Santo, il Si^gnore,
tu ^solo l'Al^tissimo, ^Cristo G^esù
^con lo ^Spirito ^Santo nella ^gloria
del ^Pad\[C]re. \[G-] \[E&] \[C]
\endverse



%%%%% RITORNELLO
\beginchorus
%\textnote{\textbf{Rit.}}

\[F]Gloria, \[B&]gloria a \[D-]Dio. \[C]
Gloria, \[F]gloria nel\[B&]l'alto dei \[D-]cie\[C]li.
\[F]Pace in \[B&]terra agli \[D-]uomi\[C]ni
di \[F]buona \[B&]volon\[F]tà. \[B&] 
\[F]Gl\[B&]o\[F]ria! \[F] \[*F]
\endchorus




\endsong
%------------------------------------------------------------
%			FINE CANZONE
%------------------------------------------------------------
%++++++++++++++++++++++++++++++++++++++++++++++++++++++++++++
%			CANZONE TRASPOSTA
%++++++++++++++++++++++++++++++++++++++++++++++++++++++++++++
\ifchorded
%decremento contatore per avere stesso numero
\addtocounter{songnum}{-1} 
\beginsong{Gloria nell'alto dei cieli}[by={Gen Verde, Esme}] 	% <<< COPIA TITOLO E AUTORE
\transpose{-3} 						% <<< TRASPOSIZIONE #TONI + - (0 nullo)
\ifchorded
	\textnote{Tonalità migliore per le chitarre}	% <<< EV COMMENTI (tonalità originale/migliore)
\fi




%%%%%% INTRODUZIONE
\ifchorded
\vspace*{\versesep}
\textnote{Intro: \qquad \qquad  }%(\eighthnote 116) % << MODIFICA IL TEMPO
% Metronomo: \eighthnote (ottavo) \quarternote (quarto) \halfnote (due quarti)
\vspace*{-\versesep}
\beginverse*

\nolyrics

%---- Prima riga -----------------------------
\vspace*{-\versesep}
\[F]   \[B&]  \[C]  \[C]	 \rep{2} % \[*D] per indicare le pennate, \rep{2} le ripetizioni

%---- Ogni riga successiva -------------------
%\vspace*{-\versesep}
%\[G] \[C]  \[D]	

%---- Ev Indicazioni -------------------------			
%\textnote{\textit{(Oppure tutta la strofa)} }	

\endverse
\fi





%%%%% RITORNELLO
\beginchorus
%\textnote{\textbf{Rit.}}

\[F]Gloria, \[B&]gloria a \[D-]Dio. \[C]
Gloria, \[F]gloria nel\[B&]l'alto dei \[D-]cie\[C]li.
\[F]Pace in \[B&]terra agli \[D-]uomi\[C]ni
di \[F]buona \[B&]volon\[F]tà. \[B&] 
\[F]Gl\[B&]o\[F]ria!
\endchorus







%%%%% STROFA
\beginverse*		%Oppure \beginverse* se non si vuole il numero di fianco
\memorize 		% <<< DECOMMENTA se si vuole utilizzarne la funzione
%\chordsoff		& <<< DECOMMENTA se vuoi una strofa senza accordi
Noi \[B&]ti lo\[F]diamo, \[G-]ti benedi\[F]ciamo,
ti \[B&]ador\[F]iamo, glo\[E&]rifichiamo \[F]te,
\[B&]ti ren\[F]diamo \[G-7]grazie per la \[F]tua immensa
\[E&]glor\[C4]ia. \[C]

\endverse


%%%%% STROFA
\beginverse*
Si^gnore ^Dio, ^glor^ia!  ^Re del ci^elo, ^glor^ia!
^Dio ^Padre, ^Dio onnipo^tente, ^glor\[C]ia! \[G-] \[E&] \[C]
\endverse



%%%%% RITORNELLO
\beginchorus
%\textnote{\textbf{Rit.}}

\[F]Gloria, \[B&]gloria a \[D-]Dio. \[C]
Gloria, \[F]gloria nel\[B&]l'alto dei \[D-]cie\[C]li.
\[F]Pace in \[B&]terra agli \[D-]uomi\[C]ni
di \[F]buona \[B&]volon\[F]tà. \[B&] 
\[F]Gl\[B&]o\[F]ria!
\endchorus



%%%%% STROFA
\beginverse*
Si\[F]gnore, Figlio uni\[E&]genito, \[B&]Gesù Cri\[F]sto,
Si\[F]gnore, Agnello di \[E&]Dio, \[B&]Figlio del Pad\[F]re.
\[F]Tu che togli i pec\[E&]cati del mondo,
a\[B&]bbi pietà  di no\[F]i;
\[F]tu che togli i pec\[E&]cati del mondo,
a\[B&]ccogli la nostra su\[F]pplica;
\[F]tu che siedi alla \[E&]destra del Padre,
\[B&]abbi pietà  di n\[C4]oi. \[C]
\endverse




%%%%% RITORNELLO
\beginchorus
%\textnote{\textbf{Rit.}}

\[F]Gloria, \[B&]gloria a \[D-]Dio. \[C]
Gloria, \[F]gloria nel\[B&]l'alto dei \[D-]cie\[C]li.
\[F]Pace in \[B&]terra agli \[D-]uomi\[C]ni
di \[F]buona \[B&]volon\[F]tà. \[B&] 
\[F]Gl\[B&]o\[F]ria!
\endchorus


%%%%% STROFA
\beginverse*
Per^chè tu ^solo il ^Santo, il Si^gnore,
tu ^solo l'Al^tissimo, ^Cristo G^esù
^con lo ^Spirito ^Santo nella ^gloria
del ^Pad\[C]re. \[G-] \[E&] \[C]
\endverse



%%%%% RITORNELLO
\beginchorus
%\textnote{\textbf{Rit.}}

\[F]Gloria, \[B&]gloria a \[D-]Dio. \[C]
Gloria, \[F]gloria nel\[B&]l'alto dei \[D-]cie\[C]li.
\[F]Pace in \[B&]terra agli \[D-]uomi\[C]ni
di \[F]buona \[B&]volon\[F]tà. \[B&] 
\[F]Gl\[B&]o\[F]ria! \[F] \[*F]
\endchorus




\endsong

\fi
%++++++++++++++++++++++++++++++++++++++++++++++++++++++++++++
%			FINE CANZONE TRASPOSTA
%++++++++++++++++++++++++++++++++++++++++++++++++++++++++++++




% ----- ALLELUIA ---------
\makeatletter \def\input@path{{Songs/Alleluia/}} \makeatother
%-------------------------------------------------------------
%			INIZIO	CANZONE
%-------------------------------------------------------------


%titolo: 	Alleluia, festa con Te
%autore: 	Fabio Avolio
%tonalita: 	Do 



%%%%%% TITOLO E IMPOSTAZONI
\beginsong{Alleluia, festa con Te}[by={Fabio Avolio}] 	% <<< MODIFICA TITOLO E AUTORE
\transpose{0} 						% <<< TRASPOSIZIONE #TONI (0 nullo)
\momenti{Acclamazione al Vangelo}							% <<< INSERISCI MOMENTI	
% momenti vanno separati da ; e vanno scelti tra:
% Ingresso; Atto penitenziale; Acclamazione al Vangelo; Dopo il Vangelo; Offertorio; Comunione; Ringraziamento; Fine; Santi; Pasqua; Avvento; Natale; Quaresima; Canti Mariani; Battesimo; Prima Comunione; Cresima; Matrimonio; Meditazione; Spezzare del pane;
\ifchorded
	%\textnote{Tonalità originale }	% <<< EV COMMENTI (tonalità originale/migliore)
\fi

%%%%%% INTRODUZIONE
\ifchorded
\vspace*{\versesep}
\textnote{Intro: \qquad \qquad  }%(\eighthnote 116) % <<  MODIFICA IL TEMPO
% Metronomo: \eighthnote (ottavo) \quarternote (quarto) \halfnote (due quarti)
\vspace*{-\versesep}
\beginverse*

\nolyrics

%---- Prima riga -----------------------------
\vspace*{-\versesep}
\[C] \[F] \[G] \[A-] \[F] \[C] \[D-] \[G]	 % \[*D] per indicare le pennate, \rep{2} le ripetizioni

%---- Ogni riga successiva -------------------
%\vspace*{-\versesep}
%\[G] \[C]  \[D]	

%---- Ev Indicazioni -------------------------			
%\textnote{\textit{(Oppure tutta la strofa)} }	

\endverse
\fi

%%%%% RITORNELLO
\beginchorus
\textnote{\textbf{Rit.}}

\[C]Allelu\[F]ia, \[G]allelu\[A-]ia, \[F]oggi è \[C]festa con \[D-]te, Ges\[G]ù.
\[C]Tu sei con \[F]noi, \[G]gioia ci \[A-]dai, \[F]allelu\[C]ia, alle\[G]lu\[C]ia.

\endchorus

%%%%% STROFA
\beginverse		%Oppure \beginverse* se non si vuole il numero di fianco
\memorize 		% <<< DECOMMENTA se si vuole utilizzarne la funzione
%\chordsoff		% <<< DECOMMENTA se vuoi una strofa senza accordi

\[C]Nella tua \[G]casa \[C]siamo ve\[F]nuti \[C]per incon\[D-]trar\[G]ti.
\[C]A te can\[G]tiamo \[A-]la nostra \[F]lode, \[C]gloria al tuo \[G]no\[C]me. \[G]

\endverse

%%%%% STROFA
\beginverse		%Oppure \beginverse* se non si vuole il numero di fianco
%\memorize 		% <<< DECOMMENTA se si vuole utilizzarne la funzione
\chordsoff		% <<< DECOMMENTA se vuoi una strofa senza accordi

^Il pane ^vivo ^che ci hai pro^messo ^dona la ^vi^ta.
^A te can^tiamo ^la nostra ^lode, ^gloria al tuo ^no^me. ^

\endverse

%%%%% STROFA
\beginverse		%Oppure \beginverse* se non si vuole il numero di fianco
%\memorize 		% <<< DECOMMENTA se si vuole utilizzarne la funzione
\chordsoff		% <<< DECOMMENTA se vuoi una strofa senza accordi

^Tu sei l'^mico ^che ci accom^pagna ^lungo il cam^mi^no.
^A te can^tiamo ^la nostra ^lode, ^gloria al tuo ^no^me. ^

\endverse

\endsong
%------------------------------------------------------------
%			FINE CANZONE
%------------------------------------------------------------ 




% ----- LITURGIA ---------
\makeatletter \def\input@path{{Songs/Liturgia/}} \makeatother
%-------------------------------------------------------------
%			INIZIO	CANZONE
%-------------------------------------------------------------


%titolo: 	Antica eterna danza
%autore: 	Gen Verde
%tonalita: 	Sol 



%%%%%% TITOLO E IMPOSTAZONI
\beginsong{Antica eterna danza}[by={Gen Verde}] 	% <<< MODIFICA TITOLO E AUTORE
\transpose{0} 						% <<< TRASPOSIZIONE #TONI (0 nullo)
\momenti{Offertorio; Prima Comunione}							% <<< INSERISCI MOMENTI	
% momenti vanno separati da ; e vanno scelti tra:
% Ingresso; Atto penitenziale; Acclamazione al Vangelo; Dopo il Vangelo; Offertorio; Comunione; Ringraziamento; Fine; Santi; Pasqua; Avvento; Natale; Quaresima; Canti Mariani; Battesimo; Prima Comunione; Cresima; Matrimonio; Meditazione;
\ifchorded
	%\textnote{Tonalità originale }	% <<< EV COMMENTI (tonalità originale/migliore)
\fi





%%%%%% INTRODUZIONE
\ifchorded
\vspace*{\versesep}
\textnote{Intro: \qquad \qquad  }%(\eighthnote 116) % << MODIFICA IL TEMPO
% Metronomo: \eighthnote (ottavo) \quarternote (quarto) \halfnote (due quarti)
\vspace*{-\versesep}
\beginverse*

\nolyrics

%---- Prima riga -----------------------------
\vspace*{-\versesep}
\[*G] \[*D]  \[*C]	 % \[*D] per indicare le pennate, \rep{2} le ripetizioni

%---- Ogni riga successiva -------------------
\vspace*{-\versesep}
\[G] \[C]  \[G]	

%---- Ev Indicazioni -------------------------			
%\textnote{\textit{(Oppure tutta la strofa)} }	

\endverse
\fi



%%%%% STROFA
\beginverse
\memorize

\[G]Spighe \[D]d'oro al \[E-]vento, 
an\[E-]tica, e\[D]terna \[C]danza
per \[A-]fare un \[D]solo \[E-]pa\[D]ne
spez\[C]zato \[A-]sulla m\[B]ensa.
\[G]Grappoli \[D]dei \[E-]colli, 
pro\[E-]fumo \[D]di le\[C]tizia
per \[A-]fare un \[D]solo \[E-]vi\[D]no 
be\[A-]vanda \[B-]della \[E-]grazia.  

\endverse



%%%%%% EV. INTERMEZZO
\beginverse*
\vspace*{1.3\versesep}
{
	\nolyrics
	
	\ifchorded
	\textnote{Intermezzo strumentale}

		
	%---- Prima riga -----------------------------
	\vspace*{-\versesep}
	\[*G] \[*D]  \[*C]	 % \[*D] per indicare le pennate, \rep{2} le ripetizioni

	%---- Ogni riga successiva -------------------
	\vspace*{-\versesep}
	\[G] \[C]  \[G] 


	\fi
	%---- Ev Indicazioni -------------------------			
	%\textnote{\textit{(ripetizione della strofa)}} 
	 
}
\vspace*{\versesep}
\endverse




%%%%% STROFA
\beginverse

^Con il pa^ne e il vi^no 
Si^gnore ^ti doni^amo
le ^nostre gi^oie ^pu^re, 
le at^tese e ^le pa^ure.
^Frutti ^del la^voro 
e ^fede n^el fu^turo,
la ^voglia ^di cam^bia^re 
e ^di ri^cominci^are.  

\endverse



%%%%%% EV. INTERMEZZO
\beginverse*
\vspace*{1.3\versesep}
{
	\nolyrics
	
	\ifchorded
	\textnote{Intermezzo strumentale}

		
	%---- Prima riga -----------------------------
	\vspace*{-\versesep}
	\[*G] \[*D]  \[*C]	 % \[*D] per indicare le pennate, \rep{2} le ripetizioni

	%---- Ogni riga successiva -------------------
	\vspace*{-\versesep}
	\[G] \[C]  \[G]	


	\fi
	%---- Ev Indicazioni -------------------------			
	%\textnote{\textit{(ripetizione della strofa)}} 
	 
}
\vspace*{\versesep}
\endverse




%%%%% STROFA
\beginverse

^Dio del^la spe^ranza, 
sor^gente ^d'ogni ^dono
ac^cogli q^uesta o^ffer^ta 
che in^sieme ^Ti porti^amo.
^Dio dell'^uni^verso 
rac^cogli ^chi è dis^perso
e ^facci ^tutti ^Chie^sa, 

\vspace*{1.3\versesep}
\textnote{\textit{(rallentando)}}
u^na ^cosa in ^Te.
\endverse
\endsong





%% ----- SANTO ---------
\makeatletter \def\input@path{{Songs/Santo/}} \makeatother
%-------------------------------------------------------------
%			INIZIO	CANZONE
%-------------------------------------------------------------


%titolo: 	Santo Milan
%autore: 	Gen Verde
%tonalita: 	Sol 



%%%%%% TITOLO E IMPOSTAZONI
\beginsong{Santo Milan}[by={Gen Verde}] 	% <<< MODIFICA TITOLO E AUTORE
\transpose{0} 						% <<< TRASPOSIZIONE #TONI (0 nullo)
\momenti{}							% <<< INSERISCI MOMENTI	
% momenti vanno separati da ; e vanno scelti tra:
% Ingresso; Atto penitenziale; Acclamazione al Vangelo; Dopo il Vangelo; Offertorio; Comunione; Ringraziamento; Fine; Santi; Pasqua; Avvento; Natale; Quaresima; Canti Mariani; Battesimo; Prima Comunione; Cresima; Matrimonio; Meditazione;
\ifchorded
	\textnote{Tonalità originale }	% <<< EV COMMENTI (tonalità originale/migliore)
\fi


%%%%%% INTRODUZIONE
\ifchorded
\vspace*{\versesep}
\textnote{Intro: \qquad \qquad  }%(\eighthnote 116) % << MODIFICA IL TEMPO
% Metronomo: \eighthnote (ottavo) \quarternote (quarto) \halfnote (due quarti)
\vspace*{-\versesep}
\beginverse*

\nolyrics

%---- Prima riga -----------------------------
\vspace*{-\versesep}
\[A] \[E]  \[C#-] \[B]	 % \[*D] per indicare le pennate, \rep{2} le ripetizioni

%---- Ogni riga successiva -------------------
\vspace*{-\versesep}
\[F#-] \[E]  \[A]  \[B]	

%---- Ev Indicazioni -------------------------			
\textnote{\textit{(come le prime due righe)} }	

\endverse
\fi








%%%%% RITORNELLO
\beginchorus

\[A]San\[E]to, \[C#-]San\[B]to,
\[F#-]Santo il Si\[E]gnore, \[A]Dio dell'uni\[B]verso.
\[A]San\[E]to, \[C#-]San\[B]to.
I \[F#-]cieli e la \[E]terra 
sono \[A]pieni della tua \[(F#-)]glo\[E]ria.

\endchorus



%%%%% STROFA
\beginverse*		%Oppure \beginverse* se non si vuole il numero di fianco
%\memorize 		% <<< DECOMMENTA se si vuole utilizzarne la funzione
%\chordsoff		% <<< DECOMMENTA se vuoi una strofa senza accordi

O\[A]sanna nel\[B]l'alto dei \[A]cie\[B]li.
O\[F#-]sanna nell'alto dei \[A]cieli.

\endverse



%%%%% RITORNELLO
\beginchorus

\[A]San\[E]to, \[C#-]San\[B]to,
\[F#-]Santo il Si\[E]gnore, \[A]Dio dell'uni\[B]verso.
\[A]San\[E]to, \[C#-]San\[B]to.
I \[F#-]cieli e la \[E]terra 
sono \[A]pieni della tua \[(F#-)]glo\[E]ria.

\endchorus




%%%%% STROFA
\beginverse*		%Oppure \beginverse* se non si vuole il numero di fianco
%\memorize 		% <<< DECOMMENTA se si vuole utilizzarne la funzione
%\chordsoff		& <<< DECOMMENTA se vuoi una strofa senza accordi

\[B]Benedetto co\[A]lui che viene
nel \[E]nome del Sig\[B]nore.
O\[A]sanna nel\[B]l'alto dei \[A]cie\[B]li.
O\[F#-]sanna nell'alto dei \[A]cieli.

\endverse


%%%%% RITORNELLO
\beginchorus

\[A]San\[E]to, \[C#-]San\[B]to,
\[F#-]Sa-\[A]a-n\[E]to.  \[*E] 

\endchorus











\endsong
%------------------------------------------------------------
%			FINE CANZONE
%------------------------------------------------------------




%++++++++++++++++++++++++++++++++++++++++++++++++++++++++++++
%			CANZONE TRASPOSTA
%++++++++++++++++++++++++++++++++++++++++++++++++++++++++++++
\ifchorded
%decremento contatore per avere stesso numero
\addtocounter{songnum}{-1} 
\beginsong{Santo Milan}[by={Gen Verde}] 	% <<< COPIA TITOLO E AUTORE
\transpose{-2} 						% <<< TRASPOSIZIONE #TONI + - (0 nullo)
\ifchorded
	\textnote{Tonalità più facile per le chitarre}	% <<< EV COMMENTI (tonalità originale/migliore)
\fi


%%%%%% INTRODUZIONE
\ifchorded
\vspace*{\versesep}
\textnote{Intro: \qquad \qquad  }%(\eighthnote 116) % << MODIFICA IL TEMPO
% Metronomo: \eighthnote (ottavo) \quarternote (quarto) \halfnote (due quarti)
\vspace*{-\versesep}
\beginverse*

\nolyrics

%---- Prima riga -----------------------------
\vspace*{-\versesep}
\[A] \[E]  \[C#-] \[B]	 % \[*D] per indicare le pennate, \rep{2} le ripetizioni

%---- Ogni riga successiva -------------------
\vspace*{-\versesep}
\[F#-] \[E]  \[A]  \[B]	

%---- Ev Indicazioni -------------------------			
\textnote{\textit{(come le prime due righe)} }	

\endverse
\fi








%%%%% RITORNELLO
\beginchorus

\[A]San\[E]to, \[C#-]San\[B]to,
\[F#-]Santo il Si\[E]gnore, \[A]Dio dell'uni\[B]verso.
\[A]San\[E]to, \[C#-]San\[B]to.
I \[F#-]cieli e la \[E]terra 
sono \[A]pieni della tua \[(F#-)]glo\[E]ria.

\endchorus



%%%%% STROFA
\beginverse*		%Oppure \beginverse* se non si vuole il numero di fianco
%\memorize 		% <<< DECOMMENTA se si vuole utilizzarne la funzione
%\chordsoff		% <<< DECOMMENTA se vuoi una strofa senza accordi

O\[A]sanna nel\[B]l'alto dei \[A]cie\[B]li.
O\[F#-]sanna nell'alto dei \[A]cieli.

\endverse



%%%%% RITORNELLO
\beginchorus

\[A]San\[E]to, \[C#-]San\[B]to,
\[F#-]Santo il Si\[E]gnore, \[A]Dio dell'uni\[B]verso.
\[A]San\[E]to, \[C#-]San\[B]to.
I \[F#-]cieli e la \[E]terra 
sono \[A]pieni della tua \[(F#-)]glo\[E]ria.

\endchorus




%%%%% STROFA
\beginverse*		%Oppure \beginverse* se non si vuole il numero di fianco
%\memorize 		% <<< DECOMMENTA se si vuole utilizzarne la funzione
%\chordsoff		& <<< DECOMMENTA se vuoi una strofa senza accordi

\[B]Benedetto co\[A]lui che viene
nel \[E]nome del Sig\[B]nore.
O\[A]sanna nel\[B]l'alto dei \[A]cie\[B]li.
O\[F#-]sanna nell'alto dei \[A]cieli.

\endverse


%%%%% RITORNELLO
\beginchorus

\[A]San\[E]to, \[C#-]San\[B]to,
\[F#-]Sa-\[A]a-n\[E]to.  \[*E] 

\endchorus






\endsong

\fi
%++++++++++++++++++++++++++++++++++++++++++++++++++++++++++++
%			FINE CANZONE TRASPOSTA
%++++++++++++++++++++++++++++++++++++++++++++++++++++++++++++





% ----- LITURGIA ---------
\makeatletter \def\input@path{{Songs/Liturgia/}} \makeatother
%-------------------------------------------------------------
%			INIZIO	CANZONE
%-------------------------------------------------------------


%titolo: 	Beato il cuore GMG 2016
%autore: Jakub Blycharz
%tonalita: 	Fa 



%%%%%% TITOLO E IMPOSTAZONI
\beginsong{Beato il cuore }[ by={Inno GMG Cracovia 2016 — J. Blycharz}, ititle={Inno GMG Cracovia 2016}]
\transpose{-2} 						% <<< TRASPOSIZIONE #TONI (0 nullo)
\momenti{Comunione; Fine; Ingresso}							% <<< INSERISCI MOMENTI	
% momenti vanno separati da ; e vanno scelti tra:
% Ingresso; Atto penitenziale; Acclamazione al Vangelo; Dopo il Vangelo; Offertorio; Comunione; Ringraziamento; Fine; Santi; Pasqua; Avvento; Natale; Quaresima; Canti Mariani; Battesimo; Prima Comunione; Cresima; Matrimonio; Meditazione;
\ifchorded
	\textnote{$\bigstar$ Tonalità migliore}	% <<< EV COMMENTI (tonalità originale/migliore)
\fi


%%%%%% INTRODUZIONE
\ifchorded
\vspace*{\versesep}
\musicnote{
\begin{minipage}{0.48\textwidth}
\textbf{Intro}
\hfill 
%( \eighthnote \, 80)   % <<  MODIFICA IL TEMPO
% Metronomo: \eighthnote (ottavo) \quarternote (quarto) \halfnote (due quarti)
\end{minipage}
} 	
\vspace*{-\versesep}
\beginverse*

\nolyrics

%---- Prima riga -----------------------------
\vspace*{-\versesep}
\[C#-]  \[A]	\[E]  % \[*D] per indicare le pennate, \rep{2} le ripetizioni

%---- Ogni riga successiva -------------------
\vspace*{-\versesep}
\[B] \[(B)] \[(B)]  \[C#-]

%---- Ev Indicazioni -------------------------			
%\textnote{\textit{(Oppure tutta la strofa)} }	

\endverse
\fi




%%%%% STROFA
\beginverse		%Oppure \beginverse* se non si vuole il numero di fianco
\memorize 		% <<< DECOMMENTA se si vuole utilizzarne la funzione
%\chordsoff		& <<< DECOMMENTA se vuoi una strofa senza accordi

\[C#-]Sei sceso \[A]dalla tua immensi\[E]tà
\[D]in nostro a\[A]iu\[E]to.
Miseri\[B]cordia  scorre  da \[F#]te
\[A]sopra \[B]tutti \[C#]noi.


\endverse


%%%%% STROFA
\beginverse*	%Oppure \beginverse* se non si vuole il numero di fianco
%\memorize 		% <<< DECOMMENTA se si vuole utilizzarne la funzione
%\chordsoff		& <<< DECOMMENTA se vuoi una strofa senza accordi

^ Persi in un ^mondo d’oscuri^tà
^lì Tu ci ^tro^vi.
Nelle tue ^braccia ci stringi e ^poi
^dai la ^vita per ^noi.


\endverse



%%%%% RITORNELLO
\beginchorus
\textnote{\textbf{Rit.}}

Beato è il \[E]cuo\[B]re che per\[C#-]do\[A]na!
Miseri\[E]cordia riceve\[B]rà da Dio in ci\[F#]elo! \rep{2}

\endchorus



%%%%% STROFA
\beginverse		%Oppure \beginverse* se non si vuole il numero di fianco
%\memorize 		% <<< DECOMMENTA se si vuole utilizzarne la funzione
%\chordsoff		% <<< DECOMMENTA se vuoi una strofa senza accordi

^ Solo il per^dono riporte^rà
^pace nel ^mon^do.
Solo il per^dono ci svele^rà
^come f^igli t^uoi.

\endverse



%%%%% RITORNELLO
\beginchorus
\textnote{\textbf{Rit.}}

Beato è il \[E]cuo\[B]re che per\[C#-]do\[A]na!
Miseri\[E]cordia riceve\[B]rà da Dio in ci\[F#]elo! \rep{2}

\endchorus




%%%%% STROFA
\beginverse		%Oppure \beginverse* se non si vuole il numero di fianco
%\memorize 		% <<< DECOMMENTA se si vuole utilizzarne la funzione
%\chordsoff		% <<< DECOMMENTA se vuoi una strofa senza accordi

^ Col sangue in ^croce hai pagato ^Tu
^le nostre ^pover^tà.
Se noi ci am^iamo e restiamo in^ te
^il mondo ^crede^rà!

\endverse



%%%%% RITORNELLO
\beginchorus
\textnote{\textbf{Rit.}}

Beato è il \[E]cuo\[B]re che per\[C#-]do\[A]na!
Miseri\[E]cordia riceve\[B]rà da Dio in ci\[F#]elo! \rep{2}

\endchorus




%%%%% BRIDGE
\beginverse*		%Oppure \beginverse* se non si vuole il numero di fianco
%\memorize 		% <<< DECOMMENTA se si vuole utilizzarne la funzione
%\chordsoff		% <<< DECOMMENTA se vuoi una strofa senza accordi
\textnote{\textbf{Bridge}}
\[A]Le nostre an\[B]gosce ed ansie\[C#-]tà
get\[A]tiamo ogni \[B]attimo in \[A]te.
Amore \[B]che non abbandona \[C#-]mai,
\[A]vivi in \[B]mezzo a \[C#]noi!

\endverse



%%%%% RITORNELLO
\beginchorus
\textnote{\textbf{Rit.}}

Beato è il \[A]cuo\[E]re che per\[F#-]do\[D]na!
Miseri\[A]cordia riceve\[E]rà da Dio in ci\[B]elo! \rep{4}

\endchorus



%%%%%% EV. INTERMEZZO
\beginverse*
\vspace*{1.3\versesep}
{
	\nolyrics
	\musicnote{Chiusura strumentale}
	
	\ifchorded

	%---- Prima riga -----------------------------
	\vspace*{-\versesep}
	\[C#-]  \[A]	\[E]  % \[*D] per indicare le pennate, \rep{2} le ripetizioni


	%---- Ogni riga successiva -------------------
	\vspace*{-\versesep}
	\[B] \[(B)] \[(B)]  \[C#-]


	\fi
	%---- Ev Indicazioni -------------------------			
	%\textnote{\textit{(ripetizione della strofa)}} 
	 
}
\vspace*{\versesep}
\endverse


\endsong
%------------------------------------------------------------
%			FINE CANZONE
%------------------------------------------------------------




%++++++++++++++++++++++++++++++++++++++++++++++++++++++++++++
%			CANZONE TRASPOSTA
%++++++++++++++++++++++++++++++++++++++++++++++++++++++++++++
\ifchorded
%decremento contatore per avere stesso numero
\addtocounter{songnum}{-1} 
\beginsong{Beato il cuore }[ by={Inno GMG Cracovia 2016 — J. Blycharz}]	% <<< COPIA TITOLO E AUTORE
\transpose{0} 						% <<< TRASPOSIZIONE #TONI + - (0 nullo)
%\preferflats SE VOGLIO FORZARE i bemolle come alterazioni
%\prefersharps SE VOGLIO FORZARE i # come alterazioni
\ifchorded
	\textnote{$\lozenge$ Tonalità originale}	% <<< EV COMMENTI (tonalità originale/migliore)
\fi


%%%%%% INTRODUZIONE
\ifchorded
\vspace*{\versesep}
\musicnote{
\begin{minipage}{0.48\textwidth}
\textbf{Intro}
\hfill 
%( \eighthnote \, 80)   % <<  MODIFICA IL TEMPO
% Metronomo: \eighthnote (ottavo) \quarternote (quarto) \halfnote (due quarti)
\end{minipage}
} 	
\vspace*{-\versesep}
\beginverse*

\nolyrics

%---- Prima riga -----------------------------
\vspace*{-\versesep}
\[C#-]  \[A]	\[E]  % \[*D] per indicare le pennate, \rep{2} le ripetizioni

%---- Ogni riga successiva -------------------
\vspace*{-\versesep}
\[B] \[(B)] \[(B)]  \[C#-]

%---- Ev Indicazioni -------------------------			
%\textnote{\textit{(Oppure tutta la strofa)} }	

\endverse
\fi




%%%%% STROFA
\beginverse		%Oppure \beginverse* se non si vuole il numero di fianco
\memorize 		% <<< DECOMMENTA se si vuole utilizzarne la funzione
%\chordsoff		& <<< DECOMMENTA se vuoi una strofa senza accordi

\[C#-]Sei sceso \[A]dalla tua immensi\[E]tà
\[D]in nostro a\[A]iu\[E]to.
Miseri\[B]cordia  scorre  da \[F#]te
\[A]sopra \[B]tutti \[C#]noi.


\endverse


%%%%% STROFA
\beginverse*	%Oppure \beginverse* se non si vuole il numero di fianco
%\memorize 		% <<< DECOMMENTA se si vuole utilizzarne la funzione
%\chordsoff		& <<< DECOMMENTA se vuoi una strofa senza accordi

^ Persi in un ^mondo d’oscuri^tà
^lì Tu ci ^tro^vi.
Nelle tue ^braccia ci stringi e ^poi
^dai la ^vita per ^noi.


\endverse



%%%%% RITORNELLO
\beginchorus
\textnote{\textbf{Rit.}}

Beato è il \[E]cuo\[B]re che per\[C#-]do\[A]na!
Miseri\[E]cordia riceve\[B]rà da Dio in ci\[F#]elo! \rep{2}

\endchorus



%%%%% STROFA
\beginverse		%Oppure \beginverse* se non si vuole il numero di fianco
%\memorize 		% <<< DECOMMENTA se si vuole utilizzarne la funzione
%\chordsoff		% <<< DECOMMENTA se vuoi una strofa senza accordi

^ Solo il per^dono riporte^rà
^pace nel ^mon^do.
Solo il per^dono ci svele^rà
^come f^igli t^uoi.

\endverse



%%%%% RITORNELLO
\beginchorus
\textnote{\textbf{Rit.}}

Beato è il \[E]cuo\[B]re che per\[C#-]do\[A]na!
Miseri\[E]cordia riceve\[B]rà da Dio in ci\[F#]elo! \rep{2}

\endchorus




%%%%% STROFA
\beginverse		%Oppure \beginverse* se non si vuole il numero di fianco
%\memorize 		% <<< DECOMMENTA se si vuole utilizzarne la funzione
%\chordsoff		% <<< DECOMMENTA se vuoi una strofa senza accordi

^ Col sangue in ^croce hai pagato ^Tu
^le nostre ^pover^tà.
Se noi ci am^iamo e restiamo in^ te
^il mondo ^crede^rà!

\endverse



%%%%% RITORNELLO
\beginchorus
\textnote{\textbf{Rit.}}

Beato è il \[E]cuo\[B]re che per\[C#-]do\[A]na!
Miseri\[E]cordia riceve\[B]rà da Dio in ci\[F#]elo! \rep{2}

\endchorus




%%%%% BRIDGE
\beginverse*		%Oppure \beginverse* se non si vuole il numero di fianco
%\memorize 		% <<< DECOMMENTA se si vuole utilizzarne la funzione
%\chordsoff		% <<< DECOMMENTA se vuoi una strofa senza accordi
\textnote{\textbf{Bridge}}
\[A]Le nostre an\[B]gosce ed ansie\[C#-]tà
get\[A]tiamo ogni \[B]attimo in \[A]te.
Amore \[B]che non abbandona \[C#-]mai,
\[A]vivi in \[B]mezzo a \[C#]noi!

\endverse



%%%%% RITORNELLO
\beginchorus
\textnote{\textbf{Rit.}}

Beato è il \[A]cuo\[E]re che per\[F#-]do\[D]na!
Miseri\[A]cordia riceve\[E]rà da Dio in ci\[B]elo! \rep{4}

\endchorus



%%%%%% EV. INTERMEZZO
\beginverse*
\vspace*{1.3\versesep}
{
	\nolyrics
	\musicnote{Chiusura strumentale}
	
	\ifchorded

	%---- Prima riga -----------------------------
	\vspace*{-\versesep}
	\[C#-]  \[A]	\[E]  % \[*D] per indicare le pennate, \rep{2} le ripetizioni


	%---- Ogni riga successiva -------------------
	\vspace*{-\versesep}
	\[B] \[(B)] \[(B)]  \[C#-]


	\fi
	%---- Ev Indicazioni -------------------------			
	%\textnote{\textit{(ripetizione della strofa)}} 
	 
}
\vspace*{\versesep}
\endverse


\endsong


\fi
%++++++++++++++++++++++++++++++++++++++++++++++++++++++++++++
%			FINE CANZONE TRASPOSTA
%++++++++++++++++++++++++++++++++++++++++++++++++++++++++++++

%-------------------------------------------------------------
%			INIZIO	CANZONE
%-------------------------------------------------------------


%titolo: 	Re dei re
%autore: 	E. Munda, G. Pretto, L. Christille
%tonalita: 	Mi- 



%%%%%% TITOLO E IMPOSTAZONI
\beginsong{Re dei re}[by={E. Munda, G. Pretto, L. Christille}] 	% <<< MODIFICA TITOLO E AUTORE
\transpose{0} 						% <<< TRASPOSIZIONE #TONI (0 nullo)
%\preferflats  %SE VOGLIO FORZARE i bemolle come alterazioni
%\prefersharps %SE VOGLIO FORZARE i # come alterazioni
\momenti{Meditazione; Ringraziamento; Comunione; Fine}							% <<< INSERISCI MOMENTI	
% momenti vanno separati da ; e vanno scelti tra:
% Ingresso; Atto penitenziale; Acclamazione al Vangelo; Dopo il Vangelo; Offertorio; Comunione; Ringraziamento; Fine; Santi; Pasqua; Avvento; Natale; Quaresima; Canti Mariani; Battesimo; Prima Comunione; Cresima; Matrimonio; Meditazione; Spezzare del pane;
\ifchorded
	%\textnote{Tonalità migliore }	% <<< EV COMMENTI (tonalità originale/migliore)
\fi


%%%%%% INTRODUZIONE
\ifchorded
\vspace*{\versesep}
\textnote{Intro: \qquad \qquad  (\quarternote 106)} % <<  MODIFICA IL TEMPO
% Metronomo: \eighthnote (ottavo) \quarternote (quarto) \halfnote (due quarti)
\vspace*{-\versesep}
\beginverse*

\nolyrics

%---- Prima riga -----------------------------
\vspace*{-\versesep}
\[E-] \[C] \[G] \[D]	 \rep{2}% \[*D] per indicare le pennate, \rep{2} le ripetizioni

%---- Ogni riga successiva -------------------
%\vspace*{-\versesep}
%\[G] \[C]  \[D]	

%---- Ev Indicazioni -------------------------			
%\textnote{\textit{(Oppure tutta la strofa)} }	

\endverse
\fi




%%%%% STROFA
\beginverse		%Oppure \beginverse* se non si vuole il numero di fianco
\memorize 		% <<< DECOMMENTA se si vuole utilizzarne la funzione
%\chordsoff		% <<< DECOMMENTA se vuoi una strofa senza accordi

\[E-]Hai solle\[C]vato \brk i nostri \[G]volti dalla \[D]polvere, 
\[E-] le nostre \[C]colpe hai por\[G]tato su di \[D]te. 
\[E-] Signore \[C]ti sei fatto \brk \[G]uomo in tutto \[D]come noi 
\[E-]per \[C]a-\[G]mo-\[D]re, 

\endverse
\beginverse*

^Figlio dell’altissimo, ^povero tra i poveri, 
^vieni a dimorare tra ^noi. 
^Dio dell’impossibile, ^Re di tutti i secoli 
^vieni nella tua mae^stà. 

\endverse



%%%%% RITORNELLO
\beginchorus
\textnote{\textbf{Rit.}}

^Re dei ^re 
i ^popoli ti acclamano, \brk i ^cieli ti proclamano 
^Re dei ^re 

^luce degli uomini, \brk ^regna con il tuo amore tra \[E-]no-\[C]o-\[G]o\[D]i 
\[E-]Oo\[C]oh \[G]\[D] \brk \[E-]Oo\[C]oh \[G]\[D]  \brk \[E-]Oo\[C]oh \[G]\[D]  
\[E-]Oh... \[E-] \[E-*]

\endchorus


%%%%%% EV. INTERMEZZO
\beginverse*
\vspace*{1.3\versesep}
{
	\nolyrics
	\textnote{Intermezzo musicale}
	
	\ifchorded

    %---- Prima riga -----------------------------
    \vspace*{-\versesep}
    \[E-] \[C] \[G] \[D]	 \rep{2}% \[*D] per indicare le pennate, \rep{2} le ripetizioni



	\fi
	%---- Ev Indicazioni -------------------------			
	%\textnote{\textit{(ripetizione della strofa)}} 
	 
}
\vspace*{\versesep}
\endverse


%%%%% STROFA
\beginverse		%Oppure \beginverse* se non si vuole il numero di fianco
%\memorize 		% <<< DECOMMENTA se si vuole utilizzarne la funzione
%\chordsoff		% <<< DECOMMENTA se vuoi una strofa senza accordi

^ Ci hai riscat^tati \brk dalla s^tretta delle ^tenebre, 
^ perché po^tessimo glo^rificare ^te. 
^ Hai river^sato in noi \brk la ^vita del tuo ^Spirito 
^per ^a-^mo-^re, 

\endverse
\beginverse*

^Figlio dell’altissimo, ^povero tra i poveri, 
^vieni a dimorare tra ^noi. 
^Dio dell’impossibile, ^Re di tutti i secoli 
^vieni nella tua mae^stà. 

\endverse



%%%%% RITORNELLO
\beginchorus
\textnote{\textbf{Rit.}}

^Re dei ^re 
i ^popoli ti acclamano, \brk i ^cieli ti proclamano 
^Re dei ^re 

^luce degli uomini, \brk ^regna con il tuo amore tra \[E-]no-\[C]o-\[G]o\[D]i 
\[E-]Oo\[C]oh \[G]\[D] \brk \[E-]Oo\[C]oh \[G]\[D]  \brk \[E-]Oo\[C]oh \[G]\[D]  
\[E-]Oh... \[E-] \[E-] \[E-]

\endchorus




%%%%% BRIDGE
\beginverse*		%Oppure \beginverse* se non si vuole il numero di fianco
%\memorize 		% <<< DECOMMENTA se si vuole utilizzarne la funzione
%\chordsoff		% <<< DECOMMENTA se vuoi una strofa senza accordi
\vspace*{1.3\versesep}
\textnote{Bridge} %<<< EV. INDICAZIONI

^Tua ^è la ^glo^ria per ^se-^em^pre, ^ 
^Tua ^è la ^glo^ria per ^se-^em^pre, ^ 
\endverse
\beginverse*
\vspace*{-\versesep}
^glo^ria, ^glo^ria, ^glo^ria, ^glo^ria.  

\endverse
\beginverse*

^Figlio dell’altissimo, ^povero tra i poveri, 
^vieni a dimorare tra ^noi. 
^Dio dell’impossibile, ^Re di tutti i secoli 
^vieni nella tua mae^stà. 

\endverse




%%%%% RITORNELLO
\beginchorus
\textnote{\textbf{Rit.}}

^Re dei ^re 
i ^popoli ti acclamano, \brk i ^cieli ti proclamano 
^Re dei ^re 

^luce degli uomini, \brk ^regna con il tuo amore tra \[E-]no-\[C]o-\[G]o\[D]i 
\[E-]Oo\[C]oh \[G]\[D] \brk \[E-]Oo\[C]oh \[G]\[D]  \brk \[E-]Oo\[C]oh \[G]\[D]  
\[E-]Oh... \[E-] \[E-] \[E-] \[E-*]

\endchorus




\endsong
%------------------------------------------------------------
%			FINE CANZONE
%------------------------------------------------------------



%++++++++++++++++++++++++++++++++++++++++++++++++++++++++++++
%			CANZONE TRASPOSTA
%++++++++++++++++++++++++++++++++++++++++++++++++++++++++++++
\ifchorded
%decremento contatore per avere stesso numero
\addtocounter{songnum}{-1} 
\beginsong{Re dei re}[by={E. Munda, G. Pretto, L. Christille}] 	% <<< MODIFICA TITOLO E AUTORE
\transpose{-2} 							% <<< TRASPOSIZIONE #TONI + - (0 nullo)
\preferflats  %SE VOGLIO FORZARE i bemolle come alterazioni
%\prefersharps %SE VOGLIO FORZARE i # come alterazioni
\ifchorded
	\textnote{Con aumento di tonalità}	% <<< EV COMMENTI (tonalità originale/migliore)
\fi




%%%%%% INTRODUZIONE
\ifchorded
\vspace*{\versesep}
\textnote{Intro: \qquad \qquad  }%(\eighthnote 116) % <<  MODIFICA IL TEMPO
% Metronomo: \eighthnote (ottavo) \quarternote (quarto) \halfnote (due quarti)
\vspace*{-\versesep}
\beginverse*

\nolyrics

%---- Prima riga -----------------------------
\vspace*{-\versesep}
\[E-] \[C] \[G] \[D]	 \rep{2}% \[*D] per indicare le pennate, \rep{2} le ripetizioni

%---- Ogni riga successiva -------------------
%\vspace*{-\versesep}
%\[G] \[C]  \[D]	

%---- Ev Indicazioni -------------------------			
%\textnote{\textit{(Oppure tutta la strofa)} }	

\endverse
\fi




%%%%% STROFA
\beginverse		%Oppure \beginverse* se non si vuole il numero di fianco
\memorize 		% <<< DECOMMENTA se si vuole utilizzarne la funzione
%\chordsoff		% <<< DECOMMENTA se vuoi una strofa senza accordi

\[E-]Hai solle\[C]vato \brk i nostri \[G]volti dalla \[D]polvere, 
\[E-] le nostre \[C]colpe hai por\[G]tato su di \[D]te. 
\[E-] Signore \[C]ti sei fatto \brk \[G]uomo in tutto \[D]come noi 
\[E-]per \[C]a-\[G]mo-\[D]re, 

\endverse
\beginverse*

^Figlio dell’altissimo, ^povero tra i poveri, 
^vieni a dimorare tra ^noi. 
^Dio dell’impossibile, ^Re di tutti i secoli 
^vieni nella tua mae^stà. 

\endverse




%%%%% RITORNELLO
\beginchorus
\textnote{\textbf{Rit.}}

^Re dei ^re 
i ^popoli ti acclamano, \brk i ^cieli ti proclamano 
^Re dei ^re 

^luce degli uomini, \brk ^regna con il tuo amore tra \[E-]no-\[C]o-\[G]o\[D]i 
\[E-]Oo\[C]oh \[G]\[D] \brk \[E-]Oo\[C]oh \[G]\[D]  \brk \[E-]Oo\[C]oh \[G]\[D]  
\[E-]Oh... \[E-] \[E-] \[E-]

\endchorus


\transpose{2}
%%%%%% EV. INTERMEZZO
\beginverse*
\vspace*{1.3\versesep}
{
	\nolyrics
	\textnote{Intermezzo con aumento di tonalità}
	
	\ifchorded

    %---- Prima riga -----------------------------
    \vspace*{-\versesep}
    \[E-] \[C] \[G] \[D]	 \rep{2}% \[*D] per indicare le pennate, \rep{2} le ripetizioni



	\fi
	%---- Ev Indicazioni -------------------------			
	%\textnote{\textit{(ripetizione della strofa)}} 
	 
}
\vspace*{\versesep}
\endverse

%%%%% STROFA
\beginverse		%Oppure \beginverse* se non si vuole il numero di fianco
%\memorize 		% <<< DECOMMENTA se si vuole utilizzarne la funzione
%\chordsoff		% <<< DECOMMENTA se vuoi una strofa senza accordi

^ Ci hai riscat^tati \brk dalla s^tretta delle ^tenebre, 
^ perché po^tessimo glo^rificare ^te. 
^ Hai river^sato in noi \brk la ^vita del tuo ^Spirito 
^per ^a-^mo-^re, 

\endverse
\beginverse*

^Figlio dell’altissimo, ^povero tra i poveri, 
^vieni a dimorare tra ^noi. 
^Dio dell’impossibile, ^Re di tutti i secoli 
^vieni nella tua mae^stà. 

\endverse



%%%%% RITORNELLO
\beginchorus
\textnote{\textbf{Rit.}}

^Re dei ^re 
i ^popoli ti acclamano, \brk i ^cieli ti proclamano 
^Re dei ^re 

^luce degli uomini, \brk ^regna con il tuo amore tra \[E-]no-\[C]o-\[G]o\[D]i 
\[E-]Oo\[C]oh \[G]\[D] \brk \[E-]Oo\[C]oh \[G]\[D]  \brk \[E-]Oo\[C]oh \[G]\[D]  
\[E-]Oh... \[E-] \[E-] \[E-]

\endchorus




%%%%% BRIDGE
\beginverse*		%Oppure \beginverse* se non si vuole il numero di fianco
%\memorize 		% <<< DECOMMENTA se si vuole utilizzarne la funzione
%\chordsoff		% <<< DECOMMENTA se vuoi una strofa senza accordi
\vspace*{1.3\versesep}
\textnote{Bridge} %<<< EV. INDICAZIONI

^Tua ^è la ^glo^ria per ^se-^em^pre, ^ 
^Tua ^è la ^glo^ria per ^se-^em^pre, ^ 
\endverse
\beginverse*
\vspace*{-\versesep}
^glo^ria, ^glo^ria, ^glo^ria, ^glo^ria.  

\endverse
\beginverse*

^Figlio dell’altissimo, ^povero tra i poveri, 
^vieni a dimorare tra ^noi. 
^Dio dell’impossibile, ^Re di tutti i secoli 
^vieni nella tua mae^stà. 

\endverse


%%%%% RITORNELLO
\beginchorus
\textnote{\textbf{Rit.}}

^Re dei ^re 
i ^popoli ti acclamano, \brk i ^cieli ti proclamano 
^Re dei ^re 

^luce degli uomini, \brk ^regna con il tuo amore tra \[E-]no-\[C]o-\[G]o\[D]i 
\[E-]Oo\[C]oh \[G]\[D] \brk \[E-]Oo\[C]oh \[G]\[D]  \brk \[E-]Oo\[C]oh \[G]\[D]  
\[E-]Oh... \[E-] \[E-] \[E-] \[E-*]

\endchorus


\endsong
\fi
%++++++++++++++++++++++++++++++++++++++++++++++++++++++++++++
%			FINE CANZONE TRASPOSTA
%++++++++++++++++++++++++++++++++++++++++++++++++++++++++++++ %MI
%titolo{Le tue meraviglie}
%autore{Casucci, Balduzzi}
%album{Verbum Panis}
%tonalita{La-}
%famiglia{Liturgica}
%gruppo{}
%momenti{Congedo;Natale}
%identificatore{le_tue_meraviglie}
%data_revisione{2011_12_31}
%trascrittore{Francesco Endrici - Manuel Toniato}
\beginsong{Le tue meraviglie}[by={Casucci, Balduzzi}]

\ifchorded
\beginverse*
\vspace*{-0.8\versesep}
{\nolyrics \[A-] \[E-] \[F] \[C] \[D-] \[A-] \[F] \[G] 
\[A-] \[E-] \[F] \[C] \[D-] \[A-] \[F] \[G] \[A-] }
\vspace*{-\versesep}
\endverse
\fi

\beginchorus
Ora \[F]lascia, o Si\[G]gnore, che io \[E-]vada in pa\[A-]ce,
perché ho \[D-]visto le tue \[C]mera\[B&]vi\[G]glie.
Il tuo \[F]popolo in \[G]festa per le \[E-]strade corre\[A-]rà
a por\[D-]tare le tue \[C]mera\[B&]vi\[G]glie!
\endchorus

\beginverse
\[A-]La tua pre\[E-]senza ha riem\[F]pito d'a\[C]more
\[A-]le nostre \[E-]vite, le \[F]nostre gior\[C]nate.
\[B&]In te una sola \[F]anima, \[G-]un solo cuore \[F]siamo noi:
\[B&]con te la luce ri\[F]splende, \brk \[G-]splende più chiara che \[C]mai!
\endverse

\beginverse
\chordsoff
La tua presenza ha inondato d'amore
le nostre vite, le nostre giornate,
fra la tua gente resterai, \brk per sempre vivo in mezzo a noi
fino ai confini del tempo: così ci accompagnerai.
\endverse

\beginchorus
Ora \[F]lascia, o Si\[G]gnore, che io \[E-]vada in pa\[A-]ce,
perché ho \[D-]visto le tue \[C]mera\[B&]vi\[G]glie.
Il tuo \[F]popolo in \[G]festa per le \[E-]strade corre\[A-]rà
a por\[D-]tare le tue \[C]mera\[B&]vi\[G]glie!
Ora \[F]lascia, o Si\[G]gnore, che io \[E-]vada in pa\[A-]ce,
perché ho \[F]visto le \[G]tue mera\[E-]vi\[A-]glie.
E il tuo \[F]popolo in \[G]festa per le \[E-]strade corre\[A-]rà
a por\[F]tare le \[G]tue mera\[F]vi\[C]glie!
\endchorus



%%%%%% FINALE
\ifchorded
\beginverse*
\vspace*{-0.5\versesep}
{
	\nolyrics
	\textbf{Finale:} \quad	
	\[A-] \[E-] \[F] \[C] \[D-] \[A-] \[F] \[G] 
	\[A-] \[E-] \[F] \[C] \[D-] \[A-] \[F] \[G] \[A-]
	 
}
\vspace*{\versesep}
\endverse
\fi





\endsong
%------------------------------------------------------------
%			FINE CANZONE
%------------------------------------------------------------





%% ----- LITURGIA ---------
%\makeatletter \def\input@path{{Songs/Liturgia/}} \makeatother
%%-------------------------------------------------------------
%			INIZIO	CANZONE
%-------------------------------------------------------------


%titolo: 	Lodate il Signore
%autore: 	Daniele Ricci
%tonalita: 	Sol 



%%%%%% TITOLO E IMPOSTAZONI
\beginsong{Lodate il Signore}[by={RnS}] 	% <<< MODIFICA TITOLO E AUTORE
\transpose{0} 						% <<< TRASPOSIZIONE #TONI (0 nullo)
%\preferflats  %SE VOGLIO FORZARE i bemolle come alterazioni
%\prefersharps %SE VOGLIO FORZARE i # come alterazioni
\momenti{}							% <<< INSERISCI MOMENTI	
% momenti vanno separati da ; e vanno scelti tra:
% Ingresso; Atto penitenziale; Acclamazione al Vangelo; Dopo il Vangelo; Offertorio; Comunione; Ringraziamento; Fine; Santi; Pasqua; Avvento; Natale; Quaresima; Canti Mariani; Battesimo; Prima Comunione; Cresima; Matrimonio; Meditazione; Spezzare del pane;
\ifchorded
	%\textnote{Tonalità migliore }	% <<< EV COMMENTI (tonalità originale/migliore)
\fi


%%%%%% INTRODUZIONE
\ifchorded
\vspace*{\versesep}
\musicnote{
\begin{minipage}{0.48\textwidth}
\textbf{Intro}
\hfill 
%( \eighthnote \, 80)   % <<  MODIFICA IL TEMPO
% Metronomo: \eighthnote (ottavo) \quarternote (quarto) \halfnote (due quarti)
\end{minipage}
} 	
\vspace*{-\versesep}
\beginverse*

\nolyrics

%---- Prima riga -----------------------------
\vspace*{-\versesep}
\[A] \[E] \[F#-] \[A]


%---- Ogni riga successiva -------------------
\vspace*{-\versesep}
\[D] \[A] \[B-] \[E]

%---- Ogni riga successiva -------------------
\vspace*{-\versesep}
\[A]



%---- Ev Indicazioni -------------------------			
\textnote{\textit{[come il ritornello]} }	

\endverse
\fi




%%%%% STROFA
\beginverse		%Oppure \beginverse* se non si vuole il numero di fianco
\memorize 		% <<< DECOMMENTA se si vuole utilizzarne la funzione
%\chordsoff		% <<< DECOMMENTA se vuoi una strofa senza accordi

Lo\[A]date il Si\[D]gnore nei c\[E]ieli,
lo\[A]date \[D]angeli \[E]suoi.
Lo\[C]date voi \[F]tutte sue schiere 
la \[E]Sua mae\[E]stà.
\endverse
\beginverse*	
Lo^date da ^tutta la ^terra,
lo^date ^popoli s^uoi.
Lo^date ^giovani e vecchi 
la ^Sua bon^tà.
\endverse

%%%%% RITORNELLO
\beginchorus
\textnote{\textbf{Rit.}}

Cantate al Si\[A]gnore, \[E] 
un cantico \[F#-]nuovo. \[A]
Sia onore al \[D]Re, 
sia gloria al \[A]Dio,
che siede sul \[B-]tro\[E7]no.
Risuoni la \[A]lode, \[E]
la benedi\[F#-]zione al Si\[C#-]gnor,
che era e che \[D]è, che sempre sa\[E]rà.
Allelu\[A]ja! \[E] Allelu\[F#-]ja! \[A]
\endchorus

%%%%%% EV. INTERMEZZO
\beginverse*
\vspace*{1.3\versesep}
{
	\nolyrics
	\textnote{\textbf{Rit.}}
    \textnote{Intermezzo strumentale}
	
	\ifchorded

	%---- Prima riga -----------------------------
	\vspace*{-\versesep}
	\[D] \[A] \[B-] \[E] 

    
	%---- Ogni riga successiva -------------------
	\vspace*{-\versesep}
   \[A]

	\fi
	%---- Ev Indicazioni -------------------------			
	%\textnote{\textit{(ripetizione della strofa)}} 
	 
}
\vspace*{\versesep}
\endverse

%%%%% STROFA
\beginverse		%Oppure \beginverse* se non si vuole il numero di fianco
\memorize 		% <<< DECOMMENTA se si vuole utilizzarne la funzione
%\chordsoff		% <<< DECOMMENTA se vuoi una strofa senza accordi

Gio^iscano ^nel crea^tore,
e^sultino i ^figli di ^Sion.
^danzino, ^facciano festa 
al ^loro ^Dio.

\endverse
\beginverse*	

^Lodino il ^nome del ^Padre,
con ^timpani in^neggino a ^Lui.
C^ieli e ^terra si prostrino 
al ^Re dei ^re. 

\endverse



%%%%% RITORNELLO
\beginchorus
\textnote{\textbf{Rit.}}

Cantate al Si\[A]gnore, \[E] 
un cantico \[F#-]nuovo. \[A]
Sia onore al \[D]Re, 
sia gloria al \[A]Dio,
che siede sul \[B-]tro\[E7]no.
Risuoni la \[A]lode, \[E]
la benedi\[F#-]zione al Si\[C#-]gnor,
che era e che \[D]è, che sempre sa\[E]rà.
Allelu\[A]ja! \[E] Allelu\[F#-]ja! \[A]
\endchorus



%%%%%% EV. INTERMEZZO
\beginverse*
\vspace*{1.3\versesep}
{
	\nolyrics
    \textnote{Intermezzo strumentale}
	
	\ifchorded

	%---- Prima riga -----------------------------
	\vspace*{-\versesep}
	\[D] \[A] \[B-] \[E] 
	\fi
	%---- Ev Indicazioni -------------------------			
	%\textnote{\textit{(ripetizione della strofa)}} 
	 
}
\vspace*{\versesep}
\endverse

%%%%% RITORNELLO
\beginchorus
\vspace*{-\versesep}
Risuoni la \[A]lode, \[E]
la benedi\[F#-]zione al Si\[C#-]gnor,
che era e che \[D]è, che sempre sa\[E]rà.
\textnote{\textit{[si alza la tonalità]}}
Allelu\[A]ja! \[F]
\endchorus



\transpose{1}
\preferflats 



%%%%% RITORNELLO
\beginchorus
\textnote{\textbf{Rit.}}

Cantate al Si\[A]gnore, \[E]
un cantico \[F#-]nuovo. \[A]
Sia onore al \[D]Re, 
sia gloria al \[A]Dio,
che siede sul \[B-]tro\[E7]no.
Risuoni la \[A]lode, \[E]
la benedi\[F#-]zione al Si\[C#-]gnor,
che era e che \[D]è, che sempre sa\[E]rà.
\endchorus




%%%%%% EV. CHIUSURA SOLO STRUMENTALE
\ifchorded
\beginchorus %oppure \beginverse*
\vspace*{1.3\versesep}
\textnote{Chiusura } %<<< EV. INDICAZIONI

\[F]A-\[G4]\[G]a-\[A]alleluja! \[A]Alleluja!   \rep{5} \quad \[A*]
\endchorus  %oppure \endverse
\fi


\endsong
%------------------------------------------------------------
%			FINE CANZONE
%------------------------------------------------------------




%%-------------------------------------------------------------
%			INIZIO	CANZONE
%-------------------------------------------------------------


%titolo: 	Luce del mondo
%autore: 	F. Pesarese, RnS
%tonalita: 	Fa 



%%%%%% TITOLO E IMPOSTAZONI
\beginsong{Luce del mondo}[by={F. Pesarese, RnS}] 	% <<< MODIFICA TITOLO E AUTORE
\transpose{0} 						% <<< TRASPOSIZIONE #TONI (0 nullo)
%\preferflats  %SE VOGLIO FORZARE i bemolle come alterazioni
%\prefersharps %SE VOGLIO FORZARE i # come alterazioni
\momenti{}							% <<< INSERISCI MOMENTI	
% momenti vanno separati da ; e vanno scelti tra:
% Ingresso; Atto penitenziale; Acclamazione al Vangelo; Dopo il Vangelo; Offertorio; Comunione; Ringraziamento; Fine; Santi; Pasqua; Avvento; Natale; Quaresima; Canti Mariani; Battesimo; Prima Comunione; Cresima; Matrimonio; Meditazione; Spezzare del pane;
\ifchorded
	%\textnote{Tonalità migliore }	% <<< EV COMMENTI (tonalità originale/migliore)
\fi


%%%%%% INTRODUZIONE
\ifchorded
\vspace*{\versesep}
\textnote{Intro: \qquad \qquad  }%(\eighthnote 116) % <<  MODIFICA IL TEMPO
% Metronomo: \eighthnote (ottavo) \quarternote (quarto) \halfnote (due quarti)
\vspace*{-\versesep}
\beginverse*

\nolyrics

%---- Prima riga -----------------------------
\vspace*{-\versesep}
\[F] \[B&]  \[F] \[B&]  \[C*]	 % \[*D] per indicare le pennate, \rep{2} le ripetizioni

%---- Ogni riga successiva -------------------
%\vspace*{-\versesep}
%\[G] \[C]  \[D]	

%---- Ev Indicazioni -------------------------			
%\textnote{\textit{(Oppure tutta la strofa)} }	

\endverse
\fi

\newchords{verse}
\newchords{chorus}


%%%%% STROFA
\beginverse		%Oppure \beginverse* se non si vuole il numero di fianco
\memorize[verse]% <<< DECOMMENTA se si vuole utilizzarne la funzione
%\chordsoff		% <<< DECOMMENTA se vuoi una strofa senza accordi

\[F]Luce del mondo sei, Signor,
\[F] il tuo Amore ci illumina
e le \[D-]tenebre che avvolgono il nostro cuor
\[B&]con la tua Luce sva\[C*]nisco\[F]no.
\[F]Luce del mondo sei, Signor,
\[F] il tuo Amore ci illumina
e le \[D-]tenebre che avvolgono il nostro cuor
\[B&]con la tua Luce sva\[C*]nisco\[D-]no,
\[B&]con la tua Luce sva\[C*]nisco\[F]no.
\endverse




%%%%% RITORNELLO
\beginchorus
\textnote{\textbf{Rit.}}
\memorize[chorus]
\[F]Luce del mondo sei,
luce che illumi\[A-]na,
\[D-]luce del mondo sei.
\[B&]Il tuo Amore, Si\[C]gnor,
\[A]mai si spegne\[D-]rà,
perché \[B&*]luce del \[C*]mondo sei,
\[B&*]luce che il\[G-*]lumi\[C]na. \[C]

\endchorus





%%%%% STROFA
\beginverse		%Oppure \beginverse* se non si vuole il numero di fianco
\replay[verse]		% <<< DECOMMENTA se si vuole utilizzarne la funzione
%\chordsoff		% <<< DECOMMENTA se vuoi una strofa senza accordi

^Gioia del mondo sei, Signor,
^ il tuo Amore ci fa cantar,
le tris^tezze che sempre ci opprimono,
^con la tua gioia sva^nisco^no.
^Gioia del mondo sei, Signor,
^ il tuo Amore ci fa cantar,
le tris^tezze che sempre ci opprimono,
^con la tua gioia sva^nisco^no,
^con la tua gioia sva^nisco^no.

\endverse




%%%%% RITORNELLO
\beginchorus
\textnote{\textbf{Rit.}}
\replay[chorus]

^Gioia del mondo sei,
gioia che fa can^tar,
^gioia del mondo sei!
^Il tuo amore, Si^gnor,
^mai si spegne^rà,
perché ^gioia del ^mondo sei,
\vspace*{\versesep}
\textnote{Si alza di tonalità}
^gioia che ^fa can^ta-\[D]re.
\endchorus


\transpose{2}


%%%%% STROFA
\beginverse		%Oppure \beginverse* se non si vuole il numero di fianco
\replay[verse]		% <<< DECOMMENTA se si vuole utilizzarne la funzione
%\chordsoff		% <<< DECOMMENTA se vuoi una strofa senza accordi


^Forza del mondo sei, Signor,
^ il tuo amore ci libera,
le ca^tene che ci legano,
^con la tua forza sva^nisco^no.
^Forza del mondo sei, Signor,
^ il tuo amore ci libera,
le ca^tene che ci legano,
^con la tua forza sva^nisco^no,
^con la tua forza sva^nisco^no.

\endverse




%%%%% RITORNELLO
\beginchorus
\textnote{\textbf{Rit.}}
\replay[chorus]


^Forza del mondo sei,
forza che ^libera,
^forza del mondo sei!
^Il tuo amore, Si^gnor,
^mai si spegne^rà,
perché ^forza del ^mondo sei,
^forza che ^libe^ra. ^
\endchorus

%%%%%% EV. INTERMEZZO
\beginverse*
\vspace*{1.3\versesep}
{
	\nolyrics
	\textnote{Intermezzo strumentale}
	
	\ifchorded

	%---- Prima riga -----------------------------
	\vspace*{-\versesep}
	\[F] \[F] 
    
	%---- Ogni riga successiva -------------------
	\vspace*{-\versesep}
	\[F] \[A-] 

	%---- Ogni riga successiva -------------------
	\vspace*{-\versesep}
	\[D-] \[D-]
 

    %---- Ogni riga successiva -------------------
	\vspace*{-\versesep}
	\[B&] \[C*] \[D-] \[B&] \[C*] \[F] 

	\fi
	%---- Ev Indicazioni -------------------------			
	%\textnote{\textit{(ripetizione della strofa)}} 
	 
}
\vspace*{\versesep}
\endverse



%%%%% STROFA
\beginverse		%Oppure \beginverse* se non si vuole il numero di fianco
%\memorize[verse]% <<< DECOMMENTA se si vuole utilizzarne la funzione
%\chordsoff		% <<< DECOMMENTA se vuoi una strofa senza accordi

\[F]Luce del mondo sei, Signor,
\[F] il tuo Amore ci illumina
e le \[D-]tenebre che avvolgono il nostro cuor
\[B&]con la tua Luce sva\[C*]nisco\[F]no.
\[F]Luce del mondo sei, Signor,
\[F] il tuo Amore ci illumina
e le \[D-]tenebre che avvolgono il nostro cuor
\[B&]con la tua Luce sva\[C*]nisco\[D-]no,
\[B&]con la tua Luce sva\[C*]nisco\[F]no.
\endverse




%%%%% RITORNELLO
\beginchorus
\textnote{\textbf{Rit.}}
%\memorize[chorus]

\[F]Luce del mondo sei,
gioia che fa can\[A-]tar,
\[D-]forza che libera.
\[B&]Il tuo Amore, Si\[C]gnor,
\[A]mai si spegne\[D-]rà,
perché \[B&*]luce del \[C*]mondo sei,
\[B&*]luce che il\[G-*]lumi\[C]na. \[C]

\endchorus

%%%%%% EV. FINALE

\beginchorus %oppure \beginverse*
\vspace*{1.3\versesep}
\textnote{Finale \textit{(rallentando)}} %<<< EV. INDICAZIONI

\[C/A-]Luce che illumi\[F]na. \[F*]

\endchorus  %oppure \endverse




\endsong
%------------------------------------------------------------
%			FINE CANZONE
%------------------------------------------------------------



%%-------------------------------------------------------------
%			INIZIO	CANZONE
%-------------------------------------------------------------


%titolo: 	Saldo è il mio cuore
%autore: 	M. Frisina
%tonalita:  Re 
%youtube: https://www.youtube.com/watch?v=cG_-iab1xcI&feature=youtu.be



%%%%%% TITOLO E IMPOSTAZONI
\beginsong{Saldo è il mio cuore}[by={M. Frisina}] 	% <<< MODIFICA TITOLO E AUTORE
\transpose{0} 						% <<< TRASPOSIZIONE #TONI (0 nullo)
%\preferflats  %SE VOGLIO FORZARE i bemolle come alterazioni
%\prefersharps %SE VOGLIO FORZARE i # come alterazioni
\momenti{}							% <<< INSERISCI MOMENTI	
% momenti vanno separati da ; e vanno scelti tra:
% Ingresso; Atto penitenziale; Acclamazione al Vangelo; Dopo il Vangelo; Offertorio; Comunione; Ringraziamento; Fine; Santi; Pasqua; Avvento; Natale; Quaresima; Canti Mariani; Battesimo; Prima Comunione; Cresima; Matrimonio; Meditazione; Spezzare del pane;
\ifchorded
	%\textnote{Tonalità migliore }	% <<< EV COMMENTI (tonalità originale/migliore)
\fi


%%%%%% INTRODUZIONE
\ifchorded
\vspace*{\versesep}
\musicnote{
\begin{minipage}{0.48\textwidth}
\textbf{Intro}
\hfill 
%( \eighthnote \, 80)   % <<  MODIFICA IL TEMPO
% Metronomo: \eighthnote (ottavo) \quarternote (quarto) \halfnote (due quarti)
\end{minipage}
} 	
\vspace*{-\versesep}
\beginverse*

\nolyrics

%---- Prima riga -----------------------------
\vspace*{-\versesep}
\[D] \[G] \quad \[D]\[A]\[D]	 % \[*D] per indicare le pennate, \rep{2} le ripetizioni

%---- Ogni riga successiva -------------------
%\vspace*{-\versesep}
%\[G] \[C]  \[D]	

%---- Ev Indicazioni -------------------------			
%\textnote{\textit{(Oppure tutta la strofa)} }	

\endverse
\fi




%%%%% STROFA
\beginverse		%Oppure \beginverse* se non si vuole il numero di fianco
\memorize 		% <<< DECOMMENTA se si vuole utilizzarne la funzione
%\chordsoff		% <<< DECOMMENTA se vuoi una strofa senza accordi


\[D]Saldo è il mio \[G]cuore, Dio \[D]\[A]mi\[D]o.
A \[D]te cante\[G]rà l'anima \[A]\[G]mi\[A]a.
De\[D]statevi \[G]arpa e \[D]\[A]ce\[B-]tra,
\[G]vo\[A]glio sve\[B-]glia\[E-]re l'au\[D]\[A]ro\[D]ra. \quad \[A]

\endverse
\beginverse	
\textbf{A ^te la mia ^lode tra le ^^gen^ti,
per^chè fino ai ^cieli \brk è il tuo ^^amo^re.
^Sorgi ed in^nalzati, o ^^Di^o,
^splen^da sul ^mondo ^la tua ^^glo^ria.\quad ^
}
\endverse
\beginverse

Con ^te noi fa^remo cose ^^gran^di.
Con ^te noi con^vertiremo il ^^mon^do.
Tu ^sei nostra ^luce e con^^for^to,
^for^za, ri^fugio, ^o Si^^gno^re. \quad ^

\endverse
\beginverse
\textbf{^Per te noi an^dremo per il ^^mon^do,
^inni cante^remo alla tua ^^glo^ria.
^Donaci la ^grazia, Si^^gno^re,
^an^nunce^remo ^il tuo a^^mo^re. \quad \[D*]
}
\endverse





\endsong
%------------------------------------------------------------
%			FINE CANZONE
%------------------------------------------------------------



%%-------------------------------------------------------------
%			INIZIO	CANZONE
%-------------------------------------------------------------


%titolo: 	Isaia 11
%autore: 	C. Rossi, S. Carocci
%tonalita: 	Do 
%youtube: https://www.youtube.com/watch?v=jf9DEnsybLc&feature=youtu.be


%%%%%% TITOLO E IMPOSTAZONI
\beginsong{Isaia 11}[by={C. Rossi, S. Carocci}] 	% <<< MODIFICA TITOLO E AUTORE
\transpose{0} 						% <<< TRASPOSIZIONE #TONI (0 nullo)
%\preferflats  %SE VOGLIO FORZARE i bemolle come alterazioni
%\prefersharps %SE VOGLIO FORZARE i # come alterazioni
\momenti{}							% <<< INSERISCI MOMENTI	
% momenti vanno separati da ; e vanno scelti tra:
% Ingresso; Atto penitenziale; Acclamazione al Vangelo; Dopo il Vangelo; Offertorio; Comunione; Ringraziamento; Fine; Santi; Pasqua; Avvento; Natale; Quaresima; Canti Mariani; Battesimo; Prima Comunione; Cresima; Matrimonio; Meditazione; Spezzare del pane;
\ifchorded
	%\textnote{Tonalità migliore }	% <<< EV COMMENTI (tonalità originale/migliore)
\fi


%%%%%% INTRODUZIONE
\ifchorded
\vspace*{\versesep}
\textnote{Intro: \qquad \qquad  }%(\eighthnote 116) % <<  MODIFICA IL TEMPO
% Metronomo: \eighthnote (ottavo) \quarternote (quarto) \halfnote (due quarti)
\vspace*{-\versesep}
\beginverse*

\nolyrics

%---- Prima riga -----------------------------
\vspace*{-\versesep}
\[C] \[G] \[A-7] \[G]	 % \[*D] per indicare le pennate, \rep{2} le ripetizioni

%---- Ogni riga successiva -------------------
\vspace*{-\versesep}
\[F]  \[F] \[E-7]  \[E7]	

%---- Ev Indicazioni -------------------------			
%\textnote{\textit{(Oppure tutta la strofa)} }	

\endverse
\fi



%%%%% RITORNELLO
\beginchorus
\textnote{\textbf{Rit.}}

\[C]Ed un vir\[G]gulto dal \[A-7]tronco di \[G]Iesse
do\[F]mani germo-oglie\[E-7]rà. \[E7]
\[C]Un ramo\[G]scello dal\[A-7]le sue ra\[G]dici
a ves\[F]sillo si \[E-]eleve\[A-]rà. 

\endchorus



%%%%% STROFA
\beginverse		%Oppure \beginverse* se non si vuole il numero di fianco
\memorize 		% <<< DECOMMENTA se si vuole utilizzarne la funzione
%\chordsoff		% <<< DECOMMENTA se vuoi una strofa senza accordi

\[D-7]Su lui sapienza, intel\[E-7]letto, consiglio,
for\[D-7]tezza e timor del Si\[E-7]gnor. \[E7]
\[D-7]La sua parola sa\[E-7]rà come verga
e dal \[D-7]male ci libere\[E-7]rà.\[E7] 

\endverse



%%%%% STROFA
\beginverse		%Oppure \beginverse* se non si vuole il numero di fianco
%\memorize 		% <<< DECOMMENTA se si vuole utilizzarne la funzione
%\chordsoff		% <<< DECOMMENTA se vuoi una strofa senza accordi

^L'agnello e il lupo ins^ieme staranno
e ^accanto al capretto viv^ran. ^
^Pascoleranno con ^l'orsa e il leone
un fan^ciullo li guide^rà.^

\endverse


%%%%% STROFA
\beginverse		%Oppure \beginverse* se non si vuole il numero di fianco
%\memorize 		% <<< DECOMMENTA se si vuole utilizzarne la funzione
%\chordsoff		% <<< DECOMMENTA se vuoi una strofa senza accordi

^Ed in quel giorno di ^nuovo il Signore
la ^mano su lui stende^rà. ^
^Come vessillo il ger^moglio di Iesse
sui ^popoli si eleve^rà.^

\endverse





%%%%%% EV. CHIUSURA SOLO STRUMENTALE
\ifchorded
\beginchorus %oppure \beginverse*
\vspace*{1.3\versesep}
\textnote{Chiusura } %<<< EV. INDICAZIONI

\[A-]

\endchorus  %oppure \endverse
\fi


\endsong
%------------------------------------------------------------
%			FINE CANZONE
%------------------------------------------------------------









%% ----- SANTO ---------
%\makeatletter \def\input@path{{Songs/Santo/}} \makeatother
%%-------------------------------------------------------------
%			INIZIO	CANZONE
%-------------------------------------------------------------


%titolo: 	Santo Zappalà
%autore: 	
%tonalita: 	Sol 



%%%%%% TITOLO E IMPOSTAZONI
\beginsong{Santo Zappalà}[by={G. Zappalà, A. Mancuso}] 	% <<< MODIFICA TITOLO E AUTORE
\transpose{0} 						% <<< TRASPOSIZIONE #TONI (0 nullo)
\momenti{Santo}							% <<< INSERISCI MOMENTI	
% momenti vanno separati da ; e vanno scelti tra:
% Ingresso; Atto penitenziale; Acclamazione al Vangelo; Dopo il Vangelo; Offertorio; Comunione; Ringraziamento; Fine; Santi; Pasqua; Avvento; Natale; Quaresima; Canti Mariani; Battesimo; Prima Comunione; Cresima; Matrimonio; Meditazione;
\ifchorded
	%\textnote{$\bigstar$ Tonalità originale }	% <<< EV COMMENTI (tonalità originale\migliore)
\fi


%%%%%% INTRODUZIONE
\ifchorded
\vspace*{\versesep}
\musicnote{
\begin{minipage}{0.48\textwidth}
\textbf{Intro}
\hfill 
%( \eighthnote \, 80)   % <<  MODIFICA IL TEMPO
% Metronomo: \eighthnote (ottavo) \quarternote (quarto) \halfnote (due quarti)
\end{minipage}
} 	
\vspace*{-\versesep}
\beginverse*

\nolyrics

%---- Prima riga -----------------------------
\vspace*{-\versesep}
\[G] \[E-] \[B-] \[D7] 	 % \[*D] per indicare le pennate, \rep{2} le ripetizioni


\endverse
\fi








%%%%% RITORNELLO
\beginchorus

\[G]Santo, \[E-]Santo, \[B-]Santo, \brk il Sign\[C]ore Dio dell’uni\[D]verso.
I \[C]cie\[B-]li e la \[C]ter\[E-]ra \brk sono \[C]pieni \[A7]della tua \[D]gloria.

\endchorus




%%%%% RITORNELLO
\beginchorus

Os\[G]an\[D]na, Os\[G]an\[D]na, \brk Os\[C]anna nell’\[D]alto dei ci\[G]eli. \[C*]\[D*] 
Os\[G]an\[D]na, Os\[G]an\[D]na, \brk Os\[C]anna nell’\[D]alto dei ci\[G]eli.
\endchorus





%%%%% STROFA
\beginverse*		%Oppure \beginverse* se non si vuole il numero di fianco
%\memorize 		% <<< DECOMMENTA se si vuole utilizzarne la funzione
%\chordsoff		% <<< DECOMMENTA se vuoi una strofa senza accordi

Benede\[E-]tto co\[C]lui che \[D]viene nel nome del Si\[G]gnore. \[C*] \[D*]

\endverse


%%%%% RITORNELLO
\beginchorus

Os\[G]an\[D]na, Os\[G]an\[D]na, \brk Os\[C]anna nell’\[D]alto dei ci\[G]eli. \[C*]\[D*] 
Os\[G]an\[D]na, Os\[G]an\[D]na, \brk Os\[C]anna nell’\[D]alto dei ci\[G]eli.
\endchorus








\endsong
%------------------------------------------------------------
%			FINE CANZONE
%------------------------------------------------------------



%
%
%
%
%
%% ----- LITURGIA ---------
%\makeatletter \def\input@path{{Songs/Liturgia/}} \makeatother
%%-------------------------------------------------------------
%			INIZIO	CANZONE
%-------------------------------------------------------------


%titolo: 	Alla tua presenza
%autore: 	F. Tiddia
%tonalita: 	Do



%%%%%% TITOLO E IMPOSTAZONI
\beginsong{Alla tua presenza}[by={F. Tiddia}] 	% <<< MODIFICA TITOLO E AUTORE
\transpose{0} 						% <<< TRASPOSIZIONE #TONI (0 nullo)
%\preferflats  %SE VOGLIO FORZARE i bemolle come alterazioni
%\prefersharps %SE VOGLIO FORZARE i # come alterazioni
\momenti{}							% <<< INSERISCI MOMENTI	
% momenti vanno separati da ; e vanno scelti tra:
% Ingresso; Atto penitenziale; Acclamazione al Vangelo; Dopo il Vangelo; Offertorio; Comunione; Ringraziamento; Fine; Santi; Pasqua; Avvento; Natale; Quaresima; Canti Mariani; Battesimo; Prima Comunione; Cresima; Matrimonio; Meditazione; Spezzare del pane;
\ifchorded
	%\textnote{$\bigstar$ Tonalità migliore }	% <<< EV COMMENTI (tonalità originale/migliore)
\fi


%%%%%% INTRODUZIONE
\ifchorded
\vspace*{\versesep}
\musicnote{
\begin{minipage}{0.48\textwidth}
\textbf{Intro}
\hfill 
%( \eighthnote \, 80)   % <<  MODIFICA IL TEMPO
% Metronomo: \eighthnote (ottavo) \quarternote (quarto) \halfnote (due quarti)
\end{minipage}
} 	
\vspace*{-\versesep}
\beginverse*

\nolyrics

%---- Prima riga -----------------------------
\vspace*{-\versesep}
\[C] \[A-7] \[D-]\[G4] % \[*D] per indicare le pennate, \rep{2} le ripetizioni

%---- Ogni riga successiva -------------------
%\vspace*{-\versesep}
%\[G] \[C]  \[D]	

%---- Ev Indicazioni -------------------------			
%\textnote{\textit{[oppure tutta la strofa]} }	

\endverse
\fi




%%%%% STROFA
\beginverse		%Oppure \beginverse* se non si vuole il numero di fianco
\memorize 		% <<< DECOMMENTA se si vuole utilizzarne la funzione
%\chordsoff		% <<< DECOMMENTA se vuoi una strofa senza accordi

\[C]Alla tua presenza \[A-7]portaci Signore, 
nei tuoi \[D-]atri noi vogliamo dimo\[G4]rar. 
\[C]Nel tuo tempio intoneremo \[A-7]inni a te, 
\[D-7]canti di lode alla tua \[G4]maestà. 

\endverse




%%%%% RITORNELLO
\beginchorus
\textnote{\textbf{Rit.}}

\[F]Il tuo Santo \[G]Spirito \brk ci \[E-]guidi là \[A-]dove sei tu, 
\[D-7]alla tua pre\[G]senza Si\[C]gnore Ge\[E&7]su-\[C]ù. 
\[F]In eterno \[G]canteremo \[E-7]gloria a te, Si\[A-7]gnor. 
\[D-7]Alla tua pre\[G]senza, \[D-7]alla tua pre\[G]senza 
per \[D-7]sempre insie\[G]me a te Ge\[C]sù.

\endchorus



\endsong
%------------------------------------------------------------
%			FINE CANZONE
%------------------------------------------------------------


%%-------------------------------------------------------------
%			INIZIO	CANZONE
%-------------------------------------------------------------


%titolo: 	Come ti ama Dio
%autore: 	Anonimo
%tonalita:  DO



%%%%%% TITOLO E IMPOSTAZONI
\beginsong{Come ti ama Dio}[by={Anonimo}] 	% <<< MODIFICA TITOLO E AUTORE
\transpose{0} 						% <<< TRASPOSIZIONE #TONI (0 nullo)
%\preferflats  %SE VOGLIO FORZARE i bemolle come alterazioni
%\prefersharps %SE VOGLIO FORZARE i # come alterazioni
\momenti{Matrimonio}							% <<< INSERISCI MOMENTI	
% momenti vanno separati da ; e vanno scelti tra:
% Ingresso; Atto penitenziale; Acclamazione al Vangelo; Dopo il Vangelo; Offertorio; Comunione; Ringraziamento; Fine; Santi; Pasqua; Avvento; Natale; Quaresima; Canti Mariani; Battesimo; Prima Comunione; Cresima; Matrimonio; Meditazione; Spezzare del pane;
\ifchorded
	%\textnote{$\bigstar$ Tonalità migliore }	% <<< EV COMMENTI (tonalità originale\migliore)
\fi


%%%%%% INTRODUZIONE
\ifchorded
\vspace*{\versesep}
\musicnote{
\begin{minipage}{0.48\textwidth}
\textbf{Intro}
\hfill 
%( \eighthnote \, 80)   % <<  MODIFICA IL TEMPO
% Metronomo: \eighthnote (ottavo) \quarternote (quarto) \halfnote (due quarti)
\end{minipage}
} 	
\vspace*{-\versesep}
\beginverse*

\nolyrics

%---- Prima riga -----------------------------
\vspace*{-\versesep}
\[C] \[A-] \[F]	 \[G] \rep{2} % \[*D] per indicare le pennate, \rep{2} le ripetizioni

%---- Ogni riga successiva -------------------
%\vspace*{-\versesep}
%\[G] \[C]  \[D]	

%---- Ev Indicazioni -------------------------			
%\textnote{\textit{[oppure tutta la strofa]} }	

\endverse
\fi




%%%%% STROFA
\beginverse		%Oppure \beginverse* se non si vuole il numero di fianco
\memorize 		% <<< DECOMMENTA se si vuole utilizzarne la funzione
%\chordsoff		% <<< DECOMMENTA se vuoi una strofa senza accordi

\[C]Io vorrei sa\[A-]perti amare \[F]come Dio
\[G]che ti prende per \[C]mano ma ti \[A-]lascia anche \[F]andare.
\[G]Vorrei saperti \[C]amare senza \[A-]farti mai \[F]domande,
\[G]felice perchè \[C]esisti e co\[A-]sì io posso \[F]darti \brk il \[G]meglio di \[C]me.

\endverse




%%%%% RITORNELLO
\beginchorus
\textnote{\textbf{Rit.}}

Con la \[G]forza del \[A-]mare, \brk l'e\[F]ternità dei \[C]giorni,
la \[G]gioia dei \[A-]voli, \brk la \[F]pace della \[C]sera,
l'im\[G]mensità del \[A-]cielo: \brk \[F]come ti ama \[C]Dio.

\endchorus



%%%%% STROFA
\beginverse		%Oppure \beginverse* se non si vuole il numero di fianco
%\memorize 		% <<< DECOMMENTA se si vuole utilizzarne la funzione
%\chordsoff		% <<< DECOMMENTA se vuoi una strofa senza accordi

\[C]Io vorrei sa\[A-]perti amare \brk \[F]come ti ama \[G]Dio
che ti cono\[C]sce e ti \[A-]accetta \brk come \[F]sei.
\[G]Tenerti fra le \[C]mani \brk come \[A-]voli nell'\[F]azzurro,
\[G]felice perchè \[C]esisti e \[A-]così io posso \[F]darti \brk il \[G]meglio di \[C]me.
 

\endverse



%%%%% RITORNELLO
\beginchorus
\textnote{\textbf{Rit.}}

Con la \[G]forza del \[A-]mare, \brk l'e\[F]ternità dei \[C]giorni,
la \[G]gioia dei \[A-]voli, \brk la \[F]pace della \[C]sera,
l'im\[G]mensità del \[A-]cielo: \brk \[F]come ti ama \[C]Dio.

\endchorus



%%%%% STROFA
\beginverse		%Oppure \beginverse* se non si vuole il numero di fianco
%\memorize 		% <<< DECOMMENTA se si vuole utilizzarne la funzione
%\chordsoff		% <<< DECOMMENTA se vuoi una strofa senza accordi

\[C]Io vorrei sa\[A-]perti amare \brk \[F]come ti ama \[G]Dio
che ti fa mi\[C]gliore con l'a\[A-]more \brk che ti \[F]dona.
\[G]Seguirti fra la \[C]gente \brk con la \[A-]gioia che hai \[F]dentro,
\[G]felice perchè \[C]esisti e \[A-]così io posso \[F]darti \brk il \[G]meglio di \[C]me.
 

\endverse


%%%%% RITORNELLO
\beginchorus
\textnote{\textbf{Rit.}}

Con la \[G]forza del \[A-]mare, \brk l'e\[F]ternità dei \[C]giorni,
la \[G]gioia dei \[A-]voli,\brk  la \[F]pace della \[C]sera,
l'im\[G]mensità del \[A-]cielo: \brk \[F]come ti ama \[C]Dio.

\endchorus



\endsong
%------------------------------------------------------------
%			FINE CANZONE
%------------------------------------------------------------



%%-------------------------------------------------------------
%			INIZIO	CANZONE
%-------------------------------------------------------------


%titolo: 	Come tu mi vuoi
%autore: 	Daniele Ricci
%tonalita: 	Sol 



%%%%%% TITOLO E IMPOSTAZONI
\beginsong{Come tu mi vuoi}[by={D. Branca}] 	% <<< MODIFICA TITOLO E AUTORE
\transpose{0} 						% <<< TRASPOSIZIONE #TONI (0 nullo)
%\preferflats  %SE VOGLIO FORZARE i bemolle come alterazioni
%\prefersharps %SE VOGLIO FORZARE i # come alterazioni
\momenti{}							% <<< INSERISCI MOMENTI	
% momenti vanno separati da ; e vanno scelti tra:
% Ingresso; Atto penitenziale; Acclamazione al Vangelo; Dopo il Vangelo; Offertorio; Comunione; Ringraziamento; Fine; Santi; Pasqua; Avvento; Natale; Quaresima; Canti Mariani; Battesimo; Prima Comunione; Cresima; Matrimonio; Meditazione; Spezzare del pane;
\ifchorded
	%\textnote{$\bigstar$ Tonalità migliore }	% <<< EV COMMENTI (tonalità originale\migliore)
\fi


%%%%%% INTRODUZIONE
\ifchorded
\vspace*{\versesep}
\musicnote{
\begin{minipage}{0.48\textwidth}
\textbf{Intro}
\hfill 
%( \eighthnote \, 80)   % <<  MODIFICA IL TEMPO
% Metronomo: \eighthnote (ottavo) \quarternote (quarto) \halfnote (due quarti)
\end{minipage}
} 	
\vspace*{-\versesep}
\beginverse*

\nolyrics

%---- Prima riga -----------------------------
\vspace*{-\versesep}
 \[G] \[C] \[D7] \[G]	 % \[*D] per indicare le pennate, \rep{2} le ripetizioni

%---- Ogni riga successiva -------------------
\vspace*{-\versesep}
\[E-] \[A-] \[D4] \[G] 	

%---- Ev Indicazioni -------------------------			
%\textnote{\textit{[oppure tutta la strofa]} }	

\endverse
\fi




%%%%% STROFA
\beginverse		%Oppure \beginverse* se non si vuole il numero di fianco
\memorize 		% <<< DECOMMENTA se si vuole utilizzarne la funzione
%\chordsoff		% <<< DECOMMENTA se vuoi una strofa senza accordi

\[G]Eccomi Signor, vengo a te mio \[A-7]Re,
\[E-]che si compia in me la tua \[G]volon\[D]tà.
\[G]Eccomi Signor, vengo a te mio \[A-7]Dio,
\[E-]plasma il cuore mio \[B-7]e di te vivrò.
\[G]Se tu lo \[C]vuoi Signo\[D]re manda \[E-]me
\[A-]e il tuo nome \[B-]annun\[G]ce\[C]rò.

\endverse




%%%%% RITORNELLO
\beginchorus
\textnote{\textbf{Rit.}}

Come tu mi \[G]vuoi io sa\[D]rò,
dove tu mi \[E-]vuoi io an\[B-7]drò.
Questa \[C]vita io voglio dona\[B-7]rla a \[E-]te
per dar \[F]gloria al tuo nome \[C]mio \[D4]Re.
\[D]Come tu mi \[C]vuoi io sa\[D]rò,
\[B7]dove tu mi \[E-]vuoi io an\[B-7]drò.
Se mi gu\[C]ida il tuo amore pa\[B7]ura non \[E-]ho,   
per se\[A-]mpre io sa\[D]rò  come tu mi vuoi.

\endchorus



%%%%%% EV. INTERMEZZO
\beginverse*
\vspace*{1.3\versesep}
{
	\nolyrics
	\textnote{Intermezzo strumentale}
	
	\ifchorded

	%---- Prima riga -----------------------------
	\vspace*{-\versesep}
	\[G] \[C] \[D7] \[G]




	\fi
	%---- Ev Indicazioni -------------------------			
	%\textnote{\textit{(ripetizione della strofa)}} 
	 
}
\vspace*{\versesep}
\endverse




%%%%% STROFA
\beginverse		%Oppure \beginverse* se non si vuole il numero di fianco
%\memorize 		% <<< DECOMMENTA se si vuole utilizzarne la funzione
%\chordsoff		% <<< DECOMMENTA se vuoi una strofa senza accordi

\[G]Eccomi Signor, vengo a te mio \[A-7]Re,
\[E-]che si compia in me la tua \[G]volon\[D]tà.
\[G]Eccomi Signor, vengo a te mio \[A-7]Dio,
\[E-]plasma il cuore mio \[B-7]e di te vivrò..
\[G]Tra le tue \[C]mani mai \[D]più vacil\[E-]lerò
\[A-]e strumento \[B-]tuo \[G]sa\[C]rò.

\endverse







%%%%% RITORNELLO
\beginchorus
\textnote{\textbf{Rit.}}

Come tu mi \[G]vuoi io sa\[D]rò,
dove tu mi \[E-]vuoi io an\[B-7]drò.
Questa \[C]vita io voglio dona\[B-7]rla a \[E-]te
per dar \[F]gloria al tuo nome \[C]mio \[D4]Re.
\[D]Come tu mi \[C]vuoi io sa\[D]rò,
\[B7]dove tu mi \[E-]vuoi io an\[B-7]drò.
Se mi gu\[C]ida il tuo amore pa\[B7]ura non \[E-]ho,   
per se\[A-]mpre io sa\[D]rò  come tu mi vuoi.

\endchorus




%%%%%% EV. FINALE

\beginchorus %oppure \beginverse*
\vspace*{1.3\versesep}
\textnote{\textbf{Finale }} %<<< EV. INDICAZIONI
come tu mi vuoi\[C] \[D].\echo{Io sarò}  \rep{4} 

\endchorus  %oppure \endverse






%%%%%% EV. CHIUSURA SOLO STRUMENTALE
\ifchorded
\beginchorus %oppure \beginverse*
\vspace*{1.3\versesep}
\textnote{Chiusura } %<<< EV. INDICAZIONI

\[G*]

\endchorus  %oppure \endverse
\fi


\endsong
%------------------------------------------------------------
%			FINE CANZONE
%------------------------------------------------------------




%%-------------------------------------------------------------
%			INIZIO	CANZONE
%-------------------------------------------------------------


%titolo: 	Dio in me
%autore: 	SERMIG
%tonalita: 	Fa 



%%%%%% TITOLO E IMPOSTAZONI
\beginsong{Dio in me}[by={Sermig}] 	% <<< MODIFICA TITOLO E AUTORE
\transpose{0} 						% <<< TRASPOSIZIONE #TONI (0 nullo)
%\preferflats  %SE VOGLIO FORZARE i bemolle come alterazioni
%\prefersharps %SE VOGLIO FORZARE i # come alterazioni
\momenti{}							% <<< INSERISCI MOMENTI	
% momenti vanno separati da ; e vanno scelti tra:
% Ingresso; Atto penitenziale; Acclamazione al Vangelo; Dopo il Vangelo; Offertorio; Comunione; Ringraziamento; Fine; Santi; Pasqua; Avvento; Natale; Quaresima; Canti Mariani; Battesimo; Prima Comunione; Cresima; Matrimonio; Meditazione; Spezzare del pane;
\ifchorded
	%\textnote{Tonalità migliore }	% <<< EV COMMENTI (tonalità originale\migliore)
\fi


%%%%%% INTRODUZIONE
\ifchorded
\vspace*{\versesep}
\musicnote{
\begin{minipage}{0.48\textwidth}
\textbf{Intro}
\hfill 
%( \eighthnote \, 80)   % <<  MODIFICA IL TEMPO
% Metronomo: \eighthnote (ottavo) \quarternote (quarto) \halfnote (due quarti)
\end{minipage}
} 	
\vspace*{-\versesep}
\beginverse*

\nolyrics

%---- Prima riga -----------------------------
\vspace*{-\versesep}
\[F] \[A-] \[G4] \[G]

%---- Ogni riga successiva -------------------
\vspace*{-\versesep}
\[F] \[A-] \[G4] \[G]	

%---- Ev Indicazioni -------------------------			
%\textnote{\textit{(Oppure tutta la strofa)} }	

\endverse
\fi




%%%%% STROFA
\beginverse		%Oppure \beginverse* se non si vuole il numero di fianco
\memorize 		% <<< DECOMMENTA se si vuole utilizzarne la funzione
%\chordsoff		% <<< DECOMMENTA se vuoi una strofa senza accordi
Sei \[C]qui total\[F]mente \[C]D\[G]i\[A-]o 
dentro m\[G]e 
sei \[C]qui total\[F]mente \[C]u\[G]o\[A-]mo 
dentro \[G]me
\[C]E \[F]vuoi \[G]che \[C]io \[G]vi\[A-]va 
per \[G]te 
\[C]Sil\[F]e-\[G]e-\[C]nzio \[G]pre\[A-]ga 
con \[G]me con \[F]me 

\endverse






%%%%%% EV. INTERMEZZO
\beginverse*
\vspace*{1.3\versesep}
{
	\nolyrics
	\textnote{Intermezzo strumentale}
	
	\ifchorded

	%---- Prima riga -----------------------------
	\vspace*{-\versesep}
	\[F] \[A-] \[G4] \[G]

	%---- Ogni riga successiva -------------------
	\vspace*{-\versesep}
	\[F] \[A-] \[G4] \[G]


	\fi
	%---- Ev Indicazioni -------------------------			
	%\textnote{\textit{(ripetizione della strofa)}} 
	 
}
\vspace*{\versesep}
\endverse


%%%%% STROFA
\beginverse		%Oppure \beginverse* se non si vuole il numero di fianco
%\memorize 		% <<< DECOMMENTA se si vuole utilizzarne la funzione
%\chordsoff		% <<< DECOMMENTA se vuoi una strofa senza accordi

Per \[C]me ti sei \[F]fatto \[C]u\[G]o\[A-]mo 
come \[G]me 
La \[C]Croce tre\[F]menda \[C]più \[G]non \[A-]è 
dopo \[G]che 
\[C]Tu \[F]l’hai \[G]re\[C]sa \[G]bene\[A-]det\[G]ta 
\[C]Sil\[F]e-\[G]e-\[C]nzio \[G]pre\[A-]ga 
con \[G]me
\[C]A\[F]de-\[G]e-\[C]sso \[G]incontr\[A-]ando \[G]me 
\[C]non \[F]tro\[G]ve\[C]re\[G]te \[A-]me 
ma \[G]Dio
 in \[F]me  \[A-] \[G4] \[G]  
 in \[F]me \[A-] \[G4] \[G]


\endverse



%%%%%% EV. CHIUSURA SOLO STRUMENTALE
\ifchorded
\beginchorus %oppure \beginverse*
\vspace*{1.3\versesep}
\textnote{Chiusura strumentale } %<<< EV. INDICAZIONI

\nolyrics
%---- Prima riga -----------------------------
\vspace*{-\versesep}
\[F] \[A-] \[G4] \[G]

%---- Ogni riga successiva -------------------
\vspace*{-\versesep}
\[F] \[A-] \[G4] \[G]

%---- Ev Indicazioni -------------------------			
%\textnote{\textit{(Oppure tutta la strofa)} }	

\endchorus  %oppure \endverse
\fi


\endsong
%------------------------------------------------------------
%			FINE CANZONE
%------------------------------------------------------------




%%-------------------------------------------------------------
%			INIZIO	CANZONE
%-------------------------------------------------------------


%titolo: 	Ti do La pace
%autore: 	SERMIG
%tonalita: 	Fa



%%%%%% TITOLO E IMPOSTAZONI
\beginsong{Ti do La pace}[by={Sermig}] 	% <<< MODIFICA TITOLO E AUTORE
\transpose{0} 						% <<< TRASPOSIZIONE #TONI (0 nullo)
%\preferflats  %SE VOGLIO FORZARE i bemolle come alterazioni
%\prefersharps %SE VOGLIO FORZARE i # come alterazioni
\momenti{}							% <<< INSERISCI MOMENTI	
% momenti vanno separati da ; e vanno scelti tra:
% Ingresso; Atto penitenziale; Acclamazione al Vangelo; Dopo il Vangelo; Offertorio; Comunione; Ringraziamento; Fine; Santi; Pasqua; Avvento; Natale; Quaresima; Canti Mariani; Battesimo; Prima Comunione; Cresima; Matrimonio; Meditazione; Spezzare del pane;
\ifchorded
	%\textnote{$\bigstar$ Tonalità migliore }	% <<< EV COMMENTI (tonalità originale\migliore)
\fi


%%%%%% INTRODUZIONE
\ifchorded
\vspace*{\versesep}
\musicnote{
\begin{minipage}{0.48\textwidth}
\textbf{Intro}
\hfill 
%( \eighthnote \, 80)   % <<  MODIFICA IL TEMPO
% Metronomo: \eighthnote (ottavo) \quarternote (quarto) \halfnote (due quarti)
\end{minipage}
} 	
\vspace*{-\versesep}
\beginverse*

\nolyrics

%---- Prima riga -----------------------------
\vspace*{-\versesep}
\[B&] \[C] \[F]	 % \[*D] per indicare le pennate, \rep{2} le ripetizioni

%---- Ogni riga successiva -------------------
%\vspace*{-\versesep}
%\[G] \[C]  \[D]	

%---- Ev Indicazioni -------------------------			
%\textnote{\textit{[oppure tutta la strofa]} }	

\endverse
\fi




%%%%% STROFA
\beginverse		%Oppure \beginverse* se non si vuole il numero di fianco
\memorize 		% <<< DECOMMENTA se si vuole utilizzarne la funzione
%\chordsoff		% <<< DECOMMENTA se vuoi una strofa senza accordi

Ti do la \[B&]pace \[C]perché ci cre\[F]do 
ti do la \[B&]pace \[C]perché la viv\[F]o 
ti do la \[D-]pace \[C]perché la vogli\[B&]o
per t\[G-]e e per tutte le don\[A4]ne 
e tu\[G-]tti gli uomini del \[A]mondo 
ti do la \[B&]pace \[C]perché ci cred\[F]o.

\endverse


%%%%% STROFA
\beginverse		%Oppure \beginverse* se non si vuole il numero di fianco
\memorize 		% <<< DECOMMENTA se si vuole utilizzarne la funzione
%\chordsoff		% <<< DECOMMENTA se vuoi una strofa senza accordi

Ti do la \[B&]pace \[C]perché ci cred\[F]o 
ti do la \[B&]pace \[C]perché la viv\[F]o 
ti do la p\[D-]ace \[C]perché la vo\[B&]glio 
per \[G-]te e per tutte le \[A4]donne
e \[G-]tutti gli uomini del \[A4]mondo


\endverse




%%%%% STROFA
\beginverse		%Oppure \beginverse* se non si vuole il numero di fianco
%\memorize 		% <<< DECOMMENTA se si vuole utilizzarne la funzione
%\chordsoff		% <<< DECOMMENTA se vuoi una strofa senza accordi
Ti do la \[B&]pace \[C]perché io spero ch\[F]e 
la pace \[B&]possa \[C]abitare semp\[F]re 
\[D-]e nel creato e in \[C]tutte le crea\[B&]ture.

\endverse


%%%%% STROFA
\beginverse		%Oppure \beginverse* se non si vuole il numero di fianco
%\memorize 		% <<< DECOMMENTA se si vuole utilizzarne la funzione
%\chordsoff		% <<< DECOMMENTA se vuoi una strofa senza accordi

Ti do la \[B&]pace \[C]perché ci cred\[F]o, 
ti do la \[B&]pace \[C]perché ci cred\[F]o, 
ti do la \[B&]pace, \[C]la voglio anche per \[F]te.  


\endverse



\endsong
%------------------------------------------------------------
%			FINE CANZONE
%------------------------------------------------------------


%%-------------------------------------------------------------
%			INIZIO	CANZONE
%-------------------------------------------------------------


%titolo: 	Santo Ricci
%autore: 	Daniele Ricci
%tonalita: 	Sol 



%%%%%% TITOLO E IMPOSTAZONI
\beginsong{Vieni o Spirito}[by={M. C. Bizzeti}] 	% <<< MODIFICA TITOLO E AUTORE
\transpose{0} 						% <<< TRASPOSIZIONE #TONI (0 nullo)
%\preferflats  %SE VOGLIO FORZARE i bemolle come alterazioni
%\prefersharps %SE VOGLIO FORZARE i # come alterazioni
\momenti{}							% <<< INSERISCI MOMENTI	
% momenti vanno separati da ; e vanno scelti tra:
% Ingresso; Atto penitenziale; Acclamazione al Vangelo; Dopo il Vangelo; Offertorio; Comunione; Ringraziamento; Fine; Santi; Pasqua; Avvento; Natale; Quaresima; Canti Mariani; Battesimo; Prima Comunione; Cresima; Matrimonio; Meditazione; Spezzare del pane;
\ifchorded
	%\textnote{$\bigstar$ Tonalità migliore }	% <<< EV COMMENTI (tonalità originale\migliore)
\fi


%%%%%% INTRODUZIONE
\ifchorded
\vspace*{\versesep}
\musicnote{
\begin{minipage}{0.48\textwidth}
\textbf{Intro}
\hfill 
%( \eighthnote \, 80)   % <<  MODIFICA IL TEMPO
% Metronomo: \eighthnote (ottavo) \quarternote (quarto) \halfnote (due quarti)
\end{minipage}
} 	
\vspace*{-\versesep}
\beginverse*

\nolyrics

%---- Prima riga -----------------------------
\vspace*{-\versesep}
\[B-] \[A] \[B-]	 % \[*D] per indicare le pennate, \rep{2} le ripetizioni

%---- Ogni riga successiva -------------------
%\vspace*{-\versesep}
%\[G] \[C]  \[D]	

%---- Ev Indicazioni -------------------------			
%\textnote{\textit{[oppure tutta la strofa]} }	

\endverse
\fi



%%%%% RITORNELLO
\beginchorus
\textnote{\textbf{Rit.}}

\[B-]Vieni o Sprito \[B-/D]Spirito di Dio,	
\[A]vieni o \[F#-]Spirito \[B-]Santo.	
\[B-]Vieni o Spirito soffia su di noi 
\[A]dona ai tuoi \[F#-]figli la \[B-]vita.

\endchorus



%%%%% STROFA
\beginverse		%Oppure \beginverse* se non si vuole il numero di fianco
%\memorize 		% <<< DECOMMENTA se si vuole utilizzarne la funzione
%\chordsoff		% <<< DECOMMENTA se vuoi una strofa senza accordi


\[G]Dona la \[A]luce ai nostri \[B-]occhi,
\[D]dona la \[A]forza ai nostri \[B-]cuori,
\[F#]dona alle menti la sapie\[B-]nza,
\[G]dona il tuo \[A]fuoco d'am\[B-]ore.

\endverse

%%%%% RITORNELLO
\beginchorus
\textnote{\textbf{Rit.}}

\[B-]Vieni o Sprito \[B-/D]Spirito di Dio,	
\[A]vieni o \[F#-]Spirito \[B-]Santo.	
\[B-]Vieni o Spirito soffia su di noi 
\[A]dona ai tuoi \[F#-]figli la \[B-]vita.

\endchorus


%%%%% STROFA
\beginverse		%Oppure \beginverse* se non si vuole il numero di fianco
%\memorize 		% <<< DECOMMENTA se si vuole utilizzarne la funzione
%\chordsoff		% <<< DECOMMENTA se vuoi una strofa senza accordi

\[G]Tu sei per \[A]noi consola\[B-]tore;
\[D]nella \[A]calura sei \[B-]riparo
\[F#]nella fatica sei \[B-]riposo
\[G]nel pianto \[A]sei conf\[B-]orto.

\endverse





%%%%% RITORNELLO
\beginchorus
\textnote{\textbf{Rit.}}

\[B-]Vieni o Sprito \[B-/D]Spirito di Dio,	
\[A]vieni o \[F#-]Spirito \[B-]Santo.	
\[B-]Vieni o Spirito soffia su di noi 
\[A]dona ai tuoi \[F#-]figli la \[B-]vita.

\endchorus


%%%%% STROFA
\beginverse		%Oppure \beginverse* se non si vuole il numero di fianco
%\memorize 		% <<< DECOMMENTA se si vuole utilizzarne la funzione
%\chordsoff		% <<< DECOMMENTA se vuoi una strofa senza accordi

\[G]Dona a \[A]tutti i tuoi \[B-]fedeli
\[D]Che \[A]confidano in \[B-]Te.
\[F#]I tuoi sette Santi \[B-]doni,
\[G]dona la \[A]gioia \[B-]eterna.

\endverse





\endsong
%------------------------------------------------------------
%			FINE CANZONE
%------------------------------------------------------------


%%-------------------------------------------------------------
%			INIZIO	CANZONE
%-------------------------------------------------------------


%titolo: 	Santo Ricci
%autore: 	Daniele Ricci
%tonalita: 	Sol 



%%%%%% TITOLO E IMPOSTAZONI
\beginsong{Io credo in te}[by={Sermig}] 	% <<< MODIFICA TITOLO E AUTORE
\transpose{0} 						% <<< TRASPOSIZIONE #TONI (0 nullo)
%\preferflats  %SE VOGLIO FORZARE i bemolle come alterazioni
%\prefersharps %SE VOGLIO FORZARE i # come alterazioni
\momenti{Congedo; Pasqua}							% <<< INSERISCI MOMENTI	
% momenti vanno separati da ; e vanno scelti tra:
% Ingresso; Atto penitenziale; Acclamazione al Vangelo; Dopo il Vangelo; Offertorio; Comunione; Ringraziamento; Fine; Santi; Pasqua; Avvento; Natale; Quaresima; Canti Mariani; Battesimo; Prima Comunione; Cresima; Matrimonio; Meditazione; Spezzare del pane;
\ifchorded
	%\textnote{$\bigstar$ Tonalità migliore }	% <<< EV COMMENTI (tonalità originale\migliore)
\fi


%%%%%% INTRODUZIONE
\ifchorded
\vspace*{\versesep}
\musicnote{
\begin{minipage}{0.48\textwidth}
\textbf{Intro}
\hfill 
%( \eighthnote \, 80)   % <<  MODIFICA IL TEMPO
% Metronomo: \eighthnote (ottavo) \quarternote (quarto) \halfnote (due quarti)
\end{minipage}
} 	
\vspace*{-\versesep}
\beginverse*

\nolyrics

%---- Prima riga -----------------------------
\vspace*{-\versesep}
\[A-] \[F] \[C]	\[G]  \[A-]% \[*D] per indicare le pennate, \rep{2} le ripetizioni

%---- Ogni riga successiva -------------------
%\vspace*{-\versesep}
%\[G] \[C]  \[D]	

%---- Ev Indicazioni -------------------------			
%\textnote{\textit{[oppure tutta la strofa]} }	

\endverse
\fi




%%%%% STROFA
\beginverse		%Oppure \beginverse* se non si vuole il numero di fianco
\memorize 		% <<< DECOMMENTA se si vuole utilizzarne la funzione
%\chordsoff		% <<< DECOMMENTA se vuoi una strofa senza accordi

\[A-]Per \[F]molti Tu sei \[C]storia, 
la \[G]pagina di un \[A-]libro 
un \[F]ruolo tea\[C]tra\[G]le. 
\[A-]Le tue \[F]urla, il tuo dol\[C]ore 
\[G]rivivono ogni \[A-]giorno 
ma \[F]non nel nostro \[C]cuo\[G]re.

\endverse




%%%%% RITORNELLO
\beginchorus
\textnote{\textbf{Rit.}}

\[A-]Io \[F]credo in \[C]Te \brk \echo{tu \[G]sei il figlio di \[A-]Dio} 
io \[F]credo in \[C]Te \brk \echo{ Tu \[G]sei risorto e \[A-]vivo} 
Io \[F]credo in \[C]Te \brk \echo{chi \[G]ha l'amore nei suoi \[A-]occhi ti \[F]riconos\[C]cerà.\[G]} \rep{2}

\endchorus



%%%%% STROFA
\beginverse		%Oppure \beginverse* se non si vuole il numero di fianco
%\memorize 		% <<< DECOMMENTA se si vuole utilizzarne la funzione
%\chordsoff		% <<< DECOMMENTA se vuoi una strofa senza accordi

^Senti l'^urlo di chi ^soffre, 
ogni ^sua lacrima ^Tu vedi, 
anche ^quelle ^non ver^sate. 
^Tu as^colti il mio si^lenzio 
^Tu conosci ogni ^stella 
Tu ^sei il Dio ^che ri^sorge 

\endverse


%%%%% RITORNELLO
\beginchorus
\textnote{\textbf{Rit.}}

\[A-]Io \[F]credo in \[C]Te \brk \echo{tu \[G]sei il figlio di \[A-]Dio} 
io \[F]credo in \[C]Te \brk \echo{ Tu \[G]sei risorto e \[A-]vivo} 
Io \[F]credo in \[C]Te \brk \echo{chi \[G]ha l'amore nei suoi \[A-]occhi ti \[F]riconos\[C]cerà.\[G]} \rep{2}

\endchorus





%%%%% BRIDGE
\beginverse*		%Oppure \beginverse* se non si vuole il numero di fianco
%\memorize 		% <<< DECOMMENTA se si vuole utilizzarne la funzione
%\chordsoff		% <<< DECOMMENTA se vuoi una strofa senza accordi
\vspace*{1.3\versesep}
\textnote{\textbf{Bridge}}

\[F]Grazie a Te io posso \[A-]credere in \[G]me         
\[F]Tu mi hai creato per \[G]amare. 

\endverse

\textnote{\textit{(si alza la tonalità)}}


\transpose{2}
%%%%% RITORNELLO
\beginchorus
\textnote{\textbf{Rit.}}

\[A-]Io \[F]credo in \[C]Te \brk \echo{tu \[G]sei il figlio di \[A-]Dio} 
io \[F]credo in \[C]Te \brk \echo{ Tu \[G]sei risorto e \[A-]vivo} 
Io \[F]credo in \[C]Te \brk \echo{chi \[G]ha l'amore nei suoi \[A-]occhi ti \[F]riconos\[C]cerà.\[G]} \rep{2}

\endchorus




%%%%%% EV. FINALE

\beginchorus %oppure \beginverse*
\vspace*{1.3\versesep}
\textnote{\textbf{Finale}} %<<< EV. INDICAZIONI

\[A-]Io \[F]credo in te… \normalfont\textit{(sospeso...)}

\endchorus  %oppure \endverse



\endsong
%------------------------------------------------------------
%			FINE CANZONE
%------------------------------------------------------------





% *  *  *  *  *  *  *  *  *  *  *  *  *  *  *  *  *  *






\end{songs}




%\ifcanzsingole
%	\relax
%\else
%	\iftitleindex
%		\ifxetex
%		\printindex[alfabetico]
%		\else
%		\printindex
%		\fi
%	\else
%	\fi
%	\ifauthorsindex
%	\printindex[autori]
%	\else
%	\fi
%	\iftematicindex
%	\printindex[tematico]
%	\else
%	\fi
%	\ifcover
%		\relax
%	\else
%		\colophon
%	\fi
%\fi
\end{document}
