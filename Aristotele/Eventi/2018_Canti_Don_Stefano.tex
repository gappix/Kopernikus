% SETTINGS
%——————————————————————————————————————————————————————
% STILE DOCUMENTO
%-------------------------------------------------------------------------------                                                                             
\documentclass[a4vert, palatino, titleindex, tematicindex, chorded, cover]{canzoniereonline}

%opzioni formato: singoli, standard (A4), a5vert, a5oriz, a6vert;
%opzioni accordi: lyric, chorded {quelli d Songs}
%opzioni font: palatino, libertine
%opzioni segno minore: "minorsign=quel che vuoi"
%opzioni indici: authorsindex, titleindex, tematicindex

%opzioi copertina: cover e nocover

\def\canzsongcolumsnumber{2} %# coolonne lungo cui disporre le canzoni




% PACCHETTI DA IMPORTARE
%-------------------------------------------------------------------------------                                                                             
\usepackage[T1]{fontenc}
\usepackage[utf8]{inputenc}
\usepackage[italian]{babel}
\usepackage{pdfpages}
\usepackage{hyperref}
\usepackage{wasysym}






% NUOVI COMANDI E VARIABILI GLOBALI
%--------------------------------------------------------------------------------
%Coomando per la suddivisione in capitoli
\renewcommand{\songchapter}{\chapter*}


%Counter globale per tenere traccia di una numerazione progressiva
%Si affianca a un altro counter già utilizzato nella classe CanzoniereOnLine "songnum" 
%che, tuttavia, si riazzera ognivolta viene definito un nuovo ambiente \beginsongs{}
\newcounter{GlobalSongCounter} 

%Ciascun capitolo contiene già tutta la logica di gestione della 
%numerazione progressiva, del DB locale da cui attingere le canzoni
%e la creazione/chiusura dell'ambiente in cui vengono importate 
%tutte le canzoni relative
\addtocounter{GlobalSongCounter}{1} %set starting song counter to 1 (0 otherwise)


%------------
\makeatletter
\newcommand*{\textoverline}[1]{$\overline{\hbox{#1}}\m@th$}
\makeatother
%-----------

%Starting Document
\begin{document}


\begin{songs}{}
\songcolumns{\canzsongcolumsnumber}
\setcounter{songnum}{\theGlobalSongCounter} %set songnum counter, otherwise would be reset


%  *  *  *  *  *  TEST SONG HERE  *  *  *  *  *  *  * ]
%set the default path inside current folder


% ----- LITURGIA ---------
\makeatletter \def\input@path{{Songs/Liturgia/}} \makeatother
%-------------------------------------------------------------
%			INIZIO	CANZONE
%-------------------------------------------------------------


%titolo: 	Popoli tutti
%autore: 	Zschech
%tonalita: 	La



%%%%%% TITOLO E IMPOSTAZONI
\beginsong{Popoli tutti acclamate}[by={Zschech}]	
\transpose{0} 						% <<< TRASPOSIZIONE #TONI (0 nullo)
\momenti{Ingresso; Comunione; Fine}							% <<< INSERISCI MOMENTI	
% momenti vanno separati da ; e vanno scelti tra:
% Ingresso; Atto penitenziale; Acclamazione al Vangelo; Dopo il Vangelo; Offertorio; Comunione; Ringraziamento; Fine; Santi; Pasqua; Avvento; Natale; Quaresima; Canti Mariani; Battesimo; Prima Comunione; Cresima; Matrimonio; Meditazione; Spezzare del pane;
\ifchorded
	%\textnote{Tonalità originale }	% <<< EV COMMENTI (tonalità originale/migliore)
\fi


%%%%%% INTRODUZIONE
\ifchorded
\vspace*{\versesep}
\textnote{Intro: \qquad \qquad  }%(\eighthnote 116) % <<  MODIFICA IL TEMPO
% Metronomo: \eighthnote (ottavo) \quarternote (quarto) \halfnote (due quarti)
\vspace*{-\versesep}
\beginverse*

\nolyrics

%---- Prima riga -----------------------------
\vspace*{-\versesep}
\[A] \[E] \[A]	 % \[*D] per indicare le pennate, \rep{2} le ripetizioni

%---- Ogni riga successiva -------------------
%\vspace*{-\versesep}
%\[G] \[C]  \[D]	

%---- Ev Indicazioni -------------------------			
%\textnote{\textit{(Oppure tutta la strofa)} }	

\endverse
\fi




%%%%% STROFA
\beginverse		%Oppure \beginverse* se non si vuole il numero di fianco
\memorize 		% <<< DECOMMENTA se si vuole utilizzarne la funzione
%\chordsoff		% <<< DECOMMENTA se vuoi una strofa senza accordi
\[A] Mio Dio, \[E] Signore, 
\[F#-]nulla è \[E]pari a \[D]te.
Ora e per \[A]sempre, \[D]voglio lo\[A]dare
\[F#-7]il Tuo grande \[G]amor \[D]per \[E4]noi. \[E] 
\[A]  Mia roccia \[E] Tu sei, 
\[F#-]pace e con\[E]forto mi \[D]dai.
Con tutto il \[A]cuore \[D]e le mie \[A]forze,
\[F#-7]sempre io Ti a\[G]do\[D]re\[E4]rò. \[E] 
\endverse


%%%%% RITORNELLO
\beginchorus
\textnote{\textbf{Rit.}}
\[A]Popoli \[F#-]tutti accla\[D]mate al Si\[E]gnore.
\[A]Gloria e po\[F#-]tenza can\[D7+]tiamo al \[E]re.
\[F#-]Mari e monti si \[D]prostrino a te,
al tuo \[E]nome, \[F#-]o Si\[E]gnore.
\[A]Canto di \[F#-]gioia per \[D]quello che \[E]fai,
per \[A]sempre Si\[F#-]gnore con \[D7+]te reste\[E]rò.
\[F#-]Non c'è promessa non \[D]c'è fedel\[E7]tà che in \[A]te.
\endchorus

%%%%% STROFA
\beginverse		%Oppure \beginverse* se non si vuole il numero di fianco
%\memorize 		% <<< DECOMMENTA se si vuole utilizzarne la funzione
%\chordsoff		% <<< DECOMMENTA se vuoi una strofa senza accordi
^ Tu luce ^ d'amore, 
^Spirito ^di Santi^tà
entra nei ^cuori di-^questi tuoi ^figli
chia^mati ad annun^cia^re il ^Re. ^
^ Tu forza ^ d'amore 
^nuova spe^ranza ci ^dai
in questo ^giorno a ^te consa^crato
^gioia immensa ^can^te^rò. ^
\endverse




\endsong
%------------------------------------------------------------
%			FINE CANZONE
%------------------------------------------------------------






% ----- GLORIA ---------
\makeatletter \def\input@path{{Songs/Gloria/}} \makeatother
%-------------------------------------------------------------
%			INIZIO	CANZONE
%-------------------------------------------------------------


%titolo: 	Gloria (esme)
%autore: 	Gen Verde
%tonalita: 	FA e RE 



%%%%%% TITOLO E IMPOSTAZONI
\beginsong{Gloria nell'alto dei cieli}[by={Gen Verde, Esme}] 	% <<< MODIFICA TITOLO E AUTORE
\transpose{0} 						% <<< TRASPOSIZIONE #TONI (0 nullo)
\momenti{}							% <<< INSERISCI MOMENTI	
% momenti vanno separati da ; e vanno scelti tra:
% Ingresso; Atto penitenziale; Acclamazione al Vangelo; Dopo il Vangelo; Offertorio; Comunione; Ringraziamento; Fine; Santi; Pasqua; Avvento; Natale; Quaresima; Canti Mariani; Battesimo; Prima Comunione; Cresima; Matrimonio; Meditazione;
\ifchorded
	%\textnote{Tonalità originale }	% <<< EV COMMENTI (tonalità originale/migliore)
\fi




%%%%%% INTRODUZIONE
\ifchorded
\vspace*{\versesep}
\textnote{Intro: \qquad \qquad  }%(\eighthnote 116) % << MODIFICA IL TEMPO
% Metronomo: \eighthnote (ottavo) \quarternote (quarto) \halfnote (due quarti)
\vspace*{-\versesep}
\beginverse*

\nolyrics

%---- Prima riga -----------------------------
\vspace*{-\versesep}
\[F]   \[B&]  \[C]  \[C]	 \rep{2} % \[*D] per indicare le pennate, \rep{2} le ripetizioni

%---- Ogni riga successiva -------------------
%\vspace*{-\versesep}
%\[G] \[C]  \[D]	

%---- Ev Indicazioni -------------------------			
%\textnote{\textit{(Oppure tutta la strofa)} }	

\endverse
\fi





%%%%% RITORNELLO
\beginchorus
%\textnote{\textbf{Rit.}}

\[F]Gloria, \[B&]gloria a \[D-]Dio. \[C]
Gloria, \[F]gloria nel\[B&]l'alto dei \[D-]cie\[C]li.
\[F]Pace in \[B&]terra agli \[D-]uomi\[C]ni
di \[F]buona \[B&]volon\[F]tà. \[B&] 
\[F]Gl\[B&]o\[F]ria!
\endchorus







%%%%% STROFA
\beginverse*		%Oppure \beginverse* se non si vuole il numero di fianco
\memorize 		% <<< DECOMMENTA se si vuole utilizzarne la funzione
%\chordsoff		& <<< DECOMMENTA se vuoi una strofa senza accordi
Noi \[B&]ti lo\[F]diamo, \[G-]ti benedi\[F]ciamo,
ti \[B&]ador\[F]iamo, glo\[E&]rifichiamo \[F]te,
\[B&]ti ren\[F]diamo \[G-7]grazie per la \[F]tua immensa
\[E&]glor\[C4]ia. \[C]

\endverse


%%%%% STROFA
\beginverse*
Si^gnore ^Dio, ^glor^ia!  ^Re del ci^elo, ^glor^ia!
^Dio ^Padre, ^Dio onnipo^tente, ^glor\[C]ia! \[G-] \[E&] \[C]
\endverse



%%%%% RITORNELLO
\beginchorus
%\textnote{\textbf{Rit.}}

\[F]Gloria, \[B&]gloria a \[D-]Dio. \[C]
Gloria, \[F]gloria nel\[B&]l'alto dei \[D-]cie\[C]li.
\[F]Pace in \[B&]terra agli \[D-]uomi\[C]ni
di \[F]buona \[B&]volon\[F]tà. \[B&] 
\[F]Gl\[B&]o\[F]ria!
\endchorus



%%%%% STROFA
\beginverse*
Si\[F]gnore, Figlio uni\[E&]genito, \[B&]Gesù Cri\[F]sto,
Si\[F]gnore, Agnello di \[E&]Dio, \[B&]Figlio del Pad\[F]re.
\[F]Tu che togli i pec\[E&]cati del mondo,
a\[B&]bbi pietà  di no\[F]i;
\[F]tu che togli i pec\[E&]cati del mondo,
a\[B&]ccogli la nostra su\[F]pplica;
\[F]tu che siedi alla \[E&]destra del Padre,
\[B&]abbi pietà  di n\[C4]oi. \[C]
\endverse




%%%%% RITORNELLO
\beginchorus
%\textnote{\textbf{Rit.}}

\[F]Gloria, \[B&]gloria a \[D-]Dio. \[C]
Gloria, \[F]gloria nel\[B&]l'alto dei \[D-]cie\[C]li.
\[F]Pace in \[B&]terra agli \[D-]uomi\[C]ni
di \[F]buona \[B&]volon\[F]tà. \[B&] 
\[F]Gl\[B&]o\[F]ria!
\endchorus


%%%%% STROFA
\beginverse*
Per^chè tu ^solo il ^Santo, il Si^gnore,
tu ^solo l'Al^tissimo, ^Cristo G^esù
^con lo ^Spirito ^Santo nella ^gloria
del ^Pad\[C]re. \[G-] \[E&] \[C]
\endverse



%%%%% RITORNELLO
\beginchorus
%\textnote{\textbf{Rit.}}

\[F]Gloria, \[B&]gloria a \[D-]Dio. \[C]
Gloria, \[F]gloria nel\[B&]l'alto dei \[D-]cie\[C]li.
\[F]Pace in \[B&]terra agli \[D-]uomi\[C]ni
di \[F]buona \[B&]volon\[F]tà. \[B&] 
\[F]Gl\[B&]o\[F]ria! \[F] \[*F]
\endchorus




\endsong
%------------------------------------------------------------
%			FINE CANZONE
%------------------------------------------------------------
%++++++++++++++++++++++++++++++++++++++++++++++++++++++++++++
%			CANZONE TRASPOSTA
%++++++++++++++++++++++++++++++++++++++++++++++++++++++++++++
\ifchorded
%decremento contatore per avere stesso numero
\addtocounter{songnum}{-1} 
\beginsong{Gloria nell'alto dei cieli}[by={Gen Verde, Esme}] 	% <<< COPIA TITOLO E AUTORE
\transpose{-3} 						% <<< TRASPOSIZIONE #TONI + - (0 nullo)
\ifchorded
	\textnote{Tonalità migliore per le chitarre}	% <<< EV COMMENTI (tonalità originale/migliore)
\fi




%%%%%% INTRODUZIONE
\ifchorded
\vspace*{\versesep}
\textnote{Intro: \qquad \qquad  }%(\eighthnote 116) % << MODIFICA IL TEMPO
% Metronomo: \eighthnote (ottavo) \quarternote (quarto) \halfnote (due quarti)
\vspace*{-\versesep}
\beginverse*

\nolyrics

%---- Prima riga -----------------------------
\vspace*{-\versesep}
\[F]   \[B&]  \[C]  \[C]	 \rep{2} % \[*D] per indicare le pennate, \rep{2} le ripetizioni

%---- Ogni riga successiva -------------------
%\vspace*{-\versesep}
%\[G] \[C]  \[D]	

%---- Ev Indicazioni -------------------------			
%\textnote{\textit{(Oppure tutta la strofa)} }	

\endverse
\fi





%%%%% RITORNELLO
\beginchorus
%\textnote{\textbf{Rit.}}

\[F]Gloria, \[B&]gloria a \[D-]Dio. \[C]
Gloria, \[F]gloria nel\[B&]l'alto dei \[D-]cie\[C]li.
\[F]Pace in \[B&]terra agli \[D-]uomi\[C]ni
di \[F]buona \[B&]volon\[F]tà. \[B&] 
\[F]Gl\[B&]o\[F]ria!
\endchorus







%%%%% STROFA
\beginverse*		%Oppure \beginverse* se non si vuole il numero di fianco
\memorize 		% <<< DECOMMENTA se si vuole utilizzarne la funzione
%\chordsoff		& <<< DECOMMENTA se vuoi una strofa senza accordi
Noi \[B&]ti lo\[F]diamo, \[G-]ti benedi\[F]ciamo,
ti \[B&]ador\[F]iamo, glo\[E&]rifichiamo \[F]te,
\[B&]ti ren\[F]diamo \[G-7]grazie per la \[F]tua immensa
\[E&]glor\[C4]ia. \[C]

\endverse


%%%%% STROFA
\beginverse*
Si^gnore ^Dio, ^glor^ia!  ^Re del ci^elo, ^glor^ia!
^Dio ^Padre, ^Dio onnipo^tente, ^glor\[C]ia! \[G-] \[E&] \[C]
\endverse



%%%%% RITORNELLO
\beginchorus
%\textnote{\textbf{Rit.}}

\[F]Gloria, \[B&]gloria a \[D-]Dio. \[C]
Gloria, \[F]gloria nel\[B&]l'alto dei \[D-]cie\[C]li.
\[F]Pace in \[B&]terra agli \[D-]uomi\[C]ni
di \[F]buona \[B&]volon\[F]tà. \[B&] 
\[F]Gl\[B&]o\[F]ria!
\endchorus



%%%%% STROFA
\beginverse*
Si\[F]gnore, Figlio uni\[E&]genito, \[B&]Gesù Cri\[F]sto,
Si\[F]gnore, Agnello di \[E&]Dio, \[B&]Figlio del Pad\[F]re.
\[F]Tu che togli i pec\[E&]cati del mondo,
a\[B&]bbi pietà  di no\[F]i;
\[F]tu che togli i pec\[E&]cati del mondo,
a\[B&]ccogli la nostra su\[F]pplica;
\[F]tu che siedi alla \[E&]destra del Padre,
\[B&]abbi pietà  di n\[C4]oi. \[C]
\endverse




%%%%% RITORNELLO
\beginchorus
%\textnote{\textbf{Rit.}}

\[F]Gloria, \[B&]gloria a \[D-]Dio. \[C]
Gloria, \[F]gloria nel\[B&]l'alto dei \[D-]cie\[C]li.
\[F]Pace in \[B&]terra agli \[D-]uomi\[C]ni
di \[F]buona \[B&]volon\[F]tà. \[B&] 
\[F]Gl\[B&]o\[F]ria!
\endchorus


%%%%% STROFA
\beginverse*
Per^chè tu ^solo il ^Santo, il Si^gnore,
tu ^solo l'Al^tissimo, ^Cristo G^esù
^con lo ^Spirito ^Santo nella ^gloria
del ^Pad\[C]re. \[G-] \[E&] \[C]
\endverse



%%%%% RITORNELLO
\beginchorus
%\textnote{\textbf{Rit.}}

\[F]Gloria, \[B&]gloria a \[D-]Dio. \[C]
Gloria, \[F]gloria nel\[B&]l'alto dei \[D-]cie\[C]li.
\[F]Pace in \[B&]terra agli \[D-]uomi\[C]ni
di \[F]buona \[B&]volon\[F]tà. \[B&] 
\[F]Gl\[B&]o\[F]ria! \[F] \[*F]
\endchorus




\endsong

\fi
%++++++++++++++++++++++++++++++++++++++++++++++++++++++++++++
%			FINE CANZONE TRASPOSTA
%++++++++++++++++++++++++++++++++++++++++++++++++++++++++++++




% ----- ALLELUIA ---------
\makeatletter \def\input@path{{Songs/Alleluia/}} \makeatother
%-------------------------------------------------------------
%			INIZIO	CANZONE
%-------------------------------------------------------------


%titolo: 	Alleluia, festa con Te
%autore: 	Fabio Avolio
%tonalita: 	Do 



%%%%%% TITOLO E IMPOSTAZONI
\beginsong{Alleluia, festa con Te}[by={Fabio Avolio}] 	% <<< MODIFICA TITOLO E AUTORE
\transpose{0} 						% <<< TRASPOSIZIONE #TONI (0 nullo)
\momenti{Acclamazione al Vangelo}							% <<< INSERISCI MOMENTI	
% momenti vanno separati da ; e vanno scelti tra:
% Ingresso; Atto penitenziale; Acclamazione al Vangelo; Dopo il Vangelo; Offertorio; Comunione; Ringraziamento; Fine; Santi; Pasqua; Avvento; Natale; Quaresima; Canti Mariani; Battesimo; Prima Comunione; Cresima; Matrimonio; Meditazione; Spezzare del pane;
\ifchorded
	%\textnote{Tonalità originale }	% <<< EV COMMENTI (tonalità originale/migliore)
\fi

%%%%%% INTRODUZIONE
\ifchorded
\vspace*{\versesep}
\textnote{Intro: \qquad \qquad  }%(\eighthnote 116) % <<  MODIFICA IL TEMPO
% Metronomo: \eighthnote (ottavo) \quarternote (quarto) \halfnote (due quarti)
\vspace*{-\versesep}
\beginverse*

\nolyrics

%---- Prima riga -----------------------------
\vspace*{-\versesep}
\[C] \[F] \[G] \[A-] \[F] \[C] \[D-] \[G]	 % \[*D] per indicare le pennate, \rep{2} le ripetizioni

%---- Ogni riga successiva -------------------
%\vspace*{-\versesep}
%\[G] \[C]  \[D]	

%---- Ev Indicazioni -------------------------			
%\textnote{\textit{(Oppure tutta la strofa)} }	

\endverse
\fi

%%%%% RITORNELLO
\beginchorus
\textnote{\textbf{Rit.}}

\[C]Allelu\[F]ia, \[G]allelu\[A-]ia, \[F]oggi è \[C]festa con \[D-]te, Ges\[G]ù.
\[C]Tu sei con \[F]noi, \[G]gioia ci \[A-]dai, \[F]allelu\[C]ia, alle\[G]lu\[C]ia.

\endchorus

%%%%% STROFA
\beginverse		%Oppure \beginverse* se non si vuole il numero di fianco
\memorize 		% <<< DECOMMENTA se si vuole utilizzarne la funzione
%\chordsoff		% <<< DECOMMENTA se vuoi una strofa senza accordi

\[C]Nella tua \[G]casa \[C]siamo ve\[F]nuti \[C]per incon\[D-]trar\[G]ti.
\[C]A te can\[G]tiamo \[A-]la nostra \[F]lode, \[C]gloria al tuo \[G]no\[C]me. \[G]

\endverse

%%%%% STROFA
\beginverse		%Oppure \beginverse* se non si vuole il numero di fianco
%\memorize 		% <<< DECOMMENTA se si vuole utilizzarne la funzione
\chordsoff		% <<< DECOMMENTA se vuoi una strofa senza accordi

^Il pane ^vivo ^che ci hai pro^messo ^dona la ^vi^ta.
^A te can^tiamo ^la nostra ^lode, ^gloria al tuo ^no^me. ^

\endverse

%%%%% STROFA
\beginverse		%Oppure \beginverse* se non si vuole il numero di fianco
%\memorize 		% <<< DECOMMENTA se si vuole utilizzarne la funzione
\chordsoff		% <<< DECOMMENTA se vuoi una strofa senza accordi

^Tu sei l'^mico ^che ci accom^pagna ^lungo il cam^mi^no.
^A te can^tiamo ^la nostra ^lode, ^gloria al tuo ^no^me. ^

\endverse

\endsong
%------------------------------------------------------------
%			FINE CANZONE
%------------------------------------------------------------


% ----- TAIZE ---------
\makeatletter \def\input@path{{Songs/Taize/}} \makeatother
%-------------------------------------------------------------
%			INIZIO	CANZONE
%-------------------------------------------------------------


%titolo: 	Bless the Lord my soul
%autore: 	Taizé
%tonalita: 	Si- 



\beginsong{Bless the lord my soul}[by={Taizé}]
\transpose{0} 						% <<< TRASPOSIZIONE #TONI (0 nullo)
\momenti{Dopo il Vangelo; Meditazione}				% <<< INSERISCI MOMENTI	
\ifchorded
	%\textnote{Tonalità originale }	% <<< EV COMMENTI (tonalità originale/migliore)
\fi
\preferflats

%%%%%% INTRODUZIONE
\ifchorded
\vspace*{\versesep}
\textnote{Intro: \qquad \qquad  }%(\eighthnote 116) % << MODIFICA IL TEMPO
% Metronomo: \eighthnote (ottavo) \quarternote (quarto) \halfnote (due quarti)
\vspace*{-\versesep}
\beginverse*

\nolyrics

%---- Prima riga -----------------------------
\vspace*{-\versesep}
\[B-] \[E]  \[B-]	 % \[*D] per indicare le pennate, \rep{2} le ripetizioni


%---- Ev Indicazioni -------------------------			
\textnote{\textit{(Oppure tutta la strofa)} }	

\endverse
\fi



%%%%% STROFA
\beginverse*
\[B-]Bless the \[E]Lord, my \[B-]soul, \brk and \[G]bless God's \[A]holy \[D]name. \[F#] 
\[B-]Bless the \[E]Lord, my \[B-]soul, \brk who \[G]leads \[A7]me into \[B-]life.
\endverse
\endsong




% ----- LITURGIA ---------
\makeatletter \def\input@path{{Songs/Liturgia/}} \makeatother
%-------------------------------------------------------------
%			INIZIO	CANZONE
%-------------------------------------------------------------


%titolo: 	Pace sia, pace a voi
%autore: 	Gen Verde, Gen Rosso
%tonalita: 	Mi 



%%%%%% TITOLO E IMPOSTAZONI
\beginsong{Pace sia, pace a voi}[by={Gen Verde, Gen Rosso}] 	% <<< MODIFICA TITOLO E AUTORE
\transpose{0} 						% <<< TRASPOSIZIONE #TONI (0 nullo)
\momenti{Offertorio}							% <<< INSERISCI MOMENTI	
% momenti vanno separati da ; e vanno scelti tra:
% Ingresso; Atto penitenziale; Acclamazione al Vangelo; Dopo il Vangelo; Offertorio; Comunione; Ringraziamento; Fine; Santi; Pasqua; Avvento; Natale; Quaresima; Canti Mariani; Battesimo; Prima Comunione; Cresima; Matrimonio; Meditazione;
\ifchorded
	%\textnote{Tonalità originale }	% <<< EV COMMENTI (tonalità originale/migliore)
\fi

%%%%%% INTRODUZIONE
\ifchorded
\vspace*{\versesep}
\musicnote{
\begin{minipage}{0.48\textwidth}
\textbf{Intro}
\hfill 
%( \eighthnote \, 80)   % <<  MODIFICA IL TEMPO
% Metronomo: \eighthnote (ottavo) \quarternote (quarto) \halfnote (due quarti)
\end{minipage}
} 	
\vspace*{-\versesep}
\beginverse*

\nolyrics

%---- Prima riga -----------------------------
\vspace*{-\versesep}
\[E] \[A] \[E] \[B] \[E] \[A] \[E]	 % \[*D] per indicare le pennate, \rep{2} le ripetizioni

%---- Ogni riga successiva -------------------
%\vspace*{-\versesep}
%\[G] \[C]  \[D]	

%---- Ev Indicazioni -------------------------			
%\textnote{\textit{(Oppure tutta la strofa)} }	

\endverse
\fi

%%%%% RITORNELLO
\beginchorus
\textnote{\textbf{Rit.}}

“Pace \[E]sia, pace a voi”: la tua \[A]pace sarà
sulla \[C#-]terra com'è nei \[B]cieli.
“Pace \[E]sia, pace a voi”: la tua \[A]pace sarà
gioia \[G]nei nostri \[D]occhi, nei \[A]cuo\[B]ri.
“Pace \[E]sia, pace a voi”: la tua \[A]pace sarà
luce \[C#-]limpida nei pen\[B]sieri.
“Pace \[E]sia, pace a voi”: la tua \[A]pace sarà
una \[E]casa per \[B]tutti. \[E]\[A]\[E]

\endchorus

%%%%% STROFA
\beginverse		%Oppure \beginverse* se non si vuole il numero di fianco
\memorize 		% <<< DECOMMENTA se si vuole utilizzarne la funzione
%\chordsoff		% <<< DECOMMENTA se vuoi una strofa senza accordi

“\[A]Pace a \[E]voi”: sia il tuo \[B]dono vi\[C#-]sibile.
“\[A]Pace a \[E]voi”: la tua e\[B]redi\[C#-]tà.
“\[A]Pace a \[E]voi”: come un \[B]canto all'u\[C#-]nisono
che \[D]sale dalle nostre cit\[B]tà.

\endverse

%%%%% STROFA
\beginverse		%Oppure \beginverse* se non si vuole il numero di fianco
%\memorize 		% <<< DECOMMENTA se si vuole utilizzarne la funzione
%\chordsoff		% <<< DECOMMENTA se vuoi una strofa senza accordi

“^Pace a ^voi”: sia un im^pronta nei ^secoli.
“^Pace a ^voi”: segno d'^uni^tà.
“^Pace a ^voi”: sia l'ab^braccio tra i ^popoli,
la ^tua promessa all'umani^tà.

\endverse

\endsong
%------------------------------------------------------------
%			FINE CANZONE
%------------------------------------------------------------




% ----- ALLELUIA ---------
\makeatletter \def\input@path{{Songs/Santo/}} \makeatother
%-------------------------------------------------------------
%			INIZIO	CANZONE
%-------------------------------------------------------------


%titolo: 	Santo Milan
%autore: 	Gen Verde
%tonalita: 	Sol 



%%%%%% TITOLO E IMPOSTAZONI
\beginsong{Santo Milan}[by={Gen Verde}] 	% <<< MODIFICA TITOLO E AUTORE
\transpose{0} 						% <<< TRASPOSIZIONE #TONI (0 nullo)
\momenti{}							% <<< INSERISCI MOMENTI	
% momenti vanno separati da ; e vanno scelti tra:
% Ingresso; Atto penitenziale; Acclamazione al Vangelo; Dopo il Vangelo; Offertorio; Comunione; Ringraziamento; Fine; Santi; Pasqua; Avvento; Natale; Quaresima; Canti Mariani; Battesimo; Prima Comunione; Cresima; Matrimonio; Meditazione;
\ifchorded
	\textnote{Tonalità originale }	% <<< EV COMMENTI (tonalità originale/migliore)
\fi


%%%%%% INTRODUZIONE
\ifchorded
\vspace*{\versesep}
\textnote{Intro: \qquad \qquad  }%(\eighthnote 116) % << MODIFICA IL TEMPO
% Metronomo: \eighthnote (ottavo) \quarternote (quarto) \halfnote (due quarti)
\vspace*{-\versesep}
\beginverse*

\nolyrics

%---- Prima riga -----------------------------
\vspace*{-\versesep}
\[A] \[E]  \[C#-] \[B]	 % \[*D] per indicare le pennate, \rep{2} le ripetizioni

%---- Ogni riga successiva -------------------
\vspace*{-\versesep}
\[F#-] \[E]  \[A]  \[B]	

%---- Ev Indicazioni -------------------------			
\textnote{\textit{(come le prime due righe)} }	

\endverse
\fi








%%%%% RITORNELLO
\beginchorus

\[A]San\[E]to, \[C#-]San\[B]to,
\[F#-]Santo il Si\[E]gnore, \[A]Dio dell'uni\[B]verso.
\[A]San\[E]to, \[C#-]San\[B]to.
I \[F#-]cieli e la \[E]terra 
sono \[A]pieni della tua \[(F#-)]glo\[E]ria.

\endchorus



%%%%% STROFA
\beginverse*		%Oppure \beginverse* se non si vuole il numero di fianco
%\memorize 		% <<< DECOMMENTA se si vuole utilizzarne la funzione
%\chordsoff		% <<< DECOMMENTA se vuoi una strofa senza accordi

O\[A]sanna nel\[B]l'alto dei \[A]cie\[B]li.
O\[F#-]sanna nell'alto dei \[A]cieli.

\endverse



%%%%% RITORNELLO
\beginchorus

\[A]San\[E]to, \[C#-]San\[B]to,
\[F#-]Santo il Si\[E]gnore, \[A]Dio dell'uni\[B]verso.
\[A]San\[E]to, \[C#-]San\[B]to.
I \[F#-]cieli e la \[E]terra 
sono \[A]pieni della tua \[(F#-)]glo\[E]ria.

\endchorus




%%%%% STROFA
\beginverse*		%Oppure \beginverse* se non si vuole il numero di fianco
%\memorize 		% <<< DECOMMENTA se si vuole utilizzarne la funzione
%\chordsoff		& <<< DECOMMENTA se vuoi una strofa senza accordi

\[B]Benedetto co\[A]lui che viene
nel \[E]nome del Sig\[B]nore.
O\[A]sanna nel\[B]l'alto dei \[A]cie\[B]li.
O\[F#-]sanna nell'alto dei \[A]cieli.

\endverse


%%%%% RITORNELLO
\beginchorus

\[A]San\[E]to, \[C#-]San\[B]to,
\[F#-]Sa-\[A]a-n\[E]to.  \[*E] 

\endchorus











\endsong
%------------------------------------------------------------
%			FINE CANZONE
%------------------------------------------------------------




%++++++++++++++++++++++++++++++++++++++++++++++++++++++++++++
%			CANZONE TRASPOSTA
%++++++++++++++++++++++++++++++++++++++++++++++++++++++++++++
\ifchorded
%decremento contatore per avere stesso numero
\addtocounter{songnum}{-1} 
\beginsong{Santo Milan}[by={Gen Verde}] 	% <<< COPIA TITOLO E AUTORE
\transpose{-2} 						% <<< TRASPOSIZIONE #TONI + - (0 nullo)
\ifchorded
	\textnote{Tonalità più facile per le chitarre}	% <<< EV COMMENTI (tonalità originale/migliore)
\fi


%%%%%% INTRODUZIONE
\ifchorded
\vspace*{\versesep}
\textnote{Intro: \qquad \qquad  }%(\eighthnote 116) % << MODIFICA IL TEMPO
% Metronomo: \eighthnote (ottavo) \quarternote (quarto) \halfnote (due quarti)
\vspace*{-\versesep}
\beginverse*

\nolyrics

%---- Prima riga -----------------------------
\vspace*{-\versesep}
\[A] \[E]  \[C#-] \[B]	 % \[*D] per indicare le pennate, \rep{2} le ripetizioni

%---- Ogni riga successiva -------------------
\vspace*{-\versesep}
\[F#-] \[E]  \[A]  \[B]	

%---- Ev Indicazioni -------------------------			
\textnote{\textit{(come le prime due righe)} }	

\endverse
\fi








%%%%% RITORNELLO
\beginchorus

\[A]San\[E]to, \[C#-]San\[B]to,
\[F#-]Santo il Si\[E]gnore, \[A]Dio dell'uni\[B]verso.
\[A]San\[E]to, \[C#-]San\[B]to.
I \[F#-]cieli e la \[E]terra 
sono \[A]pieni della tua \[(F#-)]glo\[E]ria.

\endchorus



%%%%% STROFA
\beginverse*		%Oppure \beginverse* se non si vuole il numero di fianco
%\memorize 		% <<< DECOMMENTA se si vuole utilizzarne la funzione
%\chordsoff		% <<< DECOMMENTA se vuoi una strofa senza accordi

O\[A]sanna nel\[B]l'alto dei \[A]cie\[B]li.
O\[F#-]sanna nell'alto dei \[A]cieli.

\endverse



%%%%% RITORNELLO
\beginchorus

\[A]San\[E]to, \[C#-]San\[B]to,
\[F#-]Santo il Si\[E]gnore, \[A]Dio dell'uni\[B]verso.
\[A]San\[E]to, \[C#-]San\[B]to.
I \[F#-]cieli e la \[E]terra 
sono \[A]pieni della tua \[(F#-)]glo\[E]ria.

\endchorus




%%%%% STROFA
\beginverse*		%Oppure \beginverse* se non si vuole il numero di fianco
%\memorize 		% <<< DECOMMENTA se si vuole utilizzarne la funzione
%\chordsoff		& <<< DECOMMENTA se vuoi una strofa senza accordi

\[B]Benedetto co\[A]lui che viene
nel \[E]nome del Sig\[B]nore.
O\[A]sanna nel\[B]l'alto dei \[A]cie\[B]li.
O\[F#-]sanna nell'alto dei \[A]cieli.

\endverse


%%%%% RITORNELLO
\beginchorus

\[A]San\[E]to, \[C#-]San\[B]to,
\[F#-]Sa-\[A]a-n\[E]to.  \[*E] 

\endchorus






\endsong

\fi
%++++++++++++++++++++++++++++++++++++++++++++++++++++++++++++
%			FINE CANZONE TRASPOSTA
%++++++++++++++++++++++++++++++++++++++++++++++++++++++++++++




% ----- LITURGIA ---------
\makeatletter \def\input@path{{Songs/Liturgia/}} \makeatother
%titolo{Come un prodigio}
%autore{Vezzani}
%album{}
%tonalita{Do}
%famiglia{Liturgica}
%gruppo{}
%momenti{}
%identificatore{come_un_prodigio}
%data_revisione{2017_04_26}
%trascrittore{Francesco Endrici}
\beginsong{Come un prodigio}[by={Vezzani}]
%\transpose{5}
\ifchorded
\beginverse*
\vspace*{-0.8\versesep}
{\nolyrics Intro: \[G]\[D/F#] \rep{3}}
\vspace*{-\versesep}
\endverse
\fi
\beginverse
\memorize
Signore \[E-7]tu mi scruti e co\[C9]nosci, 
sai quando \[G]seggo e quando \[D]mi alzo 
riesci a \[E-7]vedere i miei \[C9]pensieri,
sai quando \[G]io cammino \[D]e quando riposo.
\[F]Ti sono note tutte le mie \[C]vie,
la mia pa\[G]rola non è ancora sulla lingua
\[F]e tu Signore già la co\[C]nosci tut\[D]ta.
\endverse
\beginchorus
\[E-7]Sei tu che mi hai \[C9]creato
e mi hai tessuto nel \[G]seno di mia \[D]madre.
\[E-7]Tu mi hai fatto come un pro\[C9]digio,
le tue opere \[G]sono stupende
e per \[D]questo io ti lodo.
\endchorus
\ifchorded
\beginverse*
\vspace*{-\versesep}
{\nolyrics \[E-7]\[C9]\[G]\[D]}
\endverse
\fi
\beginverse
Di fronte e alle ^spalle tu mi circ^ondi 
poni su ^me la tua ^mano.
La tua sag^gezza stupenda per ^me
è troppo ^alta e io non la ^comprendo.
^Che sia al cielo o agli inferi ci ^sei,
non si può ^mai fuggire dalla tua presenza, 
^ovunque la tua mano guide^rà la ^mia.
\endverse
\beginchorus
\[E-7]Sei tu che mi hai \[C9]creato
e mi hai tessuto nel \[G]seno di mia \[D]madre.
\[E-7]Tu mi hai fatto come un pro\[C9]digio,
le tue opere \[G]sono stupende
e per \[D]questo io ti lodo.
\endchorus
\ifchorded
\beginverse*
\vspace*{-\versesep}
{\nolyrics \[E-7]\[C9]\[G]\[D]}
\endverse
\fi
\beginverse
E nel se^greto tu mi hai for^mato
mi hai intes^suto dalla ^terra.
Neanche le ^ossa ti eran na^scoste
ancora in^forme mi hanno ^visto i tuoi occhi.
^I miei giorni erano fis^sati
quando an^cora non ne esisteva uno
e ^tutto quanto era scritto ^nel tuo ^libro.
\endverse
\beginchorus
\[E-7]Sei tu che mi hai \[C9]creato
e mi hai tessuto nel \[G]seno di mia \[D]madre.
\[E-7]Tu mi hai fatto come un pro\[C9]digio,
le tue opere \[G]sono stupende
e per \[D]questo io ti lodo.
\endchorus
\ifchorded
\beginverse*
\vspace*{-\versesep}
{\nolyrics \[E-7]\[C9]\[G]\[D]}
\endverse
\fi
\beginchorus
\[E-7]Sei tu che mi hai \[C9]creato
e mi hai tessuto nel \[G]seno di mia \[D]madre.
\[E-7]Tu mi hai fatto come un pro\[C9]digio,
le tue opere \[G]sono stupende
e per \[D]questo, per questo ti \[C9]lodo
\endchorus
\endsong


%-------------------------------------------------------------
%			INIZIO	CANZONE
%-------------------------------------------------------------


%titolo: 	Benedetto il nome del Signore
%autore: 	Matt Redman
%tonalita: 	Sol > La 



%%%%%% TITOLO E IMPOSTAZONI
\beginsong{Benedetto il nome del Signore}[by={Blessed be your name — M. Redman, B. Redman}] 	% <<< MODIFICA TITOLO E AUTORE
\transpose{0} 						% <<< TRASPOSIZIONE #TONI (0 nullo)
%\preferflats  %SE VOGLIO FORZARE i bemolle come alterazioni
%\prefersharps %SE VOGLIO FORZARE i # come alterazioni
\momenti{Meditazione; Ringraziamento; Fine}							% <<< INSERISCI MOMENTI	
% momenti vanno separati da ; e vanno scelti tra:
% Ingresso; Atto penitenziale; Acclamazione al Vangelo; Dopo il Vangelo; Offertorio; Comunione; Ringraziamento; Fine; Santi; Pasqua; Avvento; Natale; Quaresima; Canti Mariani; Battesimo; Prima Comunione; Cresima; Matrimonio; Meditazione; Spezzare del pane;
\ifchorded
	%\textnote{Tonalità migliore }	% <<< EV COMMENTI (tonalità originale/migliore)
\fi


%%%%%% INTRODUZIONE
\ifchorded
\vspace*{\versesep}
\textnote{Intro: \qquad \qquad  }%(\eighthnote 116) % <<  MODIFICA IL TEMPO
% Metronomo: \eighthnote (ottavo) \quarternote (quarto) \halfnote (due quarti)
\vspace*{-\versesep}
\beginverse*

\nolyrics

%---- Prima riga -----------------------------
\vspace*{-\versesep}
\[G] \[D] \[E-] \[C] 	 % \[*D] per indicare le pennate, \rep{2} le ripetizioni

%---- Ogni riga successiva -------------------
\vspace*{-\versesep}
\[G] \[D] \[E-] \[C] \[C]	

%---- Ev Indicazioni -------------------------			
%\textnote{\textit{(Oppure tutta la strofa)} }	

\endverse
\fi




%%%%% STROFA
\beginverse		%Oppure \beginverse* se non si vuole il numero di fianco
\memorize 		% <<< DECOMMENTA se si vuole utilizzarne la funzione
%\chordsoff		% <<< DECOMMENTA se vuoi una strofa senza accordi

\[G] Lode al \[D]nome Tuo, \brk dalle \[E-]terre più \[C]floride.
Dove \[G]tutto sembra \[D]vivere,  
lode al \[C]nome Tuo.

\endverse
\beginverse*

^ Lode al ^nome Tuo,  \brk  dalle ^terre più ^aride.
Dove ^tutto sembra ^sterile, 
lode al ^nome Tuo.


\endverse
\beginverse*	

^ Tornerò a lo^darti sempre 
^ per ogni dono ^Tuo.
^ E quando scende^rà la notte 
\[E-] sempre io di\[C]rò:

\endverse





%%%%% RITORNELLO
\beginchorus
\textnote{\textbf{Rit.}}

Benedetto il \[G]nome del Si\[D]gnor,
lode al nome \[E-]Tu-u-\[C]o!
Benedetto il \[G]nome del Si\[D]gnor,
il glorioso \[E-]nome di Ge\[C]sù. 	\rep{2}

\endchorus





%%%%% STROFA
\beginverse		%Oppure \beginverse* se non si vuole il numero di fianco
%\memorize 		% <<< DECOMMENTA se si vuole utilizzarne la funzione
%\chordsoff		% <<< DECOMMENTA se vuoi una strofa senza accordi

^ Lode al ^nome Tuo, \brk quando il ^sole splende ^su di me.
Quando ^tutto è incan^tevole,
lode al ^nome Tuo.

\endverse
\beginverse*	

^ Lode al ^nome Tuo, \brk quando ^io sto da^vanti a Te.
Con il ^cuore triste e ^fragile,
lode al ^nome Tuo.

\endverse
\beginverse*	

^ Tornerò a lo^darti sempre 
^ per ogni dono ^Tuo.
^ E quando scende^rà la notte 
\[E-] sempre io di\[C]rò:

\endverse



%%%%% RITORNELLO
\beginchorus
\textnote{\textbf{Rit.}}

Benedetto il \[G]nome del Si\[D]gnor,
lode al nome \[E-]Tu-u-\[C]o!
Benedetto il \[G]nome del Si\[D]gnor,
il glorioso \[E-]nome di Ge\[C]sù. 	\rep{2}

\endchorus



%%%%% BRIDGE
\beginverse*		%Oppure \beginverse* se non si vuole il numero di fianco
%\memorize 		% <<< DECOMMENTA se si vuole utilizzarne la funzione
%\chordsoff		% <<< DECOMMENTA se vuoi una strofa senza accordi
\vspace*{1.3\versesep}
\textnote{Bridge} %<<< EV. INDICAZIONI


Tu ^doni e porti ^via.
Tu ^doni e porti ^via.
Ma ^sempre sceglie^rò
di \[E-]benedire \[C]te!  \rep{2}

\endverse
\beginverse*	

^ Tornerò a lo^darti sempre 
^ per ogni dono ^Tuo.
^ E quando scende^rà la notte 
\[E-] sempre io di\[C]rò:

\endverse


%%%%% RITORNELLO
\beginchorus
\textnote{\textbf{Rit.}}

Benedetto il \[G]nome del Si\[D]gnor,
lode al nome \[E-]Tu-u-\[C]o!
Benedetto il \[G]nome del Si\[D]gnor,
il glorioso \[E-]nome di Ge\[C]sù. 	\rep{2}

\endchorus



%%%%% BRIDGE
\beginverse*		%Oppure \beginverse* se non si vuole il numero di fianco
%\memorize 		% <<< DECOMMENTA se si vuole utilizzarne la funzione
%\chordsoff		% <<< DECOMMENTA se vuoi una strofa senza accordi

Tu ^doni e porti ^via.
Tu ^doni e porti ^via.
Ma ^sempre sceglie^rò
di \[E-]benedire \[C]te!  \rep{2}

\endverse





%%%%%% EV. INTERMEZZO
\beginverse*
\vspace*{1.3\versesep}
{
	\nolyrics
	\musicnote{Chiusura strumentale}
	
	\ifchorded

	%---- Prima riga -----------------------------
	\vspace*{-\versesep}
	\[G] \[D] \[E-] \[C] 

	%---- Ogni riga successiva -------------------
	\vspace*{-\versesep}
	\[G] \[D] \[C*]  \textit{(sospeso)}


	\fi
	%---- Ev Indicazioni -------------------------			
	%\musicnote{\textit{sospeso}} 
	 
}
\vspace*{\versesep}
\endverse


\endsong
%------------------------------------------------------------
%			FINE CANZONE
%------------------------------------------------------------



%titolo{Il canto dell'amore}
%autore{Russo}
%album{}
%tonalita{Mi}
%famiglia{Liturgica}
%gruppo{}
%momenti{}
%identificatore{il_canto_dell_amore}
%data_revisione{2011_12_31}
%trascrittore{Francesco Endrici}
\beginsong{Il canto dell'amore}[by={Russo}]
\beginverse*
Se dovrai at\[E]traversare il de\[C#-7]serto
non te\[A]mere io sarò con \[E]te
se do\[E]vrai camminare nel \[C#-7]fuoco
la sua \[A]fiamma non ti bruce\[E]rà
segui\[B]rai la mia \[A]luce nella \[E]notte \[E]
senti\[F#-]rai la mia \[B]forza nel cam\[C#-]mino \[C#-]
io s\[D]ono il tuo Dio, \[A] il Signo\[E]re. \[C#-7]\[A]\[E]
\endverse
\beginverse*
Sono ^io che ti ho fatto e plas^mato
ti ho chi^amato per no^me
io da ^sempre ti ho cono^sciuto
e ti ho ^dato il mio amo^re
perché ^tu sei pre^zioso ai miei ^occhi ^
vali ^più del più ^grande dei te^sori ^
io sa^rò con te ^ dovunque an^drai. ^^^
\endverse
\ifchorded
\beginverse*
\vspace*{-\versesep}
{\nolyrics \[B]\[A]\[E]\[E]\[D]\[A]\[B]\[B]}
\endverse
\fi
\beginverse*
Non pen^sare alle cose di ^ieri
cose ^nuove fioriscono ^già
apri^rò nel deserto sen^tieri
darò ^acqua nell'aridi^tà
perché ^tu sei pre^zioso ai miei ^occhi ^
vali ^più del più ^grande dei te^sori ^
io sa^rò con te ^ dovunque an^drai ^^^
perché \[B]tu sei pre\[A]zioso ai miei \[E]occhi \[E]
vali \[F#-]più del più \[B]grande dei te\[C#-]sori \[C#-]
io sa\[D]rò con te \[A] dovunque an\[E]drai. \[C#-7]\[A]\[E]
\endverse
\beginverse*
\[E] Io ti sa\[C#-7]rò accanto \[A]sarò con \[E]te 
\[E] per tutto il \[C#-7]tuo viaggio \[A]sarò con \[E]te. 
\[E] Io ti sa\[C#-7]rò accanto \[A]sarò con \[E]te 
\[E] per tutto il \[C#-7]tuo viaggio \[A]sarò con \[E]te. 
\endverse
\endsong


%titolo{Le tue meraviglie}
%autore{Casucci, Balduzzi}
%album{Verbum Panis}
%tonalita{La-}
%famiglia{Liturgica}
%gruppo{}
%momenti{Congedo;Natale}
%identificatore{le_tue_meraviglie}
%data_revisione{2011_12_31}
%trascrittore{Francesco Endrici - Manuel Toniato}
\beginsong{Le tue meraviglie}[by={Casucci, Balduzzi}]

\ifchorded
\beginverse*
\vspace*{-0.8\versesep}
{\nolyrics \[A-] \[E-] \[F] \[C] \[D-] \[A-] \[F] \[G] 
\[A-] \[E-] \[F] \[C] \[D-] \[A-] \[F] \[G] \[A-] }
\vspace*{-\versesep}
\endverse
\fi

\beginchorus
Ora \[F]lascia, o Si\[G]gnore, che io \[E-]vada in pa\[A-]ce,
perché ho \[D-]visto le tue \[C]mera\[B&]vi\[G]glie.
Il tuo \[F]popolo in \[G]festa per le \[E-]strade corre\[A-]rà
a por\[D-]tare le tue \[C]mera\[B&]vi\[G]glie!
\endchorus

\beginverse
\[A-]La tua pre\[E-]senza ha riem\[F]pito d'a\[C]more
\[A-]le nostre \[E-]vite, le \[F]nostre gior\[C]nate.
\[B&]In te una sola \[F]anima, \[G-]un solo cuore \[F]siamo noi:
\[B&]con te la luce ri\[F]splende, \brk \[G-]splende più chiara che \[C]mai!
\endverse

\beginverse
\chordsoff
La tua presenza ha inondato d'amore
le nostre vite, le nostre giornate,
fra la tua gente resterai, \brk per sempre vivo in mezzo a noi
fino ai confini del tempo: così ci accompagnerai.
\endverse

\beginchorus
Ora \[F]lascia, o Si\[G]gnore, che io \[E-]vada in pa\[A-]ce,
perché ho \[D-]visto le tue \[C]mera\[B&]vi\[G]glie.
Il tuo \[F]popolo in \[G]festa per le \[E-]strade corre\[A-]rà
a por\[D-]tare le tue \[C]mera\[B&]vi\[G]glie!
Ora \[F]lascia, o Si\[G]gnore, che io \[E-]vada in pa\[A-]ce,
perché ho \[F]visto le \[G]tue mera\[E-]vi\[A-]glie.
E il tuo \[F]popolo in \[G]festa per le \[E-]strade corre\[A-]rà
a por\[F]tare le \[G]tue mera\[F]vi\[C]glie!
\endchorus



%%%%%% FINALE
\ifchorded
\beginverse*
\vspace*{-0.5\versesep}
{
	\nolyrics
	\textbf{Finale:} \quad	
	\[A-] \[E-] \[F] \[C] \[D-] \[A-] \[F] \[G] 
	\[A-] \[E-] \[F] \[C] \[D-] \[A-] \[F] \[G] \[A-]
	 
}
\vspace*{\versesep}
\endverse
\fi





\endsong
%------------------------------------------------------------
%			FINE CANZONE
%------------------------------------------------------------
%-------------------------------------------------------------
%			INIZIO	CANZONE
%-------------------------------------------------------------


%titolo: 	Holy is the Lord
%autore: 	Chris Tomlin
%tonalita: 	Sol 



%%%%%% TITOLO E IMPOSTAZONI
\beginsong{Holy is the Lord}[by={C. Tomlin, L. Giglio}] 	% <<< MODIFICA TITOLO E AUTORE
\transpose{0} 						% <<< TRASPOSIZIONE #TONI (0 nullo)
\momenti{Natale}							% <<< INSERISCI MOMENTI
\ifchorded
	\textnote{Tonalità originale }
\fi


%%%%%% INTRODUZIONE
\ifchorded
\vspace*{\versesep}
\textnote{Intro: \qquad \qquad  }%(\eighthnote 116) % << MODIFICA IL TEMPO
\vspace*{-\versesep}
\beginverse*

\nolyrics

%---- Prima riga -----------------------------
\vspace*{-\versesep}
\[G] \[C]  \[D]	 \rep{2}

%---- Ogni riga successiva -------------------
%\vspace*{-\versesep}
%\[G] \[C]  \[D]	

%---- Ev Indicazioni -------------------------			
%\textnote{\textit{(Oppure tutta la strofa)} }	

\endverse
\fi



%%%%% STROFA
\beginverse		%Oppure \beginverse* se non si vuole il numero di fianco
%\memorize 		% <<< DECOMMENTA se si vuole utilizzarne la funzione
%\chordsoff		& <<< DECOMMENTA se vuoi una strofa senza accordi

We \[G]stand and \[C]lift up our \[D]hands,
for the \[E-]joy of the \[C]Lord is our str\[D]enght.
We \[G]bow down and \[C]worship Him \[D]now,
how \[E-]great how \[C]awesome is \[D]He.

\endverse




%%%%% RITORNELLO
\beginchorus
Holy is the \[G]Lord
\[C]God al\[D]mighty.
The \[E-]Earth is \[C]filled
with His \[D]glory. \rep{2} \ifchorded \quad \qquad \nolyrics \[G] \[C]  \[D]	\rep{2} \fi

\endchorus








%%%%%% FINALE

\beginchorus
\vspace*{1.3\versesep}
\textnote{Finale \textit{(rallentando)}}
The \[E-]Earth is \[C]filled
with His \[D]glory.
The \[E-]Earth is \[C]filled
with His \[D]glo-o-ory.
\endchorus



\endsong
%------------------------------------------------------------
%			FINE CANZONE
%------------------------------------------------------------














% *  *  *  *  *  *  *  *  *  *  *  *  *  *  *  *  *  *






\end{songs}




%\ifcanzsingole
%	\relax
%\else
%	\iftitleindex
%		\ifxetex
%		\printindex[alfabetico]
%		\else
%		\printindex
%		\fi
%	\else
%	\fi
%	\ifauthorsindex
%	\printindex[autori]
%	\else
%	\fi
%	\iftematicindex
%	\printindex[tematico]
%	\else
%	\fi
%	\ifcover
%		\relax
%	\else
%		\colophon
%	\fi
%\fi
\end{document}
