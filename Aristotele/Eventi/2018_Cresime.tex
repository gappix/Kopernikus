% SETTINGS
%——————————————————————————————————————————————————————
% STILE DOCUMENTO
%-------------------------------------------------------------------------------                                                                             
\documentclass[a5vert, palatino, titleindex, tematicindex, lyric, cover]{canzoniereonline}

%opzioni formato: singoli, standard (A4), a5vert, a5oriz, a6vert;
%opzioni accordi: lyric, chorded {quelli d Songs}
%opzioni font: palatino, libertine
%opzioni segno minore: "minorsign=quel che vuoi"
%opzioni indici: authorsindex, titleindex, tematicindex

%opzioi copertina: cover e nocover

\def\canzsongcolumsnumber{2} %# coolonne lungo cui disporre le canzoni




% PACCHETTI DA IMPORTARE
%-------------------------------------------------------------------------------                                                                             
\usepackage[T1]{fontenc}
\usepackage[utf8]{inputenc}
\usepackage[italian]{babel}
\usepackage{pdfpages}
\usepackage{hyperref}
\usepackage{wasysym}






% NUOVI COMANDI E VARIABILI GLOBALI
%--------------------------------------------------------------------------------
%Coomando per la suddivisione in capitoli
\renewcommand{\songchapter}{\chapter*}


%Counter globale per tenere traccia di una numerazione progressiva
%Si affianca a un altro counter già utilizzato nella classe CanzoniereOnLine "songnum" 
%che, tuttavia, si riazzera ognivolta viene definito un nuovo ambiente \beginsongs{}
\newcounter{GlobalSongCounter} 

%Ciascun capitolo contiene già tutta la logica di gestione della 
%numerazione progressiva, del DB locale da cui attingere le canzoni
%e la creazione/chiusura dell'ambiente in cui vengono importate 
%tutte le canzoni relative
\addtocounter{GlobalSongCounter}{1} %set starting song counter to 1 (0 otherwise)


%------------
\makeatletter
\newcommand*{\textoverline}[1]{$\overline{\hbox{#1}}\m@th$}
\makeatother
%-----------

%Starting Document
\begin{document}


\begin{songs}{}
\songcolumns{\canzsongcolumsnumber}
\setcounter{songnum}{\theGlobalSongCounter} %set songnum counter, otherwise would be reset


%  *  *  *  *  *  TEST SONG HERE  *  *  *  *  *  *  * ]
%set the default path inside current folder


% ----- LITURGIA ---------
\makeatletter \def\input@path{{Songs/Liturgia/}} \makeatother
%-------------------------------------------------------------
%			INIZIO	CANZONE
%-------------------------------------------------------------


%titolo: 	Un solo spirito
%autore: 	P. Sequeri
%tonalita: 	Mi 



%%%%%% TITOLO E IMPOSTAZONI
\beginsong{Un solo Spirito}[by={P. Sequeri}] 	% <<< MODIFICA TITOLO E AUTORE
\transpose{-2} 						% <<< TRASPOSIZIONE #TONI (0 nullo)

\momenti{Ingresso; Cresima; Battesimo; }							% <<< INSERISCI MOMENTI	
% momenti vanno separati da ; e vanno scelti tra:
% Ingresso; Atto penitenziale; Acclamazione al Vangelo; Dopo il Vangelo; Offertorio; Comunione; Ringraziamento; Fine; Santi; Pasqua; Avvento; Natale; Quaresima; Canti Mariani; Battesimo; Prima Comunione; Cresima; Matrimonio; Meditazione;
\ifchorded
	\textnote{$\triangle$ Tonalità alternativa }	% <<< EV COMMENTI (tonalità originale/migliore)
\fi


%%%%%% INTRODUZIONE
\ifchorded
\vspace*{\versesep}
\musicnote{
\begin{minipage}{0.48\textwidth}
\textbf{Intro}
\hfill 
%( \eighthnote \, 80)   % <<  MODIFICA IL TEMPO
% Metronomo: \eighthnote (ottavo) \quarternote (quarto) \halfnote (due quarti)
\end{minipage}
} 	
\vspace*{-\versesep}
\beginverse*

\nolyrics

%---- Prima riga -----------------------------
\vspace*{-\versesep}
\[E] \[A] \[F#-] \[B7]	 % \[*D] per indicare le pennate, \rep{2} le ripetizioni

%---- Ogni riga successiva -------------------
%\vspace*{-\versesep}
%\[G] \[C]  \[D]	

%---- Ev Indicazioni -------------------------			
%\textnote{\textit{(Oppure tutta la strofa)} }	

\endverse
\fi


%%%%% RITORNELLO
\beginchorus
\textnote{\textbf{Rit.}}

Un \[E]solo \[A]Spirito, un \[F#-]solo Bat\[B7]tesimo, 
un \[E]solo Si\[G#7]gnore: Ge\[C#]sù!
Nel \[F#-]segno dell'a\[B]more tu \[G#]sei con \[A]noi, 
nel \[F#-]nome tuo vi\[B]viamo fra\[G#]telli: \[C#7]
nel \[F#-]cuore la spe\[B]ranza che \[G#]tu ci \[A]dai, 
la \[F#-]fede che ci u\[B7]nisce can\[E]tiamo!
\endchorus


%%%%% STROFA
\beginverse		%Oppure \beginverse* se non si vuole il numero di fianco
%\memorize 		% <<< DECOMMENTA se si vuole utilizzarne la funzione
%\chordsoff		% <<< DECOMMENTA se vuoi una strofa senza accordi
\preferflats  %SE VOGLIO FORZARE i bemolle come alterazioni
Io \[E-]sono la vite e \[A-]voi siete i t\[E-]ralci miei:
il \[C]tralcio che in me non \[D]vive sfiori\[G]rà; \[B7] 
ma \[E-]se rimanete in \[A-]me il \[C]Padre mio vi da\[G]rà 
la \[F]forza di una \[B&dim/(A)]vita che non \[B7]muore mai. 

\endverse

%%%%% STROFA
\beginverse		%Oppure \beginverse* se non si vuole il numero di fianco
%\memorize 		% <<< DECOMMENTA se si vuole utilizzarne la funzione
\chordsoff		% <<< DECOMMENTA se vuoi una strofa senza accordi

Io sono la vera via e la verità:
amici vi chiamo sempre sto con voi;
chi annuncia al fratello suo la fede nel nome mio
davanti al Padre io lo riconoscerò. 


\endverse

%%%%% STROFA
\beginverse		%Oppure \beginverse* se non si vuole il numero di fianco
%\memorize 		% <<< DECOMMENTA se si vuole utilizzarne la funzione
\chordsoff		% <<< DECOMMENTA se vuoi una strofa senza accordi
 
Lo Spirito Santo in voi parlerà di me;
dovunque c'è un uomo al mondo sono io;
ognuno che crede in me fratello vostro sarà
nel segno del Battesimo rinascerà. 

\endverse





\endsong
%------------------------------------------------------------
%			FINE CANZONE
%------------------------------------------------------------

%++++++++++++++++++++++++++++++++++++++++++++++++++++++++++++
%			CANZONE TRASPOSTA
%++++++++++++++++++++++++++++++++++++++++++++++++++++++++++++
\ifchorded
%decremento contatore per avere stesso numero
\addtocounter{songnum}{-1} 
\beginsong{Un solo Spirito}[by={P. Sequeri}] 	% <<< COPIA TITOLO E AUTORE
\transpose{0} 						% <<< TRASPOSIZIONE #TONI + - (0 nullo)
%\preferflats  %SE VOGLIO FORZARE i bemolle come alterazioni
%\prefersharps %SE VOGLIO FORZARE i # come alterazioni
\ifchorded
	\textnote{$\bigstar$ Tonalità originale}	% <<< EV COMMENTI (tonalità originale/migliore)
\fi



%%%%%% INTRODUZIONE
\ifchorded
\vspace*{\versesep}
\musicnote{
\begin{minipage}{0.48\textwidth}
\textbf{Intro}
\hfill 
%( \eighthnote \, 80)   % <<  MODIFICA IL TEMPO
% Metronomo: \eighthnote (ottavo) \quarternote (quarto) \halfnote (due quarti)
\end{minipage}
} 	
\vspace*{-\versesep}
\beginverse*

\nolyrics

%---- Prima riga -----------------------------
\vspace*{-\versesep}
\[E] \[A] \[F#-] \[B7]	 % \[*D] per indicare le pennate, \rep{2} le ripetizioni

%---- Ogni riga successiva -------------------
%\vspace*{-\versesep}
%\[G] \[C]  \[D]	

%---- Ev Indicazioni -------------------------			
%\textnote{\textit{(Oppure tutta la strofa)} }	

\endverse
\fi


%%%%% RITORNELLO
\beginchorus
\textnote{\textbf{Rit.}}

Un \[E]solo \[A]Spirito, un \[F#-]solo Bat\[B7]tesimo, 
un \[E]solo Si\[G#7]gnore: Ge\[C#]sù!
Nel \[F#-]segno dell'a\[B]more tu \[G#]sei con \[A]noi, 
nel \[F#-]nome tuo vi\[B]viamo fra\[G#]telli: \[C#7]
nel \[F#-]cuore la spe\[B]ranza che \[G#]tu ci \[A]dai, 
la \[F#-]fede che ci u\[B7]nisce can\[E]tiamo!
\endchorus


%%%%% STROFA
\beginverse		%Oppure \beginverse* se non si vuole il numero di fianco
%\memorize 		% <<< DECOMMENTA se si vuole utilizzarne la funzione
%\chordsoff		% <<< DECOMMENTA se vuoi una strofa senza accordi

Io \[E-]sono la vite e \[A-]voi siete i t\[E-]ralci miei:
il \[C]tralcio che in me non \[D]vive sfiori\[G]rà; \[B7] 
ma \[E-]se rimanete in \[A-]me il \[C]Padre mio vi da\[G]rà 
la \[F]forza di una \[B&dim/(A)]vita che non \[B7]muore mai. 

\endverse

%%%%% STROFA
\beginverse		%Oppure \beginverse* se non si vuole il numero di fianco
%\memorize 		% <<< DECOMMENTA se si vuole utilizzarne la funzione
\chordsoff		% <<< DECOMMENTA se vuoi una strofa senza accordi

Io sono la vera via e la verità:
amici vi chiamo sempre sto con voi;
chi annuncia al fratello suo la fede nel nome mio
davanti al Padre io lo riconoscerò. 


\endverse

%%%%% STROFA
\beginverse		%Oppure \beginverse* se non si vuole il numero di fianco
%\memorize 		% <<< DECOMMENTA se si vuole utilizzarne la funzione
\chordsoff		% <<< DECOMMENTA se vuoi una strofa senza accordi
 
Lo Spirito Santo in voi parlerà di me;
dovunque c'è un uomo al mondo sono io;
ognuno che crede in me fratello vostro sarà
nel segno del Battesimo rinascerà. 

\endverse





\endsong
\fi
%++++++++++++++++++++++++++++++++++++++++++++++++++++++++++++
%			FINE CANZONE TRASPOSTA
%++++++++++++++++++++++++++++++++++++++++++++++++++++++++++++


% ----- GLORIA ---------
\makeatletter \def\input@path{{Songs/Gloria/}} \makeatother
%-------------------------------------------------------------
%			INIZIO	CANZONE
%-------------------------------------------------------------


%titolo: 	Gloria (esme)
%autore: 	Gen Verde
%tonalita: 	FA e RE 



%%%%%% TITOLO E IMPOSTAZONI
\beginsong{Gloria nell'alto dei cieli}[by={Gen Verde, Esme}] 	% <<< MODIFICA TITOLO E AUTORE
\transpose{0} 						% <<< TRASPOSIZIONE #TONI (0 nullo)
\momenti{}							% <<< INSERISCI MOMENTI	
% momenti vanno separati da ; e vanno scelti tra:
% Ingresso; Atto penitenziale; Acclamazione al Vangelo; Dopo il Vangelo; Offertorio; Comunione; Ringraziamento; Fine; Santi; Pasqua; Avvento; Natale; Quaresima; Canti Mariani; Battesimo; Prima Comunione; Cresima; Matrimonio; Meditazione;
\ifchorded
	%\textnote{Tonalità originale }	% <<< EV COMMENTI (tonalità originale/migliore)
\fi




%%%%%% INTRODUZIONE
\ifchorded
\vspace*{\versesep}
\textnote{Intro: \qquad \qquad  }%(\eighthnote 116) % << MODIFICA IL TEMPO
% Metronomo: \eighthnote (ottavo) \quarternote (quarto) \halfnote (due quarti)
\vspace*{-\versesep}
\beginverse*

\nolyrics

%---- Prima riga -----------------------------
\vspace*{-\versesep}
\[F]   \[B&]  \[C]  \[C]	 \rep{2} % \[*D] per indicare le pennate, \rep{2} le ripetizioni

%---- Ogni riga successiva -------------------
%\vspace*{-\versesep}
%\[G] \[C]  \[D]	

%---- Ev Indicazioni -------------------------			
%\textnote{\textit{(Oppure tutta la strofa)} }	

\endverse
\fi





%%%%% RITORNELLO
\beginchorus
%\textnote{\textbf{Rit.}}

\[F]Gloria, \[B&]gloria a \[D-]Dio. \[C]
Gloria, \[F]gloria nel\[B&]l'alto dei \[D-]cie\[C]li.
\[F]Pace in \[B&]terra agli \[D-]uomi\[C]ni
di \[F]buona \[B&]volon\[F]tà. \[B&] 
\[F]Gl\[B&]o\[F]ria!
\endchorus







%%%%% STROFA
\beginverse*		%Oppure \beginverse* se non si vuole il numero di fianco
\memorize 		% <<< DECOMMENTA se si vuole utilizzarne la funzione
%\chordsoff		& <<< DECOMMENTA se vuoi una strofa senza accordi
Noi \[B&]ti lo\[F]diamo, \[G-]ti benedi\[F]ciamo,
ti \[B&]ador\[F]iamo, glo\[E&]rifichiamo \[F]te,
\[B&]ti ren\[F]diamo \[G-7]grazie per la \[F]tua immensa
\[E&]glor\[C4]ia. \[C]

\endverse


%%%%% STROFA
\beginverse*
Si^gnore ^Dio, ^glor^ia!  ^Re del ci^elo, ^glor^ia!
^Dio ^Padre, ^Dio onnipo^tente, ^glor\[C]ia! \[G-] \[E&] \[C]
\endverse



%%%%% RITORNELLO
\beginchorus
%\textnote{\textbf{Rit.}}

\[F]Gloria, \[B&]gloria a \[D-]Dio. \[C]
Gloria, \[F]gloria nel\[B&]l'alto dei \[D-]cie\[C]li.
\[F]Pace in \[B&]terra agli \[D-]uomi\[C]ni
di \[F]buona \[B&]volon\[F]tà. \[B&] 
\[F]Gl\[B&]o\[F]ria!
\endchorus



%%%%% STROFA
\beginverse*
Si\[F]gnore, Figlio uni\[E&]genito, \[B&]Gesù Cri\[F]sto,
Si\[F]gnore, Agnello di \[E&]Dio, \[B&]Figlio del Pad\[F]re.
\[F]Tu che togli i pec\[E&]cati del mondo,
a\[B&]bbi pietà  di no\[F]i;
\[F]tu che togli i pec\[E&]cati del mondo,
a\[B&]ccogli la nostra su\[F]pplica;
\[F]tu che siedi alla \[E&]destra del Padre,
\[B&]abbi pietà  di n\[C4]oi. \[C]
\endverse




%%%%% RITORNELLO
\beginchorus
%\textnote{\textbf{Rit.}}

\[F]Gloria, \[B&]gloria a \[D-]Dio. \[C]
Gloria, \[F]gloria nel\[B&]l'alto dei \[D-]cie\[C]li.
\[F]Pace in \[B&]terra agli \[D-]uomi\[C]ni
di \[F]buona \[B&]volon\[F]tà. \[B&] 
\[F]Gl\[B&]o\[F]ria!
\endchorus


%%%%% STROFA
\beginverse*
Per^chè tu ^solo il ^Santo, il Si^gnore,
tu ^solo l'Al^tissimo, ^Cristo G^esù
^con lo ^Spirito ^Santo nella ^gloria
del ^Pad\[C]re. \[G-] \[E&] \[C]
\endverse



%%%%% RITORNELLO
\beginchorus
%\textnote{\textbf{Rit.}}

\[F]Gloria, \[B&]gloria a \[D-]Dio. \[C]
Gloria, \[F]gloria nel\[B&]l'alto dei \[D-]cie\[C]li.
\[F]Pace in \[B&]terra agli \[D-]uomi\[C]ni
di \[F]buona \[B&]volon\[F]tà. \[B&] 
\[F]Gl\[B&]o\[F]ria! \[F] \[*F]
\endchorus




\endsong
%------------------------------------------------------------
%			FINE CANZONE
%------------------------------------------------------------
%++++++++++++++++++++++++++++++++++++++++++++++++++++++++++++
%			CANZONE TRASPOSTA
%++++++++++++++++++++++++++++++++++++++++++++++++++++++++++++
\ifchorded
%decremento contatore per avere stesso numero
\addtocounter{songnum}{-1} 
\beginsong{Gloria nell'alto dei cieli}[by={Gen Verde, Esme}] 	% <<< COPIA TITOLO E AUTORE
\transpose{-3} 						% <<< TRASPOSIZIONE #TONI + - (0 nullo)
\ifchorded
	\textnote{Tonalità migliore per le chitarre}	% <<< EV COMMENTI (tonalità originale/migliore)
\fi




%%%%%% INTRODUZIONE
\ifchorded
\vspace*{\versesep}
\textnote{Intro: \qquad \qquad  }%(\eighthnote 116) % << MODIFICA IL TEMPO
% Metronomo: \eighthnote (ottavo) \quarternote (quarto) \halfnote (due quarti)
\vspace*{-\versesep}
\beginverse*

\nolyrics

%---- Prima riga -----------------------------
\vspace*{-\versesep}
\[F]   \[B&]  \[C]  \[C]	 \rep{2} % \[*D] per indicare le pennate, \rep{2} le ripetizioni

%---- Ogni riga successiva -------------------
%\vspace*{-\versesep}
%\[G] \[C]  \[D]	

%---- Ev Indicazioni -------------------------			
%\textnote{\textit{(Oppure tutta la strofa)} }	

\endverse
\fi





%%%%% RITORNELLO
\beginchorus
%\textnote{\textbf{Rit.}}

\[F]Gloria, \[B&]gloria a \[D-]Dio. \[C]
Gloria, \[F]gloria nel\[B&]l'alto dei \[D-]cie\[C]li.
\[F]Pace in \[B&]terra agli \[D-]uomi\[C]ni
di \[F]buona \[B&]volon\[F]tà. \[B&] 
\[F]Gl\[B&]o\[F]ria!
\endchorus







%%%%% STROFA
\beginverse*		%Oppure \beginverse* se non si vuole il numero di fianco
\memorize 		% <<< DECOMMENTA se si vuole utilizzarne la funzione
%\chordsoff		& <<< DECOMMENTA se vuoi una strofa senza accordi
Noi \[B&]ti lo\[F]diamo, \[G-]ti benedi\[F]ciamo,
ti \[B&]ador\[F]iamo, glo\[E&]rifichiamo \[F]te,
\[B&]ti ren\[F]diamo \[G-7]grazie per la \[F]tua immensa
\[E&]glor\[C4]ia. \[C]

\endverse


%%%%% STROFA
\beginverse*
Si^gnore ^Dio, ^glor^ia!  ^Re del ci^elo, ^glor^ia!
^Dio ^Padre, ^Dio onnipo^tente, ^glor\[C]ia! \[G-] \[E&] \[C]
\endverse



%%%%% RITORNELLO
\beginchorus
%\textnote{\textbf{Rit.}}

\[F]Gloria, \[B&]gloria a \[D-]Dio. \[C]
Gloria, \[F]gloria nel\[B&]l'alto dei \[D-]cie\[C]li.
\[F]Pace in \[B&]terra agli \[D-]uomi\[C]ni
di \[F]buona \[B&]volon\[F]tà. \[B&] 
\[F]Gl\[B&]o\[F]ria!
\endchorus



%%%%% STROFA
\beginverse*
Si\[F]gnore, Figlio uni\[E&]genito, \[B&]Gesù Cri\[F]sto,
Si\[F]gnore, Agnello di \[E&]Dio, \[B&]Figlio del Pad\[F]re.
\[F]Tu che togli i pec\[E&]cati del mondo,
a\[B&]bbi pietà  di no\[F]i;
\[F]tu che togli i pec\[E&]cati del mondo,
a\[B&]ccogli la nostra su\[F]pplica;
\[F]tu che siedi alla \[E&]destra del Padre,
\[B&]abbi pietà  di n\[C4]oi. \[C]
\endverse




%%%%% RITORNELLO
\beginchorus
%\textnote{\textbf{Rit.}}

\[F]Gloria, \[B&]gloria a \[D-]Dio. \[C]
Gloria, \[F]gloria nel\[B&]l'alto dei \[D-]cie\[C]li.
\[F]Pace in \[B&]terra agli \[D-]uomi\[C]ni
di \[F]buona \[B&]volon\[F]tà. \[B&] 
\[F]Gl\[B&]o\[F]ria!
\endchorus


%%%%% STROFA
\beginverse*
Per^chè tu ^solo il ^Santo, il Si^gnore,
tu ^solo l'Al^tissimo, ^Cristo G^esù
^con lo ^Spirito ^Santo nella ^gloria
del ^Pad\[C]re. \[G-] \[E&] \[C]
\endverse



%%%%% RITORNELLO
\beginchorus
%\textnote{\textbf{Rit.}}

\[F]Gloria, \[B&]gloria a \[D-]Dio. \[C]
Gloria, \[F]gloria nel\[B&]l'alto dei \[D-]cie\[C]li.
\[F]Pace in \[B&]terra agli \[D-]uomi\[C]ni
di \[F]buona \[B&]volon\[F]tà. \[B&] 
\[F]Gl\[B&]o\[F]ria! \[F] \[*F]
\endchorus




\endsong

\fi
%++++++++++++++++++++++++++++++++++++++++++++++++++++++++++++
%			FINE CANZONE TRASPOSTA
%++++++++++++++++++++++++++++++++++++++++++++++++++++++++++++




% ----- LITURGIA ---------
\makeatletter \def\input@path{{Songs/Liturgia/}} \makeatother
%-------------------------------------------------------------
%			INIZIO	CANZONE
%-------------------------------------------------------------


%titolo: 	Del tuo Spirito, Signore
%autore: 	Gen Verde, Gen Rosso
%tonalita: 	Re



%%%%%% TITOLO E IMPOSTAZONI
\beginsong{Del tuo Spirito, Signore}[by={Gen\ Verde, Gen\ Rosso}]	% <<< MODIFICA TITOLO E AUTORE
\transpose{0} 						% <<< TRASPOSIZIONE #TONI (0 nullo)
\momenti{Cresima; Salmi}							% <<< INSERISCI MOMENTI	
% momenti vanno separati da ; e vanno scelti tra:
% Ingresso; Atto penitenziale; Salmi; Acclamazione al Vangelo; Dopo il Vangelo; Offertorio; Comunione; Ringraziamento; Fine; Santi; Pasqua; Avvento; Natale; Quaresima; Canti Mariani; Battesimo; Prima Comunione; Cresima; Matrimonio; Meditazione; Spezzare del pane;
\ifchorded
	%\textnote{Tonalità migliore }	% <<< EV COMMENTI (tonalità originale/migliore)
\fi


%%%%%% INTRODUZIONE
\ifchorded
\vspace*{\versesep}
\musicnote{
\begin{minipage}{0.48\textwidth}
\textbf{Intro}
\hfill 
%( \eighthnote \, 80)   % <<  MODIFICA IL TEMPO
% Metronomo: \eighthnote (ottavo) \quarternote (quarto) \halfnote (due quarti)
\end{minipage}
} 	
\vspace*{-\versesep}
\beginverse*


\nolyrics

%---- Prima riga -----------------------------
\vspace*{-\versesep}
\[A*] \[B-] \[F#-*]  \[G]	 % \[*D] per indicare le pennate, \rep{2} le ripetizioni

%---- Ogni riga successiva -------------------
\vspace*{-\versesep}
\[D] \[E-*]  \[D]	

%---- Ev Indicazioni -------------------------			
\textnote{\textit{[come la seconda parte del ritornello]} }	

\endverse
\fi






\beginchorus
|\[D*] Del tuo |\[G]Spiri\[D*]to, Si|\[G*]gno\[D]re,
\[A*]è |\[B-]piena la \[F#-*]ter|\[G]ra, è |\[D]piena la \[E-*]ter|\[D]ra. 
\endchorus



\musicnote{\textit{[dolce, arpeggiato]}}

\beginverse
\memorize
|\[C]Benedici il Si|\[B&]gnore, \mbar{2}{4}\[D-]anima \mbar{4}{4}\[A-*]mi\[B&]a,
Si\[C*]gnore, \mbar{3}{4}\[F*]Dio,\[C] tu sei |\[G*]gran\[C]de!
|\[C]Sono immense, splen|\[B&]denti
\mbar{2}{4}\[D-]tutte le tue \mbar{3}{4}\[B&*]ope\[F]re e |\[G-]tutte le crea\[A*]tu\mbar{4}{4}\[D]re.
\endverse



\beginverse
%\chordsoff
^Se tu togli il tuo ^soffio ^muore ogni ^co^sa
e ^si dis^sol^ve nella ^ter^ra.
^Il tuo spirito ^scende:
^tutto si ri^cre^a e ^tutto si rin^no^va.
\endverse


\beginverse
%\chordsoff
^La tua gloria, Si^gnore, ^resti per ^sem^pre.
Gio^isci, ^Di^o, del cre^a^to.
^Questo semplice ^canto
^salga a te Si^gno^re, sei ^tu la nostra ^gio^ia.
\endverse


\endsong






% ----- ALLELUIA ---------
\makeatletter \def\input@path{{Songs/Alleluia/}} \makeatother
%-------------------------------------------------------------
%			INIZIO	CANZONE
%-------------------------------------------------------------


%titolo: 	Alleluia e poi
%autore: 	Luca Diliberto, Giuliana Monti
%tonalita: 	Do 


%%%%%% TITOLO E IMPOSTAZONI
\beginsong{Alleluia E poi}[by={L. Diliberto, G. Monti}]
\transpose{0} 						% <<< TRASPOSIZIONE #TONI (0 nullo)
\momenti{Acclamazione al Vangelo}							% <<< INSERISCI MOMENTI	
% momenti vanno separati da ; e vanno scelti tra:
% Ingresso; Atto penitenziale; Acclamazione al Vangelo; Dopo il Vangelo; Offertorio; Comunione; Ringraziamento; Fine; Santi; Pasqua; Avvento; Natale; Quaresima; Canti Mariani; Battesimo; Prima Comunione; Cresima; Matrimonio; Meditazione;
\ifchorded
	%\textnote{Tonalità originale }	% <<< EV COMMENTI (tonalità originale/migliore)
\fi




%%%%%% INTRODUZIONE
\ifchorded
\vspace*{\versesep}
\musicnote{
\begin{minipage}{0.48\textwidth}
\textbf{Intro:}
\hfill 
%( \eighthnote \, 80)   % <<  MODIFICA IL TEMPO
% Metronomo: \eighthnote (ottavo) \quarternote (quarto) \halfnote (due quarti)
\end{minipage}
} 	
\vspace*{-\versesep}
\beginverse*

\nolyrics

%---- Prima riga -----------------------------
\vspace*{-\versesep}
\[C] \[G]  \[C]	 % \[*D] per indicare le pennate, \rep{2} le ripetizioni

%---- Ogni riga successiva -------------------
%\vspace*{-\versesep}
%\[G] \[C]  \[D]	

%---- Ev Indicazioni -------------------------			
%\textnote{\textit{(Oppure tutta la strofa)} }	

\endverse
\fi



%%%%% STROFA
\beginverse		%Oppure \beginverse* se non si vuole il numero di fianco
\memorize 		% <<< DECOMMENTA se si vuole utilizzarne la funzione
%\chordsoff		& <<< DECOMMENTA se vuoi una strofa senza accordi
\[C]Chiama, ed \[G]io ver\[A-]rò da \[E-]Te:
\[F]Figlio, nel si\[C]lenzio, mi \[D]accoglie\[G]rai.
\[C]Voce e \[G]{poi\dots} la \[A-]liber\[E-]tà,
\[F]nella Tua Pa\[C]rola cam\[D7]mine\[G]rò.
\endverse



%%%%% RITORNELLO
\beginchorus
\textnote{\textbf{Rit.}}
\[C]Alleluia, \[G]alleluia, \[A-]allelu\[E-]ia,
\[F]alleluia, \[C]alle\[D7]lu\[G]ia,
\[C]Alleluia, \[G]alleluia, \[A-]allelu\[E-]ia,
\[F]alleluia, \[C]alle\[G]lu\[C]ia.

\endchorus




%%%%% STROFA
\beginverse		%Oppure \beginverse* se non si vuole il numero di fianco
%\memorize 		% <<< DECOMMENTA se si vuole utilizzarne la funzione
%\chordsoff		& <<< DECOMMENTA se vuoi una strofa senza accordi

^Danza, ed ^io ver^rò con ^Te:
^Figlio, la Tua ^strada com^prende^rò.
^Luce, e ^poi, nel ^tempo ^tuo,
^oltre il desi^derio ri^pose^rò. 
\endverse



\endsong
%------------------------------------------------------------
%			FINE CANZONE
%------------------------------------------------------------



% ----- LITURGIA ---------
\makeatletter \def\input@path{{Songs/Liturgia/}} \makeatother
%titolo{Ogni mia parola}
%autore{Gen Verde}
%album{È bello lodarti}
%tonalita{Do}
%famiglia{Liturgica}
%gruppo{}
%momenti{Acclamazione al Vangelo}
%identificatore{ogni_mia_parola}
%data_revisione{2011_12_31}
%trascrittore{Francesco Endrici}
\beginsong{Ogni mia parola}[by={Gen\ Verde}]
\beginverse*
\[C]Come la \[G]pioggia e la \[C]ne\[G]ve
\[C]scendono \[F]giù dal \[G]cielo
e \[A-]non vi ri\[G]tornano \[F]senza irri\[G]gare
e \[F]far germo\[G]gliare la \[F]ter\[G]ra,
\[C]così ogni mia Pa\[G]rola non ri\[C]tornerà a \[G]me
\[C]senza operare \[F]quanto de\[G]sidero,
\[A-]senza aver compiuto \[E-]ciò per cui l'a\[F]vevo man\[C]data.
\[F]Ogni mia Pa\[G]rola, \[F]ogni mia Pa\[G]rola \[C]
\endverse
\endsong


%-------------------------------------------------------------
%			INIZIO	CANZONE
%-------------------------------------------------------------


%titolo: 	Luce di verità
%autore: 	Becchimanzi, Scordari, Giordano
%tonalita: 	Mi



%%%%%% TITOLO E IMPOSTAZONI
\beginsong{Luce di verità}[by={G. Becchimanzi, G. Scordari, C. Giordano}] 	% <<< MODIFICA TITOLO E AUTORE
\transpose{0} 						% <<< TRASPOSIZIONE #TONI (0 nullo)
\momenti{Ingresso; Comunione; Dopo il Vangelo; Cresima}							% <<< INSERISCI MOMENTI	
% momenti vanno separati da ; e vanno scelti tra:
% Ingresso; Atto penitenziale; Acclamazione al Vangelo; Dopo il Vangelo; Offertorio; Comunione; Ringraziamento; Fine; Santi; Pasqua; Avvento; Natale; Quaresima; Canti Mariani; Battesimo; Prima Comunione; Cresima; Matrimonio; Meditazione; Spezzare del pane;
\ifchorded
	%\textnote{Tonalità originale }	% <<< EV COMMENTI (tonalità originale/migliore)
\fi

%%%%%% INTRODUZIONE
\ifchorded
\vspace*{\versesep}
\musicnote{
\begin{minipage}{0.48\textwidth}
\textbf{Intro}
\hfill 
%( \eighthnote \, 80)   % <<  MODIFICA IL TEMPO
% Metronomo: \eighthnote (ottavo) \quarternote (quarto) \halfnote (due quarti)
\end{minipage}
} 	
\vspace*{-\versesep}
\beginverse*

\nolyrics

%---- Prima riga -----------------------------
\vspace*{-\versesep}
\[E] \[A] 	\[E] \[G#-]  % \[*D] per indicare le pennate, \rep{2} le ripetizioni

%---- Ogni riga successiva -------------------
\vspace*{-\versesep}
\[A] \[E] \[(C#-7*)] \[F#7] \[A] \[B]

%---- Ev Indicazioni -------------------------			
%\textnote{\textit{(Oppure tutta la strofa)} }	

\endverse
\fi


%%%%% RITORNELLO
\beginchorus
\textnote{\textbf{Rit.}}
\[E]Luce di veri\[A]tà, \[E]fiamma di cari\[G#-]tà,
\[A]vincolo di uni\[E]tà, \[(C#-7*)]Spirito \[F#7]Santo A\[A]mo\[B]re.
\[E]Dona la liber\[A]tà, \[E]dona la santi\[G#-]tà,
\[A]fa' dell'umani\[E]tà \[(C#-7*)]il tuo \[D]canto di \[A]lo\[B]de.
\endchorus

\musicnote{\textit{(dolce, leggero)}}

%%%%% STROFA
\beginverse		%Oppure \beginverse* se non si vuole il numero di fianco
\memorize 		% <<< DECOMMENTA se si vuole utilizzarne la funzione
%\chordsoff		% <<< DECOMMENTA se vuoi una strofa senza accordi
\[C#-] Ci poni come \[B]luce sopra un \[E]mon\[A]te:
\[F#-] in noi l'umani\[E]tà vedrà il tuo \[B4]vol\[B]to. 
\[A] Ti testimonie\[B]remo fra le \[E]gen\[A]ti:
\[F#-] in noi l'umani\[E]tà vedrà il tuo \[B4]volto,  
\[B]\textbf{Spirito, vieni!}
\endverse

%%%%% STROFA
\beginverse		%Oppure \beginverse* se non si vuole il numero di fianco
%\memorize 		% <<< DECOMMENTA se si vuole utilizzarne la funzione
%\chordsoff		% <<< DECOMMENTA se vuoi una strofa senza accordi
^ Cammini accanto a ^noi lungo la ^stra^da,
^ si realizzi in ^noi la tua missi^one.^
^ Attingeremo ^forza dal tuo ^cuor^e,
^ si realizzi in ^noi la tua missi^one, 
^\textbf{Spirito, vieni!}
\endverse

%%%%% STROFA
\beginverse		%Oppure \beginverse* se non si vuole il numero di fianco
%\memorize 		% <<< DECOMMENTA se si vuole utilizzarne la funzione
%\chordsoff		% <<< DECOMMENTA se vuoi una strofa senza accordi
^ Come sigillo ^posto sul tuo ^cuo^re,
^ ci custodisci, ^Dio, nel tuo a^more.^
^ Hai dato la tua ^vita per sal^var^ci,
^ ci custodisci, ^Dio, nel tuo a^more, 
^\textbf{Spirito, vieni!}
\endverse

%%%%% STROFA
\beginverse		%Oppure \beginverse* se non si vuole il numero di fianco
%\memorize 		% <<< DECOMMENTA se si vuole utilizzarne la funzione
%\chordsoff		% <<< DECOMMENTA se vuoi una strofa senza accordi
^ Dissiperai le ^tenebre del ^ma^le,
^ esulterà in ^te la tua crea^zione.^
^ Vivremo al tuo cos^petto in e^te^rno,
^ esulterà in ^te la tua crea^zione, 
^\textbf{Spirito, vieni!}
\endverse

%%%%% STROFA
\beginverse		%Oppure \beginverse* se non si vuole il numero di fianco
%\memorize 		% <<< DECOMMENTA se si vuole utilizzarne la funzione
%\chordsoff		% <<< DECOMMENTA se vuoi una strofa senza accordi
^ Vergine del si^lenzio e della ^fe^de
^ l'Eterno ha posto in ^te la sua di^mora.^
^ Il tuo "sì" ri^suonerà per ^sem^pre:
^ l'Eterno ha posto in ^te la sua di^mora, 
^\textbf{Spirito, vieni!}
\endverse

%%%%% STROFA
\beginverse		%Oppure \beginverse* se non si vuole il numero di fianco
%\memorize 		% <<< DECOMMENTA se si vuole utilizzarne la funzione
%\chordsoff		% <<< DECOMMENTA se vuoi una strofa senza accordi
^ Tu nella Santa ^Casa accogli il ^do^no,
^ sei tu la porta ^che ci apre il C^ielo^
^ Con te la Chiesa ^canta la sua ^lo^de,
^ sei tu la porta ^che ci apre il Ci^elo, 
^\textbf{Spirito, vieni!}
\endverse

%%%%% STROFA
\beginverse		%Oppure \beginverse* se non si vuole il numero di fianco
%\memorize 		% <<< DECOMMENTA se si vuole utilizzarne la funzione
%\chordsoff		% <<< DECOMMENTA se vuoi una strofa senza accordi
^ Tu nella brezza ^parli al nostro ^cuo^re:
^ ascolteremo, ^Dio, la tua pa^rola;^
^ ci chiami a condi^videre il tuo a^mo^re:
^ ascolteremo, ^Dio, la tua pa^rola, 
^\textbf{Spirito, vieni!}
\endverse

\ifchorded
%%%%% RITORNELLO
\beginchorus
\textnote{\textbf{Finale }}
\[E]Luce di veri\[A]tà, \[E]fiamma di cari\[G#-]tà,
\[A]vincolo di uni\[E]tà, \[(C#-7*)]Spirito \[F#7]Santo A\[A]mo\[B]re.
\[E]Dona la liber\[A]tà, \[E]dona la santi\[G#-]tà,
\[A]fa' dell'umani\[E]tà \[(C#-7*)]il tuo \[D]canto di \[A]lo-\[B]o\[E]de. \[E*]
\endchorus
\fi

\endsong
%------------------------------------------------------------
%			FINE CANZONE
%------------------------------------------------------------





%-------------------------------------------------------------
%			INIZIO	CANZONE
%-------------------------------------------------------------


%titolo: 	Sei fuoco e vento
%autore: 	Andrea Testa
%tonalita: 	Do



%%%%%% TITOLO E IMPOSTAZONI
\beginsong{Sei fuoco e vento}[by={A. Testa}] 	% <<< MODIFICA TITOLO E AUTORE
\transpose{0} 						% <<< TRASPOSIZIONE #TONI (0 nullo)
\momenti{Ingresso; Comunione; Cresima}							% <<< INSERISCI MOMENTI	
% momenti vanno separati da ; e vanno scelti tra:
% Ingresso; Atto penitenziale; Acclamazione al Vangelo; Dopo il Vangelo; Offertorio; Comunione; Ringraziamento; Fine; Santi; Pasqua; Avvento; Natale; Quaresima; Canti Mariani; Battesimo; Prima Comunione; Cresima; Matrimonio; Meditazione; Spezzare del pane;
\ifchorded
	%\textnote{Tonalità originale }	% <<< EV COMMENTI (tonalità originale/migliore)
\fi


%%%%%% INTRODUZIONE
\ifchorded
\vspace*{\versesep}
\musicnote{
\begin{minipage}{0.48\textwidth}
\textbf{Intro}
\hfill 
%( \eighthnote \, 80)   % <<  MODIFICA IL TEMPO
% Metronomo: \eighthnote (ottavo) \quarternote (quarto) \halfnote (due quarti)
\end{minipage}
} 	
\vspace*{-\versesep}
\beginverse*

\nolyrics

%---- Prima riga -----------------------------
\vspace*{-\versesep}
\[A-]  % \[*D] per indicare le pennate, \rep{2} le ripetizioni

%---- Ogni riga successiva -------------------
%\vspace*{-\versesep}
%\[G] \[C]  \[D]	

%---- Ev Indicazioni -------------------------			
%\textnote{\textit{(Oppure tutta la strofa)} }	

\endverse
\fi



\beginverse*

\textnote{Recitato}	
\musicnote{\textit{(accompagnamento con gli accordi del ritornello)}}
\chordsoff
\textit{All'improvviso si sentì un rumore dal cielo 
come quando tira un forte vento
e riempì tutta la casa  dove si trovavano.
Allora videro qualcosa di simile \brk a lingue di fuoco
che si separavano e si posavano \brk	 sopra ciascuno di loro...
...e tutti furono pieni del Suo Spirito}
\endverse


%%%%% STROFA
\beginverse		%Oppure \beginverse* se non si vuole il numero di fianco
\memorize 		% <<< DECOMMENTA se si vuole utilizzarne la funzione
%\chordsoff		% <<< DECOMMENTA se vuoi una strofa senza accordi

In un \[A-]mare calmo e immobile, \brk con un \[C]cielo senza nuvole,
non si \[G]riesce a navi\[D-]gare,  \brk prose\[F]guire non si \[A-]può.
Una \[A-]brezza lieve e debole,    \brk poi di\[C]venta un vento a raffiche,
soffia \[G]forte sulle \[D-]barche    \brk e ci \[F]spinge via di \[A-]qua.
\vspace*{\versesep}
Come il \[C]vento da la \[G]forza   \brk per viag\[A-]giare in un o\[E-]ceano
così \[F]Tu ci dai lo \[C]Spirito   \brk che ci \[D-]guiderà da \[G]Te.

\endverse

%%%%% RITORNELLO
\beginchorus
\textnote{\textbf{Rit.} }

Sei come \[C]vento  \brk  che \[F]gonfia le \[C]vele,
sei come \[C]fuoco  \brk che ac\[F]cende l'a\[G]more, 
\[E] sei come l'\[A-]aria  \brk che \[E-]si respira \[F]libera
chiara \[C]luce che \[G]il cammino \[F]indica.	\rep{2}

\endchorus


%%%%%% EV. INTERMEZZO
\beginverse*
\vspace*{1.3\versesep}
{
	\nolyrics
	\textnote{Intermezzo strumentale}
	
	\ifchorded

	%---- Prima riga -----------------------------
	\vspace*{-\versesep}
	\[A-] \[A-]
	\fi
	%---- Ev Indicazioni -------------------------			
	%\textnote{\textit{(ripetizione della strofa)}} 
	 
}
\vspace*{\versesep}
\endverse






%%%%% STROFA
\beginverse		%Oppure \beginverse* se non si vuole il numero di fianco
%\memorize 		% <<< DECOMMENTA se si vuole utilizzarne la funzione
%\chordsoff		% <<< DECOMMENTA se vuoi una strofa senza accordi

Nella ^notte impenetrabile, \brk ogni ^cosa è irraggiungibile,
non puoi ^scegliere la ^strada  \brk se non ^vedi avanti a ^te.
Una ^luce fioca e debole,  \brk  sembra ^sorgere e poi crescere,
come ^fiamma che ^rigenera  \brk e che il^lumina la ^via.
\vspace*{\versesep}
Come il ^fuoco scioglie il ^gelo  \brk e rischi^ara ogni sen^tiero
così ^Tu riscaldi il ^cuore  \brk di chi ^Verbo annunce^rà.

\endverse




%%%%%% EV. CHIUSURA SOLO STRUMENTALE
\ifchorded
\beginchorus %oppure \beginverse*
\vspace*{1.3\versesep}
\textnote{Chiusura } %<<< EV. INDICAZIONI

\[C*]

\endchorus  %oppure \endverse
\fi


\endsong
%------------------------------------------------------------
%			FINE CANZONE
%------------------------------------------------------------
%-------------------------------------------------------------
%			INIZIO	CANZONE
%-------------------------------------------------------------


%titolo: 	Come fuoco vivo
%autore: 	Gen Verde, Gen Rosso
%tonalita: 	Do



%%%%%% TITOLO E IMPOSTAZONI
\beginsong{Come fuoco vivo}[by={Gen\ Verde, Gen\ Rosso}] 	% <<< MODIFICA TITOLO E AUTORE
\transpose{0} 						% <<< TRASPOSIZIONE #TONI (0 nullo)
\momenti{Comunione; Cresima}							% <<< INSERISCI MOMENTI	
% momenti vanno separati da ; e vanno scelti tra:
% Ingresso; Atto penitenziale; Acclamazione al Vangelo; Dopo il Vangelo; Offertorio; Comunione; Ringraziamento; Fine; Santi; Pasqua; Avvento; Natale; Quaresima; Canti Mariani; Battesimo; Prima Comunione; Cresima; Matrimonio; Meditazione; Spezzare del pane;
\ifchorded
	%\textnote{Tonalità migliore }	% <<< EV COMMENTI (tonalità originale/migliore)
\fi




%%%%%% INTRODUZIONE
\ifchorded
\vspace*{\versesep}
\musicnote{
\begin{minipage}{0.48\textwidth}
\textbf{Intro}
\hfill 
%( \eighthnote \, 80)   % <<  MODIFICA IL TEMPO
% Metronomo: \eighthnote (ottavo) \quarternote (quarto) \halfnote (due quarti)
\end{minipage}
} 	
\vspace*{-\versesep}
\beginverse*


\nolyrics

%---- Prima riga -----------------------------
\vspace*{-\versesep}
\[C] \[C] \[D-*]\[C] \[C] % \[*D] per indicare le pennate, \rep{2} le ripetizioni

%---- Ogni riga successiva -------------------
\vspace*{-\versesep}
  \[G*] \[A-7] \[A-7]

%---- Ogni riga successiva -------------------
\vspace*{-\versesep}
 \[E-7*] \[F] \[F] \quad \[G]\[G]

%---- Ev Indicazioni -------------------------			
%\textnote{\textit{(Oppure tutta la strofa)} }	

\endverse
\fi






%%%%% RITORNELLO
\beginchorus
\textnote{\textbf{Rit.}}
Come \[C]fuoco \[G]vivo si ac\[A-]cende in \[A-]noi
un'im\[D-7]mensa \[G]felici\[C]tà \[C]
che mai \[F]più nes\[G]suno ci \[C]toglie\[F]rà
\[D-7]perché tu \[D-7]sei ritor\[G4]nato. \[G]
Chi po\[C]trà ta\[G]cere, da \[A-]ora in \[A-]poi,
che sei \[D-]tu in cam\[G]mino con \[C]noi, \[C]
che la \[F]morte è \[G]vinta per \[C]sempre,
\[F]che \[D-7]ci hai rido\[D-7]nato la \[G4]vita? \[G]
\endchorus



%%%%% STROFA
\beginverse		%Oppure \beginverse* se non si vuole il numero di fianco
\memorize 		% <<< DECOMMENTA se si vuole utilizzarne la funzione
%\chordsoff		% <<< DECOMMENTA se vuoi una strofa senza accordi
\[A-] Spezzi il \[A-]pane da\[F]vanti a \[C]noi \[C]
mentre il \[C]sole è al tra\[G4]monto: \[G]
\[G-]o\[G-]ra gli \[A]occhi ti \[D-]vedono, \[F] sei \[F]tu! 
\[G4]Resta con \[G]noi.
\endverse




%%%%% STROFA
\beginverse		%Oppure \beginverse* se non si vuole il numero di fianco
%\memorize 		% <<< DECOMMENTA se si vuole utilizzarne la funzione
%\chordsoff		% <<< DECOMMENTA se vuoi una strofa senza accordi
^ E per ^sempre ti ^mostre^rai ^
in quel ^gesto d'a^more: ^
^ma^ni che ^ancora ^spezzano ^ pa^ne d'^eterni^tà.
\endverse



%%%%%% EV. INTERMEZZO
\beginverse*
\vspace*{1.3\versesep}
{
	
	\textnote{\textbf{Finale} \textit{[humming]}}
	
	\ifchorded

	%---- Prima riga -----------------------------
	%\vspace*{-\versesep}
	\[C]Uhmmm... \[C] \[D-*]\[C] \[C] % \[*D] per indicare le pennate, \rep{2} le ripetizioni

	%---- Ogni riga successiva -------------------
	\vspace*{-\versesep}
	\nolyrics  \[G*] \[A-7] \[A-7] 

	%---- Ogni riga successiva -------------------
	\vspace*{-\versesep}
	  \[E-7*] \[F] \[F] \quad \[G] \quad \[C*]

	\else 
	\chordsoff Uhmmm...
	\fi
	%---- Ev Indicazioni -------------------------			
	%\textnote{\textit{(ripetizione della strofa)}} 
	 
}
\vspace*{\versesep}
\endverse



\endsong
%------------------------------------------------------------
%			FINE CANZONE
%------------------------------------------------------------



%-------------------------------------------------------------
%			INIZIO	CANZONE
%-------------------------------------------------------------


%titolo: 	Lo Spirito di Cristo
%autore: 	
%tonalita: 	Mi



%%%%%% TITOLO E IMPOSTAZONI
\beginsong{Lo Spirito di Cristo}[by={}] 	% <<< MODIFICA TITOLO E AUTORE
\transpose{0} 						% <<< TRASPOSIZIONE #TONI (0 nullo)
\momenti{Ingresso; Cresima}							% <<< INSERISCI MOMENTI	
% momenti vanno separati da ; e vanno scelti tra:
% Ingresso; Atto penitenziale; Acclamazione al Vangelo; Dopo il Vangelo; Offertorio; Comunione; Ringraziamento; Fine; Santi; Pasqua; Avvento; Natale; Quaresima; Canti Mariani; Battesimo; Prima Comunione; Cresima; Matrimonio; Meditazione; Spezzare del pane;
\ifchorded
	%\textnote{Tonalità originale }	% <<< EV COMMENTI (tonalità originale/migliore)
\fi

%%%%%% INTRODUZIONE
\ifchorded
\vspace*{\versesep}
\textnote{Intro: \qquad \qquad  }%(\eighthnote 116) % <<  MODIFICA IL TEMPO
% Metronomo: \eighthnote (ottavo) \quarternote (quarto) \halfnote (due quarti)
\vspace*{-\versesep}
\beginverse*

\nolyrics

%---- Prima riga -----------------------------
\vspace*{-\versesep}
\[E] \[B] \[A] \[E]	 % \[*D] per indicare le pennate, \rep{2} le ripetizioni

%---- Ogni riga successiva -------------------
%\vspace*{-\versesep}
%\[G] \[C]  \[D]	

%---- Ev Indicazioni -------------------------			
%\textnote{\textit{(Oppure tutta la strofa)} }	

\endverse
\fi

%%%%% RITORNELLO
\beginchorus
\textnote{\textbf{Rit.}}

\[E]Lo \[B]Spirito di \[A]Cristo 
\[E]fa fio\[B]rire il de\[C#-]serto, 
\[G#-] torna la \[C#-]vita, 
\[A] noi diven\[B]tiamo testi\[E]moni di \[B]luce.

\endchorus

%%%%% STROFA
\beginverse		%Oppure \beginverse* se non si vuole il numero di fianco
\memorize 		% <<< DECOMMENTA se si vuole utilizzarne la funzione
%\chordsoff		% <<< DECOMMENTA se vuoi una strofa senza accordi

\[E] Non abbiamo rice\[B]vuto 
\[A] uno spirito di \[B]schiavitù,
\[E] ma uno spirito di a\[B]more, 
\[A] uno spirito di \[B]pace,
\[A] nel quale gri\[B]diamo 
\[A] abbà \[E]Padre,
\[A] abbà \[C#-]Pa\[B]dre.

\endverse

%%%%% STROFA
\beginverse		%Oppure \beginverse* se non si vuole il numero di fianco
%\memorize 		% <<< DECOMMENTA se si vuole utilizzarne la funzione
%\chordsoff		% <<< DECOMMENTA se vuoi una strofa senza accordi

^ Lo Spirito ^che 
^ Cristo ri^suscitò
^ darà vita ai nostri ^corpi, 
^ corpi mor^tali,  
^ e li rende^rà 
^ strumenti di sal^vezza, 
^ strumenti di sal^vez^za.

\endverse

%%%%% STROFA
\beginverse		%Oppure \beginverse* se non si vuole il numero di fianco
%\memorize 		% <<< DECOMMENTA se si vuole utilizzarne la funzione
\chordsoff		% <<< DECOMMENTA se vuoi una strofa senza accordi

Sono venuto a portare
il fuoco sulla terra
e come desidero
che divampi nel mondo 
e porti amore
ed entusiasmo in tutti i cuori.

\endverse

\endsong
%------------------------------------------------------------
%			FINE CANZONE
%------------------------------------------------------------

<<<<<<< HEAD
%++++++++++++++++++++++++++++++++++++++++++++++++++++++++++++
%			CANZONE TRASPOSTA
%++++++++++++++++++++++++++++++++++++++++++++++++++++++++++++
\ifchorded
%decremento contatore per avere stesso numero
\addtocounter{songnum}{-1} 
\beginsong{Lo Spirito di Cristo}[by={}] 	% <<< COPIA TITOLO E AUTORE
\transpose{3} 						% <<< TRASPOSIZIONE #TONI + - (0 nullo)
%\preferflats  %SE VOGLIO FORZARE i bemolle come alterazioni
%\prefersharps %SE VOGLIO FORZARE i # come alterazioni
\ifchorded
	%\textnote{Tonalità originale}	% <<< EV COMMENTI (tonalità originale/migliore)
\fi

%%%%%% INTRODUZIONE
\ifchorded
\vspace*{\versesep}
\textnote{Intro: \qquad \qquad  }%(\eighthnote 116) % <<  MODIFICA IL TEMPO
% Metronomo: \eighthnote (ottavo) \quarternote (quarto) \halfnote (due quarti)
\vspace*{-\versesep}
\beginverse*

\nolyrics

%---- Prima riga -----------------------------
\vspace*{-\versesep}
\[E] \[B] \[A] \[E]	 % \[*D] per indicare le pennate, \rep{2} le ripetizioni

%---- Ogni riga successiva -------------------
%\vspace*{-\versesep}
%\[G] \[C]  \[D]	

%---- Ev Indicazioni -------------------------			
%\textnote{\textit{(Oppure tutta la strofa)} }	

\endverse
\fi

%%%%% RITORNELLO
\beginchorus
\textnote{\textbf{Rit.}}

\[E]Lo \[B]Spirito di \[A]Cristo \[E]fa fio\[B]rire il de\[C#-]serto, \[G#-]
torna la \[C#-]vita, \[A] noi diven\[B]tiamo testi\[E]moni di \[B]luce.

\endchorus

%%%%% STROFA
\beginverse		%Oppure \beginverse* se non si vuole il numero di fianco
\memorize 		% <<< DECOMMENTA se si vuole utilizzarne la funzione
%\chordsoff		% <<< DECOMMENTA se vuoi una strofa senza accordi

\[E] Non abbiamo rice\[B]vuto
\[A] uno spirito di \[B]schiavitù,
\[E] ma uno spirito di a\[B]more,
\[A] uno spirito di \[B]pace,\[A] nel quale gri\[B]diamo
\[A] abbà \[E]Padre,\[A] abbà \[C#-]Pa\[B]dre.

\endverse

%%%%% STROFA
\beginverse		%Oppure \beginverse* se non si vuole il numero di fianco
%\memorize 		% <<< DECOMMENTA se si vuole utilizzarne la funzione
%\chordsoff		% <<< DECOMMENTA se vuoi una strofa senza accordi

Lo Spirito che Cristo
risuscitò
darà vita ai nostri corpi,
corpi mortali, e li renderà
strumenti di salvezza, strumenti di salvezza.

\endverse

%%%%% STROFA
\beginverse		%Oppure \beginverse* se non si vuole il numero di fianco
%\memorize 		% <<< DECOMMENTA se si vuole utilizzarne la funzione
%\chordsoff		% <<< DECOMMENTA se vuoi una strofa senza accordi

Sono venuto a portare
il fuoco sulla terra
e come desidero
che divampi nel mondo e porti amore
ed entusiasmo in tutti i cuori.

\endverse

\endsong

\fi
%++++++++++++++++++++++++++++++++++++++++++++++++++++++++++++
%			FINE CANZONE TRASPOSTA
%++++++++++++++++++++++++++++++++++++++++++++++++++++++++++++
=======
% %++++++++++++++++++++++++++++++++++++++++++++++++++++++++++++
% %			CANZONE TRASPOSTA
% %++++++++++++++++++++++++++++++++++++++++++++++++++++++++++++
% \ifchorded
% %decremento contatore per avere stesso numero
% \addtocounter{songnum}{-1} 
% \beginsong{Lo Spirito di Cristo}[by={}] 	% <<< COPIA TITOLO E AUTORE
% \transpose{3} 						% <<< TRASPOSIZIONE #TONI + - (0 nullo)
% %\preferflats  %SE VOGLIO FORZARE i bemolle come alterazioni
% %\prefersharps %SE VOGLIO FORZARE i # come alterazioni
% \ifchorded
% 	%\textnote{Tonalità originale}	% <<< EV COMMENTI (tonalità originale/migliore)
% \fi

% %%%%%% INTRODUZIONE
% \ifchorded
% \vspace*{\versesep}
% \textnote{Intro: \qquad \qquad  }%(\eighthnote 116) % <<  MODIFICA IL TEMPO
% % Metronomo: \eighthnote (ottavo) \quarternote (quarto) \halfnote (due quarti)
% \vspace*{-\versesep}
% \beginverse*

% \nolyrics

% %---- Prima riga -----------------------------
% \vspace*{-\versesep}
% \[E] \[B] \[A] \[E]	 % \[*D] per indicare le pennate, \rep{2} le ripetizioni

% %---- Ogni riga successiva -------------------
% %\vspace*{-\versesep}
% %\[G] \[C]  \[D]	

% %---- Ev Indicazioni -------------------------			
% %\textnote{\textit{(Oppure tutta la strofa)} }	

% \endverse
% \fi

% %%%%% RITORNELLO
% \beginchorus
% \textnote{\textbf{Rit.}}

% \[E]Lo \[B]Spirito di \[A]Cristo 
% \[E]fa fio\[B]rire il de\[C#-]serto, 
% \[G#-] torna la \[C#-]vita, 
% \[A] noi diven\[B]tiamo testi\[E]moni di \[B]luce.

% \endchorus

% %%%%% STROFA
% \beginverse		%Oppure \beginverse* se non si vuole il numero di fianco
% \memorize 		% <<< DECOMMENTA se si vuole utilizzarne la funzione
% %\chordsoff		% <<< DECOMMENTA se vuoi una strofa senza accordi

% \[E] Non abbiamo rice\[B]vuto 
% \[A] uno spirito di \[B]schiavitù,
% \[E] ma uno spirito di a\[B]more, 
% \[A] uno spirito di \[B]pace,
% \[A] nel quale gri\[B]diamo 
% \[A] abbà \[E]Padre,
% \[A] abbà \[C#-]Pa\[B]dre.

% \endverse

% %%%%% STROFA
% \beginverse		%Oppure \beginverse* se non si vuole il numero di fianco
% %\memorize 		% <<< DECOMMENTA se si vuole utilizzarne la funzione
% %\chordsoff		% <<< DECOMMENTA se vuoi una strofa senza accordi

% ^ Lo Spirito ^che 
% ^ Cristo ri^suscitò
% ^ darà vita ai nostri ^corpi, 
% ^ corpi mor^tali,  
% ^ e li rende^rà 
% ^ strumenti di sal^vezza, 
% ^ strumenti di sal^vez^za.

% \endverse

% %%%%% STROFA
% \beginverse		%Oppure \beginverse* se non si vuole il numero di fianco
% %\memorize 		% <<< DECOMMENTA se si vuole utilizzarne la funzione
% \chordsoff		% <<< DECOMMENTA se vuoi una strofa senza accordi

% Sono venuto a portare
% il fuoco sulla terra
% e come desidero
% che divampi nel mondo 
% e porti amore
% ed entusiasmo in tutti i cuori.

% \endverse

% \endsong

% \fi
% %++++++++++++++++++++++++++++++++++++++++++++++++++++++++++++
% %			FINE CANZONE TRASPOSTA
% %++++++++++++++++++++++++++++++++++++++++++++++++++++++++++++
>>>>>>> 1cf7e891cb70141482a1e55d331c97dc8203b0ef

%-------------------------------------------------------------
%			INIZIO	CANZONE
%-------------------------------------------------------------


%titolo: 	Oltre la memoria (Symbolum 80)
%autore: 	Sequeri
%tonalita: 	Sol 



%%%%%% TITOLO E IMPOSTAZONI
\beginsong{Oltre la memoria }[ititle={Symbolum 80}, by={Symbolum 80 — P. Sequeri}] 	% <<< MODIFICA TITOLO E AUTORE
\transpose{0} 						% <<< TRASPOSIZIONE #TONI (0 nullo)
\momenti{Comunione}							% <<< INSERISCI MOMENTI	
% momenti vanno separati da ; e vanno scelti tra:
% Ingresso; Atto penitenziale; Acclamazione al Vangelo; Dopo il Vangelo; Offertorio; Comunione; Ringraziamento; Fine; Santi; Pasqua; Avvento; Natale; Quaresima; Canti Mariani; Battesimo; Prima Comunione; Cresima; Matrimonio; Meditazione;
\ifchorded
	\textnote{Tonalità migliore}	% <<< EV COMMENTI (tonalità originale/migliore)
\fi





%%%%%% INTRODUZIONE
\ifchorded
\vspace*{\versesep}
\textnote{Intro: \qquad \qquad  (\eighthnote 136)}% % << MODIFICA IL TEMPO
% Metronomo: \eighthnote (ottavo) \quarternote (quarto) \halfnote (due quarti)
\vspace*{-\versesep}
\beginverse*

\nolyrics

%---- Prima riga -----------------------------
\vspace*{-\versesep}
\[D-] \[D-] \[G-]  \[D-]	 % \[*D] per indicare le pennate, \rep{2} le ripetizioni

%---- Ogni riga successiva -------------------
%\vspace*{-\versesep}
%\[G] \[C]  \[G]	

%---- Ev Indicazioni -------------------------			
%\textnote{\textit{(Oppure tutta la strofa)} }	

\endverse
\fi



%%%%% STROFA
\beginverse
\memorize


\[D-]Oltre la me\[7]moria del \[G-]tempo che ho vis\[D-]suto,
\[D-]oltre la spe\[A-]ranza 
che \[G-]serve al mio do\[A]mani, \quad \[A]
\[D-]oltre il desi\[7]derio di \[G-]vivere il pre\[D-]sente,
an\[D-]ch’io, confesso, ho chi\[A-]esto 
che \[G-]cosa è veri\[A]tà. 

\vspace{1.5\versesep}


\[D]E \[A]tu \[B-]come un desi\[F#-]derio 
\[G]che non \[A]ha me\[B-]morie, \[E]Padre bu\[A]ono, 
\[D]come una spe\[A]ranza c\[G]he non ha con\[F#-]fini,
\[G]come un \[A]tempo e\[B-]terno 
\[E7]sei per \[A]me.

\endverse


%%%%% RITORNELLO
\beginchorus
\textnote{\textbf{Rit.}}

\[F]Io \[C]so \[D-]quanto amore ch\[A-]iede 
\[B&]questa \[C]lunga at\[F]te\[(D-)]sa 
\[G]del tuo giorno, \[C]Dio; 
\[F]luce in ogni \[C]cosa \[D-]io non vedo an\[A-]cora:
\[B&]ma la \[C7]tua pa\[F]ro\[(D-)]la 
\[G]mi risch\[7]iare\[A]rà!

\endchorus




%%%%% STROFA
\beginverse

^Quando le pa^role non ^bastano all’a^more,
^quando il mio fra^tello 
do^manda più del ^pane, \quad ^
^quando l’illu^sione pro^mette un mondo ^nuovo,
anch’^io rimango in^certo 
nel ^mezzo del cam^mino.

\vspace{1.5\versesep}

^E ^tu ^Figlio tanto a^mato,
^veri^tà dell’^uomo, ^mio Si^gnore,
^come la pro^messa ^di un perdono e^terno,
^liber^tà infi^nita 
^sei per ^me.


\endverse


%%%%% RITORNELLO
\beginchorus
\textnote{\textbf{Rit.}}

\[F]Io \[C]so \[D-]quanto amore ch\[A-]iede 
\[B&]questa \[C]lunga at\[F]te\[(D-)]sa 
\[G]del tuo giorno, \[C]Dio; 
\[F]luce in ogni \[C]cosa \[D-]io non vedo an\[A-]cora:
\[B&]ma la \[C7]tua pa\[F]ro\[(D-)]la 
\[G]mi risch\[7]iare\[A]rà!

\endchorus




%%%%% STROFA
\beginverse


^Chiedo alla mia ^mente co^raggio di cer^care,
^chiedo alle mie ^mani 
la ^forza di do^nare, \quad ^
^chiedo al cuore in^certo pas^sione per la ^vita,
e ^chiedo a te fra^tello 
di ^credere con ^me.

\vspace{1.5\versesep}

^E ^tu, ^forza della ^vita,
^Spiri^to d’a^more, ^dolce Id^dio,
^grembo d’ogni ^cosa, ^tenerezza im^mensa,
^veri^tà del ^mondo 
^sei per ^me.


\endverse


%%%%% RITORNELLO
\beginchorus
\textnote{\textbf{Rit.}}

\[F]Io \[C]so \[D-]quanto amore ch\[A-]iede 
\[B&]questa \[C]lunga at\[F]te\[(D-)]sa 
\[G]del tuo giorno, \[C]Dio; 
\[F]luce in ogni \[C]cosa \[D-]io non vedo an\[A-]cora:
\[B&]ma la \[C7]tua pa\[F]ro\[(D-)]la 

\endchorus




%%%%%% EV. FINALE
\beginchorus %oppure \beginverse*
\vspace*{1.3\versesep}
\textnote{Finale \textit{(rallentando)}} %<<< EV. INDICAZIONI

\[E]mi risch\[A]iare\[D]rà!

\endchorus  %oppure \endverse






\endsong
%------------------------------------------------------------
%			FINE CANZONE
%------------------------------------------------------------




%++++++++++++++++++++++++++++++++++++++++++++++++++++++++++++
%			CANZONE TRASPOSTA
%++++++++++++++++++++++++++++++++++++++++++++++++++++++++++++
\ifchorded
%decremento contatore per avere stesso numero
\addtocounter{songnum}{-1} 
\beginsong{Oltre la memoria}[ititle={Ma la tua parola}, by={Symbolum 80 — P. Sequeri}] 	% <<< MODIFICA TITOLO E AUTORE
\transpose{2} 						% <<< TRASPOSIZIONE #TONI (0 nullo)
%\preferflats %SE VOGLIO FORZARE i bemolle come alterazioni
\prefersharps %SE VOGLIO FORZARE i # come alterazioni
\ifchorded
	\textnote{Tonalità originale}	% <<< EV COMMENTI (tonalità originale/migliore)
\fi



%%%%%% INTRODUZIONE
\ifchorded
\vspace*{\versesep}
\textnote{Intro: \qquad \qquad  (\eighthnote 136)}% % << MODIFICA IL TEMPO
% Metronomo: \eighthnote (ottavo) \quarternote (quarto) \halfnote (due quarti)
\vspace*{-\versesep}
\beginverse*

\nolyrics

%---- Prima riga -----------------------------
\vspace*{-\versesep}
\[D-] \[D-] \[G-]  \[D-]	 % \[*D] per indicare le pennate, \rep{2} le ripetizioni

%---- Ogni riga successiva -------------------
%\vspace*{-\versesep}
%\[G] \[C]  \[G]	

%---- Ev Indicazioni -------------------------			
%\textnote{\textit{(Oppure tutta la strofa)} }	

\endverse
\fi



%%%%% STROFA
\beginverse
\memorize


\[D-]Oltre la me\[7]moria del \[G-]tempo che ho vis\[D-]suto,
\[D-]oltre la spe\[A-]ranza 
che \[G-]serve al mio do\[A]mani, \quad \[A]
\[D-]oltre il desi\[7]derio di \[G-]vivere il pre\[D-]sente,
an\[D-]ch’io, confesso, ho chi\[A-]esto 
che \[G-]cosa è veri\[A]tà. 

\vspace{1.5\versesep}

\[D]E \[A]tu \[B-]come un desi\[F#-]derio 
\[G]che non \[A]ha me\[B-]morie, \[E]Padre bu\[A]ono, 
\[D]come una spe\[A]ranza c\[G]he non ha con\[F#-]fini,
\[G]come un \[A]tempo e\[B-]terno 
\[E7]sei per \[A]me.

\endverse


%%%%% RITORNELLO
\beginchorus
\textnote{\textbf{Rit.}}

\[F]Io \[C]so \[D-]quanto amore ch\[A-]iede 
\[B&]questa \[C]lunga at\[F]te\[(D-)]sa 
\[G]del tuo giorno, \[C]Dio; 
\[F]luce in ogni \[C]cosa \[D-]io non vedo an\[A-]cora:
\[B&]ma la \[C7]tua pa\[F]ro\[(D-)]la 
\[G]mi risch\[7]iare\[A]rà!

\endchorus



%%%%% STROFA
\beginverse

^Quando le pa^role non ^bastano all’a^more,
^quando il mio fra^tello 
do^manda più del ^pane, \quad ^
^quando l’illu^sione pro^mette un mondo ^nuovo,
anch’^io rimango in^certo 
nel ^mezzo del cam^mino.

\vspace{1.5\versesep}

^E ^tu ^Figlio tanto a^mato,
^veri^tà dell’^uomo, ^mio Si^gnore,
^come la pro^messa ^di un perdono e^terno,
^liber^tà infi^nita 
^sei per ^me.


\endverse


%%%%% RITORNELLO
\beginchorus
\textnote{\textbf{Rit.}}

\[F]Io \[C]so \[D-]quanto amore ch\[A-]iede 
\[B&]questa \[C]lunga at\[F]te\[(D-)]sa 
\[G]del tuo giorno, \[C]Dio; 
\[F]luce in ogni \[C]cosa \[D-]io non vedo an\[A-]cora:
\[B&]ma la \[C7]tua pa\[F]ro\[(D-)]la 
\[G]mi risch\[7]iare\[A]rà!

\endchorus




%%%%% STROFA
\beginverse


^Chiedo alla mia ^mente co^raggio di cer^care,
^chiedo alle mie ^mani 
la ^forza di do^nare, \quad ^
^chiedo al cuore in^certo pas^sione per la ^vita,
e ^chiedo a te fra^tello 
di ^credere con ^me.

\vspace{1.5\versesep}

^E ^tu, ^forza della ^vita,
^Spiri^to d’a^more, ^dolce Id^dio,
^grembo d’ogni ^cosa, ^tenerezza im^mensa,
^veri^tà del ^mondo 
^sei per ^me.


\endverse



%%%%% RITORNELLO
\beginchorus
\textnote{\textbf{Rit.}}

\[F]Io \[C]so \[D-]quanto amore ch\[A-]iede 
\[B&]questa \[C]lunga at\[F]te\[(D-)]sa 
\[G]del tuo giorno, \[C]Dio; 
\[F]luce in ogni \[C]cosa \[D-]io non vedo an\[A-]cora:
\[B&]ma la \[C7]tua pa\[F]ro\[(D-)]la 

\endchorus



%%%%%% EV. FINALE
\beginchorus %oppure \beginverse*
\vspace*{1.3\versesep}
\textnote{Finale \textit{(rallentando)}} %<<< EV. INDICAZIONI

\[E]mi risch\[A]iare\[D]rà!

\endchorus  %oppure \endverse





\endsong

\fi
%++++++++++++++++++++++++++++++++++++++++++++++++++++++++++++
%			FINE CANZONE TRASPOSTA
%++++++++++++++++++++++++++++++++++++++++++++++++++++++++++++

%titolo{Benedici, o Signore}
%autore{Gen Rosso}
%album{Se siamo uniti}
%tonalita{Si-}
%famiglia{Liturgica}
%gruppo{}
%momenti{Offertorio}
%identificatore{benedici_o_signore}
%data_revisione{2013_12_30}
%trascrittore{Francesco Endrici}
\beginsong{Benedici, o Signore}[by={Gen\ Rosso}]
\beginverse
\[B-]Nebbia e freddo, giorni lunghi e a\[A]mari
mentre il seme \[B-]muore.
\[D]Poi prodigio, antico e sempre \[A]nuovo,
del primo filo d'\[G7+]erba.
E nel \[D]vento dell'e\[A]state on\[B-]deggiano le \[D]spighe
a\[A]vremo ancora \[F#]pa\[B]ne.
\endverse
\beginchorus
\[E]Bene\[B]dici, \[E]o Si\[B]gnore,
\[A]questa o\ch{E}{f}{f}{ff}erta che por\[F#4]tiamo a \[F#]te
\[E]Facci \[B]uno \[G#-]come il \[E&-]pane
\[C#]che anche \[E]oggi hai \[B]dato a noi.
\endchorus
\beginverse
\chordsoff
^Nei filari, dopo il lungo in^verno \brk fremono le ^viti.
^La rugiada avvolge nel si^lenzio \brk i primi tralci ^verdi.
Poi co^lori dell'au^tunno, coi ^grappoli ma^turi
a^vremo ancora ^vi^no.
\endverse
\endsong





% ----- ALLELUIA ---------
\makeatletter \def\input@path{{Songs/Santo/}} \makeatother
%-------------------------------------------------------------
%			INIZIO	CANZONE
%-------------------------------------------------------------


%titolo: 	Santo Milan
%autore: 	Gen Verde
%tonalita: 	Sol 



%%%%%% TITOLO E IMPOSTAZONI
\beginsong{Santo Milan}[by={Gen Verde}] 	% <<< MODIFICA TITOLO E AUTORE
\transpose{0} 						% <<< TRASPOSIZIONE #TONI (0 nullo)
\momenti{}							% <<< INSERISCI MOMENTI	
% momenti vanno separati da ; e vanno scelti tra:
% Ingresso; Atto penitenziale; Acclamazione al Vangelo; Dopo il Vangelo; Offertorio; Comunione; Ringraziamento; Fine; Santi; Pasqua; Avvento; Natale; Quaresima; Canti Mariani; Battesimo; Prima Comunione; Cresima; Matrimonio; Meditazione;
\ifchorded
	\textnote{Tonalità originale }	% <<< EV COMMENTI (tonalità originale/migliore)
\fi


%%%%%% INTRODUZIONE
\ifchorded
\vspace*{\versesep}
\textnote{Intro: \qquad \qquad  }%(\eighthnote 116) % << MODIFICA IL TEMPO
% Metronomo: \eighthnote (ottavo) \quarternote (quarto) \halfnote (due quarti)
\vspace*{-\versesep}
\beginverse*

\nolyrics

%---- Prima riga -----------------------------
\vspace*{-\versesep}
\[A] \[E]  \[C#-] \[B]	 % \[*D] per indicare le pennate, \rep{2} le ripetizioni

%---- Ogni riga successiva -------------------
\vspace*{-\versesep}
\[F#-] \[E]  \[A]  \[B]	

%---- Ev Indicazioni -------------------------			
\textnote{\textit{(come le prime due righe)} }	

\endverse
\fi








%%%%% RITORNELLO
\beginchorus

\[A]San\[E]to, \[C#-]San\[B]to,
\[F#-]Santo il Si\[E]gnore, \[A]Dio dell'uni\[B]verso.
\[A]San\[E]to, \[C#-]San\[B]to.
I \[F#-]cieli e la \[E]terra 
sono \[A]pieni della tua \[(F#-)]glo\[E]ria.

\endchorus



%%%%% STROFA
\beginverse*		%Oppure \beginverse* se non si vuole il numero di fianco
%\memorize 		% <<< DECOMMENTA se si vuole utilizzarne la funzione
%\chordsoff		% <<< DECOMMENTA se vuoi una strofa senza accordi

O\[A]sanna nel\[B]l'alto dei \[A]cie\[B]li.
O\[F#-]sanna nell'alto dei \[A]cieli.

\endverse



%%%%% RITORNELLO
\beginchorus

\[A]San\[E]to, \[C#-]San\[B]to,
\[F#-]Santo il Si\[E]gnore, \[A]Dio dell'uni\[B]verso.
\[A]San\[E]to, \[C#-]San\[B]to.
I \[F#-]cieli e la \[E]terra 
sono \[A]pieni della tua \[(F#-)]glo\[E]ria.

\endchorus




%%%%% STROFA
\beginverse*		%Oppure \beginverse* se non si vuole il numero di fianco
%\memorize 		% <<< DECOMMENTA se si vuole utilizzarne la funzione
%\chordsoff		& <<< DECOMMENTA se vuoi una strofa senza accordi

\[B]Benedetto co\[A]lui che viene
nel \[E]nome del Sig\[B]nore.
O\[A]sanna nel\[B]l'alto dei \[A]cie\[B]li.
O\[F#-]sanna nell'alto dei \[A]cieli.

\endverse


%%%%% RITORNELLO
\beginchorus

\[A]San\[E]to, \[C#-]San\[B]to,
\[F#-]Sa-\[A]a-n\[E]to.  \[*E] 

\endchorus











\endsong
%------------------------------------------------------------
%			FINE CANZONE
%------------------------------------------------------------




%++++++++++++++++++++++++++++++++++++++++++++++++++++++++++++
%			CANZONE TRASPOSTA
%++++++++++++++++++++++++++++++++++++++++++++++++++++++++++++
\ifchorded
%decremento contatore per avere stesso numero
\addtocounter{songnum}{-1} 
\beginsong{Santo Milan}[by={Gen Verde}] 	% <<< COPIA TITOLO E AUTORE
\transpose{-2} 						% <<< TRASPOSIZIONE #TONI + - (0 nullo)
\ifchorded
	\textnote{Tonalità più facile per le chitarre}	% <<< EV COMMENTI (tonalità originale/migliore)
\fi


%%%%%% INTRODUZIONE
\ifchorded
\vspace*{\versesep}
\textnote{Intro: \qquad \qquad  }%(\eighthnote 116) % << MODIFICA IL TEMPO
% Metronomo: \eighthnote (ottavo) \quarternote (quarto) \halfnote (due quarti)
\vspace*{-\versesep}
\beginverse*

\nolyrics

%---- Prima riga -----------------------------
\vspace*{-\versesep}
\[A] \[E]  \[C#-] \[B]	 % \[*D] per indicare le pennate, \rep{2} le ripetizioni

%---- Ogni riga successiva -------------------
\vspace*{-\versesep}
\[F#-] \[E]  \[A]  \[B]	

%---- Ev Indicazioni -------------------------			
\textnote{\textit{(come le prime due righe)} }	

\endverse
\fi








%%%%% RITORNELLO
\beginchorus

\[A]San\[E]to, \[C#-]San\[B]to,
\[F#-]Santo il Si\[E]gnore, \[A]Dio dell'uni\[B]verso.
\[A]San\[E]to, \[C#-]San\[B]to.
I \[F#-]cieli e la \[E]terra 
sono \[A]pieni della tua \[(F#-)]glo\[E]ria.

\endchorus



%%%%% STROFA
\beginverse*		%Oppure \beginverse* se non si vuole il numero di fianco
%\memorize 		% <<< DECOMMENTA se si vuole utilizzarne la funzione
%\chordsoff		% <<< DECOMMENTA se vuoi una strofa senza accordi

O\[A]sanna nel\[B]l'alto dei \[A]cie\[B]li.
O\[F#-]sanna nell'alto dei \[A]cieli.

\endverse



%%%%% RITORNELLO
\beginchorus

\[A]San\[E]to, \[C#-]San\[B]to,
\[F#-]Santo il Si\[E]gnore, \[A]Dio dell'uni\[B]verso.
\[A]San\[E]to, \[C#-]San\[B]to.
I \[F#-]cieli e la \[E]terra 
sono \[A]pieni della tua \[(F#-)]glo\[E]ria.

\endchorus




%%%%% STROFA
\beginverse*		%Oppure \beginverse* se non si vuole il numero di fianco
%\memorize 		% <<< DECOMMENTA se si vuole utilizzarne la funzione
%\chordsoff		& <<< DECOMMENTA se vuoi una strofa senza accordi

\[B]Benedetto co\[A]lui che viene
nel \[E]nome del Sig\[B]nore.
O\[A]sanna nel\[B]l'alto dei \[A]cie\[B]li.
O\[F#-]sanna nell'alto dei \[A]cieli.

\endverse


%%%%% RITORNELLO
\beginchorus

\[A]San\[E]to, \[C#-]San\[B]to,
\[F#-]Sa-\[A]a-n\[E]to.  \[*E] 

\endchorus






\endsong

\fi
%++++++++++++++++++++++++++++++++++++++++++++++++++++++++++++
%			FINE CANZONE TRASPOSTA
%++++++++++++++++++++++++++++++++++++++++++++++++++++++++++++




% ----- LITURGIA ---------
\makeatletter \def\input@path{{Songs/Liturgia/}} \makeatother
%-------------------------------------------------------------
%			INIZIO	CANZONE
%-------------------------------------------------------------


%titolo: 	Pane di vita nuova
%autore: 	Frisina
%tonalita: 	Re 



%%%%%% TITOLO E IMPOSTAZONI
\beginsong{Pane di vita nuova}[by={M. Frisina}] 	% <<< MODIFICA TITOLO E AUTORE
\transpose{0} 						% <<< TRASPOSIZIONE #TONI (0 nullo)
\momenti{Comunione;  Pasqua; Spezzare del Pane;}							% <<< INSERISCI MOMENTI	
% momenti vanno separati da ; e vanno scelti tra:
% Ingresso; Atto penitenziale; Acclamazione al Vangelo; Dopo il Vangelo; Offertorio; Comunione; Ringraziamento; Fine; Santi; Pasqua; Avvento; Natale; Quaresima; Canti Mariani; Battesimo; Prima Comunione; Cresima; Matrimonio; Meditazione; Spezzare del pane;
\ifchorded
	%\textnote{Tonalità originale }	% <<< EV COMMENTI (tonalità originale/migliore)
\fi

%%%%%% INTRODUZIONE
\ifchorded
\vspace*{\versesep}
\musicnote{
\begin{minipage}{0.48\textwidth}
\textbf{Intro}
\hfill 
%( \eighthnote \, 80)   % <<  MODIFICA IL TEMPO
% Metronomo: \eighthnote (ottavo) \quarternote (quarto) \halfnote (due quarti)
\end{minipage}
} 	
\vspace*{-\versesep}
\beginverse*

\nolyrics

%---- Prima riga -----------------------------
\vspace*{-\versesep}
\[D] \[G] \[A] \[D]	 % \[*D] per indicare le pennate, \rep{2} le ripetizioni

%---- Ogni riga successiva -------------------
%\vspace*{-\versesep}
%\[G] \[C]  \[D]	

%---- Ev Indicazioni -------------------------			
%\textnote{\textit{(Oppure tutta la strofa)} }	

\endverse
\fi


%%%%% STROFA
\beginverse		%Oppure \beginverse* se non si vuole il numero di fianco
\memorize 		% <<< DECOMMENTA se si vuole utilizzarne la funzione
%\chordsoff		% <<< DECOMMENTA se vuoi una strofa senza accordi
\[D]Pane \[G]di vita \[A]nuo\[D]va, 
\[G]vero \[D]cibo dato agli \[E-]uomi\[A]ni,
\[G]nutri\[D]mento \[E-]che sostiene il \[A]mondo, 
\[B-*]do\[G]no  \[E-]splendido  di \[A]gra\[D]zia.
\endverse




%%%%% STROFA
\beginverse*		%Oppure \beginverse* se non si vuole il numero di fianco
%\memorize 		% <<< DECOMMENTA se si vuole utilizzarne la funzione
%\chordsoff		% <<< DECOMMENTA se vuoi una strofa senza accordi
^Tu sei ^sublime ^frut^to 
^di quell'^albero di ^vi^ta
^che A^damo ^non potè toc^care:
^Ora ^è in  ^Cristo a noi do^na^to.
\endverse







%%%%% RITORNELLO
\beginchorus
\textnote{\textbf{Rit.}}
\[G]Pane \[D]della \[E-]vi\[A]ta, 
\[D]sangue \[F#-]di sal\[G]vez\[A]za,
\[G]vero \[D]corpo, \[E-]vera be\[B-]vanda,
\[E-]cibo di \[B-]grazia per il \[A]mon\[D]do.
\endchorus



%%%%% STROFA
\beginverse		%Oppure \beginverse* se non si vuole il numero di fianco
%\memorize 		% <<< DECOMMENTA se si vuole utilizzarne la funzione
%\chordsoff		% <<< DECOMMENTA se vuoi una strofa senza accordi
^Sei l'A^gnello immo^la^to
^nel cui ^Sangue è la sal^vez^za,
^memo^riale ^della vera ^Pasqua
^del^la ^nuova Alle^an^za.
\endverse



%%%%% STROFA
\beginverse*		%Oppure \beginverse* se non si vuole il numero di fianco
%\memorize 		% <<< DECOMMENTA se si vuole utilizzarne la funzione
%\chordsoff		% <<< DECOMMENTA se vuoi una strofa senza accordi
^Manna ^che nel de^ser^to
^nutri il ^popolo in cam^mi^no,
^sei so^stegno e ^forza nella ^prova
^per ^la ^Chiesa in mezzo al ^mon^do.
\endverse


%%%%% STROFA
\beginverse		%Oppure \beginverse* se non si vuole il numero di fianco
%\memorize 		% <<< DECOMMENTA se si vuole utilizzarne la funzione
\chordsoff		% <<< DECOMMENTA se vuoi una strofa senza accordi
^Vino ^che ci dà ^gio^ia,
^che ri^scalda il nostro ^cuo^re,
^sei per ^noi ^il prezioso ^frutto
^del^la ^vigna del Si^gno^re.
\endverse

%%%%% STROFA
\beginverse*		%Oppure \beginverse* se non si vuole il numero di fianco
%\memorize 		% <<< DECOMMENTA se si vuole utilizzarne la funzione
\chordsoff		% <<< DECOMMENTA se vuoi una strofa senza accordi
^Dalla ^vite ai ^tral^ci
^scorre ^la vitale ^lin^fa
^che ci ^dona ^la vita di^vina,
^scor^re il ^sangue dell'a^mo^re.
\endverse



%%%%% STROFA
\beginverse
\chordsoff
^Al ban^chetto ci in^vi^ti
^che per ^noi hai prepa^ra^to,
^doni all'^uomo ^la tua Sa^pienza,
^do^ni il ^Verbo della ^vi^ta.
\endverse

%%%%% STROFA
\beginverse*		%Oppure \beginverse* se non si vuole il numero di fianco
%\memorize 		% <<< DECOMMENTA se si vuole utilizzarne la funzione
\chordsoff		% <<< DECOMMENTA se vuoi una strofa senza accordi
^Segno ^d'amore e^ter^no
^pegno ^di sublimi ^noz^ze,
^comu^nione ^nell'unico ^corpo
^che ^in ^Cristo noi for^mia^mo.
\endverse



\beginverse
\chordsoff
^Nel tuo ^Sangue è la ^vi^ta
^ed il ^fuoco dello ^Spiri^to,
^la sua ^fiamma in^cendia il nostro ^cuore 
^e ^pu^rifica il ^mon^do.
\endverse
\beginverse*
\chordsoff
^Nel pro^digio dei ^pa^ni
^tu sfa^masti ogni ^uo^mo,
^nel tuo a^more il ^povero è nu^trito
^e ^ri^ceve la tua ^vi^ta.
\endverse




\beginverse
\chordsoff
Sacerdote eterno
Tu sei vittima ed altare,
offri al Padre tutto l'universo,
sacrificio dell'amore.
\endverse
\beginverse*
\chordsoff
Il tuo Corpo è tempio
della lode della Chiesa,
dal costato tu l'hai generata,
nel tuo Sangue l'hai redenta.
\endverse






\beginverse
\chordsoff
Vero Corpo di Cristo
tratto da Maria Vergine,
dal tuo fianco doni a noi la grazia, 
per mandarci tra le genti.
\endverse
\beginverse*
\chordsoff
Dai confini del mondo,
da ogni tempo e ogni luogo
il creato a te renda grazie,
per l'eternità ti adori.
\endverse






\beginverse
\chordsoff
A te Padre la lode,
che donasti il Redentore,
e al Santo Spirito di vita 
sia per sempre onore e gloria. 
\endverse



%%%%%% EV. FINALE

\beginchorus %oppure \beginverse*
\vspace*{1.3\versesep}
\textnote{\textbf{Finale} } %<<< EV. INDICAZIONI

\[B-]\[A]A\[D]men.

\endchorus  %oppure \endverse




\endsong
%------------------------------------------------------------
%			FINE CANZONE
%------------------------------------------------------------



%titolo{Verbum panis}
%autore{Casucci, Balduzzi}
%album{Verbum panis}
%tonalita{Mi-}
%famiglia{Liturgica}
%gruppo{}
%momenti{Comunione}
%identificatore{verbum_panis}
%data_revisione{2011_12_31}
%trascrittore{Francesco Endrici}
\beginsong{Verbum panis}[by={Casucci, Balduzzi}]
\ifchorded
\beginverse*
\vspace*{-0.8\versesep}
{\nolyrics \[E-]\[(6x)]\[\vline]\[C7+]\[D]\[\vline]\[E-]}
\vspace*{-\versesep}
\endverse
\fi
\beginverse
\memorize
\[E-]Prima del \[D]tempo
prima an\[E-]cora che la \[D]terra
comin\[E-]ciasse a vive\[D]re 
\[E-]il Verbo \[D]era presso \[E-]Dio. \[D]\[E-]\[D]
\[E-]Venne nel \[D]mondo
e per \[E-]non abbando\[D]narci
in questo \[E-]viaggio ci la\[D]sciò
\[E-]tutto sé \[D]stesso come \[E-]pane. \[D] \[E-]
\endverse
\beginchorus
Verbum \[E-]caro factum est \[E-]
Verbum \[E-]panis factum est \[E-]
Verbum \[E-]caro factum est \[E-]
Verbum \[E-]panis factum \[C7+]est. \[C7+]\[D4]\[D]
\endchorus
\beginverse
\[G]Qui \[D]spezzi ancora il \[C]pane in mezzo a \[D]noi
e chi\[G]unque mange\[D]rà \[C]non avrà più \[D]fame.
\[G]Qui \[D]vive la tua \[C]chiesa intorno a \[D]te
dove o\[G]gnuno trove\[D]rà \brk \[C]la sua vera \[D]casa. \[E-]
\endverse
\beginchorus
Verbum \[E-]caro factum est \[E-]
Verbum \[E-]panis factum est \[E-]
Verbum \[E-]caro factum est \[E-]
Verbum \[E-]panis 
\endchorus
\beginverse
^Prima del ^tempo
quando l'^universo ^fu creato
^dall'oscuri^tà
^il Verbo ^era presso ^Dio. ^^^
^Venne nel ^mondo
nella ^sua miseri^cordia
Dio ha man^dato il Figlio ^suo
^tutto sé ^stesso come ^pane. ^ ^
\endverse
\beginchorus
Verbum \[E-]caro factum est \[E-]
Verbum \[E-]panis factum est \[E-]
Verbum \[E-]caro factum est \[E-]
Verbum \[E-]panis factum \[C7+]est. \[C7+]\[D4]\[D]
\endchorus
\beginverse
\[G]Qui \[D]spezzi ancora il \[C]pane in mezzo a \[D]noi
e chi\[G]unque mange\[D]rà \[C]non avrà più \[D]fame.
\[G]Qui \[D]vive la tua \[C]chiesa intorno a \[D]te
dove o\[G]gnuno trove\[D]rà \brk \[C]la sua vera \[D]casa. \[E-]
\endverse
\beginchorus
Verbum \[E-]caro factum est \[E-]
Verbum \[E-]panis factum est \[E-]
Verbum \[E-]caro factum est \[E-]
Verbum \[E-]panis factum \[E-]est. 
\endchorus
\endsong


%-------------------------------------------------------------
%			INIZIO	CANZONE
%-------------------------------------------------------------


%titolo: 	Beato il cuore GMG 2016
%autore: Jakub Blycharz
%tonalita: 	Fa 



%%%%%% TITOLO E IMPOSTAZONI
\beginsong{Beato il cuore }[ by={Inno GMG Cracovia 2016 — J. Blycharz}, ititle={Inno GMG Cracovia 2016}]
\transpose{-2} 						% <<< TRASPOSIZIONE #TONI (0 nullo)
\momenti{Comunione; Fine; Ingresso}							% <<< INSERISCI MOMENTI	
% momenti vanno separati da ; e vanno scelti tra:
% Ingresso; Atto penitenziale; Acclamazione al Vangelo; Dopo il Vangelo; Offertorio; Comunione; Ringraziamento; Fine; Santi; Pasqua; Avvento; Natale; Quaresima; Canti Mariani; Battesimo; Prima Comunione; Cresima; Matrimonio; Meditazione;
\ifchorded
	\textnote{$\bigstar$ Tonalità migliore}	% <<< EV COMMENTI (tonalità originale/migliore)
\fi


%%%%%% INTRODUZIONE
\ifchorded
\vspace*{\versesep}
\musicnote{
\begin{minipage}{0.48\textwidth}
\textbf{Intro}
\hfill 
%( \eighthnote \, 80)   % <<  MODIFICA IL TEMPO
% Metronomo: \eighthnote (ottavo) \quarternote (quarto) \halfnote (due quarti)
\end{minipage}
} 	
\vspace*{-\versesep}
\beginverse*

\nolyrics

%---- Prima riga -----------------------------
\vspace*{-\versesep}
\[C#-]  \[A]	\[E]  % \[*D] per indicare le pennate, \rep{2} le ripetizioni

%---- Ogni riga successiva -------------------
\vspace*{-\versesep}
\[B] \[(B)] \[(B)]  \[C#-]

%---- Ev Indicazioni -------------------------			
%\textnote{\textit{(Oppure tutta la strofa)} }	

\endverse
\fi




%%%%% STROFA
\beginverse		%Oppure \beginverse* se non si vuole il numero di fianco
\memorize 		% <<< DECOMMENTA se si vuole utilizzarne la funzione
%\chordsoff		& <<< DECOMMENTA se vuoi una strofa senza accordi

\[C#-]Sei sceso \[A]dalla tua immensi\[E]tà
\[D]in nostro a\[A]iu\[E]to.
Miseri\[B]cordia  scorre  da \[F#]te
\[A]sopra \[B]tutti \[C#]noi.


\endverse


%%%%% STROFA
\beginverse*	%Oppure \beginverse* se non si vuole il numero di fianco
%\memorize 		% <<< DECOMMENTA se si vuole utilizzarne la funzione
%\chordsoff		& <<< DECOMMENTA se vuoi una strofa senza accordi

^ Persi in un ^mondo d’oscuri^tà
^lì Tu ci ^tro^vi.
Nelle tue ^braccia ci stringi e ^poi
^dai la ^vita per ^noi.


\endverse



%%%%% RITORNELLO
\beginchorus
\textnote{\textbf{Rit.}}

Beato è il \[E]cuo\[B]re che per\[C#-]do\[A]na!
Miseri\[E]cordia riceve\[B]rà da Dio in ci\[F#]elo! \rep{2}

\endchorus



%%%%% STROFA
\beginverse		%Oppure \beginverse* se non si vuole il numero di fianco
%\memorize 		% <<< DECOMMENTA se si vuole utilizzarne la funzione
%\chordsoff		% <<< DECOMMENTA se vuoi una strofa senza accordi

^ Solo il per^dono riporte^rà
^pace nel ^mon^do.
Solo il per^dono ci svele^rà
^come f^igli t^uoi.

\endverse



%%%%% RITORNELLO
\beginchorus
\textnote{\textbf{Rit.}}

Beato è il \[E]cuo\[B]re che per\[C#-]do\[A]na!
Miseri\[E]cordia riceve\[B]rà da Dio in ci\[F#]elo! \rep{2}

\endchorus




%%%%% STROFA
\beginverse		%Oppure \beginverse* se non si vuole il numero di fianco
%\memorize 		% <<< DECOMMENTA se si vuole utilizzarne la funzione
%\chordsoff		% <<< DECOMMENTA se vuoi una strofa senza accordi

^ Col sangue in ^croce hai pagato ^Tu
^le nostre ^pover^tà.
Se noi ci am^iamo e restiamo in^ te
^il mondo ^crede^rà!

\endverse



%%%%% RITORNELLO
\beginchorus
\textnote{\textbf{Rit.}}

Beato è il \[E]cuo\[B]re che per\[C#-]do\[A]na!
Miseri\[E]cordia riceve\[B]rà da Dio in ci\[F#]elo! \rep{2}

\endchorus




%%%%% BRIDGE
\beginverse*		%Oppure \beginverse* se non si vuole il numero di fianco
%\memorize 		% <<< DECOMMENTA se si vuole utilizzarne la funzione
%\chordsoff		% <<< DECOMMENTA se vuoi una strofa senza accordi
\textnote{\textbf{Bridge}}
\[A]Le nostre an\[B]gosce ed ansie\[C#-]tà
get\[A]tiamo ogni \[B]attimo in \[A]te.
Amore \[B]che non abbandona \[C#-]mai,
\[A]vivi in \[B]mezzo a \[C#]noi!

\endverse



%%%%% RITORNELLO
\beginchorus
\textnote{\textbf{Rit.}}

Beato è il \[A]cuo\[E]re che per\[F#-]do\[D]na!
Miseri\[A]cordia riceve\[E]rà da Dio in ci\[B]elo! \rep{4}

\endchorus



%%%%%% EV. INTERMEZZO
\beginverse*
\vspace*{1.3\versesep}
{
	\nolyrics
	\musicnote{Chiusura strumentale}
	
	\ifchorded

	%---- Prima riga -----------------------------
	\vspace*{-\versesep}
	\[C#-]  \[A]	\[E]  % \[*D] per indicare le pennate, \rep{2} le ripetizioni


	%---- Ogni riga successiva -------------------
	\vspace*{-\versesep}
	\[B] \[(B)] \[(B)]  \[C#-]


	\fi
	%---- Ev Indicazioni -------------------------			
	%\textnote{\textit{(ripetizione della strofa)}} 
	 
}
\vspace*{\versesep}
\endverse


\endsong
%------------------------------------------------------------
%			FINE CANZONE
%------------------------------------------------------------




%++++++++++++++++++++++++++++++++++++++++++++++++++++++++++++
%			CANZONE TRASPOSTA
%++++++++++++++++++++++++++++++++++++++++++++++++++++++++++++
\ifchorded
%decremento contatore per avere stesso numero
\addtocounter{songnum}{-1} 
\beginsong{Beato il cuore }[ by={Inno GMG Cracovia 2016 — J. Blycharz}]	% <<< COPIA TITOLO E AUTORE
\transpose{0} 						% <<< TRASPOSIZIONE #TONI + - (0 nullo)
%\preferflats SE VOGLIO FORZARE i bemolle come alterazioni
%\prefersharps SE VOGLIO FORZARE i # come alterazioni
\ifchorded
	\textnote{$\lozenge$ Tonalità originale}	% <<< EV COMMENTI (tonalità originale/migliore)
\fi


%%%%%% INTRODUZIONE
\ifchorded
\vspace*{\versesep}
\musicnote{
\begin{minipage}{0.48\textwidth}
\textbf{Intro}
\hfill 
%( \eighthnote \, 80)   % <<  MODIFICA IL TEMPO
% Metronomo: \eighthnote (ottavo) \quarternote (quarto) \halfnote (due quarti)
\end{minipage}
} 	
\vspace*{-\versesep}
\beginverse*

\nolyrics

%---- Prima riga -----------------------------
\vspace*{-\versesep}
\[C#-]  \[A]	\[E]  % \[*D] per indicare le pennate, \rep{2} le ripetizioni

%---- Ogni riga successiva -------------------
\vspace*{-\versesep}
\[B] \[(B)] \[(B)]  \[C#-]

%---- Ev Indicazioni -------------------------			
%\textnote{\textit{(Oppure tutta la strofa)} }	

\endverse
\fi




%%%%% STROFA
\beginverse		%Oppure \beginverse* se non si vuole il numero di fianco
\memorize 		% <<< DECOMMENTA se si vuole utilizzarne la funzione
%\chordsoff		& <<< DECOMMENTA se vuoi una strofa senza accordi

\[C#-]Sei sceso \[A]dalla tua immensi\[E]tà
\[D]in nostro a\[A]iu\[E]to.
Miseri\[B]cordia  scorre  da \[F#]te
\[A]sopra \[B]tutti \[C#]noi.


\endverse


%%%%% STROFA
\beginverse*	%Oppure \beginverse* se non si vuole il numero di fianco
%\memorize 		% <<< DECOMMENTA se si vuole utilizzarne la funzione
%\chordsoff		& <<< DECOMMENTA se vuoi una strofa senza accordi

^ Persi in un ^mondo d’oscuri^tà
^lì Tu ci ^tro^vi.
Nelle tue ^braccia ci stringi e ^poi
^dai la ^vita per ^noi.


\endverse



%%%%% RITORNELLO
\beginchorus
\textnote{\textbf{Rit.}}

Beato è il \[E]cuo\[B]re che per\[C#-]do\[A]na!
Miseri\[E]cordia riceve\[B]rà da Dio in ci\[F#]elo! \rep{2}

\endchorus



%%%%% STROFA
\beginverse		%Oppure \beginverse* se non si vuole il numero di fianco
%\memorize 		% <<< DECOMMENTA se si vuole utilizzarne la funzione
%\chordsoff		% <<< DECOMMENTA se vuoi una strofa senza accordi

^ Solo il per^dono riporte^rà
^pace nel ^mon^do.
Solo il per^dono ci svele^rà
^come f^igli t^uoi.

\endverse



%%%%% RITORNELLO
\beginchorus
\textnote{\textbf{Rit.}}

Beato è il \[E]cuo\[B]re che per\[C#-]do\[A]na!
Miseri\[E]cordia riceve\[B]rà da Dio in ci\[F#]elo! \rep{2}

\endchorus




%%%%% STROFA
\beginverse		%Oppure \beginverse* se non si vuole il numero di fianco
%\memorize 		% <<< DECOMMENTA se si vuole utilizzarne la funzione
%\chordsoff		% <<< DECOMMENTA se vuoi una strofa senza accordi

^ Col sangue in ^croce hai pagato ^Tu
^le nostre ^pover^tà.
Se noi ci am^iamo e restiamo in^ te
^il mondo ^crede^rà!

\endverse



%%%%% RITORNELLO
\beginchorus
\textnote{\textbf{Rit.}}

Beato è il \[E]cuo\[B]re che per\[C#-]do\[A]na!
Miseri\[E]cordia riceve\[B]rà da Dio in ci\[F#]elo! \rep{2}

\endchorus




%%%%% BRIDGE
\beginverse*		%Oppure \beginverse* se non si vuole il numero di fianco
%\memorize 		% <<< DECOMMENTA se si vuole utilizzarne la funzione
%\chordsoff		% <<< DECOMMENTA se vuoi una strofa senza accordi
\textnote{\textbf{Bridge}}
\[A]Le nostre an\[B]gosce ed ansie\[C#-]tà
get\[A]tiamo ogni \[B]attimo in \[A]te.
Amore \[B]che non abbandona \[C#-]mai,
\[A]vivi in \[B]mezzo a \[C#]noi!

\endverse



%%%%% RITORNELLO
\beginchorus
\textnote{\textbf{Rit.}}

Beato è il \[A]cuo\[E]re che per\[F#-]do\[D]na!
Miseri\[A]cordia riceve\[E]rà da Dio in ci\[B]elo! \rep{4}

\endchorus



%%%%%% EV. INTERMEZZO
\beginverse*
\vspace*{1.3\versesep}
{
	\nolyrics
	\musicnote{Chiusura strumentale}
	
	\ifchorded

	%---- Prima riga -----------------------------
	\vspace*{-\versesep}
	\[C#-]  \[A]	\[E]  % \[*D] per indicare le pennate, \rep{2} le ripetizioni


	%---- Ogni riga successiva -------------------
	\vspace*{-\versesep}
	\[B] \[(B)] \[(B)]  \[C#-]


	\fi
	%---- Ev Indicazioni -------------------------			
	%\textnote{\textit{(ripetizione della strofa)}} 
	 
}
\vspace*{\versesep}
\endverse


\endsong


\fi
%++++++++++++++++++++++++++++++++++++++++++++++++++++++++++++
%			FINE CANZONE TRASPOSTA
%++++++++++++++++++++++++++++++++++++++++++++++++++++++++++++

%missing: come un canto d'amore

%-------------------------------------------------------------
%			INIZIO	CANZONE
%-------------------------------------------------------------


%titolo: 	Ave Maria
%autore: 	Casucci, Balduzzi
%tonalita: 	Re 


%%%%%% TITOLO E IMPOSTAZONI
\beginsong{Ave Maria}[by={C. Casucci, M. Balduzzi}] 	% <<< MODIFICA TITOLO E AUTORE
\transpose{0} 						% <<< TRASPOSIZIONE #TONI (0 nullo)
\momenti{Canti Mariani; Ringraziamento}							% <<< INSERISCI MOMENTI	
% momenti vanno separati da ; e vanno scelti tra:
% Ingresso; Atto penitenziale; Acclamazione al Vangelo; Dopo il Vangelo; Offertorio; Comunione; Ringraziamento; Fine; Santi; Pasqua; Avvento; Natale; Quaresima; Canti Mariani; Battesimo; Prima Comunione; Cresima; Matrimonio; Meditazione;
\ifchorded
	%\textnote{Tonalità originale }	% <<< EV COMMENTI (tonalità originale/migliore)
\fi


%%%%%% INTRODUZIONE
\ifchorded
\vspace*{\versesep}
\musicnote{
\begin{minipage}{0.48\textwidth}
\textbf{Intro}
\hfill 
(\quarternote \, 72)
%( \eighthnote \, 80)   % <<  MODIFICA IL TEMPO
% Metronomo: \eighthnote (ottavo) \quarternote (quarto) \halfnote (due quarti)
\end{minipage}
} 
\musicnote{\textit{(dolce, arpeggiato)}}	
\vspace*{-\versesep}
\beginverse*


\nolyrics

%---- Prima riga -----------------------------
\vspace*{-\versesep}
\[D]\[A]\[B-]\[G]	 % \[*D] per indicare le pennate, \rep{2} le ripetizioni

%---- Ogni riga successiva -------------------
\vspace*{-\versesep}
\[D]\[A]\[E-] \[G]

%---- Ev Indicazioni -------------------------			
\textnote{\textit{(come metà ritornello)} }	

\endverse
\fi




%%%%% RITORNELLO
\textnote{\textbf{Rit.}}
\beginchorus

\[D]A\[A]ve Ma\[B-]ria, \[G] 
\[D]\[A]a\[E-]ve, \[G]
\[D]a\[A]ve Ma\[B-]ria, \[G] 
\[D]\[A]a\[D4]ve. \[D]

\endchorus



%%%%% STROFA
\beginverse
\memorize
\[D]Donna dell'at\[D]tesa e \[B-]madre di spe\[B-]ranza
\[A]ora pro no\[G]bis.
\[D]Donna del sor\[D]riso e \[B-]madre del si\[B-]lenzio
\[A]ora pro no\[G]bis.
\[D]Donna di fron\[D]tiera e \[A]madre dell'ar\[A]dore
\[B-]ora pro no\[G]bis.
\[D]Donna del ri\[D]poso e \[A]madre del sen\[A]tiero
\[G]ora pro no\[A]bis.
\endverse




%%%%% STROFA
\beginverse
^Donna del de^serto e ^madre del re^spiro
^ora pro no^bis.
^Donna della ^sera e ^madre del ri^cordo
^ora pro no^bis.
^Donna del pre^sente e ^madre del ri^torno
^ora pro no^bis.
^Donna della ^terra e ^madre dell'a^more
^ora pro no^bis.
\endverse


\endsong
%------------------------------------------------------------
%			FINE CANZONE
%------------------------------------------------------------




%titolo{Danza la vita}
%autore{}
%album{}
%tonalita{Re}
%famiglia{Liturgica}
%gruppo{}
%momenti{Congedo}
%identificatore{danza_la_vita}
%data_revisione{2012_11_05}
%trascrittore{Francesco Endrici}
\beginsong{Danza la vita}[]
\beginverse
\[D]Canta con la \[G]voce e con il \[D]cuore, \[G]
\[D]con la bocca e \[G]con la vita, \[D] \[G]
\[D]canta senza \[G]stonature, \[D] \[G]
la \[D]verità \[G] del \[D]cuore. \[G]
\[D]Canta come \[G]cantano i viandanti 
\echo{\[D]canta come \[G]cantano i viandanti}
non \[D]solo per riem\[G]pire il tempo, 
\echo{non \[D]solo per \[G]riempire il tempo,}
\[D]Ma per soste\[G]nere lo sforzo 
\echo{\[D]Ma per soste\[G]nere lo sforzo.}
\[D]Canta \[G] e cam\[D]mina \[G]
\[D]Canta \[G] e cam\[D]mina \[G]
Se \[A]poi, credi non possa ba\[B-]stare
segui il \[E]tempo, stai \[G]pron\[A]to e
\endverse

\beginchorus
\[D]Danza la \[G]vita, al \[A]ritmo dello \[D]Spirito. 
\qquad \quad \echo{Spirito che riempi i nostri }
\[B-]Danza, \[G]danza al \[A]ritmo che c'è in \[D]te. 
\echo{cuor, danza assieme a noi. Danza}
\[G]Spirito \[A]che \[D]riempi i nostri 
\echo{la vita al ritmo dello Spirito}
\[B-]cuor. \[G]Danza assieme a \[A]no\[D]i. 
\echo{Danza, danza al ritmo che c'è in te.}
\endchorus
\beginverse
\chordsoff
Cam^mina sulle ^orme del Si^gnore, ^
non ^solo con i ^piedi ^ma ^ \brk ^usa soprat^tutto il cuore.^^
^Ama ^ chi è con ^te. ^
Cam^mina con lo ^zaino sulle spalle 
\echo{Cam^mina con lo ^zaino sulle spalle}
^la fatica a^iuta a crescere 
\echo{^la fatica a^iuta a crescere}
^nella con^divisione 
\echo{^nella con^divisione.}
^Canta ^ e cam^mina, ^
^canta  ^ e cam^mina. ^
Se ^poi, credi non possa ba^stare
segui il ^tempo, stai ^pron^to e
\endverse
\beginchorus
\chordsoff 
Rit. 
\endchorus
\endsong







% *  *  *  *  *  *  *  *  *  *  *  *  *  *  *  *  *  *






\end{songs}




%\ifcanzsingole
%	\relax
%\else
%	\iftitleindex
%		\ifxetex
%		\printindex[alfabetico]
%		\else
%		\printindex
%		\fi
%	\else
%	\fi
%	\ifauthorsindex
%	\printindex[autori]
%	\else
%	\fi
%	\iftematicindex
%	\printindex[tematico]
%	\else
%	\fi
%	\ifcover
%		\relax
%	\else
%		\colophon
%	\fi
%\fi
\end{document}
