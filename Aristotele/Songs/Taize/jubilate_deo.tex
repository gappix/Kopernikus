%-------------------------------------------------------------
%			INIZIO	CANZONE
%-------------------------------------------------------------


%titolo: 	Jubilate Deo
%autore: 	Taizè
%tonalita: 	La-



%%%%%% TITOLO E IMPOSTAZONI
\beginsong{Jubilate Deo}[by={Taizè}]	% <<< MODIFICA TITOLO E AUTORE
\transpose{-2} 						% <<< TRASPOSIZIONE #TONI (0 nullo)
\momenti{Ringraziamento; Acclamazione al Vangelo; Meditazione}							% <<< INSERISCI MOMENTI	
% momenti vanno separati da ; e vanno scelti tra:
% Ingresso; Atto penitenziale; Acclamazione al Vangelo; Dopo il Vangelo; Offertorio; Comunione; Ringraziamento; Fine; Santi; Pasqua; Avvento; Natale; Quaresima; Canti Mariani; Battesimo; Prima Comunione; Cresima; Matrimonio; Meditazione;
\ifchorded
	\textnote{Tonalità migliore per le bambine }	% <<< EV COMMENTI (tonalità originale/migliore)
\fi


%%%%%% INTRODUZIONE
\ifchorded
\vspace*{\versesep}
\textnote{Intro: \qquad \qquad  }%(\eighthnote 116) % << MODIFICA IL TEMPO
% Metronomo: \eighthnote (ottavo) \quarternote (quarto) \halfnote (due quarti)
\vspace*{-\versesep}
\beginverse*

\nolyrics

%---- Prima riga -----------------------------
\vspace*{-\versesep}
\[D] \[G] \[D] \rep{2}

%---- Ogni riga successiva -------------------
%\vspace*{-\versesep}
%\[G] \[C]  \[D]	

%---- Ev Indicazioni -------------------------			
\textnote{\textit{(Oppure tutta la strofa)} }	

\endverse
\fi



%%%%% STROFA
\beginverse*
\[D]Jubi\[G]late \[D]Deo \[(D)]omnis \[G]ter\[D]ra, 
\[(D)]servi\[G]te \[D]Domi\[G]no 
\[D]in lae\[A]titi\[D]a.
\endverse


\beginchorus
\[D]Alle\[G]luia, \[D]alle\[G]luia, \[D]in lae\[A]titi\[D]a. 
\[D]Alle\[G]luia, \[D]alle\[G]luia, \[D]in lae\[A]titi\[D]a. 
\endchorus

\endsong
%------------------------------------------------------------
%			FINE CANZONE
%------------------------------------------------------------




%++++++++++++++++++++++++++++++++++++++++++++++++++++++++++++
%			CANZONE TRASPOSTA
%++++++++++++++++++++++++++++++++++++++++++++++++++++++++++++
\ifchorded
%decremento contatore per avere stesso numero
\addtocounter{songnum}{-1} 
\beginsong{Jubilate Deo}[by={Taizè}]	% <<< MODIFICA TITOLO E AUTORE	% <<< COPIA TITOLO E AUTORE
\transpose{0} 						% <<< TRASPOSIZIONE #TONI + - (0 nullo)
%\preferflats  %SE VOGLIO FORZARE i bemolle come alterazioni
%\prefersharps %SE VOGLIO FORZARE i # come alterazioni
\ifchorded
	\textnote{Tonalità originale}	% <<< EV COMMENTI (tonalità originale/migliore)
\fi


%%%%%% INTRODUZIONE
\ifchorded
\vspace*{\versesep}
\textnote{Intro: \qquad \qquad  }%(\eighthnote 116) % << MODIFICA IL TEMPO
% Metronomo: \eighthnote (ottavo) \quarternote (quarto) \halfnote (due quarti)
\vspace*{-\versesep}
\beginverse*

\nolyrics

%---- Prima riga -----------------------------
\vspace*{-\versesep}
\[D] \[G] \[D] \rep{2}

%---- Ogni riga successiva -------------------
%\vspace*{-\versesep}
%\[G] \[C]  \[D]	

%---- Ev Indicazioni -------------------------			
\textnote{\textit{(Oppure tutta la strofa)} }	

\endverse
\fi



%%%%% STROFA
\beginverse*
\[D]Jubi\[G]late \[D]Deo \[(D)]omnis \[G]ter\[D]ra, 
\[(D)]servi\[G]te \[D]Domi\[G]no 
\[D]in lae\[A]titi\[D]a.
\endverse


\beginchorus
\[D]Alle\[G]luia, \[D]alle\[G]luia, \[D]in lae\[A]titi\[D]a. 
\[D]Alle\[G]luia, \[D]alle\[G]luia, \[D]in lae\[A]titi\[D]a. 
\endchorus

\endsong


\fi
%++++++++++++++++++++++++++++++++++++++++++++++++++++++++++++
%			FINE CANZONE TRASPOSTA
%++++++++++++++++++++++++++++++++++++++++++++++++++++++++++++
