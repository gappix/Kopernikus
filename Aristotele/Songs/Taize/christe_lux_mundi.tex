%-------------------------------------------------------------
%			INIZIO	CANZONE
%-------------------------------------------------------------

%titolo: Christe lux mundi
%autore: Taizè
%tonalita: Rem


%%%%%% TITOLO E IMPOSTAZONI
\beginsong{Christe lux mundi}[by={Taizé}] 	% <<< MODIFICA TITOLO E AUTORE
\transpose{0} 						% <<< TRASPOSIZIONE #TONI (0 nullo)
\momenti{Acclamazione al Vangelo; Meditazione}							% <<< INSERISCI MOMENTI	
% momenti vanno separati da ; e vanno scelti tra:
% Ingresso; Atto penitenziale; Acclamazione al Vangelo; Dopo il Vangelo; Offertorio; Comunione; Ringraziamento; Fine; Santi; Pasqua; Avvento; Natale; Quaresima; Canti Mariani; Battesimo; Prima Comunione; Cresima; Matrimonio; Meditazione;


%%%%%% INTRODUZIONE
\ifchorded
\vspace*{\versesep}
\textnote{Intro: \qquad \qquad  }%(\eighthnote 116) % << MODIFICA IL TEMPO(\eighthnote 88) 
% Metronomo: \eighthnote (ottavo) \quarternote (quarto) \halfnote (due quarti)
\vspace*{-\versesep}
\beginverse*

\nolyrics

%---- Prima riga -----------------------------
\vspace*{-\versesep}
\[D-] 	 % \[*D] per indicare le pennate, \rep{2} le ripetizioni

%---- Ogni riga successiva -------------------
%\vspace*{-\versesep}
%\[G] \[C]  \[D]	

%---- Ev Indicazioni -------------------------			
\textnote{\textit{(Oppure tutta la strofa)} }	

\endverse
\fi




%%%%% STROFA
\beginverse*
\[D-]Christe, lux \[A-]mun\[D-]di, 
\[D-]qui sequi\[G-]tur\[A-]te, 
ha\[F]bebit \[(*D-)]lumen \[C]vitae, \quad \[G-]lumen \[D-]vi\[C]tae...
\endverse



\endsong
%------------------------------------------------------------
%			FINE CANZONE
%------------------------------------------------------------