%-------------------------------------------------------------
%			INIZIO	CANZONE
%-------------------------------------------------------------


%titolo: 	Il Signore è la mia forza
%autore: 	Taizè
%tonalita: 	Re 



%%%%%% TITOLO E IMPOSTAZONI
\beginsong{Il Signore è la mia forza}[by={Taizè}] 	% <<< MODIFICA TITOLO E AUTORE
%\transpose{-3} 						% <<< TRASPOSIZIONE #TONI (0 nullo)
\momenti{Dopo il Vangelo; Meditazione}							% <<< INSERISCI MOMENTI	
% momenti vanno separati da ; e vanno scelti tra:
% Ingresso; Atto penitenziale; Acclamazione al Vangelo; Dopo il Vangelo; Offertorio; Comunione; Ringraziamento; Fine; Santi; Pasqua; Avvento; Natale; Quaresima; Canti Mariani; Battesimo; Prima Comunione; Cresima; Matrimonio; Meditazione;
\ifchorded
	%\textnote{Tonalità originale }	% <<< EV COMMENTI (tonalità originale/migliore)
\fi


%%%%%% INTRODUZIONE
\ifchorded
\vspace*{\versesep}
\textnote{Intro: \qquad \qquad  }%(\eighthnote 116) % << MODIFICA IL TEMPO
% Metronomo: \eighthnote (ottavo) \quarternote (quarto) \halfnote (due quarti)
\vspace*{-\versesep}
\beginverse*

\nolyrics

%---- Prima riga -----------------------------
\vspace*{-\versesep}
\[F]

%---- Ogni riga successiva -------------------
%\vspace*{-\versesep}
%\[G] \[C]  \[D]	

%---- Ev Indicazioni -------------------------			
\textnote{\textit{(oppure tutta la strofa)} }	

\endverse
\fi


\beginverse*
\[D-*]Il \[(C*)]Si\[F]gnore è la mia \[B&*]for\[C]za 
 \[D-*]ed \[(C*)]io \[F]spero in \[C]lui.
Il Si\[B&]gnore è il \[A]Salva\[D-]tor.
In \[C]lui con\[F]fido, non \[B&]ho ti\[C]mor, 
in \[A-]lui con\[D-]fido non \[B&*]ho \[(C*)]ti\[F]mor.
\endverse
\endsong
%------------------------------------------------------------
%			FINE CANZONE
%------------------------------------------------------------


