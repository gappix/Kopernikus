%-------------------------------------------------------------
%			INIZIO	CANZONE
%-------------------------------------------------------------


%titolo: 	In manus tuas Pater
%autore: 	Taizè
%tonalita: 	Re



%%%%%% TITOLO E IMPOSTAZONI
\beginsong{In manus tuas Pater}[by={Taizè}]	% <<< MODIFICA TITOLO E AUTORE
%\transpose{-2} 						% <<< TRASPOSIZIONE #TONI (0 nullo)
\momenti{ Meditazione}							% <<< INSERISCI MOMENTI	
% momenti vanno separati da ; e vanno scelti tra:
% Ingresso; Atto penitenziale; Acclamazione al Vangelo; Dopo il Vangelo; Offertorio; Comunione; Ringraziamento; Fine; Santi; Pasqua; Avvento; Natale; Quaresima; Canti Mariani; Battesimo; Prima Comunione; Cresima; Matrimonio; Meditazione;
\ifchorded
	%\textnote{Tonalità originale }	% <<< EV COMMENTI (tonalità originale\migliore)
\fi


%%%%%% INTRODUZIONE
\ifchorded
\vspace*{\versesep}
\musicnote{
\begin{minipage}{0.48\textwidth}
\textbf{Intro}
\hfill 
%( \eighthnote \, 80)   % <<  MODIFICA IL TEMPO
% Metronomo: \eighthnote (ottavo) \quarternote (quarto) \halfnote (due quarti)
\end{minipage}
} 	
\vspace*{-\versesep}
\beginverse*

\nolyrics

%---- Prima riga -----------------------------
\vspace*{-\versesep}
\[D] \[F#-] \[E-] \[D]

%---- Ogni riga successiva -------------------
%\vspace*{-\versesep}
%\[G] \[C]  \[D]	

%---- Ev Indicazioni -------------------------			
\textnote{\textit{(oppure tutta la strofa)} }	

\endverse
\fi

\beginverse*
In \[D]manus \[F#-]tuas, \[E-]Pat\[D]er 
co\[G]mmendo spiritum \[F#]meum
In \[B-]manus \[F#]tuas \[B-*]P\[A*]at\[D]er 
co\[G]mmendo \[A]spiritum \[D]meum
\endverse
\endsong

