%-------------------------------------------------------------
%			INIZIO	CANZONE
%-------------------------------------------------------------


%titolo: 	Bless the Lord my soul
%autore: 	Taizé
%tonalita: 	Si- 




%titolo{Bless the lord my soul}
%autore{Taizé}
%album{}
%tonalita{Si-}
%famiglia{Liturgica}
%gruppo{Canoni_ritornelli}
%momenti{Ritornelli;Taizé}
%identificatore{bless_the_lord}
%data_revisione{2011_12_31}
%trascrittore{Francesco Endrici - Manuel Toniato}
\beginsong{Bless the lord my soul}[by={Taizé}]
\transpose{0} 						% <<< TRASPOSIZIONE #TONI (0 nullo)
\momenti{Dopo il Vangelo; Meditazione}				% <<< INSERISCI MOMENTI	
\ifchorded
	%\textnote{Tonalità originale }	% <<< EV COMMENTI (tonalità originale/migliore)
\fi


%%%%%% INTRODUZIONE
\ifchorded
\vspace*{\versesep}
\textnote{Intro: \qquad \qquad  }%(\eighthnote 116) % << MODIFICA IL TEMPO
% Metronomo: \eighthnote (ottavo) \quarternote (quarto) \halfnote (due quarti)
\vspace*{-\versesep}
\beginverse*

\nolyrics

%---- Prima riga -----------------------------
\vspace*{-\versesep}
\[B-] \[E]  \[B-]	 % \[*D] per indicare le pennate, \rep{2} le ripetizioni


%---- Ev Indicazioni -------------------------			
\textnote{\textit{(Oppure tutta la strofa)} }	

\endverse
\fi



%%%%% STROFA
\beginverse*
\[B-]Bless the \[E]Lord, my \[B-]soul, \brk and \[G]bless God's \[A]holy \[D]name. \[F#] 
\[B-]Bless the \[E]Lord, my \[B-]soul, \brk who \[G]leads \[A7]me into \[B-]life.
\endverse
\endsong

