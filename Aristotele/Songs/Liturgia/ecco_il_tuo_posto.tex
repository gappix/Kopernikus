%-------------------------------------------------------------
%			INIZIO	CANZONE
%-------------------------------------------------------------


%titolo: 	Ecco il tuo post
%autore: 	Giombini
%tonalita: 	Do



%%%%%% TITOLO E IMPOSTAZONI
\beginsong{Ecco il tuo posto}[by={M. Giombini}] 	% <<< MODIFICA TITOLO E AUTORE
\transpose{0} 						% <<< TRASPOSIZIONE #TONI (0 nullo)
\momenti{Offertorio; Pasqua}							% <<< INSERISCI MOMENTI	
% momenti vanno separati da ; e vanno scelti tra:
% Ingresso; Atto penitenziale; Acclamazione al Vangelo; Dopo il Vangelo; Offertorio; Comunione; Ringraziamento; Fine; Santi; Pasqua; Avvento; Natale; Quaresima; Canti Mariani; Battesimo; Prima Comunione; Cresima; Matrimonio; Meditazione; Spezzare del pane;
\ifchorded
	%\textnote{Tonalità migliore }	% <<< EV COMMENTI (tonalità originale/migliore)
\fi


%%%%%% INTRODUZIONE
\ifchorded
\vspace*{\versesep}
\musicnote{
\begin{minipage}{0.48\textwidth}
\textbf{Intro}
\hfill 
%( \eighthnote \, 80)   % <<  MODIFICA IL TEMPO
% Metronomo: \eighthnote (ottavo) \quarternote (quarto) \halfnote (due quarti)
\end{minipage}
} 	
\vspace*{-\versesep}
\beginverse*
\nolyrics

%---- Prima riga -----------------------------
\vspace*{-\versesep}
\[C] \[G] \[F] \[C] % \[*D] per indicare le pennate, \rep{2} le ripetizioni


%---- Ev Indicazioni -------------------------			
\textnote{\textit{[oppure tutta la strofa]} }	

\endverse
\fi



%%%%% STROFA
\beginverse		%Oppure \beginverse* se non si vuole il numero di fianco
\memorize 		% <<< DECOMMENTA se si vuole utilizzarne la funzione
\[C]Ecco il tuo \[G]posto, \[F]vie\[C]ni, 
\[C]vieni a se\[F]derti fra \[G]noi
\[D-]e ti rac\[G]conte\[C]re\[A-]mo 
\[F]la nostra \[G]sto\[C]ria.
\endverse


%%%%% RITORNELLO
\beginchorus
\textnote{\textbf{Rit.}}
\[F]Quanto a\[C]more nel \[E]semi\[A-]nare,
\[F]quanta spe\[C]ranza nell'\[G]aspet\[C]tare,
\[A-]quanta fa\[E-]tica nel mi\[F]etere il \[C]grano
\[F]e vendem\[G]mia\[C]re.
\[F]e vendem\[G]mia\[C]re.
\endchorus



%%%%% STROFA
\beginverse		%Oppure \beginverse* se non si vuole il numero di fianco
%\memorize 		% <<< DECOMMENTA se si vuole utilizzarne la funzione
^Accanto al ^fuoco, vi^e^ni, 
v^ieni a scald^arti con ^noi:
^tutti di^vide^re^mo 
^pane e ^vi^no!
\endverse



%%%%% STROFA
\beginverse		%Oppure \beginverse* se non si vuole il numero di fianco
%\memorize 		% <<< DECOMMENTA se si vuole utilizzarne la funzione
^Ti senti^rai più ^fo^rte, 
^vieni, ri^mani con ^noi:
^uniti at^tende^re^mo 
^ogni do^ma^ni!
\endverse



\endsong
%------------------------------------------------------------
%			FINE CANZONE
%------------------------------------------------------------