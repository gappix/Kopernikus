%-------------------------------------------------------------
%			INIZIO	CANZONE
%-------------------------------------------------------------


%titolo: 	Re dei re
%autore: 	E. Munda, G. Pretto, L. Christille
%tonalita: 	Mi- 



%%%%%% TITOLO E IMPOSTAZONI
\beginsong{Re dei re}[by={E. Munda, G. Pretto, L. Christille}] 	% <<< MODIFICA TITOLO E AUTORE
\transpose{0} 						% <<< TRASPOSIZIONE #TONI (0 nullo)
%\preferflats  %SE VOGLIO FORZARE i bemolle come alterazioni
%\prefersharps %SE VOGLIO FORZARE i # come alterazioni
\momenti{Meditazione; Ringraziamento; Comunione; Fine}							% <<< INSERISCI MOMENTI	
% momenti vanno separati da ; e vanno scelti tra:
% Ingresso; Atto penitenziale; Acclamazione al Vangelo; Dopo il Vangelo; Offertorio; Comunione; Ringraziamento; Fine; Santi; Pasqua; Avvento; Natale; Quaresima; Canti Mariani; Battesimo; Prima Comunione; Cresima; Matrimonio; Meditazione; Spezzare del pane;
\ifchorded
	%\textnote{Tonalità migliore }	% <<< EV COMMENTI (tonalità originale/migliore)
\fi


%%%%%% INTRODUZIONE
\ifchorded
\vspace*{\versesep}
\textnote{Intro: \qquad \qquad  (\quarternote 106)} % <<  MODIFICA IL TEMPO
% Metronomo: \eighthnote (ottavo) \quarternote (quarto) \halfnote (due quarti)
\vspace*{-\versesep}
\beginverse*

\nolyrics

%---- Prima riga -----------------------------
\vspace*{-\versesep}
\[E-] \[C] \[G] \[D]	 \rep{2}% \[*D] per indicare le pennate, \rep{2} le ripetizioni

%---- Ogni riga successiva -------------------
%\vspace*{-\versesep}
%\[G] \[C]  \[D]	

%---- Ev Indicazioni -------------------------			
%\textnote{\textit{(Oppure tutta la strofa)} }	

\endverse
\fi




%%%%% STROFA
\beginverse		%Oppure \beginverse* se non si vuole il numero di fianco
\memorize 		% <<< DECOMMENTA se si vuole utilizzarne la funzione
%\chordsoff		% <<< DECOMMENTA se vuoi una strofa senza accordi

\[E-]Hai solle\[C]vato \brk i nostri \[G]volti dalla \[D]polvere, 
\[E-] le nostre \[C]colpe hai por\[G]tato su di \[D]te. 
\[E-] Signore \[C]ti sei fatto \brk \[G]uomo in tutto \[D]come noi 
\[E-]per \[C]a-\[G]mo-\[D]re, 

\endverse
\beginverse*

^Figlio dell’altissimo, ^povero tra i poveri, 
^vieni a dimorare tra ^noi. 
^Dio dell’impossibile, ^Re di tutti i secoli 
^vieni nella tua mae^stà. 

\endverse



%%%%% RITORNELLO
\beginchorus
\textnote{\textbf{Rit.}}

^Re \[(D)]dei ^re 
i ^popoli ti acclamano, \brk i ^cieli ti proclamano 
^Re \[(D)]dei ^re 

^luce degli uomini, \brk ^regna con il tuo amore tra \[E-]no-\[C]o-\[G]o\[D]i 
\[E-]Oo\[C]oh \[G]\[D] \brk \[E-]Oo\[C]oh \[G]\[D]  \brk \[E-]Oo\[C]oh \[G]\[D]  
\[E-]Oh... \[E-] \[E-*]

\endchorus


%%%%%% EV. INTERMEZZO
\beginverse*
\vspace*{1.3\versesep}
{
	\nolyrics
	\textnote{Intermezzo musicale}
	
	\ifchorded

    %---- Prima riga -----------------------------
    \vspace*{-\versesep}
    \[E-] \[C] \[G] \[D]	 \rep{2}% \[*D] per indicare le pennate, \rep{2} le ripetizioni



	\fi
	%---- Ev Indicazioni -------------------------			
	%\textnote{\textit{(ripetizione della strofa)}} 
	 
}
\vspace*{\versesep}
\endverse


%%%%% STROFA
\beginverse		%Oppure \beginverse* se non si vuole il numero di fianco
%\memorize 		% <<< DECOMMENTA se si vuole utilizzarne la funzione
%\chordsoff		% <<< DECOMMENTA se vuoi una strofa senza accordi

^ Ci hai riscat^tati \brk dalla s^tretta delle ^tenebre, 
^ perché po^tessimo glo^rificare ^te. 
^ Hai river^sato in noi \brk la ^vita del tuo ^Spirito 
^per ^a-^mo-^re, 

\endverse
\beginverse*

^Figlio dell’altissimo, ^povero tra i poveri, 
^vieni a dimorare tra ^noi. 
^Dio dell’impossibile, ^Re di tutti i secoli 
^vieni nella tua mae^stà. 

\endverse



%%%%% RITORNELLO
\beginchorus
\textnote{\textbf{Rit.}}

^Re \[(D)]dei ^re 
i ^popoli ti acclamano, \brk i ^cieli ti proclamano 
^Re \[(D)]dei ^re 

^luce degli uomini, \brk ^regna con il tuo amore tra \[E-]no-\[C]o-\[G]o\[D]i 
\[E-]Oo\[C]oh \[G]\[D] \brk \[E-]Oo\[C]oh \[G]\[D]  \brk \[E-]Oo\[C]oh \[G]\[D]  
\[E-]Oh... \[E-] \[E-] \[E-]

\endchorus




%%%%% BRIDGE
\beginverse*		%Oppure \beginverse* se non si vuole il numero di fianco
%\memorize 		% <<< DECOMMENTA se si vuole utilizzarne la funzione
%\chordsoff		% <<< DECOMMENTA se vuoi una strofa senza accordi
\vspace*{1.3\versesep}
\textnote{Bridge} %<<< EV. INDICAZIONI

^Tua ^è la ^glo^ria per ^se-^em^pre, ^ 
^Tua ^è la ^glo^ria per ^se-^em^pre, ^ 
\endverse
\beginverse*
\vspace*{-\versesep}
^glo^ria, ^glo^ria, ^glo^ria, ^glo^ria.  

\endverse
\beginverse*

^Figlio dell’altissimo, ^povero tra i poveri, 
^vieni a dimorare tra ^noi. 
^Dio dell’impossibile, ^Re di tutti i secoli 
^vieni nella tua mae^stà. 

\endverse




%%%%% RITORNELLO
\beginchorus
\textnote{\textbf{Rit.}}

^Re \[(D)]dei ^re 
i ^popoli ti acclamano, \brk i ^cieli ti proclamano 
^Re \[(D)]dei ^re 

^luce degli uomini, \brk ^regna con il tuo amore tra \[E-]no-\[C]o-\[G]o\[D]i 
\[E-]Oo\[C]oh \[G]\[D] \brk \[E-]Oo\[C]oh \[G]\[D]  \brk \[E-]Oo\[C]oh \[G]\[D]  
\[E-]Oh... \[E-] \[E-] \[E-] \[E-*]

\endchorus




\endsong
%------------------------------------------------------------
%			FINE CANZONE
%------------------------------------------------------------



%++++++++++++++++++++++++++++++++++++++++++++++++++++++++++++
%			CANZONE TRASPOSTA
%++++++++++++++++++++++++++++++++++++++++++++++++++++++++++++
\ifchorded
%decremento contatore per avere stesso numero
\addtocounter{songnum}{-1} 
\beginsong{Re dei re}[by={E. Munda, G. Pretto, L. Christille}] 	% <<< MODIFICA TITOLO E AUTORE
\transpose{-2} 							% <<< TRASPOSIZIONE #TONI + - (0 nullo)
\preferflats  %SE VOGLIO FORZARE i bemolle come alterazioni
%\prefersharps %SE VOGLIO FORZARE i # come alterazioni
\ifchorded
	\textnote{Con aumento di tonalità}	% <<< EV COMMENTI (tonalità originale/migliore)
\fi




%%%%%% INTRODUZIONE
\ifchorded
\vspace*{\versesep}
\textnote{Intro: \qquad \qquad  }%(\eighthnote 116) % <<  MODIFICA IL TEMPO
% Metronomo: \eighthnote (ottavo) \quarternote (quarto) \halfnote (due quarti)
\vspace*{-\versesep}
\beginverse*

\nolyrics

%---- Prima riga -----------------------------
\vspace*{-\versesep}
\[E-] \[C] \[G] \[D]	 \rep{2}% \[*D] per indicare le pennate, \rep{2} le ripetizioni

%---- Ogni riga successiva -------------------
%\vspace*{-\versesep}
%\[G] \[C]  \[D]	

%---- Ev Indicazioni -------------------------			
%\textnote{\textit{(Oppure tutta la strofa)} }	

\endverse
\fi




%%%%% STROFA
\beginverse		%Oppure \beginverse* se non si vuole il numero di fianco
\memorize 		% <<< DECOMMENTA se si vuole utilizzarne la funzione
%\chordsoff		% <<< DECOMMENTA se vuoi una strofa senza accordi

\[E-]Hai solle\[C]vato \brk i nostri \[G]volti dalla \[D]polvere, 
\[E-] le nostre \[C]colpe hai por\[G]tato su di \[D]te. 
\[E-] Signore \[C]ti sei fatto \brk \[G]uomo in tutto \[D]come noi 
\[E-]per \[C]a-\[G]mo-\[D]re, 

\endverse
\beginverse*

^Figlio dell’altissimo, ^povero tra i poveri, 
^vieni a dimorare tra ^noi. 
^Dio dell’impossibile, ^Re di tutti i secoli 
^vieni nella tua mae^stà. 

\endverse




%%%%% RITORNELLO
\beginchorus
\textnote{\textbf{Rit.}}

^Re \[(D)]dei ^re 
i ^popoli ti acclamano, \brk i ^cieli ti proclamano 
^Re \[(D)]dei ^re 

^luce degli uomini, \brk ^regna con il tuo amore tra \[E-]no-\[C]o-\[G]o\[D]i 
\[E-]Oo\[C]oh \[G]\[D] \brk \[E-]Oo\[C]oh \[G]\[D]  \brk \[E-]Oo\[C]oh \[G]\[D]  
\[E-]Oh... \[E-] \[E-] \[E-]

\endchorus


\transpose{2}
%%%%%% EV. INTERMEZZO
\beginverse*
\vspace*{1.3\versesep}
{
	\nolyrics
	\textnote{Intermezzo con aumento di tonalità}
	
	\ifchorded

    %---- Prima riga -----------------------------
    \vspace*{-\versesep}
    \[E-] \[C] \[G] \[D]	 \rep{2}% \[*D] per indicare le pennate, \rep{2} le ripetizioni



	\fi
	%---- Ev Indicazioni -------------------------			
	%\textnote{\textit{(ripetizione della strofa)}} 
	 
}
\vspace*{\versesep}
\endverse

%%%%% STROFA
\beginverse		%Oppure \beginverse* se non si vuole il numero di fianco
%\memorize 		% <<< DECOMMENTA se si vuole utilizzarne la funzione
%\chordsoff		% <<< DECOMMENTA se vuoi una strofa senza accordi

^ Ci hai riscat^tati \brk dalla s^tretta delle ^tenebre, 
^ perché po^tessimo glo^rificare ^te. 
^ Hai river^sato in noi \brk la ^vita del tuo ^Spirito 
^per ^a-^mo-^re, 

\endverse
\beginverse*

^Figlio dell’altissimo, ^povero tra i poveri, 
^vieni a dimorare tra ^noi. 
^Dio dell’impossibile, ^Re di tutti i secoli 
^vieni nella tua mae^stà. 

\endverse



%%%%% RITORNELLO
\beginchorus
\textnote{\textbf{Rit.}}

^Re \[(D)]dei ^re 
i ^popoli ti acclamano, \brk i ^cieli ti proclamano 
^Re \[(D)]dei ^re 

^luce degli uomini, \brk ^regna con il tuo amore tra \[E-]no-\[C]o-\[G]o\[D]i 
\[E-]Oo\[C]oh \[G]\[D] \brk \[E-]Oo\[C]oh \[G]\[D]  \brk \[E-]Oo\[C]oh \[G]\[D]  
\[E-]Oh... \[E-] \[E-] \[E-]

\endchorus




%%%%% BRIDGE
\beginverse*		%Oppure \beginverse* se non si vuole il numero di fianco
%\memorize 		% <<< DECOMMENTA se si vuole utilizzarne la funzione
%\chordsoff		% <<< DECOMMENTA se vuoi una strofa senza accordi
\vspace*{1.3\versesep}
\textnote{Bridge} %<<< EV. INDICAZIONI

^Tua ^è la ^glo^ria per ^se-^em^pre, ^ 
^Tua ^è la ^glo^ria per ^se-^em^pre, ^ 
\endverse
\beginverse*
\vspace*{-\versesep}
^glo^ria, ^glo^ria, ^glo^ria, ^glo^ria.  

\endverse
\beginverse*

^Figlio dell’altissimo, ^povero tra i poveri, 
^vieni a dimorare tra ^noi. 
^Dio dell’impossibile, ^Re di tutti i secoli 
^vieni nella tua mae^stà. 

\endverse


%%%%% RITORNELLO
\beginchorus
\textnote{\textbf{Rit.}}

^Re \[(D)]dei ^re 
i ^popoli ti acclamano, \brk i ^cieli ti proclamano 
^Re \[(D)]dei ^re 

^luce degli uomini, \brk ^regna con il tuo amore tra \[E-]no-\[C]o-\[G]o\[D]i 
\[E-]Oo\[C]oh \[G]\[D] \brk \[E-]Oo\[C]oh \[G]\[D]  \brk \[E-]Oo\[C]oh \[G]\[D]  
\[E-]Oh... \[E-] \[E-] \[E-] \[E-*]

\endchorus


\endsong
\fi
%++++++++++++++++++++++++++++++++++++++++++++++++++++++++++++
%			FINE CANZONE TRASPOSTA
%++++++++++++++++++++++++++++++++++++++++++++++++++++++++++++