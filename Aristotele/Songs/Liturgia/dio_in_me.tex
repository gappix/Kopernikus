%-------------------------------------------------------------
%			INIZIO	CANZONE
%-------------------------------------------------------------


%titolo: 	Dio in me
%autore: 	SERMIG
%tonalita: 	Fa 



%%%%%% TITOLO E IMPOSTAZONI
\beginsong{Dio in me}[by={Sermig}] 	% <<< MODIFICA TITOLO E AUTORE
\transpose{0} 						% <<< TRASPOSIZIONE #TONI (0 nullo)
%\preferflats  %SE VOGLIO FORZARE i bemolle come alterazioni
%\prefersharps %SE VOGLIO FORZARE i # come alterazioni
\momenti{}							% <<< INSERISCI MOMENTI	
% momenti vanno separati da ; e vanno scelti tra:
% Ingresso; Atto penitenziale; Acclamazione al Vangelo; Dopo il Vangelo; Offertorio; Comunione; Ringraziamento; Fine; Santi; Pasqua; Avvento; Natale; Quaresima; Canti Mariani; Battesimo; Prima Comunione; Cresima; Matrimonio; Meditazione; Spezzare del pane;
\ifchorded
	%\textnote{Tonalità migliore }	% <<< EV COMMENTI (tonalità originale\migliore)
\fi


%%%%%% INTRODUZIONE
\ifchorded
\vspace*{\versesep}
\musicnote{
\begin{minipage}{0.48\textwidth}
\textbf{Intro}
\hfill 
%( \eighthnote \, 80)   % <<  MODIFICA IL TEMPO
% Metronomo: \eighthnote (ottavo) \quarternote (quarto) \halfnote (due quarti)
\end{minipage}
} 	
\vspace*{-\versesep}
\beginverse*

\nolyrics

%---- Prima riga -----------------------------
\vspace*{-\versesep}
\[F] \[A-] \[G4] \[G]

%---- Ogni riga successiva -------------------
\vspace*{-\versesep}
\[F] \[A-] \[G4] \[G]	

%---- Ev Indicazioni -------------------------			
%\textnote{\textit{(Oppure tutta la strofa)} }	

\endverse
\fi




%%%%% STROFA
\beginverse		%Oppure \beginverse* se non si vuole il numero di fianco
\memorize 		% <<< DECOMMENTA se si vuole utilizzarne la funzione
%\chordsoff		% <<< DECOMMENTA se vuoi una strofa senza accordi
Sei \[C]qui total\[F]mente \[C]D\[G]i\[A-]o 
dentro m\[G]e 
sei \[C]qui total\[F]mente \[C]u\[G]o\[A-]mo 
dentro \[G]me
\[C]E \[F]vuoi \[G]che \[C]io \[G]vi\[A-]va 
per \[G]te 
\[C]Sil\[F]e-\[G]e-\[C]nzio \[G]pre\[A-]ga 
con \[G]me con \[F]me 

\endverse






%%%%%% EV. INTERMEZZO
\beginverse*
\vspace*{1.3\versesep}
{
	\nolyrics
	\textnote{Intermezzo strumentale}
	
	\ifchorded

	%---- Prima riga -----------------------------
	\vspace*{-\versesep}
	\[F] \[A-] \[G4] \[G]

	%---- Ogni riga successiva -------------------
	\vspace*{-\versesep}
	\[F] \[A-] \[G4] \[G]


	\fi
	%---- Ev Indicazioni -------------------------			
	%\textnote{\textit{(ripetizione della strofa)}} 
	 
}
\vspace*{\versesep}
\endverse


%%%%% STROFA
\beginverse		%Oppure \beginverse* se non si vuole il numero di fianco
%\memorize 		% <<< DECOMMENTA se si vuole utilizzarne la funzione
%\chordsoff		% <<< DECOMMENTA se vuoi una strofa senza accordi

Per \[C]me ti sei \[F]fatto \[C]u\[G]o\[A-]mo 
come \[G]me 
La \[C]Croce tre\[F]menda \[C]più \[G]non \[A-]è 
dopo \[G]che 
\[C]Tu \[F]l’hai \[G]re\[C]sa \[G]bene\[A-]det\[G]ta 
\[C]Sil\[F]e-\[G]e-\[C]nzio \[G]pre\[A-]ga 
con \[G]me
\[C]A\[F]de-\[G]e-\[C]sso \[G]incontr\[A-]ando \[G]me 
\[C]non \[F]tro\[G]ve\[C]re\[G]te \[A-]me 
ma \[G]Dio
 in \[F]me  \[A-] \[G4] \[G]  
 in \[F]me \[A-] \[G4] \[G]


\endverse



%%%%%% EV. CHIUSURA SOLO STRUMENTALE
\ifchorded
\beginchorus %oppure \beginverse*
\vspace*{1.3\versesep}
\textnote{Chiusura strumentale } %<<< EV. INDICAZIONI

\nolyrics
%---- Prima riga -----------------------------
\vspace*{-\versesep}
\[F] \[A-] \[G4] \[G]

%---- Ogni riga successiva -------------------
\vspace*{-\versesep}
\[F] \[A-] \[G4] \[G]

%---- Ev Indicazioni -------------------------			
%\textnote{\textit{(Oppure tutta la strofa)} }	

\endchorus  %oppure \endverse
\fi


\endsong
%------------------------------------------------------------
%			FINE CANZONE
%------------------------------------------------------------



