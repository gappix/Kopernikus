%-------------------------------------------------------------
%			INIZIO	CANZONE
%-------------------------------------------------------------


%titolo: 	Ed essi si ameranno
%autore: 	Meregalli
%tonalita: 	Re / Do -2 / La -5 / Sol -7 /



%%%%%% TITOLO E IMPOSTAZONI
\beginsong{Ed essi si ameranno}[by={G. Meregalli}]
\transpose{0} % <<< TRASPOSIZIONE #TONI + - (0 nullo)
%\preferflats %SE VOGLIO FORZARE i bemolle come alterazioni
%\prefersharps %SE VOGLIO FORZARE i # come alterazioni						
\momenti{Congedo; Pasqua; Matrimonio; Santi; Comunione;}			% <<< INSERISCI MOMENTI	
% momenti vanno separati da ; e vanno scelti tra:
% Ingresso; Atto penitenziale; Acclamazione al Vangelo; Dopo il Vangelo; Offertorio; Comunione; Ringraziamento; Fine; Santi; Pasqua; Avvento; Natale; Quaresima; Canti Mariani; Battesimo; Prima Comunione; Cresima; Matrimonio; Meditazione;
\ifchorded
	%\textnote{Tonalità originale }	% <<< EV COMMENTI (tonalità originale/migliore)
\fi




%%%%%% INTRODUZIONE
\ifchorded
\vspace*{\versesep}
\musicnote{
\begin{minipage}{0.48\textwidth}
\textbf{Intro}
\hfill 
%( \eighthnote \, 80)   % <<  MODIFICA IL TEMPO
% Metronomo: \eighthnote (ottavo) \quarternote (quarto) \halfnote (due quarti)
\end{minipage}
} 	
\vspace*{-\versesep}
\beginverse*
\nolyrics

%---- Prima riga -----------------------------
\vspace*{-\versesep}
\[D] 

%---- Ogni riga successiva -------------------
%\vspace*{-\versesep}
%\[A-] \[E-] \[F] \[C] \[D-] \[A-] \[F] \[G*] \[A-] 

%---- Ev Indicazioni -------------------------			
%\textnote{\textit{(Oppure tutta la strofa)} }	

\endverse
\fi


%%%%% STROFA
\beginverse 	%Oppure \beginverse* se non si vuole il numero di fianco
\memorize 		% <<< DECOMMENTA se si vuole utilizzarne la funzione
%\chordsoff		% <<< DECOMMENTA se vuoi una strofa senza accordi

\[D]Vai a dire alla terra 
di sve\[F#]gliare dal sonno le genti;
dì alla \[D7]folgore, al tuono e alla \[B7]voce 
di inon\[E-]dare di luce la \[A]notte;
dì alle \[E-]nuvole bianche del \[A]cielo 
di var\[D]care la soglia del \[A4]tem\[A]po.

\endverse

\beginverse*	%Oppure \beginverse* se non si vuole il numero di fianco
%\memorize 		% <<< DECOMMENTA se si vuole utilizzarne la funzione
%\chordsoff		% <<< DECOMMENTA se vuoi una strofa senza accordi

^Vai a dire alla terra 
di tre^mare al passo tonante
dei messa^ggeri di pace; pro^clama 
la mia ^legge d’amore alle ^genti;
dì che i ^vecchi delitti ho scor^dati... 
e tra ^voi non sia odio né ^guer^ra.

\endverse





%%%%% RITORNELLO
\beginchorus
\textnote{\textbf{Rit.}}
\[G]È finito questo vecchio mondo:
il \[F]cielo antico è lace\[G]rato.  \rep{2}
\endchorus

\beginchorus
\[F] Il mio popolo \[C]si radune\[G]rà. 
\[F] Il mio popolo \[C]si radune\[G]rà.
\[F] Il mio popolo \[C]si radune\[A]rà. \[A]
\endchorus








%%%%% STROFA
\beginverse 	%Oppure \beginverse* se non si vuole il numero di fianco
%\memorize 		% <<< DECOMMENTA se si vuole utilizzarne la funzione
%\chordsoff		% <<< DECOMMENTA se vuoi una strofa senza accordi


^Vai a dire alla terra 
che il Si^gnore l’ha amata da sempre,
che il suo ^servo reietto da ^molti 
si è addos^sato il peccato di ^tutti
e innal^zato ha patito la ^croce 
e, se^polto, ha rivisto la ^luce^.

\endverse

\beginverse*	%Oppure \beginverse* se non si vuole il numero di fianco
%\memorize 		% <<< DECOMMENTA se si vuole utilizzarne la funzione
%\chordsoff		% <<< DECOMMENTA se vuoi una strofa senza accordi

^Vai a dire alle genti 
di invi^tare i fratelli alla mensa;
ogni ^popolo che è sulla ^terra 
la mia ^legge proclami ed os^servi;
messag^geri di pace so^lerti 
testi^moni di pace fe^de^li.

\endverse








%%%%% RITORNELLO
\beginchorus
\textnote{\textbf{Rit.}}
\[G]Ecco che nasce il nuovo mondo:
il \[F]vecchio è termi\[G]nato.   \rep{2}
\endchorus

\beginchorus
\[F] Il mio popolo \[C]si radune\[G]rà. 
\[F] Il mio popolo \[C]si radune\[G]rà.
\[F] Il mio popolo \[C]si radune\[A]rà. \[A]
\endchorus





%%%%%% EV. FINALE
\beginchorus
\textnote{\textbf{Finale}}
\textnote{\textit{(cambia il tempo, crescendo ad ogni ripetizione)}}
\[D]Ed il giorno an\[A*]co\[G*]ra \[D*]è spun\[G*]tato \[A*]nuo\[D*]vo,
\[D]uomini di \[A*]pa\[G*]ce ed \[D*]essi \[G*]si ame\[A*]ran\[D*]no.   \rep{3}
\endchorus
\beginchorus
\[D]Ed il giorno an\[A*]co\[G*]ra \[D*]è spun\[G*]tato \[A*]nuo\[D*]vo,
\[D]uomini di \[A*]pa\[G*]ce ed \[D*]essi \[G*]si ame\[A]ra-\[A]a-n\[D*]no.   
\endchorus


\endsong
%------------------------------------------------------------
%			FINE CANZONE
%------------------------------------------------------------