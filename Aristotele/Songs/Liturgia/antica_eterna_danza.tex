%-------------------------------------------------------------
%			INIZIO	CANZONE
%-------------------------------------------------------------


%titolo: 	Antica eterna danza
%autore: 	Gen Verde
%tonalita: 	Sol 



%%%%%% TITOLO E IMPOSTAZONI
\beginsong{Antica eterna danza}[by={Gen Verde}] 	% <<< MODIFICA TITOLO E AUTORE
\transpose{0} 						% <<< TRASPOSIZIONE #TONI (0 nullo)
\momenti{Offertorio; Prima Comunione}							% <<< INSERISCI MOMENTI	
% momenti vanno separati da ; e vanno scelti tra:
% Ingresso; Atto penitenziale; Acclamazione al Vangelo; Dopo il Vangelo; Offertorio; Comunione; Ringraziamento; Fine; Santi; Pasqua; Avvento; Natale; Quaresima; Canti Mariani; Battesimo; Prima Comunione; Cresima; Matrimonio; Meditazione;
\ifchorded
	%\textnote{Tonalità originale }	% <<< EV COMMENTI (tonalità originale/migliore)
\fi





%%%%%% INTRODUZIONE
\ifchorded
\vspace*{\versesep}
\textnote{Intro: \qquad \qquad  }%(\eighthnote 116) % << MODIFICA IL TEMPO
% Metronomo: \eighthnote (ottavo) \quarternote (quarto) \halfnote (due quarti)
\vspace*{-\versesep}
\beginverse*

\nolyrics

%---- Prima riga -----------------------------
\vspace*{-\versesep}
\[*G] \[*D]  \[*C]	 % \[*D] per indicare le pennate, \rep{2} le ripetizioni

%---- Ogni riga successiva -------------------
\vspace*{-\versesep}
\[G] \[C]  \[G]	

%---- Ev Indicazioni -------------------------			
%\textnote{\textit{(Oppure tutta la strofa)} }	

\endverse
\fi



%%%%% STROFA
\beginverse
\memorize

\[G]Spighe \[D]d'oro al \[E-]vento, 
an\[E-]tica, e\[D]terna \[C]danza
per \[A-]fare un \[D]solo \[E-]pa\[D]ne
spez\[C]zato \[A-]sulla m\[B]ensa.
\[G]Grappoli \[D]dei \[E-]colli, 
pro\[E-]fumo \[D]di le\[C]tizia
per \[A-]fare un \[D]solo \[E-]vi\[D]no 
be\[A-]vanda \[B-]della \[E-]grazia.  

\endverse



%%%%%% EV. INTERMEZZO
\beginverse*
\vspace*{1.3\versesep}
{
	\nolyrics
	
	\ifchorded
	\textnote{Intermezzo strumentale}

		
	%---- Prima riga -----------------------------
	\vspace*{-\versesep}
	\[*G] \[*D]  \[*C]	 % \[*D] per indicare le pennate, \rep{2} le ripetizioni

	%---- Ogni riga successiva -------------------
	\vspace*{-\versesep}
	\[G] \[C]  \[G] 


	\fi
	%---- Ev Indicazioni -------------------------			
	%\textnote{\textit{(ripetizione della strofa)}} 
	 
}
\vspace*{\versesep}
\endverse




%%%%% STROFA
\beginverse

^Con il pa^ne e il vi^no 
Si^gnore ^ti doni^amo
le ^nostre gi^oie ^pu^re, 
le at^tese e ^le pa^ure.
^Frutti ^del la^voro 
e ^fede n^el fu^turo,
la ^voglia ^di cam^bia^re 
e ^di ri^cominci^are.  

\endverse



%%%%%% EV. INTERMEZZO
\beginverse*
\vspace*{1.3\versesep}
{
	\nolyrics
	
	\ifchorded
	\textnote{Intermezzo strumentale}

		
	%---- Prima riga -----------------------------
	\vspace*{-\versesep}
	\[*G] \[*D]  \[*C]	 % \[*D] per indicare le pennate, \rep{2} le ripetizioni

	%---- Ogni riga successiva -------------------
	\vspace*{-\versesep}
	\[G] \[C]  \[G]	


	\fi
	%---- Ev Indicazioni -------------------------			
	%\textnote{\textit{(ripetizione della strofa)}} 
	 
}
\vspace*{\versesep}
\endverse




%%%%% STROFA
\beginverse

^Dio del^la spe^ranza, 
sor^gente ^d'ogni ^dono
ac^cogli q^uesta o^ffer^ta 
che in^sieme ^Ti porti^amo.
^Dio dell'^uni^verso 
rac^cogli ^chi è dis^perso
 

\vspace*{1.3\versesep}
\textnote{\textit{(rallentando)}}
e ^facci ^tutti ^Chie^sa,
u^na ^cosa in ^Te.
\endverse
\endsong

