%-------------------------------------------------------------
%			INIZIO	CANZONE
%-------------------------------------------------------------


%titolo: 	Acqua siamo noi
%autore: 	Cento
%tonalita: 	Re



%%%%%% TITOLO E IMPOSTAZONI
\beginsong{Giullare dei campi}[by={Canto Salesiano dedicato a Don Bosco — P. Pignatelli}] 	% <<< MODIFICA TITOLO E AUTORE
\transpose{0} 						% <<< TRASPOSIZIONE #TONI (0 nullo)
\momenti{Fine; Santi}							% <<< INSERISCI MOMENTI	
% momenti vanno separati da ; e vanno scelti tra:
% Ingresso; Atto penitenziale; Acclamazione al Vangelo; Dopo il Vangelo; Offertorio; Comunione; Ringraziamento; Fine; Santi; Pasqua; Avvento; Natale; Quaresima; Canti Mariani; Battesimo; Prima Comunione; Cresima; Matrimonio; Meditazione;
\ifchorded
	%\textnote{Tonalità originale }	% <<< EV COMMENTI (tonalità originale/migliore)
\fi



%%%%%% INTRODUZIONE
\ifchorded
\vspace*{\versesep}
\musicnote{
\begin{minipage}{0.48\textwidth}
\textbf{Intro}
\hfill 
%( \eighthnote \, 80)   % <<  MODIFICA IL TEMPO
% Metronomo: \eighthnote (ottavo) \quarternote (quarto) \halfnote (due quarti)
\end{minipage}
} 	
\vspace*{-\versesep}
\beginverse*
\nolyrics

%---- Prima riga -----------------------------
\vspace*{-\versesep}
\[C] \[A-]  \[F]	\[C] % \[*D] per indicare le pennate, \rep{2} le ripetizioni

%---- Ogni riga successiva -------------------
%\vspace*{-\versesep}
%\[G] \[C]  \[D]	

%---- Ev Indicazioni -------------------------			
%\textnote{\textit{(Come la prima riga)} }	

\endverse
\fi



%%%%% STROFA
\beginverse		%Oppure \beginverse* se non si vuole il numero di fianco
\memorize 		% <<< DECOMMENTA se si vuole utilizzarne la funzione
%\chordsoff		& <<< DECOMMENTA se vuoi una strofa senza accordi

Cal\[C]zoni colore del \[A-]prato, 
un ginocchio amma\[F]ccato 
per un salto in \[C]più, 
due pi\[G7]ante un filo ti\[A-]rato,
la mela sul \[F]naso e gli amici \[G7]giù. 
Un \[D-]pezzo di pane e una \[G7]fetta di cielo, 
sa\[C]pore di festa e \[A-]tu: 
Gio\[F]vanni dei Becchi giul\[C]lare dei campi 
re\[G7]galo alla gioven\[C]tù.

\endverse



%%%%% RITORNELLO
\textnote{\textbf{Rit.}}
\beginchorus

Siete tutti \[F]ladri ragazzi mi\[G7]ei, 
non ho più il mio \[C]cuore ce l’avete \[A-]voi! 
Ma non m’inte\[F]ressa da quest’oggi in \[G7]poi 
ogni mio res\[F]piro sarà per \[C]voi. \rep{2}

\endchorus



%%%%% STROFA
\beginverse
La ^veste color della st^rada 
forse un pò consu^mata,
qualche acciacco in ^più, 
nei ^prati intorno a Val^docco 
ti chiama don ^Bosco la tua gioven^tù. 
La ^vecchia tettoia e una ^piccola stanza
fra ^spiagge infinite in ^cuor, 
un ^fischio per Corso Re^gina, uno sguardo 
pro^fondo sentono l’a^more!

\endverse



%%%%% STROFA
\beginverse
%\chordsoff
Un ^eco color della ^storia, 
tesoro dei ^campi 
che oggi non è ^più, 
il ^vecchio pilone del ^sogno, 
il ragazzo sul ^filo non esiste ^più. 
L’an^tica fontana del ^grande cortile 
non ^getta più acqua e ^tu... 
as^petti qualcuno che ^ancora racconti 
l’a^more alla gioven^tù. 
\endverse




\endsong
%------------------------------------------------------------
%			FINE CANZONE
%------------------------------------------------------------