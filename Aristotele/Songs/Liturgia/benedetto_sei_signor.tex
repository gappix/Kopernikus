%-------------------------------------------------------------
%			INIZIO	CANZONE
%-------------------------------------------------------------


%titolo: 	Benedetto sei Signor
%autore: 	La Rocca, Capacchione
%tonalita: 	Sol 



%%%%%% TITOLO E IMPOSTAZONI
\beginsong{Benedetto sei Signor}[by={La Rocca, Capacchione}] 	% <<< MODIFICA TITOLO E AUTORE
\transpose{0} 						% <<< TRASPOSIZIONE #TONI (0 nullo)
\momenti{Offertorio; Pasqua}							% <<< INSERISCI MOMENTI	
% momenti vanno separati da ; e vanno scelti tra:
% Ingresso; Atto penitenziale; Acclamazione al Vangelo; Dopo il Vangelo; Offertorio; Comunione; Ringraziamento; Fine; Santi; Pasqua; Avvento; Natale; Quaresima; Canti Mariani; Battesimo; Prima Comunione; Cresima; Matrimonio; Meditazione; Spezzare del pane;
\ifchorded
	%\textnote{Tonalità migliore }	% <<< EV COMMENTI (tonalità originale/migliore)
\fi


%%%%%% INTRODUZIONE
\ifchorded
\vspace*{\versesep}
\textnote{Intro: \qquad \qquad  }%(\eighthnote 116) % <<  MODIFICA IL TEMPO
% Metronomo: \eighthnote (ottavo) \quarternote (quarto) \halfnote (due quarti)
\vspace*{-\versesep}
\beginverse*

\nolyrics

%---- Prima riga -----------------------------
\vspace*{-\versesep}
\[F] \[D-] \[B&*]  \[C*] \[F] \rep{2}	 % \[*D] per indicare le pennate, \rep{2} le ripetizioni

%---- Ogni riga successiva -------------------
%\vspace*{-\versesep}
%\[G] \[C]  \[D]	

%---- Ev Indicazioni -------------------------			
%\textnote{\textit{(Oppure tutta la strofa)} }	

\endverse
\fi




%%%%% RITORNELLO
\beginchorus
\textnote{\textbf{Rit.}}

\[F]Gloria, \[D-]gloria, bene\[B&*]detto \[C*]sei Si\[F]gnor!
\[F]Gloria, \[D-]gloria, bene\[B&*]detto \[C*]sei Si\[F]gnor!

\endchorus



%%%%% STROFA
\beginverse		%Oppure \beginverse* se non si vuole il numero di fianco
\memorize 		% <<< DECOMMENTA se si vuole utilizzarne la funzione
%\chordsoff		% <<< DECOMMENTA se vuoi una strofa senza accordi

Questo \[F]pane e questo \[D-]vino 
noi li \[G]presentiamo a \[C]Te:
sono i \[F]doni del tuo \[D-]amore,
 nutri\[G]mento dell'umani\[C]tà.
Bene\[A]detto sei Tu, Si\[D-]gnore, 
per la \[B&]mensa che prepari a \[C]noi;
fà che in\[A]torno a questo al\[D-]tare 
sia \[B&*]grande la fra\[C*]terni\[F]tà.

\endverse









%%%%% STROFA
\beginverse		%Oppure \beginverse* se non si vuole il numero di fianco
%\memorize 		% <<< DECOMMENTA se si vuole utilizzarne la funzione
%\chordsoff		% <<< DECOMMENTA se vuoi una strofa senza accordi

Questa ^vita che ci ^doni 
noi la ^presentiamo a ^Te,
nella ^fede i nostri ^giorni 
noi vi^vremo con semplici^tà.
Bene^detto sei Tu, Si^gnore, 
per il ^pane che ci done^rai;
fà che al ^mondo possiamo ^dare 
una ^vera soli^darie^tà.
\endverse




\endsong
%------------------------------------------------------------
%			FINE CANZONE
%------------------------------------------------------------

