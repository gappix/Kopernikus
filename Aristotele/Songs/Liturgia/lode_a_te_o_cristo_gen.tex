%-------------------------------------------------------------
%			INIZIO	CANZONE
%-------------------------------------------------------------


%titolo: 	Lode a te o Cristo
%autore: 	Gen Verde
%tonalita: 	F#-



%%%%%% TITOLO E IMPOSTAZONI
\beginsong{Lode a te o Cristo}[by={Gen Verde}] 	% <<< MODIFICA TITOLO E AUTORE
\transpose{0} 						% <<< TRASPOSIZIONE #TONI (0 nullo)
%\preferflats  %SE VOGLIO FORZARE i bemolle come alterazioni
%\prefersharps %SE VOGLIO FORZARE i # come alterazioni
\momenti{Acclamazione al Vangelo; Quaresima}							% <<< INSERISCI MOMENTI	
% momenti vanno separati da ; e vanno scelti tra:
% Ingresso; Atto penitenziale; Acclamazione al Vangelo; Dopo il Vangelo; Offertorio; Comunione; Ringraziamento; Fine; Santi; Pasqua; Avvento; Natale; Quaresima; Canti Mariani; Battesimo; Prima Comunione; Cresima; Matrimonio; Meditazione; Spezzare del pane;
\ifchorded
	%\textnote{Tonalità migliore }	% <<< EV COMMENTI (tonalità originale/migliore)
\fi

%%%%%% INTRODUZIONE
\ifchorded
\vspace*{\versesep}
\musicnote{
\begin{minipage}{0.48\textwidth}
\textbf{Intro}
\hfill 
%( \eighthnote \, 80)   % <<  MODIFICA IL TEMPO
% Metronomo: \eighthnote (ottavo) \quarternote (quarto) \halfnote (due quarti)
\end{minipage}
} 	
\vspace*{-\versesep}
\beginverse*

\nolyrics

%---- Prima riga -----------------------------
\vspace*{-\versesep}
\[F#-] \[E] \[D] \[F#-]	 % \[*D] per indicare le pennate, \rep{2} le ripetizioni

%---- Ogni riga successiva -------------------
%\vspace*{-\versesep}
%\[G] \[C]  \[D]	

%---- Ev Indicazioni -------------------------			
%\textnote{\textit{(Oppure tutta la strofa)} }	

\endverse
\fi




%%%%% RITORNELLO
\beginchorus
\textnote{\textbf{Rit.}}


\[F#-]Lode a \[E]te, o \[D]Cri\[F#-]sto,
\[B-]Re di e\[D]ter\[E*]na \[F#-]gloria.  \rep{2}

\endchorus



%%%%%% EV. INTERMEZZO
\beginverse*
\vspace*{1.3\versesep}
{
	\nolyrics
	\textnote{Intermezzo strumentale}
	
	\ifchorded

	%---- Prima riga -----------------------------
	\vspace*{-\versesep}
	\[B-] \[E]  \[F#-]	 \rep{2}




	\fi
	%---- Ev Indicazioni -------------------------			
	%\textnote{\textit{(ripetizione della strofa)}} 
	 
}
\vspace*{\versesep}
\endverse



%%%%% STROFA
\beginverse		%Oppure \beginverse* se non si vuole il numero di fianco
\memorize 		% <<< DECOMMENTA se si vuole utilizzarne la funzione
%\chordsoff		% <<< DECOMMENTA se vuoi una strofa senza accordi

Si\[F#-]gnore, tu \[E]sei vera\[D]men\[F#-]te 
\[F#-]il Salva\[E]tore del \[D]mon\[E]do,
\[F#-]dammi dell’\[E]acqua \[D]vi\[C#-]va 
per\[D7]ché non \[E]abbia più \[F#-]sete.

\endverse






%%%%% STROFA
\beginverse		%Oppure \beginverse* se non si vuole il numero di fianco
%\memorize 		% <<< DECOMMENTA se si vuole utilizzarne la funzione
%\chordsoff		% <<< DECOMMENTA se vuoi una strofa senza accordi

Chi ^beve ^di quest’^acq^ua 
^avrà di ^nuovo ^se^te,
ma chi ^beve dell’^acqua che ^io gli da^rò
^non av^rà mai più ^sete.

\endverse







%%%%%% EV. INTERMEZZO
\beginverse*
\vspace*{1.3\versesep}
{
	\nolyrics
	\musicnote{Finale strumentale}
	
	\ifchorded

	%---- Prima riga -----------------------------
	\vspace*{-\versesep}
	\[F#-] \[E] \[D] \[F#-]	 % \[*D] per indicare le pennate, \rep{2} le ripetizioni


	%---- Ogni riga successiva -------------------
	\vspace*{-\versesep}
	\[F#-] \[E] \[D] \[C#-]	 % \[*D] per indicare le pennate, \rep{2} le ripetizioni

	%---- Ogni riga successiva -------------------
	\vspace*{-\versesep}
	\[F#-] \[E] \[D] \[C#-]	 % \[*D] per indicare le pennate, \rep{2} le ripetizioni

	%---- Prima riga -----------------------------
	\vspace*{-\versesep}
	\[D] \[E] \[F#-*]	 % \[*D] per indicare le pennate, \rep{2} le ripetizioni


	\fi
	%---- Ev Indicazioni -------------------------			
	%\textnote{\textit{(ripetizione della strofa)}} 
	 
}
\vspace*{\versesep}
\endverse

\endsong
%------------------------------------------------------------
%			FINE CANZONE
%------------------------------------------------------------

