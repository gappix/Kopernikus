%-------------------------------------------------------------
%			INIZIO	CANZONE
%-------------------------------------------------------------


%titolo: 	Nulla è impossibile a Dio
%autore: 	C. Burgio
%tonalita: 	Re>Fa



%%%%%% TITOLO E IMPOSTAZONI
\beginsong{Nulla è impossibile a Dio}[by={C. Burgio}] 	% <<< MODIFICA TITOLO E AUTORE
\transpose{3} 						% <<< TRASPOSIZIONE #TONI (0 nullo)
\momenti{Ingresso; Prima Comunione}							% <<< INSERISCI MOMENTI	
% momenti vanno separati da ; e vanno scelti tra:
% Ingresso; Atto penitenziale; Acclamazione al Vangelo; Dopo il Vangelo; Offertorio; Comunione; Ringraziamento; Fine; Santi; Pasqua; Avvento; Natale; Quaresima; Canti Mariani; Battesimo; Prima Comunione; Cresima; Matrimonio; Meditazione; Spezzare del pane;
\ifchorded
	\textnote{Tonalità migliore}	% <<< EV COMMENTI (tonalità originale/migliore)
\fi


%%%%%% INTRODUZIONE
\ifchorded
\vspace*{\versesep}
\textnote{Intro: \qquad \qquad  }%(\eighthnote 116) % <<  MODIFICA IL TEMPO
% Metronomo: \eighthnote (ottavo) \quarternote (quarto) \halfnote (due quarti)
\vspace*{-\versesep}
\beginverse*

\nolyrics

%---- Prima riga -----------------------------
\vspace*{-\versesep}
\[D] \[A] \[G] \[A]	 % \[*D] per indicare le pennate, \rep{2} le ripetizioni

%---- Ogni riga successiva -------------------
%\vspace*{-\versesep}
%\[G] \[C]  \[D]	

%---- Ev Indicazioni -------------------------			
%\textnote{\textit{(Oppure tutta la strofa)} }	

\endverse
\fi




%%%%% STROFA
\beginverse		%Oppure \beginverse* se non si vuole il numero di fianco
\memorize 		% <<< DECOMMENTA se si vuole utilizzarne la funzione
%\chordsoff		% <<< DECOMMENTA se vuoi una strofa senza accordi
\[D]Quando Dio ci chia\[A]mò 
ed il \[G]tempo ci do\[A]nò,
come un \[D]padre fidu\[A]cioso 
nel suo \[G]cuore ci por\[A]tò.

\[B-]Egli fece di \[F#-]noi 
una s\[G]toria un solo \[A]popolo;
\[B-]forte, lui, ci gui\[F#-]dò 
sulle \[G]strade che con\[E-]ducono 
alla \[F#]libertà.

\endverse




%%%%% RITORNELLO
\beginchorus
\textnote{\textbf{Rit.}}

\[D]Ecco il grande mis\[A]tero 
dai \[B-]secoli annun\[F#-]ciato:
“\[G]Nulla è impos\[D]sibile a \[A]Dio”. \[A7]
\[D]Nasce nuova spe\[A7]ranza 
si \[B-]compie ormai la pro\[F#-]messa:
“\[G]Nulla è impos\[D]sibile a \[A]Di\[D]o”.

\endchorus



%%%%% STROFA
\beginverse		%Oppure \beginverse* se non si vuole il numero di fianco
%\memorize 		% <<< DECOMMENTA se si vuole utilizzarne la funzione
%\chordsoff		% <<< DECOMMENTA se vuoi una strofa senza accordi

^Quando venne tra ^noi, 
come ^figlio e “Dio con ^noi”, 
fatto ^uomo in Ma^ria, 
la sal^vezza ci por^tò. 

^Noi credemmo in ^Lui, 
vivo ^segno della ^Verità; 
^imparammo da ^Lui 
che l’^Amore non ha ^prezzo,
non pos^siede mai. 

\endverse



%%%%% STROFA
\beginverse		%Oppure \beginverse* se non si vuole il numero di fianco
%\memorize 		% <<< DECOMMENTA se si vuole utilizzarne la funzione
%\chordsoff		% <<< DECOMMENTA se vuoi una strofa senza accordi

^Quando poi ci las^ciò 
e dal ^padre rito^rnò, 
il Si^gnore dalla ^croce 
il suo ^spirito do^nò. 

^Nuova vita per ^noi 
questa ^grazia che ci il^lumina, 
^è memoria tra ^noi 
dell’A^more che ci ac^coglie
e non ci ^lascia mai.

\endverse




\endsong
%------------------------------------------------------------
%			FINE CANZONE
%------------------------------------------------------------

%++++++++++++++++++++++++++++++++++++++++++++++++++++++++++++
%			CANZONE TRASPOSTA
%++++++++++++++++++++++++++++++++++++++++++++++++++++++++++++
\ifchorded
%decremento contatore per avere stesso numero
\addtocounter{songnum}{-1} 
\beginsong{Nulla è impossibile a Dio}[by={C. Burgio}] 	% <<< MODIFICA TITOLO E AUTORE 
\transpose{0} 						% <<< TRASPOSIZIONE #TONI + - (0 nullo)
%\preferflats  %SE VOGLIO FORZARE i bemolle come alterazioni
%\prefersharps %SE VOGLIO FORZARE i # come alterazioni
\ifchorded
	\textnote{Tonalità originale}	% <<< EV COMMENTI (tonalità originale/migliore)
\fi


%%%%%% INTRODUZIONE
\ifchorded
\vspace*{\versesep}
\textnote{Intro: \qquad \qquad  }%(\eighthnote 116) % <<  MODIFICA IL TEMPO
% Metronomo: \eighthnote (ottavo) \quarternote (quarto) \halfnote (due quarti)
\vspace*{-\versesep}
\beginverse*

\nolyrics

%---- Prima riga -----------------------------
\vspace*{-\versesep}
\[D] \[A] \[G] \[A]	 % \[*D] per indicare le pennate, \rep{2} le ripetizioni

%---- Ogni riga successiva -------------------
%\vspace*{-\versesep}
%\[G] \[C]  \[D]	

%---- Ev Indicazioni -------------------------			
%\textnote{\textit{(Oppure tutta la strofa)} }	

\endverse
\fi




%%%%% STROFA
\beginverse		%Oppure \beginverse* se non si vuole il numero di fianco
\memorize 		% <<< DECOMMENTA se si vuole utilizzarne la funzione
%\chordsoff		% <<< DECOMMENTA se vuoi una strofa senza accordi
\[D]Quando Dio ci chia\[A]mò 
ed il \[G]tempo ci do\[A]nò,
come un \[D]padre fidu\[A]cioso 
nel suo \[G]cuore ci por\[A]tò.

\[B-]Egli fece di \[F#-]noi 
una s\[G]toria un solo \[A]popolo;
\[B-]forte, lui, ci gui\[F#-]dò 
sulle \[G]strade che con\[E-]ducono 
alla \[F#]libertà.

\endverse




%%%%% RITORNELLO
\beginchorus
\textnote{\textbf{Rit.}}

\[D]Ecco il grande mis\[A]tero 
dai \[B-]secoli annun\[F#-]ciato:
“\[G]Nulla è impos\[D]sibile a \[A]Dio”. \[A7]
\[D]Nasce nuova spe\[A7]ranza 
si \[B-]compie ormai la pro\[F#-]messa:
“\[G]Nulla è impos\[D]sibile a \[A]Di\[D]o”.

\endchorus



%%%%% STROFA
\beginverse		%Oppure \beginverse* se non si vuole il numero di fianco
%\memorize 		% <<< DECOMMENTA se si vuole utilizzarne la funzione
%\chordsoff		% <<< DECOMMENTA se vuoi una strofa senza accordi

^Quando venne tra ^noi, 
come ^figlio e “Dio con ^noi”, 
fatto ^uomo in Ma^ria, 
la sal^vezza ci por^tò. 

^Noi credemmo in ^Lui, 
vivo ^segno della ^Verità; 
^imparammo da ^Lui 
che l’^Amore non ha ^prezzo,
non pos^siede mai. 

\endverse



%%%%% STROFA
\beginverse		%Oppure \beginverse* se non si vuole il numero di fianco
%\memorize 		% <<< DECOMMENTA se si vuole utilizzarne la funzione
%\chordsoff		% <<< DECOMMENTA se vuoi una strofa senza accordi

^Quando poi ci las^ciò 
e dal ^padre rito^rnò, 
il Si^gnore dalla ^croce 
il suo ^spirito do^nò. 

^Nuova vita per ^noi 
questa ^grazia che ci il^lumina, 
^è memoria tra ^noi 
dell’A^more che ci ac^coglie
e non ci ^lascia mai.

\endverse




\endsong


\fi
%++++++++++++++++++++++++++++++++++++++++++++++++++++++++++++
%			FINE CANZONE TRASPOSTA
%++++++++++++++++++++++++++++++++++++++++++++++++++++++++++++
