%-------------------------------------------------------------
%			INIZIO	CANZONE
%-------------------------------------------------------------


%titolo: 	La Tue meraviglie
%autore: 	Casucci, Balduzzi
%tonalita: 	Fa 



%%%%%% TITOLO E IMPOSTAZONI
\beginsong{Le tue meraviglie}[by={G. Casucci, M. Balduzzi}]
\transpose{0} 						% <<< TRASPOSIZIONE #TONI (0 nullo)
\momenti{Congedo; Ringraziamento}							% <<< INSERISCI MOMENTI	
% momenti vanno separati da ; e vanno scelti tra:
% Ingresso; Atto penitenziale; Acclamazione al Vangelo; Dopo il Vangelo; Offertorio; Comunione; Ringraziamento; Fine; Santi; Pasqua; Avvento; Natale; Quaresima; Canti Mariani; Battesimo; Prima Comunione; Cresima; Matrimonio; Meditazione;
\ifchorded
	%\textnote{Tonalità originale }	% <<< EV COMMENTI (tonalità originale/migliore)
\fi


%%%%%% INTRODUZIONE
\ifchorded
\vspace*{\versesep}
\musicnote{
\begin{minipage}{0.48\textwidth}
\textbf{Intro}
\hfill 
%( \eighthnote \, 80)   % <<  MODIFICA IL TEMPO
% Metronomo: \eighthnote (ottavo) \quarternote (quarto) \halfnote (due quarti)
\end{minipage}
} 	
\vspace*{-\versesep}
\beginverse*
\nolyrics

%---- Prima riga -----------------------------
\vspace*{-\versesep}
\[A-] \[E-] \[F] \[C] \[D-] \[A-] \[F] \[G]	 % \[*D] per indicare le pennate, \rep{2} le ripetizioni

%---- Ogni riga successiva -------------------
\vspace*{-\versesep}
\[A-] \[E-] \[F] \[C] \[D-] \[A-] \[F] \[*G] \[A-] 

%---- Ev Indicazioni -------------------------			
%\textnote{\textit{(Oppure tutta la strofa)} }	

\endverse
\fi


%%%%% RITORNELLO
\beginchorus
\textnote{\textbf{Rit.}}
Ora \[F]lascia, o Si\[G]gnore, che io \[E-]vada in pa\[A-]ce,
perché ho \[D-]visto le tue \[C]mera\[B&]vi\[G]glie.
Il tuo \[F]popolo in \[G]festa per le \[E-]strade corre\[A-]rà
a por\[D-]tare le tue \[C]mera\[B&]vi\[G]glie!
\endchorus




%%%%% STROFA
\beginverse		%Oppure \beginverse* se non si vuole il numero di fianco
\memorize 		% <<< DECOMMENTA se si vuole utilizzarne la funzione
%\chordsoff		% <<< DECOMMENTA se vuoi una strofa senza accordi
\vspace*{1.3\versesep}
\musicnote{\textit{(da piano a forte in un crescendo)}}
\[A-]La tua pre\[E-]senza ha riem\[F]pito d'a\[C]more
\[A-]le nostre \[E-]vite, le \[F]nostre gior\[C]nate.
\[B&]In te una sola \[F]anima, \[G-]un solo cuore \[F]siamo noi:
\[B&]con te la luce ri\[F]splende, \brk \[G-]splende più chiara che \[C]mai!
\endverse



%%%%% STROFA
\beginverse

^La tua pre^senza ha inon^dato d'a^more
^le nostre ^vite, le ^nostre gior^nate,
^fra la tua gente ^resterai, \brk ^per sempre vivo in ^mezzo a noi
^fino ai confini del ^tempo: ^così ci accompagne^rai.

\endverse




%%%%% RITORNELLO
\beginchorus
\textnote{\textbf{Rit.}}
Ora \[F]lascia, o Si\[G]gnore, che io \[E-]vada in pa\[A-]ce,
perché ho \[D-]visto le tue \[C]mera\[B&]vi\[G]glie.
Il tuo \[F]popolo in \[G]festa per le \[E-]strade corre\[A-]rà
a por\[D-]tare le tue \[C]mera\[B&]vi\[G]glie!
Ora \[F]lascia, o Si\[G]gnore, che io \[E-]vada in pa\[A-]ce,
perché ho \[F]visto le \[G]tue mera\[E-]vi\[A-]glie.
E il tuo \[F]popolo in \[G]festa per le \[E-]strade corre\[A-]rà
a por\[F]tare le \[G]tue mera\[F]vi\[C]glie!
\endchorus





%%%%%% EV. FINALE
\ifchorded
\beginverse*
\vspace*{1.3\versesep}
{
	\nolyrics
	\textnote{Finale strumentale}
	
	
	%---- Prima riga -----------------------------
	\vspace*{-\versesep}
	\[A-] \[E-] \[F] \[C] \[D-] \[A-] \[F] \[G] 

	%---- Ogni riga successiva -------------------
	\vspace*{-\versesep}
	\[A-] \[E-] \[F] \[C] \[D-] \[A-] \[F] \[*G] \[A-]

}	
\vspace*{\versesep}
\endverse
\fi



\endsong
%------------------------------------------------------------
%			FINE CANZONE
%------------------------------------------------------------