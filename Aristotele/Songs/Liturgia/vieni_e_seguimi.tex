%-------------------------------------------------------------
%			INIZIO	CANZONE
%-------------------------------------------------------------


%titolo: 	Vieni e seguimi
%autore: 	Gen Rosso
%tonalita: 	La



%%%%%% TITOLO E IMPOSTAZONI
\beginsong{Vieni e seguimi}[by={Gen Rosso}] 	% <<< MODIFICA TITOLO E AUTORE
\transpose{0} 						% <<< TRASPOSIZIONE #TONI (0 nullo)
\momenti{Comunione; Fine; }							% <<< INSERISCI MOMENTI	
% momenti vanno separati da ; e vanno scelti tra:
% Ingresso; Atto penitenziale; Acclamazione al Vangelo; Dopo il Vangelo; Offertorio; Comunione; Ringraziamento; Fine; Santi; Pasqua; Avvento; Natale; Quaresima; Canti Mariani; Battesimo; Prima Comunione; Cresima; Matrimonio; Meditazione; Spezzare del pane;
\ifchorded
	%\textnote{Tonalità originale }	% <<< EV COMMENTI (tonalità originale/migliore)
\fi


%%%%%% INTRODUZIONE
\ifchorded
\vspace*{\versesep}
\textnote{Intro: \qquad \qquad  }%(\eighthnote 116) % <<  MODIFICA IL TEMPO
% Metronomo: \eighthnote (ottavo) \quarternote (quarto) \halfnote (due quarti)
\vspace*{-\versesep}
\beginverse*

\nolyrics

%---- Prima riga -----------------------------
\vspace*{-\versesep}
\[A] \[B-] \[A]	 % \[*D] per indicare le pennate, \rep{2} le ripetizioni

%---- Ogni riga successiva -------------------
%\vspace*{-\versesep}
%\[G] \[C]  \[D]	

%---- Ev Indicazioni -------------------------			
%\textnote{\textit{(Oppure tutta la strofa)} }	

\endverse
\fi



%%%%% STROFA
\beginverse		%Oppure \beginverse* se non si vuole il numero di fianco
\memorize 		% <<< DECOMMENTA se si vuole utilizzarne la funzione
%\chordsoff		% <<< DECOMMENTA se vuoi una strofa senza accordis
Lascia \[A]che il mondo \[B-]vada per la sua \[A]strada.
Lascia \[C#-]che l'uomo ri\[F#-]torni alla sua \[E]casa.
Lascia \[D]che la gente accumuli la sua for\[A]tuna.
\endverse


\beginchorus
Ma \[E]tu, tu \[D]vieni e \[A]seguimi, \[E]tu, \[D] vieni e \[A]seguimi.
\endchorus


\beginverse
Lascia ^che la barca in ^mare spieghi la ^vela.
Lascia ^che trovi af^fetto chi segue il ^cuore.
Lascia ^che dall'albero cadano i frutti ma^turi.
\endverse



\beginchorus
Ma \[E]tu, tu \[D]vieni e \[A]seguimi, \[E]tu, \[D] vieni e \[F#-]seguimi.
\endchorus



\beginverse
E sa\[F#]rai luce per gli \[B]uomini
e sa\[F#]rai sale della \[C#-]terra \[E]
e nel mondo de\[F#]serto aprirai
una \[B]strada nuova. \rep{2}
E per \[F#]questa strada, \[G#-]va', \[F#]va',
e \[B]non voltarti indietro, \[F#]va'
e \[B]non voltarti indietro\[F#]{\dots}
\endverse


\endsong
%------------------------------------------------------------
%			FINE CANZONE
%------------------------------------------------------------