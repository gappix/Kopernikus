%-------------------------------------------------------------
%			INIZIO	CANZONE
%-------------------------------------------------------------


%titolo: 	Lo Spirito di Cristo
%autore: 	
%tonalita: 	Mi



%%%%%% TITOLO E IMPOSTAZONI
\beginsong{Lo Spirito di Cristo}[by={}] 	% <<< MODIFICA TITOLO E AUTORE
\transpose{0} 						% <<< TRASPOSIZIONE #TONI (0 nullo)
\momenti{Ingresso; Cresima}							% <<< INSERISCI MOMENTI	
% momenti vanno separati da ; e vanno scelti tra:
% Ingresso; Atto penitenziale; Acclamazione al Vangelo; Dopo il Vangelo; Offertorio; Comunione; Ringraziamento; Fine; Santi; Pasqua; Avvento; Natale; Quaresima; Canti Mariani; Battesimo; Prima Comunione; Cresima; Matrimonio; Meditazione; Spezzare del pane;
\ifchorded
	%\textnote{Tonalità originale }	% <<< EV COMMENTI (tonalità originale/migliore)
\fi

%%%%%% INTRODUZIONE
\ifchorded
\vspace*{\versesep}
\musicnote{
\begin{minipage}{0.48\textwidth}
\textbf{Intro}
\hfill 
%( \eighthnote \, 80)   % <<  MODIFICA IL TEMPO
% Metronomo: \eighthnote (ottavo) \quarternote (quarto) \halfnote (due quarti)
\end{minipage}
} 	
\vspace*{-\versesep}
\beginverse*
\nolyrics

%---- Prima riga -----------------------------
\vspace*{-\versesep}
\[E] \[B] \[A] 	 % \[*D] per indicare le pennate, \rep{2} le ripetizioni

%---- Ogni riga successiva -------------------
%\vspace*{-\versesep}
%\[G] \[C]  \[D]	

%---- Ev Indicazioni -------------------------			
%\textnote{\textit{(Oppure tutta la strofa)} }	

\endverse
\fi

%%%%% RITORNELLO
\beginchorus
\textnote{\textbf{Rit.}}

\[E]Lo \[B]Spirito di \[A]Cristo 
\[E] fa fio\[B]rire il de\[C#-]serto, 
\[G#-] torna la \[C#-]vita, 
\[A] noi diven\[B]tiamo testi\[E]moni di \[B]luce.

\endchorus

%%%%% STROFA
\beginverse		%Oppure \beginverse* se non si vuole il numero di fianco
\memorize 		% <<< DECOMMENTA se si vuole utilizzarne la funzione
%\chordsoff		% <<< DECOMMENTA se vuoi una strofa senza accordi

\[E] Non abbiamo rice\[B]vuto 
\[A] uno spirito di \[B]schiavitù,
\[E] ma uno spirito di a\[B]more, 
\[A] uno spirito di \[B]pace,
\[A] nel quale gri\[B]diamo 
\[A] abbà \[E]Padre,
\[A] abbà \[C#-]Pa\[B]dre.

\endverse

%%%%% STROFA
\beginverse		%Oppure \beginverse* se non si vuole il numero di fianco
%\memorize 		% <<< DECOMMENTA se si vuole utilizzarne la funzione
%\chordsoff		% <<< DECOMMENTA se vuoi una strofa senza accordi

^ Lo Spirito ^che 
^ Cristo ri^suscitò
^ darà vita ai nostri ^corpi, 
^ corpi mor^tali,  
^ e li rende^rà 
^ strumenti di sal^vezza, 
^ strumenti di sal^vez^za.

\endverse

%%%%% STROFA
\beginverse		%Oppure \beginverse* se non si vuole il numero di fianco
%\memorize 		% <<< DECOMMENTA se si vuole utilizzarne la funzione
\chordsoff		% <<< DECOMMENTA se vuoi una strofa senza accordi

Sono venuto a portare
il fuoco sulla terra
e come desidero
che divampi nel mondo 
e porti amore
ed entusiasmo in tutti i cuori.

\endverse

\endsong
%------------------------------------------------------------
%			FINE CANZONE
%------------------------------------------------------------

% %++++++++++++++++++++++++++++++++++++++++++++++++++++++++++++
% %			CANZONE TRASPOSTA
% %++++++++++++++++++++++++++++++++++++++++++++++++++++++++++++
% \ifchorded
% %decremento contatore per avere stesso numero
% \addtocounter{songnum}{-1} 
% \beginsong{Lo Spirito di Cristo}[by={}] 	% <<< COPIA TITOLO E AUTORE
% \transpose{3} 						% <<< TRASPOSIZIONE #TONI + - (0 nullo)
% %\preferflats  %SE VOGLIO FORZARE i bemolle come alterazioni
% %\prefersharps %SE VOGLIO FORZARE i # come alterazioni
% \ifchorded
% 	%\textnote{Tonalità originale}	% <<< EV COMMENTI (tonalità originale/migliore)
% \fi

% %%%%%% INTRODUZIONE
% \ifchorded
% \vspace*{\versesep}
% \textnote{Intro: \qquad \qquad  }%(\eighthnote 116) % <<  MODIFICA IL TEMPO
% % Metronomo: \eighthnote (ottavo) \quarternote (quarto) \halfnote (due quarti)
% \vspace*{-\versesep}
% \beginverse*

% \nolyrics

% %---- Prima riga -----------------------------
% \vspace*{-\versesep}
% \[E] \[B] \[A] 	 % \[*D] per indicare le pennate, \rep{2} le ripetizioni

% %---- Ogni riga successiva -------------------
% %\vspace*{-\versesep}
% %\[G] \[C]  \[D]	

% %---- Ev Indicazioni -------------------------			
% %\textnote{\textit{(Oppure tutta la strofa)} }	

% \endverse
% \fi

% %%%%% RITORNELLO
% \beginchorus
% \textnote{\textbf{Rit.}}

% \[E]Lo \[B]Spirito di \[A]Cristo 
% \[E] fa fio\[B]rire il de\[C#-]serto, 
% \[G#-] torna la \[C#-]vita, 
% \[A] noi diven\[B]tiamo testi\[E]moni di \[B]luce.

% \endchorus

% %%%%% STROFA
% \beginverse		%Oppure \beginverse* se non si vuole il numero di fianco
% \memorize 		% <<< DECOMMENTA se si vuole utilizzarne la funzione
% %\chordsoff		% <<< DECOMMENTA se vuoi una strofa senza accordi

% \[E] Non abbiamo rice\[B]vuto 
% \[A] uno spirito di \[B]schiavitù,
% \[E] ma uno spirito di a\[B]more, 
% \[A] uno spirito di \[B]pace,
% \[A] nel quale gri\[B]diamo 
% \[A] abbà \[E]Padre,
% \[A] abbà \[C#-]Pa\[B]dre.

% \endverse

% %%%%% STROFA
% \beginverse		%Oppure \beginverse* se non si vuole il numero di fianco
% %\memorize 		% <<< DECOMMENTA se si vuole utilizzarne la funzione
% %\chordsoff		% <<< DECOMMENTA se vuoi una strofa senza accordi

% ^ Lo Spirito ^che 
% ^ Cristo ri^suscitò
% ^ darà vita ai nostri ^corpi, 
% ^ corpi mor^tali,  
% ^ e li rende^rà 
% ^ strumenti di sal^vezza, 
% ^ strumenti di sal^vez^za.

% \endverse

% %%%%% STROFA
% \beginverse		%Oppure \beginverse* se non si vuole il numero di fianco
% %\memorize 		% <<< DECOMMENTA se si vuole utilizzarne la funzione
% \chordsoff		% <<< DECOMMENTA se vuoi una strofa senza accordi

% Sono venuto a portare
% il fuoco sulla terra
% e come desidero
% che divampi nel mondo 
% e porti amore
% ed entusiasmo in tutti i cuori.

% \endverse

% \endsong

% \fi
% %++++++++++++++++++++++++++++++++++++++++++++++++++++++++++++
% %			FINE CANZONE TRASPOSTA
% %++++++++++++++++++++++++++++++++++++++++++++++++++++++++++++
