%-------------------------------------------------------------
%			INIZIO	CANZONE
%-------------------------------------------------------------


%titolo: 	Come fuoco vivo
%autore: 	Gen Verde, Gen Rosso
%tonalita: 	Do



%%%%%% TITOLO E IMPOSTAZONI
\beginsong{Come fuoco vivo}[by={Gen\ Verde, Gen\ Rosso}] 	% <<< MODIFICA TITOLO E AUTORE
\transpose{0} 						% <<< TRASPOSIZIONE #TONI (0 nullo)
\momenti{Comunione; Cresima}							% <<< INSERISCI MOMENTI	
% momenti vanno separati da ; e vanno scelti tra:
% Ingresso; Atto penitenziale; Acclamazione al Vangelo; Dopo il Vangelo; Offertorio; Comunione; Ringraziamento; Fine; Santi; Pasqua; Avvento; Natale; Quaresima; Canti Mariani; Battesimo; Prima Comunione; Cresima; Matrimonio; Meditazione; Spezzare del pane;
\ifchorded
	%\textnote{Tonalità migliore }	% <<< EV COMMENTI (tonalità originale/migliore)
\fi




%%%%%% INTRODUZIONE
\ifchorded
\vspace*{\versesep}
\musicnote{
\begin{minipage}{0.48\textwidth}
\textbf{Intro}
\hfill 
%( \eighthnote \, 80)   % <<  MODIFICA IL TEMPO
% Metronomo: \eighthnote (ottavo) \quarternote (quarto) \halfnote (due quarti)
\end{minipage}
} 	
\vspace*{-\versesep}
\beginverse*


\nolyrics

%---- Prima riga -----------------------------
\vspace*{-\versesep}
\[C] \[C] \[D-*]\[C] \[C] % \[*D] per indicare le pennate, \rep{2} le ripetizioni

%---- Ogni riga successiva -------------------
\vspace*{-\versesep}
  \[G*] \[A-7] \[A-7]

%---- Ogni riga successiva -------------------
\vspace*{-\versesep}
 \[E-7*] \[F] \[F] \quad \[G]\[G]

%---- Ev Indicazioni -------------------------			
%\textnote{\textit{(Oppure tutta la strofa)} }	

\endverse
\fi






%%%%% RITORNELLO
\beginchorus
\textnote{\textbf{Rit.}}
Come \[C]fuoco \[G]vivo si ac\[A-]cende in \[A-]noi
un'im\[D-7]mensa \[G]felici\[C]tà \[C]
che mai \[F]più nes\[G]suno ci \[C]toglie\[F]rà
\[D-7]perché tu \[D-7]sei ritor\[G4]nato. \[G]
Chi po\[C]trà ta\[G]cere, da \[A-]ora in \[A-]poi,
che sei \[D-]tu in cam\[G]mino con \[C]noi, \[C]
che la \[F]morte è \[G]vinta per \[C]sempre,
\[F]che \[D-7]ci hai rido\[D-7]nato la \[G4]vita? \[G]
\endchorus



%%%%% STROFA
\beginverse		%Oppure \beginverse* se non si vuole il numero di fianco
\memorize 		% <<< DECOMMENTA se si vuole utilizzarne la funzione
%\chordsoff		% <<< DECOMMENTA se vuoi una strofa senza accordi
\[A-] Spezzi il \[A-]pane da\[F]vanti a \[C]noi \[C]
mentre il \[C]sole è al tra\[G4]monto: \[G]
\[G-]o\[G-]ra gli \[A]occhi ti \[D-]vedono, \[F] sei \[F]tu! 
\[G4]Resta con \[G]noi.
\endverse




%%%%% STROFA
\beginverse		%Oppure \beginverse* se non si vuole il numero di fianco
%\memorize 		% <<< DECOMMENTA se si vuole utilizzarne la funzione
%\chordsoff		% <<< DECOMMENTA se vuoi una strofa senza accordi
^ E per ^sempre ti ^mostre^rai ^
in quel ^gesto d'a^more: ^
^ma^ni che ^ancora ^spezzano ^ pa^ne d'^eterni^tà.
\endverse



%%%%%% EV. INTERMEZZO
\beginverse*
\vspace*{1.3\versesep}
{
	
	\textnote{\textbf{Finale} \textit{(humming)}}
	
	\ifchorded

	%---- Prima riga -----------------------------
	%\vspace*{-\versesep}
	\[C]Uhmmm... \[C] \[D-*]\[C] \[C] % \[*D] per indicare le pennate, \rep{2} le ripetizioni

	%---- Ogni riga successiva -------------------
	\vspace*{-\versesep}
	\nolyrics  \[G*] \[A-7] \[A-7] 

	%---- Ogni riga successiva -------------------
	\vspace*{-\versesep}
	  \[E-7*] \[F] \[F] \quad \[G] \quad \[C*]

	\else 
	\chordsoff Uhmmm...
	\fi
	%---- Ev Indicazioni -------------------------			
	%\textnote{\textit{(ripetizione della strofa)}} 
	 
}
\vspace*{\versesep}
\endverse



\endsong
%------------------------------------------------------------
%			FINE CANZONE
%------------------------------------------------------------


