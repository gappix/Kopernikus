%-------------------------------------------------------------
%			INIZIO	CANZONE
%-------------------------------------------------------------


%titolo: 	Salmo 8
%autore: 	Meregalli
%tonalita: 	Mi-



%%%%%% TITOLO E IMPOSTAZONI
\beginsong{Salmo 8}[by={G. Meregalli}]	% <<< MODIFICA TITOLO E AUTORE
\transpose{-2} 						% <<< TRASPOSIZIONE #TONI (0 nullo)
\preferflats 
\momenti{Ringraziamento}							% <<< INSERISCI MOMENTI	
% momenti vanno separati da ; e vanno scelti tra:
% Ingresso; Atto penitenziale; Acclamazione al Vangelo; Dopo il Vangelo; Offertorio; Comunione; Ringraziamento; Fine; Santi; Pasqua; Avvento; Natale; Quaresima; Canti Mariani; Battesimo; Prima Comunione; Cresima; Matrimonio; Meditazione; Spezzare del pane;
\ifchorded
	\textnote{$\bigstar$ Tonalità migliore }	% <<< EV COMMENTI (tonalità originale/migliore)
\fi


%%%%%% INTRODUZIONE
\ifchorded
\vspace*{\versesep}
\musicnote{
\begin{minipage}{0.48\textwidth}
\textbf{Intro}
\hfill 
( \eighthnote \, 76)   % <<  MODIFICA IL TEMPO
% Metronomo: \eighthnote (ottavo) \quarternote (quarto) \halfnote (due quarti)
\end{minipage}
} 	
\vspace*{-\versesep}
\beginverse*
\nolyrics

%---- Prima riga -----------------------------
\vspace*{-\versesep}
\[E-] \[E-] \[A-] \[E-]	 % \[*D] per indicare le pennate, \rep{2} le ripetizioni

%---- Ogni riga successiva -------------------
%\vspace*{-\versesep}
%\[G] \[C]  \[D]	

%---- Ev Indicazioni -------------------------			
%\textnote{\textit{(Oppure tutta la strofa)} }	

\endverse
\fi




%%%%% RITORNELLO
\beginchorus
\textnote{\textbf{Rit.}}
\[E-]Se guardo il c\[E-]ielo, \brk\[A-]  la luna e le \[E-]stelle,
opere che \[C]Tu  \brk con le \[D]dita hai model\[B-]lato,
\[B7] che cosa \[C]è,  \brk \[D]perché Te ne \[E-7]curi,
\[C7+] che cosa \[A-]è, \brk  per\[D7]ché te ne ri\[G*]cor\[B7*]di
\[E-]l’uo\[9]mo, \[C]l’uo\[A-7]mo,  \brk \[E-]l’uomo?
\endchorus




%%%%% STROFA
\beginverse*		%Oppure \beginverse* se non si vuole il numero di fianco
%\memorize 		% <<< DECOMMENTA se si vuole utilizzarne la funzione
%\chordsoff
\[(E-)]Eppure l’hai \[C]fatto \brk\[D] poco meno degli \[E-]angeli,
di \[(E-)]gloria e di o\[C]nore \brk\[D]  lo hai coro\[E-]nato
gli hai \[C]dato po\[D]tere \brk  sulle \[G]opere delle tue \[B7]mani,
su \[C]tutte le \[D]cose \brk  che \[G]tu avevi cre\[B7]ato:
gli u\[C]ccelli del \[E-]cielo, \brk  i \[C]pesci del \[E-]mare,
le \[C]greggi e gli ar\[E-]menti, \brk  gli ani\[A-*]mali \[B7*]della cam\[E-]pagna.
\endverse






\endsong
%------------------------------------------------------------
%			FINE CANZONE
%------------------------------------------------------------

%++++++++++++++++++++++++++++++++++++++++++++++++++++++++++++
%			CANZONE TRASPOSTA
%++++++++++++++++++++++++++++++++++++++++++++++++++++++++++++
\ifchorded
%decremento contatore per avere stesso numero
\addtocounter{songnum}{-1} 
\beginsong{Salmo 8}[by={Meregalli}]	% <<< COPIA TITOLO E AUTORE
\transpose{0} 						% <<< TRASPOSIZIONE #TONI + - (0 nullo)
%\preferflats  %SE VOGLIO FORZARE i bemolle come alterazioni
%\prefersharps %SE VOGLIO FORZARE i # come alterazioni
\ifchorded
	\textnote{$\lozenge$ Tonalità originale}	% <<< EV COMMENTI (tonalità originale/migliore)
\fi


%%%%%% INTRODUZIONE
\ifchorded
\vspace*{\versesep}
\musicnote{
\begin{minipage}{0.48\textwidth}
\textbf{Intro}
\hfill 
( \eighthnote \, 76)   % <<  MODIFICA IL TEMPO
% Metronomo: \eighthnote (ottavo) \quarternote (quarto) \halfnote (due quarti)
\end{minipage}
} 	
\vspace*{-\versesep}
\beginverse*

\nolyrics

%---- Prima riga -----------------------------
\vspace*{-\versesep}
\[E-] \[E-] \[A-] \[E-]	 % \[*D] per indicare le pennate, \rep{2} le ripetizioni

%---- Ogni riga successiva -------------------
%\vspace*{-\versesep}
%\[G] \[C]  \[D]	

%---- Ev Indicazioni -------------------------			
%\textnote{\textit{(Oppure tutta la strofa)} }	

\endverse
\fi




%%%%% RITORNELLO
\beginchorus
\textnote{\textbf{Rit.}}
\[E-]Se guardo il c\[E-]ielo, \brk\[A-]  la luna e le \[E-]stelle,
opere che \[C]Tu  \brk con le \[D]dita hai model\[B-]lato,
\[B7] che cosa \[C]è,  \brk \[D]perché Te ne \[E-7]curi,
\[C7+] che cosa \[A-]è, \brk  per\[D7]ché te ne ri\[G*]cor\[B7*]di
\[E-]l’uo\[9]mo, \[C]l’uo\[A-7]mo,  \brk \[E-]l’uomo?
\endchorus




%%%%% STROFA
\beginverse		%Oppure \beginverse* se non si vuole il numero di fianco
%\memorize 		% <<< DECOMMENTA se si vuole utilizzarne la funzione
%\chordsoff
\[(E-)]Eppure l’hai \[C]fatto \brk\[D] poco meno degli \[E-]angeli,
di \[(E-)]gloria e di o\[C]nore \brk\[D]  lo hai coro\[E-]nato
gli hai \[C]dato po\[D]tere \brk  sulle \[G]opere delle tue \[B7]mani,
su \[C]tutte le \[D]cose \brk  che \[G]tu avevi cre\[B7]ato:
gli u\[C]ccelli del \[E-]cielo, \brk  i \[C]pesci del \[E-]mare,
le \[C]greggi e gli ar\[E-]menti, \brk  gli ani\[A-*]mali \[B7*]della cam\[E-]pagna.
\endverse



\endsong
\fi
%++++++++++++++++++++++++++++++++++++++++++++++++++++++++++++
%			FINE CANZONE TRASPOSTA
%++++++++++++++++++++++++++++++++++++++++++++++++++++++++++++
