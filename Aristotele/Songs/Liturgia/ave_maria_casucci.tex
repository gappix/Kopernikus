
%-------------------------------------------------------------
%			INIZIO	CANZONE
%-------------------------------------------------------------


%titolo: 	Ave Maria
%autore: 	Casucci, Balduzzi
%tonalita: 	Re 


%%%%%% TITOLO E IMPOSTAZONI
\beginsong{Ave Maria}[by={C. Casucci, M. Balduzzi}] 	% <<< MODIFICA TITOLO E AUTORE
\transpose{0} 						% <<< TRASPOSIZIONE #TONI (0 nullo)
\momenti{Canti Mariani; Ringraziamento}							% <<< INSERISCI MOMENTI	
% momenti vanno separati da ; e vanno scelti tra:
% Ingresso; Atto penitenziale; Acclamazione al Vangelo; Dopo il Vangelo; Offertorio; Comunione; Ringraziamento; Fine; Santi; Pasqua; Avvento; Natale; Quaresima; Canti Mariani; Battesimo; Prima Comunione; Cresima; Matrimonio; Meditazione;
\ifchorded
	%\textnote{Tonalità originale }	% <<< EV COMMENTI (tonalità originale/migliore)
\fi


%%%%%% INTRODUZIONE
\ifchorded
\vspace*{\versesep}
\musicnote{
\begin{minipage}{0.48\textwidth}
\textbf{Intro}
\hfill 
(\quarternote \, 72)
%( \eighthnote \, 80)   % <<  MODIFICA IL TEMPO
% Metronomo: \eighthnote (ottavo) \quarternote (quarto) \halfnote (due quarti)
\end{minipage}
} 
\musicnote{\textit{[dolce, arpeggiato]}}	
\vspace*{-\versesep}
\beginverse*


\nolyrics

%---- Prima riga -----------------------------
\vspace*{-\versesep}
\[D]\[A]\[B-]\[G]	 % \[*D] per indicare le pennate, \rep{2} le ripetizioni

%---- Ogni riga successiva -------------------
\vspace*{-\versesep}
\[D]\[A]\[E-] \[G]

%---- Ev Indicazioni -------------------------			
\textnote{\textit{[come mezzo ritornello]} }	

\endverse
\fi




%%%%% RITORNELLO
\textnote{\textbf{Rit.}}
\beginchorus

\[D]A\[A]ve Ma\[B-]ria, \[G] 
\[D]\[A]a\[E-]ve, \[G]
\[D]a\[A]ve Ma\[B-]ria, \[G] 
\[D]\[A]a\[D4]ve. \[D]

\endchorus



%%%%% STROFA
\beginverse
\memorize
\[D]Donna dell'at\[D]tesa e \[B-]madre di spe\[B-]ranza
\[A]ora pro no\[G]bis.
\[D]Donna del sor\[D]riso e \[B-]madre del si\[B-]lenzio
\[A]ora pro no\[G]bis.
\[D]Donna di fron\[D]tiera e \[A]madre dell'ar\[A]dore
\[B-]ora pro no\[G]bis.
\[D]Donna del ri\[D]poso e \[A]madre del sen\[A]tiero
\[G]ora pro no\[A]bis.
\endverse




%%%%% STROFA
\beginverse
^Donna del de^serto e ^madre del re^spiro
^ora pro no^bis.
^Donna della ^sera e ^madre del ri^cordo
^ora pro no^bis.
^Donna del pre^sente e ^madre del ri^torno
^ora pro no^bis.
^Donna della ^terra e ^madre dell'a^more
^ora pro no^bis.
\endverse


\endsong
%------------------------------------------------------------
%			FINE CANZONE
%------------------------------------------------------------



