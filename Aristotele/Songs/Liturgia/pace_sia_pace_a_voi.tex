%-------------------------------------------------------------
%			INIZIO	CANZONE
%-------------------------------------------------------------


%titolo: 	Pace sia, pace a voi
%autore: 	Gen Verde, Gen Rosso
%tonalita: 	Mi 



%%%%%% TITOLO E IMPOSTAZONI
\beginsong{Pace sia, pace a voi}[by={Gen Verde, Gen Rosso}] 	% <<< MODIFICA TITOLO E AUTORE
\transpose{0} 						% <<< TRASPOSIZIONE #TONI (0 nullo)
\momenti{Offertorio}							% <<< INSERISCI MOMENTI	
% momenti vanno separati da ; e vanno scelti tra:
% Ingresso; Atto penitenziale; Acclamazione al Vangelo; Dopo il Vangelo; Offertorio; Comunione; Ringraziamento; Fine; Santi; Pasqua; Avvento; Natale; Quaresima; Canti Mariani; Battesimo; Prima Comunione; Cresima; Matrimonio; Meditazione;
\ifchorded
	%\textnote{Tonalità originale }	% <<< EV COMMENTI (tonalità originale/migliore)
\fi

%%%%%% INTRODUZIONE
\ifchorded
\vspace*{\versesep}
\musicnote{
\begin{minipage}{0.48\textwidth}
\textbf{Intro}
\hfill 
%( \eighthnote \, 80)   % <<  MODIFICA IL TEMPO
% Metronomo: \eighthnote (ottavo) \quarternote (quarto) \halfnote (due quarti)
\end{minipage}
} 	
\vspace*{-\versesep}
\beginverse*

\nolyrics

%---- Prima riga -----------------------------
\vspace*{-\versesep}
\[E] \[A] \[E] \[B] \[E] \[A] \[E]	 % \[*D] per indicare le pennate, \rep{2} le ripetizioni

%---- Ogni riga successiva -------------------
%\vspace*{-\versesep}
%\[G] \[C]  \[D]	

%---- Ev Indicazioni -------------------------			
%\textnote{\textit{(Oppure tutta la strofa)} }	

\endverse
\fi

%%%%% RITORNELLO
\beginchorus
\textnote{\textbf{Rit.}}

“Pace \[E]sia, pace a voi”: la tua \[A]pace sarà
sulla \[C#-]terra com'è nei \[B]cieli.
“Pace \[E]sia, pace a voi”: la tua \[A]pace sarà
gioia \[G]nei nostri \[D]occhi, nei \[A]cuo\[B]ri.
“Pace \[E]sia, pace a voi”: la tua \[A]pace sarà
luce \[C#-]limpida nei pen\[B]sieri.
“Pace \[E]sia, pace a voi”: la tua \[A]pace sarà
una \[E]casa per \[B]tutti. \[E]\[A]\[E]

\endchorus

%%%%% STROFA
\beginverse		%Oppure \beginverse* se non si vuole il numero di fianco
\memorize 		% <<< DECOMMENTA se si vuole utilizzarne la funzione
%\chordsoff		% <<< DECOMMENTA se vuoi una strofa senza accordi

“\[A]Pace a \[E]voi”: sia il tuo \[B]dono vi\[C#-]sibile.
“\[A]Pace a \[E]voi”: la tua e\[B]redi\[C#-]tà.
“\[A]Pace a \[E]voi”: come un \[B]canto all'u\[C#-]nisono
che \[D]sale dalle nostre cit\[B]tà.

\endverse

%%%%% STROFA
\beginverse		%Oppure \beginverse* se non si vuole il numero di fianco
%\memorize 		% <<< DECOMMENTA se si vuole utilizzarne la funzione
%\chordsoff		% <<< DECOMMENTA se vuoi una strofa senza accordi

“^Pace a ^voi”: sia un'im^pronta nei ^secoli.
“^Pace a ^voi”: segno d'^uni^tà.
“^Pace a ^voi”: sia l'ab^braccio tra i ^popoli,
la ^tua promessa all'umani^tà.

\endverse

\endsong
%------------------------------------------------------------
%			FINE CANZONE
%------------------------------------------------------------
