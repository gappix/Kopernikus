%-------------------------------------------------------------
%			INIZIO	CANZONE
%-------------------------------------------------------------


%titolo: 	Spirito Santo, discendi tra noi
%autore: 	
%tonalita: 	Do 



%%%%%% TITOLO E IMPOSTAZONI  O forse (Gesù risuscitò) Sequeri
\beginsong{Lo spirito di Dio (dal cielo scenda)}[by={G. Maggio, F. Filsetti}] 	% <<< MODIFICA TITOLO E AUTORE
\transpose{0} 						% <<< TRASPOSIZIONE #TONI (0 nullo)
%\preferflats  %SE VOGLIO FORZARE i bemolle come alterazioni
%\prefersharps %SE VOGLIO FORZARE i # come alterazioni
\momenti{Cresima}							% <<< INSERISCI MOMENTI	
% momenti vanno separati da ; e vanno scelti tra:
% Ingresso; Atto penitenziale; Acclamazione al Vangelo; Dopo il Vangelo; Offertorio; Comunione; Ringraziamento; Fine; Santi; Pasqua; Avvento; Natale; Quaresima; Canti Mariani; Battesimo; Prima Comunione; Cresima; Matrimonio; Meditazione; Spezzare del pane;
\ifchorded
	%\textnote{Tonalità migliore }	% <<< EV COMMENTI (tonalità originale/migliore)
\fi


%%%%%% INTRODUZIONE
\ifchorded
\vspace*{\versesep}
\textnote{Intro: \qquad \qquad  }%(\eighthnote 116) % <<  MODIFICA IL TEMPO
% Metronomo: \eighthnote (ottavo) \quarternote (quarto) \halfnote (due quarti)
\vspace*{-\versesep}
\beginverse*

\nolyrics

%---- Prima riga -----------------------------
\vspace*{-\versesep}
\[C] \[A-] \[F] \[C]	 % \[*D] per indicare le pennate, \rep{2} le ripetizioni

%---- Ogni riga successiva -------------------
%\vspace*{-\versesep}
%\[G] \[C]  \[D]	

%---- Ev Indicazioni -------------------------			
%\textnote{\textit{(Oppure tutta la strofa)} }	

\endverse
\fi








%%%%% RITORNELLO
\beginchorus
\textnote{\textbf{Rit.}}

Lo \[]Spirito di \[]Dio, Ge\[]sù risusci\[]tò:
la \[]nostra vita un \[]giorno
con \[]Lui \[]risorge\[]rà! \[]
Lo \[]Spirito di \[]Dio è a\[]more ed umiltà:
è il dono del Signore la nostra libertà!


\endchorus



%%%%% STROFA
\beginverse		%Oppure \beginverse* se non si vuole il numero di fianco
\memorize 		% <<< DECOMMENTA se si vuole utilizzarne la funzione
%\chordsoff		% <<< DECOMMENTA se vuoi una strofa senza accordi

I segni dello Spirito son gioia e pace,
coraggio di sperare, volontà d’amare;
per mano ti conduce lungo il tuo cammino:
un cuore nuovo avrai!


\endverse

%%%%% STROFA
\beginverse		%Oppure \beginverse* se non si vuole il numero di fianco
%\memorize 		% <<< DECOMMENTA se si vuole utilizzarne la funzione
\chordsoff		% <<< DECOMMENTA se vuoi una strofa senza accordi

La vita fa rinascere dov’è la morte,
riporta la fiducia dove c’è il dolore;
la forza sa trovarti di una mano amica:
perdono e pace avrai!


\endverse



%%%%% STROFA
\beginverse		%Oppure \beginverse* se non si vuole il numero di fianco
%\memorize 		% <<< DECOMMENTA se si vuole utilizzarne la funzione
\chordsoff		% <<< DECOMMENTA se vuoi una strofa senza accordi

Di molti nell’amore fa una cosa sola,
al Padre ch’è nei cieli porta la tua voce;
il nome del Signore che ce l’ha donato,
a tutti annuncerai!

\endverse

\endsong
%------------------------------------------------------------
%			FINE CANZONE
%------------------------------------------------------------


