%-------------------------------------------------------------
%			INIZIO	CANZONE
%-------------------------------------------------------------


%titolo: 	Ti do La pace
%autore: 	SERMIG
%tonalita: 	Fa



%%%%%% TITOLO E IMPOSTAZONI
\beginsong{Ti do la pace}[by={Sermig}] 	% <<< MODIFICA TITOLO E AUTORE
\transpose{0} 						% <<< TRASPOSIZIONE #TONI (0 nullo)
%\preferflats  %SE VOGLIO FORZARE i bemolle come alterazioni
%\prefersharps %SE VOGLIO FORZARE i # come alterazioni
\momenti{Offertorio}							% <<< INSERISCI MOMENTI	
% momenti vanno separati da ; e vanno scelti tra:
% Ingresso; Atto penitenziale; Acclamazione al Vangelo; Dopo il Vangelo; Offertorio; Comunione; Ringraziamento; Fine; Santi; Pasqua; Avvento; Natale; Quaresima; Canti Mariani; Battesimo; Prima Comunione; Cresima; Matrimonio; Meditazione; Spezzare del pane;
\ifchorded
	%\textnote{$\bigstar$ Tonalità migliore }	% <<< EV COMMENTI (tonalità originale\migliore)
\fi


%%%%%% INTRODUZIONE
\ifchorded
\vspace*{\versesep}
\musicnote{
\begin{minipage}{0.48\textwidth}
\textbf{Intro}
\hfill 
%( \eighthnote \, 80)   % <<  MODIFICA IL TEMPO
% Metronomo: \eighthnote (ottavo) \quarternote (quarto) \halfnote (due quarti)
\end{minipage}
} 	
\vspace*{-\versesep}
\beginverse*

\nolyrics

%---- Prima riga -----------------------------
\vspace*{-\versesep}
\[B&] \[C] \[F]	 % \[*D] per indicare le pennate, \rep{2} le ripetizioni

%---- Ogni riga successiva -------------------
%\vspace*{-\versesep}
%\[G] \[C]  \[D]	

%---- Ev Indicazioni -------------------------			
%\textnote{\textit{[oppure tutta la strofa]} }	

\endverse
\fi




%%%%% STROFA
\beginverse		%Oppure \beginverse* se non si vuole il numero di fianco
\memorize 		% <<< DECOMMENTA se si vuole utilizzarne la funzione
%\chordsoff		% <<< DECOMMENTA se vuoi una strofa senza accordi

Ti do la \[B&]pace \[C]perché ci cre\[F]do 
ti do la \[B&]pace \[C]perché la viv\[F]o 
ti do la \[D-]pace \[C]perché la vogli\[B&]o
per t\[G-]e e per tutte le don\[A4]ne 
e tu\[G-]tti gli uomini del \[A]mondo 
ti do la \[B&]pace \[C]perché ci cred\[F]o.

\endverse


%%%%% STROFA
\beginverse*		%Oppure \beginverse* se non si vuole il numero di fianco
\memorize 		% <<< DECOMMENTA se si vuole utilizzarne la funzione
%\chordsoff		% <<< DECOMMENTA se vuoi una strofa senza accordi

Ti do la \[B&]pace \[C]perché ci cred\[F]o 
ti do la \[B&]pace \[C]perché la viv\[F]o 
ti do la p\[D-]ace \[C]perché la vo\[B&]glio 
per \[G-]te e per tutte le \[A4]donne
e \[G-]tutti gli uomini del \[A4]mondo


\endverse




%%%%% STROFA
\beginverse*		%Oppure \beginverse* se non si vuole il numero di fianco
%\memorize 		% <<< DECOMMENTA se si vuole utilizzarne la funzione
%\chordsoff		% <<< DECOMMENTA se vuoi una strofa senza accordi
Ti do la \[B&]pace \[C]perché io spero ch\[F]e 
la pace \[B&]possa \[C]abitare semp\[F]re 
\[D-]e nel creato e in \[C]tutte le crea\[B&]ture.

\endverse


%%%%% STROFA
\beginverse*		%Oppure \beginverse* se non si vuole il numero di fianco
%\memorize 		% <<< DECOMMENTA se si vuole utilizzarne la funzione
%\chordsoff		% <<< DECOMMENTA se vuoi una strofa senza accordi

Ti do la \[B&]pace \[C]perché ci cred\[F]o, 
ti do la \[B&]pace \[C]perché ci cred\[F]o, 
ti do la \[B&]pace, \[C]la voglio anche per \[F]te.  


\endverse



\endsong
%------------------------------------------------------------
%			FINE CANZONE
%------------------------------------------------------------

