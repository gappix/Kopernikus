%-------------------------------------------------------------
%			INIZIO	CANZONE
%-------------------------------------------------------------


%titolo: 	Santo Ricci
%autore: 	Daniele Ricci
%tonalita: 	Sol 



%%%%%% TITOLO E IMPOSTAZONI
\beginsong{Fammi conoscere}[by={P. Ruaro}] 	% <<< MODIFICA TITOLO E AUTORE
\transpose{-2} 						% <<< TRASPOSIZIONE #TONI (0 nullo)
\momenti{Ingresso; Acclamazione al Vangelo; Salmi; Quaresima}							% <<< INSERISCI MOMENTI	
% momenti vanno separati da ; e vanno scelti tra:
% Ingresso; Atto penitenziale; Acclamazione al Vangelo; Dopo il Vangelo; Offertorio; Comunione; Ringraziamento; Fine; Santi; Pasqua; Avvento; Natale; Quaresima; Canti Mariani; Battesimo; Prima Comunione; Cresima; Matrimonio; Meditazione; Spezzare del pane;
\ifchorded
	\textnote{$\bigstar$ Tonalità migliore }	% <<< EV COMMENTI (tonalità originale/migliore)
\fi


%%%%%% INTRODUZIONE
\ifchorded
\vspace*{\versesep}
\musicnote{
\begin{minipage}{0.48\textwidth}
\textbf{Intro}
\hfill 
%( \eighthnote \, 80)   % <<  MODIFICA IL TEMPO
% Metronomo: \eighthnote (ottavo) \quarternote (quarto) \halfnote (due quarti)
\end{minipage}
} 	
\vspace*{-\versesep}
\beginverse*


\nolyrics

%---- Prima riga -----------------------------
\vspace*{-\versesep}
\[E] \[F#-] \[A] \[E] % \[*D] per indicare le pennate, \rep{2} le ripetizioni

%---- Ogni riga successiva -------------------
%\vspace*{-\versesep}
%\[G] \[C]  \[D]	

%---- Ev Indicazioni -------------------------			
%\textnote{\textit{(Oppure tutta la strofa)} }	

\endverse
\fi




%%%%% RITORNELLO
\beginchorus
\textnote{\textbf{Rit.}}

\[E]Fammi co\[F#-]noscere la \[Gdim/(A)]tua volon\[G#-]tà,
\[A]parla, Ti as\[E]colto, Si\[F#-]gno\[B]re !
\[A] La mia fe\[B]licità è \[G#]fare il tuo vo\[C#-7]lere:
\[A*]porte\[B7*]rò con \[E*]me la \[A*]tua Pa\[B7]ro\[E]la. \[E*] \quad \[D*]


\endchorus


\preferflats
%%%%% STROFA
\beginverse		%Oppure \beginverse* se non si vuole il numero di fianco
\memorize 		% <<< DECOMMENTA se si vuole utilizzarne la funzione
%\chordsoff		% <<< DECOMMENTA se vuoi una strofa senza accordi
\vspace*{1.3\versesep}
\musicnote{ \textit{(dolce, arpeggiato)}} %<<< EV. INDICAZIONI
\[G]Lampada ai miei \[D]passi 
\[C]è la tua Pa\[D]rola
\[E-]luce sul \[C]mio cam\[G]mi\[D]no.
\[G]Ogni gi\[D]orno \[C]la mia volon\[D]tà
\[C]trova una guida in \[G]te. \[B]\[7]

\endverse







%%%%% STROFA
\beginverse		%Oppure \beginverse* se non si vuole il numero di fianco
%\memorize 		% <<< DECOMMENTA se si vuole utilizzarne la funzione
%\chordsoff		% <<< DECOMMENTA se vuoi una strofa senza accordi

^Porterò con ^me 
i ^tuoi insegna^menti:
^danno al mio ^cuore gi^o^ia!
^La tua Pa^rola è f^onte di ^luce:
^dona saggezza ai ^sempli^ci. ^

\endverse





%%%%% STROFA
\beginverse		%Oppure \beginverse* se non si vuole il numero di fianco
%\memorize 		% <<< DECOMMENTA se si vuole utilizzarne la funzione
%\chordsoff		% <<< DECOMMENTA se vuoi una strofa senza accordi

^La mia ^bocca 
im^pari la tua ^lode
^sempre ti ^renda gr^az^ie.
^Ogni mo^mento ^canti la tua ^lode
^la mia speranza è in ^te. ^ ^ 

\endverse









\endsong
%------------------------------------------------------------
%			FINE CANZONE
%------------------------------------------------------------

%++++++++++++++++++++++++++++++++++++++++++++++++++++++++++++
%			CANZONE TRASPOSTA
%++++++++++++++++++++++++++++++++++++++++++++++++++++++++++++
\ifchorded
%decremento contatore per avere stesso numero
\addtocounter{songnum}{-1} 
\beginsong{Fammi conoscere}[by={P. Ruaro}] 	% <<< MODIFICA TITOLO E AUTORE
\transpose{0} 						% <<< TRASPOSIZIONE #TONI + - (0 nullo)
%\preferflats  %SE VOGLIO FORZARE i bemolle come alterazioni
%\prefersharps %SE VOGLIO FORZARE i # come alterazioni
\ifchorded
	\textnote{$\lozenge$ Tonalità originale}	% <<< EV COMMENTI (tonalità originale/migliore)
\fi



%%%%%% INTRODUZIONE
\ifchorded
\vspace*{\versesep}
\musicnote{
\begin{minipage}{0.48\textwidth}
\textbf{Intro}
\hfill 
%( \eighthnote \, 80)   % <<  MODIFICA IL TEMPO
% Metronomo: \eighthnote (ottavo) \quarternote (quarto) \halfnote (due quarti)
\end{minipage}
} 	
\vspace*{-\versesep}
\beginverse*


\nolyrics

%---- Prima riga -----------------------------
\vspace*{-\versesep}
\[E] \[F#-] \[A] \[E] % \[*D] per indicare le pennate, \rep{2} le ripetizioni

%---- Ogni riga successiva -------------------
%\vspace*{-\versesep}
%\[G] \[C]  \[D]	

%---- Ev Indicazioni -------------------------			
%\textnote{\textit{(Oppure tutta la strofa)} }	

\endverse
\fi




%%%%% RITORNELLO
\beginchorus
\textnote{\textbf{Rit.}}

\[E]Fammi co\[F#-]noscere la \[Gdim/(A)]tua volon\[G#-]tà,
\[A]parla, Ti as\[E]colto, Si\[F#-]gno\[B]re !
\[A] La mia fe\[B]licità è \[G#]fare il tuo vo\[C#-7]lere:
\[A*]porte\[B7*]rò con \[E*]me la \[A*]tua Pa\[B7]ro\[E]la. \[E*] \quad \[D*]


\endchorus


%%%%% STROFA
\beginverse		%Oppure \beginverse* se non si vuole il numero di fianco
\memorize 		% <<< DECOMMENTA se si vuole utilizzarne la funzione
%\chordsoff		% <<< DECOMMENTA se vuoi una strofa senza accordi
\vspace*{1.3\versesep}
\musicnote{ \textit{[dolce, arpeggiato]}} %<<< EV. INDICAZIONI
\[G]Lampada ai miei \[D]passi 
\[C]è la tua Pa\[D]rola
\[E-]luce sul \[C]mio cam\[G]mi\[D]no.
\[G]Ogni gi\[D]orno \[C]la mia volon\[D]tà
\[C]trova una guida in \[G]te. \[B]\[7]

\endverse








%%%%% STROFA
\beginverse		%Oppure \beginverse* se non si vuole il numero di fianco
%\memorize 		% <<< DECOMMENTA se si vuole utilizzarne la funzione
%\chordsoff		% <<< DECOMMENTA se vuoi una strofa senza accordi

^Porterò con ^me 
i ^tuoi insegna^menti:
^danno al mio ^cuore gi^o^ia!
^La tua Pa^rola è f^onte di ^luce:
^dona saggezza ai ^sempli^ci. ^

\endverse





%%%%% STROFA
\beginverse		%Oppure \beginverse* se non si vuole il numero di fianco
%\memorize 		% <<< DECOMMENTA se si vuole utilizzarne la funzione
%\chordsoff		% <<< DECOMMENTA se vuoi una strofa senza accordi

^La mia ^bocca 
im^pari la tua ^lode
^sempre ti ^renda gr^az^ie.
^Ogni mo^mento ^canti la tua ^lode
^la mia speranza è in ^te. ^ ^ 

\endverse









\endsong
\fi
%++++++++++++++++++++++++++++++++++++++++++++++++++++++++++++
%			FINE CANZONE TRASPOSTA
%++++++++++++++++++++++++++++++++++++++++++++++++++++++++++++

