%-------------------------------------------------------------
%			INIZIO	CANZONE
%-------------------------------------------------------------


%titolo: 	Sulla tua parola
%autore: 	Paci Varnavà
%tonalita: 	La- / Re- 



%%%%%% TITOLO E IMPOSTAZONI
\beginsong{Sulla tua parola}[ititle={ Pietro vai},by={ Pietro vai — P. Paci, S. Varnavà}]
%\transpose{5}\preferflats						% <<< TRASPOSIZIONE #TONI (0 nullo)
\momenti{Comunione; Congedo}							% <<< INSERISCI MOMENTI	
% momenti vanno separati da ; e vanno scelti tra:
% Ingresso; Atto penitenziale; Acclamazione al Vangelo; Dopo il Vangelo; Offertorio; Comunione; Ringraziamento; Fine; Santi; Pasqua; Avvento; Natale; Quaresima; Canti Mariani; Battesimo; Prima Comunione; Cresima; Matrimonio; Meditazione;
\ifchorded
	\textnote{$\bigstar$ Tonalità migliore}	% <<< EV COMMENTI (tonalità originale/migliore)
\fi


%%%%%% INTRODUZIONE
\ifchorded
\vspace*{\versesep}
\musicnote{
\begin{minipage}{0.48\textwidth}
\textbf{Intro}
\hfill 
%( \eighthnote \, 80)   % <<  MODIFICA IL TEMPO
% Metronomo: \eighthnote (ottavo) \quarternote (quarto) \halfnote (due quarti)
\end{minipage}
} 	
\vspace*{-\versesep}
\beginverse*

\nolyrics

%---- Prima riga -----------------------------
\vspace*{-\versesep}
\[A-] \[D-]  \[A-]	 % \[*D] per indicare le pennate, \rep{2} le ripetizioni

%---- Ogni riga successiva -------------------
%\vspace*{-\versesep}
%\[G] \[C]  \[D]	

%---- Ev Indicazioni -------------------------			
%\textnote{\textit{(Oppure tutta la strofa)} }	

\endverse
\fi




%%%%% STROFA
\beginverse		%Oppure \beginverse* se non si vuole il numero di fianco
\memorize 		% <<< DECOMMENTA se si vuole utilizzarne la funzione
%\chordsoff		& <<< DECOMMENTA se vuoi una strofa senza accordi

Si\[A-]gnore, ho pe\[D-]scato tutto il \[A-]giorno,
le \[F]reti son ri\[G]maste sempre \[C]vuote, \[E]
s'è \[D-]fatto \[E]tardi, a \[A-*]casa \[E*]ora ri\[A-]torno,
Si\[F]gnore, son de\[G]luso e me ne \[A-]vado.
\endverse



\beginverse*
La ^vita con ^me è ^sempre stata \[E]dura
e ^niente mai mi ^dà soddisfa^zione, ^
la ^strada in cui mi ^guidi è ^in^si^cura,
sono ^stanco e ora ^non aspetto \[E]più.
\endverse



%%%%% RITORNELLO
\beginchorus
\textnote{\textbf{Rit.}}
\[A]Pietro, \[E]vai, \[F#-]fidati di \[C#-]me,
\[D]getta ancora in \[A]acqua le tue \[E]reti,
\[F#-]prendi ancora il \[C#-]largo, sulla \[D]mia pa\[A]rola,
\[D]con la mia po\[A]tenza, \[B-]io ti fa\[F#-]rò \[D]
pesca\[E]tore di \[D]uomi\[A]ni.
\endchorus




%%%%% STROFA
\beginverse
Ma^estro, dimmi ^cosa devo ^fare,
in^segnami, Si^gnore, dove an^dare, ^
Ge^sù, dammi la ^forza ^di ^par^tire,
la ^forza di la^sciare le mie ^cose:
\endverse



\beginverse*
que^sta fa^miglia che mi ^son cre\[E]ato,
le ^barche che a fa^tica ho conqui^stato, ^
la ^casa, la mia ^terra, la ^mi^a ^gente,
Si^gnore, dammi ^tu una fede \[E]forte.
\endverse


%%%%% RITORNELLO
\beginchorus
\textnote{\textbf{Rit.}}
\[A]Pietro, \[E]vai, \[F#-]fidati di \[C#-]me,
\[D]la mia Chiesa \[A]su te fonde\[E]rò;
\[F#-]manderò lo \[C#-]Spirito, ti \[D]darà co\[A]raggio,
\[D]donerà la \[A]forza \[B-]dell'a\[F#-]mor \[D]
per il \[E]Regno di \[D]Di\[A]o.
\endchorus

\beginchorus
\[A]Pietro, \[E]vai, \[F#-]fidati di \[C#-]me,
\[D]getta ancora in \[A]acqua le tue \[E]reti,
\[F#-]prendi ancora il \[C#-]largo, sulla \[D]mia pa\[A]rola,
\[D]con la mia po\[A]tenza, \[B-]io ti fa\[F#-]rò \[D]
pesca\[E]tore di \[D]uomi\[A]ni.
\endchorus

\endsong
%------------------------------------------------------------
%			FINE CANZONE
%------------------------------------------------------------





%++++++++++++++++++++++++++++++++++++++++++++++++++++++++++++
%			CANZONE TRASPOSTA
%++++++++++++++++++++++++++++++++++++++++++++++++++++++++++++
\ifchorded
%decremento contatore per avere stesso numero
\addtocounter{songnum}{-1} 
\beginsong{Sulla tua parola }[by={Pietro vai — P. Paci, S. Varnavà}]
\transpose{5}\prefersharps						% <<< TRASPOSIZIONE #TONI (0 nullo)
%\preferflats SE VOGLIO FORZARE i bemolle come alterazioni
%\prefersharps SE VOGLIO FORZARE i # come alterazioni
\ifchorded
	\textnote{$\lozenge$ Tonalità originale}	% <<< EV COMMENTI (tonalità originale/migliore)
\fi



%%%%%% INTRODUZIONE
\ifchorded
\vspace*{\versesep}
\musicnote{
\begin{minipage}{0.48\textwidth}
\textbf{Intro}
\hfill 
%( \eighthnote \, 80)   % <<  MODIFICA IL TEMPO
% Metronomo: \eighthnote (ottavo) \quarternote (quarto) \halfnote (due quarti)
\end{minipage}
} 	
\vspace*{-\versesep}
\beginverse*

\nolyrics

%---- Prima riga -----------------------------
\vspace*{-\versesep}
\[A-] \[D-]  \[A-]	 % \[*D] per indicare le pennate, \rep{2} le ripetizioni

%---- Ogni riga successiva -------------------
%\vspace*{-\versesep}
%\[G] \[C]  \[D]	

%---- Ev Indicazioni -------------------------			
%\textnote{\textit{(Oppure tutta la strofa)} }	

\endverse
\fi




%%%%% STROFA
\beginverse		%Oppure \beginverse* se non si vuole il numero di fianco
\memorize 		% <<< DECOMMENTA se si vuole utilizzarne la funzione
%\chordsoff		& <<< DECOMMENTA se vuoi una strofa senza accordi

Si\[A-]gnore, ho pe\[D-]scato tutto il \[A-]giorno,
le \[F]reti son ri\[G]maste sempre \[C]vuote, \[E]
s'è \[D-]fatto \[E]tardi, a \[A-*]casa \[E*]ora ri\[A-]torno,
Si\[F]gnore, son de\[G]luso e me ne \[A-]vado.
\endverse



\beginverse*
La ^vita con ^me è ^sempre stata \[E]dura
e ^niente mai mi ^dà soddisfa^zione, ^
la ^strada in cui mi ^guidi è ^in^si^cura,
sono ^stanco e ora ^non aspetto \[E]più.
\endverse



%%%%% RITORNELLO
\beginchorus
\textnote{\textbf{Rit.}}
\[A]Pietro, \[E]vai, \[F#-]fidati di \[C#-]me,
\[D]getta ancora in \[A]acqua le tue \[E]reti,
\[F#-]prendi ancora il \[C#-]largo, sulla \[D]mia pa\[A]rola,
\[D]con la mia po\[A]tenza, \[B-]io ti fa\[F#-]rò \[D]
pesca\[E]tore di \[D]uomi\[A]ni.
\endchorus




%%%%% STROFA
\beginverse
Ma^estro, dimmi ^cosa devo ^fare,
in^segnami, Si^gnore, dove an^dare, ^
Ge^sù, dammi la ^forza ^di ^par^tire,
la ^forza di la^sciare le mie ^cose:
\endverse



\beginverse*
que^sta fa^miglia che mi ^son cre\[E]ato,
le ^barche che a fa^tica ho conqui^stato, ^
la ^casa, la mia ^terra, la ^mi^a ^gente,
Si^gnore, dammi ^tu una fede \[E]forte.
\endverse


%%%%% RITORNELLO
\beginchorus
\textnote{\textbf{Rit.}}
\[A]Pietro, \[E]vai, \[F#-]fidati di \[C#-]me,
\[D]la mia Chiesa \[A]su te fonde\[E]rò;
\[F#-]manderò lo \[C#-]Spirito, ti \[D]darà co\[A]raggio,
\[D]donerà la \[A]forza \[B-]dell'a\[F#-]mor \[D]
per il \[E]Regno di \[D]Di\[A]o.
\endchorus

%%%%% RITORNELLO
\beginchorus
\[A]Pietro, \[E]vai, \[F#-]fidati di \[C#-]me,
\[D]getta ancora in \[A]acqua le tue \[E]reti,
\[F#-]prendi ancora il \[C#-]largo, sulla \[D]mia pa\[A]rola,
\[D]con la mia po\[A]tenza, \[B-]io ti fa\[F#-]rò \[D]
pesca\[E]tore di \[D]uomi\[A]ni.
\endchorus


\endsong

\fi
%++++++++++++++++++++++++++++++++++++++++++++++++++++++++++++
%			FINE CANZONE TRASPOSTA
%++++++++++++++++++++++++++++++++++++++++++++++++++++++++++++
