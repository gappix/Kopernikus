%-------------------------------------------------------------
%			INIZIO	CANZONE
%-------------------------------------------------------------


%titolo: 	Fate questo in memoria di me
%autore: 	Burgio
%tonalita: 	Do



%%%%%% TITOLO E IMPOSTAZONI
\beginsong{Fate questo in memoria di me}[by={C. Burgio}] 	% <<< MODIFICA TITOLO E AUTORE
\transpose{0} 						% <<< TRASPOSIZIONE #TONI (0 nullo)
\momenti{Comunione; Offertorio; Spezzare del pane; Pasqua; Meditazione; Prima Comunione; }							% <<< INSERISCI MOMENTI	
% momenti vanno separati da ; e vanno scelti tra:
% Ingresso; Atto penitenziale; Acclamazione al Vangelo; Dopo il Vangelo; Offertorio; Comunione; Ringraziamento; 
% Fine; Santi; Pasqua; Avvento; Natale; Quaresima; Canti Mariani; Battesimo; Prima Comunione; Cresima; Matrimonio; Meditazione; Spezzare del pane;
\ifchorded
	%\textnote{Tonalità migliore }	% <<< EV COMMENTI (tonalità originale/migliore)
\fi



%%%%%% INTRODUZIONE
\ifchorded
\vspace*{\versesep}
\musicnote{
\begin{minipage}{0.48\textwidth}
\textbf{Intro}
\hfill 
%( \eighthnote \, 80)   % <<  MODIFICA IL TEMPO
% Metronomo: \eighthnote (ottavo) \quarternote (quarto) \halfnote (due quarti)
\end{minipage}
} 	
\vspace*{-\versesep}
\beginverse*

\nolyrics

%---- Prima riga -----------------------------
\vspace*{-\versesep}
\[C]\[G] \[F] \[C] % \[*D] per indicare le pennate, \rep{2} le ripetizioni

%---- Ogni riga successiva -------------------
%\vspace*{-\versesep}
%\[F] \[G]  \[C]	

%---- Ev Indicazioni -------------------------			
%\textnote{\textit{(Oppure tutta la strofa)} }	

\endverse
\fi



%%%%% STROFA
\beginverse		%Oppure \beginverse* se non si vuole il numero di fianco
\memorize 		% <<< DECOMMENTA se si vuole utilizzarne la funzione
\[C]Quando nell’\[G]ultima \[F]cena, Si\[E-]gnore,
\[F] spezzando il \[E-]pane \[F*] ti \[E*]desti a \[A-]noi,
\[G-7]ecco ap\[C7]rimmo i nostri \[F4]occhi, \[E]
ve\[A-]demmo il \[D-]Tuo immenso A\[C]more,
cre\[A]demmo alla tua \[D-]voce che di\[G*]ce\[F*]va: \[G]
\endverse


%%%%% RITORNELLO
\beginchorus
\textnote{\textbf{Rit.}}
Questo è il \[C]corpo \[F*] che è \[G*]dato per \[C]voi
questo \[C]calice  \[F] è la nuova alle\[G]anza \[E]
nel mio \[A-]sangue \[D-] ch’è versato \brk per \[C*]v\[E*]o\[A-]i
fate \[D-*]que\[C*]sto \[F*] in me\[G*]moria di \[F]me. \[C]
\endchorus




%%%%% STROFA
\beginverse	%Oppure \beginverse* se non si vuole il numero di fianco
%\memorize 		% <<< DECOMMENTA se si vuole utilizzarne la funzione
^Quando nell’^ultima ^cena, Si^gnore,
^ versando il ^vino, ^ ti ^desti a ^noi,
^ecco sve^lasti il gran Mis^tero, ^
il ^dono di un’^Alleanza ^nuova,
per ^sempre stabi^lita con ^n^o^i

\endverse



%%%%% STROFA
\beginverse		%Oppure \beginverse* se non si vuole il numero di fianco
%\memorize 		% <<< DECOMMENTA se si vuole utilizzarne la funzione
^Ora anche ^noi, Tuoi ^figli a^mati,
^ saremo ^dono ^ per ^ogni u^omo,
^prendici e ^guida i nostri ^passi, ^
do^vunque il tuo ^Spirito ci ^porti,
sa^remo la tua ^voce che ^di^ce:^
\endverse


%%%%% RITORNELLO
\beginchorus
\vspace*{1.3\versesep}
\textnote{\textbf{Finale} }
Non te\[C]mete \[F*] sarò \[G*]sempre con \[C]voi
e por\[C]tate \[F] il Vangelo nel \[G]mondo \[E]
ogni \[A-]uomo \[D-] riconosca il \brk mio a\[C*]m\[E*]o\[A-]re
fate \[D-*]que\[C*]sto \[F*] in me\[G*]moria di \[F]me. \[C*]
\endchorus





\endsong
%------------------------------------------------------------
%			FINE CANZONE
%------------------------------------------------------------