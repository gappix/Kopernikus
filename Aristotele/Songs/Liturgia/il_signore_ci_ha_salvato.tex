%-------------------------------------------------------------
%			INIZIO	CANZONE
%-------------------------------------------------------------


%titolo: 	Il Signore ci ha salvato
%autore: 	L. Capello, A. Roncari
%tonalita: 	RE-



%%%%%% TITOLO E IMPOSTAZONI
\beginsong{Il Signore ci ha salvato}[by={L. Capello, A. Roncari}] 	% <<< MODIFICA TITOLO E AUTORE
\transpose{0} 						% <<< TRASPOSIZIONE #TONI (0 nullo)
%\preferflats  %SE VOGLIO FORZARE i bemolle come alterazioni
%\prefersharps %SE VOGLIO FORZARE i # come alterazioni
\momenti{Quaresima}							% <<< INSERISCI MOMENTI	
% momenti vanno separati da ; e vanno scelti tra:
% Ingresso; Atto penitenziale; Acclamazione al Vangelo; Dopo il Vangelo; Offertorio; Comunione; Ringraziamento; Fine; Santi; Pasqua; Avvento; Natale; Quaresima; Canti Mariani; Battesimo; Prima Comunione; Cresima; Matrimonio; Meditazione; Spezzare del pane;
\ifchorded
	%\textnote{Tonalità migliore }	% <<< EV COMMENTI (tonalità originale\migliore)
\fi


%%%%%% INTRODUZIONE
\ifchorded
\vspace*{\versesep}
\textnote{Intro: \qquad \qquad  }%(\eighthnote 116) % <<  MODIFICA IL TEMPO
% Metronomo: \eighthnote (ottavo) \quarternote (quarto) \halfnote (due quarti)
\vspace*{-\versesep}
\beginverse*

\nolyrics

%---- Prima riga -----------------------------
\vspace*{-\versesep}
\[D-] \[G-] \[D-]	 % \[*D] per indicare le pennate, \rep{2} le ripetizioni

%---- Ogni riga successiva -------------------
%\vspace*{-\versesep}
%\[G] \[C]  \[D]	

%---- Ev Indicazioni -------------------------			
%\textnote{\textit{(Oppure tutta la strofa)} }	

\endverse
\fi




%%%%% STROFA
\beginverse		%Oppure \beginverse* se non si vuole il numero di fianco
\memorize 		% <<< DECOMMENTA se si vuole utilizzarne la funzione
%\chordsoff		% <<< DECOMMENTA se vuoi una strofa senza accordi

Il Sign\[D-]ore ci ha sal\[G-]vato dai ne\[D-]mici
nel pass\[G-]aggio \[C7]del Mar \[F]Rosso:
\[G-]l'acqua che ha travolto gli Egi\[D-]ziani
fu per \[G-]noi \[A7]la sal\[D-]vezza.\[D7]
\endverse




%%%%% RITORNELLO
\beginchorus
\textnote{\textbf{Rit.}}

"Se cono\[G-]scessi il \[C]dono di \[F]Dio
e chi è co\[G-]lui che ti chi\[A7]ede da \[D-]bere,
lo preghe\[B&]resti tu \[C]stesso di \[F]darti
quell'acqua \[G-]viva che ti salve\[A7]rà".

\endchorus



%%%%% STROFA
\beginverse		%Oppure \beginverse* se non si vuole il numero di fianco
%\memorize 		% <<< DECOMMENTA se si vuole utilizzarne la funzione
%\chordsoff		% <<< DECOMMENTA se vuoi una strofa senza accordi

Era\[D-]vamo pro\[G-]strati nel de\[D-]serto,
consu\[G-]mati \[C7]dalla se\[F]te:
qu\[G-]ando fu percossa la \[D-]roccia,
zampi\[G-]llò \[A7]una so\[D-]rgente.\[D7]

\endverse




%%%%% STROFA
\beginverse		%Oppure \beginverse* se non si vuole il numero di fianco
%\memorize 		% <<< DECOMMENTA se si vuole utilizzarne la funzione
%\chordsoff		% <<< DECOMMENTA se vuoi una strofa senza accordi
\chordsoff
Dal\[D-]le mura del te\[G-]mpio di \[D-]Dio
sgorga \[G-]un fi\[C7]ume d'acqua \[F]viva:
\[G-]tutto quel che l'acqua \[D-]toccherà
na\[G-]scerà \[A7]a nuova vi\[D-]ta.\[D7]
\endverse

%%%%% STROFA
\beginverse		%Oppure \beginverse* se non si vuole il numero di fianco
%\memorize 		% <<< DECOMMENTA se si vuole utilizzarne la funzione
%\chordsoff		% <<< DECOMMENTA se vuoi una strofa senza accordi
\chordsoff
Ve\[D-]nga a me chi ha se\[G-]te e chi mi \[D-]cerca,
si di\[G-]sseti co\[C7]lui che in me cr\[F]ede:
\[G-]fiumi d'acqua viva scorre\[D-]ranno
dal mio \[G-]cuo\[A7]re tra\[D-]fitto.\[D7]
\endverse

%%%%% STROFA
\beginverse		%Oppure \beginverse* se non si vuole il numero di fianco
%\memorize 		% <<< DECOMMENTA se si vuole utilizzarne la funzione
%\chordsoff		% <<< DECOMMENTA se vuoi una strofa senza accordi
\chordsoff
Sul\[D-]la croce il Fi\[G-]glio di \[D-]Dio
fu tra\[G-]fitto \[C7]da una la\[F]ncia:
dal cuore dell'Agnello imm\[D-]olato
sca\[G-]turà sa\[A7]ngue ed a\[D-]cqua.\[D7]
\endverse


%%%%% STROFA
\beginverse		%Oppure \beginverse* se non si vuole il numero di fianco
%\memorize 		% <<< DECOMMENTA se si vuole utilizzarne la funzione
%\chordsoff		% <<< DECOMMENTA se vuoi una strofa senza accordi
\chordsoff
Chi \[D-]berrà l'acqu\[G-]a viva che io \[D-]dono
non a\[G-]vrà mai più \[C7]sete in e\[F]terno:
\[G-]in lui diventerà una sor\[D-]gente
za\[G-]mpilla\[A7]nte per se\[D-]mpre.\[D7]
\endverse




\endsong
%------------------------------------------------------------
%			FINE CANZONE
%------------------------------------------------------------



