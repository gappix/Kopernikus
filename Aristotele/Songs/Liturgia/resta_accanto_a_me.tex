%-------------------------------------------------------------
%			INIZIO	CANZONE
%-------------------------------------------------------------


%titolo: 	Resta accanto a me
%autore: 	Gen Verde
%tonalita: 	Mi 



%%%%%% TITOLO E IMPOSTAZONI
\beginsong{Resta accanto a me}[by={Gen Verde}] 	% <<< MODIFICA TITOLO E AUTORE
\transpose{0} 						% <<< TRASPOSIZIONE #TONI (0 nullo)
\momenti{Ringraziamento; Congedo}							% <<< INSERISCI MOMENTI	
% momenti vanno separati da ; e vanno scelti tra:
% Ingresso; Atto penitenziale; Acclamazione al Vangelo; Dopo il Vangelo; Offertorio; Comunione; Ringraziamento; Fine; Santi; Pasqua; Avvento; Natale; Quaresima; Canti Mariani; Battesimo; Prima Comunione; Cresima; Matrimonio; Meditazione;
\ifchorded
	\textnote{$\lozenge$ Tonalità originale }	% <<< EV COMMENTI (tonalità originale/migliore)
\fi

%%%%%% INTRODUZIONE
\ifchorded
\vspace*{\versesep}
\musicnote{
\begin{minipage}{0.48\textwidth}
\textbf{Intro}
\hfill 
%( \eighthnote \, 80)   % <<  MODIFICA IL TEMPO
% Metronomo: \eighthnote (ottavo) \quarternote (quarto) \halfnote (due quarti)
\end{minipage}
} 	
\vspace*{-\versesep}
\beginverse*

\nolyrics

%---- Prima riga -----------------------------
\vspace*{-\versesep}
\[A] \[E]  \[B]	 % \[*D] per indicare le pennate, \rep{2} le ripetizioni

%---- Ogni riga successiva -------------------
%\vspace*{-\versesep}
%\[G] \[C]  \[D]	

%---- Ev Indicazioni -------------------------			
%\textnote{\textit{(Oppure tutta la strofa)} }	

\endverse
\fi

%%%%% RITORNELLO
\beginchorus
\textnote{\textbf{Rit.}}

\[E]Ora \[B]vado \[A]sulla mia \[E]strada
\[F#-]con l'a\[G#-]more \[A]tuo che mi \[B]guida.
\[E]O Si\[B]gnore, o\[A]vunque io \[E]vada,
\[A]resta ac\[B]canto a \[E]me.
\[E]Io ti \[B]prego, \[A]stammi vi\[E]cino
\[F#-]ogni \[G#-]passo \[A]del mio cam\[B]mino,
\[E]ogni \[B]notte, \[A]ogni mat\[E]tino,
\[A]resta ac\[B]canto a \[E]me.

\endchorus

%%%%% STROFA
\beginverse		%Oppure \beginverse* se non si vuole il numero di fianco
\memorize 		% <<< DECOMMENTA se si vuole utilizzarne la funzione
%\chordsoff		& <<< DECOMMENTA se vuoi una strofa senza accordi

\[B]Il tuo sguardo \[A]puro sia luce per \[C#-]me
\[B]e la tua Pa\[A]rola sia voce per \[E]me.
\[A]Che io trovi il \[B]senso del mio andare
\[C#-]so\[B]lo in \[E]te, nel \[B]tuo fedele amare il \[A]mio per\[E]ché.

\endverse

%%%%% STROFA
\beginverse		%Oppure \beginverse* se non si vuole il numero di fianco
%\memorize 		% <<< DECOMMENTA se si vuole utilizzarne la funzione
%\chordsoff		& <<< DECOMMENTA se vuoi una strofa senza accordi

^Fa' che chi mi ^guarda non veda che ^te.
^Fa' che chi mi a^scolta non senta che ^te,
^e chi pensa a ^me, fa' che nel cuore
^pen^si a ^te e ^trovi quell'amore \brk che ^hai dato a ^me.
\endverse

%%%%%% EV. FINALE

\beginchorus %oppure \beginverse*
\vspace*{1.3\versesep}
%\textnote{Finale \textit{(rallentando)}} %<<< EV. INDICAZIONI

\[E]Ora \[B]vado \[A]sulla mia \[E]strada
\[A]resta ac\[B]canto a \[E]me.

\endchorus  %oppure \endverse




\endsong
%------------------------------------------------------------
%			FINE CANZONE
%------------------------------------------------------------

%++++++++++++++++++++++++++++++++++++++++++++++++++++++++++++
%			CANZONE TRASPOSTA
%++++++++++++++++++++++++++++++++++++++++++++++++++++++++++++
\ifchorded
%decremento contatore per avere stesso numero
\addtocounter{songnum}{-1} 
\beginsong{Resta accanto a me}[by={Gen Verde}] 	% <<< COPIA TITOLO E AUTORE
\transpose{-2} 						% <<< TRASPOSIZIONE #TONI + - (0 nullo)
%\preferflats SE VOGLIO FORZARE i bemolle come alterazioni
%\prefersharps SE VOGLIO FORZARE i # come alterazioni
\ifchorded
	\textnote{$\bigstar$ Tonalità migliore}	% <<< EV COMMENTI (tonalità originale/migliore)
\fi

%%%%%% INTRODUZIONE
\ifchorded
\vspace*{\versesep}
\musicnote{
\begin{minipage}{0.48\textwidth}
\textbf{Intro}
\hfill 
%( \eighthnote \, 80)   % <<  MODIFICA IL TEMPO
% Metronomo: \eighthnote (ottavo) \quarternote (quarto) \halfnote (due quarti)
\end{minipage}
} 	
\vspace*{-\versesep}
\beginverse*
\nolyrics

%---- Prima riga -----------------------------
\vspace*{-\versesep}
\[A] \[E]  \[B]	 % \[*D] per indicare le pennate, \rep{2} le ripetizioni

%---- Ogni riga successiva -------------------
%\vspace*{-\versesep}
%\[G] \[C]  \[D]	

%---- Ev Indicazioni -------------------------			
%\textnote{\textit{(Oppure tutta la strofa)} }	

\endverse
\fi

%%%%% RITORNELLO
\beginchorus
\textnote{\textbf{Rit.}}

\[E]Ora \[B]vado \[A]sulla mia \[E]strada
\[F#-]con l'a\[G#-]more \[A]tuo che mi \[B]guida.
\[E]O Si\[B]gnore, o\[A]vunque io \[E]vada,
\[A]resta ac\[B]canto a \[E]me.
\[E]Io ti \[B]prego, \[A]stammi vi\[E]cino
\[F#-]ogni \[G#-]passo \[A]del mio cam\[B]mino,
\[E]ogni \[B]notte, \[A]ogni mat\[E]tino,
\[A]resta ac\[B]canto a \[E]me.

\endchorus

%%%%% STROFA
\beginverse		%Oppure \beginverse* se non si vuole il numero di fianco
\memorize 		% <<< DECOMMENTA se si vuole utilizzarne la funzione
%\chordsoff		& <<< DECOMMENTA se vuoi una strofa senza accordi

\[B]Il tuo sguardo \[A]puro sia luce per \[C#-]me
\[B]e la tua Pa\[A]rola sia voce per \[E]me.
\[A]Che io trovi il \[B]senso del mio andare
\[C#-]so\[B]lo in \[E]te, nel \[B]tuo fedele amare il \[A]mio per\[E]ché.

\endverse

%%%%% STROFA
\beginverse		%Oppure \beginverse* se non si vuole il numero di fianco
%\memorize 		% <<< DECOMMENTA se si vuole utilizzarne la funzione
%\chordsoff		& <<< DECOMMENTA se vuoi una strofa senza accordi

^Fa' che chi mi ^guarda non veda che ^te.
^Fa' che chi mi a^scolta non senta che ^te,
^e chi pensa a ^me, fa' che nel cuore
^pen^si a ^te e ^trovi quell'amore \brk che ^hai dato a ^me.
\endverse

%%%%%% EV. FINALE

\beginchorus %oppure \beginverse*
\vspace*{1.3\versesep}
%\textnote{Finale \textit{(rallentando)}} %<<< EV. INDICAZIONI

\[E]Ora \[B]vado \[A]sulla mia \[E]strada
\[A]resta ac\[B]canto a \[E]me.

\endchorus  %oppure \endverse

\endsong

\fi
%++++++++++++++++++++++++++++++++++++++++++++++++++++++++++++
%			FINE CANZONE TRASPOSTA
%++++++++++++++++++++++++++++++++++++++++++++++++++++++++++++
