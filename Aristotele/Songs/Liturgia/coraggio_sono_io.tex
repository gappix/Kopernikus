%-------------------------------------------------------------
%			INIZIO	CANZONE
%-------------------------------------------------------------


%titolo: 	Coraggio sono io
%autore: 	E. Bertoglio, A. Testa, C. Pastori
%tonalita: 	Re 



%%%%%% TITOLO E IMPOSTAZONI
\beginsong{Coraggio sono io}[by={E. Bertoglio, A. Testa, C. Pastori}] 	
\transpose{0} 						% <<< TRASPOSIZIONE #TONI (0 nullo)
\momenti{Comunione; Ringraziamento; Congedo}	% <<< INSERISCI MOMENTI	
% momenti vanno separati da ; e vanno scelti tra:
% Ingresso; Atto penitenziale; Acclamazione al Vangelo; Dopo il Vangelo; Offertorio; Comunione; Ringraziamento; Fine; Santi; Pasqua; Avvento; Natale; Quaresima; Canti Mariani; Battesimo; Prima Comunione; Cresima; Matrimonio; Meditazione; Spezzare del pane;
\ifchorded
	%\textnote{Tonalità originale }	% <<< EV COMMENTI (tonalità originale/migliore)
\fi


%%%%%% INTRODUZIONE
\ifchorded
\vspace*{\versesep}
\musicnote{
\begin{minipage}{0.48\textwidth}
Intro
\hfill 
(\quarternote \, 80)
\end{minipage}
} 
% Metronomo: \eighthnote (ottavo)  (quarto) \halfnote (due quarti)
\vspace*{-\versesep}
\beginverse*

\nolyrics

%---- Prima riga -----------------------------
\vspace*{-\versesep}
 \[D]   \[D] 	 % \[*D] per indicare le pennate, \rep{2} le ripetizioni

%---- Ogni riga successiva -------------------
%\vspace*{-\versesep}
%\[G] \[C]  \[D]	

%---- Ev Indicazioni -------------------------			
%\textnote{\textit{(Oppure tutta la strofa)} }	

\endverse
\fi




%%%%% STROFA
\beginverse		%Oppure \beginverse* se non si vuole il numero di fianco
\memorize 		% <<< DECOMMENTA se si vuole utilizzarne la funzione
%\chordsoff		% <<< DECOMMENTA se vuoi una strofa senza accordi

\[D]Strade vuote e silen\[D7+]ziose,
vie de\[E-]serte e sconosci\[B-]ute
è una v\[G]ita che ora scorre senza di \[A]te.
Mi ri\[D]trovo dentro a un \[D7+]mare
di incer\[E-]tezze e turba\[B-]menti,
la fa\[G]tica di un cammino senza di \[A]te.

\[D]Ma \[A]tu,
mano a\[G]mica di ogni u\[D]omo,
pre\[E-]senza che sosti\[E7]ene an\[A]cora.
\[D]Ma \[A]tu,
che ora g\[G]uidi il mio cam\[D]mino,
il\[G]lumina la via da\[E7]vanti a \[A]me.

\endverse







%%%%% RITORNELLO
\beginchorus
\textnote{\textbf{Rit.}}

\[D]No, non a\[G]vere pa\[A7]ura,
se nel \[D]buio il tuo c\[G]uore
un \[B-]giorno perde\[C]ra\[A]i.
\[D]Io v\[G]errò da \[A7]te,
come un \[D]padre \[G]ti di\[A*]rò:
"coraggio, \[G]sono \[D]io!".

\endchorus



%%%%% STROFA
\beginverse		%Oppure \beginverse* se non si vuole il numero di fianco
%\memorize 		% <<< DECOMMENTA se si vuole utilizzarne la funzione
%\chordsoff		% <<< DECOMMENTA se vuoi una strofa senza accordi

Il res^piro di una ^vita
è ^silenzio di de^serto
fatto ^di parole vuote senza di ^te.
Dove ^sono le risp^oste
alle m^ille e più do^mande,
il mio ^cuore non sa amare senza di ^te.

^Ma ^tu,
luce ^nella notte ^buia,
mi ai^uterai a rag^giunger la ^meta.
^Ma ^tu,
fonte ^viva della ^fede,
sa^rai per sempre qui vi^cino a ^me.

\endverse




%%%%% RITORNELLO
\beginchorus
\textnote{\textbf{Rit.}}

\[D]No, non a\[G]vere pa\[A7]ura,
se nel \[D]buio il tuo c\[G]uore
un \[B-]giorno perde\[C]ra\[A]i.
\[D]Io v\[G]errò da \[A7]te,
come un \[D]padre \[G]ti di\[A*]rò:
"coraggio, \[G]sono \[D]io!".

\endchorus


%%%%%% EV. FINALE

\beginchorus %oppure \beginverse*
\vspace*{1.3\versesep}
\textnote{\textbf{Finale} \textit{(rallentando)}} %<<< EV. INDICAZIONI

\[(D)]"Coraggio, \[G]sono \[D]io!".
\[D*]"Coraggio, \[G*]sono \[D]io!".

\endchorus  %oppure \endverse


\endsong
%------------------------------------------------------------
%			FINE CANZONE
%------------------------------------------------------------


