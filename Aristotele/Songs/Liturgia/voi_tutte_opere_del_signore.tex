%-------------------------------------------------------------
%			INIZIO	CANZONE
%-------------------------------------------------------------


%titolo: 	Voi tutte opere del Signore
%autore: 	Rossi
%tonalita: 	Sol 



%%%%%% TITOLO E IMPOSTAZONI
\beginsong{Voi tutte opere del Signore}[by={Rossi}] 	% <<< MODIFICA TITOLO E AUTORE
\transpose{0} 						% <<< TRASPOSIZIONE #TONI (0 nullo)
\momenti{Ringraziamento}							% <<< INSERISCI MOMENTI	
% momenti vanno separati da ; e vanno scelti tra:
% Ingresso; Atto penitenziale; Acclamazione al Vangelo; Dopo il Vangelo; Offertorio; Comunione; Ringraziamento; Fine; Santi; Pasqua; Avvento; Natale; Quaresima; Canti Mariani; Battesimo; Prima Comunione; Cresima; Matrimonio; Meditazione; Spezzare del pane;
\ifchorded
	%\textnote{Tonalità originale }	% <<< EV COMMENTI (tonalità originale/migliore)
\fi

%%%%%% INTRODUZIONE
\ifchorded
\vspace*{\versesep}
\textnote{Intro: \qquad \qquad  }%(\eighthnote 116) % <<  MODIFICA IL TEMPO
% Metronomo: \eighthnote (ottavo) \quarternote (quarto) \halfnote (due quarti)
\vspace*{-\versesep}
\beginverse*

\nolyrics

%---- Prima riga -----------------------------
\vspace*{-\versesep}
\[D] %\[G] \[B-] \[G]	 % \[*D] per indicare le pennate, \rep{2} le ripetizioni

%---- Ogni riga successiva -------------------
%\vspace*{-\versesep}
%\[G] \[C]  \[D]	

%---- Ev Indicazioni -------------------------			
\textnote{\textit{(Oppure tutta la prima sequenza)} }	

\endverse
\fi

%%%%% STROFA
\beginverse		%Oppure \beginverse* se non si vuole il numero di fianco
\memorize 		% <<< DECOMMENTA se si vuole utilizzarne la funzione
%\chordsoff		% <<< DECOMMENTA se vuoi una strofa senza accordi

\[(A*)]Voi \[D]tut\[(A*)]te \[D]opere \[G]del Si\[D]gnore,  \brk  \[G]bene\[D]dite il Si\[A]gno\[D]re!
\[(A*)]Voi \[D]tut\[(A*)]ti \[D]Angeli \[G]del Si\[D]gnore, \brk  \[G]bene\[D]dite il Si\[A]gno\[D]re!
E \[F#]voi, o \[B-]cieli, \[E]voi, o \[A]acque, \brk  \[D]bene\[A]dite il Si\[E]gno\[A]re!
\[(A*)]Voi \[D]tut\[(A*)]te \[D]opere \[G]del Si\[D]gnore,  \brk  \[G]bene\[D]dite il Si\[A]gno\[D]re!
\endverse

%%%%% STROFA
\beginverse		%Oppure \beginverse* se non si vuole il numero di fianco
%\memorize 		% <<< DECOMMENTA se si vuole utilizzarne la funzione
%\chordsoff		% <<< DECOMMENTA se vuoi una strofa senza accordi

^Voi ^tut^te po^tenze e ^astri del c^ielo,  \brk   ^bene^dite il Si^gno^re!
^Voi ^tut^te p^iogge, ru^giade e ^nevi,  \brk  ^bene^dite il Si^gno^re!
Voi ^sole e ^luna, ^voi, o ^venti,  \brk ^bene^dite il Si^gno^re!
^Voi ^tut^te ^opere ^del Si^gnore,   \brk  ^bene^dite il Si^gno^re!
\endverse


%%%%% STROFA
\beginverse		%Oppure \beginverse* se non si vuole il numero di fianco
%\memorize 		% <<< DECOMMENTA se si vuole utilizzarne la funzione
%\chordsoff		% <<< DECOMMENTA se vuoi una strofa senza accordi

^Voi ^fuo^co e ca^lore, ^freddo e ^caldo,   \brk   ^bene^dite il Si^gno^re!
^voi ^luc^e e ^tenebre, ^ghiaccio e ^freddo,  \brk ^bene^dite il Si^gno^re!
Voi ^notti e ^giorni, ^lampi e ^nubi,  \brk  ^bene^dite il Si^gno^re!
^Voi ^tut^te ^opere ^del Si^gnore,  \brk   ^bene^dite il Si^gno^re!
\endverse


%%%%% STROFA
\beginverse		%Oppure \beginverse* se non si vuole il numero di fianco
%\memorize 		% <<< DECOMMENTA se si vuole utilizzarne la funzione
\chordsoff		% <<< DECOMMENTA se vuoi una strofa senza accordi

La terra tutta lodi il Signore:     benedite il Signore!
Voi tutti viventi lodate il Signore,    benedite il Signore!
Voi monti e colli, mari e fiumi,    benedite il Signore!
Voi tutte opere del Signore,    benedite il Signore!

\endverse


%%%%% STROFA
\beginverse		%Oppure \beginverse* se non si vuole il numero di fianco
%\memorize 		% <<< DECOMMENTA se si vuole utilizzarne la funzione
\chordsoff		% <<< DECOMMENTA se vuoi una strofa senza accordi

Voi tutti pesci e mostri del mare,  benedite il Signore!
Voi tutte belve feroci e armenti,   benedite il Signore!
Voi acque e fonti, voi uccelli,     benedite il Signore!
Voi tutte opere del Signore,    benedite il Signore!

\endverse


%%%%% STROFA
\beginverse		%Oppure \beginverse* se non si vuole il numero di fianco
%\memorize 		% <<< DECOMMENTA se si vuole utilizzarne la funzione
\chordsoff		% <<< DECOMMENTA se vuoi una strofa senza accordi

Voi tutti uomini del Signore,   benedite il Signore!
E voi sacerdoti del Signore,    benedite il Signore!
Voi popolo santo, eletto da Dio,    benedite il Signore!
Voi tutte opere del Signore,    benedite il Signore!
\endverse
\endsong
%------------------------------------------------------------
%			FINE CANZONE
%------------------------------------------------------------