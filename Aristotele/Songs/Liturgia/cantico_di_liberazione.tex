%-------------------------------------------------------------
%			INIZIO	CANZONE
%-------------------------------------------------------------


%titolo: 	Cantico di liberazione
%autore: 	
%tonalita: 	La- 



%%%%%% TITOLO E IMPOSTAZONI
\beginsong{Cantico di liberazione}[by={Canto della Veglia Pasquale}] 	% <<< MODIFICA TITOLO E AUTORE
\transpose{0} 						% <<< TRASPOSIZIONE #TONI (0 nullo)
\momenti{Pasqua}							% <<< INSERISCI MOMENTI	
% momenti vanno separati da ; e vanno scelti tra:
% Ingresso; Atto penitenziale; Acclamazione al Vangelo; Dopo il Vangelo; Offertorio; Comunione; Ringraziamento; Fine; Santi; Pasqua; Avvento; Natale; Quaresima; Canti Mariani; Battesimo; Prima Comunione; Cresima; Matrimonio; Meditazione;
\ifchorded
	%\textnote{Tonalità originale }	% <<< EV COMMENTI (tonalità originale/migliore)
\fi




%%%%%% INTRODUZIONE
\ifchorded
\vspace*{\versesep}
\musicnote{
\begin{minipage}{0.48\textwidth}
\textbf{Intro}
\hfill 
%( \eighthnote \, 80)   % <<  MODIFICA IL TEMPO
% Metronomo: \eighthnote (ottavo) \quarternote (quarto) \halfnote (due quarti)
\end{minipage}
} 	
\vspace*{-\versesep}
\beginverse*


\nolyrics

%---- Prima riga -----------------------------
\vspace*{-\versesep}
\[A-] \[G] \[A-] \[G] \rep{2}	 % \[*D] per indicare le pennate, \rep{2} le ripetizioni

%---- Ogni riga successiva -------------------
%\vspace*{-\versesep}
%\[G] \[C]  \[D]	

%---- Ev Indicazioni -------------------------			
%\textnote{\textit{(Oppure tutta la strofa)} }	

\endverse
\fi




%%%%% STROFA
\beginverse		%Oppure \beginverse* se non si vuole il numero di fianco
\memorize 		% <<< DECOMMENTA se si vuole utilizzarne la funzione
%\chordsoff		% <<< DECOMMENTA se vuoi una strofa senza accordi

\[A-]Voglio can\[G]tare in o\[E-]nore di \[A-]Dio
\[D-]perché mi\[C]rabile \[A-]è la sua \[D-]gloria
\[G-]amore, \[C]forza e mio \[A-]can\[A-]to è il Si\[D-]gnore
\[G-]solo a lui \[A]devo la \[D-]mia sal\[B&]vezza:
\[A-]lo esalte\[F]rò, \[A-]è 
il \[G]Dio di mio \[A-]pa\[G]dre  \qquad \[A-] \[G] 

 


\endverse








%%%%% STROFA
\beginverse		%Oppure \beginverse* se non si vuole il numero di fianco
%\memorize 		% <<< DECOMMENTA se si vuole utilizzarne la funzione
%\chordsoff		% <<< DECOMMENTA se vuoi una strofa senza accordi



^Stettero im^mobili le ^acque di^vise
^per riscat^tare il tuo ^popolo, o ^Dio
^poi l'ira ^tua volò ^so^pra il mar ^Rosso,
^carri ed e^sercito ^di Fara^one,
^fior di guerri^e^ri 
som^mersero l'^on^de \qquad  ^ ^ 



\endverse



%%%%%% EV. INTERMEZZO
\beginverse*
\vspace*{1.3\versesep}
{
	\nolyrics
	\textnote{Intermezzo strumentale}
	
	%---- Ev Indicazioni -------------------------			
	\textnote{\textit{[ripetizione dell'intera strofa]}} 
	 
}
\vspace*{\versesep}
\endverse



%%%%% STROFA
\beginverse		%Oppure \beginverse* se non si vuole il numero di fianco
%\memorize 		% <<< DECOMMENTA se si vuole utilizzarne la funzione
%\chordsoff		% <<< DECOMMENTA se vuoi una strofa senza accordi


^Disse il ne^mico: io l'^insegui^rò,
^raggiunge^rò la mia ^preda Isra^ele,
^sguaine^rò la mia ^spa^da ro^vente,
^divide^rò il bot^tino dei ^vinti,
^la mia ^ma^no 
li s^termine^rà ^ \qquad  ^ ^ 

\endverse





%%%%% STROFA
\beginverse		%Oppure \beginverse* se non si vuole il numero di fianco
%\memorize 		% <<< DECOMMENTA se si vuole utilizzarne la funzione
%\chordsoff		% <<< DECOMMENTA se vuoi una strofa senza accordi



^Ma l'ira ^tua soffiò ^sopra il mar ^Rosso,
^acque im^mense copr^iron le schi^ere
^si river^sarono ^ad ^un tuo ^gesto
^e ricop^rirono ^carri e guerri^eri
^che come pi^e^tre 
rag^giunsero il ^fon^do \qquad  ^ ^ 

\endverse




%%%%%% EV. INTERMEZZO
\beginverse*
\vspace*{1.3\versesep}
{
	\nolyrics
	\textnote{Intermezzo strumentale}
	
	%---- Ev Indicazioni -------------------------			
	\textnote{\textit{[ripetizione dell'intera strofa]}} 
	 
}
\vspace*{\versesep}
\endverse


%%%%% STROFA
\beginverse		%Oppure \beginverse* se non si vuole il numero di fianco
%\memorize 		% <<< DECOMMENTA se si vuole utilizzarne la funzione
%\chordsoff		% <<< DECOMMENTA se vuoi una strofa senza accordi


^Chi è come ^te fra gli ^dei, Si^gnore,
^chi è come ^te maes^toso e po^tente
^che ope^rasti un pro^di^gio gran^dioso
^la tua ^destra sten^desti o ^Dio
^il mare apr^is^ti 
a sal^vare i tuoi ^ser^vi \qquad  ^ ^


\endverse





%%%%% STROFA
\beginverse		%Oppure \beginverse* se non si vuole il numero di fianco
%\memorize 		% <<< DECOMMENTA se si vuole utilizzarne la funzione
%\chordsoff		% <<< DECOMMENTA se vuoi una strofa senza accordi

^Questo tuo ^popolo, che ^hai riscat^tato,
^ora lo gu^idi tu ^solo be^nigno
con ^forza e a^more lo ^stai ^condu^cendo
^alla tua ^santa di^mora di^vina
^che le tue ^ma^ni, 
Si^gnore, han fon^da^to. 

\endverse

%%%%%% EV. INTERMEZZO
\beginverse*
\vspace*{1.3\versesep}
{
	\nolyrics
	\textnote{Chiusura strumentale}
	
	\ifchorded

	%---- Prima riga -----------------------------
	\vspace*{-\versesep}
	\[A-] \[G] \[A-] \[G]  
	%---- Ogni riga successiva -------------------
	\vspace*{-\versesep}
	\[A-] \[G*]  \textit{[sospeso...]}


	\fi
	%---- Ev Indicazioni -------------------------			
	%\textnote{\textit{(ripetizione della strofa)}} 
	 
}
\vspace*{\versesep}
\endverse


\endsong
%------------------------------------------------------------
%			FINE CANZONE
%------------------------------------------------------------

