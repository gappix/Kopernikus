%-------------------------------------------------------------
%			INIZIO	CANZONE
%-------------------------------------------------------------


%titolo: 	Cantico dei redenti
%autore: 	Marani
%tonalita: 	Sol 



%%%%%% TITOLO E IMPOSTAZONI
\beginsong{Cantico dei redenti}[ititle={Il Signore è la mia salvezza}, by={Marani}] 	% <<< MODIFICA TITOLO E AUTORE
\transpose{0} 						% <<< TRASPOSIZIONE #TONI (0 nullo)
\momenti{Ingresso; Fine; Comunione}							% <<< INSERISCI MOMENTI	
% momenti vanno separati da ; e vanno scelti tra:
% Ingresso; Atto penitenziale; Acclamazione al Vangelo; Dopo il Vangelo; Offertorio; Comunione; Ringraziamento; Fine; Santi; Pasqua; Avvento; Natale; Quaresima; Canti Mariani; Battesimo; Prima Comunione; Cresima; Matrimonio; Meditazione; Spezzare del pane;
\ifchorded
	%\textnote{Tonalità migliore }	% <<< EV COMMENTI (tonalità originale/migliore)
\fi


%%%%%% INTRODUZIONE
\ifchorded
\vspace*{\versesep}
\textnote{Intro: \qquad \qquad  }%(\eighthnote 116) % <<  MODIFICA IL TEMPO
% Metronomo: \eighthnote (ottavo) \quarternote (quarto) \halfnote (due quarti)
\vspace*{-\versesep}
\beginverse*

\nolyrics

%---- Prima riga -----------------------------
\vspace*{-\versesep}
\[E-] \[D] \[E-]	 % \[*D] per indicare le pennate, \rep{2} le ripetizioni

%---- Ogni riga successiva -------------------
%\vspace*{-\versesep}
%\[G] \[C]  \[D]	

%---- Ev Indicazioni -------------------------			
%\textnote{\textit{(Oppure tutta la strofa)} }	

\endverse
\fi



%%%%% RITORNELLO
\beginchorus
\textnote{\textbf{Rit.}}
Il Si\[E-]gnore è la \[D]mia sal\[E-]vezza
e con \[G]lui non \[D]temo \[E-]più 
perché ho nel \[A-]cuore \[B7]la cer\[E-]tezza  
la sal\[C]vezza è \[D]qui con \[E-]me.
\endchorus



%%%%% STROFA
\beginverse		%Oppure \beginverse* se non si vuole il numero di fianco
\memorize 		% <<< DECOMMENTA se si vuole utilizzarne la funzione
%\chordsoff		% <<< DECOMMENTA se vuoi una strofa senza accordi
Ti \[E-]lodo Si\[D]gnore per\[C]ché 
un giorno \[E-]eri lon\[D7]tano da \[G]me, \[D]
\[D] ora invece sei tor\[E-]nato
e mi \[C]hai pre\[D]so con \[E-]te.
\endverse

\beginverse
%\chordsoff
Ber^rete con ^gioia alle ^fonti
alle ^fonti ^della sal^vez^za 
^ e quel giorno voi di^rete:
lo^date il Signore,
invo^cate il suo ^nome.
\endverse

\beginverse
%\chordsoff
F^ate co^noscere ai ^popoli
tutto ^quello che ^lui ha com^piu^to
^ e ricordino per ^sempre
ri^cordino sempre
che il ^suo nome è ^grande.
\endverse

\beginverse
%\chordsoff
Can^tate a chi ha ^fatto gran^dezze
e sia ^fatto sa^pere nel ^mon^do;
^ grida forte la tua ^gioia, 
abi^tante di Sion,
perché ^grande con te è il Si^gnore.
\endverse
\endsong

