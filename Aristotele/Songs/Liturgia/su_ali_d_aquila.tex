%-------------------------------------------------------------
%			INIZIO	CANZONE
%-------------------------------------------------------------


%titolo: 	Su ali d'acquila
%autore: 	Daniele Ricci
%tonalita: 	FA 



%%%%%% TITOLO E IMPOSTAZONI
\beginsong{Su ali d'aquila}[by={J. M. Joncas}] 	% <<< MODIFICA TITOLO E AUTORE
\transpose{0} 						% <<< TRASPOSIZIONE #TONI (0 nullo)
\momenti{Congedo; Matrimonio}							% <<< INSERISCI MOMENTI	
% momenti vanno separati da ; e vanno scelti tra:
% Ingresso; Atto penitenziale; Acclamazione al Vangelo; Dopo il Vangelo; Offertorio; Comunione; Ringraziamento; Fine; Santi; Pasqua; Avvento; Natale; Quaresima; Canti Mariani; Battesimo; Prima Comunione; Cresima; Matrimonio; Meditazione;
\ifchorded
	\textnote{$\bigstar$ Tonalità originale }	% <<< EV COMMENTI (tonalità originale/migliore)
\fi


%%%%%% INTRODUZIONE
\ifchorded
\vspace*{\versesep}
\musicnote{
\begin{minipage}{0.48\textwidth}
\textbf{Intro}
\hfill 
%( \eighthnote \, 80)   % <<  MODIFICA IL TEMPO
% Metronomo: \eighthnote (ottavo) \quarternote (quarto) \halfnote (due quarti)
\end{minipage}
} 	
\vspace*{-\versesep}
\beginverse*

\nolyrics

%---- Prima riga -----------------------------
\vspace*{-\versesep}
\[F7+] \[C7+] \[F7+] \[C7+] 	 % \[*D] per indicare le pennate, \rep{2} le ripetizioni

%---- Ogni riga successiva -------------------
%\vspace*{-\versesep}
%\[G] \[C]  \[D]	

%---- Ev Indicazioni -------------------------			
%\textnote{\textit{(Oppure tutta la strofa)} }	

\endverse
\fi




%%%%% STROFA
\beginverse		%Oppure \beginverse* se non si vuole il numero di fianco
%\memorize 		% <<< DECOMMENTA se si vuole utilizzarne la funzione
%\chordsoff		& <<< DECOMMENTA se vuoi una strofa senza accordi

\[F7+]Tu che abiti al \[C7+]riparo del Signore
\[F7+]e che dimori alla sua \[C7+]ombra
\[E&]di al Signore mio \[C-]Rifugio,
mia \[F-]roccia in cui con\[G]fido.

\endverse




%%%%% RITORNELLO
\beginchorus
\textnote{\textbf{Rit.}}

E ti ri\[C]alzerà, \[7+]ti solleverà
su ali d'\[D-]aquila \[G]ti regge\[7]rà
sulla \[G-7]brezza dell'\[C]alba ti \[F]farà bril\[D-]lar
co\[G*]me il \[A-]sole, \[E-7]  co\[F]sì nelle sue \[G4]\[7]mani vi\[C]vrai.

\endchorus


%%%%% STROFA
\beginverse		%Oppure \beginverse* se non si vuole il numero di fianco
%\memorize 		% <<< DECOMMENTA se si vuole utilizzarne la funzione
\chordsoff		% <<< DECOMMENTA se vuoi una strofa senza accordi
Dal laccio del cacciatore ti libererà
e dalla carestia che ti distrugge
poi ti coprirà con le sue ali
e rifugio troverai.

\endverse



%%%%% STROFA
\beginverse		%Oppure \beginverse* se non si vuole il numero di fianco
%\memorize 		% <<< DECOMMENTA se si vuole utilizzarne la funzione
\chordsoff		% <<< DECOMMENTA se vuoi una strofa senza accordi
Perché ai suoi angeli da dato un comando
di preservarti in tutte le tue vie
ti porteranno sulle loro mani
contro la pietra non inciamperai.

\endverse



%%%%%% EV. FINALE

\beginchorus %oppure \beginverse*
\vspace*{1.3\versesep}
\textnote{\textbf{Finale} } %<<< EV. INDICAZIONI

E ti ri\[C]alzerò, \[7+]ti solleverò
su ali d'\[D-]aquila \[G]ti regge\[7]rò
sulla \[G-7]brezza dell'\[C]alba ti \[F]farò bril\[D-]lar
co\[G]me il \[A-]sole, \[E-7]  co\[F]sì nelle mie \[G4]\[7]mani vi\[C]vrai.


\endchorus  %oppure \endverse




\endsong
%------------------------------------------------------------
%			FINE CANZONE
%------------------------------------------------------------




%++++++++++++++++++++++++++++++++++++++++++++++++++++++++++++
%			CANZONE TRASPOSTA
%++++++++++++++++++++++++++++++++++++++++++++++++++++++++++++
\ifchorded
%decremento contatore per avere stesso numero
\addtocounter{songnum}{-1} 
\beginsong{Su ali d'aquila}[by={J. M. Joncas}]	% <<< COPIA TITOLO E AUTORE
\transpose{-3} 						% <<< TRASPOSIZIONE #TONI + - (0 nullo)
\ifchorded
	\textnote{$\triangle$ Tonalità migliore per le chitarre}	% <<< EV COMMENTI (tonalità originale/migliore)
\fi


%%%%%% INTRODUZIONE
\ifchorded
\vspace*{\versesep}
\musicnote{
\begin{minipage}{0.48\textwidth}
\textbf{Intro}
\hfill 
%( \eighthnote \, 80)   % <<  MODIFICA IL TEMPO
% Metronomo: \eighthnote (ottavo) \quarternote (quarto) \halfnote (due quarti)
\end{minipage}
} 	
\vspace*{-\versesep}
\beginverse*

\nolyrics

%---- Prima riga -----------------------------
\vspace*{-\versesep}
\[F7+] \[C7+] \[F7+] \[C7+] 	 % \[*D] per indicare le pennate, \rep{2} le ripetizioni

%---- Ogni riga successiva -------------------
%\vspace*{-\versesep}
%\[G] \[C]  \[D]	

%---- Ev Indicazioni -------------------------			
%\textnote{\textit{(Oppure tutta la strofa)} }	

\endverse
\fi




%%%%% STROFA
\beginverse		%Oppure \beginverse* se non si vuole il numero di fianco
%\memorize 		% <<< DECOMMENTA se si vuole utilizzarne la funzione
%\chordsoff		& <<< DECOMMENTA se vuoi una strofa senza accordi

\[F7+]Tu che abiti al \[C7+]riparo del Signore
\[F7+]e che dimori alla sua \[C7+]ombra
\[E&]dì al Signore mio \[C-]Rifugio,
mia \[F-]roccia in cui con\[G]fido.

\endverse




%%%%% RITORNELLO
\beginchorus
\textnote{\textbf{Rit.}}

E ti ri\[C]alzerà, \[7+]ti solleverà
su ali d'\[D-]aquila \[G]ti regge\[7]rà
sulla \[G-7]brezza dell'\[C]alba ti \[F]farà bril\[D-]lar
co\[G]me il \[A-]sole, \[E-7]  co\[F]sì nelle sue \[G4]\[7]mani vi\[C]vrai.

\endchorus



%%%%% STROFA
\beginverse		%Oppure \beginverse* se non si vuole il numero di fianco
%\memorize 		% <<< DECOMMENTA se si vuole utilizzarne la funzione
\chordsoff		% <<< DECOMMENTA se vuoi una strofa senza accordi
Dal laccio del cacciatore ti libererà
e dalla carestia che ti distrugge
poi ti coprirà con le sue ali
e rifugio troverai.

\endverse


%%%%% STROFA
\beginverse		%Oppure \beginverse* se non si vuole il numero di fianco
%\memorize 		% <<< DECOMMENTA se si vuole utilizzarne la funzione
\chordsoff		% <<< DECOMMENTA se vuoi una strofa senza accordi
Non devi temer i terrori della notte
né freccia che vola di giorno
mille cadranno al tuo fianco
ma nulla ti colpirà
\endverse


%%%%% STROFA
\beginverse		%Oppure \beginverse* se non si vuole il numero di fianco
%\memorize 		% <<< DECOMMENTA se si vuole utilizzarne la funzione
\chordsoff		% <<< DECOMMENTA se vuoi una strofa senza accordi
Perché ai suoi angeli ha dato un comando
di preservarti in tutte le tue vie
ti porteranno sulle loro mani
contro la pietra non inciamperai.

\endverse



%%%%%% EV. FINALE

\beginchorus %oppure \beginverse*
\vspace*{1.3\versesep}
\textnote{\textbf{Finale} } %<<< EV. INDICAZIONI

E ti ri\[C]alzerò, \[7+]ti solleverò
su ali d'\[D-]aquila \[G]ti regge\[7]rò
sulla \[G-7]brezza dell'\[C]alba ti \[F]farò bril\[D-]lar
co\[G*]me il \[A-]sole, \[E-7]  co\[F]sì nelle mie \[G4]\[7]mani vi\[C]vrai.


\endchorus  %oppure \endverse




\endsong

\fi
%++++++++++++++++++++++++++++++++++++++++++++++++++++++++++++
%			FINE CANZONE TRASPOSTA
%++++++++++++++++++++++++++++++++++++++++++++++++++++++++++++
