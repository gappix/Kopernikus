%-------------------------------------------------------------
%			INIZIO	CANZONE
%-------------------------------------------------------------


%titolo: 	Non di solo pane
%autore: 	Tuttiattavola — Oratori Bresso e Lainate
%tonalita: 	Mi 



%%%%%% TITOLO E IMPOSTAZONI
\beginsong{Non di solo pane}[by={Tuttiattavola — Oratori Bresso e Lainate}] 	% <<< MODIFICA TITOLO E AUTORE
\transpose{-2} 						% <<< TRASPOSIZIONE #TONI (0 nullo)
%\preferflats  %SE VOGLIO FORZARE i bemolle come alterazioni
%\prefersharps %SE VOGLIO FORZARE i # come alterazioni
\momenti{Meditazione; Comunione; Ringraziamento}							% <<< INSERISCI MOMENTI	
% momenti vanno separati da ; e vanno scelti tra:
% Ingresso; Atto penitenziale; Acclamazione al Vangelo; Dopo il Vangelo; Offertorio; Comunione; 
% Ringraziamento; Fine; Santi; Pasqua; Avvento; Natale; Quaresima; Canti Mariani; Battesimo; 
% Prima Comunione; Cresima; Matrimonio; Meditazione; Spezzare del pane;
\ifchorded
	\textnote{$\bigstar$ Tonalità migliore }	% <<< EV COMMENTI (tonalità originale/migliore)
\fi


%%%%%% INTRODUZIONE
\ifchorded
\vspace*{\versesep}
\musicnote{
\begin{minipage}{0.48\textwidth}
\textbf{Intro}
\hfill 
%( \eighthnote \, 80)   % <<  MODIFICA IL TEMPO
% Metronomo: \eighthnote (ottavo) \quarternote (quarto) \halfnote (due quarti)
\end{minipage}
} 	
\vspace*{-\versesep}
\beginverse*

\nolyrics

%---- Prima riga -----------------------------
\vspace*{-\versesep}
\[C#-] \[A] \[E] \[B] \rep{2}	 % \[*D] per indicare le pennate, \rep{2} le ripetizioni

%---- Ogni riga successiva -------------------
%\vspace*{-\versesep}
%\[G] \[C]  \[D]	

%---- Ev Indicazioni -------------------------			
%\textnote{\textit{(Oppure tutta la strofa)} }	

\endverse
\fi




%%%%% STROFA
\beginverse		%Oppure \beginverse* se non si vuole il numero di fianco
\memorize 		% <<< DECOMMENTA se si vuole utilizzarne la funzione
%\chordsoff		% <<< DECOMMENTA se vuoi una strofa senza accordi

Non di solo \[C#-]pane ogni giorno vi\[A]vrai,
ma di ogni Pa\[E]rola che il Padre ti dona  \brk ti nutri\[B]rai.
Fedele cus\[C#-]tode del gesto d'a\[A]more,
nel pane spez\[E]zato memoria preziosa \brk  conserve\[B]rai.

\endverse
\beginverse*		%Oppure \beginverse* se non si vuole il numero di fianco
%\memorize 		% <<< DECOMMENTA se si vuole utilizzarne la funzione
%\chordsoff		% <<< DECOMMENTA se vuoi una strofa senza accordi

Sulla nostra ^terra che lui ci do^nò
c'è posto per ^tutti se apri il tuo cuore \brk all'umani^tà.
Per vivere in^sieme ci vuole co^raggio
un tenero s^guardo di misericordia e fraterni^tà.

\endverse



%%%%% RITORNELLO
\beginchorus
\textnote{\textbf{Rit.}}

Tu \[E]sai Signore,
co\[B]me saziare
quel \[A]vuoto che sento dentro \[C#-]me.
Perchè o\[A]gnuno sa dare
la \[E]parte migliore di \[B]sé
insieme a \[B]te.

\endchorus


%%%%%% EV. INTERMEZZO
\beginverse*
\vspace*{1.3\versesep}
{
	\nolyrics
	\textnote{Intermezzo strumentale}
	
	\ifchorded

	%---- Prima riga -----------------------------
	\vspace*{-\versesep}
	\[C#-] \[A] \[E] \[B]


	\fi
	%---- Ev Indicazioni -------------------------			
	%\textnote{\textit{(ripetizione della strofa)}} 
	 
}
\vspace*{\versesep}
\endverse



%%%%% STROFA
\beginverse		%Oppure \beginverse* se non si vuole il numero di fianco
%\memorize 		% <<< DECOMMENTA se si vuole utilizzarne la funzione
%\chordsoff		% <<< DECOMMENTA se vuoi una strofa senza accordi

Non di solo ^pane \brk ogni giorno viv^rai, 
avrai sete di ^pace guardando la croce \brk con umil^tà.
Alle tue mille do^mande \brk troverai le ris^poste,
se camminerai ac^canto a un fratello \brk in difficol^tà.

\endverse
\beginverse*		%Oppure \beginverse* se non si vuole il numero di fianco
%\memorize 		% <<< DECOMMENTA se si vuole utilizzarne la funzione
%\chordsoff		% <<< DECOMMENTA se vuoi una strofa senza accordi
Spalanca il ^cuore, ascolta il Si^gnore.
Tieni aperte le ^porte in costante ricerca \brk della veri^tà.
Coltiva la ^Fede, che luce sa^rà,
perchè siamo in ^viaggio insieme ad un Altro \brk che ci guida ^già.

\endverse



%%%%% RITORNELLO
\beginchorus
\textnote{\textbf{Rit.}}

Tu \[E]sai Signore,
co\[B]me saziare
quel \[A]vuoto che sento dentro \[C#-]me.
Perchè o\[A]gnuno sa dare
la \[E]parte migliore di \[B]sé
insieme a \[B]te.

\endchorus



%%%%% BRIDGE
\beginverse*		%Oppure \beginverse* se non si vuole il numero di fianco
%\memorize 		% <<< DECOMMENTA se si vuole utilizzarne la funzione
%\chordsoff		% <<< DECOMMENTA se vuoi una strofa senza accordi
\vspace*{1.3\versesep}
\textnote{\textbf{Bridge}}

Non è \[E]facile,
quando \[B]niente va,
aspet\[A]tare che un sorriso 
torni ad \[C#-]esser pane per la vita.
Ar\[E]rendersi
non a\[B]iuterà
prendi \[A]posto alla mia tavo\[B*]la:
chi ama gioia guste\[C#-]rà!
\endverse


%%%%%% EV. INTERMEZZO
\beginverse*
\vspace*{1.3\versesep}
{
	\nolyrics
	\textnote{Intermezzo strumentale}
	
	\ifchorded

	%---- Prima riga -----------------------------
	\vspace*{-\versesep}
	 \[A] \[E] \[B]


	\fi
	%---- Ev Indicazioni -------------------------			
	%\textnote{\textit{(ripetizione della strofa)}} 
	 
}
\vspace*{\versesep}
\endverse


%%%%% RITORNELLO
\beginchorus
\textnote{\textbf{Rit.}}

Tu \[E]sai Signore,
co\[B]me saziare
quel \[A]vuoto che sento dentro \[C#-]me.
Perchè o\[A]gnuno sa dare
la \[E]parte migliore di \[B]sé
insieme a \[B]te. \rep{2} 

\endchorus



%%%%%% EV. CHIUSURA SOLO STRUMENTALE
\ifchorded
\beginchorus %oppure \beginverse*
\vspace*{1.3\versesep}
\textnote{Chiusura } %<<< EV. INDICAZIONI

\[E*]

\endchorus  %oppure \endverse
\fi


\endsong
%------------------------------------------------------------
%			FINE CANZONE
%------------------------------------------------------------


%++++++++++++++++++++++++++++++++++++++++++++++++++++++++++++
%			CANZONE TRASPOSTA
%++++++++++++++++++++++++++++++++++++++++++++++++++++++++++++
\ifchorded
%decremento contatore per avere stesso numero
\addtocounter{songnum}{-1} 
\beginsong{Non di solo pane}[by={Tuttiattavola — Oratori Bresso e Lainate}] 	% <<< COPIA TITOLO E AUTORE
\transpose{0} 						% <<< TRASPOSIZIONE #TONI + - (0 nullo)
%\preferflats  %SE VOGLIO FORZARE i bemolle come alterazioni
%\prefersharps %SE VOGLIO FORZARE i # come alterazioni
\ifchorded
	\textnote{$\lozenge$ Tonalità originale}	% <<< EV COMMENTI (tonalità originale/migliore)
\fi

%%%%%% INTRODUZIONE
\ifchorded
\vspace*{\versesep}
\musicnote{
\begin{minipage}{0.48\textwidth}
\textbf{Intro}
\hfill 
%( \eighthnote \, 80)   % <<  MODIFICA IL TEMPO
% Metronomo: \eighthnote (ottavo) \quarternote (quarto) \halfnote (due quarti)
\end{minipage}
} 	
\vspace*{-\versesep}
\beginverse*

\nolyrics

%---- Prima riga -----------------------------
\vspace*{-\versesep}
\[C#-] \[A] \[E] \[B] \rep{2}	 % \[*D] per indicare le pennate, \rep{2} le ripetizioni

%---- Ogni riga successiva -------------------
%\vspace*{-\versesep}
%\[G] \[C]  \[D]	

%---- Ev Indicazioni -------------------------			
%\textnote{\textit{(Oppure tutta la strofa)} }	

\endverse
\fi




%%%%% STROFA
\beginverse		%Oppure \beginverse* se non si vuole il numero di fianco
\memorize 		% <<< DECOMMENTA se si vuole utilizzarne la funzione
%\chordsoff		% <<< DECOMMENTA se vuoi una strofa senza accordi

Non di solo \[C#-]pane ogni giorno vi\[A]vrai,
ma di ogni Pa\[E]rola che il Padre ti dona  \brk ti nutri\[B]rai.
Fedele cus\[C#-]tode del gesto d'a\[A]more,
nel pane spez\[E]zato memoria preziosa \brk  conserve\[B]rai.

\endverse
\beginverse*		%Oppure \beginverse* se non si vuole il numero di fianco
%\memorize 		% <<< DECOMMENTA se si vuole utilizzarne la funzione
%\chordsoff		% <<< DECOMMENTA se vuoi una strofa senza accordi

Sulla nostra ^terra che lui ci do^nò
c'è posto per ^tutti se apri il tuo cuore \brk all'umani^tà.
Per vivere in^sieme ci vuole co^raggio
un tenero s^guardo di misericordia e fraterni^tà.

\endverse



%%%%% RITORNELLO
\beginchorus
\textnote{\textbf{Rit.}}

Tu \[E]sai Signore,
co\[B]me saziare
quel \[A]vuoto che sento dentro \[C#-]me.
Perchè o\[A]gnuno sa dare
la \[E]parte migliore di \[B]sé
insieme a \[B]te.

\endchorus


%%%%%% EV. INTERMEZZO
\beginverse*
\vspace*{1.3\versesep}
{
	\nolyrics
	\textnote{Intermezzo strumentale}
	
	\ifchorded

	%---- Prima riga -----------------------------
	\vspace*{-\versesep}
	\[C#-] \[A] \[E] \[B]


	\fi
	%---- Ev Indicazioni -------------------------			
	%\textnote{\textit{(ripetizione della strofa)}} 
	 
}
\vspace*{\versesep}
\endverse



%%%%% STROFA
\beginverse		%Oppure \beginverse* se non si vuole il numero di fianco
%\memorize 		% <<< DECOMMENTA se si vuole utilizzarne la funzione
%\chordsoff		% <<< DECOMMENTA se vuoi una strofa senza accordi

Non di solo ^pane \brk ogni giorno viv^rai, 
avrai sete di ^pace guardando la croce \brk con umil^tà.
Alle tue mille do^mande \brk troverai le ris^poste,
se camminerai ac^canto a un fratello \brk in difficol^tà.

\endverse
\beginverse*		%Oppure \beginverse* se non si vuole il numero di fianco
%\memorize 		% <<< DECOMMENTA se si vuole utilizzarne la funzione
%\chordsoff		% <<< DECOMMENTA se vuoi una strofa senza accordi
Spalanca il ^cuore, ascolta il Si^gnore.
Tieni aperte le ^porte in costante ricerca \brk della veri^tà.
Coltiva la ^Fede, che luce sa^rà,
perchè siamo in ^viaggio insieme ad un Altro \brk che ci guida ^già.

\endverse



%%%%% RITORNELLO
\beginchorus
\textnote{\textbf{Rit.}}

Tu \[E]sai Signore,
co\[B]me saziare
quel \[A]vuoto che sento dentro \[C#-]me.
Perchè o\[A]gnuno sa dare
la \[E]parte migliore di \[B]sé
insieme a \[B]te.

\endchorus



%%%%% BRIDGE
\beginverse*		%Oppure \beginverse* se non si vuole il numero di fianco
%\memorize 		% <<< DECOMMENTA se si vuole utilizzarne la funzione
%\chordsoff		% <<< DECOMMENTA se vuoi una strofa senza accordi
\vspace*{1.3\versesep}
\textnote{\textbf{Bridge}}

Non è \[E]facile,
quando \[B]niente va,
aspet\[A]tare che un sorriso 
torni ad \[C#-]esser pane per la vita.
Ar\[E]rendersi
non a\[B]iuterà
prendi \[A]posto alla mia tavo\[B*]la:
chi ama gioia guste\[C#-]rà!
\endverse


%%%%%% EV. INTERMEZZO
\beginverse*
\vspace*{1.3\versesep}
{
	\nolyrics
	\textnote{Intermezzo strumentale}
	
	\ifchorded

	%---- Prima riga -----------------------------
	\vspace*{-\versesep}
	 \[A] \[E] \[B]


	\fi
	%---- Ev Indicazioni -------------------------			
	%\textnote{\textit{(ripetizione della strofa)}} 
	 
}
\vspace*{\versesep}
\endverse


%%%%% RITORNELLO
\beginchorus
\textnote{\textbf{Rit.}}

Tu \[E]sai Signore,
co\[B]me saziare
quel \[A]vuoto che sento dentro \[C#-]me.
Perchè o\[A]gnuno sa dare
la \[E]parte migliore di \[B]sé
insieme a \[B]te. \rep{2} 

\endchorus



%%%%%% EV. CHIUSURA SOLO STRUMENTALE
\ifchorded
\beginchorus %oppure \beginverse*
\vspace*{1.3\versesep}
\textnote{Chiusura } %<<< EV. INDICAZIONI

\[E*]

\endchorus  %oppure \endverse
\fi


\endsong

\fi
%++++++++++++++++++++++++++++++++++++++++++++++++++++++++++++
%			FINE CANZONE TRASPOSTA
%++++++++++++++++++++++++++++++++++++++++++++++++++++++++++++


