%-------------------------------------------------------------
%			INIZIO	CANZONE
%-------------------------------------------------------------


%titolo: 	Mani che si stringono
%autore: 	M. Gioia
%tonalita: 	Do 



%%%%%% TITOLO E IMPOSTAZONI
\beginsong{Mani che si stringono}[by={M. Gioia}] 	% <<< MODIFICA TITOLO E AUTORE
\transpose{0} 						% <<< TRASPOSIZIONE #TONI (0 nullo)
%\preferflats  %SE VOGLIO FORZARE i bemolle come alterazioni
%\prefersharps %SE VOGLIO FORZARE i # come alterazioni
\momenti{}							% <<< INSERISCI MOMENTI	
% momenti vanno separati da ; e vanno scelti tra:
% Ingresso; Atto penitenziale; Acclamazione al Vangelo; Dopo il Vangelo; Offertorio; Comunione; Ringraziamento; Fine; Santi; Pasqua; Avvento; Natale; Quaresima; Canti Mariani; Battesimo; Prima Comunione; Cresima; Matrimonio; Meditazione; Spezzare del pane;
\ifchorded
	%\textnote{Tonalità migliore }	% <<< EV COMMENTI (tonalità originale/migliore)
\fi

%%%%%% INTRODUZIONE
\ifchorded
\vspace*{\versesep}
\musicnote{
\begin{minipage}{0.48\textwidth}
\textbf{Intro}
\hfill 
%( \eighthnote \, 80)   % <<  MODIFICA IL TEMPO
% Metronomo: \eighthnote (ottavo) \quarternote (quarto) \halfnote (due quarti)
\end{minipage}
} 	
\vspace*{-\versesep}
\beginverse*
\nolyrics

%---- Prima riga -----------------------------
\vspace*{-\versesep}
\[C] \[F] \[C]	 % \[*D] per indicare le pennate, \rep{2} le ripetizioni

%---- Ogni riga successiva -------------------
%\vspace*{-\versesep}
%\[G] \[C]  \[D]	

%---- Ev Indicazioni -------------------------			
%\textnote{\textit{(Oppure tutta la strofa)} }	

\endverse
\fi




%%%%% STROFA
\beginverse		%Oppure \beginverse* se non si vuole il numero di fianco
\memorize 		% <<< DECOMMENTA se si vuole utilizzarne la funzione
%\chordsoff		% <<< DECOMMENTA se vuoi una strofa senza accordi

\[C]Il sole scende \[F] \brk è quasi notte or\[C]mai, \[C7]
dai re\[F]stiamo ancora insieme un \[G]po',  \brk  meno buio sa\[C]rà. \[C7]
La pa\[F]rola del Si\[G]gnore  \brk come \[E-]luce in mezzo a \[A-]noi,
ci ris\[D7]chiara,\[F] e ci unisce a \[G4]Lui. \[G]


\endverse




%%%%% RITORNELLO
\beginchorus
\textnote{\textbf{Rit.}}

\[C]Mani \[F]che si stringono \[C]forte,
\[D-7]in un cerchio di \[C]sguardi 
\[F]che s'incrocia\[G4]no, \[G]
come un ab\[A-]braccio stret\[D-]to-o,
per sen\[G]tire che la \[C7+]chiesa vive,
che \[F7+]vive dentro \[B&]noi, insieme a \[G4]noi. \[G]
E so-no \[C]mani \[F4]che si tendono in \[C]alto,
\[D-7]che si aprono \[C]grandi 
\[F]per raccoglie\[G4]re \[G]
quella \[A-]forza immensa-a \[D-7]
che il Si\[G]gnore mette in \[C7+]fondo al cuore,
annun\[F7+]ciare a tutti \[B&]che Dio ci \[G4]ama. \[G]


\endchorus



%%%%% STROFA
\beginverse		%Oppure \beginverse* se non si vuole il numero di fianco
%\memorize 		% <<< DECOMMENTA se si vuole utilizzarne la funzione
%\chordsoff		% <<< DECOMMENTA se vuoi una strofa senza accordi

^Noi veglieremo ^con le lampa^de, ^
aspet^tando nella notte, ^ \brk  finché giorno sa^rà. ^
E la ^voce del Si^gnore, \brk  all'improv^viso giunge^rà,
saremo ^pronti,  \brk ^ saremo amici s^uoi. ^


\endverse



%%%%%% EV. FINALE

\beginchorus %oppure \beginverse*
\vspace*{1.3\versesep}
\textnote{\textbf{Finale} } %<<< EV. INDICAZIONI

...Dio ci \[F]ama! \[F] \[C]

\endchorus  %oppure \endverse





\endsong
%------------------------------------------------------------
%			FINE CANZONE
%------------------------------------------------------------


