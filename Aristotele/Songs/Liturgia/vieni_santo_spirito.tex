%-------------------------------------------------------------
%			INIZIO	CANZONE
%-------------------------------------------------------------


%titolo: 	Vieni Santo Spirito
%autore: 	Gen Verde
%tonalita: 	Do 



%%%%%% TITOLO E IMPOSTAZONI
\beginsong{Vieni Santo Spirito}[by={Gen Verde}] 	% <<< MODIFICA TITOLO E AUTORE
\transpose{0} 						% <<< TRASPOSIZIONE #TONI (0 nullo)
%\preferflats  %SE VOGLIO FORZARE i bemolle come alterazioni
%\prefersharps %SE VOGLIO FORZARE i # come alterazioni
\momenti{Cresima}							% <<< INSERISCI MOMENTI	
% momenti vanno separati da ; e vanno scelti tra:
% Ingresso; Atto penitenziale; Acclamazione al Vangelo; Dopo il Vangelo; Offertorio; Comunione; Ringraziamento; Fine; Santi; Pasqua; Avvento; Natale; Quaresima; Canti Mariani; Battesimo; Prima Comunione; Cresima; Matrimonio; Meditazione; Spezzare del pane;
\ifchorded
	%\textnote{Tonalità migliore }	% <<< EV COMMENTI (tonalità originale/migliore)
\fi


%%%%%% INTRODUZIONE
\ifchorded
\vspace*{\versesep}
\textnote{Intro: \qquad \qquad  }%(\eighthnote 116) % <<  MODIFICA IL TEMPO
% Metronomo: \eighthnote (ottavo) \quarternote (quarto) \halfnote (due quarti)
\vspace*{-\versesep}
\beginverse*

\nolyrics

%---- Prima riga -----------------------------
\vspace*{-\versesep}
\[C] \[A-] \[F] \[G] 	 % \[*D] per indicare le pennate, \rep{2} le ripetizioni

%---- Ogni riga successiva -------------------
%\vspace*{-\versesep}
%\[G] \[C]  \[D]	

%---- Ev Indicazioni -------------------------			
%\textnote{\textit{(Oppure tutta la strofa)} }	

\endverse
\fi










%%%%% RITORNELLO
\beginchorus
\textnote{\textbf{Rit.}}
\[C]Vieni \[G]Santo \[A-]Spirito,  \brk \[F]manda a \[C]noi dal \[G4]cie\[G]lo,
un \[C]ra-a-g\[F]gio di \[C]lu-u-\[G]ce,  \brk un \[F]raggio di \[C]luce.
\endchorus
\beginchorus
\[C]Vieni \[G]Padre dei \[A-]poveri, \brk \[F]vieni da\[C]tore dei \[G4]do-o-\[G]ni,
\[C]lu-u-\[F]ce dei \[C]cuo-o-\[G]ri,  \brk \[F]luce dei \[C]cuori.
\endchorus



%%%%% STROFA
\beginverse		%Oppure \beginverse* se non si vuole il numero di fianco
\memorize 		% <<< DECOMMENTA se si vuole utilizzarne la funzione
%\chordsoff		% <<< DECOMMENTA se vuoi una strofa senza accordi
\[A-]Consola\[F]tore per\[G]fe-e-t\[A-]to, \brk \[F]ospite \[G]dolce dell’\[C]a-a-ni\[E]ma,
\[A-]dolcissimo sol\[G]lievo,  \brk  \[A-]dolcissimo sol\[E7]lievo.
\[A-]Nella fa\[F]tica ri\[G]po-o-\[A-]so, \brk \[F]nel ca\[G]lore ri\[C]pa-a-\[E]ro,
nel \[A-]pianto con\[G]forto,  \brk  \[A-]ne-el pianto con\[E]forto. \[E]
\endverse



%%%%% RITORNELLO
\beginchorus
\textnote{\textbf{Rit.}}
\[C]Vieni \[G]Santo \[A-]Spirito,  \brk \[F]manda a \[C]noi dal \[G4]cie\[G]lo,
un \[C]ra-a-g\[F]gio di \[C]lu-u-\[G]ce,  \brk un \[F]raggio di \[C]luce.
\endchorus


%%%%% STROFA
\beginverse		%Oppure \beginverse* se non si vuole il numero di fianco
%\memorize 		% <<< DECOMMENTA se si vuole utilizzarne la funzione
%\chordsoff		% <<< DECOMMENTA se vuoi una strofa senza accordi

^Luce ^be-e-a^tissi^ma,  \brk in^vadi i ^nostri ^cuo-o-^ri,
^senza la tua forza ^nu-ulla, \brk  ^nulla è nell’^uomo.
^Lava ^ciò che è ^sordi^do, \brk  ^scalda ^ciò che è ^geli^do,
^rialza chi è ca^duto, \brk  ri^alza chi è ca^duto. ^

\endverse



%%%%% RITORNELLO
\beginchorus
\textnote{\textbf{Rit.}}
\[C]Vieni \[G]Padre dei \[A-]poveri, \brk \[F]vieni da\[C]tore dei \[G4]do-o-\[G]ni,
\[C]lu-u-\[F]ce dei \[C]cuo-o-\[G]ri,  \brk \[F]luce dei \[C]cuori.
\endchorus



%%%%%% EV. INTERMEZZO
\beginverse*
\vspace*{1.3\versesep}
{
	\nolyrics
	\textnote{Intermezzo strumentale}
	
	\ifchorded

	%---- Prima riga -----------------------------
	\vspace*{-\versesep}
	\[C] \[A-]  \[F]\[G] \[A] \[F#]


	\fi
	%---- Ev Indicazioni -------------------------			
	\textnote{\textit{(Si alza la tonalità)}} 
	 
}
\vspace*{\versesep}
\endverse


%%%%% STROFA
\beginverse		%Oppure \beginverse* se non si vuole il numero di fianco
%\memorize 		% <<< DECOMMENTA se si vuole utilizzarne la funzione
%\chordsoff		% <<< DECOMMENTA se vuoi una strofa senza accordi
\transpose{2}
^Dona ai ^tuoi fe^de-e-^li,  \brk ^che in ^te con^fida^no;
^i sette santi ^doni,  \brk ^i sette santi ^doni.
^Dona vir^tù e ^premi^o, \brk  ^dona ^morte ^sa-a-n^ta,
^dona eterna ^gioia, \brk  ^dona eterna ^gioia. ^
\endverse



%%%%% RITORNELLO
\beginchorus
\textnote{\textbf{Rit.}}
\transpose{2}
\[C]Vieni \[G]Santo \[A-]Spirito,  \brk \[F]manda a \[C]noi dal \[G4]cie\[G]lo,
un \[C]ra-a-g\[F]gio di \[C]lu-u-\[G]ce,  \brk un \[F]raggio di \[C]luce.
\endchorus
\beginchorus
\transpose{2}
\[C]Vieni \[G]Padre dei \[A-]poveri, \brk \[F]vieni da\[C]tore dei \[G4]do-o-\[G]ni,
\[C]lu-u-\[F]ce dei \[C]cuo-o-\[G]ri,  \brk \[F]luce dei \[C]cuori.
\endchorus



\endsong
%------------------------------------------------------------
%			FINE CANZONE
%------------------------------------------------------------


