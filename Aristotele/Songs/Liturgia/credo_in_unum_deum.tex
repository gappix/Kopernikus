%-------------------------------------------------------------
%			INIZIO	CANZONE
%-------------------------------------------------------------


%titolo: 	Credo in unum Deum
%autore: 	M. Balduzzi, C. Casucci
%tonalita: 	Sol 



%%%%%% TITOLO E IMPOSTAZONI
\beginsong{Credo in unum Deum}[by={M. Balduzzi, C. Casucci}] 	% <<< MODIFICA TITOLO E AUTORE
\transpose{0} 						% <<< TRASPOSIZIONE #TONI (0 nullo)
\preferflats  %SE VOGLIO FORZARE i bemolle come alterazioni
%\prefersharps %SE VOGLIO FORZARE i # come alterazioni
\momenti{Meditazione}							% <<< INSERISCI MOMENTI	
% momenti vanno separati da ; e vanno scelti tra:
% Ingresso; Atto penitenziale; Acclamazione al Vangelo; Dopo il Vangelo; Offertorio; Comunione; Ringraziamento; Fine; Santi; Pasqua; Avvento; Natale; Quaresima; Canti Mariani; Battesimo; Prima Comunione; Cresima; Matrimonio; Meditazione; Spezzare del pane;
\ifchorded
	%\textnote{Tonalità migliore }	% <<< EV COMMENTI (tonalità originale/migliore)
\fi


%%%%%% INTRODUZIONE
\ifchorded
\vspace*{\versesep}
\textnote{Intro: \qquad \qquad  }%(\eighthnote 116) % <<  MODIFICA IL TEMPO
% Metronomo: \eighthnote (ottavo) \quarternote (quarto) \halfnote (due quarti)
\vspace*{-\versesep}
\beginverse*

\nolyrics

%---- Prima riga -----------------------------
\vspace*{-\versesep}
\[E-] \quad \[C] 	 % \[*D] per indicare le pennate, \rep{2} le ripetizioni

%---- Ogni riga successiva -------------------
\vspace*{-\versesep}
\[G] \qquad \[D]	

%---- Ev Indicazioni -------------------------			
%\textnote{\textit{(Oppure tutta la strofa)} }	

\endverse
\fi


 

%%%%% STROFA
\beginverse		%Oppure \beginverse* se non si vuole il numero di fianco
\memorize 		% <<< DECOMMENTA se si vuole utilizzarne la funzione
%\chordsoff		% <<< DECOMMENTA se vuoi una strofa senza accordi

\[E-]Credo in unum \[C]Deum,
\[G]Patrem omnipo\[D]tentem.
\[E-]Credo in unum \[C]Deum,
\[G]factorem coeli et \[D]terrae.
\[E-]Visibilium om\[C]nium,
\[G]et invisi\[D]bilium.
\[A-]Credo in unum \[C]Deum, \[D]A-a-a-\[E-]men!          

\endverse



\transpose{2}
%\preferflats 
%\prefersharps 

%%%%% STROFA
\beginverse		%Oppure \beginverse* se non si vuole il numero di fianco
%\memorize 		% <<< DECOMMENTA se si vuole utilizzarne la funzione
%\chordsoff		% <<< DECOMMENTA se vuoi una strofa senza accordi

^Credo in unum ^Deum,
^Dominum Jesum ^Christum.
^Credo in unum ^Deum,
^Filium Dei uni^genitum.
^Et ex Patre na^tum,
^ante omnia sae^cula.
^Credo in unum ^Deum, ^A-a-a-^men!      

\endverse





\transpose{2}
%\preferflats 
\prefersharps 
%%%%% STROFA
\beginverse		%Oppure \beginverse* se non si vuole il numero di fianco
%\memorize 		% <<< DECOMMENTA se si vuole utilizzarne la funzione
%\chordsoff		% <<< DECOMMENTA se vuoi una strofa senza accordi

^Credo in unum ^Deum,
^Spiritum sanc^tum.
^Credo in unum ^Deum,
^Dominum et vivifi^cantem.
^Qui       ex         Pa^tre,
^filioque proce^dit.
^Credo in unum ^Deum, ^A-a-a-^men! 

\endverse

\transpose{-3}
%\preferflats 
\prefersharps 
%%%%% STROFA
\beginverse		%Oppure \beginverse* se non si vuole il numero di fianco
%\memorize 		% <<< DECOMMENTA se si vuole utilizzarne la funzione
%\chordsoff		% <<< DECOMMENTA se vuoi una strofa senza accordi
^Credo in unum ^Deum,
^Patrem omnipo^tentem.
^Credo in unum ^Deum,
^Dominum Jesum ^Christum.
^Credo in unum ^Deum,
^Spiritum sanc^tum.
^Credo in unum ^Deum, ^A-a-a-^men!

\endverse

%%%%%% EV. FINALE

\beginchorus %oppure \beginverse*
\vspace*{1.3\versesep}
\textnote{Finale } %<<< EV. INDICAZIONI
\transpose{-1}
%\preferflats 
\prefersharps 
A-a-a\[C-]men!  A-a-a\[C#]men! 
A-a-a-\[G#]men!

\endchorus  %oppure \endverse



\endsong
%------------------------------------------------------------
%			FINE CANZONE
%------------------------------------------------------------


