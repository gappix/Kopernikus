%-------------------------------------------------------------
%			INIZIO	CANZONE
%-------------------------------------------------------------


%titolo: 	Ecco quel che abbiamo
%autore: 	Gen Verde
%tonalita: 	Do



%%%%%% TITOLO E IMPOSTAZONI
\beginsong{Ecco quel che abbiamo}[by={Gen\ Verde}]	% <<< MODIFICA TITOLO E AUTORE
\transpose{0} 						% <<< TRASPOSIZIONE #TONI (0 nullo)
\momenti{Offertorio}							% <<< INSERISCI MOMENTI	
% momenti vanno separati da ; e vanno scelti tra:
% Ingresso; Atto penitenziale; Acclamazione al Vangelo; Dopo il Vangelo; Offertorio; Comunione; Ringraziamento; Fine; Santi; Pasqua; Avvento; Natale; Quaresima; Canti Mariani; Battesimo; Prima Comunione; Cresima; Matrimonio; Meditazione; Spezzare del pane;
\ifchorded
	%\textnote{Tonalità migliore }	% <<< EV COMMENTI (tonalità originale/migliore)
\fi




%%%%%% INTRODUZIONE
\ifchorded
\vspace*{\versesep}
\musicnote{
\begin{minipage}{0.48\textwidth}
\textbf{Intro}
\hfill 
%( \eighthnote \, 80)   % <<  MODIFICA IL TEMPO
% Metronomo: \eighthnote (ottavo) \quarternote (quarto) \halfnote (due quarti)
\end{minipage}
} 	
\vspace*{-\versesep}
\beginverse*

\nolyrics

%---- Prima riga -----------------------------
\vspace*{-\versesep}
\[C] \[E-] \[F] \[C]	 % \[*D] per indicare le pennate, \rep{2} le ripetizioni

%---- Ogni riga successiva -------------------
%\vspace*{-\versesep}
%\[G] \[C]  \[D]	

%---- Ev Indicazioni -------------------------			
%\textnote{\textit{(Oppure tutta la strofa)} }	

\endverse
\fi




%%%%% STROFA
\beginverse		%Oppure \beginverse* se non si vuole il numero di fianco
\memorize 		% <<< DECOMMENTA se si vuole utilizzarne la funzione
%\chordsoff		% <<< DECOMMENTA se vuoi una strofa senza accordi

\[C]Ecco quel che ab\[C7+/E-]biamo, 
 nulla \[F]ci appartiene, or\[C]mai.
Ecco i \[A-]frutti della \[E-]terra, che Tu \[F]moltipliche\[G]rai. \[G7]
\[C]Ecco queste \[C7+/E-]mani, puoi u\[F]sarle, se lo \[C]vuoi, 
per di\[A-]videre nel \[E-]mondo il pane 
 \[F*]che Tu hai \[G7*]dato a \[C]noi. \[C]
\endverse





%%%%% RITORNELLO
\beginchorus
\textnote{\textbf{Rit.}}
\[A-] Solo una goccia hai messo \[E-]fra le mani \[E-7]mie,
solo una goccia che tu \[A]ora chiedi a \[A7]me,
\textit{\[D-7](ah.....\[G]ah..... \[E-]ah.....\[A]ah)}
una \[D-7]goccia che in mano a \[F7]te,
una \[D-7]pioggia divente\[E-7]rà 
e la \[F]terra feconde\[C]rà.
\endchorus





%%%%% STROFA
\beginverse		%Oppure \beginverse* se non si vuole il numero di fianco
%\memorize 		% <<< DECOMMENTA se si vuole utilizzarne la funzione
%\chordsoff		% <<< DECOMMENTA se vuoi una strofa senza accordi
^Sulle strade, il ^vento 
da lon^tano porte^rà
il pro^fumo del fru^mento, che ^tutti avvolge^rà. ^
^E sarà l'a^more che il rac^colto sparti^rà
e il mi^racolo del ^pane 
in terra ^si ri^pete^rà. ^
\endverse


%%%%% RITORNELLO
\beginchorus
\textnote{\textbf{Rit.}}
\[A-] Le nostre gocce, pioggia \[E-]fra le mani \[E-7]tue,
saranno linfa di una \[A]nuova civil\[A7]tà
\textit{\[D-7](ah.....\[G]ah..... \[E-]ah.....\[A]ah)}
e la \[D-7]terra prepare\[F7]rà 
la \[D-7]festa del pane \[E-7]che 
ogni \[F]uomo condivide\[C]rà.
\endchorus




\endsong
%------------------------------------------------------------
%			FINE CANZONE
%------------------------------------------------------------




