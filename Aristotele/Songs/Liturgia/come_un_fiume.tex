%-------------------------------------------------------------
%			INIZIO	CANZONE
%-------------------------------------------------------------


%titolo: 	Come un fiume
%autore: 	Paci, Preti
%tonalita: 	Do 



%%%%%% TITOLO E IMPOSTAZONI
\beginsong{Come un fiume}[by={Paci, Preti}] 	% <<< MODIFICA TITOLO E AUTORE
\transpose{0} 						% <<< TRASPOSIZIONE #TONI (0 nullo)
\momenti{Comunione; Avvento}							% <<< INSERISCI MOMENTI	
% momenti vanno separati da ; e vanno scelti tra:
% Ingresso; Atto penitenziale; Acclamazione al Vangelo; Dopo il Vangelo; Offertorio; Comunione; Ringraziamento; Fine; Santi; Pasqua; Avvento; Natale; Quaresima; Canti Mariani; Battesimo; Prima Comunione; Cresima; Matrimonio; Meditazione;
\ifchorded
	%\textnote{Tonalità originale }	% <<< EV COMMENTI (tonalità originale/migliore)
\fi

%%%%%% INTRODUZIONE
\ifchorded
\vspace*{\versesep}
\textnote{Intro: \qquad \qquad  }%(\eighthnote 116) % << MODIFICA IL TEMPO
% Metronomo: \eighthnote (ottavo) \quarternote (quarto) \halfnote (due quarti)
\vspace*{-\versesep}
\beginverse*

\nolyrics

%---- Prima riga -----------------------------
\vspace*{-\versesep}
\[C] \[G]  \[C]	 % \[*D] per indicare le pennate, \rep{2} le ripetizioni

%---- Ogni riga successiva -------------------
%\vspace*{-\versesep}
%\[G] \[C]  \[D]	

%---- Ev Indicazioni -------------------------			
%\textnote{\textit{(Oppure tutta la strofa)} }	

\endverse
\fi

%%%%% RITORNELLO
\beginchorus
\textnote{\textbf{Rit.}}

Come un \[C]fiume in piena che \brk la sabbia \[G]non può arrestare
come l'\[C7]onda che dal mare si di\[F]stende sulla riva
ti pre\[F-]ghiamo Padre \brk che così si \[C]sciolga il nostro amore
e l'a\[D]more dove ar\[D7]riva sciolga il \[G]dubbio e la \[G7]paura. 

\endchorus

%%%%% STROFA
\beginverse		%Oppure \beginverse* se non si vuole il numero di fianco
\memorize 		% <<< DECOMMENTA se si vuole utilizzarne la funzione
%\chordsoff		& <<< DECOMMENTA se vuoi una strofa senza accordi

Come un \[C]pesce che risale a nuoto \[G]fino alla sorgente
va a sco\[C7]prire dove nasce e si di\ch{F}{f}{f}{ff}onde la sua vita
ti pre\[F-]ghiamo Padre che noi risa\[C]liamo la corrente
fino ad \[G]arrivare alla vita \[F]nell'a\[C]more.  \[(G7)]

\endverse

%%%%% RITORNELLO
\beginchorus
\textnote{\textbf{Rit.}}

Come un \[C]fiume in piena che \brk la sabbia \[G]non può arrestare
come l'\[C7]onda che dal mare si di\[F]stende sulla riva
ti pre\[F-]ghiamo Padre \brk che così si \[C]sciolga il nostro amore
e l'a\[D]more dove ar\[D7]riva sciolga il \[G]dubbio e la \[G7]paura. 

\endchorus

%%%%% STROFA
\beginverse		%Oppure \beginverse* se non si vuole il numero di fianco
%\memorize 		% <<< DECOMMENTA se si vuole utilizzarne la funzione
%\chordsoff		& <<< DECOMMENTA se vuoi una strofa senza accordi

Come l'^erba che germoglia cresce ^senza far rumore
ama il ^giorno della pioggia si addor^menta sotto il sole
ti pre^ghiamo Padre che così in un ^giorno di silenzio
anche in ^noi germogli questa vita ^nell'a^more.  \[A7]

\endverse

\transpose{2}

%%%%% RITORNELLO
\beginchorus
\textnote{\textbf{Rit.}}

Come un \[C]fiume in piena che \brk la sabbia \[G]non può arrestare
come l'\[C7]onda che dal mare si di\[F]stende sulla riva
ti pre\[F-]ghiamo Padre \brk che così si \[C]sciolga il nostro amore
e l'a\[D]more dove ar\[D7]riva sciolga il \[G]dubbio e la \[G7]paura. 

\endchorus

%%%%% STROFA
\beginverse		%Oppure \beginverse* se non si vuole il numero di fianco
%\memorize 		% <<< DECOMMENTA se si vuole utilizzarne la funzione
%\chordsoff		& <<< DECOMMENTA se vuoi una strofa senza accordi

Come un ^albero che affonda le ra^dici nella terra
e su ^quella terra un uomo costru^isce la sua casa
ti pre^ghiamo Padre buono di por^tarci alla tua casa
dove ^vivere una vita piena ^nell'a^more. 
\endverse

\endsong
%------------------------------------------------------------
%			FINE CANZONE
%------------------------------------------------------------
