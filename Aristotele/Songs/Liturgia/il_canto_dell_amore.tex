%-------------------------------------------------------------
%			INIZIO	CANZONE
%-------------------------------------------------------------


%titolo: 	Il canto dell'amore
%autore: 	F. Russo
%tonalita: 	Mi 



%%%%%% TITOLO E IMPOSTAZONI
\beginsong{Il canto dell'amore}[by={F. Russo}] 	% <<< MODIFICA TITOLO E AUTORE
\transpose{0} 						% <<< TRASPOSIZIONE #TONI (0 nullo)
\momenti{Comunione; Meditazione; Ringraziamento}							% <<< INSERISCI MOMENTI	
% momenti vanno separati da ; e vanno scelti tra:
% Ingresso; Atto penitenziale; Acclamazione al Vangelo; Dopo il Vangelo; Offertorio; Comunione; Ringraziamento; Fine; Santi; Pasqua; Avvento; Natale; Quaresima; Canti Mariani; Battesimo; Prima Comunione; Cresima; Matrimonio; Meditazione; Spezzare del pane;
\ifchorded
	%\textnote{Tonalità migliore }	% <<< EV COMMENTI (tonalità originale/migliore)
\fi


%%%%%% INTRODUZIONE
\ifchorded
\vspace*{\versesep}
\musicnote{
\begin{minipage}{0.48\textwidth}
\textbf{Intro}
\hfill 
%( \eighthnote \, 80)   % <<  MODIFICA IL TEMPO
% Metronomo: \eighthnote (ottavo) \quarternote (quarto) \halfnote (due quarti)
\end{minipage}
} 	
\vspace*{-\versesep}
\beginverse*

\nolyrics

%---- Prima riga -----------------------------
\vspace*{-\versesep}
\[E]  \[C#-] \[A] \[E]% \[*D] per indicare le pennate, \rep{2} le ripetizioni

%---- Ogni riga successiva -------------------
%\vspace*{-\versesep}
%\[A*] \[E*] \[A]

%---- Ev Indicazioni -------------------------			
%\textnote{\textit{(Oppure tutta la strofa)} }	

\endverse
\fi





%%%%% STROFA
\beginverse		%Oppure \beginverse* se non si vuole il numero di fianco
\memorize 		% <<< DECOMMENTA se si vuole utilizzarne la funzione
%\chordsoff		% <<< DECOMMENTA se vuoi una strofa senza accordi
Se dov\[E]rai attraversare il de\[C#-]serto
non te\[A]mere io sarò con \[E]te
se do\[E]vrai camminare nel \[C#-]fuoco
la sua \[A]fiamma non ti bruce\[E]rà
segui\[B]rai la mia \[A]luce nella \[E]notte \[E]
senti\[F#-]rai la mia \[B]forza nel cam\[C#-7]mino \[C#-7]
io \[D9]sono il tuo Dio, \[A9] il Signo-o\[E]re. \[C#-] \[A] \[E]
\endverse


%%%%% STROFA
\beginverse		%Oppure \beginverse* se non si vuole il numero di fianco
%\memorize 		% <<< DECOMMENTA se si vuole utilizzarne la funzione
%\chordsoff		% <<< DECOMMENTA se vuoi una strofa senza accordi
Sono ^io che ti ho fatto e pla^smato
ti ho chia^mato per ^nome
io da ^sempre ti ho conosc^iuto
e ti ho ^dato il mio a^more
perché ^tu sei pre^zioso ai miei ^occhi ^
vali ^più del più ^grande dei te^sori ^
io sa^rò con te ^ dovunque an^drai. ^ ^ ^
\endverse


%%%%% STROFA
\beginverse		%Oppure \beginverse* se non si vuole il numero di fianco
%\memorize 		% <<< DECOMMENTA se si vuole utilizzarne la funzione
%\chordsoff		% <<< DECOMMENTA se vuoi una strofa senza accordi
Non pen^sare alle cose di ^ieri
cose ^nuove fioriscono ^già
apri^rò nel deserto senti^eri
darò ^acqua nell'aridi^tà
perché ^tu sei prez^ioso ai miei ^occhi ^
vali ^più del più ^grande dei te^sori ^
io sa^rò con te ^ dovunque and^rai ^ ^ ^
\endverse



%%%%% STROFA
\beginverse*		%Oppure \beginverse* se non si vuole il numero di fianco
%\memorize 		% <<< DECOMMENTA se si vuole utilizzarne la funzione
%\chordsoff		% <<< DECOMMENTA se vuoi una strofa senza accordi
perché \[B]tu sei prezi\[A]oso ai miei \[E]occhi \[E]
vali \[F#-]più del più \[B]grande dei te\[C#-7]sori \[C#-7]
io sa\[D9]rò con te \[A9] dovunque andr\[E]ai. \[E]
\endverse

%%%%% STROFA
\beginverse*		%Oppure \beginverse* se non si vuole il numero di fianco
%\memorize 		% <<< DECOMMENTA se si vuole utilizzarne la funzione
%\chordsoff		% <<< DECOMMENTA se vuoi una strofa senza accordi
\vspace*{1.3\versesep}
\textnote{\textbf{Finale} } %<<< EV. INDICAZIONI
\[E] Io ti sa\[C#-]rò accanto \[A]sarò con \[E]te 
\[E] per tutto il \[C#-]tuo viaggio \[A]sarò con \[E]te 
\[E] io ti sa\[C#-]rò accanto \[A]sarò con \[E]te 
\[E] per tutto il \[C#-]tuo viaggio \[A]sarò con \[E]te \[E*]
\endverse







\endsong
%------------------------------------------------------------
%			FINE CANZONE
%------------------------------------------------------------


