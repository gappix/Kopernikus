%-------------------------------------------------------------
%			INIZIO	CANZONE
%-------------------------------------------------------------


%titolo: 	Mani
%autore: 	Colombo
%tonalita: 	Do 



%%%%%% TITOLO E IMPOSTAZONI
\beginsong{Mani}[by={F. Colombo}] 	% <<< MODIFICA TITOLO E AUTORE
\transpose{-2} 						% <<< TRASPOSIZIONE #TONI (0 nullo)
\momenti{Comunione; Congedo}							% <<< INSERISCI MOMENTI	
% momenti vanno separati da ; e vanno scelti tra:
% Ingresso; Atto penitenziale; Acclamazione al Vangelo; Dopo il Vangelo; Offertorio; Comunione; Ringraziamento; Fine; Santi; Pasqua; Avvento; Natale; Quaresima; Canti Mariani; Battesimo; Prima Comunione; Cresima; Matrimonio; Meditazione; Spezzare del pane;
\ifchorded
	%\textnote{Tonalità originale }	% <<< EV COMMENTI (tonalità originale/migliore)
\fi

%%%%%% INTRODUZIONE
\ifchorded
\vspace*{\versesep}
\musicnote{
\begin{minipage}{0.48\textwidth}
\textbf{Intro}
\hfill 
%( \eighthnote \, 80)   % <<  MODIFICA IL TEMPO
% Metronomo: \eighthnote (ottavo) \quarternote (quarto) \halfnote (due quarti)
\end{minipage}
} 	
\vspace*{-\versesep}
\beginverse*

\nolyrics

%---- Prima riga -----------------------------
\vspace*{-\versesep}
\[D] \[G] \[A] \[D]	 % \[*D] per indicare le pennate, \rep{2} le ripetizioni

%---- Ogni riga successiva -------------------
%\vspace*{-\versesep}
%\[G] \[C]  \[D]	

%---- Ev Indicazioni -------------------------			
%\textnote{\textit{(Oppure tutta la strofa)} }	

\endverse
\fi

%%%%% STROFA
\beginverse		%Oppure \beginverse* se non si vuole il numero di fianco
\memorize 		% <<< DECOMMENTA se si vuole utilizzarne la funzione
%\chordsoff		% <<< DECOMMENTA se vuoi una strofa senza accordi

Vor\[D]rei che le pa\[G]role mu\[A]tassero in pre\[D]ghiera
e \[G]rivederti o \[B-]Padre che \[G]dipingevi il \[A]cielo.
Sa\[D]pessi quante \[G]volte guar\[A]dando questo \[D]mondo
vor\[G]rei che tu tor\[B-]nassi a \[G]ritoc\[A]carne il \[D]cuore.
\vspace{\versesep}
Vor\[B-]rei che le mie \[F#-]mani a\[G]vessero la \[D]forza
per \[G]sostenere chi \[E]non può cammi\[A]nare.
Vor\[B-]rei che questo \[F#-]cuore che e\[G]splode in senti\[D]menti
\[G]diventasse \[B-]culla per \[G]chi non ha più \[A]madre.

\endverse

%%%%% RITORNELLO
\beginchorus
\textnote{\textbf{Rit.}}

\[D]Mani, prendi queste mie \[A]mani,
fanne vita, fanne a\[G]more
braccia aperte per ri\[B-]ceve\[A]re chi è solo.
\[D]Cuore, prendi questo mio \[A]cuore,
fa' che si spalanchi al \[G]mondo
germogliando per quegli \[B-]occhi
che non \[A]sanno pianger \[G]più. \[A]

\endchorus

%%%%% STROFA
\beginverse		%Oppure \beginverse* se non si vuole il numero di fianco
%\memorize 		% <<< DECOMMENTA se si vuole utilizzarne la funzione
%\chordsoff		% <<< DECOMMENTA se vuoi una strofa senza accordi

Sei ^tu lo spazio ^che de^sidero da ^sempre,
so ^che mi stringe^rai e ^mi terrai la ^mano.
^Fa' che le mie ^strade si ^perdano nel ^buio
ed ^io cammini ^dove cam^mine^resti ^Tu.
\vspace{\versesep}
Tu ^soffio della ^vita ^prendi la mia giovi^nezza
con ^le contraddi^zioni e le falsi^tà.
Stru^mento fa' che ^sia per an^nunciare il ^Regno
a ^chi per queste ^vie Tu ^chiami Be^ati.

\endverse

%%%%% STROFA
\beginverse		%Oppure \beginverse* se non si vuole il numero di fianco
%\memorize 		% <<< DECOMMENTA se si vuole utilizzarne la funzione
%\chordsoff		% <<< DECOMMENTA se vuoi una strofa senza accordi

Noi ^giovani di un ^mondo che can^cella i senti^menti
e in^scatola le ^forze nell'a^sfalto di cit^tà.
\vspace{\versesep}
Siamo \[B-]stanchi di guar\[F#-]dare 
siamo s\[G]tanchi di gri\[D]dare
ci hai chia\[G]mati siamo \[B-]Tuoi cam\[G]mineremo in\[A]sieme.

\endverse



%%%%% RITORNELLO
\beginchorus
\textnote{\textbf{Rit.}}

\[D]Mani, prendi queste mie \[A]mani,
fanne vita, fanne a\[G]more
braccia aperte per ri\[B-]ceve\[A]re chi è solo.
\[D]Cuore, prendi questo mio \[A]cuore,
fa' che si spalanchi al \[G]mondo
germogliando per quegli \[B-]occhi
che non \[A]sanno pianger\dots

\vspace{\versesep}

\[D]Mani, prendi queste nostre \[A]mani,
fanne vita, fanne a\[G]more
braccia aperte per ri\[B-]ceve\[A]re chi è solo.
\[D]Cuori, prendi questi nostri \[A]cuori,
 a che siano testi\[G]moni
che tu chiami ogni \[B-]uomo 
a far \[A]festa con \[G]Dio. \[A]  \[D*]

\endchorus
\endsong
%------------------------------------------------------------
%			FINE CANZONE
%------------------------------------------------------------



















%++++++++++++++++++++++++++++++++++++++++++++++++++++++++++++
%			CANZONE TRASPOSTA
%++++++++++++++++++++++++++++++++++++++++++++++++++++++++++++
% \ifchorded
% %decremento contatore per avere stesso numero
% \addtocounter{songnum}{-1} 
% \beginsong{Mani}[by={Colombo}] 	% <<< COPIA TITOLO E AUTORE
% 					% <<< TRASPOSIZIONE #TONI + - (0 nullo)
% %\preferflats  %SE VOGLIO FORZARE i bemolle come alterazioni
% %\prefersharps %SE VOGLIO FORZARE i # come alterazioni
% \ifchorded
% 	%\textnote{Tonalità originale}	% <<< EV COMMENTI (tonalità originale/migliore)
% \fi


% %%%%%% INTRODUZIONE
% \ifchorded
% \vspace*{\versesep}
% \textnote{Intro: \qquad \qquad  }%(\eighthnote 116) % <<  MODIFICA IL TEMPO
% % Metronomo: \eighthnote (ottavo) \quarternote (quarto) \halfnote (due quarti)
% \vspace*{-\versesep}
% \beginverse*

% \nolyrics

% %---- Prima riga -----------------------------
% \vspace*{-\versesep}
% \[D] \[G] \[A] \[D]	 % \[*D] per indicare le pennate, \rep{2} le ripetizioni

% %---- Ogni riga successiva -------------------
% %\vspace*{-\versesep}
% %\[G] \[C]  \[D]	

% %---- Ev Indicazioni -------------------------			
% %\textnote{\textit{(Oppure tutta la strofa)} }	

% \endverse
% \fi

% %%%%% STROFA
% \beginverse		%Oppure \beginverse* se non si vuole il numero di fianco
% \memorize 		% <<< DECOMMENTA se si vuole utilizzarne la funzione
% %\chordsoff		% <<< DECOMMENTA se vuoi una strofa senza accordi

% Vor\[D]rei che le pa\[G]role mu\[A]tassero in pre\[D]ghiera
% e \[G]rivederti o \[B-]Padre che \[G]dipingevi il \[A]cielo.
% Sa\[D]pessi quante \[G]volte guar\[A]dando questo \[D]mondo
% vor\[G]rei che tu tor\[B-]nassi a \[G]ritoc\[A]carne il \[D]cuore.
% \vspace{\versesep}
% Vor\[B-]rei che le mie \[F#-]mani a\[G]vessero la \[D]forza
% per \[G]sostenere chi \[E]non può cammi\[A]nare.
% Vor\[B-]rei che questo \[F#-]cuore che e\[G]splode in senti\[D]menti
% \[G]diventasse \[B-]culla per \[G]chi non ha più \[A]madre.

% \endverse

% %%%%% RITORNELLO
% \beginchorus
% \textnote{\textbf{Rit.}}

% \[D]Mani, prendi queste mie \[A]mani,
% fanne vita, fanne a\[G]more
% braccia aperte per ri\[B-]ceve\[A]re chi è solo.
% \[D]Cuore, prendi questo mio \[A]cuore,
% fa' che si spalanchi al \[G]mondo
% germogliando per quegli \[B-]occhi
% che non \[A]sanno pianger \[G]più. \[A]

% \endchorus

% %%%%% STROFA
% \beginverse		%Oppure \beginverse* se non si vuole il numero di fianco
% %\memorize 		% <<< DECOMMENTA se si vuole utilizzarne la funzione
% %\chordsoff		% <<< DECOMMENTA se vuoi una strofa senza accordi

% Sei ^tu lo spazio ^che de^sidero da ^sempre,
% so ^che mi stringe^rai e ^mi terrai la ^mano.
% ^Fa' che le mie ^strade si ^perdano nel ^buio
% ed ^io cammini ^dove cam^mine^resti ^Tu.
% \vspace{\versesep}
% Tu ^soffio della ^vita ^prendi la mia giovi^nezza
% con ^le contraddi^zioni e le falsi^tà.
% Stru^mento fa' che ^sia per an^nunciare il ^Regno
% a ^chi per queste ^vie Tu ^chiami Be^ati.

% \endverse

% %%%%% STROFA
% \beginverse		%Oppure \beginverse* se non si vuole il numero di fianco
% %\memorize 		% <<< DECOMMENTA se si vuole utilizzarne la funzione
% %\chordsoff		% <<< DECOMMENTA se vuoi una strofa senza accordi

% Noi ^giovani di un ^mondo che can^cella i senti^menti
% e in^scatola le ^forze nell'a^sfalto di cit^tà.
% \vspace{\versesep}
% Siamo \[B-]stanchi di guar\[F#-]dare 
% siamo s\[G]tanchi di gri\[D]dare
% ci hai chia\[G]mati siamo \[B-]Tuoi cam\[G]mineremo in\[A]sieme.

% \endverse




% %%%%% RITORNELLO
% \beginchorus
% \textnote{\textbf{Rit.}}

% \[D]Mani, prendi queste mie \[A]mani,
% fanne vita, fanne a\[G]more
% braccia aperte per ri\[B-]ceve\[A]re chi è solo.
% \[D]Cuore, prendi questo mio \[A]cuore,
% fa' che si spalanchi al \[G]mondo
% germogliando per quegli \[B-]occhi
% che non \[A]sanno pianger\dots

% \vspace{\versesep}

% \[D]Mani, prendi queste nostre \[A]mani,
% fanne vita, fanne a\[G]more
% braccia aperte per ri\[B-]ceve\[A]re chi è solo.
% \[D]Cuori, prendi questi nostri \[A]cuori,
%  a che siano testi\[G]moni
% che tu chiami ogni \[B-]uomo 
% a far \[A]festa con \[G]Dio. \[A]  \[D*]

% \endchorus

% \endsong

% \fi
%++++++++++++++++++++++++++++++++++++++++++++++++++++++++++++
%			FINE CANZONE TRASPOSTA
%++++++++++++++++++++++++++++++++++++++++++++++++++++++++++++