%-------------------------------------------------------------
%			INIZIO	CANZONE
%-------------------------------------------------------------


%titolo: 	Vocazione
%autore: 	Sequeri
%tonalita: 	Do 



%%%%%% TITOLO E IMPOSTAZONI
\beginsong{Vocazione}[by={P. Sequeri}] 	% <<< MODIFICA TITOLO E AUTORE
\transpose{0} 						% <<< TRASPOSIZIONE #TONI (0 nullo)
\momenti{Comunione; Cresima}							% <<< INSERISCI MOMENTI	
% momenti vanno separati da ; e vanno scelti tra:
% Ingresso; Atto penitenziale; Acclamazione al Vangelo; Dopo il Vangelo; Offertorio; Comunione; Ringraziamento; Fine; Santi; Pasqua; Avvento; Natale; Quaresima; Canti Mariani; Battesimo; Prima Comunione; Cresima; Matrimonio; Meditazione; Spezzare del pane;
\ifchorded
	%\textnote{Tonalità originale }	% <<< EV COMMENTI (tonalità originale/migliore)
\fi


%%%%%% INTRODUZIONE
\ifchorded
\vspace*{\versesep}
\musicnote{
\begin{minipage}{0.48\textwidth}
\textbf{Intro}
\hfill 
%( \eighthnote \, 80)   % <<  MODIFICA IL TEMPO
% Metronomo: \eighthnote (ottavo) \quarternote (quarto) \halfnote (due quarti)
\end{minipage}
} 	
\vspace*{-\versesep}
\beginverse*
\nolyrics

%---- Prima riga -----------------------------
\vspace*{-\versesep}
\[C] \[G] \[F]	  \[F] \rep{2} % \[*D] per indicare le pennate, \rep{2} le ripetizioni

%---- Ogni riga successiva -------------------
%\vspace*{-\versesep}
%\[G] \[C]  \[D]	

%---- Ev Indicazioni -------------------------			
%\textnote{\textit{(Oppure tutta la strofa)} }	

\endverse
\fi



%%%%% STROFA
\beginverse		%Oppure \beginverse* se non si vuole il numero di fianco
\memorize 		% <<< DECOMMENTA se si vuole utilizzarne la funzione
%\chordsoff		% <<< DECOMMENTA se vuoi una strofa senza accordi
\[C]Era un giorno \[G]come tanti \[F]altri,
e quel \[G]giorno Lui pas\[C]sò. \[F*]\[C*]\[G*]
\[C]Era un uomo \[G]come tutti gli \[F]altri,
e pas\[G]sando mi chia\[C]mò \[F*]\[C*]\[E*]
\[A-]come lo sa\[E-]pesse che il mio \[F]nome \brk era \[G]proprio quello
\[C]come mai ve\[G]desse proprio \[F]me
nella sua \[G]vita, non lo \[C]so. \[F*]\[C*]\[G*]
\[C]Era un giorno \[G]come tanti \[F]altri
e quel \[G]giorno mi chia\[C]mò. \[F*]\[C*]\[E*]
\endverse



%%%%% RITORNELLO
\beginchorus
\textnote{\textbf{Rit.}}
\[A-]Tu \[E-]Dio, \[F]che conosci il \[G]nome mio
\[A-]fa' \[E-]che \[F]ascoltando \[G]la tua voce
\[C]io ri\[G]cordi dove \[F]porta la mia \[G]strada
\[C]nella \[G]vita, all'in\[F]contro con \[C]Te. \[F*]\[C*]\[G*]
\endchorus



%%%%% STROFA
\beginverse		%Oppure \beginverse* se non si vuole il numero di fianco
%\memorize 		% <<< DECOMMENTA se si vuole utilizzarne la funzione
%\chordsoff		% <<< DECOMMENTA se vuoi una strofa senza accordi
^Era l'alba ^triste e senza ^vita,
e ^qualcuno mi chia^mò ^^^
^era un uomo ^come tanti ^altri,
ma la ^voce, quella ^no. ^^^
^Quante volte un ^uomo
con il ^nome giusto ^mi ha chiamato
^una volta ^sola l'ho sen^tito
pronun^ciare con a^more. ^^^
^Era un uomo ^come nessun ^altro
e quel ^giorno mi chia^mò. ^^^
\endverse

\ifchorded
%%%%% RITORNELLO
\beginchorus
\textnote{\textbf{Rit.}}
\[A-]Tu \[E-]Dio, \[F]che conosci il \[G]nome mio
\[A-]fa' \[E-]che \[F]ascoltando \[G]la tua voce
\[C]io ri\[G]cordi dove \[F]porta la mia \[G]strada
\[C]nella \[G]vita, all'in\[F]contro con \[C]Te. \[C]
\endchorus
\fi
\endsong
%------------------------------------------------------------
%			FINE CANZONE
%------------------------------------------------------------


