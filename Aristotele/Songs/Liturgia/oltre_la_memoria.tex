%-------------------------------------------------------------
%			INIZIO	CANZONE
%-------------------------------------------------------------


%titolo: 	Oltre la memoria (Symbolum 80)
%autore: 	Sequeri
%tonalita: 	Sol 



%%%%%% TITOLO E IMPOSTAZONI
\beginsong{Oltre la memoria }[ititle={Symbolum 80}, by={Ma la tua parola — Symbolum 80 — P. Sequeri}] 	% <<< MODIFICA TITOLO E AUTORE
\transpose{0} 						% <<< TRASPOSIZIONE #TONI (0 nullo)
\momenti{Comunione}							% <<< INSERISCI MOMENTI	
% momenti vanno separati da ; e vanno scelti tra:
% Ingresso; Atto penitenziale; Acclamazione al Vangelo; Dopo il Vangelo; Offertorio; Comunione; Ringraziamento; Fine; Santi; Pasqua; Avvento; Natale; Quaresima; Canti Mariani; Battesimo; Prima Comunione; Cresima; Matrimonio; Meditazione;
\ifchorded
	\textnote{$\bigstar$ Tonalità migliore}	% <<< EV COMMENTI (tonalità originale/migliore)
\fi



%%%%%% INTRODUZIONE
\ifchorded
\vspace*{\versesep}
\musicnote{
\begin{minipage}{0.48\textwidth}
\textbf{Intro}
\hfill 
( \eighthnote \, 136)   % <<  MODIFICA IL TEMPO
% Metronomo: \eighthnote (ottavo) \quarternote (quarto) \halfnote (due quarti)
\end{minipage}
} 	
\vspace*{-\versesep}
\beginverse*



\nolyrics

%---- Prima riga -----------------------------
\vspace*{-\versesep}
\[D-] \[D-] \[G-]  \[D-]	 % \[*D] per indicare le pennate, \rep{2} le ripetizioni

%---- Ogni riga successiva -------------------
%\vspace*{-\versesep}
%\[G] \[C]  \[G]	

%---- Ev Indicazioni -------------------------			
%\textnote{\textit{(Oppure tutta la strofa)} }	

\endverse
\fi



%%%%% STROFA
\beginverse
\memorize


\[D-]Oltre la me\[7]moria del \[G-]tempo che ho vis\[D-]suto,
\[D-]oltre la spe\[A-]ranza 
che \[G-]serve al mio do\[A]mani, \quad \[A]
\[D-]oltre il desi\[7]derio di \[G-]vivere il pre\[D-]sente,
an\[D-]ch’io, confesso, ho chi\[A-]esto 
che \[G-]cosa è veri\[A]tà. 

\vspace{1.5\versesep}


\[D]E \[A]tu \[B-]come un desi\[F#-]derio 
\[G]che non \[A]ha me\[B-]morie, \[E]Padre bu\[A]ono, 
\[D]come una spe\[A]ranza c\[G]he non ha con\[F#-]fini,
\[G]come un \[A]tempo e\[B-]terno 
\[E7]sei per \[A]me.

\endverse


%%%%% RITORNELLO
\beginchorus
\textnote{\textbf{Rit.}}

\[F]Io \[C]so \[D-]quanto amore ch\[A-]iede 
\[B&*]questa \[C*]lunga at\[F]te\[(D-)]sa 
\[G]del tuo giorno, \[C]Dio; 
\[F]luce in ogni \[C]cosa \[D-]io non vedo an\[A-]cora:
\[B&*]ma la \[C7*]tua pa\[F]ro\[(D-)]la 
\[G]mi risch\[7]iare\[A]rà!

\endchorus




%%%%% STROFA
\beginverse

^Quando le pa^role non ^bastano all’a^more,
^quando il mio fra^tello 
do^manda più del ^pane, \quad ^
^quando l’illu^sione pro^mette un mondo ^nuovo,
anch’^io rimango in^certo 
nel ^mezzo del cam^mino.

\vspace{1.5\versesep}

^E ^tu ^Figlio tanto a^mato,
^veri^tà dell’^uomo, ^mio Si^gnore,
^come la pro^messa ^di un perdono e^terno,
^liber^tà infi^nita 
^sei per ^me.


\endverse


%%%%% RITORNELLO
\beginchorus
\textnote{\textbf{Rit.}}

\[F]Io \[C]so \[D-]quanto amore ch\[A-]iede 
\[B&*]questa \[C*]lunga at\[F]te\[(D-)]sa 
\[G]del tuo giorno, \[C]Dio; 
\[F]luce in ogni \[C]cosa \[D-]io non vedo an\[A-]cora:
\[B&*]ma la \[C7*]tua pa\[F]ro\[(D-)]la 
\[G]mi risch\[7]iare\[A]rà!

\endchorus




%%%%% STROFA
\beginverse


^Chiedo alla mia ^mente co^raggio di cer^care,
^chiedo alle mie ^mani 
la ^forza di do^nare, \quad ^
^chiedo al cuore in^certo pas^sione per la ^vita,
e ^chiedo a te fra^tello 
di ^credere con ^me.

\vspace{1.5\versesep}

^E ^tu, ^forza della ^vita,
^Spiri^to d’a^more, ^dolce Id^dio,
^grembo d’ogni ^cosa, ^tenerezza im^mensa,
^veri^tà del ^mondo 
^sei per ^me.


\endverse


%%%%% RITORNELLO
\beginchorus
\textnote{\textbf{Rit.}}

\[F]Io \[C]so \[D-]quanto amore ch\[A-]iede 
\[B&*]questa \[C*]lunga at\[F]te\[(D-)]sa 
\[G]del tuo giorno, \[C]Dio; 
\[F]luce in ogni \[C]cosa \[D-]io non vedo an\[A-]cora:
\[B&*]ma la \[C7*]tua pa\[F]ro\[(D-)]la 

\endchorus




%%%%%% EV. FINALE
\beginchorus %oppure \beginverse*
\vspace*{1.3\versesep}
\textnote{\textbf{Finale} \textit{(rallentando)}} %<<< EV. INDICAZIONI

\[E]mi risch\[A]iare\[D]rà!

\endchorus  %oppure \endverse






\endsong
%------------------------------------------------------------
%			FINE CANZONE
%------------------------------------------------------------




%++++++++++++++++++++++++++++++++++++++++++++++++++++++++++++
%			CANZONE TRASPOSTA
%++++++++++++++++++++++++++++++++++++++++++++++++++++++++++++
\ifchorded
%decremento contatore per avere stesso numero
\addtocounter{songnum}{-1} 
\beginsong{Oltre la memoria}[ititle={Ma la tua parola}, by={Ma la tua parola — Symbolum 80 — P. Sequeri}] 	% <<< MODIFICA TITOLO E AUTORE
\transpose{2} 						% <<< TRASPOSIZIONE #TONI (0 nullo)
%\preferflats %SE VOGLIO FORZARE i bemolle come alterazioni
\prefersharps %SE VOGLIO FORZARE i # come alterazioni
\ifchorded
	\textnote{$\lozenge$ Tonalità originale}	% <<< EV COMMENTI (tonalità originale/migliore)
\fi




%%%%%% INTRODUZIONE
\ifchorded
\vspace*{\versesep}
\musicnote{
\begin{minipage}{0.48\textwidth}
\textbf{Intro}
\hfill 
( \eighthnote \, 136)   % <<  MODIFICA IL TEMPO
% Metronomo: \eighthnote (ottavo) \quarternote (quarto) \halfnote (due quarti)
\end{minipage}
} 	
\vspace*{-\versesep}
\beginverse*


\nolyrics

%---- Prima riga -----------------------------
\vspace*{-\versesep}
\[D-] \[D-] \[G-]  \[D-]	 % \[*D] per indicare le pennate, \rep{2} le ripetizioni

%---- Ogni riga successiva -------------------
%\vspace*{-\versesep}
%\[G] \[C]  \[G]	

%---- Ev Indicazioni -------------------------			
%\textnote{\textit{(Oppure tutta la strofa)} }	

\endverse
\fi



%%%%% STROFA
\beginverse
\memorize


\[D-]Oltre la me\[7]moria del \[G-]tempo che ho vis\[D-]suto,
\[D-]oltre la spe\[A-]ranza 
che \[G-]serve al mio do\[A]mani, \quad \[A]
\[D-]oltre il desi\[7]derio di \[G-]vivere il pre\[D-]sente,
an\[D-]ch’io, confesso, ho chi\[A-]esto 
che \[G-]cosa è veri\[A]tà. 

\vspace{1.5\versesep}

\[D]E \[A]tu \[B-]come un desi\[F#-]derio 
\[G]che non \[A]ha me\[B-]morie, \[E]Padre bu\[A]ono, 
\[D]come una spe\[A]ranza c\[G]he non ha con\[F#-]fini,
\[G]come un \[A]tempo e\[B-]terno 
\[E7]sei per \[A]me.

\endverse


%%%%% RITORNELLO
\beginchorus
\textnote{\textbf{Rit.}}

\[F]Io \[C]so \[D-]quanto amore ch\[A-]iede 
\[B&*]questa \[C*]lunga at\[F]te\[(D-)]sa 
\[G]del tuo giorno, \[C]Dio; 
\[F]luce in ogni \[C]cosa \[D-]io non vedo an\[A-]cora:
\[B&*]ma la \[C7*]tua pa\[F]ro\[(D-)]la 
\[G]mi risch\[7]iare\[A]rà!

\endchorus



%%%%% STROFA
\beginverse

^Quando le pa^role non ^bastano all’a^more,
^quando il mio fra^tello 
do^manda più del ^pane, \quad ^
^quando l’illu^sione pro^mette un mondo ^nuovo,
anch’^io rimango in^certo 
nel ^mezzo del cam^mino.

\vspace{1.5\versesep}

^E ^tu ^Figlio tanto a^mato,
^veri^tà dell’^uomo, ^mio Si^gnore,
^come la pro^messa ^di un perdono e^terno,
^liber^tà infi^nita 
^sei per ^me.


\endverse


%%%%% RITORNELLO
\beginchorus
\textnote{\textbf{Rit.}}

\[F]Io \[C]so \[D-]quanto amore ch\[A-]iede 
\[B&*]questa \[C*]lunga at\[F]te\[(D-)]sa 
\[G]del tuo giorno, \[C]Dio; 
\[F]luce in ogni \[C]cosa \[D-]io non vedo an\[A-]cora:
\[B&*]ma la \[C7*]tua pa\[F]ro\[(D-)]la 
\[G]mi risch\[7]iare\[A]rà!

\endchorus




%%%%% STROFA
\beginverse


^Chiedo alla mia ^mente co^raggio di cer^care,
^chiedo alle mie ^mani 
la ^forza di do^nare, \quad ^
^chiedo al cuore in^certo pas^sione per la ^vita,
e ^chiedo a te fra^tello 
di ^credere con ^me.

\vspace{1.5\versesep}

^E ^tu, ^forza della ^vita,
^Spiri^to d’a^more, ^dolce Id^dio,
^grembo d’ogni ^cosa, ^tenerezza im^mensa,
^veri^tà del ^mondo 
^sei per ^me.


\endverse



%%%%% RITORNELLO
\beginchorus
\textnote{\textbf{Rit.}}

\[F]Io \[C]so \[D-]quanto amore ch\[A-]iede 
\[B&*]questa \[C*]lunga at\[F]te\[(D-)]sa 
\[G]del tuo giorno, \[C]Dio; 
\[F]luce in ogni \[C]cosa \[D-]io non vedo an\[A-]cora:
\[B&*]ma la \[C7*]tua pa\[F]ro\[(D-)]la 

\endchorus



%%%%%% EV. FINALE
\beginchorus %oppure \beginverse*
\vspace*{1.3\versesep}
\textnote{\textbf{Finale} \textit{(rallentando)}} %<<< EV. INDICAZIONI

\[E]mi risch\[A]iare\[D]rà!

\endchorus  %oppure \endverse





\endsong

\fi
%++++++++++++++++++++++++++++++++++++++++++++++++++++++++++++
%			FINE CANZONE TRASPOSTA
%++++++++++++++++++++++++++++++++++++++++++++++++++++++++++++
