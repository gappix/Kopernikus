%-------------------------------------------------------------
%			INIZIO	CANZONE
%-------------------------------------------------------------


%titolo: 	Come tu mi vuoi
%autore: 	Daniele Ricci
%tonalita: 	Sol 



%%%%%% TITOLO E IMPOSTAZONI
\beginsong{Come tu mi vuoi}[by={D. Branca}] 	% <<< MODIFICA TITOLO E AUTORE
\transpose{0} 						% <<< TRASPOSIZIONE #TONI (0 nullo)
%\preferflats  %SE VOGLIO FORZARE i bemolle come alterazioni
%\prefersharps %SE VOGLIO FORZARE i # come alterazioni
\momenti{Comunione, Ringraziamento; Cresima; Prima Comunione}							% <<< INSERISCI MOMENTI	
% momenti vanno separati da ; e vanno scelti tra:
% Ingresso; Atto penitenziale; Acclamazione al Vangelo; Dopo il Vangelo; Offertorio; Comunione; Ringraziamento; Fine; Santi; Pasqua; Avvento; Natale; Quaresima; Canti Mariani; Battesimo; Prima Comunione; Cresima; Matrimonio; Meditazione; Spezzare del pane;
\ifchorded
	%\textnote{$\bigstar$ Tonalità migliore }	% <<< EV COMMENTI (tonalità originale\migliore)
\fi


%%%%%% INTRODUZIONE
\ifchorded
\vspace*{\versesep}
\musicnote{
\begin{minipage}{0.48\textwidth}
\textbf{Intro}
\hfill 
%( \eighthnote \, 80)   % <<  MODIFICA IL TEMPO
% Metronomo: \eighthnote (ottavo) \quarternote (quarto) \halfnote (due quarti)
\end{minipage}
} 	
\vspace*{-\versesep}
\beginverse*

\nolyrics

%---- Prima riga -----------------------------
\vspace*{-\versesep}
 \[G] \[C] \[D7] \[G]	 % \[*D] per indicare le pennate, \rep{2} le ripetizioni

%---- Ogni riga successiva -------------------
\vspace*{-\versesep}
\[E-] \[A-] \[D4] \[G] 	

%---- Ev Indicazioni -------------------------			
%\textnote{\textit{[oppure tutta la strofa]} }	

\endverse
\fi




%%%%% STROFA
\beginverse		%Oppure \beginverse* se non si vuole il numero di fianco
\memorize 		% <<< DECOMMENTA se si vuole utilizzarne la funzione
%\chordsoff		% <<< DECOMMENTA se vuoi una strofa senza accordi

\[G]Eccomi Signor, \[A-7]vengo a te mio Re,
\[E-]che si compia in me la tua \[G]volon\[D]tà.
\[G]Eccomi Signor, \[A-7]vengo a te mio Dio,
\[E-]plasma il cuore mio \[B-7]e di te vivrò.
\[G]Se tu lo \[C]vuoi Si\[D]gnore manda \[E-]me
\[A-]e il tuo nome \[B-]annun\[G]ce\[C]rò.

\endverse




%%%%% RITORNELLO
\beginchorus
\textnote{\textbf{Rit.}}

Come tu mi \[G]vuoi io sa\[D]rò,
dove tu mi \[E-]vuoi io an\[B-7]drò.
Questa \[C]vita io voglio dona\[B-7]rla a \[E-]te
per dar \[F]gloria al tuo nome \[C]mio \[D4]Re.
\[D]Come tu mi \[C]vuoi io sa\[D]rò,
\[B7]dove tu mi \[E-]vuoi io an\[B-7]drò.
Se mi gu\[C]ida il tuo amore pa\[B7]ura non \[E-]ho,   
per se\[A-]mpre io sa\[D]rò  come tu mi vuoi.

\endchorus



%%%%%% EV. INTERMEZZO
\beginverse*
\vspace*{1.3\versesep}
{
	\nolyrics
	\textnote{Intermezzo strumentale}
	
	\ifchorded

	%---- Prima riga -----------------------------
	\vspace*{-\versesep}
	\[G] \[C] \[D7] \[G]




	\fi
	%---- Ev Indicazioni -------------------------			
	%\textnote{\textit{(ripetizione della strofa)}} 
	 
}
\vspace*{\versesep}
\endverse




%%%%% STROFA
\beginverse		%Oppure \beginverse* se non si vuole il numero di fianco
%\memorize 		% <<< DECOMMENTA se si vuole utilizzarne la funzione
%\chordsoff		% <<< DECOMMENTA se vuoi una strofa senza accordi

\[G]Eccomi Signor, \[A-7]vengo a te mio Re,
\[E-]che si compia in me la tua \[G]volon\[D]tà.
\[G]Eccomi Signor, \[A-7]vengo a te mio Dio,
\[E-]plasma il cuore mio \[B-7]e di te vivrò..
\[G]Tra le tue \[C]mani mai \[D]più vacil\[E-]lerò
\[A-]e strumento \[B-]tuo \[G]sa\[C]rò.

\endverse







%%%%% RITORNELLO
\beginchorus
\textnote{\textbf{Rit.}}

Come tu mi \[G]vuoi io sa\[D]rò,
dove tu mi \[E-]vuoi io an\[B-7]drò.
Questa \[C]vita io voglio dona\[B-7]rla a \[E-]te
per dar \[F]gloria al tuo nome \[C]mio \[D4]Re.
\[D]Come tu mi \[C]vuoi io sa\[D]rò,
\[B7]dove tu mi \[E-]vuoi io an\[B-7]drò.
Se mi gu\[C]ida il tuo amore pa\[B7]ura non \[E-]ho,   
per se\[A-]mpre io sa\[D]rò  come tu mi vuoi.

\endchorus




%%%%%% EV. FINALE

\beginchorus %oppure \beginverse*
\vspace*{1.3\versesep}
\textnote{\textbf{Finale }} %<<< EV. INDICAZIONI
come tu mi vuoi\[C] \[D].\echo{Io sarò}  \rep{4} 

\endchorus  %oppure \endverse






%%%%%% EV. CHIUSURA SOLO STRUMENTALE
\ifchorded
\beginchorus %oppure \beginverse*
\vspace*{1.3\versesep}
\textnote{Chiusura } %<<< EV. INDICAZIONI

\[G*]

\endchorus  %oppure \endverse
\fi


\endsong
%------------------------------------------------------------
%			FINE CANZONE
%------------------------------------------------------------



