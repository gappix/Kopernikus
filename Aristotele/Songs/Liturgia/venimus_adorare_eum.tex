%-------------------------------------------------------------
%			INIZIO	CANZONE
%-------------------------------------------------------------


%titolo: 	Venimus adorare eum
%autore: 	Lissen
%tonalita: 	Re



%%%%%% TITOLO E IMPOSTAZONI
\beginsong{Venimus adorare eum}[by={Inno della XX GMG, Colonia 2005 — Linssen}]
	% <<< MODIFICA TITOLO E AUTORE
\transpose{0} 						% <<< TRASPOSIZIONE #TONI (0 nullo)
\momenti{Ringraziamento}							% <<< INSERISCI MOMENTI	
% momenti vanno separati da ; e vanno scelti tra:
% Ingresso; Atto penitenziale; Acclamazione al Vangelo; Dopo il Vangelo; Offertorio; Comunione; Ringraziamento; Fine; Santi; Pasqua; Avvento; Natale; Quaresima; Canti Mariani; Battesimo; Prima Comunione; Cresima; Matrimonio; Meditazione; Spezzare del pane;
\ifchorded
	%\textnote{Tonalità migliore }	% <<< EV COMMENTI (tonalità originale/migliore)
\fi


%%%%%% INTRODUZIONE
\ifchorded
\vspace*{\versesep}
\textnote{Intro: \qquad \qquad  }%(\eighthnote 116) % <<  MODIFICA IL TEMPO
% Metronomo: \eighthnote (ottavo) \quarternote (quarto) \halfnote (due quarti)
\vspace*{-\versesep}
\beginverse*

\nolyrics

%---- Prima riga -----------------------------
\vspace*{-\versesep}
\[D] \[G] \[A4] \[A]% \[*D] per indicare le pennate, \rep{2} le ripetizioni

%---- Ogni riga successiva -------------------
%\vspace*{-\versesep}
%\[G] \[C]  \[D]	

%---- Ev Indicazioni -------------------------			
%\textnote{\textit{(Oppure tutta la strofa)} }	

\endverse
\fi




\beginverse
\[D]Chiedi perché par\[G]tire \[E-]dal proprio \[A]regno
\[B-]solo per inse\[G]guire \[E-]una stella e per\[A]ché
\[G]per un Bimbo pi\[E7]egano quelle gi\[D]nocchia da \[B-]Re?
\[E-] Tu la risposta  \[A]sai che è:
\endverse

\beginchorus
Ve\[D]nimus adorare Eum, Em\[C]manuel \[G]Dio con noi
\[D]Venimus adorare \[C]Eum, Em\[A]manuel 
Ve\[D]nimus adorare Eum, Em\[C]manuel \[G]Dio con noi
\[D]Venimus adorare \[C]Eum, Em\[G]manuel 
\endchorus

\chordsoff
\beginverse
Chiedi perché lasciare sui monti il gregge
solo per ascoltare un canto e perché
per un Bimbo piegano quelle ginocchia, perché?
Tu la risposta  sai che è:
\endverse

\beginverse
Ecco da lontano per adorarlo \brk siamo giunti anche noi,
noi, tutti figli suoi, profeti e sacerdoti ormai.
Nel pane e nel vino noi siamo in lui e lui è in noi:
e un canto qui si alza già.
\endverse



\endsong
%------------------------------------------------------------
%			FINE CANZONE
%------------------------------------------------------------






