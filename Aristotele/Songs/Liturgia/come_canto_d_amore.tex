%-------------------------------------------------------------
%			INIZIO	CANZONE
%-------------------------------------------------------------


%titolo:    Come canto d'amore
%autore: 	P. Sequeri
%tonalita: 	Re 



%%%%%% TITOLO E IMPOSTAZONI
\beginsong{Come canto d'amore}[by={P. Sequeri}] 	% <<< MODIFICA TITOLO E AUTORE
\transpose{0} 						% <<< TRASPOSIZIONE #TONI (0 nullo)
\momenti{Ingresso; Cresima; Ringraziamento; Fine}							% <<< INSERISCI MOMENTI	
% momenti vanno separati da ; e vanno scelti tra:
% Ingresso; Atto penitenziale; Acclamazione al Vangelo; Dopo il Vangelo; Offertorio; Comunione; Ringraziamento; Fine; Santi; Pasqua; Avvento; Natale; Quaresima; Canti Mariani; Battesimo; Prima Comunione; Cresima; Matrimonio; Meditazione; Spezzare del pane;
\ifchorded
	%\textnote{Tonalità originale }	% <<< EV COMMENTI (tonalità originale/migliore)
\fi


%%%%%% INTRODUZIONE
\ifchorded
\vspace*{\versesep}
\textnote{Intro: \qquad \qquad  }%(\eighthnote 116) % <<  MODIFICA IL TEMPO
% Metronomo: \eighthnote (ottavo) \quarternote (quarto) \halfnote (due quarti)
\vspace*{-\versesep}
\beginverse*

\nolyrics

%---- Prima riga -----------------------------
\vspace*{-\versesep}
 \[D]  \[E-] \[D] \[E-]	 % \[*D] per indicare le pennate, \rep{2} le ripetizioni

%---- Ogni riga successiva -------------------
%\vspace*{-\versesep}
%\[G] \[C]  \[D]	

%---- Ev Indicazioni -------------------------			
%\textnote{\textit{(Oppure tutta la strofa)} }	

\endverse
\fi




%%%%% STROFA
\beginverse		%Oppure \beginverse* se non si vuole il numero di fianco
\memorize 		% <<< DECOMMENTA se si vuole utilizzarne la funzione
%\chordsoff		% <<< DECOMMENTA se vuoi una strofa senza accordi

\[D]Con il mio canto, \[E-]dolce Signore, 
\[F#-]voglio danzare \[G]questa mia \[D]gioia,
\[G]voglio destare \[E-]tutte le cose,
\[E7]un mondo nuovo \[A]voglio cantare.

\endverse

%%%%% STROFA
\beginverse*		%Oppure \beginverse* se non si vuole il numero di fianco
%\memorize 		% <<< DECOMMENTA se si vuole utilizzarne la funzione
%\chordsoff		% <<< DECOMMENTA se vuoi una strofa senza accordi

^Con il mio canto, ^dolce Signore, 
^voglio riempire ^lunghi si^lenzi,
^voglio abitare s^guardi di pace,
^il tuo perdono ^voglio cantare.

\endverse





%%%%% RITORNELLO
\beginchorus
\textnote{\textbf{Rit.}}

\[D]Tu \[E-]sei \[(E-)]per \[F#-]me, 
\[G]co\[D]me un \[G]can\[E-]to \[(E-)]d'a\[E7]mo\[A]re.
\[D]Res\[E-]ta \[(E-)]con \[F#-]noi, 
\[G]fi\[D]no al \[G]nuo\[E-]vo \[(E-)]mat\[E7]ti\[A]no.

\endchorus



%%%%% STROFA
\beginverse		%Oppure \beginverse* se non si vuole il numero di fianco
%\memorize 		% <<< DECOMMENTA se si vuole utilizzarne la funzione
%\chordsoff		% <<< DECOMMENTA se vuoi una strofa senza accordi

^Con il mio canto, ^dolce Signore, 
^voglio plasmare ^gesti d'a^more,
^voglio arrivare ^oltre la morte,
^la tua speranza ^voglio cantare.

\endverse

%%%%% STROFA
\beginverse*		%Oppure \beginverse* se non si vuole il numero di fianco
%\memorize 		% <<< DECOMMENTA se si vuole utilizzarne la funzione
%\chordsoff		% <<< DECOMMENTA se vuoi una strofa senza accordi

^Con il mio canto, ^dolce Signore, 
^voglio gettare ^semi di ^luce,
^voglio sognare ^cose mai viste,
^la tua bellezza ^voglio cantare.

\endverse



%%%%% RITORNELLO
\beginchorus
\textnote{\textbf{Rit.}}

\[D]Tu \[E-]sei \[(E-)]per \[F#-]me, 
\[G]co\[D]me un \[G]can\[E-]to  \[(E-)]d'a\[E7]mo\[A]re.
\[D]Res\[E-]ta \[(E-)]con \[F#-]noi, 
\[G]fi\[D]no al \[G]nuo\[E-]vo \[(E-)]mat\[E7]ti\[A]no.

\endchorus

%%%%% STROFA
\beginverse*		%Oppure \beginverse* se non si vuole il numero di fianco
%\memorize 		% <<< DECOMMENTA se si vuole utilizzarne la funzione
%\chordsoff		% <<< DECOMMENTA se vuoi una strofa senza accordi
\textnote{\textit{Cambio di tempo: molto più lento}}
\[D*] Se tu mi ascolti, \[E-*] dolce Signore,
\[F#-*] questo mio canto \[G*] sarà una \[D*]vita,
\[G*] e sarà bello \[E-*]  vivere insieme,
\[E7*] finché la vita un \[A*]canto sarà.

\endverse


%%%%% STROFA
\beginverse*		%Oppure \beginverse* se non si vuole il numero di fianco
%\memorize 		% <<< DECOMMENTA se si vuole utilizzarne la funzione
%\chordsoff		% <<< DECOMMENTA se vuoi una strofa senza accordi
\vspace*{\versesep}
\textnote{\textit{Dall'inizio con il tempo più veloce}}
\vspace*{-\versesep}
Con il mio canto...
\endverse


\endsong
%------------------------------------------------------------
%			FINE CANZONE
%------------------------------------------------------------


