%-------------------------------------------------------------
%			INIZIO	CANZONE
%-------------------------------------------------------------


%titolo: 	Santo Ricci
%autore: 	Daniele Ricci
%tonalita: 	Sol 



%%%%%% TITOLO E IMPOSTAZONI
\beginsong{Canto per te Gesù}[by={G. Boretti, J. Akepsimas, A. Fant, M. Deflorian}] 	% <<< MODIFICA TITOLO E AUTORE
\transpose{0} 						% <<< TRASPOSIZIONE #TONI (0 nullo)
%\preferflats  %SE VOGLIO FORZARE i bemolle come alterazioni
%\prefersharps %SE VOGLIO FORZARE i # come alterazioni
\momenti{Ingresso; Comunione}							% <<< INSERISCI MOMENTI	
% momenti vanno separati da ; e vanno scelti tra:
% Ingresso; Atto penitenziale; Acclamazione al Vangelo; Dopo il Vangelo; Offertorio; Comunione; Ringraziamento; Fine; Santi; Pasqua; Avvento; Natale; Quaresima; Canti Mariani; Battesimo; Prima Comunione; Cresima; Matrimonio; Meditazione; Spezzare del pane;
\ifchorded
	%\textnote{Tonalità migliore }	% <<< EV COMMENTI (tonalità originale/migliore)
\fi




%%%%%% INTRODUZIONE
\ifchorded
\vspace*{\versesep}
\musicnote{
\begin{minipage}{0.48\textwidth}
\textbf{Intro}
\hfill 
%( \eighthnote \, 80)   % <<  MODIFICA IL TEMPO
% Metronomo: \eighthnote (ottavo) \quarternote (quarto) \halfnote (due quarti)
\end{minipage}
} 	
\vspace*{-\versesep}
\beginverse*


\nolyrics

%---- Prima riga -----------------------------
\vspace*{-\versesep}
\[F] \[C] \[F]	 % \[*D] per indicare le pennate, \rep{2} le ripetizioni

%---- Ogni riga successiva -------------------
%\vspace*{-\versesep}
%\[G] \[C]  \[D]	

%---- Ev Indicazioni -------------------------			
%\textnote{\textit{(Oppure tutta la strofa)} }	

\endverse
\fi


%%%%% RITORNELLO
\beginchorus
\textnote{\textbf{Rit.}}

\[F]Canto per \[D-]Te, Gesù,
\[B&]canto per \[F]Te!
\[G-]Oggi mi \[F]rendi fe\[G]li\[C]ce.
\[F]Canto per \[D-]Te, Gesù,
\[B&]canto per \[F]Te.
\[G-]Vieni e \[F*]vivi \[C*]con \[F]me.

\endchorus




%%%%% STROFA
\beginverse		%Oppure \beginverse* se non si vuole il numero di fianco
\memorize 		% <<< DECOMMENTA se si vuole utilizzarne la funzione
%\chordsoff		% <<< DECOMMENTA se vuoi una strofa senza accordi

Hai \[F]scelto la mia \[D-]casa
per \[B&]abitar con \[C]me:
è \[G-]piccola, Si\[D-]gnore,
ma \[B&]io vivrò con \[C]Te.

\endverse







%%%%% STROFA
\beginverse		%Oppure \beginverse* se non si vuole il numero di fianco
%\memorize 		% <<< DECOMMENTA se si vuole utilizzarne la funzione
%\chordsoff		% <<< DECOMMENTA se vuoi una strofa senza accordi


Nel ^Pane che man^giamo,
Ge^sù, Tu vieni a ^noi:
e un ^giorno su nel ^cielo
sa^rem fratelli ^tuoi.

\endverse

%%%%% STROFA
\beginverse		%Oppure \beginverse* se non si vuole il numero di fianco
%\memorize 		% <<< DECOMMENTA se si vuole utilizzarne la funzione
\chordsoff		% <<< DECOMMENTA se vuoi una strofa senza accordi


Mi parli con amore
di grandi verità:
ti ascolterò, Signore,
con gioia e fedeltà.


\endverse


%%%%% STROFA
\beginverse		%Oppure \beginverse* se non si vuole il numero di fianco
%\memorize 		% <<< DECOMMENTA se si vuole utilizzarne la funzione
\chordsoff		% <<< DECOMMENTA se vuoi una strofa senza accordi



Gesù, Tu sei risorto
e vivi in mezzo a noi:
fedele nostro amico,
ci rendi amici tuoi.


\endverse

%%%%% STROFA
\beginverse		%Oppure \beginverse* se non si vuole il numero di fianco
%\memorize 		% <<< DECOMMENTA se si vuole utilizzarne la funzione
%\chordsoff		% <<< DECOMMENTA se vuoi una strofa senza accordi


I ^bimbi che abbrac^ciavi
ti a^mavano, Ge^sù,
a ^Te noi promett^iamo
di a^marti sempre ^più.

\endverse


%%%%% STROFA
\beginverse		%Oppure \beginverse* se non si vuole il numero di fianco
%\memorize 		% <<< DECOMMENTA se si vuole utilizzarne la funzione
\chordsoff		% <<< DECOMMENTA se vuoi una strofa senza accordi


Passavi tra la gente,
chiedevi la bontà:
con Te l'amore vero
più facile sarà.

\endverse


%%%%% STROFA
\beginverse		%Oppure \beginverse* se non si vuole il numero di fianco
%\memorize 		% <<< DECOMMENTA se si vuole utilizzarne la funzione
\chordsoff		% <<< DECOMMENTA se vuoi una strofa senza accordi


La strada spesso è buia:
il Sole chi sarà?
Amici non temete:
Gesù vi guiderà.

\endverse


%%%%% STROFA
\beginverse		%Oppure \beginverse* se non si vuole il numero di fianco
%\memorize 		% <<< DECOMMENTA se si vuole utilizzarne la funzione
\chordsoff		% <<< DECOMMENTA se vuoi una strofa senza accordi



Quest'oggi ho tanti amici
che pregano con me:
ma il mondo è ancor più grande
io l'amerò con Te!

\endverse


%%%%% STROFA
\beginverse		%Oppure \beginverse* se non si vuole il numero di fianco
%\memorize 		% <<< DECOMMENTA se si vuole utilizzarne la funzione
\chordsoff		% <<< DECOMMENTA se vuoi una strofa senza accordi


La gioia del mio cuore
a tutti canterò
Perché Gesù mi ha detto:
"Per sempre ti amerò!"

\endverse

%%%%% STROFA
\beginverse		%Oppure \beginverse* se non si vuole il numero di fianco
%\memorize 		% <<< DECOMMENTA se si vuole utilizzarne la funzione
%\chordsoff		% <<< DECOMMENTA se vuoi una strofa senza accordi

Lo ^Spirito che ^doni
la ^terra cambie^rà:
Si^gnore, noi spe^riamo
nel ^Regno che ver^rà.

\endverse


\endsong
%------------------------------------------------------------
%			FINE CANZONE
%------------------------------------------------------------

