%-------------------------------------------------------------
%			INIZIO	CANZONE
%-------------------------------------------------------------


%titolo: 	Spirito Santo, discendi tra noi
%autore: 	
%tonalita: 	Do 



%%%%%% TITOLO E IMPOSTAZONI
\beginsong{Spirito Santo, discendi tra noi}[by={}] 	% <<< MODIFICA TITOLO E AUTORE
\transpose{0} 						% <<< TRASPOSIZIONE #TONI (0 nullo)
%\preferflats  %SE VOGLIO FORZARE i bemolle come alterazioni
%\prefersharps %SE VOGLIO FORZARE i # come alterazioni
\momenti{Cresima}							% <<< INSERISCI MOMENTI	
% momenti vanno separati da ; e vanno scelti tra:
% Ingresso; Atto penitenziale; Acclamazione al Vangelo; Dopo il Vangelo; Offertorio; Comunione; Ringraziamento; Fine; Santi; Pasqua; Avvento; Natale; Quaresima; Canti Mariani; Battesimo; Prima Comunione; Cresima; Matrimonio; Meditazione; Spezzare del pane;
\ifchorded
	%\textnote{Tonalità migliore }	% <<< EV COMMENTI (tonalità originale/migliore)
\fi


%%%%%% INTRODUZIONE
\ifchorded
\vspace*{\versesep}
\textnote{Intro: \qquad \qquad  }%(\eighthnote 116) % <<  MODIFICA IL TEMPO
% Metronomo: \eighthnote (ottavo) \quarternote (quarto) \halfnote (due quarti)
\vspace*{-\versesep}
\beginverse*

\nolyrics

%---- Prima riga -----------------------------
\vspace*{-\versesep}
\[C] \[A-] \[F] \[C]	 % \[*D] per indicare le pennate, \rep{2} le ripetizioni

%---- Ogni riga successiva -------------------
%\vspace*{-\versesep}
%\[G] \[C]  \[D]	

%---- Ev Indicazioni -------------------------			
%\textnote{\textit{(Oppure tutta la strofa)} }	

\endverse
\fi








%%%%% RITORNELLO
\beginchorus
\textnote{\textbf{Rit.}}

S\[C]pirito \[A-]Santo, dis\[F]cendi tra \[G]noi:
\[C]la nostra \[F]fede ha bi\[G]sogno di \[E]te;
\[A-]al nostro c\[F]uore in\[A-]segna ad a\[G7]mare
\[C]e la spe\[F]ranza non \[D-*]toglier\[G7*]ci \[C]mai!


\endchorus



%%%%% STROFA
\beginverse		%Oppure \beginverse* se non si vuole il numero di fianco
\memorize 		% <<< DECOMMENTA se si vuole utilizzarne la funzione
%\chordsoff		% <<< DECOMMENTA se vuoi una strofa senza accordi

\[C]Tu sei il \[A-]dono pro\[D-]messo dal \[G7]Padre;
sei \[C]fuoco di a\[A-]more, sor\[D-]gente di \[G7]vita.


\endverse

%%%%% STROFA
\beginverse		%Oppure \beginverse* se non si vuole il numero di fianco
%\memorize 		% <<< DECOMMENTA se si vuole utilizzarne la funzione
\chordsoff		% <<< DECOMMENTA se vuoi una strofa senza accordi

Tu vivi con noi e sei nostra forza:
sostienici sempre nel nostro cammino.


\endverse



%%%%% STROFA
\beginverse		%Oppure \beginverse* se non si vuole il numero di fianco
%\memorize 		% <<< DECOMMENTA se si vuole utilizzarne la funzione
\chordsoff		% <<< DECOMMENTA se vuoi una strofa senza accordi

Tu sei sapienza che vince ogni errore;
di te ci fidiamo e avremo la luce.

\endverse

\endsong
%------------------------------------------------------------
%			FINE CANZONE
%------------------------------------------------------------


