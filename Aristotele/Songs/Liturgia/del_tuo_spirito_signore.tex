%-------------------------------------------------------------
%			INIZIO	CANZONE
%-------------------------------------------------------------


%titolo: 	Del tuo Spirito, Signore
%autore: 	Gen Verde, Gen Rosso
%tonalita: 	Re



%%%%%% TITOLO E IMPOSTAZONI
\beginsong{Del tuo Spirito, Signore}[by={Gen\ Verde, Gen\ Rosso}]	% <<< MODIFICA TITOLO E AUTORE
\transpose{0} 						% <<< TRASPOSIZIONE #TONI (0 nullo)
\momenti{Cresima; Salmi}							% <<< INSERISCI MOMENTI	
% momenti vanno separati da ; e vanno scelti tra:
% Ingresso; Atto penitenziale; Salmi; Acclamazione al Vangelo; Dopo il Vangelo; Offertorio; Comunione; Ringraziamento; Fine; Santi; Pasqua; Avvento; Natale; Quaresima; Canti Mariani; Battesimo; Prima Comunione; Cresima; Matrimonio; Meditazione; Spezzare del pane;
\ifchorded
	%\textnote{Tonalità migliore }	% <<< EV COMMENTI (tonalità originale/migliore)
\fi


%%%%%% INTRODUZIONE
\ifchorded
\vspace*{\versesep}
\musicnote{
\begin{minipage}{0.48\textwidth}
\textbf{Intro}
\hfill 
%( \eighthnote \, 80)   % <<  MODIFICA IL TEMPO
% Metronomo: \eighthnote (ottavo) \quarternote (quarto) \halfnote (due quarti)
\end{minipage}
} 	
\vspace*{-\versesep}
\beginverse*


\nolyrics

%---- Prima riga -----------------------------
\vspace*{-\versesep}
\[A*] \[B-] \[F#-*]  \[G]	 % \[*D] per indicare le pennate, \rep{2} le ripetizioni

%---- Ogni riga successiva -------------------
\vspace*{-\versesep}
\[D] \[E-*]  \[D]	

%---- Ev Indicazioni -------------------------			
\textnote{\textit{[come la seconda parte del ritornello]} }	

\endverse
\fi






\beginchorus
|\[D*] Del tuo |\[G]Spiri\[D*]to, Si|\[G*]gno\[D]re,
\[A*]è |\[B-]piena la \[F#-*]ter|\[G]ra, è |\[D]piena la \[E-*]ter|\[D]ra. 
\endchorus



\musicnote{\textit{[dolce, arpeggiato]}}

\beginverse
\memorize
|\[C]Benedici il Si|\[B&]gnore, \mbar{2}{4}\[D-]anima \mbar{4}{4}\[A-*]mi\[B&]a,
Si\[C*]gnore, \mbar{3}{4}\[F*]Dio,\[C] tu sei |\[G*]gran\[C]de!
|\[C]Sono immense, splen|\[B&]denti
\mbar{2}{4}\[D-]tutte le tue \mbar{3}{4}\[B&*]ope\[F]re e |\[G-]tutte le crea\[A*]tu\mbar{4}{4}\[D]re.
\endverse



\beginverse
%\chordsoff
^Se tu togli il tuo ^soffio ^muore ogni ^co^sa
e ^si dis^sol^ve nella ^ter^ra.
^Il tuo spirito ^scende:
^tutto si ri^cre^a e ^tutto si rin^no^va.
\endverse


\beginverse
%\chordsoff
^La tua gloria, Si^gnore, ^resti per ^sem^pre.
Gio^isci, ^Di^o, del cre^a^to.
^Questo semplice ^canto
^salga a te Si^gno^re, sei ^tu la nostra ^gio^ia.
\endverse


\endsong

