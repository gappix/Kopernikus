%-------------------------------------------------------------
%			INIZIO	CANZONE
%-------------------------------------------------------------


%titolo: 	Come l'aurora verrai
%autore: 	Gen Verde
%tonalita: 	Re-



%%%%%% TITOLO E IMPOSTAZONI
\beginsong{Come l'aurora verrai}[by={Gen\ Verde}]	% <<< MODIFICA TITOLO E AUTORE
\transpose{0} 						% <<< TRASPOSIZIONE #TONI (0 nullo)
\momenti{Avvento}							% <<< INSERISCI MOMENTI	
% momenti vanno separati da ; e vanno scelti tra:
% Ingresso; Atto penitenziale; Acclamazione al Vangelo; Dopo il Vangelo; Offertorio; Comunione; Ringraziamento; Fine; Santi; Pasqua; Avvento; Natale; Quaresima; Canti Mariani; Battesimo; Prima Comunione; Cresima; Matrimonio; Meditazione; Spezzare del pane;
\ifchorded
	%\textnote{Tonalità migliore }	% <<< EV COMMENTI (tonalità originale/migliore)
\fi


%%%%%% INTRODUZIONE
\ifchorded
\vspace*{\versesep}
\musicnote{
\begin{minipage}{0.48\textwidth}
\textbf{Intro}
\hfill 
%( \eighthnote \, 80)   % <<  MODIFICA IL TEMPO
% Metronomo: \eighthnote (ottavo) \quarternote (quarto) \halfnote (due quarti)
\end{minipage}
} 	
\vspace*{-\versesep}
\beginverse*

\nolyrics

%---- Prima riga -----------------------------
\vspace*{-\versesep}
\[D-] \[A-] \[D-]	 % \[*D] per indicare le pennate, \rep{2} le ripetizioni

%---- Ogni riga successiva -------------------
%\vspace*{-\versesep}
%\[G] \[C]  \[D]	

%---- Ev Indicazioni -------------------------			
%\textnote{\textit{(Oppure tutta la strofa)} }	

\endverse
\fi





%%%%% STROFA
\beginverse		%Oppure \beginverse* se non si vuole il numero di fianco
\memorize 		% <<< DECOMMENTA se si vuole utilizzarne la funzione
%\chordsoff		% <<< DECOMMENTA se vuoi una strofa senza accordi
\[D-] Come l'au\[A-7]rora ver\[D-]rai
le tenebre in \[A-]luce cambie\[B&]rai
tu per \[F]noi, Si\[B&]gnore. \[C]
\[D-] Come la pi\[A-7]oggia ca\[D-]drai
sui nostri de\[A-]serti scende\[B&]rai
scorre\[F]{rà l'a}\[B&]{mo}re\[C]
\endverse




%%%%% RITORNELLO
\beginchorus
\textnote{\textbf{Rit.}}
\[B&] Tutti i nostri senti\[C]eri \brk percorrer\[D-*]{ai,} \[A-*] \[D-] 
\[B&] tutti i figli dis\[C]persi \brk raccoglier\[D-*]{ai,} \[A-*] \[D-] 
\[B&] chiamerai da ogni \[C]terra
il tuo \[A]popo\[B&]lo,
\[G-] in eterno ti a\[G-7]vremo con \[C]{noi.} \[C] 
\endchorus




%%%%% STROFA
\beginverse		%Oppure \beginverse* se non si vuole il numero di fianco
%\memorize 		% <<< DECOMMENTA se si vuole utilizzarne la funzione
\chordsoff
Re di giustizia sarai,
le spade in aratri forgerai:
ci darai la pace.
Lupo ed agnello vedrai
insieme sui prati dove mai
tornerà la notte.
\endverse




%%%%% STROFA
\beginverse		%Oppure \beginverse* se non si vuole il numero di fianco
%\memorize 		% <<< DECOMMENTA se si vuole utilizzarne la funzione
\chordsoff
Dio di salvezza tu sei
e come una stella sorgerai
su di noi per sempre.
E chi non vede, vedrà,
chi ha chiusi gli orecchi sentirà,
canterà di gioia.
\endverse




\endsong
%------------------------------------------------------------
%			FINE CANZONE
%------------------------------------------------------------