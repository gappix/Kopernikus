%-------------------------------------------------------------
%			INIZIO	CANZONE
%-------------------------------------------------------------


%titolo: 	E la gioia
%autore: 	
%tonalita: 	Do



%%%%%% TITOLO E IMPOSTAZONI
\beginsong{E la gioia}[by={Autore sconosciuto}] 	% <<< MODIFICA TITOLO E AUTORE
\transpose{0} 						% <<< TRASPOSIZIONE #TONI (0 nullo)
%\preferflats  %SE VOGLIO FORZARE i bemolle come alterazioni
%\prefersharps %SE VOGLIO FORZARE i # come alterazioni
\momenti{Matrimonio; Congedo; Comunione; Ringraziamento}				 			% <<< INSERISCI MOMENTI	
% momenti vanno separati da ; e vanno scelti tra:
% Ingresso; Atto penitenziale; Acclamazione al Vangelo; 
% Dopo il Vangelo; Offertorio; Comunione; Ringraziamento; 
% Fine; Santi; Pasqua; Avvento; Natale; Quaresima; Canti Mariani; 
% Battesimo; Prima Comunione; Cresima; Matrimonio; Meditazione; Spezzare del pane;
\ifchorded
	%\textnote{Tonalità migliore }	% <<< EV COMMENTI (tonalità originale/migliore)
\fi

%%%%%% INTRODUZIONE
\ifchorded
\vspace*{\versesep}
\musicnote{
\begin{minipage}{0.48\textwidth}
\textbf{Intro}
\hfill 
%( \eighthnote \, 80)   % <<  MODIFICA IL TEMPO
% Metronomo: \eighthnote (ottavo) \quarternote (quarto) \halfnote (due quarti)
\end{minipage}
} 	
\vspace*{-\versesep}
\beginverse*


\nolyrics

%---- Prima riga -----------------------------
\vspace*{-\versesep}
\[C] \[G] \[D] 	\[E-] % \[*D] per indicare le pennate, \rep{2} le ripetizioni

%---- Ogni riga successiva -------------------
\vspace*{-\versesep}
\[C] \[G] \[D]   \[D]	

%---- Ev Indicazioni -------------------------			
%\textnote{\textit{(Oppure tutta la strofa)} }	

\endverse
\fi




%%%%% STROFA
\beginverse		%Oppure \beginverse* se non si vuole il numero di fianco
\memorize 		% <<< DECOMMENTA se si vuole utilizzarne la funzione
%\chordsoff		% <<< DECOMMENTA se vuoi una strofa senza accordi


\[C] Voglio cantare con \[G]voi \brk la  \[D]gioia di essere \[E-]qui
\[C] per dare una ris\[G]posta \brk a \[D]chi ci ha chia\[E-]mati.
\[C] E insieme ci ha vo\[G]luti \brk per \[D]dare spe\[E-]ranza
a \[C]tutti quelli \[C]che sono intorno a \[D]noi.

\endverse


%%%%% RITORNELLO
\beginchorus
\textnote{\textbf{Rit.}}

E la \[C]gioia entra \[E-]dentro 
e scom\[D]bussola o\[E-]gni cosa 
\[C]perché va \[G]dritta al \[D]cuor. \[D]
Vuole \[C]esser di \[E-]tutti 
\[D]non te\[E-]niamola per \[C]noi
ma but\[G]tiamola \[D]fuori! \[D]

\endchorus


%%%%% STROFA
\beginverse		%Oppure \beginverse* se non si vuole il numero di fianco
%\memorize 		% <<< DECOMMENTA se si vuole utilizzarne la funzione
%\chordsoff		% <<< DECOMMENTA se vuoi una strofa senza accordi

^ Hai lasciato la ^tua casa, le ^tue comodi^tà,
^ rischiando un po-o’ di ^te, le ^tue sicu^rezze.
^ Per cerca-are qualche ^cosa
che ^dia fuoco al ^cuore,
e tro^vare così il ^senso della ^vita.

\endverse



%%%%% RITORNELLO
\beginchorus
\textnote{\textbf{Rit.}}

E la \[C]gioia entra \[E-]dentro 
e scom\[D]bussola o\[E-]gni cosa 
\[C]perché va \[G]dritta al \[D]cuor. \[D]
Vuole \[C]esser di \[E-]tutti 
\[D]non te\[E-]niamola per \[C]noi
ma but\[G]tiamola \[D]fuori! \[D]

\endchorus




%%%%% STROFA
\beginverse		%Oppure \beginverse* se non si vuole il numero di fianco
%\memorize 		% <<< DECOMMENTA se si vuole utilizzarne la funzione
%\chordsoff		% <<< DECOMMENTA se vuoi una strofa senza accordi

^ Non opporre resis^tenza \brk alla ^voce che ^chiama,
^ lasciati invece an^dare, \brk ^senza te^mere,
^ vedrai che arrive^rai \brk a pro^vare cos’è l’a^more
che ^prende ogni ^cosa e poi tutto ^dà.

\endverse

%%%%% RITORNELLO
\beginchorus
\textnote{\textbf{Rit.}}

E la \[C]gioia entra \[E-]dentro 
e scom\[D]bussola o\[E-]gni cosa 
\[C]perché va \[G]dritta al \[D]cuor. \[D]
Vuole \[C]esser di \[E-]tutti 
\[D]non te\[E-]niamola per \[C]noi
ma but\[G]tiamola \[D]fuori! \[D]  \rep{3}

\endchorus




\endsong
%------------------------------------------------------------
%			FINE CANZONE
%------------------------------------------------------------

