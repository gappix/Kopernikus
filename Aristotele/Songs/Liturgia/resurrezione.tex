%-------------------------------------------------------------
%			INIZIO	CANZONE
%-------------------------------------------------------------


%titolo: 	Resurrezione
%autore: 	Gen Rosso
%tonalita: 	Re e Do 



%%%%%% TITOLO E IMPOSTAZONI
\beginsong{Resurrezione}[by={Gen Rosso}] 	% <<< MODIFICA TITOLO E AUTORE
\transpose{-2} 						% <<< TRASPOSIZIONE #TONI (0 nullo)
\momenti{Pasqua}							% <<< INSERISCI MOMENTI	
% momenti vanno separati da ; e vanno scelti tra:
% Ingresso; Atto penitenziale; Acclamazione al Vangelo; Dopo il Vangelo; Offertorio; Comunione; Ringraziamento; Fine; Santi; Pasqua; Avvento; Natale; Quaresima; Canti Mariani; Battesimo; Prima Comunione; Cresima; Matrimonio; Meditazione;
\ifchorded
	\textnote{Tonalità migliore per le bambine }	% <<< EV COMMENTI (tonalità originale/migliore)
\fi


%%%%%% INTRODUZIONE
\ifchorded
\vspace*{\versesep}
\textnote{Intro: \qquad \qquad  }%(\eighthnote 116) % <<  MODIFICA IL TEMPO
% Metronomo: \eighthnote (ottavo) \quarternote (quarto) \halfnote (due quarti)
\vspace*{-\versesep}
\beginverse*

\nolyrics

%---- Prima riga -----------------------------
\vspace*{-\versesep}
\[D] \[G] \[D]	 \[G] % \[*D] per indicare le pennate, \rep{2} le ripetizioni

%---- Ogni riga successiva -------------------
%\vspace*{-\versesep}
%\[G] \[C]  \[D]	

%---- Ev Indicazioni -------------------------			
%\textnote{\textit{(Oppure tutta la strofa)} }	

\endverse
\fi




%%%%% STROFA
\beginverse		%Oppure \beginverse* se non si vuole il numero di fianco
\memorize 		% <<< DECOMMENTA se si vuole utilizzarne la funzione
%\chordsoff		& <<< DECOMMENTA se vuoi una strofa senza accordi

Che \[D]gioia ci hai \[G]dato, Si\[D]gnore del \[G]cielo
Si\[D]gnore del \[G]grande uni\[A]verso!
Che \[D]gioia ci hai \[G]dato, ves\[D]tito di \[G]luce
ves\[D]tito di \[A]gloria infi\[B-]ni\[G]ta,
ves\[D]tito di \[A]gloria infi\[G]n\[*A]i\[D]ta!

\endverse




%%%%%% EV. INTERMEZZO
\beginverse*
\vspace*{1.3\versesep}
{
	\nolyrics
	\textnote{Intermezzo strumentale}
	
	\ifchorded

	%---- Prima riga -----------------------------
	\vspace*{-\versesep}
	 \[G]  \[D]	 \[G] 


	\fi
	%---- Ev Indicazioni -------------------------			
	%\textnote{\textit{(ripetizione della strofa)}} 
	 
}
\vspace*{\versesep}
\endverse


%%%%% STROFA
\beginverse		%Oppure \beginverse* se non si vuole il numero di fianco
%\memorize 		% <<< DECOMMENTA se si vuole utilizzarne la funzione
%\chordsoff		& <<< DECOMMENTA se vuoi una strofa senza accordi

Ve^derti ri^sorto, ve^derti Si^gnore,
il ^cuore sta ^per impaz^zire!
Tu ^sei ritor^nato, Tu ^sei qui tra ^noi
e a^desso Ti a^vremo per ^sem^pre,
e a^desso Ti a^vremo per ^se^m^pre.

\endverse




%%%%%% EV. INTERMEZZO
\beginverse*
\vspace*{1.3\versesep}
{
	\nolyrics
	\textnote{Intermezzo strumentale}
	
	\ifchorded

	%---- Prima riga -----------------------------
	\vspace*{-\versesep}
	\[G]  \[D]	 \[G] 


	\fi
	%---- Ev Indicazioni -------------------------			
	%\textnote{\textit{(ripetizione della strofa)}} 
	 
}
\vspace*{\versesep}
\endverse



%%%%% STROFA
\beginverse		%Oppure \beginverse* se non si vuole il numero di fianco
%\memorize 		% <<< DECOMMENTA se si vuole utilizzarne la funzione
%\chordsoff		& <<< DECOMMENTA se vuoi una strofa senza accordi

^ Chi cercate, ^donne, quag^giù,
chi cercate, ^donne, quag^giù?
Quello che era ^morto non è ^qui:
è ri^sorto, sì! come a^veva detto ^anche a voi,
^voi gridate a ^tutti che^
^è risorto ^Lui,
a ^tutti che^
^è risor^to ^Lui!

\endverse




%%%%%% EV. INTERMEZZO
\beginverse*
\vspace*{1.3\versesep}
{
	\nolyrics
	\textnote{Intermezzo strumentale}
	
	\ifchorded

	%---- Prima riga -----------------------------
	\vspace*{-\versesep}
	 \[G]  \[D]	 \[G] 


	\fi
	%---- Ev Indicazioni -------------------------			
	%\textnote{\textit{(ripetizione della strofa)}} 
	 
}
\vspace*{\versesep}
\endverse




%%%%% STROFA
\beginverse		%Oppure \beginverse* se non si vuole il numero di fianco
%\memorize 		% <<< DECOMMENTA se si vuole utilizzarne la funzione
%\chordsoff		& <<< DECOMMENTA se vuoi una strofa senza accordi

^ Tu hai vinto il ^mondo, Ge^sù,
Tu hai vinto il ^mondo, Ge^sù,
liberiamo ^la felici^tà!
E la ^morte, no, non e^siste più, l’hai ^vinta Tu
e ^hai salvato ^tutti noi, ^
^uomini con ^Te,
^tutti noi, ^
^uomini ^con ^Te.

\endverse



%%%%%% EV. FINALE

\beginverse*
\vspace*{1.3\versesep}
\textnote{Finale:} %<<< EV. INDICAZIONI

\[G]Uomini con te, uomini con te - e
\echo{\[D]Che gioia ci hai \[G]dato 
ti a\[D]vremo per \[G]sem\[D]pre - \[D]e!} \[*D]

\endverse




\endsong
%------------------------------------------------------------
%			FINE CANZONE
%------------------------------------------------------------

%++++++++++++++++++++++++++++++++++++++++++++++++++++++++++++
%			CANZONE TRASPOSTA
%++++++++++++++++++++++++++++++++++++++++++++++++++++++++++++
\ifchorded
%decremento contatore per avere stesso numero
\addtocounter{songnum}{-1} 
\beginsong{Resurrezione}[by={Gen Rosso}] 	% <<< COPIA TITOLO E AUTORE
\transpose{0} 						% <<< TRASPOSIZIONE #TONI + - (0 nullo)
%\preferflats  %SE VOGLIO FORZARE i bemolle come alterazioni
%\prefersharps %SE VOGLIO FORZARE i # come alterazioni
\ifchorded
	\textnote{Tonalità originale}	% <<< EV COMMENTI (tonalità originale/migliore)
\fi


%%%%%% INTRODUZIONE
\ifchorded
\vspace*{\versesep}
\textnote{Intro: \qquad \qquad  }%(\eighthnote 116) % <<  MODIFICA IL TEMPO
% Metronomo: \eighthnote (ottavo) \quarternote (quarto) \halfnote (due quarti)
\vspace*{-\versesep}
\beginverse*

\nolyrics

%---- Prima riga -----------------------------
\vspace*{-\versesep}
\[D] \[G] \[D]	 \[G] % \[*D] per indicare le pennate, \rep{2} le ripetizioni

%---- Ogni riga successiva -------------------
%\vspace*{-\versesep}
%\[G] \[C]  \[D]	

%---- Ev Indicazioni -------------------------			
%\textnote{\textit{(Oppure tutta la strofa)} }	

\endverse
\fi




%%%%% STROFA
\beginverse		%Oppure \beginverse* se non si vuole il numero di fianco
\memorize 		% <<< DECOMMENTA se si vuole utilizzarne la funzione
%\chordsoff		& <<< DECOMMENTA se vuoi una strofa senza accordi

Che \[D]gioia ci hai \[G]dato, Si\[D]gnore del \[G]cielo
Si\[D]gnore del \[G]grande uni\[A]verso!
Che \[D]gioia ci hai \[G]dato, ves\[D]tito di \[G]luce
ves\[D]tito di \[A]gloria infi\[B-]ni\[G]ta,
ves\[D]tito di \[A]gloria infi\[G]n\[*A]i\[D]ta!

\endverse




%%%%%% EV. INTERMEZZO
\beginverse*
\vspace*{1.3\versesep}
{
	\nolyrics
	\textnote{Intermezzo strumentale}
	
	\ifchorded

	%---- Prima riga -----------------------------
	\vspace*{-\versesep}
	 \[G]  \[D]	 \[G] 


	\fi
	%---- Ev Indicazioni -------------------------			
	%\textnote{\textit{(ripetizione della strofa)}} 
	 
}
\vspace*{\versesep}
\endverse


%%%%% STROFA
\beginverse		%Oppure \beginverse* se non si vuole il numero di fianco
%\memorize 		% <<< DECOMMENTA se si vuole utilizzarne la funzione
%\chordsoff		& <<< DECOMMENTA se vuoi una strofa senza accordi

Ve^derti ri^sorto, ve^derti Si^gnore,
il ^cuore sta ^per impaz^zire!
Tu ^sei ritor^nato, Tu ^sei qui tra ^noi
e a^desso Ti a^vremo per ^sem^pre,
e a^desso Ti a^vremo per ^se^m^pre.

\endverse




%%%%%% EV. INTERMEZZO
\beginverse*
\vspace*{1.3\versesep}
{
	\nolyrics
	\textnote{Intermezzo strumentale}
	
	\ifchorded

	%---- Prima riga -----------------------------
	\vspace*{-\versesep}
	\[G]  \[D]	 \[G] 


	\fi
	%---- Ev Indicazioni -------------------------			
	%\textnote{\textit{(ripetizione della strofa)}} 
	 
}
\vspace*{\versesep}
\endverse



%%%%% STROFA
\beginverse		%Oppure \beginverse* se non si vuole il numero di fianco
%\memorize 		% <<< DECOMMENTA se si vuole utilizzarne la funzione
%\chordsoff		& <<< DECOMMENTA se vuoi una strofa senza accordi

^ Chi cercate, ^donne, quag^giù,
chi cercate, ^donne, quag^giù?
Quello che era ^morto non è ^qui:
è ri^sorto, sì! come a^veva detto ^anche a voi,
^voi gridate a ^tutti che^
^è risorto ^Lui,
a ^tutti che^
^è risor^to ^Lui!

\endverse




%%%%%% EV. INTERMEZZO
\beginverse*
\vspace*{1.3\versesep}
{
	\nolyrics
	\textnote{Intermezzo strumentale}
	
	\ifchorded

	%---- Prima riga -----------------------------
	\vspace*{-\versesep}
	 \[G]  \[D]	 \[G] 


	\fi
	%---- Ev Indicazioni -------------------------			
	%\textnote{\textit{(ripetizione della strofa)}} 
	 
}
\vspace*{\versesep}
\endverse




%%%%% STROFA
\beginverse		%Oppure \beginverse* se non si vuole il numero di fianco
%\memorize 		% <<< DECOMMENTA se si vuole utilizzarne la funzione
%\chordsoff		& <<< DECOMMENTA se vuoi una strofa senza accordi

^ Tu hai vinto il ^mondo, Ge^sù,
Tu hai vinto il ^mondo, Ge^sù,
liberiamo ^la felici^tà!
E la ^morte, no, non e^siste più, l’hai ^vinta Tu
e ^hai salvato ^tutti noi, ^
^uomini con ^Te,
^tutti noi, ^
^uomini ^con ^Te.

\endverse



%%%%%% EV. FINALE

\beginverse*
\vspace*{1.3\versesep}
\textnote{Finale:} %<<< EV. INDICAZIONI

\[G]Uomini con te, uomini con te - e
\echo{\[D]Che gioia ci hai \[G]dato 
ti a\[D]vremo per \[G]sem\[D]pre - \[D]e!} \[*D]

\endverse




\endsong

\fi
%++++++++++++++++++++++++++++++++++++++++++++++++++++++++++++
%			FINE CANZONE TRASPOSTA
%++++++++++++++++++++++++++++++++++++++++++++++++++++++++++++

