%-------------------------------------------------------------
%			INIZIO	CANZONE
%-------------------------------------------------------------


%titolo: 	Dove vita è davvero
%autore: 	E. Sarini
%tonalita: 	Sol 



%%%%%% TITOLO E IMPOSTAZONI
\beginsong{Dove vita è davvero}[by={E. Sarini}] 
\transpose{0} 						% <<< TRASPOSIZIONE #TONI (0 nullo)
\momenti{Comunione; Congedo}							% <<< INSERISCI MOMENTI	
% momenti vanno separati da ; e vanno scelti tra:
% Ingresso; Atto penitenziale; Acclamazione al Vangelo; Dopo il Vangelo; Offertorio; Comunione; Ringraziamento; Fine; Santi; Pasqua; Avvento; Natale; Quaresima; Canti Mariani; Battesimo; Prima Comunione; Cresima; Matrimonio; Meditazione; Spezzare del pane;
\ifchorded
	%\textnote{Tonalità originale }	% <<< EV COMMENTI (tonalità originale/migliore)
\fi

%%%%%% INTRODUZIONE
\ifchorded
\vspace*{\versesep}
\musicnote{
\begin{minipage}{0.48\textwidth}
\textbf{Intro}
\hfill 
%( \eighthnote \, 80)   % <<  MODIFICA IL TEMPO
% Metronomo: \eighthnote (ottavo) \quarternote (quarto) \halfnote (due quarti)
\end{minipage}
} 	
\vspace*{-\versesep}
\beginverse*


\nolyrics

%---- Prima riga -----------------------------
\vspace*{-\versesep}
\[C] \[G] \[C]	 % \[*D] per indicare le pennate, \rep{2} le ripetizioni

%---- Ogni riga successiva -------------------
%\vspace*{-\versesep}
%\[G] \[C]  \[D]	

%---- Ev Indicazioni -------------------------			
\textnote{\textit{(oppure la prima riga)} }	

\endverse
\fi




%%%%% STROFA
\beginverse		%Oppure \beginverse* se non si vuole il numero di fianco
\memorize 		% <<< DECOMMENTA se si vuole utilizzarne la funzione
%\chordsoff		% <<< DECOMMENTA se vuoi una strofa senza accordi

\[C]Cerchi un sor\[G]riso negli \[A-]occhi degli u\[C]omini
\[F]sogni avven\[C]ture che 
il \[D7]tempo porta con \[G]sé
\[C]danzi da \[G]sempre la \[E]gioia di \[A-]vivere
\[F] hai conosci\[C]uto l'uomo 
\[F] che ti ha par\[C]lato di un te\[F]so\[G]ro.

\endverse




%%%%% RITORNELLO
\beginchorus
\textnote{\textbf{Rit.}}

\[C] E quel te\[F]soro sai co\[C]s'è  \quad \[F*]\quad \[G*]
\[C] è la tua \[F]vita nell'a\[G]more 
\[F] è la gioia \[G]di chi annuncia 
l'u\[C*]omo che \[G*]torne\[F*]rà \[F]
e allora \[E-]sciogli i tuoi \[A-]piedi e va'
\[E-] tendi le \[A-]mani e va’ 
dove \[G#]vita è dav\[G7]vero.

\endchorus



%%%%% STROFA
\beginverse		%Oppure \beginverse* se non si vuole il numero di fianco
%\memorize 		% <<< DECOMMENTA se si vuole utilizzarne la funzione
%\chordsoff		% <<< DECOMMENTA se vuoi una strofa senza accordi

^Vivi nel ^mondo la ^storia degli u^omini
^apri il tuo ^cuore a chi 
nel ^mondo ha chiesto di ^te
^chiedi emozi^oni che ^corrano li^bere
^ ed hai cre^duto all'uomo 
^ che ti ha par^lato di un te^so^ro.

\endverse




%%%%% STROFA
\beginverse		%Oppure \beginverse* se non si vuole il numero di fianco
%\memorize 		% <<< DECOMMENTA se si vuole utilizzarne la funzione
%\chordsoff		% <<< DECOMMENTA se vuoi una strofa senza accordi

^Canti la ^pace nei ^gesti degli ^uomini
^offri spe^ranza a chi 
da ^tempo domanda un per^ché.
^Vivi l'at^tesa del ^giorno che ^libera
^ ed hai a^mato l'uomo 
^ che ti ha par^lato di un te^so^ro.

\endverse





\endsong
%------------------------------------------------------------
%			FINE CANZONE
%------------------------------------------------------------


