%-------------------------------------------------------------
%			INIZIO	CANZONE
%-------------------------------------------------------------


%titolo: 	Tu sarai profeta
%autore: 	M. Frisina
%tonalita: 	Re 



%%%%%% TITOLO E IMPOSTAZONI
\beginsong{Tu sarai profeta}[by={M. Frisina}] 	% <<< MODIFICA TITOLO E AUTORE
\transpose{0} 						% <<< TRASPOSIZIONE #TONI (0 nullo)
%\preferflats  %SE VOGLIO FORZARE i bemolle come alterazioni
%\prefersharps %SE VOGLIO FORZARE i # come alterazioni
\momenti{}							% <<< INSERISCI MOMENTI	
% momenti vanno separati da ; e vanno scelti tra:
% Ingresso; Atto penitenziale; Acclamazione al Vangelo; Dopo il Vangelo; Offertorio; Comunione; Ringraziamento; Fine; Santi; Pasqua; Avvento; Natale; Quaresima; Canti Mariani; Battesimo; Prima Comunione; Cresima; Matrimonio; Meditazione; Spezzare del pane;
\ifchorded
	%\textnote{Tonalità migliore }	% <<< EV COMMENTI (tonalità originale\migliore)
\fi


%%%%%% INTRODUZIONE
\ifchorded
\vspace*{\versesep}
\textnote{Intro: \qquad \qquad  }%(\eighthnote 116) % <<  MODIFICA IL TEMPO
% Metronomo: \eighthnote (ottavo) \quarternote (quarto) \halfnote (due quarti)
\vspace*{-\versesep}
\beginverse*

\nolyrics

%---- Prima riga -----------------------------
\vspace*{-\versesep}
\[D] \[A] \[B-]	 \[F#-] % \[*D] per indicare le pennate, \rep{2} le ripetizioni

%---- Ogni riga successiva -------------------
\vspace*{-\versesep}
\[G] \[D]  \[A] \[A]	

%---- Ev Indicazioni -------------------------			
%\textnote{\textit{(Oppure tutta la strofa)} }	

\endverse
\fi




%%%%% STROFA
\beginverse		%Oppure \beginverse* se non si vuole il numero di fianco
\memorize 		% <<< DECOMMENTA se si vuole utilizzarne la funzione
%\chordsoff		% <<< DECOMMENTA se vuoi una strofa senza accordi

\[D]Una \[A]luce che ri\[B-]schia\[F#-]ra,
\[D]una \[G]lampada che \[D]ar\[A]de,
\[B-*]u\[G*]na \[G]voce che pro\[A]cla\[B-]ma
\[E-]la Pa\[G]rola di sal\[A4]vezza.
\[D]Precur\[A]sore nella \[B-]gio\[F#-]ia,
\[D]precu\[G]rsore nel \[D]dolo\[A]re,
\[B-*]tu \[G*]che \[G]sveli nel \[A]perdo\[B-]no
\[E-]l'annunzio \[G]di miseri\[A]cord\[D]ia.

\endverse




%%%%% RITORNELLO
\beginchorus
\textnote{\textbf{Rit.}}

Tu sa\[G]ra\[D]i pro\[E-]feta di salve\[B-]zza
\[G]fino ai con\[D]fini della \[A4]terra,
porte\[G]ra\[D]i la \[E-]mia Pa\[B-]rola,
\[G]risplende\[D]rai della mia \[A]lu\[D]ce.

\endchorus



%%%%% STROFA
\beginverse		%Oppure \beginverse* se non si vuole il numero di fianco
%\memorize 		% <<< DECOMMENTA se si vuole utilizzarne la funzione
%\chordsoff		% <<< DECOMMENTA se vuoi una strofa senza accordi

\[D]Forte \[A]amico dello \[B-]Spo\[F#-]so,
\[D]che \[G]gioisci alla sua \[D]vo\[A]ce,
\[B-*]tu \[G*]cam\[G]mini per il \[A]mon\[B-]do
\[E-]per pre\[G]cedere il \[A4]Signore.
\[D]Stende\[A]rò la mia \[B-]ma\[F#-]no
\[D]e por\[G]rò sulla tua \[D]boc\[A]ca
\[B-*]la \[G*]po\[G]tente mia Pa\[A]ro\[B-]la
\[E-]che con\[G]vertirà il \[A]mon\[D]do.

\endverse



\endsong
%------------------------------------------------------------
%			FINE CANZONE
%------------------------------------------------------------


