%-------------------------------------------------------------
%			INIZIO	CANZONE
%-------------------------------------------------------------


%titolo: 	Benedite e acclamate
%autore: 	A. Casale
%tonalita: 	Fa 



%%%%%% TITOLO E IMPOSTAZONI
\beginsong{Benedite e acclamate}[by={A. Casale}] 	% <<< MODIFICA TITOLO E AUTORE
\transpose{0} 						% <<< TRASPOSIZIONE #TONI (0 nullo)
\momenti{Ingresso; Matrimonio; }							% <<< INSERISCI MOMENTI	
% momenti vanno separati da ; e vanno scelti tra:
% Ingresso; Atto penitenziale; Acclamazione al Vangelo; Dopo il Vangelo; Offertorio; Comunione; Ringraziamento; Fine; Santi; Pasqua; Avvento; Natale; Quaresima; Canti Mariani; Battesimo; Prima Comunione; Cresima; Matrimonio; Meditazione; Spezzare del pane;
\ifchorded
	\textnote{$\bigstar$ Tonalità migliore }	% <<< EV COMMENTI (tonalità originale/migliore)
\fi



%%%%%% INTRODUZIONE
\ifchorded
\vspace*{\versesep}
\musicnote{
\begin{minipage}{0.48\textwidth}
\textbf{Intro}
\hfill 
(\eighthnote \, 88)
%( \eighthnote \, 80)   % <<  MODIFICA IL TEMPO
% Metronomo: \eighthnote (ottavo) \quarternote (quarto) \halfnote (due quarti)
\end{minipage}
} 	
\vspace*{-\versesep}
\beginverse*


\nolyrics

%---- Prima riga -----------------------------
\vspace*{-\versesep}
\[F] \[C] \[D-]	 % \[*D] per indicare le pennate, \rep{2} le ripetizioni

%---- Ogni riga successiva -------------------
\vspace*{-\versesep}
\[B&] \[F]  \[C] \[B&]  \[B&] 	

%---- Ev Indicazioni -------------------------			
%\textnote{\textit{(Oppure tutta la strofa)} }	

\endverse
\fi








%%%%% RITORNELLO
\beginchorus
\textnote{\textbf{Rit.}}

Bene\[F]dite e accla\[C]mate
il Si\[D-]gnore di tutta la terra,
egli \[G-]compie prodigi
chia\[B&]mando ogni uomo all'a\[C]more. \[C]
Bene\[F]dite e accla\[C]mate
il Si\[D-]gnore di tutta la terra,
egli \[G-]compie prodigi 
u\[B&]nendo un uomo e una \[C]donna.

\endchorus



%%%%% STROFA
\beginverse		%Oppure \beginverse* se non si vuole il numero di fianco
\memorize 		% <<< DECOMMENTA se si vuole utilizzarne la funzione
%\chordsoff		% <<< DECOMMENTA se vuoi una strofa senza accordi

\[D-]Guarda, Signore, 
questo \[G-]patto d'amore
che per \[B&]grazia questi \[E&]sposi
con\[C4]sacrano da\[C]vanti a Te.
E sia \[D-]segno di speranza
e sia \[G-]segno che rivela
il mi\[B&]stero dell'a\[E&]more:
la \[C4]fonte del cre\[C]ato.

\endverse



%%%%% STROFA
\beginverse		%Oppure \beginverse* se non si vuole il numero di fianco
%\memorize 		% <<< DECOMMENTA se si vuole utilizzarne la funzione
%\chordsoff		% <<< DECOMMENTA se vuoi una strofa senza accordi

^Padre di ogni bene,
Tu, che ^guidi i loro passi,
bene^dici nel tuo ^nome
i ^sogni che li u^niscono.
E sia ^pace sullo sposo
e sia ^vita nella sposa, 
la Tua ^mano mostri an^cora
la ^Tua benevo^lenza.

\endverse



\endsong
%------------------------------------------------------------
%			FINE CANZONE
%------------------------------------------------------------

%++++++++++++++++++++++++++++++++++++++++++++++++++++++++++++
%			CANZONE TRASPOSTA
%++++++++++++++++++++++++++++++++++++++++++++++++++++++++++++
\ifchorded
%decremento contatore per avere stesso numero
\addtocounter{songnum}{-1} 
\beginsong{Benedite e acclamate}[by={A. Casale}] 	% <<< COPIA TITOLO E AUTORE
\transpose{+2} 						% <<< TRASPOSIZIONE #TONI + - (0 nullo)
%\preferflats  %SE VOGLIO FORZARE i bemolle come alterazioni
%\prefersharps %SE VOGLIO FORZARE i # come alterazioni
\ifchorded
	\textnote{$\lozenge$ Tonalità originale}	% <<< EV COMMENTI (tonalità originale/migliore)
\fi



%%%%%% INTRODUZIONE
\ifchorded
\vspace*{\versesep}
\musicnote{
\begin{minipage}{0.48\textwidth}
\textbf{Intro}
\hfill 
(\eighthnote \, 88)
%( \eighthnote \, 80)   % <<  MODIFICA IL TEMPO
% Metronomo: \eighthnote (ottavo) \quarternote (quarto) \halfnote (due quarti)
\end{minipage}
} 	
\vspace*{-\versesep}
\beginverse*


\nolyrics

%---- Prima riga -----------------------------
\vspace*{-\versesep}
\[F] \[C] \[D-]	 % \[*D] per indicare le pennate, \rep{2} le ripetizioni

%---- Ogni riga successiva -------------------
\vspace*{-\versesep}
\[B&] \[F]  \[C] \[B&]  \[B&] 	

%---- Ev Indicazioni -------------------------			
%\textnote{\textit{(Oppure tutta la strofa)} }	

\endverse
\fi








%%%%% RITORNELLO
\beginchorus
\textnote{\textbf{Rit.}}

Bene\[F]dite e accla\[C]mate
il Si\[D-]gnore di tutta la terra,
egli \[G-]compie prodigi
chia\[B&]mando ogni uomo all'a\[C]more. \[C]
Bene\[F]dite e accla\[C]mate
il Si\[D-]gnore di tutta la terra,
egli \[G-]compie prodigi 
u\[B&]nendo un uomo e una \[C]donna.

\endchorus



%%%%% STROFA
\beginverse		%Oppure \beginverse* se non si vuole il numero di fianco
\memorize 		% <<< DECOMMENTA se si vuole utilizzarne la funzione
%\chordsoff		% <<< DECOMMENTA se vuoi una strofa senza accordi

\[D-]Guarda, Signore, 
questo \[G-]patto d'amore
che per \[B&]grazia questi \[E&]sposi
con\[C4]sacrano da\[C]vanti a Te.
E sia \[D-]segno di speranza
e sia \[G-]segno che rivela
il mi\[B&]stero dell'a\[E&]more:
la \[C4]fonte del cre\[C]ato.

\endverse



%%%%% STROFA
\beginverse		%Oppure \beginverse* se non si vuole il numero di fianco
%\memorize 		% <<< DECOMMENTA se si vuole utilizzarne la funzione
%\chordsoff		% <<< DECOMMENTA se vuoi una strofa senza accordi

^Padre di ogni bene,
Tu, che ^guidi i loro passi,
bene^dici nel tuo ^nome
i ^sogni che li u^niscono.
E sia ^pace sullo sposo
e sia ^vita nella sposa, 
la Tua ^mano mostri an^cora
la ^Tua benevo^lenza.

\endverse



\endsong
\fi
%++++++++++++++++++++++++++++++++++++++++++++++++++++++++++++
%			FINE CANZONE TRASPOSTA
%++++++++++++++++++++++++++++++++++++++++++++++++++++++++++++

