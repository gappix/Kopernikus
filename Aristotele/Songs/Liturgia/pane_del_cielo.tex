%-------------------------------------------------------------
%			INIZIO	CANZONE
%-------------------------------------------------------------


%titolo: 	Pane del cielo
%autore: 	Gen rosso
%tonalita: 	Do 



%%%%%% TITOLO E IMPOSTAZONI
\beginsong{Pane del Cielo}[by={Gen Rosso}] 	% <<< MODIFICA TITOLO E AUTORE
\transpose{0} 						% <<< TRASPOSIZIONE #TONI (0 nullo)
\momenti{Offertorio; Spezzare del pane;}							% <<< INSERISCI MOMENTI	
% momenti vanno separati da ; e vanno scelti tra:
% Ingresso; Atto penitenziale; Acclamazione al Vangelo; Dopo il Vangelo; Offertorio; Comunione; Ringraziamento; Fine; Santi; Pasqua; Avvento; Natale; Quaresima; Canti Mariani; Battesimo; Prima Comunione; Cresima; Matrimonio; Meditazione;
\ifchorded
	%\textnote{Tonalità originale }	% <<< EV COMMENTI (tonalità originale/migliore)
\fi


%%%%%% INTRODUZIONE
\ifchorded
\vspace*{\versesep}
\textnote{Intro: \qquad \qquad  }%(\eighthnote 116) % <<  MODIFICA IL TEMPO
% Metronomo: \eighthnote (ottavo) \quarternote (quarto) \halfnote (due quarti)
\vspace*{-\versesep}
\beginverse*

\nolyrics

%---- Prima riga -----------------------------
\vspace*{-\versesep}
\[C] \[E-] \[F]	 \[C] % \[*D] per indicare le pennate, \rep{2} le ripetizioni

%---- Ogni riga successiva -------------------
%\vspace*{-\versesep}
%\[G] \[C]  \[D]	

%---- Ev Indicazioni -------------------------			
\textnote{\textit{(Come la prima riga)} }	

\endverse
\fi



%%%%% RITORNELLO
\beginchorus
\textnote{\textbf{Rit.}}
\[C]Pane del \[E-]cielo, \[F]sei Tu Ge\[C]sù,
\[A-]via d'a\[D-]more: \[F]Tu ci fai come \[C]Te. \rep{2}
\endchorus


%%%%% STROFA
\beginverse		%Oppure \beginverse* se non si vuole il numero di fianco
\memorize 		% <<< DECOMMENTA se si vuole utilizzarne la funzione
%\chordsoff		% <<< DECOMMENTA se vuoi una strofa senza accordi
\[F]No, non è ri\[D-]masta fredda la \[G]terra;
\[E-]Tu sei ri\[F]masto con \[C]noi \[F] per nutrirci di \[C]Te.
\[A-]Pane di \[G]vita, \[A-] 
ed infiam\[G]mare col tuo a\[E-]more
\[G]tutta l'u\[F]mani\[C]tà.
\endverse


%%%%% STROFA
\beginverse		%Oppure \beginverse* se non si vuole il numero di fianco
%\memorize 		% <<< DECOMMENTA se si vuole utilizzarne la funzione
%\chordsoff		% <<< DECOMMENTA se vuoi una strofa senza accordi
^Sì, il cielo è ^qui su questa ^terra;
^Tu sei ri^masto con ^noi ^ ma ci porti con ^Te
^nella tua ^casa ^ 
dove vi^vremo insieme a ^Te
^tutta l'e^terni^tà.
\endverse



%%%%% STROFA
\beginverse		%Oppure \beginverse* se non si vuole il numero di fianco
%\memorize 		% <<< DECOMMENTA se si vuole utilizzarne la funzione
%\chordsoff		% <<< DECOMMENTA se vuoi una strofa senza accordi
^No, la morte non ^può farci pa^ura;
^Tu sei ri^masto con ^noi, ^ e chi vive di ^Te
^vive per ^sempre. ^
Sei Dio con ^noi, sei Dio per ^noi,
^Dio in ^mezzo a ^noi.
\endverse

\endsong
%------------------------------------------------------------
%			FINE CANZONE
%------------------------------------------------------------


