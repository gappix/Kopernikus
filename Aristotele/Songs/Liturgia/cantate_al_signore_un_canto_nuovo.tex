%-------------------------------------------------------------
%			INIZIO	CANZONE
%-------------------------------------------------------------


%titolo: 	Cantate al Signore
%autore: 	Fallormi
%tonalita: 	Sol 



%%%%%% TITOLO E IMPOSTAZONI
\beginsong{Cantate al Signore}[by={M. Fallormi}] 	% <<< MODIFICA TITOLO E AUTORE
\transpose{-3} 						% <<< TRASPOSIZIONE #TONI (0 nullo)
\momenti{Ingresso}							% <<< INSERISCI MOMENTI	
% momenti vanno separati da ; e vanno scelti tra:
% Ingresso; Atto penitenziale; Acclamazione al Vangelo; Dopo il Vangelo; Offertorio; Comunione; Ringraziamento; Fine; Santi; Pasqua; Avvento; Natale; Quaresima; Canti Mariani; Battesimo; Prima Comunione; Cresima; Matrimonio; Meditazione; Spezzare del pane;
\ifchorded
	%\textnote{Tonalità migliore }	% <<< EV COMMENTI (tonalità originale/migliore)
\fi




%%%%%% INTRODUZIONE
\ifchorded
\vspace*{\versesep}
\musicnote{
\begin{minipage}{0.48\textwidth}
\textbf{Intro}
\hfill 
%( \eighthnote \, 80)   % <<  MODIFICA IL TEMPO
% Metronomo: \eighthnote (ottavo) \quarternote (quarto) \halfnote (due quarti)
\end{minipage}
} 	
\vspace*{-\versesep}
\beginverse*


\nolyrics

%---- Prima riga -----------------------------
\vspace*{-\versesep}
\[G]\[D]\[C]\[D] \rep{2}	 % \[*D] per indicare le pennate, \rep{2} le ripetizioni

%---- Ogni riga successiva -------------------
%\vspace*{-\versesep}
%\[G] \[C]  \[D]	

%---- Ev Indicazioni -------------------------			
\textnote{\textit{(oppure tutto il ritornello)} }	

\endverse
\fi



%%%%% RITORNELLO
\beginchorus
\textnote{\textbf{Rit.}}
Can\[G]tate al Si\[D]gnore un \[C]canto \[G]nuovo,
\[C]perché ha com\[G]piuto pro\[A-7]di|\[D4]gi.
\[(D*)]Ha |\[G]manife\[D]stato la \[C]sua sal\[G]vezza,
\[C]su tutti i |\[B-7/G]popo\[E-7*]li la |\[A-7]sua bon\[G]tà.
\endchorus


%%%%%% EV. INTERMEZZO
\beginverse*
\vspace*{1.3\versesep}
{
	\nolyrics
	\textnote{Intermezzo strumentale}
	
	\ifchorded

	%---- Prima riga -----------------------------
	\vspace*{-\versesep}
	\[G]\[D]\[C]\[D] \rep{2}


	\fi
	%---- Ev Indicazioni -------------------------			
	%\textnote{\textit{(ripetizione della strofa)}} 
	 
}
\vspace*{\versesep}
\endverse

\beginverse
\memorize
\[G] Egli \[C]si è ricor\[G]dato
\[(G)] della \[E-7]sua |\[C]fedel\[D]tà.
\[C] I con\[D]fini \[B-7]della \[E-7]terra 
\[C]hanno ve\[G]duto 
la sal\[A-7]vezza \[D7]del Si\[G]gnor.
\endverse


\beginverse
^ Esul^tiamo di ^gioia 
^ accla^miamo ^al Si^gnor. 
^ Con un ^suono ^melo^dioso: 
^cantiamo in^sieme
lode e ^gloria al ^nostro ^Re.
\endverse




\beginverse
^ Frema il ^mare e la ^terra, 
^ il Si^gno^re ver^rà! 
^ Con giu^dizio ^di giu^stizia, 
^con retti^tudine 
nel ^mondo ^porte\[E&]ra-a-\[F]a. \quad  \[F]
\endverse

%%%%% RITORNELLO
\beginchorus
\textnote{\textbf{Rit.}}
\transpose{3}
Can\[G]tate al Si\[D]gnore un \[C]canto \[G]nuovo,
\[C]perché ha com\[G]piuto pro\[A-7]di|\[D4]gi.
\[(D*)]Ha |\[G]manife\[D]stato la \[C]sua sal\[G]vezza, 
\vspace*{0.5\versesep}
\textnote{\textit{(rallentando)}}
\[C]su tutti i |\[B-7/G]popo\[E-7*]li 
la |\[A-7]sua \[D]bon\[G]tà.
\endchorus


\endsong
%------------------------------------------------------------
%			FINE CANZONE
%------------------------------------------------------------




