
%-------------------------------------------------------------
%			INIZIO	CANZONE
%-------------------------------------------------------------


%titolo: 	I cieli narrano
%autore: 	Frisina
%tonalita: 	Re > Do



%%%%%% TITOLO E IMPOSTAZONI
\beginsong{I cieli narrano}[by={M. Frisina}]	% <<< MODIFICA TITOLO E AUTORE
\transpose{-2} 						% <<< TRASPOSIZIONE #TONI (0 nullo)
\momenti{Ingresso}							% <<< INSERISCI MOMENTI	
% momenti vanno separati da ; e vanno scelti tra:
% Ingresso; Atto penitenziale; Acclamazione al Vangelo; Dopo il Vangelo; Offertorio; Comunione; Ringraziamento; Fine; Santi; Pasqua; Avvento; Natale; Quaresima; Canti Mariani; Battesimo; Prima Comunione; Cresima; Matrimonio; Meditazione; Spezzare del pane;
\ifchorded
	%\textnote{Tonalità migliore }	% <<< EV COMMENTI (tonalità originale/migliore)
\fi


%%%%%% INTRODUZIONE
\ifchorded
\vspace*{\versesep}
\musicnote{
\begin{minipage}{0.48\textwidth}
\textbf{Intro}
\hfill 
%( \eighthnote \, 80)   % <<  MODIFICA IL TEMPO
% Metronomo: \eighthnote (ottavo) \quarternote (quarto) \halfnote (due quarti)
\end{minipage}
} 	
\vspace*{-\versesep}
\beginverse*
\nolyrics

%---- Prima riga -----------------------------
\vspace*{-\versesep}
\[D] \[G] \[D] \[G]	 % \[*D] per indicare le pennate, \rep{2} le ripetizioni

%---- Ogni riga successiva -------------------
%\vspace*{-\versesep}
%\[G] \[C]  \[D]	

%---- Ev Indicazioni -------------------------			
%\textnote{\textit{(Oppure tutta la strofa)} }	

\endverse
\fi

\beginchorus
I \[D]cieli \[G]narrano la \[D]gloria di \[A]Dio
e il \[B-]firma\[G]mento annunzia \[D]l'opera \[A]sua.
Al\[B-]lelu\[E-]ia al\[A]lelu\[D]ia \brk al\[B-]lelu\[G]ia alle\[E-7]\[A*]lu\[D]ia.
\endchorus

\beginverse
\memorize
Il \[D]giorno al \[G]giorno ne a\ch{D}{f}{f}{ff}ida il mes\[A]saggio
la \[B-]notte alla \[G]notte ne tras\[D]mette no\[A]tizia
non \[B-]è un lin\[G]guaggio non \[A]sono pa\[D]role 
di \[B-]cui non si \[E]oda il \[A]suo\[A7]no.
\endverse

\beginverse
%\chordsoff
Là ^pose una ^tenda per il ^sole che ^sorge
è ^come uno s^poso dalla s^tanza nu^ziale
e^sulta ^come un ^prode che ^corre 
con ^gioia la ^sua s^tra^da.
\endverse

\beginverse
%\chordsoff
Lui s^orge dall'u^ltimo est^remo del ci^elo
e ^la sua ^corsa l'altro es^tremo rag^giunge.
Nes^suna ^delle crea^ture po^trà 
mai sot^trarsi al ^suo ca^lo^re. 
\endverse

\beginverse
%\chordsoff
La ^legge di ^Dio rinfr^anca ^l'anima
la ^testimoni^anza del Si^gnore è ve^race.
Gio^isce il ^cuore ai suoi ^giusti pre^cetti 
che ^danno la ^luce agli ^oc^chi.
\endverse
\endsong


