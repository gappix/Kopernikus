%-------------------------------------------------------------
%			INIZIO	CANZONE
%-------------------------------------------------------------


%titolo: 	Benedici O Signore
%autore: 	Gen Rosso
%tonalita: 	Do-/Si-



%%%%%% TITOLO E IMPOSTAZONI
\beginsong{Benedici o Signore}[ititle={Nebbia e freddo}, by={Gen Rosso}] 	% <<< MODIFICA TITOLO E AUTORE
\transpose{-3} 						% <<< TRASPOSIZIONE #TONI (0 nullo)
\momenti{Offertorio}							% <<< INSERISCI MOMENTI	
% momenti vanno separati da ; e vanno scelti tra:
% Ingresso; Atto penitenziale; Acclamazione al Vangelo; Dopo il Vangelo; Offertorio; Comunione; Ringraziamento; Fine; Santi; Pasqua; Avvento; Natale; Quaresima; Canti Mariani; Battesimo; Prima Comunione; Cresima; Matrimonio; Meditazione; Spezzare del pane;
\ifchorded
	\textnote{Tonalità migliore}	% <<< EV COMMENTI (tonalità originale/migliore)
\fi


%%%%%% INTRODUZIONE
\ifchorded
\vspace*{\versesep}
\textnote{Intro: \qquad \qquad  (\eighthnote 120)} % <<  MODIFICA IL TEMPO
% Metronomo: \eighthnote (ottavo) \quarternote (quarto) \halfnote (due quarti)
\vspace*{-\versesep}
\beginverse*

\nolyrics

%---- Prima riga -----------------------------
\vspace*{-\versesep}
\[C-] \[B&] \[C-]	 % \[*D] per indicare le pennate, \rep{2} le ripetizioni

%---- Ogni riga successiva -------------------
%\vspace*{-\versesep}
%\[G] \[C]  \[D]	

%---- Ev Indicazioni -------------------------			
%\textnote{\textit{(Oppure tutta la strofa)} }	

\endverse
\fi




%%%%% STROFA
\beginverse*		%Oppure \beginverse* se non si vuole il numero di fianco
\memorize 		% <<< DECOMMENTA se si vuole utilizzarne la funzione
%\chordsoff		% <<< DECOMMENTA se vuoi una strofa senza accordi

\[C-]Nebbia e freddo, giorni lunghi e a\[B&]mari
mentre il seme mu\[C-]ore.
\[E&]Poi il prodigio antico e sempre n\[B&]uovo
del primo filo d'e\[A&]rba.
E nel \[E&]vento dell'es\[B&]tate on\[C-]deggiano le s\[E&]pighe;
av\[B&]remo ancora \[G]pa\[C]ne.


\endverse




%%%%% RITORNELLO
\beginchorus
%\textnote{\textbf{Rit.}}

\[F]Bene\[C]dici, \[F]o Si\[C]gnore,
\[B&]questa of\[F]ferta che por\[G4]tiamo a Te.
F\[F]acci \[C]uno, \[A-]come il \[E-]pane,
\[D]che anche \[F]oggi hai \[C]dato a noi.

\endchorus



%%%%%% EV. INTERMEZZO
\beginverse*
\vspace*{1.3\versesep}
{
	\nolyrics
	\textnote{Ev. intermezzo strumentale}
	
	\ifchorded

	%---- Prima riga -----------------------------
	\vspace*{-\versesep}
	\[C-] \[B&] \[C-]


	\fi
	%---- Ev Indicazioni -------------------------			
	%\textnote{\textit{(ripetizione della strofa)}} 
	 
}
\vspace*{\versesep}
\endverse



%%%%% STROFA
\beginverse*		%Oppure \beginverse* se non si vuole il numero di fianco
%\memorize 		% <<< DECOMMENTA se si vuole utilizzarne la funzione
%\chordsoff		% <<< DECOMMENTA se vuoi una strofa senza accordi

^Nei filari, dopo il lungo in^verno
fremono le ^viti.
^La rugiada avvolge nel si^lenzio
i primi tralci ^verdi.
Poi i co^lori dell'au^tunno coi ^grappoli ma^turi;
av^remo ancora ^vi^no.


\endverse




%%%%% RITORNELLO
\beginchorus
%\textnote{\textbf{Rit.}}

\[F]Bene\[C]dici, \[F]o Si\[C]gnore,
\[B&]questa of\[F]ferta che por\[G4]tiamo a Te.
F\[F]acci \[C]uno, \[A-]come il \[E-]vino,
\[D]che anche \[F]oggi hai \[C]dato a noi.

\endchorus





\endsong
%------------------------------------------------------------
%			FINE CANZONE
%------------------------------------------------------------





%++++++++++++++++++++++++++++++++++++++++++++++++++++++++++++
%			CANZONE TRASPOSTA
%++++++++++++++++++++++++++++++++++++++++++++++++++++++++++++
\ifchorded
%decremento contatore per avere stesso numero
\addtocounter{songnum}{-1} 
\beginsong{Benedici o Signore}[by={Gen Rosso}] 	% <<< COPIA TITOLO E AUTORE
\transpose{0} 						% <<< TRASPOSIZIONE #TONI + - (0 nullo)
%\preferflats  %SE VOGLIO FORZARE i bemolle come alterazioni
%\prefersharps %SE VOGLIO FORZARE i # come alterazioni
\ifchorded
	\textnote{Tonalità originale}	% <<< EV COMMENTI (tonalità originale/migliore)
\fi


%%%%%% INTRODUZIONE
\ifchorded
\vspace*{\versesep}
\textnote{Intro: \qquad \qquad  (\eighthnote 120)} % <<  MODIFICA IL TEMPO
% Metronomo: \eighthnote (ottavo) \quarternote (quarto) \halfnote (due quarti)
\vspace*{-\versesep}
\beginverse*

\nolyrics

%---- Prima riga -----------------------------
\vspace*{-\versesep}
\[C-] \[B&] \[C-]	 % \[*D] per indicare le pennate, \rep{2} le ripetizioni

%---- Ogni riga successiva -------------------
%\vspace*{-\versesep}
%\[G] \[C]  \[D]	

%---- Ev Indicazioni -------------------------			
%\textnote{\textit{(Oppure tutta la strofa)} }	

\endverse
\fi



%%%%% STROFA
\beginverse*		%Oppure \beginverse* se non si vuole il numero di fianco
\memorize 		% <<< DECOMMENTA se si vuole utilizzarne la funzione
%\chordsoff		% <<< DECOMMENTA se vuoi una strofa senza accordi

\[C-]Nebbia e freddo, giorni lunghi e a\[B&]mari
mentre il seme mu\[C-]ore.
\[E&]Poi il prodigio antico e sempre n\[B&]uovo
del primo filo d'e\[A&]rba.
E nel \[E&]vento dell'es\[B&]tate on\[C-]deggiano le s\[E&]pighe;
av\[B&]remo ancora \[G]pa\[C]ne.


\endverse




%%%%% RITORNELLO
\beginchorus
%\textnote{\textbf{Rit.}}

\[F]Bene\[C]dici, \[F]o Si\[C]gnore,
\[B&]questa of\[F]ferta che por\[G4]tiamo a Te.
F\[F]acci \[C]uno, \[A-]come il \[E-]pane,
\[D]che anche \[F]oggi hai \[C]dato a noi.

\endchorus



%%%%%% EV. INTERMEZZO
\beginverse*
\vspace*{1.3\versesep}
{
	\nolyrics
	\textnote{Ev. intermezzo strumentale}
	
	\ifchorded

	%---- Prima riga -----------------------------
	\vspace*{-\versesep}
	\[C-] \[B&] \[C-]


	\fi
	%---- Ev Indicazioni -------------------------			
	%\textnote{\textit{(ripetizione della strofa)}} 
	 
}
\vspace*{\versesep}
\endverse



%%%%% STROFA
\beginverse*		%Oppure \beginverse* se non si vuole il numero di fianco
%\memorize 		% <<< DECOMMENTA se si vuole utilizzarne la funzione
%\chordsoff		% <<< DECOMMENTA se vuoi una strofa senza accordi

^Nei filari, dopo il lungo in^verno
fremono le ^viti.
^La rugiada avvolge nel si^lenzio
i primi tralci ^verdi.
Poi i co^lori dell'au^tunno coi ^grappoli ma^turi;
av^remo ancora ^vi^no.


\endverse




%%%%% RITORNELLO
\beginchorus
%\textnote{\textbf{Rit.}}

\[F]Bene\[C]dici, \[F]o Si\[C]gnore,
\[B&]questa of\[F]ferta che por\[G4]tiamo a Te.
F\[F]acci \[C]uno, \[A-]come il \[E-]vino,
\[D]che anche \[F]oggi hai \[C]dato a noi.

\endchorus





\endsong


\fi
%++++++++++++++++++++++++++++++++++++++++++++++++++++++++++++
%			FINE CANZONE TRASPOSTA
%++++++++++++++++++++++++++++++++++++++++++++++++++++++++++++
