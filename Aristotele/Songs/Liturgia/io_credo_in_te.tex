%-------------------------------------------------------------
%			INIZIO	CANZONE
%-------------------------------------------------------------


%titolo: 	Santo Ricci
%autore: 	Daniele Ricci
%tonalita: 	Sol 



%%%%%% TITOLO E IMPOSTAZONI
\beginsong{Io credo in te}[by={}] 	% <<< MODIFICA TITOLO E AUTORE
\transpose{0} 						% <<< TRASPOSIZIONE #TONI (0 nullo)
%\preferflats  %SE VOGLIO FORZARE i bemolle come alterazioni
%\prefersharps %SE VOGLIO FORZARE i # come alterazioni
\momenti{}							% <<< INSERISCI MOMENTI	
% momenti vanno separati da ; e vanno scelti tra:
% Ingresso; Atto penitenziale; Acclamazione al Vangelo; Dopo il Vangelo; Offertorio; Comunione; Ringraziamento; Fine; Santi; Pasqua; Avvento; Natale; Quaresima; Canti Mariani; Battesimo; Prima Comunione; Cresima; Matrimonio; Meditazione; Spezzare del pane;
\ifchorded
	%\textnote{$\bigstar$ Tonalità migliore }	% <<< EV COMMENTI (tonalità originale\migliore)
\fi


%%%%%% INTRODUZIONE
\ifchorded
\vspace*{\versesep}
\musicnote{
\begin{minipage}{0.48\textwidth}
\textbf{Intro}
\hfill 
%( \eighthnote \, 80)   % <<  MODIFICA IL TEMPO
% Metronomo: \eighthnote (ottavo) \quarternote (quarto) \halfnote (due quarti)
\end{minipage}
} 	
\vspace*{-\versesep}
\beginverse*

\nolyrics

%---- Prima riga -----------------------------
\vspace*{-\versesep}
\[A-] \[F] \[C]	\[G]  \[A-]% \[*D] per indicare le pennate, \rep{2} le ripetizioni

%---- Ogni riga successiva -------------------
%\vspace*{-\versesep}
%\[G] \[C]  \[D]	

%---- Ev Indicazioni -------------------------			
%\textnote{\textit{[oppure tutta la strofa]} }	

\endverse
\fi




%%%%% STROFA
\beginverse		%Oppure \beginverse* se non si vuole il numero di fianco
\memorize 		% <<< DECOMMENTA se si vuole utilizzarne la funzione
%\chordsoff		% <<< DECOMMENTA se vuoi una strofa senza accordi

\[A-]Per \[F]molti Tu sei \[C]storia, 
la \[G]pagina di un \[A-]libro 
un \[F]ruolo tea\[C]tra\[G]le. 
\[A-]Le tue \[F]urla, il tuo dol\[C]ore 
\[G]rivivono ogni \[A-]giorno 
ma \[F]non nel nostro \[C]cuo\[G]re.

\endverse




%%%%% RITORNELLO
\beginchorus
\textnote{\textbf{Rit.}}

\[A-]Io \[F]credo in \[C]Te \brk \echo{tu \[G]sei il figlio di \[A-]Dio} 
io \[F]credo in \[C]Te \brk \echo{ Tu \[G]sei risorto e \[A-]vivo} 
Io \[F]credo in \[C]Te \brk \echo{chi \[G]ha l'amore nei suoi \[A-]occhi ti \[F]riconos\[C]cerà.\[G]} \rep{2}

\endchorus



%%%%% STROFA
\beginverse		%Oppure \beginverse* se non si vuole il numero di fianco
%\memorize 		% <<< DECOMMENTA se si vuole utilizzarne la funzione
%\chordsoff		% <<< DECOMMENTA se vuoi una strofa senza accordi

^Senti l'^urlo di chi ^soffre, 
ogni ^sua lacrima ^Tu vedi, 
anche ^quelle ^non ver^sate. 
^Tu as^colti il mio si^lenzio 
^Tu conosci ogni ^stella 
Tu ^sei il Dio ^che ri^sorge 

\endverse


%%%%% RITORNELLO
\beginchorus
\textnote{\textbf{Rit.}}

\[A-]Io \[F]credo in \[C]Te \brk \echo{tu \[G]sei il figlio di \[A-]Dio} 
io \[F]credo in \[C]Te \brk \echo{ Tu \[G]sei risorto e \[A-]vivo} 
Io \[F]credo in \[C]Te \brk \echo{chi \[G]ha l'amore nei suoi \[A-]occhi ti \[F]riconos\[C]cerà.\[G]} \rep{2}

\endchorus





%%%%% BRIDGE
\beginverse*		%Oppure \beginverse* se non si vuole il numero di fianco
%\memorize 		% <<< DECOMMENTA se si vuole utilizzarne la funzione
%\chordsoff		% <<< DECOMMENTA se vuoi una strofa senza accordi
\vspace*{1.3\versesep}
\textnote{\textbf{Bridge}}

\[F]Grazie a Te io posso \[A-]credere in \[G]me         
\[F]Tu mi hai creato per \[G]amare. 

\endverse

\textnote{\textit{(si alza la tonalità)}}


\transpose{2}
%%%%% RITORNELLO
\beginchorus
\textnote{\textbf{Rit.}}

\[A-]Io \[F]credo in \[C]Te \brk \echo{tu \[G]sei il figlio di \[A-]Dio} 
io \[F]credo in \[C]Te \brk \echo{ Tu \[G]sei risorto e \[A-]vivo} 
Io \[F]credo in \[C]Te \brk \echo{chi \[G]ha l'amore nei suoi \[A-]occhi ti \[F]riconos\[C]cerà.\[G]} \rep{2}

\endchorus




%%%%%% EV. FINALE

\beginchorus %oppure \beginverse*
\vspace*{1.3\versesep}
\textnote{\textbf{Finale}} %<<< EV. INDICAZIONI

\[A-]Io \[F]credo in te… \normalfont\textit{(sospeso...)}

\endchorus  %oppure \endverse



\endsong
%------------------------------------------------------------
%			FINE CANZONE
%------------------------------------------------------------



