%-------------------------------------------------------------
%			INIZIO	CANZONE
%-------------------------------------------------------------


%titolo: 	Pane di vita nuova
%autore: 	Frisina
%tonalita: 	Re 



%%%%%% TITOLO E IMPOSTAZONI
\beginsong{Pane di vita nuova}[by={Frisina}] 	% <<< MODIFICA TITOLO E AUTORE
\transpose{0} 						% <<< TRASPOSIZIONE #TONI (0 nullo)
\momenti{Comunione; Spezzare del pane; Pasqua}							% <<< INSERISCI MOMENTI	
% momenti vanno separati da ; e vanno scelti tra:
% Ingresso; Atto penitenziale; Acclamazione al Vangelo; Dopo il Vangelo; Offertorio; Comunione; Ringraziamento; Fine; Santi; Pasqua; Avvento; Natale; Quaresima; Canti Mariani; Battesimo; Prima Comunione; Cresima; Matrimonio; Meditazione; Spezzare del pane;
\ifchorded
	%\textnote{Tonalità originale }	% <<< EV COMMENTI (tonalità originale/migliore)
\fi


%%%%%% INTRODUZIONE
\ifchorded
\vspace*{\versesep}
\textnote{Intro: \qquad \qquad  }%(\eighthnote 116) % <<  MODIFICA IL TEMPO
% Metronomo: \eighthnote (ottavo) \quarternote (quarto) \halfnote (due quarti)
\vspace*{-\versesep}
\beginverse*

\nolyrics

%---- Prima riga -----------------------------
\vspace*{-\versesep}
\[D] \[G] \[A] \[D]	 % \[*D] per indicare le pennate, \rep{2} le ripetizioni

%---- Ogni riga successiva -------------------
%\vspace*{-\versesep}
%\[G] \[C]  \[D]	

%---- Ev Indicazioni -------------------------			
%\textnote{\textit{(Oppure tutta la strofa)} }	

\endverse
\fi


%%%%% STROFA
\beginverse		%Oppure \beginverse* se non si vuole il numero di fianco
\memorize 		% <<< DECOMMENTA se si vuole utilizzarne la funzione
%\chordsoff		% <<< DECOMMENTA se vuoi una strofa senza accordi
\[D]Pane \[G]di vita \[A]nuo\[D]va, 
\[G]vero \[D]cibo dato agli \[E-]uomi\[A]ni,
\[G]nutri\[D]mento \[E-]che sostiene il \[A]mondo, 
\[B-*]do\[G]no  \[E-]splendido  di \[A]gra\[D]zia.
\endverse




%%%%% STROFA
\beginverse*		%Oppure \beginverse* se non si vuole il numero di fianco
%\memorize 		% <<< DECOMMENTA se si vuole utilizzarne la funzione
%\chordsoff		% <<< DECOMMENTA se vuoi una strofa senza accordi
^Tu sei ^sublime ^frut^to 
^di quell'^albero di ^vi^ta
^che A^damo ^non potè toc^care:
^Ora ^è in  ^Cristo a noi do^na^to.
\endverse







%%%%% RITORNELLO
\beginchorus
\textnote{\textbf{Rit.}}
\[G]Pane \[D]della \[E-]vi\[A]ta, 
\[D]sangue \[F#-]di sal\[G]vez\[A]za,
\[G]vero \[D]corpo, \[E-]vera be\[B-]vanda,
\[E-]cibo di \[B-]grazia per il \[A]mon\[D]do.
\endchorus



%%%%% STROFA
\beginverse		%Oppure \beginverse* se non si vuole il numero di fianco
%\memorize 		% <<< DECOMMENTA se si vuole utilizzarne la funzione
%\chordsoff		% <<< DECOMMENTA se vuoi una strofa senza accordi
^Sei l'A^gnello immo^la^to
^nel cui ^Sangue è la sal^vez^za,
^memo^riale ^della vera ^Pasqua
^del^la ^nuova Alle^an^za.
\endverse



%%%%% STROFA
\beginverse*		%Oppure \beginverse* se non si vuole il numero di fianco
%\memorize 		% <<< DECOMMENTA se si vuole utilizzarne la funzione
%\chordsoff		% <<< DECOMMENTA se vuoi una strofa senza accordi
^Manna ^che nel de^ser^to
^nutri il ^popolo in cam^mi^no,
^sei so^stegno e ^forza nella ^prova
^per ^la ^Chiesa in mezzo al ^mon^do.
\endverse


%%%%% STROFA
\beginverse		%Oppure \beginverse* se non si vuole il numero di fianco
%\memorize 		% <<< DECOMMENTA se si vuole utilizzarne la funzione
\chordsoff		% <<< DECOMMENTA se vuoi una strofa senza accordi
^Vino ^che ci dà ^gio^ia,
^che ri^scalda il nostro ^cuo^re,
^sei per ^noi ^il prezioso ^frutto
^del^la ^vigna del Si^gno^re.
\endverse

%%%%% STROFA
\beginverse*		%Oppure \beginverse* se non si vuole il numero di fianco
%\memorize 		% <<< DECOMMENTA se si vuole utilizzarne la funzione
\chordsoff		% <<< DECOMMENTA se vuoi una strofa senza accordi
^Dalla ^vite ai ^tral^ci
^scorre ^la vitale ^lin^fa
^che ci ^dona ^la vita di^vina,
^scor^re il ^sangue dell'a^mo^re.
\endverse



%%%%% STROFA
\beginverse
\chordsoff
^Al ban^chetto ci in^vi^ti
^che per ^noi hai prepa^ra^to,
^doni all'^uomo ^la tua Sa^pienza,
^do^ni il ^Verbo della ^vi^ta.
\endverse

%%%%% STROFA
\beginverse*		%Oppure \beginverse* se non si vuole il numero di fianco
%\memorize 		% <<< DECOMMENTA se si vuole utilizzarne la funzione
\chordsoff		% <<< DECOMMENTA se vuoi una strofa senza accordi
^Segno ^d'amore e^ter^no
^pegno ^di sublimi ^noz^ze,
^comu^nione ^nell'unico ^corpo
^che ^in ^Cristo noi for^mia^mo.
\endverse



\beginverse
\chordsoff
^Nel tuo ^Sangue è la ^vi^ta
^ed il ^fuoco dello ^Spiri^to,
^la sua ^fiamma in^cendia il nostro ^cuore 
^e ^pu^rifica il ^mon^do.
\endverse
\beginverse*
\chordsoff
^Nel pro^digio dei ^pa^ni
^tu sfa^masti ogni ^uo^mo,
^nel tuo a^more il ^povero è nu^trito
^e ^ri^ceve la tua ^vi^ta.
\endverse




\beginverse
\chordsoff
Sacerdote eterno
Tu sei vittima ed altare,
offri al Padre tutto l'universo,
sacrificio dell'amore.
\endverse
\beginverse*
\chordsoff
Il tuo Corpo è tempio
della lode della Chiesa,
dal costato tu l'hai generata,
nel tuo Sangue l'hai redenta.
\endverse






\beginverse
\chordsoff
Vero Corpo di Cristo
tratto da Maria Vergine,
dal tuo fianco doni a noi la grazia, 
per mandarci tra le genti.
\endverse
\beginverse*
\chordsoff
Dai confini del mondo,
da ogni tempo e ogni luogo
il creato a te renda grazie,
per l'eternità ti adori.
\endverse






\beginverse
\chordsoff
A te Padre la lode,
che donasti il Redentore,
e al Santo Spirito di vita 
sia per sempre onore e gloria. 
\endverse



%%%%%% EV. FINALE

\beginchorus %oppure \beginverse*
\vspace*{1.3\versesep}
\textnote{Finale } %<<< EV. INDICAZIONI

\[B-]\[A]A\[D]men.

\endchorus  %oppure \endverse




\endsong
%------------------------------------------------------------
%			FINE CANZONE
%------------------------------------------------------------


