%-------------------------------------------------------------
%			INIZIO	CANZONE
%-------------------------------------------------------------


%titolo: 	Voglio esaltare
%autore: 	Giampiero Colombo
%tonalita: 	MIm e REm 



%%%%%% TITOLO E IMPOSTAZONI
\beginsong{Voglio esaltare}[by={Giampiero Colombo}] 	% <<< MODIFICA TITOLO E AUTORE
\transpose{0} 						% <<< TRASPOSIZIONE #TONI (0 nullo)
\momenti{Ingresso; Avvento}							% <<< INSERISCI MOMENTI	
% momenti vanno separati da ; e vanno scelti tra:
% Ingresso; Atto penitenziale; Acclamazione al Vangelo; Dopo il Vangelo; Offertorio; Comunione; Ringraziamento; Fine; Santi; Pasqua; Avvento; Natale; Quaresima; Canti Mariani; Battesimo; Prima Comunione; Cresima; Matrimonio; Meditazione;
\ifchorded
	%\textnote{Tonalità originale }	% <<< EV COMMENTI (tonalità originale/migliore)
\fi


%%%%%% INTRODUZIONE
\ifchorded
\vspace*{\versesep}
\textnote{Intro: \qquad \qquad  }%(\eighthnote 116) % << MODIFICA IL TEMPO
% Metronomo: \eighthnote (ottavo) \quarternote (quarto) \halfnote (due quarti)
\vspace*{-\versesep}
\beginverse*

\nolyrics

%---- Prima riga -----------------------------
\vspace*{-\versesep}
\[E-] \[B-] \[E-] \[B-]  	 % \[*D] per indicare le pennate, \rep{2} le ripetizioni

%---- Ogni riga successiva -------------------
%\vspace*{-\versesep}
%\[G] \[C]  \[D]	

%---- Ev Indicazioni -------------------------			
%\textnote{\textit{(Oppure tutta la strofa)} }	

\endverse
\fi




%%%%% STROFA
\beginverse		%Oppure \beginverse* se non si vuole il numero di fianco
\memorize 		% <<< DECOMMENTA se si vuole utilizzarne la funzione
%\chordsoff		% <<< DECOMMENTA se vuoi una strofa senza accordi

\[E-]Voglio esal\[B-]tare \brk il \[E-]nome del Dio \[B-]nostro: 
\[C]è Lui la mia \[D]liber\[G]tà! \[B7]
\[E-]Ecco il ma\[B-]ttino, \brk \[E-]gioia di sal\[B-]vezza, 
un \[C]canto sta nas\[D]cendo in \[E-]noi.

\endverse




%%%%% RITORNELLO
\beginchorus
\textnote{\textbf{Rit.}}

\[A-]Vieni, o Si\[D7]gnore, \[G7]luce del cam\[C7]mino,
\[A-]fuoco che nel \[B]cuore 
ac\[E-]cen\[D]de il \[G]"sì". \[*E7]
\[A-]Lieto il tuo pas\[D7]saggio, \[G7]ritmi la spe\[C7]ranza,
\[A-]Padre della \[B7]veri\[E-]tà. \[B-] \[E-] \[B-]

\endchorus




%%%%% STROFA
\beginverse		%Oppure \beginverse* se non si vuole il numero di fianco
%\memorize 		% <<< DECOMMENTA se si vuole utilizzarne la funzione
%\chordsoff		% <<< DECOMMENTA se vuoi una strofa senza accordi
^Voglio esal^tare \brk il ^nome del Dio ^nostro
^grande nella ^fedel^tà! ^
^Egli mi ha ^posto \brk ^sull'alto suo ^monte.
^Roccia che non ^crolla ^mai.

\endverse


%%%%% STROFA
\beginverse		%Oppure \beginverse* se non si vuole il numero di fianco
%\memorize 		% <<< DECOMMENTA se si vuole utilizzarne la funzione
\chordsoff		% <<< DECOMMENTA se vuoi una strofa senza accordi

Voglio Annunciare il dono crocifisso
di Cristo, il Dio con noi!
Perchè della morte lui si prende gioco,
Figlio che ci attira a sé!

\endverse


\endsong
%------------------------------------------------------------
%			FINE CANZONE
%------------------------------------------------------------




%++++++++++++++++++++++++++++++++++++++++++++++++++++++++++++
%			CANZONE TRASPOSTA
%++++++++++++++++++++++++++++++++++++++++++++++++++++++++++++
\ifchorded
%decremento contatore per avere stesso numero
\addtocounter{songnum}{-1} 
\beginsong{Voglio esaltare}[by={Giampiero Colombo}]  	% <<< COPIA TITOLO E AUTORE
\transpose{-2} 						% <<< TRASPOSIZIONE #TONI + - (0 nullo)
\ifchorded
	\textnote{Tonalità migliore per le bambine}	% <<< EV COMMENTI (tonalità originale/migliore)
\fi



%%%%%% INTRODUZIONE
\ifchorded
\vspace*{\versesep}
\textnote{Intro: \qquad \qquad  }%(\eighthnote 116) % << MODIFICA IL TEMPO
% Metronomo: \eighthnote (ottavo) \quarternote (quarto) \halfnote (due quarti)
\vspace*{-\versesep}
\beginverse*

\nolyrics

%---- Prima riga -----------------------------
\vspace*{-\versesep}
\[E-] \[B-] \[E-] \[B-]  	 % \[*D] per indicare le pennate, \rep{2} le ripetizioni

%---- Ogni riga successiva -------------------
%\vspace*{-\versesep}
%\[G] \[C]  \[D]	

%---- Ev Indicazioni -------------------------			
%\textnote{\textit{(Oppure tutta la strofa)} }	

\endverse
\fi




%%%%% STROFA
\beginverse		%Oppure \beginverse* se non si vuole il numero di fianco
\memorize 		% <<< DECOMMENTA se si vuole utilizzarne la funzione
%\chordsoff		% <<< DECOMMENTA se vuoi una strofa senza accordi

\[E-]Voglio esal\[B-]tare \brk il \[E-]nome del Dio \[B-]nostro: 
\[C]è Lui la mia \[D]liber\[G]tà! \[B7]
\[E-]Ecco il ma\[B-]ttino, \brk \[E-]gioia di sal\[B-]vezza, 
un \[C]canto sta nas\[D]cendo in \[E-]noi.

\endverse




%%%%% RITORNELLO
\beginchorus
\textnote{\textbf{Rit.}}

\[A-]Vieni, o Si\[D7]gnore, \[G7]luce del cam\[C7]mino,
\[A-]fuoco che nel \[B]cuore 
ac\[E-]cen\[D]de il \[G]"sì". \[*E7]
\[A-]Lieto il tuo pas\[D7]saggio, \[G7]ritmi la spe\[C7]ranza,
\[A-]Padre della \[B7]veri\[E-]tà. \[B-] \[E-] \[B-]

\endchorus




%%%%% STROFA
\beginverse		%Oppure \beginverse* se non si vuole il numero di fianco
%\memorize 		% <<< DECOMMENTA se si vuole utilizzarne la funzione
%\chordsoff		% <<< DECOMMENTA se vuoi una strofa senza accordi
^Voglio esal^tare \brk il ^nome del Dio ^nostro
^grande nella ^fedel^tà! ^
^Egli mi ha ^posto \brk ^sull'alto suo ^monte.
^Roccia che non ^crolla ^mai.

\endverse


%%%%% STROFA
\beginverse		%Oppure \beginverse* se non si vuole il numero di fianco
%\memorize 		% <<< DECOMMENTA se si vuole utilizzarne la funzione
\chordsoff		% <<< DECOMMENTA se vuoi una strofa senza accordi

Voglio Annunciare il dono crocifisso
di Cristo, il Dio con noi!
Perchè della morte lui si prende gioco,
Figlio che ci attira a sé!

\endverse


\endsong

\fi
%++++++++++++++++++++++++++++++++++++++++++++++++++++++++++++
%			FINE CANZONE TRASPOSTA
%++++++++++++++++++++++++++++++++++++++++++++++++++++++++++++
