%-------------------------------------------------------------
%			INIZIO	CANZONE
%-------------------------------------------------------------


%titolo: 	Cantico di liberazione
%autore: 	
%tonalita: 	La- 



%%%%%% TITOLO E IMPOSTAZONI
\beginsong{Cantico di liberazione}[by={}] 	% <<< MODIFICA TITOLO E AUTORE
\transpose{0} 						% <<< TRASPOSIZIONE #TONI (0 nullo)
\momenti{Pasqua}							% <<< INSERISCI MOMENTI	
% momenti vanno separati da ; e vanno scelti tra:
% Ingresso; Atto penitenziale; Acclamazione al Vangelo; Dopo il Vangelo; Offertorio; Comunione; Ringraziamento; Fine; Santi; Pasqua; Avvento; Natale; Quaresima; Canti Mariani; Battesimo; Prima Comunione; Cresima; Matrimonio; Meditazione;
\ifchorded
	%\textnote{Tonalità originale }	% <<< EV COMMENTI (tonalità originale/migliore)
\fi

%%%%%% INTRODUZIONE
\ifchorded
\vspace*{\versesep}
\textnote{Intro: \qquad \qquad  }%(\eighthnote 116) % <<  MODIFICA IL TEMPO
% Metronomo: \eighthnote (ottavo) \quarternote (quarto) \halfnote (due quarti)
\vspace*{-\versesep}
\beginverse*

\nolyrics

%---- Prima riga -----------------------------
\vspace*{-\versesep}
\[A-] \[G] \[E-] \[A-]	 % \[*D] per indicare le pennate, \rep{2} le ripetizioni

%---- Ogni riga successiva -------------------
%\vspace*{-\versesep}
%\[G] \[C]  \[D]	

%---- Ev Indicazioni -------------------------			
%\textnote{\textit{(Oppure tutta la strofa)} }	

\endverse
\fi

%%%%% STROFA
\beginverse		%Oppure \beginverse* se non si vuole il numero di fianco
\memorize 		% <<< DECOMMENTA se si vuole utilizzarne la funzione
%\chordsoff		% <<< DECOMMENTA se vuoi una strofa senza accordi

\[A-]Voglio cant\[G]are in on\[E-]ore di D\[A-]io
\[D-]perché mir\[C]abile \[A-]è la sua gl\[D-]oria
\[G-]amore, f\[C]orza e mio c\[A-]anto è il Sign\[D-]ore
\[G-]solo a lui d\[A]evo la m\[D-]ia salv\[Bb]ezza:
\[A]lo esalter\[F]ò, \[A-]è il D\[G]io di mio p\[A-]adre. \[A-]

\endverse

%%%%% STROFA
\beginverse		%Oppure \beginverse* se non si vuole il numero di fianco
%\memorize 		% <<< DECOMMENTA se si vuole utilizzarne la funzione
\chordsoff		% <<< DECOMMENTA se vuoi una strofa senza accordi

^Stettero imm^obili le ^acque div^ise
^per riscatt^are il tuo p^opolo, o D^io
^poi l'ira t^ua volò s^opra il mar R^osso,
^carri ed es^ercito d^i Fara^one,
^fior di guerri^eri som^ersero l'^onde. ^

\endverse
%%%%% STROFA
\beginverse		%Oppure \beginverse* se non si vuole il numero di fianco
%\memorize 		% <<< DECOMMENTA se si vuole utilizzarne la funzione
\chordsoff		% <<< DECOMMENTA se vuoi una strofa senza accordi

Disse il nemico: io l'inseguirò,
raggiungerò la mia preda Israele,
sguainerò la mia spada rovente,
dividerò il bottino dei vinti,
la mia mano li sterminerà.

\endverse

%%%%% STROFA
\beginverse		%Oppure \beginverse* se non si vuole il numero di fianco
%\memorize 		% <<< DECOMMENTA se si vuole utilizzarne la funzione
\chordsoff		% <<< DECOMMENTA se vuoi una strofa senza accordi

Ma l'ira tua soffiò sopra il mar Rosso,
acque immense copriron le schiere
si riversarono ad un tuo gesto
e ricoprirono carri e guerrieri
che come pietre raggiunsero il fondo.

\endverse

%%%%% STROFA
\beginverse		%Oppure \beginverse* se non si vuole il numero di fianco
%\memorize 		% <<< DECOMMENTA se si vuole utilizzarne la funzione
\chordsoff		% <<< DECOMMENTA se vuoi una strofa senza accordi

Chi è come te fra gli dei, Signore,
chi è come te maestoso e potente
che operasti un prodigio grandioso
la tua destra stendesti o Dio
il mare apristi a salvare i tuoi servi

\endverse

%%%%% STROFA
\beginverse		%Oppure \beginverse* se non si vuole il numero di fianco
%\memorize 		% <<< DECOMMENTA se si vuole utilizzarne la funzione
\chordsoff		% <<< DECOMMENTA se vuoi una strofa senza accordi

Questo tuo popolo, che hai riscattato,
ora lo guidi tu solo benigno
con forza e amore lo stai conducendo
alla tua santa dimora divina
che le tue mani, Signore, han fondato.

\endverse

\endsong
%------------------------------------------------------------
%			FINE CANZONE
%------------------------------------------------------------