%-------------------------------------------------------------
%			INIZIO	CANZONE
%-------------------------------------------------------------


%titolo: 	Venite Fedeli
%autore: 	Stefani, Wade
%tonalita: 	Fa 



%%%%%% TITOLO E IMPOSTAZONI
\beginsong{Venite fedeli}[by={Stefani, Wade}]	% <<< MODIFICA TITOLO E AUTORE
\transpose{0} 						% <<< TRASPOSIZIONE #TONI (0 nullo)
\momenti{Natale}							% <<< INSERISCI MOMENTI	
% momenti vanno separati da ; e vanno scelti tra:
% Ingresso; Atto penitenziale; Acclamazione al Vangelo; Dopo il Vangelo; Offertorio; Comunione; Ringraziamento; Fine; Santi; Pasqua; Avvento; Natale; Quaresima; Canti Mariani; Battesimo; Prima Comunione; Cresima; Matrimonio; Meditazione;
\ifchorded
	%\textnote{Tonalità originale }	% <<< EV COMMENTI (tonalità originale/migliore)
\fi



%%%%%% INTRODUZIONE
\ifchorded
\vspace*{\versesep}
\textnote{Intro: \qquad \qquad  }%(\eighthnote 116) % << MODIFICA IL TEMPO
% Metronomo: \eighthnote (ottavo) \quarternote (quarto) \halfnote (due quarti)
\vspace*{-\versesep}
\beginverse*

\nolyrics

%---- Prima riga -----------------------------
\vspace*{-\versesep}
\[(*F)] \[B&] \[G-] \[C]  \[*F]  \[*B&] \[F] \[C] \[F]	 % \[*D] per indicare le pennate, \rep{2} le ripetizioni

%---- Ogni riga successiva -------------------
%\vspace*{-\versesep}
%\[G] \[C]  \[D]	

%---- Ev Indicazioni -------------------------			
\textnote{\textit{(come la seconda riga del ritornello)} }	

\endverse
\fi



%%%%% STROFA
\beginverse
\memorize
Ve\[F]nite, fe\[C]deli, 
\[*F]l'an\[*C]ge\[*F]lo \[*B&]ci in\[F]vi\[C]ta, 
v\[D-]eni\[*C]te, \[*G]v\[*C]eni-\[(*D-)]i-\[*C]te 
\[*F]a Be\[C]tle-\[G7]em-\[C]me.
\endverse




%%%%% RITORNELLO
\beginchorus
\textnote{\textbf{Rit.}}
\[F]Na\[*B&]sce \[*F]per \[B&]no\[F]i 
\[*C]Cri\[*A-]sto \[*D-]Sal\[(*G-)]va\[*C]to-\[*G]o-\[C]re.
Ve\[F]nite, ado\[F]ria\[C]mo, \brk\[F]venite, \[*F]a\[*B&]dori\[F]a\[C]mo, 
v\[B&]enite, a\[(G-)]do\[C]ria\[*D-]mo \brk \[*B&]il Si\[F]gno\[C7]re Ge\[F]sù!
\endchorus




%%%%% STROFA
\beginverse
La ^luce del ^mo-o-ndo 
^bril^la in ^u^na ^grot^ta: 
la ^fe^de ^ci ^gui-^i-^da 
^a Be^tle-^em-^me.
\endverse


\beginverse
\chordsoff
La ^notte ri^splende, ^tut^to il ^mon^do at^ten^de; 
se^guia^mo i ^pa^^sto^ri ^a Be^^tle^me.
\endverse


\beginverse
\chordsoff
Il ^Figlio di ^Dio, ^Re ^dell'^u^ni^ver^so, 
si é ^fat^to ^bam^^bi^no ^a Be^^tlem^me.
\endverse


\beginverse
\chordsoff
«Sia ^gloria nei ^cieli, ^pa^ce ^sul^la ^ter^ra» 
un ^an^ge^lo an^^nun^cia ^a Be^^tlem^me.
\endverse



\endsong
%------------------------------------------------------------
%			FINE CANZONE
%------------------------------------------------------------


% %++++++++++++++++++++++++++++++++++++++++++++++++++++++++++++
% %			CANZONE TRASPOSTA
% %++++++++++++++++++++++++++++++++++++++++++++++++++++++++++++
% \ifchorded
% %decremento contatore per avere stesso numero
% \addtocounter{songnum}{-1} 
% \beginsong{Venite fedeli}[by={Stefani, Wade}]	% <<< COPIA TITOLO E AUTORE
% \transpose{-3} 						% <<< TRASPOSIZIONE #TONI + - (0 nullo)
% \ifchorded
% 	\textnote{Tonalità più facile da suonare}	% <<< EV COMMENTI (tonalità originale/migliore)
% \fi


% %%%%%% INTRODUZIONE
% \ifchorded
% \vspace*{\versesep}
% \textnote{Intro: \qquad \qquad  }%(\eighthnote 116) % << MODIFICA IL TEMPO
% % Metronomo: \eighthnote (ottavo) \quarternote (quarto) \halfnote (due quarti)
% \vspace*{-\versesep}
% \beginverse*

% \nolyrics

% %---- Prima riga -----------------------------
% \vspace*{-\versesep}
% \[(*F)] \[B&] \[G-] \[C]  \[*F]  \[*B&] \[F] \[C] \[F]	 % \[*D] per indicare le pennate, \rep{2} le ripetizioni

% %---- Ogni riga successiva -------------------
% %\vspace*{-\versesep}
% %\[G] \[C]  \[D]	

% %---- Ev Indicazioni -------------------------			
% \textnote{\textit{(come la seconda riga del ritornello)} }	

% \endverse
% \fi



% %%%%% STROFA
% \beginverse
% \memorize
% Ve\[F]nite, fe\[C]deli, 
% \[*F]l'an\[*C]ge\[*F]lo \[*B&]ci in\[F]vi\[C]ta, 
% v\[D-]eni\[*C]te, \[*G]v\[*C]eni-\[(*D-)]i-\[*C]te 
% \[*F]a Be\[C]tle-\[G7]em-\[C]me.
% \endverse




% %%%%% RITORNELLO
% \beginchorus
% \textnote{\textbf{Rit.}}
% \[F]Na\[*B&]sce \[*F]per \[B&]no\[F]i 
% \[*C]Cri\[*A-]sto \[*D-]Sal\[(*G-)]va\[*C]to-\[*G]o-\[C]re.
% Ve\[F]nite, ado\[F]ria\[C]mo, \brk\[F]venite, \[*F]a\[*B&]dori\[F]a\[C]mo, 
% v\[B&]enite, a\[(G-)]do\[C]ria\[*D-]mo \brk \[*B&]il Si\[F]gno\[C7]re Ge\[F]sù!
% \endchorus




% %%%%% STROFA
% \beginverse
% La ^luce del ^mo-o-ndo 
% ^bril^la in ^u^na ^grot^ta: 
% la ^fe^de ^ci ^gui-^i-^da 
% ^a Be^tle-^em-^me.
% \endverse


% \beginverse
% \chordsoff
% La ^notte ri^splende, ^tut^to il ^mon^do at^ten^de; 
% se^guia^mo i ^pa^^sto^ri ^a Be^^tle^me.
% \endverse


% \beginverse
% \chordsoff
% Il ^Figlio di ^Dio, ^Re ^dell'^u^ni^ver^so, 
% si é ^fat^to ^bam^^bi^no ^a Be^^tlem^me.
% \endverse


% \beginverse
% \chordsoff
% «Sia ^gloria nei ^cieli, ^pa^ce ^sul^la ^ter^ra» 
% un ^an^ge^lo an^^nun^cia ^a Be^^tlem^me.
% \endverse



% \endsong


% \fi
% %++++++++++++++++++++++++++++++++++++++++++++++++++++++++++++
% %			FINE CANZONE TRASPOSTA
% %++++++++++++++++++++++++++++++++++++++++++++++++++++++++++++
