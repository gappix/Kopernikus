%-------------------------------------------------------------
%			INIZIO	CANZONE
%-------------------------------------------------------------


%titolo: 	E sei rimasto qui
%autore: 	Gen\ Rosso
%tonalita: 	Fa 



%%%%%% TITOLO E IMPOSTAZONI
\beginsong{E sei rimasto qui}[by={Gen\ Rosso}]
\transpose{0} 						% <<< TRASPOSIZIONE #TONI (0 nullo)
%\preferflats  %SE VOGLIO FORZARE i bemolle come alterazioni
%\prefersharps %SE VOGLIO FORZARE i # come alterazioni
\momenti{Comunione; Cresima}							% <<< INSERISCI MOMENTI	
% momenti vanno separati da ; e vanno scelti tra:
% Ingresso; Atto penitenziale; Acclamazione al Vangelo; Dopo il Vangelo; Offertorio; Comunione; Ringraziamento; Fine; Santi; Pasqua; Avvento; Natale; Quaresima; Canti Mariani; Battesimo; Prima Comunione; Cresima; Matrimonio; Meditazione; Spezzare del pane;
\ifchorded
	%\textnote{Tonalità migliore }	% <<< EV COMMENTI (tonalità originale/migliore)
\fi



%%%%%% INTRODUZIONE
\ifchorded
\vspace*{\versesep}
\musicnote{
\begin{minipage}{0.48\textwidth}
\textbf{Intro}
\hfill 
%( \eighthnote \, 80)   % <<  MODIFICA IL TEMPO
% Metronomo: \eighthnote (ottavo) \quarternote (quarto) \halfnote (due quarti)
\end{minipage}
} 	
\vspace*{-\versesep}
\beginverse*


\nolyrics

%---- Prima riga -----------------------------
\vspace*{-\versesep}
 \[F] \[C] \[F] 	 % \[*D] per indicare le pennate, \rep{2} le ripetizioni

%---- Ogni riga successiva -------------------
\vspace*{-\versesep}
\[F] \[C] \[B&]	

%---- Ev Indicazioni -------------------------			
%\textnote{\textit{(Oppure tutta la strofa)} }	

\endverse
\fi



%%%%% STROFA
\beginverse		%Oppure \beginverse* se non si vuole il numero di fianco
\memorize 		% <<< DECOMMENTA se si vuole utilizzarne la funzione
%\chordsoff		% <<< DECOMMENTA se vuoi una strofa senza accordi
\[F] Perché la sete d'infi\[B&/F]nito? \brk \[G-] Perché la fame d'immor\[D-]tali\[C]tà?
\[F] Sei Tu che hai messo dentro \[B&/F]l'uomo  \brk \[G-] il desiderio dell'e\[F/C]terni\[C]tà!
Ma \[G-]Tu sapevi \[F/A]che quel vuoto \[B&]lo colmavi \[F/A]Tu,
per \[G-]questo sei ve\[F/A]nuto in mezzo a \[C]noi.
\endverse




%%%%% RITORNELLO
\beginchorus
\textnote{\textbf{Rit.}}
E \[F]sei rimasto qui, \[B&/F]visibile mistero.
E \[F]sei rimasto qui, \[D-]cuore del mondo in\[C]tero.
E \[B&]rimarrai con noi fin\[A-]ché quest'uni\[D-]verso gire\[G]rà.
Sal\[F]vezza dell'u\[C]mani\[F]tà. \[C] 
\endchorus





%%%%% STROFA
\beginverse		%Oppure \beginverse* se non si vuole il numero di fianco
%\memorize 		% <<< DECOMMENTA se si vuole utilizzarne la funzione
%\chordsoff		% <<< DECOMMENTA se vuoi una strofa senza accordi
\[D] Si apre il cielo del fu\[G/D]turo,   \brk \[E-] il muro della morte or\[B-]mai non \[A]c'è.
\[D]Tu, Pane vivo, ci fai \[G/B]Uno:\brk \[E-]  richiami tutti i figli at\[D]torno a \[A]Te.
E \[E-]doni il tuo \[D/F#]Spirito che \[G]lascia dentro \[D/F#]noi
il \[E-]germe della  \[D/F#]sua immortali\[A]tà.
\endverse






%%%%% RITORNELLO
\beginchorus
\textnote{\textbf{Rit.}}
\[D]Sei rimasto qui, \[G/D]visibile mistero.
\[D]Sei rimasto qui, \[B-]cuore del mondo in\[A]tero.
E \[G]rimarrai con noi fin\[F#-]ché quest'uni\[B-]verso gire\[E]rà.
Sal\[D]vezza dell'u\[A]mani\[D]tà. \[C] 
\endchorus




%%%%% STROFA
\beginverse		%Oppure \beginverse* se non si vuole il numero di fianco
%\memorize 		% <<< DECOMMENTA se si vuole utilizzarne la funzione
%\chordsoff		% <<< DECOMMENTA se vuoi una strofa senza accordi
^ Presenza vera nel mi^stero, \brk ^  ma più reale di ogni ^realtà, ^
^ da te ogni cosa prende ^vita  \brk ^ e tutto un giorno a te ri^torne^rà.
Var^cando l'infi^nito tutti ^troveremo in ^Te
un ^Sole immenso ^di felici\[G-]tà. \[A-]\[B&]\[C]
\endverse




%%%%% STROFA
\beginverse*		%Oppure \beginverse* se non si vuole il numero di fianco
%\memorize 		% <<< DECOMMENTA se si vuole utilizzarne la funzione
%\chordsoff		% <<< DECOMMENTA se vuoi una strofa senza accordi
\[F/A]Noi,  \[B&/D]trasformati in \[C]Te, sa\[F]remo il seme \[B&]che
fa\[G-]rà fiorire l'\[F/A]universo \[B&]nella Trini\[C]tà.
\[F/A]Noi,  \[B&/D]trasformati in \[C]Te, sa\[F]remo il seme \[B&]che
fa\[G-]rà fiorire \[F/A]tutto l'uni\[B&]verso insieme a \[C]Te.
\endverse


%%%%% RITORNELLO
\beginchorus
\textnote{\textbf{Rit.}}
E \[G]sei rimasto qui, vi\[C]sibile mistero.
\[G]Sei rimasto qui, \[E-]cuore del mondo in\[D]tero.
E \[C]rimarrai con noi \brk fin\[B-]ché quest'uni\[E-]verso gire\[A]rà.  \[D] 
\[G]Sei rimasto qui, vi\[C]sibile mistero.
\[G]Sei rimasto qui, \[E-]cuore del mondo in\[D]tero.
E \[C]rimarrai con noi fin\[B-]ché quest'uni\[E-]verso gire\[A]rà.
\[G/D]Ieri oggi e sempre. \[A-] \[G/B] 
\[C]Sal\[C7+/D]vezza dell'u\[D]manità. \[G] 

\endchorus



%%%%%% EV. CHIUSURA SOLO STRUMENTALE
\ifchorded
\beginchorus %oppure \beginverse*
\vspace*{1.3\versesep}
\textnote{Chiusura strumentale} %<<< EV. INDICAZIONI

\[D] \[G] \[G/B] \[D] \[G] \[C] \[G] 

\endchorus  %oppure \endverse
\fi


\endsong
%------------------------------------------------------------
%			FINE CANZONE
%------------------------------------------------------------



