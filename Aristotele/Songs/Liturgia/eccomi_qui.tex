%titolo: 	Eccomi qui
%autore: 	Diliberto Monti
%tonalita: 	Do 



%%%%%% TITOLO E IMPOSTAZONI
\beginsong{Eccomi qui}[by={D. Monti}] 	% <<< MODIFICA TITOLO E AUTORE
\transpose{0} 						% <<< TRASPOSIZIONE #TONI (0 nullo)
\momenti{Ingresso; Dopo il Vangelo}							% <<< INSERISCI MOMENTI	
% momenti vanno separati da ; e vanno scelti tra:
% Ingresso; Atto penitenziale; Acclamazione al Vangelo; Dopo il Vangelo; Offertorio; Comunione; Ringraziamento; Fine; Santi; Pasqua; Avvento; Natale; Quaresima; Canti Mariani; Battesimo; Prima Comunione; Cresima; Matrimonio; Meditazione;
\ifchorded
	%\textnote{Tonalità originale }	% <<< EV COMMENTI (tonalità originale/migliore)
\fi




%%%%%% INTRODUZIONE
\ifchorded
\vspace*{\versesep}
\musicnote{
\begin{minipage}{0.48\textwidth}
\textbf{Intro}
\hfill 
%( \eighthnote \, 80)   % <<  MODIFICA IL TEMPO
% Metronomo: \eighthnote (ottavo) \quarternote (quarto) \halfnote (due quarti)
\end{minipage}
} 	
\vspace*{-\versesep}
\beginverse*
\nolyrics

%---- Prima riga -----------------------------
\vspace*{-\versesep}
\[C*] \[E7*] \[A-] 	 % \[*D] per indicare le pennate, \rep{2} le ripetizioni

%---- Ogni riga successiva -------------------
\vspace*{-\versesep}
\[F] \[G] \[G] 

%---- Ev Indicazioni -------------------------			
%\textnote{\textit{(Oppure tutta la strofa)} }	

\endverse
\fi


%%%%% RITORNELLO
\beginchorus
\textnote{\textbf{Rit.}}

\[C*] E\[E7*]ccomi \[A-]qui, \brk  \[F] di nuovo a \[G]te Signore, 
\[C*] e\[E7*]ccomi \[A-]qui: \brk  \[F] accetta \[G]la mia vita;
\[C] non dire \[G]no  \brk a \[F]chi si affi da a \[C]te,
\[A-] mi accoglie\[D-]rai \brk \[F] per sempre \[G]nel tuo amore.\[F]\[C]

\endchorus



%%%%% STROFA
\beginverse		%Oppure \beginverse* se non si vuole il numero di fianco
\memorize 		% <<< DECOMMENTA se si vuole utilizzarne la funzione
%\chordsoff		& <<< DECOMMENTA se vuoi una strofa senza accordi

Q\[C]uando hai scelto di \[G]vivere quag\[A-]giù, \[A-7]
q\[F]uando hai voluto che \[G]fossimo figli t\[C]uoi, \[C7]
\[F] ti sei do\[C]nato ad \[D-]una come \[A-]noi 
\[A-7] e hai cammi\[D-]nato sulle st\[F]rade dell'\[G]uomo.

\endverse

%%%%% STROFA
\beginverse		%Oppure \beginverse* se non si vuole il numero di fianco
%\memorize 		% <<< DECOMMENTA se si vuole utilizzarne la funzione
%\chordsoff		% <<< DECOMMENTA se vuoi una strofa senza accordi

P^rima che il Padre ti ^richiamasse a ^sé ^
p^rima del buio che il tuo g^rido spezze^rà ^
^ tu hai pro^messo di ^non lasciarci p^iù
^ di accompag^narci sulle st^rade del ^mondo.

\endverse

%%%%% STROFA
\beginverse		%Oppure \beginverse* se non si vuole il numero di fianco
%\memorize 		% <<< DECOMMENTA se si vuole utilizzarne la funzione
%\chordsoff		% <<< DECOMMENTA se vuoi una strofa senza accordi

^Ora ti prego con^ducimi con ^te ^
^nella fatica di ser^vir la veri^tà ^
^ sarò vi^cino a ^chi ti invoche^rà
^ e mi guide^rai sulle st^rade dell'^uomo.

\endverse

\endsong
%------------------------------------------------------------
%			FINE CANZONE
%------------------------------------------------------------
