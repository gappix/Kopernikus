%-------------------------------------------------------------
%			INIZIO	CANZONE
%-------------------------------------------------------------


%titolo: 	È più bello insieme
%autore: 	Gen Verde
%tonalita: 	Re 



%%%%%% TITOLO E IMPOSTAZONI
\beginsong{E' più bello insieme}[by={Gen Verde}] 	% <<< MODIFICA TITOLO E AUTORE
\transpose{0} 						% <<< TRASPOSIZIONE #TONI (0 nullo)
\momenti{Fine}							% <<< INSERISCI MOMENTI	
% momenti vanno separati da ; e vanno scelti tra:
% Ingresso; Atto penitenziale; Acclamazione al Vangelo; Dopo il Vangelo; Offertorio; Comunione; Ringraziamento; Fine; Santi; Pasqua; Avvento; Natale; Quaresima; Canti Mariani; Battesimo; Prima Comunione; Cresima; Matrimonio; Meditazione; Spezzare del pane;
\ifchorded
	%\textnote{Tonalità originale }	% <<< EV COMMENTI (tonalità originale/migliore)
\fi


%%%%%% INTRODUZIONE
\ifchorded
\vspace*{\versesep}
\musicnote{
\begin{minipage}{0.48\textwidth}
\textbf{Intro}
\hfill 
%( \eighthnote \, 80)   % <<  MODIFICA IL TEMPO
% Metronomo: \eighthnote (ottavo) \quarternote (quarto) \halfnote (due quarti)
\end{minipage}
} 	
\vspace*{-\versesep}
\beginverse*


\nolyrics

%---- Prima riga -----------------------------
\vspace*{-\versesep}
\[D] \[A] \[G] \[D]	 % \[*D] per indicare le pennate, \rep{2} le ripetizioni

%---- Ogni riga successiva -------------------
\vspace*{-\versesep}
\[G] \[D]\[E-]\[A7]

%---- Ev Indicazioni -------------------------			
%\textnote{\textit{(%Oppure tutta la strofa)} }	

\endverse
\fi



%%%%% STROFA
\beginverse		%Oppure \beginverse* se non si vuole il numero di fianco
\memorize 		% <<< DECOMMENTA se si vuole utilizzarne la funzione
%\chordsoff		% <<< DECOMMENTA se vuoi una strofa senza accordi
\[D] Dietro i volti \[A]sconosciuti 
\[G] della gente \[D4]che mi \[D]sfiora, 
\[G] quanta vita, \[D]quante attese 
\[E-]di felici\[A4]tà, \[A]
\[D] quanti atti\[A]mi vissuti, 
\[G] mondi da sco\[D4]prire an\[D]cora, 
\[G] splendidi uni\[D]versi accanto a \[A4]me. \[A7]
\endverse


%%%%% RITORNELLO
\beginchorus
\textnote{\textbf{Rit.}}
\[G]È più \[A]bello insieme
è un \[B-]dono grande l'\[7]altra gente,
\[G]è più \[A]bello insieme. \[B-7] \rep{2}
\endchorus


%%%%% STROFA
\beginverse		%Oppure \beginverse* se non si vuole il numero di fianco
%\memorize 		% <<< DECOMMENTA se si vuole utilizzarne la funzione
%\chordsoff		% <<< DECOMMENTA se vuoi una strofa senza accordi
^ E raccolgo ^nel mio cuore 
^ la speranza ed ^il do^lore 
^ il silenzio, il ^pianto 
della ^gente attorno a ^me. ^
^ In quel pianto, in ^quel sorriso 
^ è il mio pianto, il ^mio sor^riso 
^ chi mi vive ac^canto è un altro ^me. ^
\endverse


%%%%% RITORNELLO
\beginchorus
\textnote{\textbf{Rit.}}
\[G]È più \[A]bello insieme
è un \[B-]dono grande l'\[7]altra gente,
\[G]è più \[A]bello insieme. \[B-7] \rep{2}
\endchorus



%%%%%% EV. INTERMEZZO
\beginverse*
\vspace*{1.3\versesep}
{
	\nolyrics
	\textnote{[cambio di tonalità]}
	
	\ifchorded

	%---- Prima riga -----------------------------
	\vspace*{-\versesep}
	\[B-7]

	%---- Ogni riga successiva -------------------
	\vspace*{-\versesep}
	\[A*] \[G*]  \[B] \[B]
	\fi
	%---- Ev Indicazioni -------------------------			
	%\textnote{\textit{(ripetizione della strofa)}} 
	 
}
\vspace*{\versesep}
\endverse



%%%%% STROFA
\beginverse		%Oppure \beginverse* se non si vuole il numero di fianco
%\memorize 		% <<< DECOMMENTA se si vuole utilizzarne la funzione
%\chordsoff		% <<< DECOMMENTA se vuoi una strofa senza accordi

\transpose{2}
^ Fra le case e i ^grattacieli, 
^ fra le antenne ^lassù in ^alto, 
^ così traspa^rente il cielo 
^non l'ho visto ^mai. ^
^ E la luce ^getta veli 
^ di colore ^sull'a^sfalto 
^ ora che can^tate assieme a ^me. ^
\endverse


%%%%% RITORNELLO
\beginchorus
\textnote{\textbf{Rit.}}
\transpose{2}
\[G]È più \[A]bello insieme
è un \[B-]dono grande l'\[7]altra gente,
\[G]è più \[A]bello insieme. \[B-7] \rep{6} 

\endchorus


%%%%%% EV. FINALE
\ifchorded
\beginchorus %oppure \beginverse*
\vspace*{1.3\versesep}
\textnote{\textbf{Finale} } %<<< EV. INDICAZIONI
\transpose{2}
\[G]è più \[A]bello insieme. \[D]

\endchorus  %oppure \endverse
\fi

\endsong
%------------------------------------------------------------
%			FINE CANZONE
%------------------------------------------------------------


