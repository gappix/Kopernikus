%-------------------------------------------------------------
%			INIZIO	CANZONE
%-------------------------------------------------------------


%titolo: 	Ti ringrazio mi Signore
%autore: 	P. Sequeri
%tonalita: 	Re



%%%%%% TITOLO E IMPOSTAZONI
\beginsong{Ti ringrazio mio Signore}[by={P. Sequeri}] 	% <<< MODIFICA TITOLO E AUTORE
\transpose{0} 						% <<< TRASPOSIZIONE #TONI (0 nullo)
%\preferflats  %SE VOGLIO FORZARE i bemolle come alterazioni
%\prefersharps %SE VOGLIO FORZARE i # come alterazioni
\momenti{Congedo; Prima Comunione}							% <<< INSERISCI MOMENTI	
% momenti vanno separati da ; e vanno scelti tra:
% Ingresso; Atto penitenziale; Acclamazione al Vangelo; Dopo il Vangelo; Offertorio; 
%Comunione; Ringraziamento; Fine; Santi; Pasqua; Avvento; Natale; Quaresima; Canti Mariani; 
%Battesimo; Prima Comunione; Cresima; Matrimonio; Meditazione; Spezzare del pane;
\ifchorded
	%\textnote{Tonalità migliore }	% <<< EV COMMENTI (tonalità originale/migliore)
\fi

%%%%%% INTRODUZIONE
\ifchorded
\vspace*{\versesep}
\musicnote{
\begin{minipage}{0.48\textwidth}
\textbf{Intro}
\hfill 
( \quarternote \, 120)   % <<  MODIFICA IL TEMPO
% Metronomo: \eighthnote (ottavo) \quarternote (quarto) \halfnote (due quarti)
\end{minipage}
} 	
\vspace*{-\versesep}
\beginverse*

\nolyrics

%---- Prima riga -----------------------------
\vspace*{-\versesep}
\[D] \[G] \[A7]	 \rep{2}% \[*D] per indicare le pennate, \rep{2} le ripetizioni

%---- Ogni riga successiva -------------------
%\vspace*{-\versesep}
%\[G] \[C]  \[D]	

%---- Ev Indicazioni -------------------------			
%\textnote{\textit{(Oppure tutta la strofa)} }	

\endverse
\fi

%%%%% STROFA
\beginverse		%Oppure \beginverse* se non si vuole il numero di fianco
\memorize 		% <<< DECOMMENTA se si vuole utilizzarne la funzione
%\chordsoff		% <<< DECOMMENTA se vuoi una strofa senza accordi

A\[D]matevi l’un l’\[B-]altro
come \[E-]Lui ha amato \[A]noi:
e si\[D]ate per \[B-]sempre suoi a\[G]mi\[D]ci;
e \[E-]quello che fa\[A]rete
al più \[F#]piccolo tra \[B-]voi,
cre\[E-]dete, l’a\[A]vete fatto a \[D*]Lui. \[G*] \[D*]

\endverse

%%%%% RITORNELLO
\beginchorus
\textnote{\textbf{Rit.}}

Ti ring\[D]razio \[B-]mio Sig\[G]nore \brk e \[A7]non ho più pa\[D]ura, \[B-]
per\[G]ché, \[A7]con la mia \[D]mano
\[B-]nella \[G]mano \[A7]degli amici \[D7]miei, \[B-]
cam\[G]mino tra la \[A7]gente della \[D*]mia \[F#*]cit\[B-]tà
e \[G]non mi \[A]sento più \[D7]solo; \[B-]
non \[G]sento la stan\[A7]chezza
e guardo \[D*]dritto \[F#*]avanti a \[B-]me,
per\[G]ché sulla mia \[A7]strada ci sei \[D*]tu. \[G*] \[D*]

\endchorus

%%%%% STROFA
\beginverse		%Oppure \beginverse* se non si vuole il numero di fianco
%\memorize 		% <<< DECOMMENTA se si vuole utilizzarne la funzione
%\chordsoff		% <<< DECOMMENTA se vuoi una strofa senza accordi

Se a^mate vera^mente perdo^natevi tra ^voi,
nel ^cuore di og^nuno ci sia ^pa^ce,
il ^Padre che dai ^cieli vede ^tutti i figli ^suoi
con ^gioia a ^voi perdone^rà. ^ ^

\endverse

%%%%% STROFA
\beginverse		%Oppure \beginverse* se non si vuole il numero di fianco
%\memorize 		% <<< DECOMMENTA se si vuole utilizzarne la funzione
%\chordsoff		% <<< DECOMMENTA se vuoi una strofa senza accordi

Sa^rete suoi a^mici se vi ^amate tra di ^voi
e ^questo è ^tutto il suo Van^ge^lo
l’ a^more non ha ^prezzo, non mi^sura ciò che ^dà
l’a^more con^fini non ne ^ha. ^ ^

\endverse

\endsong
%------------------------------------------------------------
%			FINE CANZONE
%------------------------------------------------------------