%-------------------------------------------------------------
%			INIZIO	CANZONE
%-------------------------------------------------------------


%titolo: 	Angelo di Dio
%autore: 	G. Mezzalira
%tonalita: 	Mib



%%%%%% TITOLO E IMPOSTAZONI
\beginsong{Angelo di Dio}[by={G. Mezzalira}]
\transpose{-3} 						% <<< TRASPOSIZIONE #TONI (0 nullo)
%\preferflats  %SE VOGLIO FORZARE i bemolle come alterazioni
%\prefersharps %SE VOGLIO FORZARE i # come alterazioni
\momenti{Ringraziamento; Meditazione}							% <<< INSERISCI MOMENTI	
% momenti vanno separati da ; e vanno scelti tra:
% Ingresso; Atto penitenziale; Acclamazione al Vangelo; Dopo il Vangelo; Offertorio; Comunione; Ringraziamento; 
% Fine; Santi; Pasqua; Avvento; Natale; Quaresima; Canti Mariani; Battesimo; Prima Comunione; Cresima; Matrimonio; Meditazione; Spezzare del pane;
\ifchorded
	%\textnote{Tonalità migliore }	% <<< EV COMMENTI (tonalità originale\migliore)
\fi


%%%%%% INTRODUZIONE
\ifchorded
\vspace*{\versesep}
\musicnote{
\begin{minipage}{0.48\textwidth}
\textbf{Intro}
\hfill 
%( \eighthnote \, 80)   % <<  MODIFICA IL TEMPO
% Metronomo: \eighthnote (ottavo) \quarternote (quarto) \halfnote (due quarti)
\end{minipage}
} 	
\vspace*{-\versesep}
\beginverse*

\nolyrics

%---- Prima riga -----------------------------
\vspace*{-\versesep}
\[E&] \[A&] \[B&7]	\[E&] % \[*D] per indicare le pennate, \rep{2} le ripetizioni

%---- Ogni riga successiva -------------------
%\vspace*{-\versesep}
%\[G] \[C]  \[D]	

%---- Ev Indicazioni -------------------------			
%\textnote{\textit{(Oppure tutta la strofa)} }	

\endverse
\fi


%%%%% RITORNELLO
\beginchorus
\textnote{\textbf{Rit.}}
\[E&]Angelo di \[A&]Dio, \brk che \[B&7]sei il mio cus\[G-]tode,
il\[C-]lumina custo\[F7]disci \brk \[B&]reggi-e governa \[E&]me 
che ti \[A&]fui affi\[Ddim/B&]dato 
\[G-]dalla pietà ce\[C-]leste 
\[E&]A-a-am\[A&]en \[C7*] \[F-]A-a-\[B&*]a\[E&]men 
\endchorus




%%%%% STROFA
\beginverse		%Oppure \beginverse* se non si vuole il numero di fianco
\memorize 		% <<< DECOMMENTA se si vuole utilizzarne la funzione
%\chordsoff		% <<< DECOMMENTA se vuoi una strofa senza accordi

\[C]Guida i miei \[F]pensieri, \[D-]segui i miei \[G]passi,
\[C]dammi la tua \[A-]forza per \[G]restare nella \[C]luce
\[C]la gioia \[F]incontrerò \[G]la pace \[C]porterò
\[A-]Angelo \[F]Santo \[C]veglia \[G]su di \[C]me.


\endverse





%%%%% STROFA
\beginverse		%Oppure \beginverse* se non si vuole il numero di fianco
%\memorize 		% <<< DECOMMENTA se si vuole utilizzarne la funzione
%\chordsoff		% <<< DECOMMENTA se vuoi una strofa senza accordi
^Guida la mia ^vita, ^segui il mio la^voro,
^dammi la tua ^forza per dif^fondere l’a^more
^Gesù io ^annunzierò, ^paura ^non avrò 
^Angelo ^Santo ^veglia ^su di ^me.
\endverse





\endsong
%------------------------------------------------------------
%			FINE CANZONE
%------------------------------------------------------------



