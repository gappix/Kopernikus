%-------------------------------------------------------------
%			INIZIO	CANZONE
%-------------------------------------------------------------


%titolo: 	Voi siete di Dio
%autore: 	Balduzzi, Casucci, Savelli 
%tonalita: 	Sol 



%%%%%% TITOLO E IMPOSTAZONI
\beginsong{Voi siete di Dio}[by={M. Balduzzi, C. Casucci, W. Savelli}] 	% <<< MODIFICA TITOLO E AUTORE
\transpose{0} 						% <<< TRASPOSIZIONE #TONI (0 nullo)
\momenti{Ingresso; Comunione; Ringraziamento; Meditazione}							% <<< INSERISCI MOMENTI	
% momenti vanno separati da ; e vanno scelti tra:
% Ingresso; Atto penitenziale; Acclamazione al Vangelo; Dopo il Vangelo; Offertorio; Comunione; Ringraziamento; Fine; Santi; Pasqua; Avvento; Natale; Quaresima; Canti Mariani; Battesimo; Prima Comunione; Cresima; Matrimonio; Meditazione; Spezzare del pane;
\ifchorded
	%\textnote{Tonalità migliore }	% <<< EV COMMENTI (tonalità originale/migliore)
\fi


%%%%%% INTRODUZIONE
\ifchorded
\vspace*{\versesep}
\musicnote{
\begin{minipage}{0.48\textwidth}
\textbf{Intro}
\hfill 
%( \eighthnote \, 80)   % <<  MODIFICA IL TEMPO
% Metronomo: \eighthnote (ottavo) \quarternote (quarto) \halfnote (due quarti)
\end{minipage}
} 	
\vspace*{-\versesep}
\beginverse*
\nolyrics

%---- Prima riga -----------------------------
\vspace*{-\versesep}
\[G] \[G] \[D] \[G] \rep{2}	 % \[*D] per indicare le pennate, \rep{2} le ripetizioni

%---- Ogni riga successiva -------------------
%\vspace*{-\versesep}
%\[G] \[C]  \[D]	

%---- Ev Indicazioni -------------------------			
\textnote{\textit{(oppure tutta la strofa)} }	

\endverse
\fi




%%%%% STROFA
\beginverse		%Oppure \beginverse* se non si vuole il numero di fianco
\memorize 		% <<< DECOMMENTA se si vuole utilizzarne la funzione
%\chordsoff		% <<< DECOMMENTA se vuoi una strofa senza accordi

\[G] Tutte le stelle della \[D]not\[G]te 
\[G] le nebulose e le co\[D]me\[E-]te 
\[G] il sole su una ragna\[D]te\[G]la 
\textbf{\[C] è tutto vostro e voi \[G]sie\[D]te di \[G]Dio.} 

\endverse



%%%%% STROFA
\beginverse		%Oppure \beginverse* se non si vuole il numero di fianco
%\memorize 		% <<< DECOMMENTA se si vuole utilizzarne la funzione
%\chordsoff		% <<< DECOMMENTA se vuoi una strofa senza accordi
\transpose{-3} 
^ Tutte le rose della ^vi^ta 
^ il grano, i prati, i fili d’^er^ba 
^il mare, i fiumi, le mon^ta^gne 
\textbf{\[C] è tutto vostro e voi \[E-]sie\[D]te di \[C]Dio.} 

\endverse




%%%%% STROFA
\beginverse		%Oppure \beginverse* se non si vuole il numero di fianco
%\memorize 		% <<< DECOMMENTA se si vuole utilizzarne la funzione
%\chordsoff		% <<< DECOMMENTA se vuoi una strofa senza accordi

^ Tutte le musiche e le ^dan^ze, 
^ i grattacieli, le astro^na^vi 
^ i quadri, i libri, le cul^tu^re
\textbf{\[C] è tutto vostro e voi \[G]sie\[D]te di \[G]Dio.}  


\endverse




%%%%% STROFA
\beginverse		%Oppure \beginverse* se non si vuole il numero di fianco
%\memorize 		% <<< DECOMMENTA se si vuole utilizzarne la funzione
%\chordsoff		% <<< DECOMMENTA se vuoi una strofa senza accordi
\transpose{-3} 
^ Tutte le volte che per^do^no 
^ quando sorrido, quando ^pian^go 
^ quando mi accorgo di chi ^so^no 
\textbf{\[C] è tutto vostro e voi \[E-]sie\[D]te di \[C]Dio.} 

\endverse






%%%%%% EV. FINALE

\beginchorus %oppure \beginverse*
\transpose{-3} 
\[C]E’ tutto nostro e noi \[G]sia\[D]mo di \[G]Dio. 


\endchorus  %oppure \endverse




\endsong
%------------------------------------------------------------
%			FINE CANZONE
%------------------------------------------------------------


