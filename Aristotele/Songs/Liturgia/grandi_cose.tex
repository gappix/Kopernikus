%-------------------------------------------------------------
%			INIZIO	CANZONE
%-------------------------------------------------------------


%titolo: 	Grandi cose
%autore: 	Gen Rosso
%tonalita: 	Do 



%%%%%% TITOLO E IMPOSTAZONI
\beginsong{Grandi cose}[by={Gen\ Rosso}]% <<< MODIFICA TITOLO E AUTORE
\transpose{0} 						% <<< TRASPOSIZIONE #TONI (0 nullo)
\momenti{Ingresso;  Dopo il Vangelo; Fine}							% <<< INSERISCI MOMENTI	
% momenti vanno separati da ; e vanno scelti tra:
% Ingresso; Atto penitenziale; Acclamazione al Vangelo; Dopo il Vangelo; Offertorio; Comunione; Ringraziamento; Fine; Santi; Pasqua; Avvento; Natale; Quaresima; Canti Mariani; Battesimo; Prima Comunione; Cresima; Matrimonio; Meditazione; Spezzare del pane;
\ifchorded
	%\textnote{Tonalità migliore }	% <<< EV COMMENTI (tonalità originale/migliore)
\fi


%%%%%% INTRODUZIONE
\ifchorded
\vspace*{\versesep}
\textnote{Intro: \qquad \qquad  }%(\eighthnote 116) % <<  MODIFICA IL TEMPO
% Metronomo: \eighthnote (ottavo) \quarternote (quarto) \halfnote (due quarti)
\vspace*{-\versesep}
\beginverse*

\nolyrics

%---- Prima riga -----------------------------
\vspace*{-\versesep}
\[C] \[G] \[C] \[G]  % \[*D] per indicare le pennate, \rep{2} le ripetizioni
 
%---- Ogni riga successiva -------------------
%\vspace*{-\versesep}
%\[G] \[C]  \[D]	

%---- Ev Indicazioni -------------------------			
%\textnote{\textit{(Oppure tutta la strofa)} }	

\endverse
\fi




\beginchorus
\[C]Grandi \[G]cose ha fatto \[D-]il Si\[A-]gnore per noi,
\[C] ha fatto \[F]germogliare i \[C]fiori tra le \[G4]rocce. \[G]
\[C]Grandi \[G]cose ha fatto \[D-]il Si\[A-]gnore per noi,
\[C] ci ha ripor\[F]tati liberi \[C]alla nostra \[G4]terra.
Ed \[A-]ora possiamo can\[D-]tare, possiamo gri\[E-]dare
l'amore che \[F]Dio ha versato su \[G4]noi. \[G]
\endchorus



\beginverse*
\[C]Tu che \[G]sai strap\[D-]pare dalla \[A-]morte, 
\[C] hai solle\[F]vato il nostro \[C]viso dalla \[G]polvere.
\[C]Tu che \[G]hai sen\[D-]tito il nostro \[A-]pianto, 
\[C] nel nostro \[F]cuore hai messo un \[C]seme di \[G]felicità.
\endverse


%%%%%% EV. FINALE
\ifchorded
\beginchorus %oppure \beginverse*
\vspace*{1.3\versesep}
\textnote{Chiusura} %<<< EV. INDICAZIONI

\[C*]

\endchorus  %oppure \endverse
\fi

\endsong
%------------------------------------------------------------
%			FINE CANZONE
%------------------------------------------------------------

