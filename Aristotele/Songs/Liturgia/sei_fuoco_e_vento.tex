%-------------------------------------------------------------
%			INIZIO	CANZONE
%-------------------------------------------------------------


%titolo: 	Sei fuoco e vento
%autore: 	Andrea Testa
%tonalita: 	Do



%%%%%% TITOLO E IMPOSTAZONI
\beginsong{Sei fuoco e vento}[by={Andrea Testa}] 	% <<< MODIFICA TITOLO E AUTORE
\transpose{0} 						% <<< TRASPOSIZIONE #TONI (0 nullo)
\momenti{Ingresso; Comunione}							% <<< INSERISCI MOMENTI	
% momenti vanno separati da ; e vanno scelti tra:
% Ingresso; Atto penitenziale; Acclamazione al Vangelo; Dopo il Vangelo; Offertorio; Comunione; Ringraziamento; Fine; Santi; Pasqua; Avvento; Natale; Quaresima; Canti Mariani; Battesimo; Prima Comunione; Cresima; Matrimonio; Meditazione; Spezzare del pane;
\ifchorded
	%\textnote{Tonalità originale }	% <<< EV COMMENTI (tonalità originale/migliore)
\fi


%%%%%% INTRODUZIONE
\ifchorded
\vspace*{\versesep}
\textnote{Intro: \qquad \qquad  }%(\eighthnote 116) % <<  MODIFICA IL TEMPO
% Metronomo: \eighthnote (ottavo) \quarternote (quarto) \halfnote (due quarti)
\vspace*{-\versesep}
\beginverse*

\nolyrics

%---- Prima riga -----------------------------
\vspace*{-\versesep}
\[A-] \[C] \[G] \[D-] \[F] \[A-] % \[*D] per indicare le pennate, \rep{2} le ripetizioni

%---- Ogni riga successiva -------------------
%\vspace*{-\versesep}
%\[G] \[C]  \[D]	

%---- Ev Indicazioni -------------------------			
%\textnote{\textit{(Oppure tutta la strofa)} }	

\endverse
\fi

%%%%% STROFA
\beginverse		%Oppure \beginverse* se non si vuole il numero di fianco
\memorize 		% <<< DECOMMENTA se si vuole utilizzarne la funzione
%\chordsoff		% <<< DECOMMENTA se vuoi una strofa senza accordi

In un \[A-]mare calmo e immobile, con un \[C]cielo senza nuvole,
non si \[G]riesce a navi\[D-]gare, prose\[F]guire non si \[A-]può.
Una \[A-]brezza lieve e debole, poi di\[C]venta un vento a raffiche,
soffia \[G]forte sulle \[D-]barche e ci \[F]spinge via di \[A-]qua.
Come il \[C]vento da la \[G]forza per viag\[A-]giare in un o\[E-]ceano
così \[F]Tu ci dai lo \[C]Spirito che ci \[D-]guiderà da \[G]Te.

\endverse

%%%%% RITORNELLO
\beginchorus
\textnote{\textbf{Rit.} \rep{2}}

Sei come \[C]vento che \[F]gonfia le \[C]vele,
sei come fuoco che ac\[F]cende l'a\[G]more, \[E]
sei come l'\[A-]aria che \[E-]si respira \[F]libera
chiara \[C]luce che \[G]il cammino \[F]indica.

\endchorus

%%%%% STROFA
\beginverse		%Oppure \beginverse* se non si vuole il numero di fianco
%\memorize 		% <<< DECOMMENTA se si vuole utilizzarne la funzione
\chordsoff		% <<< DECOMMENTA se vuoi una strofa senza accordi

Nella ^notte impenetrabile, ogni ^cosa è irraggiungibile,
non puoi ^scegliere la ^strada se non ^vedi avanti a ^te.
Una ^luce fioca e debole, sembra ^sorgere e poi crescere,
come ^fiamma che ^rigenera e che il^lumina la ìvia.
Come il ^fuoco scioglie il ^gelo e rischi^ara ogni sen^tiero
Così ^Tu riscaldi il ^cuore di chi ^Verbo annunce^rà.

\endverse

\endsong
%------------------------------------------------------------
%			FINE CANZONE
%------------------------------------------------------------