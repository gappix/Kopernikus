%-------------------------------------------------------------
%			INIZIO	CANZONE
%-------------------------------------------------------------


%titolo: 	Sei fuoco e vento
%autore: 	Andrea Testa
%tonalita: 	Do



%%%%%% TITOLO E IMPOSTAZONI
\beginsong{Sei fuoco e vento}[by={A. Testa}] 	% <<< MODIFICA TITOLO E AUTORE
\transpose{0} 						% <<< TRASPOSIZIONE #TONI (0 nullo)
\momenti{Ingresso; Comunione}							% <<< INSERISCI MOMENTI	
% momenti vanno separati da ; e vanno scelti tra:
% Ingresso; Atto penitenziale; Acclamazione al Vangelo; Dopo il Vangelo; Offertorio; Comunione; Ringraziamento; Fine; Santi; Pasqua; Avvento; Natale; Quaresima; Canti Mariani; Battesimo; Prima Comunione; Cresima; Matrimonio; Meditazione; Spezzare del pane;
\ifchorded
	%\textnote{Tonalità originale }	% <<< EV COMMENTI (tonalità originale/migliore)
\fi


%%%%%% INTRODUZIONE
\ifchorded
\vspace*{\versesep}
\textnote{Intro: \qquad \qquad  }%(\eighthnote 116) % <<  MODIFICA IL TEMPO
% Metronomo: \eighthnote (ottavo) \quarternote (quarto) \halfnote (due quarti)
\vspace*{-\versesep}
\beginverse*

\nolyrics

%---- Prima riga -----------------------------
\vspace*{-\versesep}
\[A-]  % \[*D] per indicare le pennate, \rep{2} le ripetizioni

%---- Ogni riga successiva -------------------
%\vspace*{-\versesep}
%\[G] \[C]  \[D]	

%---- Ev Indicazioni -------------------------			
%\textnote{\textit{(Oppure tutta la strofa)} }	

\endverse
\fi



\beginverse*

\textnote{Recitato}	
\musicnote{\textit{(Accompagnamento con gli accordi del ritornello)}}
\chordsoff
\textit{All'improvviso si sentì un rumore dal cielo 
come quando tira un forte vento
e riempì tutta la casa  dove si trovavano.
Allora videro qualcosa di simile \brk a lingue di fuoco
che si separavano e si posavano \brk	 sopra ciascuno di loro...
...e tutti furono pieni del Suo Spirito}
\endverse


%%%%% STROFA
\beginverse		%Oppure \beginverse* se non si vuole il numero di fianco
\memorize 		% <<< DECOMMENTA se si vuole utilizzarne la funzione
%\chordsoff		% <<< DECOMMENTA se vuoi una strofa senza accordi

In un \[A-]mare calmo e immobile, \brk con un \[C]cielo senza nuvole,
non si \[G]riesce a navi\[D-]gare,  \brk prose\[F]guire non si \[A-]può.
Una \[A-]brezza lieve e debole,    \brk poi di\[C]venta un vento a raffiche,
soffia \[G]forte sulle \[D-]barche    \brk e ci \[F]spinge via di \[A-]qua.
\vspace*{\versesep}
Come il \[C]vento da la \[G]forza   \brk per viag\[A-]giare in un o\[E-]ceano
così \[F]Tu ci dai lo \[C]Spirito   \brk che ci \[D-]guiderà da \[G]Te.

\endverse

%%%%% RITORNELLO
\beginchorus
\textnote{\textbf{Rit.} }

Sei come \[C]vento  \brk  che \[F]gonfia le \[C]vele,
sei come \[C]fuoco  \brk che ac\[F]cende l'a\[G]more, 
\[E] sei come l'\[A-]aria  \brk che \[E-]si respira \[F]libera
chiara \[C]luce che \[G]il cammino \[F]indica.	\rep{2}

\endchorus


%%%%%% EV. INTERMEZZO
\beginverse*
\vspace*{1.3\versesep}
{
	\nolyrics
	\textnote{Breve intermezzo strumentale}
	
	\ifchorded

	%---- Prima riga -----------------------------
	\vspace*{-\versesep}
	\[A-] \[A-]
	\fi
	%---- Ev Indicazioni -------------------------			
	%\textnote{\textit{(ripetizione della strofa)}} 
	 
}
\vspace*{\versesep}
\endverse






%%%%% STROFA
\beginverse		%Oppure \beginverse* se non si vuole il numero di fianco
%\memorize 		% <<< DECOMMENTA se si vuole utilizzarne la funzione
%\chordsoff		% <<< DECOMMENTA se vuoi una strofa senza accordi

Nella ^notte impenetrabile, \brk ogni ^cosa è irraggiungibile,
non puoi ^scegliere la ^strada  \brk se non ^vedi avanti a ^te.
Una ^luce fioca e debole,  \brk  sembra ^sorgere e poi crescere,
come ^fiamma che ^rigenera  \brk e che il^lumina la ^via.
\vspace*{\versesep}
Come il ^fuoco scioglie il ^gelo  \brk e rischi^ara ogni sen^tiero
così ^Tu riscaldi il ^cuore  \brk di chi ^Verbo annunce^rà.

\endverse




%%%%%% EV. CHIUSURA SOLO STRUMENTALE
\ifchorded
\beginchorus %oppure \beginverse*
\vspace*{1.3\versesep}
\textnote{Chiusura } %<<< EV. INDICAZIONI

\[C*]

\endchorus  %oppure \endverse
\fi


\endsong
%------------------------------------------------------------
%			FINE CANZONE
%------------------------------------------------------------