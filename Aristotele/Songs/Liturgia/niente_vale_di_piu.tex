%-------------------------------------------------------------
%			INIZIO	CANZONE
%-------------------------------------------------------------


%titolo: 	Niente vale di più
%autore: 	
%tonalita: 	Re

%%%%%% TITOLO E IMPOSTAZONI
\beginsong{Niente vale di più}[by={}] 	% <<< MODIFICA TITOLO E AUTORE
\transpose{0} 						% <<< TRASPOSIZIONE #TONI (0 nullo)
\momenti{Comunione; Congedo; Prima Comunione}							% <<< INSERISCI MOMENTI	
% momenti vanno separati da ; e vanno scelti tra:
% Ingresso; Atto penitenziale; Acclamazione al Vangelo; Dopo il Vangelo; Offertorio; Comunione; Ringraziamento; Fine; Santi; Pasqua; Avvento; Natale; Quaresima; Canti Mariani; Battesimo; Prima Comunione; Cresima; Matrimonio; Meditazione;
\ifchorded
	%\textnote{Tonalità originale }	% <<< EV COMMENTI (tonalità originale/migliore)
\fi

%%%%%% INTRODUZIONE
\ifchorded
\vspace*{\versesep}
\musicnote{
\begin{minipage}{0.48\textwidth}
\textbf{Intro}
\hfill 
%( \eighthnote \, 80)   % <<  MODIFICA IL TEMPO
% Metronomo: \eighthnote (ottavo) \quarternote (quarto) \halfnote (due quarti)
\end{minipage}
} 	
\vspace*{-\versesep}
\beginverse*
\nolyrics

%---- Prima riga -----------------------------
\vspace*{-\versesep}
\[D] \[A] \[D]	 % \[*D] per indicare le pennate, \rep{2} le ripetizioni

%---- Ogni riga successiva -------------------
%\vspace*{-\versesep}
%\[G] \[C]  \[D]	

%---- Ev Indicazioni -------------------------			
%\textnote{\textit{(Oppure tutta la strofa)} }	

\endverse
\fi

%%%%% STROFA
\beginverse		%Oppure \beginverse* se non si vuole il numero di fianco
\memorize 		% <<< DECOMMENTA se si vuole utilizzarne la funzione
%\chordsoff		& <<< DECOMMENTA se vuoi una strofa senza accordi

Non \[D]so cosa voglio rag\[A]giungere
non \[B-]so quali stelle rac\[F#-]cogliere
pe\[G]rò ho una \[A]gioia da \[D]vive\[B-]re
dai \[E-]dammi la mano cam\[A]mina con \[A7]me.
Io \[D]credo in un mondo fan\[A]tastico
che u\[B-]nisce il sorriso degli \[F#-]uomini
non \[G]dirmi che è un \[A]sogno impos\[D]sibi\[B-]le
se as\[C]colti il tuo cuore al\[A]lora sapr\[A7]ai:

\endverse


%%%%% RITORNELLO
\beginchorus
\textnote{\textbf{Rit.}}

Che \[D]niente è più bello di una \[A]vita vissuta,
di una \[B-]pace donata di un a\[G]more fe\[A]dele,
di un fra\[G]tello che \[A]crede.
Che \[D]niente è più grande di una \[A]voce che chiama
il tuo \[B-]nome nel mondo di una \[G]vita che an\[A]nuncia, 
la pa\[G]rola che \[A]salva. 
Ti \[G]svelo un segreto: se \[D]cerchi un amico 
il Si\[Bb]gnore ti \[C]sta amando \[D]già 
il Si\[Bb]gnore ti \[C]sta amando \[D]già. \[A]
\endchorus

%%%%% STROFA
\beginverse		%Oppure \beginverse* se non si vuole il numero di fianco
%\memorize 		% <<< DECOMMENTA se si vuole utilizzarne la funzione
%\chordsoff		% <<< DECOMMENTA se vuoi una strofa senza accordi

Io ^chiedo il coraggio di ^vivere 
fra^tello alle voci che at^tendono 
spe^ranze che ^volano ^libe^re 
più ^alte del sole ^raggiungono ^Te. 
Non ^so quali volti co^noscerò 
e ^quante illusioni attra^verserai
se un ^giorno si ^leverà i^nuti^le 
as^colta il tuo cuore al^lora sapr^ai:

\endverse

%%%%% STROFA
\beginverse		%Oppure \beginverse* se non si vuole il numero di fianco
%\memorize 		% <<< DECOMMENTA se si vuole utilizzarne la funzione
%\chordsoff		% <<< DECOMMENTA se vuoi una strofa senza accordi

Per ^ogni momento che ^tu mi dai
do^mando la forza di ^credere 
nel ^gesto d'a^more che ^libe^ra 
e ^questo mio canto pregh^iera sa^rà. 
Se ^scopri con gioia la ^verità 
racch^iusa negli occhi degli ^uomini 
se ^cerchi un te^soro per ^vive^re 
as^colta il tuo cuore al^lora sapr^ai:

\endverse

%%%%%% EV. CHIUSURA SOLO STRUMENTALE
\ifchorded
\beginchorus %oppure \beginverse*
\vspace*{1.3\versesep}
\textnote{Chiusura } %<<< EV. INDICAZIONI

\[D*]

\endchorus  %oppure \endverse
\fi



\endsong
%------------------------------------------------------------
%			FINE CANZONE
%------------------------------------------------------------
