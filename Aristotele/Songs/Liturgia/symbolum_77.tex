%-------------------------------------------------------------
%			INIZIO	CANZONE
%-------------------------------------------------------------


%titolo: 	Symbolum 77 (Tu sei la mia vita)
%autore: 	Pierangelo Sequeri
%tonalita: 	Mi- 



%%%%%% TITOLO E IMPOSTAZONI
\beginsong{Symbolum 77 (Tu sei la mia vita)}[by={Pierangelo Sequeri}] 	% <<< MODIFICA TITOLO E AUTORE
\transpose{0} 						% <<< TRASPOSIZIONE #TONI (0 nullo)
%\preferflats  %SE VOGLIO FORZARE i bemolle come alterazioni
%\prefersharps %SE VOGLIO FORZARE i # come alterazioni
\momenti{Comunione}							% <<< INSERISCI MOMENTI	
% momenti vanno separati da ; e vanno scelti tra:
% Ingresso; Atto penitenziale; Acclamazione al Vangelo; Dopo il Vangelo; Offertorio; Comunione; Ringraziamento; Fine; Santi; Pasqua; Avvento; Natale; Quaresima; Canti Mariani; Battesimo; Prima Comunione; Cresima; Matrimonio; Meditazione; Spezzare del pane;
\ifchorded
	%\textnote{Tonalità migliore }	% <<< EV COMMENTI (tonalità originale/migliore)
\fi


%%%%%% INTRODUZIONE
\ifchorded
\vspace*{\versesep}
\textnote{Intro: \qquad \qquad  }%(\eighthnote 116) % <<  MODIFICA IL TEMPO
% Metronomo: \eighthnote (ottavo) \quarternote (quarto) \halfnote (due quarti)
\vspace*{-\versesep}
\beginverse*

\nolyrics

%---- Prima riga -----------------------------
\vspace*{-\versesep}
\[E-] \[C] \[D] \[E-]	 % \[*D] per indicare le pennate, \rep{2} le ripetizioni

%---- Ogni riga successiva -------------------
%\vspace*{-\versesep}
%\[G] \[C]  \[D]	

%---- Ev Indicazioni -------------------------			
%\textnote{\textit{(Oppure tutta la strofa)} }	

\endverse
\fi

%%%%% STROFA
\beginverse		%Oppure \beginverse* se non si vuole il numero di fianco
\memorize 		% <<< DECOMMENTA se si vuole utilizzarne la funzione
%\chordsoff		% <<< DECOMMENTA se vuoi una strofa senza accordi

\[E-]Tu sei la mia \[C]vita \[D]altro io non \[G]ho.
\[E-]Tu sei la mia \[C]strada, \[D]la mia veri\[B7]tà.
\[A-]Nella tua pa\[D7]rola \[G]io cammine\[E-]rò
\[C]finché avrò res\[D]piro, fino a \[G]quando Tu vorr\[B7]ai:
\[A-]non avrò pa\[D7]ura sai \[G]se Tu sei con \[E-]me,
\[C]io ti prego \[D]resta con \[E-]me.

\endverse

%%%%% STROFA
\beginverse		%Oppure \beginverse* se non si vuole il numero di fianco
\memorize 		% <<< DECOMMENTA se si vuole utilizzarne la funzione
\chordsoff		% <<< DECOMMENTA se vuoi una strofa senza accordi

^Credo in Te, Si^gnore, ^nato da Mar^ia,
^Figlio eterno e ^santo, ^uomo come ^noi,
^morto per a^more, ^vivo in mezzo a ^noi:
^una cosa ^sola con il ^Padre e con i ^tuoi,
^fino a quando, ^io lo so, ^Tu ritorner^ai
^per aprirci il ^regno di ^Dio.

\endverse

%%%%% STROFA
\beginverse		%Oppure \beginverse* se non si vuole il numero di fianco
\memorize 		% <<< DECOMMENTA se si vuole utilizzarne la funzione
\chordsoff		% <<< DECOMMENTA se vuoi una strofa senza accordi

^Tu sei la mia ^forza, ^altro io non ^ho. 
^Tu sei la mia ^pace, ^la mia liber^tà. 
^Niente nella ^vita ^ci separe^rà,
^so che la tua ^mano forte ^non mi lasce^rà. 
^So che da ogni ^male ^Tu mi libere^rai 
^e nel tuo ^perdono viv^rò.

\endverse

%%%%% STROFA
\beginverse		%Oppure \beginverse* se non si vuole il numero di fianco
\memorize 		% <<< DECOMMENTA se si vuole utilizzarne la funzione
\chordsoff		% <<< DECOMMENTA se vuoi una strofa senza accordi

^Padre della ^vita ^noi crediamo in ^Te. 
^Figlio Salva^tore ^noi speriamo in ^Te. 
^Spirito d'a^more ^vieni in mezzo a ^noi,
^Tu da mille ^strade ci ra^duni in uni^tà 
^e per mille ^strade poi ^dove Tu vor^rai 
^noi saremo il ^seme di ^Dio.

\endverse

\endsong
%------------------------------------------------------------
%			FINE CANZONE
%------------------------------------------------------------