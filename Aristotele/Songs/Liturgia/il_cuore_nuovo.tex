%-------------------------------------------------------------
%			INIZIO	CANZONE
%-------------------------------------------------------------


%titolo: 	Il cuore nuovo
%autore: 	D. MAchetta
%tonalita: 	Do-



%%%%%% TITOLO E IMPOSTAZONI
\beginsong{Il cuore nuovo}[by={D. Machetta}] 	% <<< MODIFICA TITOLO E AUTORE
%#ALIAS Io verrò a salvarvi
\transpose{2} 						% <<< TRASPOSIZIONE #TONI (0 nullo)
\preferflats  %SE VOGLIO FORZARE i bemolle come alterazioni
%\prefersharps %SE VOGLIO FORZARE i # come alterazioni
\momenti{}							% <<< INSERISCI MOMENTI	
% momenti vanno separati da ; e vanno scelti tra:
% Ingresso; Atto penitenziale; Acclamazione al Vangelo; Dopo il Vangelo; Offertorio; Comunione; Ringraziamento; Fine; Santi; Pasqua; Avvento; Natale; Quaresima; Canti Mariani; Battesimo; Prima Comunione; Cresima; Matrimonio; Meditazione; Spezzare del pane;
\ifchorded
	%\textnote{Tonalità migliore }	% <<< EV COMMENTI (tonalità originale\migliore)
\fi


%%%%%% INTRODUZIONE
\ifchorded
\vspace*{\versesep}
\textnote{Intro: \qquad \qquad  }%(\eighthnote 116) % <<  MODIFICA IL TEMPO
% Metronomo: \eighthnote (ottavo) \quarternote (quarto) \halfnote (due quarti)
\vspace*{-\versesep}
\beginverse*

\nolyrics

%---- Prima riga -----------------------------
\vspace*{-\versesep}
\[C-] \[G-] \[C-]	 % \[*D] per indicare le pennate, \rep{2} le ripetizioni

%---- Ogni riga successiva -------------------
%\vspace*{-\versesep}
%\[G] \[C]  \[D]	

%---- Ev Indicazioni -------------------------			
%\textnote{\textit{(Oppure tutta la strofa)} }	

\endverse
\fi




%%%%% STROFA
\beginverse		%Oppure \beginverse* se non si vuole il numero di fianco
\memorize 		% <<< DECOMMENTA se si vuole utilizzarne la funzione
%\chordsoff		% <<< DECOMMENTA se vuoi una strofa senza accordi

\[C-]Io verrò a salvarvi tra le \[G-]genti,
vi condur\[A&]rò nella vostra di\[G-]mora.
\[C-]Spargerò su voi torrenti d'\[G-]acque:
da ogni \[A&]colpa sarete la\[C]vati.

\endverse




%%%%% RITORNELLO
\beginchorus
\textnote{\textbf{Rit.}}

\[E&]Dio ci da\[F]rà un cuore \[B&]nuovo,
\[E&]porrà in \[F]noi uno spirito \[G]nuovo.

\endchorus



%%%%% STROFA
\beginverse		%Oppure \beginverse* se non si vuole il numero di fianco
%\memorize 		% <<< DECOMMENTA se si vuole utilizzarne la funzione
%\chordsoff		% <<< DECOMMENTA se vuoi una strofa senza accordi

\[C-]Voglio liberarvi dai \[G-]peccati,
abbatte\[A&]rò ogni falso di\[G-]o.
\[C-]Tolgo il vostro cuore di pi\[G-]etra
per rega\[A&]larvi un cuore di \[C]carne.

\endverse



%%%%% STROFA
\beginverse		%Oppure \beginverse* se non si vuole il numero di fianco
%\memorize 		% <<< DECOMMENTA se si vuole utilizzarne la funzione
%\chordsoff		% <<< DECOMMENTA se vuoi una strofa senza accordi

\[C-]Voi osserverete la mia \[G-]legge
e abite\[A&]rete la terra dei \[G-]padri.
\[C-]Voi sarete il popolo fe\[G-]dele
e io sa\[A&]rò il vostro Dio per \[C]sempre.

\endverse




\endsong
%------------------------------------------------------------
%			FINE CANZONE
%------------------------------------------------------------


