%-------------------------------------------------------------
%			INIZIO	CANZONE
%-------------------------------------------------------------


%titolo: 	Santo Ricci
%autore: 	Daniele Ricci
%tonalita: 	Sol 



%%%%%% TITOLO E IMPOSTAZONI
\beginsong{Vieni o Spirito}[by={M. C. Bizzeti}] 	% <<< MODIFICA TITOLO E AUTORE
\transpose{0} 						% <<< TRASPOSIZIONE #TONI (0 nullo)
%\preferflats  %SE VOGLIO FORZARE i bemolle come alterazioni
%\prefersharps %SE VOGLIO FORZARE i # come alterazioni
\momenti{Ingresso; Battesimo; Cresima}							% <<< INSERISCI MOMENTI	
% momenti vanno separati da ; e vanno scelti tra:
% Ingresso; Atto penitenziale; Acclamazione al Vangelo; Dopo il Vangelo; Offertorio; Comunione; Ringraziamento; Fine; Santi; Pasqua; Avvento; Natale; Quaresima; Canti Mariani; Battesimo; Prima Comunione; Cresima; Matrimonio; Meditazione; Spezzare del pane;
\ifchorded
	%\textnote{$\bigstar$ Tonalità migliore }	% <<< EV COMMENTI (tonalità originale\migliore)
\fi


%%%%%% INTRODUZIONE
\ifchorded
\vspace*{\versesep}
\musicnote{
\begin{minipage}{0.48\textwidth}
\textbf{Intro}
\hfill 
%( \eighthnote \, 80)   % <<  MODIFICA IL TEMPO
% Metronomo: \eighthnote (ottavo) \quarternote (quarto) \halfnote (due quarti)
\end{minipage}
} 	
\vspace*{-\versesep}
\beginverse*

\nolyrics

%---- Prima riga -----------------------------
\vspace*{-\versesep}
\[B-] \[A] \[B-]	 % \[*D] per indicare le pennate, \rep{2} le ripetizioni

%---- Ogni riga successiva -------------------
%\vspace*{-\versesep}
%\[G] \[C]  \[D]	

%---- Ev Indicazioni -------------------------			
%\textnote{\textit{[oppure tutta la strofa]} }	

\endverse
\fi



%%%%% RITORNELLO
\beginchorus
\textnote{\textbf{Rit.}}

\[B-]Vieni o Sprito \[B-/D]Spirito di Dio,	
\[A]vieni o \[F#-]Spirito \[B-]Santo.	
\[B-]Vieni o Spirito soffia su di noi 
\[A]dona ai tuoi \[F#-]figli la \[B-]vita.

\endchorus



%%%%% STROFA
\beginverse		%Oppure \beginverse* se non si vuole il numero di fianco
%\memorize 		% <<< DECOMMENTA se si vuole utilizzarne la funzione
%\chordsoff		% <<< DECOMMENTA se vuoi una strofa senza accordi


\[G]Dona la \[A]luce ai nostri \[B-]occhi,
\[D]dona la \[A]forza ai nostri \[B-]cuori,
\[F#]dona alle menti la sapie\[B-]nza,
\[G]dona il tuo \[A]fuoco d'am\[B-]ore.

\endverse

%%%%% RITORNELLO
\beginchorus
\textnote{\textbf{Rit.}}

\[B-]Vieni o Sprito \[B-/D]Spirito di Dio,	
\[A]vieni o \[F#-]Spirito \[B-]Santo.	
\[B-]Vieni o Spirito soffia su di noi 
\[A]dona ai tuoi \[F#-]figli la \[B-]vita.

\endchorus


%%%%% STROFA
\beginverse		%Oppure \beginverse* se non si vuole il numero di fianco
%\memorize 		% <<< DECOMMENTA se si vuole utilizzarne la funzione
%\chordsoff		% <<< DECOMMENTA se vuoi una strofa senza accordi

\[G]Tu sei per \[A]noi consola\[B-]tore;
\[D]nella \[A]calura sei \[B-]riparo
\[F#]nella fatica sei \[B-]riposo
\[G]nel pianto \[A]sei conf\[B-]orto.

\endverse





%%%%% RITORNELLO
\beginchorus
\textnote{\textbf{Rit.}}

\[B-]Vieni o Sprito \[B-/D]Spirito di Dio,	
\[A]vieni o \[F#-]Spirito \[B-]Santo.	
\[B-]Vieni o Spirito soffia su di noi 
\[A]dona ai tuoi \[F#-]figli la \[B-]vita.

\endchorus


%%%%% STROFA
\beginverse		%Oppure \beginverse* se non si vuole il numero di fianco
%\memorize 		% <<< DECOMMENTA se si vuole utilizzarne la funzione
%\chordsoff		% <<< DECOMMENTA se vuoi una strofa senza accordi

\[G]Dona a \[A]tutti i tuoi \[B-]fedeli
\[D]Che \[A]confidano in \[B-]Te.
\[F#]I tuoi sette Santi \[B-]doni,
\[G]dona la \[A]gioia \[B-]eterna.

\endverse





\endsong
%------------------------------------------------------------
%			FINE CANZONE
%------------------------------------------------------------

