%-------------------------------------------------------------
%			INIZIO	CANZONE
%-------------------------------------------------------------


%titolo: 	Beato il cuore GMG 2016
%autore: 	Casucci, Balduzzi
%tonalita: 	Fa 



%%%%%% TITOLO E IMPOSTAZONI
\beginsong{Beato il cuore (Inno GMG 2016)}[by={Jakub Blycharz, GMG Cracovia 2016}]
\transpose{-2} 						% <<< TRASPOSIZIONE #TONI (0 nullo)
\momenti{Comunione; Fine; Ingresso}							% <<< INSERISCI MOMENTI	
% momenti vanno separati da ; e vanno scelti tra:
% Ingresso; Atto penitenziale; Acclamazione al Vangelo; Dopo il Vangelo; Offertorio; Comunione; Ringraziamento; Fine; Santi; Pasqua; Avvento; Natale; Quaresima; Canti Mariani; Battesimo; Prima Comunione; Cresima; Matrimonio; Meditazione;
\ifchorded
	\textnote{Tonalità migliore}	% <<< EV COMMENTI (tonalità originale/migliore)
\fi


%%%%%% INTRODUZIONE
\ifchorded
\vspace*{\versesep}
\textnote{Intro: \qquad \qquad  }%(\eighthnote 116) % << MODIFICA IL TEMPO
% Metronomo: \eighthnote (ottavo) \quarternote (quarto) \halfnote (due quarti)
\vspace*{-\versesep}
\beginverse*

\nolyrics

%---- Prima riga -----------------------------
\vspace*{-\versesep}
\[C#-]  \[A]	\[E]  % \[*D] per indicare le pennate, \rep{2} le ripetizioni

%---- Ogni riga successiva -------------------
\vspace*{-\versesep}
\[B] \[(B)] \[(B)]  \[C#-]

%---- Ev Indicazioni -------------------------			
%\textnote{\textit{(Oppure tutta la strofa)} }	

\endverse
\fi




%%%%% STROFA
\beginverse		%Oppure \beginverse* se non si vuole il numero di fianco
\memorize 		% <<< DECOMMENTA se si vuole utilizzarne la funzione
%\chordsoff		& <<< DECOMMENTA se vuoi una strofa senza accordi

\[C#-]Sei sceso \[A]dalla tua immensi\[E]tà
\[D]in nostro a\[A]iu\[E]to.
Miseri\[B]cordia  scorre  da \[F#]te
\[A]sopra \[B]tutti \[C#]noi.


\endverse


%%%%% STROFA
\beginverse*	%Oppure \beginverse* se non si vuole il numero di fianco
%\memorize 		% <<< DECOMMENTA se si vuole utilizzarne la funzione
%\chordsoff		& <<< DECOMMENTA se vuoi una strofa senza accordi

^ Persi in un ^mondo d’oscuri^tà
^lì Tu ci ^tro^vi.
Nelle tue ^braccia ci stringi e ^poi
^dai la ^vita per ^noi.


\endverse



%%%%% RITORNELLO
\beginchorus
\textnote{\textbf{Rit.}}

Beato è il \[E]cuo\[B]re che per\[C#-]do\[A]na!
Miseri\[E]cordia riceve\[B]rà da Dio in ci\[F#]elo! \rep{2}

\endchorus



%%%%% STROFA
\beginverse		%Oppure \beginverse* se non si vuole il numero di fianco
%\memorize 		% <<< DECOMMENTA se si vuole utilizzarne la funzione
%\chordsoff		% <<< DECOMMENTA se vuoi una strofa senza accordi

^ Solo il per^dono riporte^rà
^pace nel ^mon^do.
Solo il per^dono ci svele^rà
^come f^igli t^uoi.

\endverse



%%%%% RITORNELLO
\beginchorus
\textnote{\textbf{Rit.}}

Beato è il \[E]cuo\[B]re che per\[C#-]do\[A]na!
Miseri\[E]cordia riceve\[B]rà da Dio in ci\[F#]elo! \rep{2}

\endchorus




%%%%% STROFA
\beginverse		%Oppure \beginverse* se non si vuole il numero di fianco
%\memorize 		% <<< DECOMMENTA se si vuole utilizzarne la funzione
%\chordsoff		% <<< DECOMMENTA se vuoi una strofa senza accordi

^ Col sangue in ^croce hai pagato ^Tu
^le nostre ^pover^tà.
Se noi ci am^iamo e restiamo in^ te
^il mondo ^crede^rà!

\endverse



%%%%% RITORNELLO
\beginchorus
\textnote{\textbf{Rit.}}

Beato è il \[E]cuo\[B]re che per\[C#-]do\[A]na!
Miseri\[E]cordia riceve\[B]rà da Dio in ci\[F#]elo! \rep{2}

\endchorus




%%%%% BRIDGE
\beginverse*		%Oppure \beginverse* se non si vuole il numero di fianco
%\memorize 		% <<< DECOMMENTA se si vuole utilizzarne la funzione
%\chordsoff		% <<< DECOMMENTA se vuoi una strofa senza accordi
\textnote{Bridge}
\[A]Le nostre an\[B]gosce ed ansie\[C#-]tà
get\[A]tiamo ogni \[B]attimo in \[A]te.
Amore \[B]che non abbandona \[C#-]mai,
\[A]vivi in \[B]mezzo a \[C#]noi!

\endverse



%%%%% RITORNELLO
\beginchorus
\textnote{\textbf{Rit.}}

Beato è il \[A]cuo\[E]re che per\[F#-]do\[D]na!
Miseri\[A]cordia riceve\[E]rà da Dio in ci\[B]elo! \rep{4}

\endchorus



%%%%%% EV. INTERMEZZO
\beginverse*
\vspace*{1.3\versesep}
{
	\nolyrics
	\musicnote{Chiusura}
	
	\ifchorded

	%---- Prima riga -----------------------------
	\vspace*{-\versesep}
	\[C#-]  \[A]	\[E]  % \[*D] per indicare le pennate, \rep{2} le ripetizioni


	%---- Ogni riga successiva -------------------
	\vspace*{-\versesep}
	\[B] \[(B)] \[(B)]  \[C#-]


	\fi
	%---- Ev Indicazioni -------------------------			
	%\textnote{\textit{(ripetizione della strofa)}} 
	 
}
\vspace*{\versesep}
\endverse


\endsong
%------------------------------------------------------------
%			FINE CANZONE
%------------------------------------------------------------




%++++++++++++++++++++++++++++++++++++++++++++++++++++++++++++
%			CANZONE TRASPOSTA
%++++++++++++++++++++++++++++++++++++++++++++++++++++++++++++
\ifchorded
%decremento contatore per avere stesso numero
\addtocounter{songnum}{-1} 
\beginsong{Beato il cuore (Inno GMG 2016)}[by={Jakub Blycharz, GMG Cracovia 2016}]	% <<< COPIA TITOLO E AUTORE
\transpose{0} 						% <<< TRASPOSIZIONE #TONI + - (0 nullo)
%\preferflats SE VOGLIO FORZARE i bemolle come alterazioni
%\prefersharps SE VOGLIO FORZARE i # come alterazioni
\ifchorded
	\textnote{Tonalità originale}	% <<< EV COMMENTI (tonalità originale/migliore)
\fi


%%%%%% INTRODUZIONE
\ifchorded
\vspace*{\versesep}
\textnote{Intro: \qquad \qquad  }%(\eighthnote 116) % << MODIFICA IL TEMPO
% Metronomo: \eighthnote (ottavo) \quarternote (quarto) \halfnote (due quarti)
\vspace*{-\versesep}
\beginverse*

\nolyrics

%---- Prima riga -----------------------------
\vspace*{-\versesep}
\[C#-]  \[A]	\[E]  % \[*D] per indicare le pennate, \rep{2} le ripetizioni

%---- Ogni riga successiva -------------------
\vspace*{-\versesep}
\[B] \[(B)] \[(B)]  \[C#-]

%---- Ev Indicazioni -------------------------			
%\textnote{\textit{(Oppure tutta la strofa)} }	

\endverse
\fi




%%%%% STROFA
\beginverse		%Oppure \beginverse* se non si vuole il numero di fianco
\memorize 		% <<< DECOMMENTA se si vuole utilizzarne la funzione
%\chordsoff		& <<< DECOMMENTA se vuoi una strofa senza accordi

\[C#-]Sei sceso \[A]dalla tua immensi\[E]tà
\[D]in nostro a\[A]iu\[E]to.
Miseri\[B]cordia  scorre  da \[F#]te
\[A]sopra \[B]tutti \[C#]noi.


\endverse


%%%%% STROFA
\beginverse*	%Oppure \beginverse* se non si vuole il numero di fianco
%\memorize 		% <<< DECOMMENTA se si vuole utilizzarne la funzione
%\chordsoff		& <<< DECOMMENTA se vuoi una strofa senza accordi

^ Persi in un ^mondo d’oscuri^tà
^lì Tu ci ^tro^vi.
Nelle tue ^braccia ci stringi e ^poi
^dai la ^vita per ^noi.


\endverse



%%%%% RITORNELLO
\beginchorus
\textnote{\textbf{Rit.}}

Beato è il \[E]cuo\[B]re che per\[C#-]do\[A]na!
Miseri\[E]cordia riceve\[B]rà da Dio in ci\[F#]elo! \rep{2}

\endchorus



%%%%% STROFA
\beginverse		%Oppure \beginverse* se non si vuole il numero di fianco
%\memorize 		% <<< DECOMMENTA se si vuole utilizzarne la funzione
%\chordsoff		% <<< DECOMMENTA se vuoi una strofa senza accordi

^ Solo il per^dono riporte^rà
^pace nel ^mon^do.
Solo il per^dono ci svele^rà
^come f^igli t^uoi.

\endverse



%%%%% RITORNELLO
\beginchorus
\textnote{\textbf{Rit.}}

Beato è il \[E]cuo\[B]re che per\[C#-]do\[A]na!
Miseri\[E]cordia riceve\[B]rà da Dio in ci\[F#]elo! \rep{2}

\endchorus




%%%%% STROFA
\beginverse		%Oppure \beginverse* se non si vuole il numero di fianco
%\memorize 		% <<< DECOMMENTA se si vuole utilizzarne la funzione
%\chordsoff		% <<< DECOMMENTA se vuoi una strofa senza accordi

^ Col sangue in ^croce hai pagato ^Tu
^le nostre ^pover^tà.
Se noi ci am^iamo e restiamo in^ te
^il mondo ^crede^rà!

\endverse



%%%%% RITORNELLO
\beginchorus
\textnote{\textbf{Rit.}}

Beato è il \[E]cuo\[B]re che per\[C#-]do\[A]na!
Miseri\[E]cordia riceve\[B]rà da Dio in ci\[F#]elo! \rep{2}

\endchorus




%%%%% BRIDGE
\beginverse*		%Oppure \beginverse* se non si vuole il numero di fianco
%\memorize 		% <<< DECOMMENTA se si vuole utilizzarne la funzione
%\chordsoff		% <<< DECOMMENTA se vuoi una strofa senza accordi
\textnote{Bridge}
\[A]Le nostre an\[B]gosce ed ansie\[C#-]tà
get\[A]tiamo ogni \[B]attimo in \[A]te.
Amore \[B]che non abbandona \[C#-]mai,
\[A]vivi in \[B]mezzo a \[C#]noi!

\endverse



%%%%% RITORNELLO
\beginchorus
\textnote{\textbf{Rit.}}

Beato è il \[A]cuo\[E]re che per\[F#-]do\[D]na!
Miseri\[A]cordia riceve\[E]rà da Dio in ci\[B]elo! \rep{4}

\endchorus



%%%%%% EV. INTERMEZZO
\beginverse*
\vspace*{1.3\versesep}
{
	\nolyrics
	\musicnote{Chiusura}
	
	\ifchorded

	%---- Prima riga -----------------------------
	\vspace*{-\versesep}
	\[C#-]  \[A]	\[E]  % \[*D] per indicare le pennate, \rep{2} le ripetizioni


	%---- Ogni riga successiva -------------------
	\vspace*{-\versesep}
	\[B] \[(B)] \[(B)]  \[C#-]


	\fi
	%---- Ev Indicazioni -------------------------			
	%\textnote{\textit{(ripetizione della strofa)}} 
	 
}
\vspace*{\versesep}
\endverse


\endsong


\fi
%++++++++++++++++++++++++++++++++++++++++++++++++++++++++++++
%			FINE CANZONE TRASPOSTA
%++++++++++++++++++++++++++++++++++++++++++++++++++++++++++++
