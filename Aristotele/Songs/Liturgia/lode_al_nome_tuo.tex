%-------------------------------------------------------------
%			INIZIO	CANZONE
%-------------------------------------------------------------


%titolo: 	Benedetto il nome del Signore
%autore: 	Matt Redman
%tonalita: 	Sol > La 



%%%%%% TITOLO E IMPOSTAZONI
\beginsong{Lode al nome Tuo}[by={Blessed be your name — M. Redman, B. Redman}] 	% <<< MODIFICA TITOLO E AUTORE
\transpose{0} 						% <<< TRASPOSIZIONE #TONI (0 nullo)
%\preferflats  %SE VOGLIO FORZARE i bemolle come alterazioni
%\prefersharps %SE VOGLIO FORZARE i # come alterazioni
\momenti{Meditazione; Ringraziamento; Fine}							% <<< INSERISCI MOMENTI	
% momenti vanno separati da ; e vanno scelti tra:
% Ingresso; Atto penitenziale; Acclamazione al Vangelo; Dopo il Vangelo; Offertorio; Comunione; Ringraziamento; Fine; Santi; Pasqua; Avvento; Natale; Quaresima; Canti Mariani; Battesimo; Prima Comunione; Cresima; Matrimonio; Meditazione; Spezzare del pane;
\ifchorded
	%\textnote{Tonalità migliore }	% <<< EV COMMENTI (tonalità originale/migliore)
\fi


%%%%%% INTRODUZIONE
\ifchorded
\vspace*{\versesep}
\textnote{Intro: \qquad \qquad  }%(\eighthnote 116) % <<  MODIFICA IL TEMPO
% Metronomo: \eighthnote (ottavo) \quarternote (quarto) \halfnote (due quarti)
\vspace*{-\versesep}
\beginverse*

\nolyrics

%---- Prima riga -----------------------------
\vspace*{-\versesep}
\[G] \[D] \[E-] \[C] 	 % \[*D] per indicare le pennate, \rep{2} le ripetizioni

%---- Ogni riga successiva -------------------
\vspace*{-\versesep}
\[G] \[D] \[E-] \[C] \[C]	

%---- Ev Indicazioni -------------------------			
%\textnote{\textit{(Oppure tutta la strofa)} }	

\endverse
\fi




%%%%% STROFA
\beginverse		%Oppure \beginverse* se non si vuole il numero di fianco
\memorize 		% <<< DECOMMENTA se si vuole utilizzarne la funzione
%\chordsoff		% <<< DECOMMENTA se vuoi una strofa senza accordi

\[G] Lode al \[D]nome Tuo, \brk dalle \[E-]terre più \[C]floride.
Dove \[G]tutto sembra \[D]vivere,  
lode al \[C]nome Tuo.

\endverse
\beginverse*

^ Lode al ^nome Tuo,  \brk  dalle ^terre più ^aride.
Dove ^tutto sembra ^sterile, 
lode al ^nome Tuo.


\endverse
\beginverse*	

^ Tornerò a lo^darti sempre 
^ per ogni dono ^Tuo.
^ E quando scende^rà la notte 
\[E-] sempre io di\[C]rò:

\endverse





%%%%% RITORNELLO
\beginchorus
\textnote{\textbf{Rit.}}

Benedetto il \[G]nome del Si\[D]gnor,
lode al nome \[E-]Tu-u-\[C]o!
Benedetto il \[G]nome del Si\[D]gnor,
il glorioso \[E-]nome di Ge\[C]sù. 	

\endchorus





%%%%% STROFA
\beginverse		%Oppure \beginverse* se non si vuole il numero di fianco
%\memorize 		% <<< DECOMMENTA se si vuole utilizzarne la funzione
%\chordsoff		% <<< DECOMMENTA se vuoi una strofa senza accordi

^ Lode al ^nome Tuo, \brk quando il ^sole splende ^su di me.
Quando ^tutto è incan^tevole,
lode al ^nome Tuo.

\endverse
\beginverse*	

^ Lode al ^nome Tuo, \brk quando ^io sto da^vanti a Te.
Con il ^cuore triste e ^fragile,
lode al ^nome Tuo.

\endverse
\beginverse*	

^ Tornerò a lo^darti sempre 
^ per ogni dono ^Tuo.
^ E quando scende^rà la notte 
\[E-] sempre io di\[C]rò:

\endverse



%%%%% RITORNELLO
\beginchorus
\textnote{\textbf{Rit.}}

Benedetto il \[G]nome del Si\[D]gnor,
lode al nome \[E-]Tu-u-\[C]o!
Benedetto il \[G]nome del Si\[D]gnor,
il glorioso \[E-]nome di Ge\[C]sù. 	

\endchorus



%%%%% BRIDGE
\beginverse*		%Oppure \beginverse* se non si vuole il numero di fianco
%\memorize 		% <<< DECOMMENTA se si vuole utilizzarne la funzione
%\chordsoff		% <<< DECOMMENTA se vuoi una strofa senza accordi
\vspace*{1.3\versesep}
\textnote{Bridge} %<<< EV. INDICAZIONI


Tu ^doni e porti ^via.
Tu ^doni e porti ^via.
Ma ^sempre sceglie^rò
di \[E-]benedire \[C]te!  

\endverse
\beginverse*	

^ Tornerò a lo^darti sempre 
^ per ogni dono ^Tuo.
^ E quando scende^rà la notte 
\[E-] sempre io di\[C]rò:

\endverse


%%%%% RITORNELLO
\beginchorus
\textnote{\textbf{Rit.}}

Benedetto il \[G]nome del Si\[D]gnor,
lode al nome \[E-]Tu-u-\[C]o! \rep{3}
Benedetto il \[G]nome del Si\[D]gnor,
il glorioso \[E-]nome di Ge\[C]sù. 	

\endchorus



%%%%% BRIDGE
\beginverse*		%Oppure \beginverse* se non si vuole il numero di fianco
%\memorize 		% <<< DECOMMENTA se si vuole utilizzarne la funzione
%\chordsoff		% <<< DECOMMENTA se vuoi una strofa senza accordi

Tu ^doni e porti ^via.
Tu ^doni e porti ^via.
Ma ^sempre sceglie^rò
di \[E-]benedire \[C]te!  \rep{2}

\endverse





%%%%%% EV. INTERMEZZO
\beginverse*
\vspace*{1.3\versesep}
{
	\nolyrics
	\musicnote{Chiusura strumentale}
	
	\ifchorded

	%---- Prima riga -----------------------------
	\vspace*{-\versesep}
	\[G] \[D] \[E-] \[C] 

	%---- Ogni riga successiva -------------------
	\vspace*{-\versesep}
	\[G] \[D] \[C*]  \textit{(sospeso)}


	\fi
	%---- Ev Indicazioni -------------------------			
	%\musicnote{\textit{sospeso}} 
	 
}
\vspace*{\versesep}
\endverse


\endsong
%------------------------------------------------------------
%			FINE CANZONE
%------------------------------------------------------------


