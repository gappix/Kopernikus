%-------------------------------------------------------------
%			INIZIO	CANZONE
%-------------------------------------------------------------


%titolo: 	Luce dei miei passi (XdiQua)
%autore: 	N. Cermenati, M. Violato, E. Porro, A. Motti
%tonalita: 	Mi



%%%%%% TITOLO E IMPOSTAZONI
\beginsong{Luce dei miei passi (XdiQua)}[by={N. Cermenati, M. Violato, E. Porro, A. Motti}] 	% <<< MODIFICA TITOLO E AUTORE
\transpose{-2} 						% <<< TRASPOSIZIONE #TONI (0 nullo)
\momenti{Comunione; Ringraziamento; Meditazione; Fine}			% <<< INSERISCI MOMENTI	
% momenti vanno separati da ; e vanno scelti tra:
% Ingresso; Atto penitenziale; Acclamazione al Vangelo; Dopo il Vangelo; Offertorio; Comunione; Ringraziamento; Fine; Santi; Pasqua; Avvento; Natale; Quaresima; Canti Mariani; Battesimo; Prima Comunione; Cresima; Matrimonio; Meditazione; Spezzare del pane;
\ifchorded
	\textnote{Tonalità migliore}	% <<< EV COMMENTI (tonalità originale/migliore)
\fi


%%%%%% INTRODUZIONE
\ifchorded
\vspace*{\versesep}
\textnote{Intro: \qquad \qquad }% (\eighthnote 120)} % <<  MODIFICA IL TEMPO
% Metronomo: \eighthnote (ottavo) \quarternote (quarto) \halfnote (due quarti)
\vspace*{-\versesep}
\beginverse*

\nolyrics

%---- Prima riga -----------------------------
\vspace*{-\versesep}
\[E] \[A] \[E]	 \[B]  % \[*D] per indicare le pennate, \rep{2} le ripetizioni

%---- Ogni riga successiva -------------------
\vspace*{-\versesep}
\[E] \[A] \[E]	 \[B]  	

%---- Ev Indicazioni -------------------------			
%\textnote{\textit{(Oppure tutta la strofa)} }	

\endverse
\fi



%%%%% STROFA
\beginverse		%Oppure \beginverse* se non si vuole il numero di fianco
\memorize 		% <<< DECOMMENTA se si vuole utilizzarne la funzione
%\chordsoff		% <<< DECOMMENTA se vuoi una strofa senza accordi

\[E]Nel cammino del\[A]la mia vita \brk \[E]cerco quel che \[B4]sono
\[E]ogni passo in \[A]questo mondo \brk \[E]sogno la mia liber\[B]tà
\[E]ma la strada a \[A]volte è buia, \brk  \[E]perdo la fi\[B4]ducia 
\[E]ho bisogno di un \[A]po’ di luce, \brk il \[E]sole chi sa\[B]rà?

\endverse


\beginverse*		%Oppure \beginverse* se non si vuole il numero di fianco

Ma se guardo il \[F#-]volto tuo, \brk io l’Amore \[A]vedo in Te.
guardo la tua \[F#-]croce e  \brk Tu speranza \[A]sei per \[B]me.

\endverse


%%%%% RITORNELLO
\beginchorus
\textnote{\textbf{Rit.}}

Sono \[E]qui Signore, sono \[A]qui 
\[F#-]luce dei miei passi è la \[A]tua pa\[B]rola! 
(Sono) \[E]qui Signore, sono \[A]qui 
\[F#-]lungo la mia strada, ogni \[A]giorno, 
tu sei \[B]guida per \[E]me! \[A] 
\endchorus


%%%%%% EV. INTERMEZZO
\beginverse*
\vspace*{1.3\versesep}
{
	\nolyrics
	\textnote{Breve intermezzo}
	
	\ifchorded

	%---- Prima riga -----------------------------
	\vspace*{-\versesep}
	\[E] \[A]  \[E]	\[A]   \[B] \[B] 
	 
}
\vspace*{\versesep}
\endverse



%%%%% STROFA
\beginverse		%Oppure \beginverse* se non si vuole il numero di fianco
%\memorize 		% <<< DECOMMENTA se si vuole utilizzarne la funzione
%\chordsoff		% <<< DECOMMENTA se vuoi una strofa senza accordi

^Nel cammino del^la mia vita \brk in^contro il mio fra^tello
^condivido ^la mia strada, \brk la ^meta è perdi^qua.
^ma la strada a ^volte è dura, \brk ^perdo la spe^ranza
^cerco solo un ^punto fermo, \brk in ^chi lo trove^rò?

\endverse


\beginverse*		%Oppure \beginverse* se non si vuole il numero di fianco

Ma se guardo il \[F#-]volto tuo, \brk io l’Amore \[A]vedo in Te.
guardo la tua \[F#-]croce e  \brk Tu speranza \[A]sei per \[B]me.

\endverse

%%%%% RITORNELLO
\beginchorus
\textnote{\textbf{Rit.}}

Sono \[E]qui Signore, sono \[A]qui 
\[F#-]luce dei miei passi è la \[A]tua pa\[B]rola! 
(Sono) \[E]qui Signore, sono \[A]qui 
\[F#-]lungo la mia strada, ogni \[A]giorno, 
tu sei \[B]guida per \[A]me! \quad \[A] \[E]  \[E]
\endchorus

\beginverse*		%Oppure \beginverse* se non si vuole il numero di fianco

\[(E)]Ma se guardo il \[F#-]volto tuo, \brk io l’Amore \[A]vedo in Te.
guardo la tua \[F#-]croce e  \brk Tu speranza \[A]sei per \[B]me. \[C#]

\endverse

\vspace*{1.3\versesep}
\textnote{ \textit{(Cambia la tonalità)}} %<<< EV. INDICAZIONI


%%%%% RITORNELLO
\beginchorus
\textnote{\textbf{Rit.}}
\transpose{2}
Sono \[E]qui Signore, sono \[A]qui 
\[F#-]luce dei miei passi è la \[A]tua pa\[B]rola! 
(Sono) \[E]qui Signore, sono \[A]qui 
\[F#-]lungo la mia strada, ogni \[A]giorno, 
tu sei \[B]guida per \[A]me! \[A]
\endchorus



%%%%%% FINALE

\beginchorus %oppure \beginverse*
\vspace*{1.3\versesep}
\textnote{Finale } %<<< EV. INDICAZIONI

\[F#] tu sei \[C#]guida per \[B]me! \[B] 
\[F#*] tu sei \[C#*]guida per \[B]me!  \[B] \[F#*]

\endchorus  %oppure \endverse
\endsong
%------------------------------------------------------------
%			FINE CANZONE
%------------------------------------------------------------




%++++++++++++++++++++++++++++++++++++++++++++++++++++++++++++
%			CANZONE TRASPOSTA
%++++++++++++++++++++++++++++++++++++++++++++++++++++++++++++
\ifchorded
%decremento contatore per avere stesso numero
\addtocounter{songnum}{-1} 
\beginsong{Luce dei miei passi (XdiQua)}[by={N. Cermenati, M. Violato, E. Porro, A. Motti}] 	% <<< COPIA TITOLO E AUTORE
\transpose{0} 						% <<< TRASPOSIZIONE #TONI + - (0 nullo)
%\preferflats  %SE VOGLIO FORZARE i bemolle come alterazioni
%\prefersharps %SE VOGLIO FORZARE i # come alterazioni
\ifchorded
	\textnote{Tonalità originale}	% <<< EV COMMENTI (tonalità originale/migliore)
\fi


%%%%%% INTRODUZIONE
\ifchorded
\vspace*{\versesep}
\textnote{Intro: \qquad \qquad }% (\eighthnote 120)} % <<  MODIFICA IL TEMPO
% Metronomo: \eighthnote (ottavo) \quarternote (quarto) \halfnote (due quarti)
\vspace*{-\versesep}
\beginverse*

\nolyrics

%---- Prima riga -----------------------------
\vspace*{-\versesep}
\[E] \[A] \[E]	 \[B]  % \[*D] per indicare le pennate, \rep{2} le ripetizioni

%---- Ogni riga successiva -------------------
\vspace*{-\versesep}
\[E] \[A] \[E]	 \[B]  	

%---- Ev Indicazioni -------------------------			
%\textnote{\textit{(Oppure tutta la strofa)} }	

\endverse
\fi



%%%%% STROFA
\beginverse		%Oppure \beginverse* se non si vuole il numero di fianco
\memorize 		% <<< DECOMMENTA se si vuole utilizzarne la funzione
%\chordsoff		% <<< DECOMMENTA se vuoi una strofa senza accordi

\[E]Nel cammino del\[A]la mia vita \brk \[E]cerco quel che \[B4]sono
\[E]ogni passo in \[A]questo mondo \brk \[E]sogno la mia liber\[B]tà
\[E]ma la strada a \[A]volte è buia, \brk  \[E]perdo la fi\[B4]ducia 
\[E]ho bisogno di un \[A]po’ di luce, \brk il \[E]sole chi sa\[B]rà?

\endverse


\beginverse*		%Oppure \beginverse* se non si vuole il numero di fianco

Ma se guardo il \[F#-]volto tuo, \brk io l’Amore \[A]vedo in Te.
guardo la tua \[F#-]croce e  \brk Tu speranza \[A]sei per \[B]me.

\endverse


%%%%% RITORNELLO
\beginchorus
\textnote{\textbf{Rit.}}

Sono \[E]qui Signore, sono \[A]qui 
\[F#-]luce dei miei passi è la \[A]tua pa\[B]rola! 
(Sono) \[E]qui Signore, sono \[A]qui 
\[F#-]lungo la mia strada, ogni \[A]giorno, 
tu sei \[B]guida per \[E]me! \[A] 
\endchorus


%%%%%% EV. INTERMEZZO
\beginverse*
\vspace*{1.3\versesep}
{
	\nolyrics
	\textnote{Breve intermezzo}
	
	\ifchorded

	%---- Prima riga -----------------------------
	\vspace*{-\versesep}
	\[E] \[A]  \[E]	\[A]   \[B] \[B] 
	 
}
\vspace*{\versesep}
\endverse



%%%%% STROFA
\beginverse		%Oppure \beginverse* se non si vuole il numero di fianco
%\memorize 		% <<< DECOMMENTA se si vuole utilizzarne la funzione
%\chordsoff		% <<< DECOMMENTA se vuoi una strofa senza accordi

^Nel cammino del^la mia vita \brk in^contro il mio fra^tello
^condivido ^la mia strada, \brk la ^meta è perdi^qua.
^ma la strada a ^volte è dura, \brk ^perdo la spe^ranza
^cerco solo un ^punto fermo, \brk in ^chi lo trove^rò?

\endverse


\beginverse*		%Oppure \beginverse* se non si vuole il numero di fianco

Ma se guardo il \[F#-]volto tuo, \brk io l’Amore \[A]vedo in Te.
guardo la tua \[F#-]croce e  \brk Tu speranza \[A]sei per \[B]me.

\endverse

%%%%% RITORNELLO
\beginchorus
\textnote{\textbf{Rit.}}

Sono \[E]qui Signore, sono \[A]qui 
\[F#-]luce dei miei passi è la \[A]tua pa\[B]rola! 
(Sono) \[E]qui Signore, sono \[A]qui 
\[F#-]lungo la mia strada, ogni \[A]giorno, 
tu sei \[B]guida per \[A]me! \quad \[A] \[E]  \[E]
\endchorus

\beginverse*		%Oppure \beginverse* se non si vuole il numero di fianco

\[(E)]Ma se guardo il \[F#-]volto tuo, \brk io l’Amore \[A]vedo in Te.
guardo la tua \[F#-]croce e  \brk Tu speranza \[A]sei per \[B]me. \[C#]

\endverse

\vspace*{1.3\versesep}
\textnote{ \textit{(Cambia la tonalità)}} %<<< EV. INDICAZIONI


%%%%% RITORNELLO
\beginchorus
\textnote{\textbf{Rit.}}
\transpose{2}
Sono \[E]qui Signore, sono \[A]qui 
\[F#-]luce dei miei passi è la \[A]tua pa\[B]rola! 
(Sono) \[E]qui Signore, sono \[A]qui 
\[F#-]lungo la mia strada, ogni \[A]giorno, 
tu sei \[B]guida per \[A]me! \[A]
\endchorus



%%%%%% FINALE

\beginchorus %oppure \beginverse*
\vspace*{1.3\versesep}
\textnote{Finale } %<<< EV. INDICAZIONI

\[F#] tu sei \[C#]guida per \[B]me! \[B] 
\[F#*] tu sei \[C#*]guida per \[B]me!  \[B] \[F#*]

\endchorus  %oppure \endverse
\endsong

\fi
%++++++++++++++++++++++++++++++++++++++++++++++++++++++++++++
%			FINE CANZONE TRASPOSTA
%++++++++++++++++++++++++++++++++++++++++++++++++++++++++++++