%-------------------------------------------------------------
%			INIZIO	CANZONE
%-------------------------------------------------------------


%titolo: 	Mi hai tenuto per mano
%autore: 	Meregalli
%tonalita: 	 



%%%%%% TITOLO E IMPOSTAZONI
\beginsong{Mi hai tenuto per mano}[by={G. Meregalli}] 	% <<< MODIFICA TITOLO E AUTORE
\transpose{0} 						% <<< TRASPOSIZIONE #TONI (0 nullo)
\momenti{Comunione; Cresima; Prima Comunione}							% <<< INSERISCI MOMENTI	
% momenti vanno separati da ; e vanno scelti tra:
% Ingresso; Atto penitenziale; Acclamazione al Vangelo; Dopo il Vangelo; Offertorio; Comunione; Ringraziamento; Fine; Santi; Pasqua; Avvento; Natale; Quaresima; Canti Mariani; Battesimo; Prima Comunione; Cresima; Matrimonio; Meditazione; Spezzare del pane;
\ifchorded
	%\textnote{Tonalità originale }	% <<< EV COMMENTI (tonalità originale/migliore)
\fi

%%%%%% INTRODUZIONE
\ifchorded
\vspace*{\versesep}
\musicnote{
\begin{minipage}{0.48\textwidth}
\textbf{Intro}
\hfill 
%( \eighthnote \, 80)   % <<  MODIFICA IL TEMPO
% Metronomo: \eighthnote (ottavo) \quarternote (quarto) \halfnote (due quarti)
\end{minipage}
} 	
\vspace*{-\versesep}
\beginverse*
\nolyrics

%---- Prima riga -----------------------------
\vspace*{-\versesep}
\[C] \rep{4} % \[*D] per indicare le pennate, \rep{2} le ripetizioni

%---- Ogni riga successiva -------------------
%\vspace*{-\versesep}
%\[G] \[C]  \[D]	

%---- Ev Indicazioni -------------------------			
%\textnote{\textit{(Oppure tutta la strofa)} }	

\endverse
\fi




%%%%% STROFA
\beginverse		%Oppure \beginverse* se non si vuole il numero di fianco
%\memorize 		% <<< DECOMMENTA se si vuole utilizzarne la funzione
%\chordsoff		% <<< DECOMMENTA se vuoi una strofa senza accordi

\textnote{\textbf{Mistero dell'esistenza}}

\[C]Gli occhi videro il mondo
durante giorni che non ricordo.
Il \[G]tempo seguì i miei passi
per un mistero che non co\[C]nosco.
Un \[A-]uomo tese le mani,
raccolse il bimbo che aveva in \[G]dono
e \[F]buona vidi mia \[G]madre:
il mio res\[A-]piro fu pianto e \[E]sogno. \[E]

\vspace*{\versesep}

Il \[C]tempo indurì le \[G]mani,
ogni mio \[A-]sogno si fece \[E]terra.
La \[F]strada a chi non ha \[G]ali,
il primo \[A-]passo fu sfida e \[E]rabbia;
pa\[C]role a nutrir la \[G]voce
e luci \[A-]senza veder la \[E]strada:
sul \[F]ciglio stavo se\[G]duto
e fui bam\[A-]bino tra stelle e \[E]sabbia.

\endverse



%%%%%% EV. INTERMEZZO
\ifchorded
\beginverse*
\vspace*{1.3\versesep}
{
	\nolyrics
	\textnote{\textit{[il tempo cambia]}}
	

	%---- Prima riga -----------------------------
	\vspace*{-\versesep}
	\[E] \[E]  \[E]

	%---- Ev Indicazioni -------------------------			
	%\textnote{\textit{(ripetizione della strofa)}} 
	 
}
\endverse
\fi



%%%%% RITORNELLO
\beginchorus
%\textnote{\textbf{Rit.}}
\memorize


\[C]Io non ho più pa\[G]ura,
\[F]vedo un torrente di \[G]volti
che scorre tra gli \[C]argini
di una stessa avven\[G]tura:
\[F] chi è vicino alla \[C]fonte,
\[F] chi già lascia la \[C]valle
\[F] chi ha percorso \[B&]tempo
la grande pia\[C]nura... \[C]  \[C] \[C] 


\endchorus
\beginchorus


Da\[F]vanti al mio \[C]nome 
\[B&] ti sei fer\[C]mato
e dal \[F]primo res\[C]piro
\[B&] mi hai te\[C]nuto per \[F]ma\[C]no,
\[B&] mi hai te\[C]nuto per \[F]ma\[C]no, 
\[B&] mi hai te\[C]nuto per \[F]ma\[F]no, \[F] \[F]


\endchorus




%%%%%% EV. INTERMEZZO
\beginverse*
\vspace*{1.3\versesep}
{
	\nolyrics
	%\textnote{Intermezzo strumentale}
	
	\ifchorded

	%---- Prima riga -----------------------------
	\vspace*{-\versesep}
	\[C] \rep{4}


	\fi
	%---- Ev Indicazioni -------------------------			
	%\textnote{\textit{(ripetizione della strofa)}} 
	 
}
\vspace*{\versesep}
\endverse



%%%%% STROFA
\beginverse		%Oppure \beginverse* se non si vuole il numero di fianco
%\memorize 		% <<< DECOMMENTA se si vuole utilizzarne la funzione
\chordsoff		% <<< DECOMMENTA se vuoi una strofa senza accordi

\textnote{\textbf{Mistero dell’Incarnazione}}
 
Fu ^quando mi seppi dono
che dire grazie fu dire molto.
A ^tanti chiesi una meta:
ebbi il consiglio di stare ^pronto.
Te^metti la verità
quando il coraggio si fece or^goglio.
Qual^cuno mi disse:^cerca 
con l’umil^tà di chi accoglie un ^dono. ^

\vspace*{\versesep}

La ^storia si è fatta ^casa;
il tempo, at^tesa di una sal^vezza.
L’a^more di chi ha cre^ato
si è fatto im^pronta di ogni spe^ranza. 
La ^carne si è fatta ^segno,
custode e ^tempio di tene^rezza.
Il ^Padre si è fatto ^Figlio
e fu bam^bino fra stelle e ^sabbia.
\endverse




%%%%% RITORNELLO
\beginchorus
%\textnote{\textbf{Rit.}}
\chordsoff

^Io non ho più pa^ura,
^vedo la luce e ri^torna la pace,
la str^ada si è fatta si^cura;
^ ora vedo la f^onte
^ e la grande pia^nura:
^ la potenza crea^trice si è fatta crea^tura. ^ ^ ^


\endchorus
\beginchorus


Da\[F]vanti al mio \[C]volto 
\[B&] ti sei chi\[C]nato
e  fra\[F]tello nel v\[C]iaggio
\[B&] mi hai te\[C]nuto per \[F]ma\[C]no,
\[B&] mi hai te\[C]nuto per \[F]ma\[C]no, 
\[B&] mi hai te\[C]nuto per \[F]ma\[F]no, \[F] \[F]

\endchorus




%%%%%% EV. INTERMEZZO
\beginverse*
\vspace*{1.3\versesep}
{
	\nolyrics
	%\textnote{Intermezzo strumentale}
	
	\ifchorded

	%---- Prima riga -----------------------------
	\vspace*{-\versesep}
	\[C] \rep{4}


	\fi
	%---- Ev Indicazioni -------------------------			
	%\textnote{\textit{(ripetizione della strofa)}} 
	 
}
\vspace*{\versesep}
\endverse






%%%%% STROFA
\beginverse		%Oppure \beginverse* se non si vuole il numero di fianco
%\memorize 		% <<< DECOMMENTA se si vuole utilizzarne la funzione
\chordsoff		% <<< DECOMMENTA se vuoi una strofa senza accordi

\textnote{\textbf{Mistero della morte}}

Ho visto semi d’amore unire mani,
destini e corpi,
portare consolazione
a chi conobbe dolore e morte;
ho visto colmare abissi
con un abbraccio riconciliante,
ho visto una croce infissa
dove il perdono si fece sangue.

\vspace*{\versesep}

La strada si è fatta folla,
il monte, pietra dell’agonia.
Il cielo si è fatto tenda
della promessa e della memoria.
La voce si è fatta grido,
le mani preda della follia.
Il corpo si è fatto pane
per tutti gli uomini senza storia.

\endverse



%%%%% RITORNELLO
\beginchorus
%\textnote{\textbf{Rit.}}
\chordsoff

^Io non ho più pa^ura,
^vedo l’amore che ^scende la valle
e con^sola la grande pia^nura;
^ è sospinto alla ^foce 
^ sopra un legno cru^dele:
^ condivide la ^sorte di ogni crea^tura. ^ ^ ^


\endchorus
\beginchorus


Da\[F]vanti a una c\[C]roce, 
\[B&] ti sei pie\[C]gato,
le tue \[F]mani nel l\[C]egno,
\[B&] mi han te\[C]nuto per \[F]ma\[C]no,
\[B&] mi han te\[C]nuto per \[F]ma\[C]no, 
\[B&] mi han te\[C]nuto per \[F]ma\[F]no, \[F] \[F]

\endchorus




%%%%%% EV. INTERMEZZO
\beginverse*
\vspace*{1.3\versesep}
{
	\nolyrics
	%\textnote{Intermezzo strumentale}
	
	\ifchorded

	%---- Prima riga -----------------------------
	\vspace*{-\versesep}
	\[C] \rep{4}


	\fi
	%---- Ev Indicazioni -------------------------			
	%\textnote{\textit{(ripetizione della strofa)}} 
	 
}
\vspace*{\versesep}
\endverse






%%%%% STROFA
\beginverse		%Oppure \beginverse* se non si vuole il numero di fianco
%\memorize 		% <<< DECOMMENTA se si vuole utilizzarne la funzione
\chordsoff		% <<< DECOMMENTA se vuoi una strofa senza accordi

\textnote{\textbf{Mistero della risurrezione}}


La storia si fece luce durante un’ora di poesia.
L’amore divenne forza che,
dolce, vince ogni resistenza.
La terra si fece altare di imprevedibile liturgia.
La morte si fece figlia riconciliata dell’esistenza.

\vspace*{\versesep}

Fu allora che vidi il cielo
sacra dimora dell’infinito;
fu allora che amai la terra
forma possibile dell’assoluto;
feconda di una speranza
che lega volti, cuori e destini,
accolgo la verità
fino a che tutto non sia compiuto.

\endverse



%%%%% RITORNELLO
\beginchorus
%\textnote{\textbf{Rit.}}
\chordsoff

^Io non ho più pa^ura,
^vedo una grande dis^tesa di pace
acc^ogliere l’acqua più ^pura:
^ chi è vicino alla ^fonte, 
^ chi già lascia la ^valle,
^ chi è già parte di ^Dio nella grande dis^tesa. ^ ^ ^



\endchorus
\beginchorus


Da\[F]vanti al mio \[C]nome 
\[B&] ti sei fer\[C]mato
e dal \[F]primo res\[C]piro
\[B&] mi hai te\[C]nuto per \[F]ma\[C]no,
\[B&] mi hai te\[C]nuto per \[F]ma\[C]no, 
\[B&] mi hai te\[C]nuto per \[F]mano...





\endchorus







%%%%%% EV. FINALE
\ifchorded
\beginchorus %oppure \beginverse*
\vspace*{1.3\versesep}
\textnote{Finale strumentale} %<<< EV. INDICAZIONI
\nolyrics

\[C] \[B&] \[C] \[B&] \[C] 

%---- Ogni riga successiva -------------------
\vspace*{-\versesep}
\[F] \[F] \[F] \[F*]

\endchorus  %oppure \endverse
\fi



\endsong
%------------------------------------------------------------
%			FINE CANZONE
%------------------------------------------------------------



