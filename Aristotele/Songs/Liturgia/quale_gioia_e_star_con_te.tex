%-------------------------------------------------------------
%			INIZIO	CANZONE
%-------------------------------------------------------------


%titolo: 	Quale gioia è star con te
%autore: 	D. Branca, L. Ciancio
%tonalita: 	Sol 



%%%%%% TITOLO E IMPOSTAZONI
\beginsong{Quale gioia è star con te}[by={D. Branca, L. Ciancio}]	% <<< MODIFICA TITOLO E AUTORE
\transpose{0} 						% <<< TRASPOSIZIONE #TONI (0 nullo)
%\preferflats  %SE VOGLIO FORZARE i bemolle come alterazioni
%\prefersharps %SE VOGLIO FORZARE i # come alterazioni
\momenti{Comunione}							% <<< INSERISCI MOMENTI	
% momenti vanno separati da ; e vanno scelti tra:
% Ingresso; Atto penitenziale; Acclamazione al Vangelo; Dopo il Vangelo; Offertorio; Comunione; Ringraziamento; Fine; Santi; Pasqua; Avvento; Natale; Quaresima; Canti Mariani; Battesimo; Prima Comunione; Cresima; Matrimonio; Meditazione; Spezzare del pane;
\ifchorded
	%\textnote{Tonalità migliore }	% <<< EV COMMENTI (tonalità originale/migliore)
\fi


%%%%%% INTRODUZIONE
\ifchorded
\vspace*{\versesep}
\musicnote{
\begin{minipage}{0.48\textwidth}
\textbf{Intro}
\hfill 
%( \eighthnote \, 80)   % <<  MODIFICA IL TEMPO
% Metronomo: \eighthnote (ottavo) \quarternote (quarto) \halfnote (due quarti)
\end{minipage}
} 	
\vspace*{-\versesep}
\beginverse*

\nolyrics

%---- Prima riga -----------------------------
\vspace*{-\versesep}
\[D] \[A] \[E-] \[B-] % \[*D] per indicare le pennate, \rep{2} le ripetizioni

%---- Ogni riga successiva -------------------
\vspace*{-\versesep}
 \[D] \[A] \[G] \[A] \rep{2}


%---- Ev Indicazioni -------------------------			
%\textnote{\textit{(Oppure tutta la strofa)} }	

\endverse
\fi



%%%%% STROFA
\beginverse		%Oppure \beginverse* se non si vuole il numero di fianco
\memorize 		% <<< DECOMMENTA se si vuole utilizzarne la funzione
%\chordsoff		% <<< DECOMMENTA se vuoi una strofa senza accordi
\[D4]Ogni volta che \[D]ti cerco, \brk \[E-] ogni volta che ti invoco,
\[D]Sempre mi ac\[B-]cogli Si\[A4]gnor. \[A] 
\[B-7]Grandi sono i tuoi prodigi,\brk \[G2] Tu sei buono verso tutti, 
\[D/F#]Santo Tu \[G]regni tra \[A4]noi. \[A] 
\endverse


%%%%% RITORNELLO
\beginchorus
\textnote{\textbf{Rit.}}
Qua-le \[D]gioia è star con \[A]Te Gesù \[E-]vivo e vi\[B-]cino,
\[D]bello è dar \[A]lode a Te, \[G]Tu sei il Si\[A]gnor.
Quale \[D]dono è aver cre\[A]duto in Te \brk \[E-]che non mi abban\[B-]doni,
\[D]Io per sempre a\[A]biterò 
\[G]la Tua \[A]casa, mio \[D]Re.
\endchorus





%%%%%% EV. INTERMEZZO
\beginverse*
\vspace*{1.3\versesep}
{
	\nolyrics
	\textnote{Intermezzo strumentale}
	
	\ifchorded

	%---- Prima riga -----------------------------
    \vspace*{-\versesep}
     \[A] \[E-] \[B-] % \[*D] per indicare le pennate, \rep{2} le ripetizioni

    %---- Ogni riga successiva -------------------
    \vspace*{-\versesep}
    \[D] \[A] \[G] \[A] 


	\fi
	%---- Ev Indicazioni -------------------------			
	%\textnote{\textit{(ripetizione della strofa)}} 
	 
}
\vspace*{\versesep}
\endverse





%%%%% STROFA
\beginverse		%Oppure \beginverse* se non si vuole il numero di fianco
%\memorize 		% <<< DECOMMENTA se si vuole utilizzarne la funzione
%\chordsoff		% <<< DECOMMENTA se vuoi una strofa senza accordi
^Hai guarito il ^mio dolore, \brk ^ hai cambiato questo cuore,
^oggi ri^nasco, Si^gnor. ^
^Grandi sono i tuoi prodigi, \brk ^ Tu sei buono verso tutti, 
^santo Tu ^regni tra ^noi. ^
\endverse


%%%%% RITORNELLO
\beginchorus
\textnote{\textbf{Rit.}}
Qua-le \[D]gioia è star con \[A]Te Gesù \[E-]vivo e vi\[B-]cino,
\[D]bello è dar \[A]lode a Te, \[G]Tu sei il Si\[A]gnor.
Quale \[D]dono è aver cre\[A]duto in Te \brk \[E-]che non mi abban\[B-]doni,
\[D]Io per sempre a\[A]biterò 
\[G]la Tua \[A]casa, mio \[D]Re.
\endchorus





%%%%%% EV. INTERMEZZO
\beginverse*
\vspace*{1.3\versesep}
{
	\nolyrics
	\textnote{Intermezzo strumentale}
	\textnote{(aumento di tonalità)}
	
	\ifchorded

	%---- Prima riga -----------------------------
    \vspace*{-\versesep}
    \[A] \[E-] \[B-] % \[*D] per indicare le pennate, \rep{2} le ripetizioni

    %---- Ogni riga successiva -------------------
    \vspace*{-\versesep}
    \[D] \[A] \[G] \[A]  \quad \[E]


	\fi
	%---- Ev Indicazioni -------------------------			
	%\textnote{\textit{(ripetizione della strofa)}} 
	 
}
\vspace*{\versesep}
\endverse



%%%%% STROFA
\beginverse		%Oppure \beginverse* se non si vuole il numero di fianco
%\memorize 		% <<< DECOMMENTA se si vuole utilizzarne la funzione
%\chordsoff		% <<< DECOMMENTA se vuoi una strofa senza accordi
\[E]Hai salvato la mia vita, \[F#-7]hai aperto la mia bocca,
\[E/G#]Canto per \[C#-]Te, mio Si\[B4]gnor.\[B] 
\[C#-7]Grandi sono i tuoi prodigi, \brk \[A]tu sei buono verso tutti,
\[E]santo Tu \[A]regni tra \[B]noi. \[B] 
\endverse




%%%%% RITORNELLO

\beginchorus
\textnote{\textbf{Rit.}}
Quale \[E]gioia è star con \[B]Te Gesù \[F#-]vivo e \[C#-]vicino,
\[E]Bello è dar \[B7]lode a Te, \[A]Tu sei il Si\[B7]gnor.
Quale \[E]dono è aver creduto in \[B]Te \brk \[F#-]che non mi abban\[C#-]doni.
\[E]Io per sempre a\[B7]biterò 
\[A]la Tua \[B]casa, \brk mio \[E4/C#]Re.
\endchorus

%%%%% BRIDGE
\beginverse*		%Oppure \beginverse* se non si vuole il numero di fianco
%\memorize 		% <<< DECOMMENTA se si vuole utilizzarne la funzione
%\chordsoff		% <<< DECOMMENTA se vuoi una strofa senza accordi
\vspace*{1.3\versesep}
\textnote{\textbf{Bridge}}
Ti lode\[B4/G#]rò \echo{ti loderò}, \brk Ti adore\[C#4]rò \echo{ti adorerò}
Ti cante\[A/B]rò, cante\[B&/C]remo.
\endverse




%%%%% RITORNELLO
\prefersharps
\beginchorus
\textnote{\textbf{Rit.}}
Quale \[F]gioia è star con \[C7]Te Gesù \brk \[G-]vivo e \[D-]vicino,
\[F]bello è dar \[C7]lode a Te, \brk \[B&]Tu sei il Si\[C7]gnor.
Quale \[F]dono è aver cre\[C7]duto in Te \brk \[G-]che non mi abban\[D-]doni.
\[F]Io per sempre a\[C7]biterò 
\[B&]la Tua \[C]casa, \brk mio \[F4]Re. \[F]
\[C4]la Tua casa, mio \[F4]Re. \[F] 
\[C4]Tu sei il Signor... \quad \[F*]mio \[F*]Re!
\endchorus  %oppure \endverse








\endsong
%------------------------------------------------------------
%			FINE CANZONE
%------------------------------------------------------------




