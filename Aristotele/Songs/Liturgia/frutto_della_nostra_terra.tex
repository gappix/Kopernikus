%titolo{Frutto della nostra terra}
%autore{Buttazzo}
%-------------------------------------------------------------
%			INIZIO	CANZONE
%-------------------------------------------------------------


%titolo: 	Frutto della nostra terra
%autore: 	Buttazzo
%tonalita: 	Sol 

%%%%%% TITOLO E IMPOSTAZONI
\beginsong{Frutto della nostra terra}[by={Buttazzo}]
\transpose{0} 						% <<< TRASPOSIZIONE #TONI (0 nullo)
\momenti{Offertorio}							% <<< INSERISCI MOMENTI	
% momenti vanno separati da ; e vanno scelti tra:
% Ingresso; Atto penitenziale; Acclamazione al Vangelo; Dopo il Vangelo; Offertorio; Comunione; Ringraziamento; Fine; Santi; Pasqua; Avvento; Natale; Quaresima; Canti Mariani; Battesimo; Prima Comunione; Cresima; Matrimonio; Meditazione;
\ifchorded
	%\textnote{Tonalità originale }	% <<< EV COMMENTI (tonalità originale/migliore)
\fi





%%%%%% INTRODUZIONE
\ifchorded
\vspace*{\versesep}
\textnote{Intro: \qquad \qquad  }%(\eighthnote 116) % << MODIFICA IL TEMPO
% Metronomo: \eighthnote (ottavo) \quarternote (quarto) \halfnote (due quarti)
\vspace*{-\versesep}
\beginverse*

\nolyrics

%---- Prima riga -----------------------------
\vspace*{-\versesep}
\[G] \[D]  \[C]	 \[D] \[D]% \[*D] per indicare le pennate, \rep{2} le ripetizioni

%---- Ogni riga successiva -------------------
%\vspace*{-\versesep}
%\[G] \[C]  \[D]	

%---- Ev Indicazioni -------------------------			
\textnote{\textit{(Melodia del pianoforte o clarinetto)} }	

\endverse
\fi



%%%%% STROFA
\beginverse		%Oppure \beginverse* se non si vuole il numero di fianco
\memorize 		% <<< DECOMMENTA se si vuole utilizzarne la funzione
%\chordsoff		& <<< DECOMMENTA se vuoi una strofa senza accordi
\[G]Frutto della nostra \[C]terra 
\[G]del lavoro di ogni \[D]uomo
\[E-]pane della nostra \[B-]vita, 
cibo \[C]della quotidiani\[D]tà. \[D]
\[G]Tu che lo prendevi un \[C]giorno, 
 \[G]lo spezzavi per i \[D]tuoi
\[E-]oggi vieni in questo \[B-]pane, 
 cibo \[C]vero dell'umani\[D]tà.
\endverse




%%%%% RITORNELLO
\beginchorus
\textnote{\textbf{Rit.}}
E sarò \[G]pane, e sarò \[D]vino,
nella mia \[E-]vita, nelle tue \[B-]mani.
Ti accoglie\[C]rò dentro di \[D]me,
farò di \[E-]me un'offerta \[C]viva,
un sacri\[A-]ficio \[D] gradito a \[G]Te.
\endchorus





%%%%%% EV. INTERMEZZO
\beginverse*
\vspace*{1.3\versesep}
{
	\nolyrics
	\textnote{Breve intermezzo strumentale}
	
	\ifchorded

	%---- Prima riga -----------------------------
	\vspace*{-\versesep}
	\[C]  \[D]	 \[C]  

	\fi
	%---- Ev Indicazioni -------------------------			
	%\textnote{\textit{(ripetizione della strofa)}} 
	 
}
\vspace*{\versesep}
\endverse



%%%%% STROFA
\beginverse		%Oppure \beginverse* se non si vuole il numero di fianco
%\memorize 		% <<< DECOMMENTA se si vuole utilizzarne la funzione
%\chordsoff		% <<< DECOMMENTA se vuoi una strofa senza accordi
%\chordsoff
^Frutto della nostra ^terra
 ^del lavoro di ogni ^uomo
^vino delle nostre ^vigne 
 sulla ^mensa dei fratelli ^tuoi.  ^
^Tu che lo prendevi un ^giorno, 
 ^lo bevevi con i ^tuoi
^oggi vieni in questo ^vino 
 e ti ^doni per la vita ^mia.
\endverse





%%%%% RITORNELLO
\beginchorus
\textnote{\textbf{Rit.}}
E sarò \[G]pane, e sarò \[D]vino,
nella mia \[E-]vita, nelle tue \[B-]mani.
Ti accoglie\[C]rò dentro di \[D]me,
farò di \[E-]me un'offerta \[C]viva,
un sacri\[A-]ficio \[D] gradito a \[E-]Te,
\[C] un sacri\[A-]ficio \[D] gradito a \[G]Te.
\endchorus





%%%%%% EV. INTERMEZZO
\beginverse*
\vspace*{1.3\versesep}
{
	\nolyrics
	\musicnote{Chiusura strumentale}
	
	\ifchorded

	%---- Prima riga -----------------------------
	\vspace*{-\versesep}
	\[C] \[G]  \[C]	 \[*G] 


	\fi
	%---- Ev Indicazioni -------------------------			
	%\textnote{\textit{(ripetizione della strofa)}} 
	 
}
\vspace*{\versesep}
\endverse


\endsong
%------------------------------------------------------------
%			FINE CANZONE
%------------------------------------------------------------


