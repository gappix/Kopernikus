%-------------------------------------------------------------
%			INIZIO	CANZONE
%-------------------------------------------------------------


%titolo: 	Andate per le strade
%autore: 	Roncari, Capello
%tonalita: 	Si-



%%%%%% TITOLO E IMPOSTAZONI
\beginsong{Andate per le strade}[by={Roncari, Capello}] 	% <<< MODIFICA TITOLO E AUTORE
\transpose{0} 						% <<< TRASPOSIZIONE #TONI (0 nullo)
\momenti{Fine}							% <<< INSERISCI MOMENTI	
% momenti vanno separati da ; e vanno scelti tra:
% Ingresso; Atto penitenziale; Acclamazione al Vangelo; Dopo il Vangelo; Offertorio; Comunione; Ringraziamento; Fine; Santi; Pasqua; Avvento; Natale; Quaresima; Canti Mariani; Battesimo; Prima Comunione; Cresima; Matrimonio; Meditazione;
\ifchorded
	%\textnote{Tonalità originale }	% <<< EV COMMENTI (tonalità originale/migliore)
\fi


%%%%%% INTRODUZIONE
\ifchorded
\vspace*{\versesep}
\textnote{Intro: \qquad \qquad  }%(\eighthnote 116) % << MODIFICA IL TEMPO
% Metronomo: \eighthnote (ottavo) \quarternote (quarto) \halfnote (due quarti)
\vspace*{-\versesep}
\beginverse*

\nolyrics

%---- Prima riga -----------------------------
\vspace*{-\versesep}
\[B-] \[A] \[*B-]	 % \[*D] per indicare le pennate, \rep{2} le ripetizioni

%---- Ogni riga successiva -------------------
%\vspace*{-\versesep}
%\[G] \[C]  \[D]	

%---- Ev Indicazioni -------------------------			
%\textnote{\textit{(Oppure tutta la strofa)} }

\endverse
\fi

%%%%% RITORNELLO
\beginchorus
\textnote{\textbf{Rit.}}

An\[(B-)]date per le \[D]strade in \[G]tutto il \[A]mondo,
chia\[F#]mate i miei a\[B-]mici \[A]per far \[D]festa,
c'è un \[F#]posto per cia\[G]scuno \[A]alla mia \[B-]mensa.  \[*B-]

\endchorus

%%%%% STROFA
\beginverse		%Oppure \beginverse* se non si vuole il numero di fianco
\memorize 		% <<< DECOMMENTA se si vuole utilizzarne la funzione
%\chordsoff		& <<< DECOMMENTA se vuoi una strofa senza accordi

Nel \[(D)]vostro cam\[G]mino annun\[A]ciate il Van\[D]gelo 
di\[B-]cendo “È vi\[E-]cino il \[F#]Regno dei \[B-]cieli!”.
Gua\[D7]rite i ma\[G]lati, mon\[A]date i leb\[D]brosi,
ren\[B-]dete la \[F#-]vita a \[C#]chi l'ha per\[F#]duta. \[F#] \quad \[*B-] 

\endverse

\beginverse
%\chordsoff
^Vi è stato do^nato con a^more gra^tuito: 
u^gualmente do^nate con ^gioia e per a^more. 
Con ^voi non pren^dete ^né oro né ar^gento, 
per^ché l'ope^raio ha di^ritto al suo ^cibo. ^ \quad ^
\endverse
\beginverse
%\chordsoff
En^trando in una ^casa do^natele la ^pace: 
se ^c'è chi vi ri^fiuta e ^non accoglie il ^dono, 
la ^pace torni a ^voi e u^scite dalla ^casa, 
^scuotendo la ^polvere dai ^vostri cal^zari.  ^ \quad ^
\endverse
\beginverse
%\chordsoff
^Ecco, io vi ^mando a^gnelli in mezzo ai ^lupi: 
siate ^dunque avve^duti come ^sono i ser^penti, 
ma ^liberi e ^chiari co^me le co^lombe; 
do^vrete soppor^tare pri^gioni e tribu^nali. ^ \quad ^
\endverse
\beginverse
%\chordsoff
Nes^suno è più ^grande del ^proprio ma^estro, 
né il ^servo è più im^portante ^del suo pa^drone. 
Se ^hanno odiato ^me, odie^ranno anche ^voi; 
ma ^voi non te^mete, io non ^vi lascio ^soli! ^ \quad ^
\endverse

\endsong
%------------------------------------------------------------
%			FINE CANZONE
%------------------------------------------------------------