%-------------------------------------------------------------
%			INIZIO	CANZONE
%-------------------------------------------------------------


%titolo: 	Re di gloria
%autore: 	F. Marranzino, A. De\ LucaDaniele Ricci
%tonalita: 	do



%%%%%% TITOLO E IMPOSTAZONI
\beginsong{Re di gloria}[by={F. Marranzino, A. De\ Luca}]	% <<< MODIFICA TITOLO E AUTORE
\transpose{-3} 						% <<< TRASPOSIZIONE #TONI (0 nullo)
%\preferflats  %SE VOGLIO FORZARE i bemolle come alterazioni
%\prefersharps %SE VOGLIO FORZARE i # come alterazioni
\momenti{}							% <<< INSERISCI MOMENTI	
% momenti vanno separati da ; e vanno scelti tra:
% Ingresso; Atto penitenziale; Acclamazione al Vangelo; Dopo il Vangelo; Offertorio; Comunione; Ringraziamento; Fine; Santi; Pasqua; Avvento; Natale; Quaresima; Canti Mariani; Battesimo; Prima Comunione; Cresima; Matrimonio; Meditazione; Spezzare del pane;
\ifchorded
	%\textnote{Tonalità migliore }	% <<< EV COMMENTI (tonalità originale/migliore)
\fi


%%%%%% INTRODUZIONE
\ifchorded
\vspace*{\versesep}
\musicnote{
\begin{minipage}{0.48\textwidth}
\textbf{Intro}
\hfill 
%( \eighthnote \, 80)   % <<  MODIFICA IL TEMPO
% Metronomo: \eighthnote (ottavo) \quarternote (quarto) \halfnote (due quarti)
\end{minipage}
} 	
\vspace*{-\versesep}
\beginverse*

\nolyrics

%---- Prima riga -----------------------------
\vspace*{-\versesep}
\[G] \[C] \[D] \[G] \[G]	 % \[*D] per indicare le pennate, \rep{2} le ripetizioni

%---- Ogni riga successiva -------------------
%\vspace*{-\versesep}
%\[G] \[C]  \[D]	

%---- Ev Indicazioni -------------------------			
%\textnote{\textit{(Oppure tutta la strofa)} }	

\endverse
\fi



%%%%% STROFA
\beginverse		%Oppure \beginverse* se non si vuole il numero di fianco
\memorize 		% <<< DECOMMENTA se si vuole utilizzarne la funzione
%\chordsoff		% <<< DECOMMENTA se vuoi una strofa senza accordi
\[G]Ho incontrato Te Gesù \brk e ogni \[D]cosa in me è cambiata
\[A-]tutta la mia \[A-7]vita ora ti \[A-*]ap\[D*]par\[A-7*]tie\[D]ne
\[G]tutto il mio passato io lo a\ch{C*}{f}{f}{ff}i\[B7*]do a \[E-]Te  
Ge\[A-]sù Re di \[C7+]gloria mio Si\[D4]gnor\[D7].
\endverse





\beginverse*
^Tutto in Te riposa,  \brk la mia ^mente il mio cuore
^trovo pace in ^Te Signor,  \brk Tu mi ^dai ^la ^gio^ia
^voglio stare insieme a Te,  \brk non lasc^iar^ti ^mai 
Ge^sù Re di ^gloria  \brk mio Si^gnor.^
\endverse





%%%%% RITORNELLO
\beginchorus
\textnote{\textbf{Rit.}}
Dal Tuo a\[G]more chi \[C7+]mi separe\[D4]rà \[D] 
sulla \[A-7]croce hai \[C7+]dato la vita per \[D]me \[D]
una co\[G]rona di \[C7+]gloria mi da\[D4]rai \[D] 
quando un \[A-7*]gior\[C*]no \[D]ti ve\[G]drò.
\endchorus





%%%%% STROFA
\beginverse		%Oppure \beginverse* se non si vuole il numero di fianco
%\memorize 		% <<< DECOMMENTA se si vuole utilizzarne la funzione
%\chordsoff		% <<< DECOMMENTA se vuoi una strofa senza accordi
^Tutto in Te riposa,  \brk la mia ^mente il mio cuore
^trovo pace in ^Te Signor,  \brk Tu mi ^dai ^la ^gio^ia
^voglio stare insieme a Te,  \brk non lasc^iar^ti ^mai 
Ge^sù Re di ^gloria  \brk mio Si^gnor.^
\endverse



%%%%% RITORNELLO
\beginchorus
\textnote{\textbf{Rit.}}
Dal Tuo a\[G]more chi \[C7+]mi separe\[D4]rà \[D] 
sulla \[A-7]croce hai \[C7+]dato la vita per \[D]me \[D]
una co\[G]rona di \[C7+]gloria mi da\[D4]rai \[D] 
quando un \[A-7*]gior\[C*]no \[D]ti ve\[E&]drò. \[E&]

\vspace*{0.2\versesep}
\textnote{\textit{(si alza la tonalità)}}

\preferflats %---------
\[E&7]Dal Tuo a\[A&]more chi \[D&7+]mi separe\[E&]rà \ldots
\endchorus

\preferflats %------------

%%%%%% EV. INTERMEZZO
\beginverse*
\vspace*{0.3\versesep}
{
	\nolyrics
	\textnote{\textit{(continua strumentale)}}
	
	\ifchorded

	%---- Prima riga -----------------------------
	\vspace*{-\versesep}
	\[B&-7] \[D&7+] \[E&]  \[E&7]

	%---- Ogni riga successiva -------------------
	\vspace*{-\versesep}
	\[A&] \[D&7+] \[E&4] \[E&]


	\fi
	%---- Ev Indicazioni -------------------------			
	%\textnote{\textit{(ripetizione della strofa)}} 
	 
}
\vspace*{\versesep}
\endverse
\beginverse*\bfseries
Io ti a\[B&-7*]spet\[D&7+*]to \[E&]mio Signo\[F-]r \[A&] 
io ti a\[B&-7*]spet\[A&*]to \[C]mio Signo\[F-]r \[A&] 
io ti a\[B&-7*]spet\[A&*]to \[E&]mio \[A&4]Re! \[A&] \[A&*]
\endverse



\endsong
%------------------------------------------------------------
%			FINE CANZONE
%------------------------------------------------------------

