%-------------------------------------------------------------
%			INIZIO	CANZONE
%-------------------------------------------------------------


%titolo: 	Invochiamo la tua presenza
%autore: 	A. Napolitano, D. Bruno
%tonalita:  Re-



%%%%%% TITOLO E IMPOSTAZONI
\beginsong{Invochiamo la tua presenza}[by={A. Napolitano, D. Bruno}] 	% <<< MODIFICA TITOLO E AUTORE
\transpose{-1} 						% <<< TRASPOSIZIONE #TONI (0 nullo)
%\preferflats  %SE VOGLIO FORZARE i bemolle come alterazioni
%\prefersharps %SE VOGLIO FORZARE i # come alterazioni
\momenti{Cresima; Ringraziamento; Meditazione}							% <<< INSERISCI MOMENTI	
% momenti vanno separati da ; e vanno scelti tra:
% Ingresso; Atto penitenziale; Acclamazione al Vangelo; Dopo il Vangelo; Offertorio; Comunione; Ringraziamento; Fine; Santi; Pasqua; Avvento; Natale; Quaresima; Canti Mariani; Battesimo; Prima Comunione; Cresima; Matrimonio; Meditazione; Spezzare del pane;
\ifchorded
	%\textnote{Tonalità migliore }	% <<< EV COMMENTI (tonalità originale\migliore)
\fi

%%%%%% INTRODUZIONE
\ifchorded
\vspace*{\versesep}
\musicnote{
\begin{minipage}{0.48\textwidth}
\textbf{Intro}
\hfill 
(\quarternote  \, 60) 
%( \eighthnote \, 80)   % <<  MODIFICA IL TEMPO
% Metronomo: \eighthnote (ottavo) \quarternote (quarto) \halfnote (due quarti)
\end{minipage}
} 	
\vspace*{-\versesep}
\beginverse*


\nolyrics

%---- Prima riga -----------------------------
\vspace*{-\versesep}
\[D-] \[B&]  \[F] \[C]	 \rep{2} % \[*D] per indicare le pennate, \rep{2} le ripetizioni

%---- Ogni riga successiva -------------------
%\vspace*{-\versesep}
%\[G] \[C]  \[D]	

%---- Ev Indicazioni -------------------------			
%\textnote{\textit{(Oppure tutta la strofa)} }	

\endverse
\fi




%%%%% STROFA
\beginverse		%Oppure \beginverse* se non si vuole il numero di fianco
\memorize 		% <<< DECOMMENTA se si vuole utilizzarne la funzione
%\chordsoff		% <<< DECOMMENTA se vuoi una strofa senza accordi
\[D-]Invochiamo la \[B&]tua presenza,   \brk \[F]vieni Si\[C]gnor
\[D-]Invochiamo la \[B&]tua presenza,   \brk \[F]scendi su di \[C]noi.
\[G-]Vieni consola\[D-]tore,   \brk dona \[B&]pace ed um\[C]iltà.
\[G-]Acqua viva dà \[D-]amore,   \brk questo \[B&]cuore apriamo a \[A]te. \[A]
\endverse




%%%%% RITORNELLO
\beginchorus
\textnote{\textbf{Rit.}}
\[D-]Vieni spirito, \[B&]vieni spirito,  \brk \[F]scendi su di no\[C]i
\[D-]Vieni spirito, \[B&]vieni spirito,  \brk \[F]scendi su di no\[C]i.
\[B&]Vieni su \[C]noi \[A]Maranath\[D-]à,  \brk \[B&]vieni su \[C]noi spi-ri\textit{(-to.)}
\[D-]Vieni spirito, \[B&]vieni spirito, \brk  \[F]scendi su di no\[C]i
\[D-]Vieni spirito, \[B&]vieni spirito, \brk  \[F]scendi su di no\[C]i.
Scendi su di no\[D-]i  
\endchorus



%%%%%% EV. INTERMEZZO
\beginverse*
\vspace*{1.3\versesep}
{
	\nolyrics
	\textnote{Intermezzo strumentale}
	
	\ifchorded

	%---- Prima riga -----------------------------
	\vspace*{-\versesep}
     \[B&]  \[F] \[C]	 
	%---- Ogni riga successiva -------------------
	\vspace*{-\versesep}
	\[D-]  \[B&]  \[F] \[C]


	\fi
	%---- Ev Indicazioni -------------------------			
	%\textnote{\textit{(ripetizione della strofa)}} 
	 
}
\vspace*{\versesep}
\endverse


%%%%% STROFA
\beginverse		%Oppure \beginverse* se non si vuole il numero di fianco
%\memorize 		% <<< DECOMMENTA se si vuole utilizzarne la funzione
%\chordsoff		% <<< DECOMMENTA se vuoi una strofa senza accordi

\[D-]Invochiamo la \[B&]tua presenza, \brk \[F]vieni Si\[C]gnor
\[D-]Invochiamo la \[B&]tua presenza, \brk  \[F]scendi su di \[C]noi.
\[G-]Vieni luce dei \[D-]cuori, \brk  dona \[B&]forza e fe\[C]deltà.
\[G-]Fuoco eterno d'\[D-]amore,  \brk questa \[B&]vita offriamo a \[A]te. \[A]

\endverse



%%%%% RITORNELLO
\beginchorus
\textnote{\textbf{Rit.}}
\[D-]Vieni spirito, \[B&]vieni spirito,  \brk \[F]scendi su di no\[C]i
\[D-]Vieni spirito, \[B&]vieni spirito,  \brk \[F]scendi su di no\[C]i.
\[B&]Vieni su \[C]noi \[A]Maranath\[D-]à,  \brk \[B&]vieni su \[C]noi spi-ri\textit{(-to.)}
\[D-]Vieni spirito, \[B&]vieni spirito, \brk  \[F]scendi su di no\[C]i
\[D-]Vieni spirito, \[B&]vieni spirito, \brk  \[F]scendi su di no\[C]i.
Scendi su di \echo{noi}  
\endchorus
%%%%%INTERLUDIO
\beginverse*		
\musicnote{\textit{[dolce, arpeggiato]}}
\[D-]Vieni spirito, \[B&]vieni spirito, \brk  \[F]scendi su di no\[C]i
\[D-]Vieni spirito, \[B&]vieni spirito, \brk  \[F]scendi su di no\[C]i. \[C]
\endverse
\beginchorus
\textnote{[cambio di tonalità]}
\transpose{3}
\[D-]Vieni spirito, \[B&]vieni spirito,  \brk \[F]scendi su di no\[C]i
\[D-]Vieni spirito, \[B&]vieni spirito,  \brk \[F]scendi su di no\[C]i.
\[B&]Vieni su \[C]noi \[A]Maranath\[D-]à,  \brk \[B&]vieni su \[C]noi spi-ri\textit{(-to.)}
\[D-]Vieni spirito, \[B&]vieni spirito, \brk  \[F]scendi su di no\[C]i
\[D-]Vieni spirito, \[B&]vieni spirito, \brk  \[F]scendi su di no\[C]i.
Scendi su di no\[D-]i  
\endchorus



%%%%%% EV. CHIUSURA SOLO STRUMENTALE
\beginverse*
\vspace*{1.3\versesep}
{   
    \ifchorded
	\nolyrics
	\textnote{Chiusura strumentale}
	\transpose{3}
	

	%---- Prima riga -----------------------------
	\vspace*{-\versesep}
     \[B&]  \[F] \[C]	 
	%---- Ogni riga successiva -------------------
	\vspace*{-\versesep}
	\[D-]  \[B&]  \[F] \[C]


	\fi
	%---- Ev Indicazioni -------------------------			
	%\textnote{\textit{(ripetizione della strofa)}} 
	 
}
\vspace*{\versesep}
\endverse



\endsong
%------------------------------------------------------------
%			FINE CANZONE
%------------------------------------------------------------


