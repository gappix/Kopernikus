%-------------------------------------------------------------
%			INIZIO	CANZONE
%-------------------------------------------------------------


%titolo: 	Il pane del cammino
%autore: 	Sequeri
%tonalita: 	Do



%%%%%% TITOLO E IMPOSTAZONI
\beginsong{Il pane del cammino}[by={P. Sequeri}] 	% <<< MODIFICA TITOLO E AUTORE
\transpose{0} 						% <<< TRASPOSIZIONE #TONI (0 nullo)
\momenti{}							% <<< INSERISCI MOMENTI	
% momenti vanno separati da ; e vanno scelti tra:
% Ingresso; Atto penitenziale; Acclamazione al Vangelo; Dopo il Vangelo; Offertorio; Comunione; Ringraziamento; Fine; Santi; Pasqua; Avvento; Natale; Quaresima; Canti Mariani; Battesimo; Prima Comunione; Cresima; Matrimonio; Meditazione; Spezzare del pane;
\ifchorded
	%\textnote{Tonalità migliore }	% <<< EV COMMENTI (tonalità originale/migliore)
\fi


%%%%%% INTRODUZIONE
\ifchorded
\vspace*{\versesep}
\musicnote{
\begin{minipage}{0.48\textwidth}
\textbf{Intro}
\hfill 
( \eighthnote \, 76)   % <<  MODIFICA IL TEMPO
% Metronomo: \eighthnote (ottavo) \quarternote (quarto) \halfnote (due quarti)
\end{minipage}
} 	
\vspace*{-\versesep}
\beginverse*

\nolyrics

%---- Prima riga -----------------------------
\vspace*{-\versesep}
\[C] \[G] \[C]  % \[*D] per indicare le pennate, \rep{2} le ripetizioni

%---- Ogni riga successiva -------------------
%\vspace*{-\versesep}
%\[A*] \[E*] \[A]

%---- Ev Indicazioni -------------------------			
%\textnote{\textit{(Oppure tutta la strofa)} }	

\endverse
\fi



%%%%% RITORNELLO
\beginchorus
\textnote{\textbf{Rit.}}
\[C]Il tuo \[G7]popolo in cam\[A-7]mi\[C]no
\[F]cerca in \[D7]te la \[C4]gui\[G7]da.
\[C]Sulla \[G7]strada verso il \[A-7]re\[C]gno
\[F]sei so\[D7]stegno col tuo \[C]cor\[G7]po:
\[E-]resta \[A7]sempre con \[D-]noi, 
\[F]o Sig\[C]no-\[G]o-\[C]re!
\endchorus




%%%%% STROFA
\beginverse		%Oppure \beginverse* se non si vuole il numero di fianco
\memorize 		% <<< DECOMMENTA se si vuole utilizzarne la funzione
%\chordsoff		% <<< DECOMMENTA se vuoi una strofa senza accordi
È il tuo \[C-]pane, Ge\[F-]sù, che ci dà \[C-]forza
e \[C-]rende più si\[B&]curo il nostro \[C-]passo.
Se il vi\[C-]gore nel cam\[G7]mino si svi\[G#]lisce 
la tua \[F-]mano dona l\[D7]ieta la spe\[G4]ranza.
\endverse


%%%%% STROFA
\beginverse		%Oppure \beginverse* se non si vuole il numero di fianco
%\memorize 		% <<< DECOMMENTA se si vuole utilizzarne la funzione
\chordsoff		% <<< DECOMMENTA se vuoi una strofa senza accordi
È il tuo vino, Gesù, che ci disseta
e sveglia in noi l'ardore di seguirti.
Se la gioia cede il passo alla stanchezza,
la tua voce fa rinascere freschezza.
\endverse




%%%%% STROFA
\beginverse		%Oppure \beginverse* se non si vuole il numero di fianco
%\memorize 		% <<< DECOMMENTA se si vuole utilizzarne la funzione
\chordsoff		% <<< DECOMMENTA se vuoi una strofa senza accordi
È il tuo Corpo, Gesù, che ci fa Chiesa,
fratelli sulle strade della vita.
Se il rancore toglie luce all’amicizia,
dal tuo cuore nasce giovane il perdono.
\endverse



%%%%% STROFA
\beginverse		%Oppure \beginverse* se non si vuole il numero di fianco
%\memorize 		% <<< DECOMMENTA se si vuole utilizzarne la funzione
\chordsoff		% <<< DECOMMENTA se vuoi una strofa senza accordi
È il tuo Sangue, Gesù, il segno eterno
dell’unico linguaggio dell’amore.
Se il donarsi come te richiede fede,
nel tuo Spirito sfidiamo l’incertezza.
\endverse



%%%%% STROFA
\beginverse		%Oppure \beginverse* se non si vuole il numero di fianco
%\memorize 		% <<< DECOMMENTA se si vuole utilizzarne la funzione
\chordsoff		% <<< DECOMMENTA se vuoi una strofa senza accordi
È il tuo Dono, Gesù, la vera fonte
del gesto coraggioso di chi annuncia.
Se la Chiesa non è aperta ad ogni uomo,
il tuo fuoco le rivela la missione.
\endverse



\endsong
%------------------------------------------------------------
%			FINE CANZONE
%------------------------------------------------------------


