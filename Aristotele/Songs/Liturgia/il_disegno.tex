%-------------------------------------------------------------
%			INIZIO	CANZONE
%-------------------------------------------------------------


%titolo: 	Il disegno
%autore: 	
%tonalita: 	La-



%%%%%% TITOLO E IMPOSTAZONI
\beginsong{Il disegno}[by={A. Marani}] 	% <<< MODIFICA TITOLO E AUTORE
\transpose{0} 						% <<< TRASPOSIZIONE #TONI (0 nullo)
\momenti{Comunione}							% <<< INSERISCI MOMENTI	
% momenti vanno separati da ; e vanno scelti tra:
% Ingresso; Atto penitenziale; Acclamazione al Vangelo; Dopo il Vangelo; Offertorio; Comunione; Ringraziamento; Fine; Santi; Pasqua; Avvento; Natale; Quaresima; Canti Mariani; Battesimo; Prima Comunione; Cresima; Matrimonio; Meditazione; Spezzare del pane;
\ifchorded
	%\textnote{Tonalità migliore }	% <<< EV COMMENTI (tonalità originale/migliore)
\fi

%%%%%% INTRODUZIONE
\ifchorded
\vspace*{\versesep}
\musicnote{
\begin{minipage}{0.48\textwidth}
\textbf{Intro}
\hfill 
%( \eighthnote \, 80)   % <<  MODIFICA IL TEMPO
% Metronomo: \eighthnote (ottavo) \quarternote (quarto) \halfnote (due quarti)
\end{minipage}
} 	
\vspace*{-\versesep}
\beginverse*

\nolyrics

%---- Prima riga -----------------------------
\vspace*{-\versesep}
\[A-] \[F] \[G] \[C] \[E7] \rep{2}	 % \[*D] per indicare le pennate, \rep{2} le ripetizioni

%---- Ogni riga successiva -------------------
%\vspace*{-\versesep}
%\[G] \[C]  \[D]	

%---- Ev Indicazioni -------------------------			
%\textnote{\textit{(Oppure tutta la strofa)} }	

\endverse
\fi



%%%%% STROFA
\beginverse		%Oppure \beginverse* se non si vuole il numero di fianco
\memorize 		% <<< DECOMMENTA se si vuole utilizzarne la funzione
%\chordsoff		% <<< DECOMMENTA se vuoi una strofa senza accordi
Nel \[A-]mare del si\[F]lenzio \brk una \[G]voce si al\[C]zò, \[E7]
da una \[A-]notte senza con\[F]fini \brk  una \[G]luce bril\[C]lò \[E7]
dove non \[A-]c'era niente quel \[E7]giorno.
\endverse



%%%%% RITORNELLO
\beginchorus
\textnote{\textbf{Rit.}}
A\[A-]vevi scritto \[D-]già  \brk il mio \[G]nome lassù nel \[C]cie\[E7]lo, 
a\[A-]vevi scritto \[D-]già \brk  la mia \[G]vita insieme a \[C]Te, \[E7]
avevi s\[A-]critto già di \[E7]me.
\endchorus





%%%%% STROFA
\beginverse		%Oppure \beginverse* se non si vuole il numero di fianco
%\memorize 		% <<< DECOMMENTA se si vuole utilizzarne la funzione
%\chordsoff		% <<< DECOMMENTA se vuoi una strofa senza accordi
E ^quando la tua ^mente \brk  fece ^splendere le ^stel^le,
e ^quando le tue ^mani  \brk model^larono la ^ter^ra
dove non ^c'era niente quel ^giorno.
\endverse

%%%%% STROFA
\beginverse		%Oppure \beginverse* se non si vuole il numero di fianco
%\memorize 		% <<< DECOMMENTA se si vuole utilizzarne la funzione
%\chordsoff		% <<< DECOMMENTA se vuoi una strofa senza accordi

E ^quando hai calco^lato  \brk la ^profondità del ^cie^lo, 
e ^quando hai colo^rato \brk  ogni ^fiore della ^ter^ra
dove non ^c'era niente quel ^giorno.

\endverse

%%%%% STROFA
\beginverse		%Oppure \beginverse* se non si vuole il numero di fianco
%\memorize 		% <<< DECOMMENTA se si vuole utilizzarne la funzione
%\chordsoff		% <<< DECOMMENTA se vuoi una strofa senza accordi

E ^quando hai dise^gnato  \brk le ^nubi e le mon^ta^gne,
e ^quando hai dise^gnato \brk il ^cammino di ogni ^uo^mo
l'avevi ^fatto anche per ^me.

\endverse


%%%%%% EV. FINALE

\beginchorus %oppure \beginverse*
\vspace*{1.3\versesep}
\textnote{\textbf{Finale} } %<<< EV. INDICAZIONI

Se \[A-]ieri non sa\[D-]pevo  \brk oggi \[G]ho incontrato \[C]Te \[E7]
e \[A-]la mia liber\[D-]tà \brk  è il tuo di\[G]segno su di \[C]me, \[E7]
non cerche\[A-]rò più niente per\[E7]ché
Tu mi salve\[A-]rai.

\endchorus 




%%%%%% EV. CHIUSURA SOLO STRUMENTALE
\ifchorded
\beginchorus %oppure \beginverse*
\vspace*{1.3\versesep}
\musicnote{Chiusura strumentale} %<<< EV. INDICAZIONI

\[D-] \[F] \[G] \[A-]

\endchorus  %oppure \endverse
\fi

\endsong
%------------------------------------------------------------
%			FINE CANZONE
%------------------------------------------------------------