%-------------------------------------------------------------
%			INIZIO	CANZONE
%-------------------------------------------------------------


%titolo: 	Noi saremo il pane
%autore: 	Fusco
%tonalita: 	Do 



%%%%%% TITOLO E IMPOSTAZONI
\beginsong{Noi saremo il pane}[by={M. G. Fusco}] 	% <<< MODIFICA TITOLO E AUTORE
\transpose{0} 						% <<< TRASPOSIZIONE #TONI (0 nullo)
\momenti{Offertorio; Prima Comunione; Spezzare del Pane}							% <<< INSERISCI MOMENTI	
% momenti vanno separati da ; e vanno scelti tra:
% Ingresso; Atto penitenziale; Acclamazione al Vangelo; Dopo il Vangelo; Offertorio; Comunione; Ringraziamento; Fine; Spezzare del Pane; Santi; Pasqua; Avvento; Natale; Quaresima; Canti Mariani; Battesimo; Prima Comunione; Cresima; Matrimonio; Meditazione;
\ifchorded
	%\textnote{Tonalità originale }	% <<< EV COMMENTI (tonalità originale/migliore)
\fi


%%%%%% INTRODUZIONE
\ifchorded
\vspace*{\versesep}
\musicnote{
\begin{minipage}{0.48\textwidth}
\textbf{Intro}
\hfill 
%( \eighthnote \, 80)   % <<  MODIFICA IL TEMPO
% Metronomo: \eighthnote (ottavo) \quarternote (quarto) \halfnote (due quarti)
\end{minipage}
} 	
\vspace*{-\versesep}
\beginverse*

\nolyrics

%---- Prima riga -----------------------------
\vspace*{-\versesep}
\[C] \[F]  \[C]	 \rep{2} % \[*D] per indicare le pennate, \rep{2} le ripetizioni

%---- Ogni riga successiva -------------------
%\vspace*{-\versesep}
%\[G] \[C]  \[D]	

%---- Ev Indicazioni -------------------------			
%\textnote{\textit{(Oppure tutta la strofa)} }	

\endverse
\fi




%%%%% STROFA
\beginverse		%Oppure \beginverse* se non si vuole il numero di fianco
\memorize 		% <<< DECOMMENTA se si vuole utilizzarne la funzione
%\chordsoff		& <<< DECOMMENTA se vuoi una strofa senza accordi

Un \[C]chicco da \[F]solo che \[C]fa?
Non fa un \[F]campo di grano né un \[G]pane!
Un \[C]chicco da \[F]solo non po\[C]trà
esser la \[F]gioia di chi ha \[G]fame!
\[D-7]Ma uniti in\[E-]sieme
tanti \[D-7]chicchi un solo \[G]pane!

\endverse




%%%%% RITORNELLO
\beginchorus
\textnote{\textbf{Rit.}}

\[C]Noi saremo il \[E-]pane, \[F]noi sarem l'a\[E-]more
\[F]noi sarem la \[G]gioia per un \[A-]mondo che ha
fame d'infi\[D-]ni\[G]to!
\[C]Noi saremo il \[E-]pane, \[F]noi sarem l'a\[E-]more
\[F]noi sarem la \[G]gioia per un \[A-]mondo che ha
\[G]fame di \[C]Te!

\endchorus



%%%%%% EV. INTERMEZZO
\beginverse*
\vspace*{1.3\versesep}
{
	\nolyrics
	\textnote{Intermezzo strumentale}
	
	\ifchorded

	%---- Prima riga -----------------------------
	\vspace*{-\versesep}
	\[C] \[F]  \[C]	 

	\fi
	%---- Ev Indicazioni -------------------------			
	%\textnote{\textit{(ripetizione della strofa)}} 
	 
}
\vspace*{\versesep}
\endverse




%%%%% STROFA
\beginverse		%Oppure \beginverse* se non si vuole il numero di fianco
%\memorize 		% <<< DECOMMENTA se si vuole utilizzarne la funzione
\chordsoff		% <<< DECOMMENTA se vuoi una strofa senza accordi

Un ^acino ^solo che ^fa?
Non è ^uva che matura sui ^colli!
Un ^uomo ^solo non po^trà
essere ^"segno" dell'a^more
^ma noi invi^tati
tutti in^sieme Chiesa ^viva!

\endverse






\endsong
%------------------------------------------------------------
%			FINE CANZONE
%------------------------------------------------------------

