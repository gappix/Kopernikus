%-------------------------------------------------------------
%			INIZIO	CANZONE
%-------------------------------------------------------------


%titolo: 	Come un prodigio
%autore: 	D. Vezzani
%tonalita: 	La-



%%%%%% TITOLO E IMPOSTAZONI
\beginsong{Come un prodigio}[by={D. Vezzani}] 	% <<< MODIFICA TITOLO E AUTORE
\transpose{0} 						% <<< TRASPOSIZIONE #TONI (0 nullo)
%\preferflats  %SE VOGLIO FORZARE i bemolle come alterazioni
%\prefersharps %SE VOGLIO FORZARE i # come alterazioni
\momenti{Comunione; Ringraziamento; Meditazione}							% <<< INSERISCI MOMENTI	
% momenti vanno separati da ; e vanno scelti tra:
% Ingresso; Atto penitenziale; Acclamazione al Vangelo; Dopo il Vangelo; Offertorio; Comunione; Ringraziamento; Fine; Santi; Pasqua; Avvento; Natale; Quaresima; Canti Mariani; Battesimo; Prima Comunione; Cresima; Matrimonio; Meditazione; Spezzare del pane;
\ifchorded
	%\textnote{Tonalità migliore }	% <<< EV COMMENTI (tonalità originale/migliore)
\fi



%%%%%% INTRODUZIONE
\ifchorded
\vspace*{\versesep}
\musicnote{
\begin{minipage}{0.48\textwidth}
\textbf{Intro}
\hfill 
%( \eighthnote \, 80)   % <<  MODIFICA IL TEMPO
% Metronomo: \eighthnote (ottavo) \quarternote (quarto) \halfnote (due quarti)
\end{minipage}
} 	
\vspace*{-\versesep}
\beginverse*

\nolyrics

%---- Prima riga -----------------------------
\vspace*{-\versesep}
\[A-] \[F] \[C] \[G]	 % \[*D] per indicare le pennate, \rep{2} le ripetizioni

%---- Ogni riga successiva -------------------
%\vspace*{-\versesep}
%\[G] \[C]  \[D]	

%---- Ev Indicazioni -------------------------			
%\textnote{\textit{(Oppure tutta la strofa)} }	

\endverse
\fi




%%%%% STROFA
\beginverse		%Oppure \beginverse* se non si vuole il numero di fianco
\memorize 		% <<< DECOMMENTA se si vuole utilizzarne la funzione
%\chordsoff		% <<< DECOMMENTA se vuoi una strofa senza accordi

Signore \[A-]Tu mi scruti, e co\[F]nosci,
sai quando \[C]seggo e quando mi \[G]alzo.
Riesci a ve\[A-]dere i miei pensi\[F]eri,
sai quando \[C]io cammino e \[G]quando riposo.
Ti \[B&]sono note tutte le mie \[F]vie,
la mia pa\[C]rola non è ancora sulla lingua,
e \[B&]Tu, Signore, già la co\[F]nosci \[G]tutta.

\endverse




%%%%% RITORNELLO
\beginchorus
\textnote{\textbf{Rit.}}

\[A-]Sei tu che mi hai cre\[F]ato
e mi hai tessuto nel \[C]seno di mia \[G]madre.
\[A-]Tu mi hai fatto come un pro\[F]digio  
le Tue opere \[C]sono stupende,
e per \[G]questo ti lo\[A-]do.

\endchorus






%%%%%% EV. INTERMEZZO
\beginverse*
\vspace*{1.3\versesep}
{
	\nolyrics
	\textnote{Intermezzo strumentale}
	
	\ifchorded

	%---- Prima riga -----------------------------
	\vspace*{-\versesep}
	\[F] \[C] \[G]


	\fi
	%---- Ev Indicazioni -------------------------			
	%\textnote{\textit{(ripetizione della strofa)}} 
	 
}
\vspace*{\versesep}
\endverse




%%%%% STROFA
\beginverse		%Oppure \beginverse* se non si vuole il numero di fianco
%\memorize 		% <<< DECOMMENTA se si vuole utilizzarne la funzione
%\chordsoff		% <<< DECOMMENTA se vuoi una strofa senza accordi

Di fronte e alle \[A-]spalle, Tu mi cir\[F]condi,
poni su \[C]me la tua \[G]mano.
La Tua sag\[A-]gezza, stupenda per \[F]me,
è troppo \[C]alta e io non \[G]la comprendo.
Che \[B&]sia in cielo o agli inferi ci \[F]sei,
non si può \[C]mai fuggire dalla tua presenza.
O\[B&]vunque la Tua mano guide\[F]rà la \[G]mia.


\endverse



%%%%% RITORNELLO
\beginchorus
\textnote{\textbf{Rit.}}

\[A-]Sei tu che mi hai cre\[F]ato
e mi hai tessuto nel \[C]seno di mia \[G]madre.
\[A-]Tu mi hai fatto come un pro\[F]digio  
le Tue opere \[C]sono stupende,
e per \[G]questo ti lo\[A-]do.

\endchorus





%%%%%% EV. INTERMEZZO
\beginverse*
\vspace*{1.3\versesep}
{
	\nolyrics
	\textnote{Intermezzo strumentale}
	
	\ifchorded

	%---- Prima riga -----------------------------
	\vspace*{-\versesep}
	 \[F] \[C] \[G]


	\fi
	%---- Ev Indicazioni -------------------------			
	%\textnote{\textit{(ripetizione della strofa)}} 
	 
}
\vspace*{\versesep}
\endverse




%%%%% STROFA
\beginverse		%Oppure \beginverse* se non si vuole il numero di fianco
%\memorize 		% <<< DECOMMENTA se si vuole utilizzarne la funzione
%\chordsoff		% <<< DECOMMENTA se vuoi una strofa senza accordi

E nel se\[A-]greto, Tu mi hai for\[F]mato,
mi hai intes\[C]suto dalla \[G]terra.
Neanche le \[A-]ossa, Ti eran na\[F]scoste
ancora in\[C]forme mi hanno \[G]visto i Tuoi occhi.
\[B&]I miei giorni erano fis\[F]sati
quando an\[C]cora non ne esisteva uno,
e \[B&]tutto quanto era scritto \[F]nel tuo \[G]libro.


\endverse





%%%%% RITORNELLO
\beginchorus
\textnote{\textbf{Rit.}}

\[A-]Sei tu che mi hai cre\[F]ato
e mi hai tessuto nel \[C]seno di mia \[G]madre.
\[A-]Tu mi hai fatto come un pro\[F]digio  
le Tue opere \[C]sono stupende,
e per \[G]questo ti lo\[A-]do.

\endchorus

%%%%%% EV. INTERMEZZO
\beginverse*
\vspace*{1.3\versesep}
{
	\nolyrics
	\textnote{Intermezzo strumentale}
	
	\ifchorded

	%---- Prima riga -----------------------------
	\vspace*{-\versesep}
	 \[F] \[C] \[G]

	%---- Ogni riga successiva -------------------
	\vspace*{-\versesep}
	\[A-] \[F] \[C] \[G]

	\fi
	%---- Ev Indicazioni -------------------------			
	%\textnote{\textit{(ripetizione della strofa)}} 
	 
}
\vspace*{\versesep}
\endverse



%%%%% RITORNELLO
\beginchorus
\textnote{\textbf{Rit.}}

\[A-]Sei tu che mi hai cre\[F]ato
e mi hai tessuto nel \[C]seno di mia \[G]madre.
\[A-]Tu mi hai fatto come un pro\[F]digio  
le Tue opere \[C]sono stupende,


\endchorus



%%%%%% EV. FINALE

\beginchorus %oppure \beginverse*
\vspace*{1.3\versesep}
\textnote{\textbf{Finale} \textit{[rallentando]}} %<<< EV. INDICAZIONI

e per \[G*]questo..
per questo ti \[F*]lo-odo. \quad \[C*]

\endchorus  %oppure \endverse



\endsong
%------------------------------------------------------------
%			FINE CANZONE
%------------------------------------------------------------

