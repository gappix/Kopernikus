%-------------------------------------------------------------
%			INIZIO	CANZONE
%-------------------------------------------------------------


%titolo: 	Camminerò
%autore: 	
%tonalita: 	Do e La



%%%%%% TITOLO E IMPOSTAZONI
\beginsong{Camminerò}[by={G. Moscati}] 	% <<< MODIFICA TITOLO E AUTORE
\transpose{-3} 						% <<< TRASPOSIZIONE #TONI (0 nullo)
\momenti{Congedo; Prima Comunione}							% <<< INSERISCI MOMENTI	
% momenti vanno separati da ; e vanno scelti tra:
% Ingresso; Atto penitenziale; Acclamazione al Vangelo; Dopo il Vangelo; Offertorio; Comunione; Ringraziamento; Fine; Santi; Pasqua; Avvento; Natale; Quaresima; Canti Mariani; Battesimo; Prima Comunione; Cresima; Matrimonio; Meditazione; Spezzare del pane;
\ifchorded
	%\textnote{Tonalità originale }	% <<< EV COMMENTI (tonalità originale/migliore)
\fi

%%%%%% INTRODUZIONE
\ifchorded
\vspace*{\versesep}
\musicnote{
\begin{minipage}{0.48\textwidth}
\textbf{Intro}
\hfill 
%( \eighthnote \, 80)   % <<  MODIFICA IL TEMPO
% Metronomo: \eighthnote (ottavo) \quarternote (quarto) \halfnote (due quarti)
\end{minipage}
} 	
\vspace*{-\versesep}
\beginverse*

\nolyrics

%---- Prima riga -----------------------------
\vspace*{-\versesep}
\[C] \[F] \[G7] \[C]	 % \[*D] per indicare le pennate, \rep{2} le ripetizioni

%---- Ogni riga successiva -------------------
%\vspace*{-\versesep}
%\[G] \[C]  \[D]	

%---- Ev Indicazioni -------------------------			
%\textnote{\textit{(Oppure tutta la strofa)} }	

\endverse
\fi

%%%%% STROFA
\beginverse		%Oppure \beginverse* se non si vuole il numero di fianco
\memorize 		% <<< DECOMMENTA se si vuole utilizzarne la funzione
%\chordsoff		% <<< DECOMMENTA se vuoi una strofa senza accordi

Mi hai chia\[C]mato dal nulla Si\[F]gnore
e mi hai \[G7]dato il dono della \[C]vita
tu mi hai preso e mi hai messo per \[F]strada
e mi hai \[G7]detto di cammi\[C]nar.
Verso un mondo che non ha con\[F]fini no
verso \[C]mete da raggiunge\[G7]re oramai
verso il \[C]regno dell'A\[G7]more
che è \[F]sempre un pò più in \[C]là!

\endverse

%%%%% RITORNELLO
\beginchorus
\textnote{\textbf{Rit.}}

Cammine\[F]rò senza stan\[C]carmi
e vole\[F]rò sui monti più \[C]alti
e trove\[F]rò la forza di an\[C]dare, 
sempre più a\[F]vanti, \[G7*]sì!
io cammine\[C]rò, cammine\[F]rò
con te vi\[C]cino, io non ca\[F]drò
e cammine\[C]rò, cammine\[G]rò. \[C]

\endchorus

%%%%% STROFA
\beginverse		%Oppure \beginverse* se non si vuole il numero di fianco
%\memorize 		% <<< DECOMMENTA se si vuole utilizzarne la funzione
%\chordsoff		% <<< DECOMMENTA se vuoi una strofa senza accordi

In ogni is^tante ti sento vi^cino,
Tu dai il ^senso alle cose che io ^faccio.
La tua luce mi indica la ^strada
e mi in^vita a cammi^nar.
Verso un mondo che non ha con^fini no
verso ^mete da raggiunge^re oramai
verso il ^regno dell'A^more
che è ^sempre un pò più in ^là!

\endverse

%%%%% RITORNELLO
\beginchorus
\textnote{\textbf{Rit.}}

Cammine\[F]rò senza stan\[C]carmi
e vole\[F]rò sui monti più \[C]alti
e trove\[F]rò la forza di an\[C]dare, 
sempre più a\[F]vanti, \[G7*]sì!
io cammine\[C]rò, cammine\[F]rò
con te vi\[C]cino, io non ca\[F]drò
e cammine\[C]rò, cammine\[G]rò. \[C] \[C*]

\endchorus

\endsong
%------------------------------------------------------------
%			FINE CANZONE
%------------------------------------------------------------


%++++++++++++++++++++++++++++++++++++++++++++++++++++++++++++
%			CANZONE TRASPOSTA
%++++++++++++++++++++++++++++++++++++++++++++++++++++++++++++
% \ifchorded
% %decremento contatore per avere stesso numero
% \addtocounter{songnum}{-1} 
% \beginsong{Camminerò}[by={G. Moscati}] 	% <<< COPIA TITOLO E AUTORE
% 						% <<< TRASPOSIZIONE #TONI + - (0 nullo)
% %\preferflats  %SE VOGLIO FORZARE i bemolle come alterazioni
% %\prefersharps %SE VOGLIO FORZARE i # come alterazioni
% \ifchorded
% 	%\textnote{Tonalità originale}	% <<< EV COMMENTI (tonalità originale/migliore)
% \fi


% %%%%%% INTRODUZIONE
% \ifchorded
% \vspace*{\versesep}
% \textnote{Intro: \qquad \qquad  }%(\eighthnote 116) % <<  MODIFICA IL TEMPO
% % Metronomo: \eighthnote (ottavo) \quarternote (quarto) \halfnote (due quarti)
% \vspace*{-\versesep}
% \beginverse*

% \nolyrics

% %---- Prima riga -----------------------------
% \vspace*{-\versesep}
% \[C] \[F] \[G7] \[C]	 % \[*D] per indicare le pennate, \rep{2} le ripetizioni

% %---- Ogni riga successiva -------------------
% %\vspace*{-\versesep}
% %\[G] \[C]  \[D]	

% %---- Ev Indicazioni -------------------------			
% %\textnote{\textit{(Oppure tutta la strofa)} }	

% \endverse
% \fi

% %%%%% STROFA
% \beginverse		%Oppure \beginverse* se non si vuole il numero di fianco
% \memorize 		% <<< DECOMMENTA se si vuole utilizzarne la funzione
% %\chordsoff		% <<< DECOMMENTA se vuoi una strofa senza accordi

% Mi hai chia\[C]mato dal nulla Si\[F]gnore
% e mi hai \[G7]dato il dono della \[C]vita
% tu mi hai preso e mi hai messo per \[F]strada
% e mi hai \[G7]detto di cammi\[C]nar.
% Verso un mondo che non ha con\[F]fini no
% verso \[C]mete da raggiunge\[G7]re oramai
% verso il \[C]regno dell'A\[G7]more
% che è \[F]sempre un pò più in \[C]là!

% \endverse
% %%%%% RITORNELLO
% \beginchorus
% \textnote{\textbf{Rit.}}

% Cammine\[F]rò senza stan\[C]carmi
% e vole\[F]rò sui monti più \[C]alti
% e trove\[F]rò la forza di an\[C]dare, 
% sempre più a\[F]vanti, \[*G7]sì!
% io cammine\[C]rò, cammine\[F]rò
% con te vi\[C]cino, io non ca\[F]drò
% e cammine\[C]rò, cammine\[G]rò. \[C]

% \endchorus

% %%%%% STROFA
% \beginverse		%Oppure \beginverse* se non si vuole il numero di fianco
% %\memorize 		% <<< DECOMMENTA se si vuole utilizzarne la funzione
% %\chordsoff		% <<< DECOMMENTA se vuoi una strofa senza accordi

% In ogni is^tante ti sento vi^cino,
% Tu dai il ^senso alle cose che io ^faccio.
% La tua luce mi indica la ^strada
% e mi in^vita a cammi^nar.
% Verso un mondo che non ha con^fini no
% verso ^mete da raggiunge^re oramai
% verso il ^regno dell'A^more
% che è ^sempre un pò più in ^là!

% \endverse

% %%%%% RITORNELLO
% \beginchorus
% \textnote{\textbf{Rit.}}

% Cammine\[F]rò senza stan\[C]carmi
% e vole\[F]rò sui monti più \[C]alti
% e trove\[F]rò la forza di an\[C]dare, 
% sempre più a\[F]vanti, \[*G7]sì!
% io cammine\[C]rò, cammine\[F]rò
% con te vi\[C]cino, io non ca\[F]drò
% e cammine\[C]rò, cammine\[G]rò. \[C] \[C*]

% \endchorus

% \endsong
% %------------------------------------------------------------
% %			FINE CANZONE
% %------------------------------------------------------------



% \fi
%++++++++++++++++++++++++++++++++++++++++++++++++++++++++++++
%			FINE CANZONE TRASPOSTA
%++++++++++++++++++++++++++++++++++++++++++++++++++++++++++++