%-------------------------------------------------------------
%			INIZIO	CANZONE
%-------------------------------------------------------------


%titolo: 	Blessed be your name
%autore: 	Matt Redman
%tonalita: 	Sol > La 



%%%%%% TITOLO E IMPOSTAZONI
\beginsong{Blessed be Your name}[by={M. Redman, B. Redman}] 	% <<< MODIFICA TITOLO E AUTORE
\transpose{0} 						% <<< TRASPOSIZIONE #TONI (0 nullo)
%\preferflats  %SE VOGLIO FORZARE i bemolle come alterazioni
%\prefersharps %SE VOGLIO FORZARE i # come alterazioni
\momenti{Meditazione; Ringraziamento; Congedo}							% <<< INSERISCI MOMENTI	
% momenti vanno separati da ; e vanno scelti tra:
% Ingresso; Atto penitenziale; Acclamazione al Vangelo; Dopo il Vangelo; Offertorio; Comunione; Ringraziamento; Fine; Santi; Pasqua; Avvento; Natale; Quaresima; Canti Mariani; Battesimo; Prima Comunione; Cresima; Matrimonio; Meditazione; Spezzare del pane;
\ifchorded
	%\textnote{Tonalità migliore }	% <<< EV COMMENTI (tonalità originale/migliore)
\fi


%%%%%% INTRODUZIONE
\ifchorded
\vspace*{\versesep}
\musicnote{
\begin{minipage}{0.48\textwidth}
\textbf{Intro}
\hfill 
%( \eighthnote \, 80)   % <<  MODIFICA IL TEMPO
% Metronomo: \eighthnote (ottavo) \quarternote (quarto) \halfnote (due quarti)
\end{minipage}
} 	
\vspace*{-\versesep}
\beginverse*

\nolyrics

%---- Prima riga -----------------------------
\vspace*{-\versesep}
\[G] \[D] \[E-] \[C] 	 % \[*D] per indicare le pennate, \rep{2} le ripetizioni

%---- Ogni riga successiva -------------------
\vspace*{-\versesep}
\[G] \[D] \[E-] \[C] \[C]	

%---- Ev Indicazioni -------------------------			
%\textnote{\textit{(Oppure tutta la strofa)} }	

\endverse
\fi




%%%%% STROFA
\beginverse		%Oppure \beginverse* se non si vuole il numero di fianco
\memorize 		% <<< DECOMMENTA se si vuole utilizzarne la funzione
%\chordsoff		% <<< DECOMMENTA se vuoi una strofa senza accordi

\[G] Blessed \[D]Be Your Name,  \brk in the \[E-]land that is \[C]plentiful, 
where Your \[G]streams of a\[D]bundance flow,
blessed \[C]be Your name.

\endverse
\beginverse*	

^ Blessed ^be Your name,  \brk when i'm ^found in the ^desert place,
though i ^walk through the ^wilderness,
blessed ^be Your name.


\endverse
\beginverse*		

^ Every blessing ^you pour out, i'll 
^ turn back to ^praise,
^ when the darkness ^closes in, Lord,
\[E-] still i will \[C]say:

\endverse


%%%%% RITORNELLO
\beginchorus
\textnote{\textbf{Rit.}}

Blessed be the \[G]name of the \[D]Lord!
Blessed be Your \[E-]name! \[C]
Blessed be the \[G]name of the \[D]Lord!
Blessed be Your \[E-]glorious \[C]name! 	\rep{2}

\endchorus


%%%%% STROFA
\beginverse		%Oppure \beginverse* se non si vuole il numero di fianco
%\memorize 		% <<< DECOMMENTA se si vuole utilizzarne la funzione
%\chordsoff		% <<< DECOMMENTA se vuoi una strofa senza accordi

^ Blessed ^be Your name, \brk when the ^sun's shining ^down on me,
when the ^world is all as ^it should be,
blessed b^e Your name.

\endverse
\beginverse*	

^ Blessed ^be Your name, \brk on the ^road marked with ^suffering,
though there is ^pain in the ^offering,
blessed ^be your name.

\endverse
\beginverse*		

^ Every blessing ^you pour out, i'll 
^ turn back to ^praise,
^ when the darkness ^closes in, Lord,
\[E-] still i will \[C]say:

\endverse


%%%%% RITORNELLO
\beginchorus
\textnote{\textbf{Rit.}}

Blessed be the \[G]name of the \[D]Lord!
Blessed be Your \[E-]name! \[C]
Blessed be the \[G]name of the \[D]Lord!
Blessed be Your \[E-]glorious \[C]name! 	\rep{2}

\endchorus


%%%%% BRIDGE
\beginverse*		%Oppure \beginverse* se non si vuole il numero di fianco
%\memorize 		% <<< DECOMMENTA se si vuole utilizzarne la funzione
%\chordsoff		% <<< DECOMMENTA se vuoi una strofa senza accordi
\vspace*{1.3\versesep}
\textnote{\textbf{Bridge}} %<<< EV. INDICAZIONI

You ^give and take ^away.
You ^give and take ^away.
My ^heart will choose to ^say,
Lord, bl\[E-]essed be Your \[C]name!  \rep{2}

\endverse
\beginverse*		

^ Every blessing ^you pour out, i'll 
^ turn back to ^praise,
^ when the darkness ^closes in, Lord,
\[E-] still i will \[C]say:

\endverse



%%%%% RITORNELLO
\beginchorus
\textnote{\textbf{Rit.}}

Blessed be the \[G]name of the \[D]Lord!
Blessed be Your \[E-]name! \[C]
Blessed be the \[G]name of the \[D]Lord!
Blessed be Your \[E-]glorious \[C]name! 	\rep{2}

\endchorus





%%%%% BRIDGE
\beginverse*		%Oppure \beginverse* se non si vuole il numero di fianco
%\memorize 		% <<< DECOMMENTA se si vuole utilizzarne la funzione
%\chordsoff		% <<< DECOMMENTA se vuoi una strofa senza accordi

You ^give and take ^away.
You ^give and take ^away.
My ^heart will choose to ^say,
Lord, bl\[E-]essed be Your \[C]name!  \rep{2}

\endverse



%%%%%% EV. INTERMEZZO
\beginverse*
\vspace*{1.3\versesep}
{
	\nolyrics
	\musicnote{Chiusura strumentale}
	
	\ifchorded

	%---- Prima riga -----------------------------
	\vspace*{-\versesep}
	\[G] \[D] \[E-] \[C] 

	%---- Ogni riga successiva -------------------
	\vspace*{-\versesep}
	\[G] \[D] \[C*]  \textit{(sospeso...)}


	\fi
	%---- Ev Indicazioni -------------------------			
	%\musicnote{\textit{sospeso}} 
	 
}
\vspace*{\versesep}
\endverse


\endsong
%------------------------------------------------------------
%			FINE CANZONE
%------------------------------------------------------------


