
%-------------------------------------------------------------
%			INIZIO	CANZONE
%-------------------------------------------------------------


%titolo: 	Camminiamo sulla strada
%autore: 	Varnavà, spiritual
%tonalita: 	Re


%%%%%% TITOLO E IMPOSTAZONI
\beginsong{Camminiamo sulla strada}[ititle={O when the saints}, by={Canto Spiritual tradizionale, Testo di S. Varnavà}] 	% <<< MODIFICA TITOLO E AUTORE
\transpose{-2} 						% <<< TRASPOSIZIONE #TONI (0 nullo)
\momenti{Santi}							% <<< INSERISCI MOMENTI	
% momenti vanno separati da ; e vanno scelti tra:
% Ingresso; Atto penitenziale; Acclamazione al Vangelo; Dopo il Vangelo; Offertorio; Comunione; Ringraziamento; Fine; Santi; Pasqua; Avvento; Natale; Quaresima; Canti Mariani; Battesimo; Prima Comunione; Cresima; Matrimonio; Meditazione;
\ifchorded
	%\textnote{Tonalità originale }	% <<< EV COMMENTI (tonalità originale/migliore)
\fi


%%%%%% INTRODUZIONE
\ifchorded
\vspace*{\versesep}
\musicnote{
\begin{minipage}{0.48\textwidth}
\textbf{Intro}
\hfill 
%( \eighthnote \, 80)   % <<  MODIFICA IL TEMPO
% Metronomo: \eighthnote (ottavo) \quarternote (quarto) \halfnote (due quarti)
\end{minipage}
} 	
\vspace*{-\versesep}
\beginverse*

\nolyrics

%---- Prima riga -----------------------------
\vspace*{-\versesep}
\[E] \[A]  \[E] \[B] \[E]	 % \[*D] per indicare le pennate, \rep{2} le ripetizioni

%---- Ogni riga successiva -------------------
%\vspace*{-\versesep}
%\[G] \[C]  \[D]	

%---- Ev Indicazioni -------------------------			
\textnote{\textit{[come la seconda parte della strofa]} }	

\endverse
\fi





%%%%% STROFA
\beginverse		%Oppure \beginverse* se non si vuole il numero di fianco
\memorize 		% <<< DECOMMENTA se si vuole utilizzarne la funzione
%\chordsoff		% <<< DECOMMENTA se vuoi una strofa senza accordi

Cammi\[E]niamo \[7]sulla \[A]strada
che han per\[E]corso i santi \[B7]tuoi
tutti \[E]ci ri\[7]trove\[A]remo
dove e\[E]terno \[B7]splende il \[E]sol.
\endverse





%%%%% RITORNELLO
\beginchorus
\textnote{\textbf{Rit.}}
E \[E]quando in ciel dei santi tuoi
la grande schiera arrive\[B7]rà
o Si\[E]gnor co\[7]me vor\[A]rei \[A-]
che ci \[E]fosse un \[B7]posto per \[E]me.


E \[(E)]quando il sol si spegnerà
e quando il sol si spegne\[B7]rà
o Si\[E]gnor co\[7]me vor\[A]rei \[A-]
che ci \[E]fosse un \[B7]posto per \[E]me.
\endchorus




%%%%% STROFA
\beginverse		%Oppure \beginverse* se non si vuole il numero di fianco
%\memorize 		% <<< DECOMMENTA se si vuole utilizzarne la funzione
%\chordsoff		% <<< DECOMMENTA se vuoi una strofa senza accordi
C'è chi ^dice ^che la ^vita
sia tris^tezza sia do^lor
ma io ^so che vi^ene il ^giorno
in cui ^tutto ^cambie^rà.
\endverse


%%%%% RITORNELLO
\beginchorus
\textnote{\textbf{Rit.}}
E \[E]quando in ciel risuonerà
la tromba che tutti chiame\[B7]rà
o Si\[E]gnor co\[7]me vor\[A]rei \[A-]
che ci \[E]fosse un \[B7]posto per \[E]me.

Il \[(E)]giorno che la terra e il ciel
a nuova vita risorge\[B7]ran
o Si\[E]gnor co\[7]me vor\[A]rei \[A-]
che ci \[E]fosse un \[B7]posto per \[E]me.
\endchorus
\endsong

