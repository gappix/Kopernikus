%-------------------------------------------------------------
%			INIZIO	CANZONE
%-------------------------------------------------------------


%titolo: 	Annunceremo che tu
%autore: 	Auricchio
%tonalita: 	Do 


%%%%%% TITOLO E IMPOSTAZONI
\beginsong{Annunceremo che Tu}[by={P. Auricchio}]
\transpose{-2}
\momenti{Ingresso; Congedo}							% <<< INSERISCI MOMENTI	
% momenti vanno separati da ; e vanno scelti tra:
% Ingresso; Atto penitenziale; Acclamazione al Vangelo; Dopo il Vangelo; Offertorio; Comunione; Ringraziamento; Fine; Santi; Pasqua; Avvento; Natale; Quaresima; Canti Mariani; Battesimo; Prima Comunione; Cresima; Matrimonio; Meditazione;
\ifchorded
	%\textnote{Tonalità originale }	% <<< EV COMMENTI (tonalità originale/migliore)
\fi



%%%%%% INTRODUZIONE
\ifchorded
\vspace*{\versesep}
\musicnote{
\begin{minipage}{0.48\textwidth}
\textbf{Intro}
\hfill 
%( \eighthnote \, 80)   % <<  MODIFICA IL TEMPO
% Metronomo: \eighthnote (ottavo) \quarternote (quarto) \halfnote (due quarti)
\end{minipage}
} 	
\vspace*{-\versesep}
\beginverse*
\nolyrics

%---- Prima riga -----------------------------
\vspace*{-\versesep}
\[C7]% \[*D] per indicare le pennate, \rep{2} le ripetizioni

%---- Ogni riga successiva -------------------
\vspace*{-\versesep}
\[G] \[D]  \[G]	 \[D]% \[*D] per indicare le pennate, \rep{2} le ripetizioni	

%---- Ogni riga successiva -------------------
\vspace*{-\versesep}
\[A-] \[C]  \[D]	

%---- Ev Indicazioni -------------------------			
\textnote{\textit{(come il ritornello)} }	

\endverse
\fi




%%%%% RITORNELLO
\beginchorus
\textnote{\textbf{Rit.}}
\[(D7)] Annunceremo che \[G]Tu sei Veri\[D]tà, 
lo grideremo dai \[G]tetti della nostra cit\[D]tà, 
senza paura \[A-]anche tu \[C] lo puoi can\[D]tare.
\endchorus




%%%%% STROFA
\beginverse
\memorize
\[G] E non te\[C]mere dai,  
\[G] che non ci \[D]vuole poi tanto, 
\[G] quello che \[C]non si sa  
\[G] non reste\[D]rà nascosto. 
\[B-] Se ti parlo nel \[C]buio, lo dirai nella \[D]luce,
ogni giorno è il mo\[C]mento di credere in \[D]me.
\endverse





%%%%% STROFA
\beginverse
%\memorize 		% <<< DECOMMENTA se si vuole utilizzarne la funzione
%\chordsoff		& <<< DECOMMENTA se vuoi una strofa senza accordi
^ Con il co^raggio tu  
^ porterai la Pa^rola che salva,  
^ anche se ^ci sarà 
^ chi non vuole ac^cogliere il dono, 
^ tu non devi fer^marti, ma continua a lot^tare, 
il mio Spirito ^sempre ti accompagne^rà.
\endverse





%%%%% STROFA
\beginverse
%\memorize 		% <<< DECOMMENTA se si vuole utilizzarne la funzione
%\chordsoff		& <<< DECOMMENTA se vuoi una strofa senza accordi
^ Non ti abban^dono mai,  
^ io sono il ^Dio fedele, 
^ conosco il ^cuore tuo. 
^ Ogni tuo pen^siero mi è noto,  
^ la tua vita è pre^ziosa, vale più di ogni ^cosa,
il segno più ^grande del mio amore per ^te.
\endverse

\endsong
%------------------------------------------------------------
%			FINE CANZONE
%------------------------------------------------------------