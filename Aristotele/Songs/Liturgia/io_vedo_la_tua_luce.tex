%-------------------------------------------------------------
%			INIZIO	CANZONE
%-------------------------------------------------------------


%titolo: 	Io vedo la Tua luce
%autore: 	Pierangelo Sequeri
%tonalita: 	Do



%%%%%% TITOLO E IMPOSTAZONI
\beginsong{Io vedo la Tua luce}[by={Sybolum 79 — P. Sequeri}] 	% <<< MODIFICA TITOLO E AUTORE
\transpose{0} 						% <<< TRASPOSIZIONE #TONI (0 nullo)
\momenti{Comunione; Ringraziamento; Avvento; Cresima}							% <<< INSERISCI MOMENTI	
% momenti vanno separati da ; e vanno scelti tra:
% Ingresso; Atto penitenziale; Acclamazione al Vangelo; Dopo il Vangelo; Offertorio; Comunione; Ringraziamento; Fine; Santi; Pasqua; Avvento; Natale; Quaresima; Canti Mariani; Battesimo; Prima Comunione; Cresima; Matrimonio; Meditazione; Spezzare del pane;
\ifchorded
	%\textnote{Tonalità migliore }	% <<< EV COMMENTI (tonalità originale/migliore)
\fi

%%%%%% INTRODUZIONE
\ifchorded
\vspace*{\versesep}
\musicnote{
\begin{minipage}{0.48\textwidth}
\textbf{Intro}
\hfill 
%( \eighthnote \, 80)   % <<  MODIFICA IL TEMPO
% Metronomo: \eighthnote (ottavo) \quarternote (quarto) \halfnote (due quarti)
\end{minipage}
} 	
\vspace*{-\versesep}
\beginverse*

\nolyrics

%---- Prima riga -----------------------------
\vspace*{-\versesep}
\[C] \[G] \[C] \[G]	 % \[*D] per indicare le pennate, \rep{2} le ripetizioni

%---- Ogni riga successiva -------------------
%\vspace*{-\versesep}
%\[G] \[C]  \[D]	

%---- Ev Indicazioni -------------------------			
%\textnote{\textit{(Oppure tutta la strofa)} }	

\endverse
\fi

%%%%% STROFA
\beginverse		%Oppure \beginverse* se non si vuole il numero di fianco
\memorize 		% <<< DECOMMENTA se si vuole utilizzarne la funzione
%\chordsoff		% <<< DECOMMENTA se vuoi una strofa senza accordi

\[C]Tu sei prima \[G]d’ogni cosa, \[C]prima d’ogni \[G]tempo,
d’ogni \[C]mio pensiero: \[G]prima \[C]della \[G]vita.

\endverse

%%%%% STROFA
\beginverse*	%Oppure \beginverse* se non si vuole il numero di fianco
%\memorize 		% <<< DECOMMENTA se si vuole utilizzarne la funzione
\chordsoff		% <<< DECOMMENTA se vuoi una strofa senza accordi

Una voce udimmo che gridava nel deserto
preparate la venuta del Signore.

\endverse

%%%%% STROFA
\beginverse*		%Oppure \beginverse* se non si vuole il numero di fianco
%\memorize 		% <<< DECOMMENTA se si vuole utilizzarne la funzione
\chordsoff		% <<< DECOMMENTA se vuoi una strofa senza accordi

Tu sei la Parola eterna della quale vivo
che mi pronunciò soltanto per amore.

\endverse

%%%%% STROFA
\beginverse*	%Oppure \beginverse* se non si vuole il numero di fianco
%\memorize 		% <<< DECOMMENTA se si vuole utilizzarne la funzione
\chordsoff		% <<< DECOMMENTA se vuoi una strofa senza accordi

E ti abbiamo udito predicare sulle strade
della nostra incomprensione senza fine. 

\endverse

%%%%% RITORNELLO
\beginchorus
\textnote{\textbf{Rit.}}

\[C]Io \[E7]ora so chi \[A-7]sei \[C7]
\[F]io \[D-]sento la tua \[E-]voce \[E-]
\[F]io \[A-]vedo la tua \[E-]luce \[C]
\[D-]io so che tu sei \[G]qui. \[G7]
\[C]E \[E7]sulla tua pa\[A-7]rola \[C7]
\[F]io \[D-]credo nell’a\[E-]more \[E-]
\[F]Io \[A-]vivo nella \[E-]pace \[C]
\[F]io \[G]so che torne\[C]rai.  \[G] \[C] \[G]

\endchorus

%%%%% STROFA
\beginverse		%Oppure \beginverse* se non si vuole il numero di fianco
%\memorize 		% <<< DECOMMENTA se si vuole utilizzarne la funzione
%\chordsoff		% <<< DECOMMENTA se vuoi una strofa senza accordi

^Tu sei l’appa^rire dell’imm^ensa tene^rezza
di un ^Amore che nes^suno ha visto ^mai.^

\endverse

%%%%% STROFA
\beginverse*		%Oppure \beginverse* se non si vuole il numero di fianco
%\memorize 		% <<< DECOMMENTA se si vuole utilizzarne la funzione
\chordsoff		% <<< DECOMMENTA se vuoi una strofa senza accordi

Ci fu dato il lieto annuncio della tua venuta
noi abbiamo visto un uomo come noi.

\endverse

%%%%% STROFA
\beginverse*		%Oppure \beginverse* se non si vuole il numero di fianco
%\memorize 		% <<< DECOMMENTA se si vuole utilizzarne la funzione
\chordsoff		% <<< DECOMMENTA se vuoi una strofa senza accordi

Tu sei verità che non tramonta,sei la vita 
che non muore,sei la via di un mondo nuovo.

\endverse

%%%%% STROFA
\beginverse*		%Oppure \beginverse* se non si vuole il numero di fianco
%\memorize 		% <<< DECOMMENTA se si vuole utilizzarne la funzione
\chordsoff		% <<< DECOMMENTA se vuoi una strofa senza accordi

E ti abbiamo visto stabilire la tua tenda
tra la nostra indifferenza d’ogni giorno.

\endverse


%%%%%% EV. CHIUSURA SOLO STRUMENTALE
\ifchorded
\beginchorus %oppure \beginverse*
\vspace*{1.3\versesep}
\textnote{Chiusura } %<<< EV. INDICAZIONI

\[C] \[G*] \textnote{[sospeso...]}

\endchorus  %oppure \endverse
\fi

\endsong
%------------------------------------------------------------
%			FINE CANZONE
%------------------------------------------------------------


% %++++++++++++++++++++++++++++++++++++++++++++++++++++++++++++
% %			CANZONE TRASPOSTA
% %++++++++++++++++++++++++++++++++++++++++++++++++++++++++++++
% \ifchorded
% %decremento contatore per avere stesso numero
% \addtocounter{songnum}{-1} 
% \beginsong{Io vedo la Tua Luce}[by={Pierangelo Sequeri}] 	% <<< COPIA TITOLO E AUTORE
% \transpose{+2} 						% <<< TRASPOSIZIONE #TONI + - (0 nullo)
% %\preferflats  %SE VOGLIO FORZARE i bemolle come alterazioni
% %\prefersharps %SE VOGLIO FORZARE i # come alterazioni
% \ifchorded
% 	%\textnote{Tonalità originale}	% <<< EV COMMENTI (tonalità originale/migliore)
% \fi


% %%%%%% INTRODUZIONE
% \ifchorded
% \vspace*{\versesep}
% \textnote{Intro: \qquad \qquad  }%(\eighthnote 116) % <<  MODIFICA IL TEMPO
% % Metronomo: \eighthnote (ottavo) \quarternote (quarto) \halfnote (due quarti)
% \vspace*{-\versesep}
% \beginverse*

% \nolyrics

% %---- Prima riga -----------------------------
% \vspace*{-\versesep}
% \[C] \[G] \[C] \[G]	 % \[*D] per indicare le pennate, \rep{2} le ripetizioni

% %---- Ogni riga successiva -------------------
% %\vspace*{-\versesep}
% %\[G] \[C]  \[D]	

% %---- Ev Indicazioni -------------------------			
% %\textnote{\textit{(Oppure tutta la strofa)} }	

% \endverse
% \fi

% %%%%% STROFA
% \beginverse		%Oppure \beginverse* se non si vuole il numero di fianco
% \memorize 		% <<< DECOMMENTA se si vuole utilizzarne la funzione
% %\chordsoff		% <<< DECOMMENTA se vuoi una strofa senza accordi

% \[C]Tu sei prima \[G]d’ogni cosa, \[C]prima d’ogni \[G]tempo,
% d’ogni \[C]mio pensiero: \[G]prima \[C]della \[G]vita.

% \endverse

% %%%%% STROFA
% \beginverse		%Oppure \beginverse* se non si vuole il numero di fianco
% %\memorize 		% <<< DECOMMENTA se si vuole utilizzarne la funzione
% %\chordsoff		% <<< DECOMMENTA se vuoi una strofa senza accordi

% Una voce udimmo che gridava nel deserto
% preparate la venuta del Signore.

% \endverse

% %%%%% STROFA
% \beginverse		%Oppure \beginverse* se non si vuole il numero di fianco
% %\memorize 		% <<< DECOMMENTA se si vuole utilizzarne la funzione
% %\chordsoff		% <<< DECOMMENTA se vuoi una strofa senza accordi

% Tu sei la Parola eterna della quale vivo
% che mi pronunciò soltanto per amore.

% \endverse

% %%%%% STROFA
% \beginverse		%Oppure \beginverse* se non si vuole il numero di fianco
% %\memorize 		% <<< DECOMMENTA se si vuole utilizzarne la funzione
% %\chordsoff		% <<< DECOMMENTA se vuoi una strofa senza accordi

% E ti abbiamo udito predicare sulle strade
% della nostra incomprensione senza fine. 

% \endverse

% %%%%% RITORNELLO
% \beginchorus
% \textnote{\textbf{Rit.}}

% \[C]Io \[E7]ora so chi \[A-7]sei
% \[C7]io \[F]sento \[D-]la tua \[E-]voce
% \[F]io \[A-]vedo la tua \[E-]luce \[C]
% \[D-]io \[G]so che tu sei \[G7]qui.
% \[C]E \[E7]sulla tua pa\[A-7]rola \[C7]
% \[F]io \[D-]credo nell’a\[E-]more
% \[F]Io \[A-]vivo nella \[E-]pace \[C]
% \[F]io \[G]so che torne\[C]rai.

% \endchorus

% %%%%% STROFA
% \beginverse		%Oppure \beginverse* se non si vuole il numero di fianco
% %\memorize 		% <<< DECOMMENTA se si vuole utilizzarne la funzione
% %\chordsoff		% <<< DECOMMENTA se vuoi una strofa senza accordi

% Tu sei l’apparire dell’immensa tenerezza
% di un Amore che nessuno ha visto mai.

% \endverse

% %%%%% STROFA
% \beginverse		%Oppure \beginverse* se non si vuole il numero di fianco
% %\memorize 		% <<< DECOMMENTA se si vuole utilizzarne la funzione
% %\chordsoff		% <<< DECOMMENTA se vuoi una strofa senza accordi

% Ci fu dato il lieto annuncio della tua venuta
% noi abbiamo visto un uomo come noi.

% \endverse

% %%%%% STROFA
% \beginverse		%Oppure \beginverse* se non si vuole il numero di fianco
% %\memorize 		% <<< DECOMMENTA se si vuole utilizzarne la funzione
% %\chordsoff		% <<< DECOMMENTA se vuoi una strofa senza accordi

% Tu sei verità che non tramonta,sei la vita 
% che non muore,sei la via di un mondo nuovo.

% \endverse

% %%%%% STROFA
% \beginverse		%Oppure \beginverse* se non si vuole il numero di fianco
% %\memorize 		% <<< DECOMMENTA se si vuole utilizzarne la funzione
% %\chordsoff		% <<< DECOMMENTA se vuoi una strofa senza accordi

% E ti abbiamo visto stabilire la tua tenda
% tra la nostra indifferenza d’ogni giorno.

% \endverse

% \endsong


% \fi
% %++++++++++++++++++++++++++++++++++++++++++++++++++++++++++++
% %			FINE CANZONE TRASPOSTA
% %++++++++++++++++++++++++++++++++++++++++++++++++++++++++++++