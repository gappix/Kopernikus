%-------------------------------------------------------------
%			INIZIO	CANZONE
%-------------------------------------------------------------


%titolo: 	Andiamo ed annunciamo
%autore: 	V. Di Mauro
%tonalita: 	Re 



%%%%%% TITOLO E IMPOSTAZONI
\beginsong{Andiamo ed annunciamo}[by={V. Di Mauro}] 	% <<< MODIFICA TITOLO E AUTORE
\transpose{0} 						% <<< TRASPOSIZIONE #TONI (0 nullo)
\momenti{Fine; Cresima}							% <<< INSERISCI MOMENTI	
% momenti vanno separati da ; e vanno scelti tra:
% Ingresso; Atto penitenziale; Acclamazione al Vangelo; Dopo il Vangelo; Offertorio; Comunione; Ringraziamento; Fine; Santi; Pasqua; Avvento; Natale; Quaresima; Canti Mariani; Battesimo; Prima Comunione; Cresima; Matrimonio; Meditazione; Spezzare del pane;
\ifchorded
	%\textnote{Tonalità migliore }	% <<< EV COMMENTI (tonalità originale/migliore)
\fi

%%%%%% INTRODUZIONE
\ifchorded
\vspace*{\versesep}
\textnote{Intro: \qquad \qquad  }%(\eighthnote 116) % <<  MODIFICA IL TEMPO
% Metronomo: \eighthnote (ottavo) \quarternote (quarto) \halfnote (due quarti)
\vspace*{-\versesep}
\beginverse*

\nolyrics

%---- Prima riga -----------------------------
\vspace*{-\versesep}
\[D] \[D] \[B-] 	 % \[*D] per indicare le pennate, \rep{2} le ripetizioni

%---- Ogni riga successiva -------------------
\vspace*{-\versesep}
\[G] \[A] \[D]

%---- Ev Indicazioni -------------------------			
%\textnote{\textit{(Oppure tutta la strofa)} }	

\endverse
\fi

%%%%% RITORNELLO
\beginchorus
\textnote{\textbf{Rit.}}

Andi\[D]amo e annunciamo a tutto il \[B-]mondo
che il Si\[G]gnore è ve\[A]nuto in mezzo a \[D]noi
e se ami\[G]amo come \[A]Lui ci \[D*]ha a\[F#-*]ma\[B-]to,
ogni \[G]giorno con \[A]noi cammine\[D*]rà. \[G*] \[D]

\endchorus

%%%%% STROFA
\beginverse		%Oppure \beginverse* se non si vuole il numero di fianco
\memorize 		% <<< DECOMMENTA se si vuole utilizzarne la funzione
%\chordsoff		% <<< DECOMMENTA se vuoi una strofa senza accordi

La \[D]terra percor\[B-]rete inseg\[E-]nando ad ogni \[A7]uomo
il \[D]mio comanda\[B-]mento, la \[E-]legge dell’A\[A7]more:
par\[D]late con la \[A]vita e \[G]non con le pa\[A]role;
chi \[D]vede il vostro a\[B-]more da \[E-]solo capi\[A7]rà.

\endverse

%%%%% STROFA
\beginverse		%Oppure \beginverse* se non si vuole il numero di fianco
%\memorize 		% <<< DECOMMENTA se si vuole utilizzarne la funzione
\chordsoff		% <<< DECOMMENTA se vuoi una strofa senza accordi

Com^prenderà chi ^sbaglia e ^chi non spera ^più,
a ^Me ritorne^rà se ^voi l’accoglie^rete
nel ^nome di quel ^Padre, che at^tende e poi per^dona,
del ^Figlio e dello ^Spirito, che a ^voi è stato ^dato!

\endverse

%%%%% STROFA
\beginverse		%Oppure \beginverse* se non si vuole il numero di fianco
%\memorize 		% <<< DECOMMENTA se si vuole utilizzarne la funzione
\chordsoff		% <<< DECOMMENTA se vuoi una strofa senza accordi

Se ^lungo e fati^coso vi ^sembrerà il ca^mino
che agli altri vi con^duce per ^far conoscer ^Me,
non ^rallentate il ^passo, ma ^ricordate ^sempre
che ^ovunque in ogni ^strada, con ^voi cammine^rò!

\endverse

\endsong
%------------------------------------------------------------
%			FINE CANZONE
%------------------------------------------------------------