%-------------------------------------------------------------
%			INIZIO	CANZONE
%-------------------------------------------------------------


%titolo: 	Benedici il Signore anima mia
%autore: 	Frisina
%tonalita: 	Sol 



%%%%%% TITOLO E IMPOSTAZONI
\beginsong{Benedici il Signore anima mia}[by={Frisina}] 	% <<< MODIFICA TITOLO E AUTORE
\transpose{0} 						% <<< TRASPOSIZIONE #TONI (0 nullo)
\momenti{}							% <<< INSERISCI MOMENTI	
% momenti vanno separati da ; e vanno scelti tra:
% Ingresso; Atto penitenziale; Acclamazione al Vangelo; Dopo il Vangelo; Offertorio; Comunione; Ringraziamento; Fine; Santi; Pasqua; Avvento; Natale; Quaresima; Canti Mariani; Battesimo; Prima Comunione; Cresima; Matrimonio; Meditazione; Spezzare del pane;
\ifchorded
	%\textnote{Tonalità originale }	% <<< EV COMMENTI (tonalità originale/migliore)
\fi

%%%%%% INTRODUZIONE
\ifchorded
\vspace*{\versesep}
\textnote{Intro: \qquad \qquad  }%(\eighthnote 116) % <<  MODIFICA IL TEMPO
% Metronomo: \eighthnote (ottavo) \quarternote (quarto) \halfnote (due quarti)
\vspace*{-\versesep}
\beginverse*

\nolyrics

%---- Prima riga -----------------------------
\vspace*{-\versesep}
\[A-] \[G] \[A-]	 % \[*D] per indicare le pennate, \rep{2} le ripetizioni

%---- Ogni riga successiva -------------------
%\vspace*{-\versesep}
%\[G] \[C]  \[D]	

%---- Ev Indicazioni -------------------------			
\textnote{\textit{(Oppure tutta la prima sequenza)} }	

\endverse
\fi


%%%%% RITORNELLO
\beginchorus
\textnote{\textbf{Rit.}}
Bene\[A-]dici il Si\[G]gnore, anima \[C]mia
Quant'è in \[F]me bene\[G]dica il suo \[C]nome
Non di\[D-]mentiche\[G]rò tutti i \[E]suoi bene\[A-]fici.
Bene\[F]dici il Si\[E-]gnore, anima \[A-]mia.
\endchorus

%%%%% STROFA
\beginverse
\memorize
Lui per\[F]dona \[G]tutte le tue \[C]colpe
e ti \[D-]salva dalla \[G]mor\[A-]te.
Ti co\[F]rona di \[G]grazia e ti \[E]sazia di \[A-]beni
nella \[F]tua giovi\[G]nez\[E]za.
\endverse


%%%%% STROFA
\beginverse
%\chordsoff
Il Si^gnore a^gisce con giu^stizia
con a^more verso i ^pove^ri
rive^lò a Mo^sè le sue ^vie, ad Isra^ele
le sue ^grandi ^ope^re.
\endverse

%%%%% STROFA
\beginverse
\chordsoff
Il Signore è buono e pietoso
lento all'ira e grande nell'amor.
Non conserva in eterno il suo sdegno e la sua ira
verso i nostri peccati.
\endverse

%%%%% STROFA
\beginverse
\chordsoff
Come dista Oriente da Occidente
allontana le tue colpe.
Perché sa che di polvere siam tutti noi plasmati,
come l'erba i nostri giorni.
\endverse


\beginchorus
\vspace*{1.3\versesep}
\textnote{Finale }
%\chordsoff
Bene^dite il Si^gnore voi  ^angeli,
voi ^tutti su^oi mi^nistri.
Bene^ditelo voi ^tutte sue ^opere e do^míni.
Bene\[F]dicilo \[E-]tu, anima \[A-]mia.
\endchorus


\endsong

