%-------------------------------------------------------------
%			INIZIO	CANZONE
%-------------------------------------------------------------


%titolo: 	Popoli tutti
%autore: 	Zschech
%tonalita: 	La



%%%%%% TITOLO E IMPOSTAZONI
\beginsong{Popoli tutti acclamate}[by={Zschech}]	
\transpose{0} 						% <<< TRASPOSIZIONE #TONI (0 nullo)
\momenti{Ingresso; Comunione; Fine}							% <<< INSERISCI MOMENTI	
% momenti vanno separati da ; e vanno scelti tra:
% Ingresso; Atto penitenziale; Acclamazione al Vangelo; Dopo il Vangelo; Offertorio; Comunione; Ringraziamento; Fine; Santi; Pasqua; Avvento; Natale; Quaresima; Canti Mariani; Battesimo; Prima Comunione; Cresima; Matrimonio; Meditazione; Spezzare del pane;
\ifchorded
	%\textnote{Tonalità originale }	% <<< EV COMMENTI (tonalità originale/migliore)
\fi


%%%%%% INTRODUZIONE
\ifchorded
\vspace*{\versesep}
\textnote{Intro: \qquad \qquad  }%(\eighthnote 116) % <<  MODIFICA IL TEMPO
% Metronomo: \eighthnote (ottavo) \quarternote (quarto) \halfnote (due quarti)
\vspace*{-\versesep}
\beginverse*

\nolyrics

%---- Prima riga -----------------------------
\vspace*{-\versesep}
\[A] \[E] \[A]	 % \[*D] per indicare le pennate, \rep{2} le ripetizioni

%---- Ogni riga successiva -------------------
%\vspace*{-\versesep}
%\[G] \[C]  \[D]	

%---- Ev Indicazioni -------------------------			
%\textnote{\textit{(Oppure tutta la strofa)} }	

\endverse
\fi




%%%%% STROFA
\beginverse		%Oppure \beginverse* se non si vuole il numero di fianco
\memorize 		% <<< DECOMMENTA se si vuole utilizzarne la funzione
%\chordsoff		% <<< DECOMMENTA se vuoi una strofa senza accordi
\[A] Mio Dio, \[E] Signore, 
\[F#-]nulla è \[E]pari a \[D]te.
Ora e per \[A]sempre, \[D]voglio lo\[A]dare
\[F#-7]il Tuo grande \[G]amor \[D]per \[E4]noi. \[E] 
\[A]  Mia roccia \[E] Tu sei, 
\[F#-]pace e con\[E]forto mi \[D]dai.
Con tutto il \[A]cuore \[D]e le mie \[A]forze,
\[F#-7]sempre io Ti a\[G]do\[D]re\[E4]rò. \[E] 
\endverse


%%%%% RITORNELLO
\beginchorus
\textnote{\textbf{Rit.}}
\[A]Popoli \[F#-]tutti accla\[D]mate al Si\[E]gnore.
\[A]Gloria e po\[F#-]tenza can\[D7+]tiamo al \[E]re.
\[F#-]Mari e monti si \[D]prostrino a te,
al tuo \[E]nome, \[F#-]o Si\[E]gnore.
\[A]Canto di \[F#-]gioia per \[D]quello che \[E]fai,
per \[A]sempre Si\[F#-]gnore con \[D7+]te reste\[E]rò.
\[F#-]Non c'è promessa non \[D]c'è fedel\[E7]tà che in \[A]te.
\endchorus

%%%%% STROFA
\beginverse		%Oppure \beginverse* se non si vuole il numero di fianco
%\memorize 		% <<< DECOMMENTA se si vuole utilizzarne la funzione
%\chordsoff		% <<< DECOMMENTA se vuoi una strofa senza accordi
^ Tu luce ^ d'amore, 
^Spirito ^di Santi^tà
entra nei ^cuori di-^questi tuoi ^figli
chia^mati ad annun^cia^re il ^Re. ^
^ Tu forza ^ d'amore 
^nuova spe^ranza ci ^dai
in questo ^giorno a ^te consa^crato
^gioia immensa ^can^te^rò. ^
\endverse




\endsong
%------------------------------------------------------------
%			FINE CANZONE
%------------------------------------------------------------



