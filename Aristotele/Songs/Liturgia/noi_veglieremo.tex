%-------------------------------------------------------------
%			INIZIO	CANZONE
%-------------------------------------------------------------


%titolo: 	Noi veglieremo
%autore: 	Machetta
%tonalita: 	Re 



%%%%%% TITOLO E IMPOSTAZONI
\beginsong{Noi veglieremo}[by={D. Machetta}]	% <<< MODIFICA TITOLO E AUTORE
\transpose{-2} 						% <<< TRASPOSIZIONE #TONI (0 nullo)
\momenti{Avvento; Congedo}							% <<< INSERISCI MOMENTI	
% momenti vanno separati da ; e vanno scelti tra:
% Ingresso; Atto penitenziale; Acclamazione al Vangelo; Dopo il Vangelo; Offertorio; Comunione; Ringraziamento; Fine; Santi; Pasqua; Avvento; Natale; Quaresima; Canti Mariani; Battesimo; Prima Comunione; Cresima; Matrimonio; Meditazione; Spezzare del pane;
\ifchorded
	%\textnote{Tonalità migliore }	% <<< EV COMMENTI (tonalità originale/migliore)
\fi


%%%%%% INTRODUZIONE
\ifchorded
\vspace*{\versesep}
\musicnote{
\begin{minipage}{0.48\textwidth}
\textbf{Intro}
\hfill 
%( \eighthnote \, 80)   % <<  MODIFICA IL TEMPO
% Metronomo: \eighthnote (ottavo) \quarternote (quarto) \halfnote (due quarti)
\end{minipage}
} 	
\vspace*{-\versesep}
\beginverse*

\nolyrics

%---- Prima riga -----------------------------
\vspace*{-\versesep}
\[D*] \[G] \[E-*]\[A*] \[D]	 % \[*D] per indicare le pennate, \rep{2} le ripetizioni

%---- Ogni riga successiva -------------------
%\vspace*{-\versesep}
%\[G] \[C]  \[D]	

%---- Ev Indicazioni -------------------------			
%\textnote{\textit{(Oppure tutta la strofa)} }	

\endverse
\fi







%%%%% RITORNELLO
\beginchorus
\textnote{\textbf{Rit.}}
Nella \[D*]notte, o \[G]Dio, \[E-7*] noi \[A*]veglie\[D]remo,
con le \[B-]lampade, vestiti a \[F#-]festa: \[B]  
presto \[E-*]arri\[G]verai  e \[A*]sarà \[D]giorno.
\endchorus




%%%%% STROFA
\beginverse		%Oppure \beginverse* se non si vuole il numero di fianco
\memorize 		% <<< DECOMMENTA se si vuole utilizzarne la funzione
%\chordsoff		% <<< DECOMMENTA se vuoi una strofa senza accordi
\[B-*]Ralle\[E-]gratevi in at\[A]tesa del Si\[D7+]gnore:
improv\[B-]visa giungerà la sua \[E-7]voce. \[A9]
Quando \[G-]Lui verrà, sarete \[D7+]pronti
e vi \[E-]chiamerà “\[G]amici” per \[F#]sempre. \[A7]  
\endverse



%%%%% STROFA
\beginverse		%Oppure \beginverse* se non si vuole il numero di fianco
%\memorize 		% <<< DECOMMENTA se si vuole utilizzarne la funzione
%\chordsoff		% <<< DECOMMENTA se vuoi una strofa senza accordi
^Raccogli^ete per il ^giorno della ^vita
dove ^tutto sarà giovane in e^terno. ^
Quando ^lui verrà sarete ^pronti
e vi ^chiamerà “^amici” per ^sempre. ^
\endverse

\endsong
%------------------------------------------------------------
%			FINE CANZONE
%------------------------------------------------------------

