% SETTINGS
%——————————————————————————————————————————————————————
% STILE DOCUMENTO
%-------------------------------------------------------------------------------                                                                             
\documentclass[a4vert, palatino, titleindex, tematicindex, chorded, cover]{canzoniereonline}

%opzioni formato: singoli, standard (A4), a5vert, a5oriz, a6vert;
%opzioni accordi: lyric, chorded {quelli d Songs}
%opzioni font: palatino, libertine
%opzioni segno minore: "minorsign=quel che vuoi"
%opzioni indici: authorsindex, titleindex, tematicindex

%opzioi copertina: cover e nocover

\def\canzsongcolumsnumber{2} %# coolonne lungo cui disporre le canzoni




% PACCHETTI DA IMPORTARE
%-------------------------------------------------------------------------------                                                                             
\usepackage[T1]{fontenc}
\usepackage[utf8]{inputenc}
\usepackage[italian]{babel}
\usepackage{pdfpages}
\usepackage{hyperref}
\usepackage{wasysym}






% NUOVI COMANDI E VARIABILI GLOBALI
%--------------------------------------------------------------------------------
%Coomando per la suddivisione in capitoli
\renewcommand{\songchapter}{\chapter*}


%Counter globale per tenere traccia di una numerazione progressiva
%Si affianca a un altro counter già utilizzato nella classe CanzoniereOnLine "songnum" 
%che, tuttavia, si riazzera ognivolta viene definito un nuovo ambiente \beginsongs{}
\newcounter{GlobalSongCounter} 

%Ciascun capitolo contiene già tutta la logica di gestione della 
%numerazione progressiva, del DB locale da cui attingere le canzoni
%e la creazione/chiusura dell'ambiente in cui vengono importate 
%tutte le canzoni relative
\addtocounter{GlobalSongCounter}{1} %set starting song counter to 1 (0 otherwise)


% RIGHE PER LA COPERTINA
%————————————————————————————————————————————————————————————————————————


%titoletto
\renewcommand{\titolettocop}{Tutti i dubbi della Brunella — BETA 1.X1} 	


%TITOLO					
\renewcommand{\titolocop}{KOPERNIKUS}  			


%Sottotitolo
\renewcommand{\sottotitolocop}{La rivoluzione dei canzonieri} 	


%fondo pagina
\renewcommand{\piede}{\today}							

%------------
\makeatletter
\newcommand*{\textoverline}[1]{$\overline{\hbox{#1}}\m@th$}
\makeatother
%-----------

%Starting Document
\begin{document}

% COLOPHON
%-------------------------------------------------------------------------------
\ifcover
	\firstpage
	%\colophon
\else
	\relax
\fi


\begin{songs}{}
\songcolumns{\canzsongcolumsnumber}
\setcounter{songnum}{\theGlobalSongCounter} %set songnum counter, otherwise would be reset


%  *  *  *  *  *  TEST SONG HERE  *  *  *  *  *  *  * ]
%*****************************************************
%-------------------------------------------------------------
%			INIZIO	CANZONE
%-------------------------------------------------------------


%titolo: 	Angelo di Dio
%autore: 	G. Mezzalira
%tonalita: 	Mib



%%%%%% TITOLO E IMPOSTAZONI
\beginsong{Angelo di Dio}[by={G. Mezzalira}]
\transpose{-3} 						% <<< TRASPOSIZIONE #TONI (0 nullo)
%\preferflats  %SE VOGLIO FORZARE i bemolle come alterazioni
%\prefersharps %SE VOGLIO FORZARE i # come alterazioni
\momenti{}							% <<< INSERISCI MOMENTI	
% momenti vanno separati da ; e vanno scelti tra:
% Ingresso; Atto penitenziale; Acclamazione al Vangelo; Dopo il Vangelo; Offertorio; Comunione; Ringraziamento; Fine; Santi; Pasqua; Avvento; Natale; Quaresima; Canti Mariani; Battesimo; Prima Comunione; Cresima; Matrimonio; Meditazione; Spezzare del pane;
\ifchorded
	%\textnote{Tonalità migliore }	% <<< EV COMMENTI (tonalità originale\migliore)
\fi


%%%%%% INTRODUZIONE
\ifchorded
\vspace*{\versesep}
\musicnote{
\begin{minipage}{0.48\textwidth}
\textbf{Intro}
\hfill 
%( \eighthnote \, 80)   % <<  MODIFICA IL TEMPO
% Metronomo: \eighthnote (ottavo) \quarternote (quarto) \halfnote (due quarti)
\end{minipage}
} 	
\vspace*{-\versesep}
\beginverse*

\nolyrics

%---- Prima riga -----------------------------
\vspace*{-\versesep}
\[E&] \[A&] \[B&7]	\[E&] % \[*D] per indicare le pennate, \rep{2} le ripetizioni

%---- Ogni riga successiva -------------------
%\vspace*{-\versesep}
%\[G] \[C]  \[D]	

%---- Ev Indicazioni -------------------------			
%\textnote{\textit{(Oppure tutta la strofa)} }	

\endverse
\fi


%%%%% RITORNELLO
\beginchorus
\textnote{\textbf{Rit.}}
\[E&]Angelo di \[A&]Dio, \brk che \[B&7]sei il mio cus\[G-]tode,
il\[C-]lumina custo\[F7]disci \brk \[B&]reggi-e governa \[E&]me 
che ti \[A&]fui affi\[Ddim/B&]dato 
\[G-]dalla pietà ce\[C-]leste 
\[E&]A-a-am\[A&]en \[C7*] \[F-]A-a-\[B&*]a\[E&]men 
\endchorus




%%%%% STROFA
\beginverse		%Oppure \beginverse* se non si vuole il numero di fianco
\memorize 		% <<< DECOMMENTA se si vuole utilizzarne la funzione
%\chordsoff		% <<< DECOMMENTA se vuoi una strofa senza accordi

\[C]Guida i miei \[F]pensieri, \[D-]segui i miei \[G]passi,
\[C]dammi la tua \[A-]forza per \[G]restare nella \[C]luce
\[C]la gioia \[F]incontrerò \[G]la pace \[C]porterò
\[A-]Angelo \[F]Santo \[C]veglia \[G]su di \[C]me.


\endverse





%%%%% STROFA
\beginverse		%Oppure \beginverse* se non si vuole il numero di fianco
%\memorize 		% <<< DECOMMENTA se si vuole utilizzarne la funzione
%\chordsoff		% <<< DECOMMENTA se vuoi una strofa senza accordi
^Guida la mia ^vita, ^segui il mio la^voro,
^dammi la tua ^forza per dif^fondere l’a^more
^Gesù io ^annunzierò, ^paura ^non avrò 
^Angelo ^Santo ^veglia ^su di ^me.
\endverse





\endsong
%------------------------------------------------------------
%			FINE CANZONE
%------------------------------------------------------------




%titolo{Apri le tue braccia}
%autore{Machetta}
%album{Una voce che ti cerca}
%tonalita{Re-}
%famiglia{Liturgica}
%gruppo{}
%momenti{Ingresso;Quaresima;Riconciliazione;Conversione}
%identificatore{apri_le_tue_braccia}
%data_revisione{2011_12_31}
%trascrittore{Francesco Endrici}
\beginsong{Apri le tue braccia}[by={Machetta}]
\beginverse
Hai cer\[D-]cato la \[C]libertà lon\[D-]tano,
hai tro\[A-]vato la \[E&]noia e le ca\[B&]tene;
hai va\[G-]ga\[D-]to \[G-]senza \[D-]via, \[G-]solo, \[E&] con la tua \[C]fame.
\endverse
\beginchorus
\[F]A\[C]pri le tue \[D-]brac\[A-]cia, \[B&]corri in\[G-]contro al \[C]Padre;
\[F]oggi \[D] la sua \[G-]ca\[B&]sa sarà in \[F]fe\[C7]sta per \[F]te.
\endchorus
\beginverse
\chordsoff
Se vor^rai spez^zare le ca^tene
trove^rai la ^strada dell'a^more;
la tua ^gio^ia ^cante^rai: ^questa ^ è liber^tà.
\endverse
\beginverse
\chordsoff
I tuoi ^occhi ri^cercano l'az^zurro;
c'è una ^casa che a^spetta il tuo ri^torno,
e la ^pa^ce ^torne^rà: ^questa ^ è liber^tà.
\endverse
\endsong


%-------------------------------------------------------------
%			INIZIO	CANZONE
%-------------------------------------------------------------


%titolo: 	Cantico delle creature
%autore: 	Varnavà, Mancinoni
%tonalita: 	Mi-



%%%%%% TITOLO E IMPOSTAZONI
\beginsong{Cantico delle creature}[by={Per la terra e le tue creature — S. Varnavà, R. Mancinoni}] 	% <<< MODIFICA TITOLO E AUTORE
\transpose{0} 						% <<< TRASPOSIZIONE #TONI (0 nullo)
%\preferflats  %SE VOGLIO FORZARE i bemolle come alterazioni
%\prefersharps %SE VOGLIO FORZARE i # come alterazioni
\momenti{}							% <<< INSERISCI MOMENTI	
% momenti vanno separati da ; e vanno scelti tra:
% Ingresso; Atto penitenziale; Acclamazione al Vangelo; Dopo il Vangelo; Offertorio; Comunione; Ringraziamento; Fine; Santi; Pasqua; Avvento; Natale; Quaresima; Canti Mariani; Battesimo; Prima Comunione; Cresima; Matrimonio; Meditazione; Spezzare del pane;
\ifchorded
	%\textnote{Tonalità migliore }	% <<< EV COMMENTI (tonalità originale/migliore)
\fi




%%%%%% INTRODUZIONE
\ifchorded
\vspace*{\versesep}
\musicnote{
\begin{minipage}{0.48\textwidth}
\textbf{Intro}
\hfill 
%( \eighthnote \, 80)   % <<  MODIFICA IL TEMPO
% Metronomo: \eighthnote (ottavo) \quarternote (quarto) \halfnote (due quarti)
\end{minipage}
} 	
\vspace*{-\versesep}
\beginverse*


\nolyrics

%---- Prima riga -----------------------------
\vspace*{-\versesep}
\[B-] \[E-] \[B-]	 % \[*D] per indicare le pennate, \rep{2} le ripetizioni

%---- Ogni riga successiva -------------------
%\vspace*{-\versesep}
%\[G] \[C]  \[D]	

%---- Ev Indicazioni -------------------------			
%\textnote{\textit{(Oppure tutta la strofa)} }	

\endverse
\fi






%%%%% STROFA
\beginverse		%Oppure \beginverse* se non si vuole il numero di fianco
\memorize 		% <<< DECOMMENTA se si vuole utilizzarne la funzione
%\chordsoff		% <<< DECOMMENTA se vuoi una strofa senza accordi
\[B-]Laudato \[E-]sii mi Si\[B-]gnore,
\[A]per frate \[B-]sole, \[F#]sora \[B-]luna,
\[B-]frate vento, il \[E-]cielo, \[B-]le stelle,
\[A]per sora \[B-]acqua, \[F#]frate \[B-]focu.
\endverse



%%%%% RITORNELLO
\beginchorus
\textnote{\textbf{Rit.}}
\[G]Lau\[A]dato
\[D*]sii \[F#*]mi Si\[B-]gnore,
\[E-]per la \[B-]terra e \[F#]le tue crea\[B-]ture
\[G]Lau\[A]dato
\[D*]sii \[F#*]mi Si\[B-]gnore,
\[E-]per la \[B-]terra e \[F#]le tue crea\[B-]ture
\endchorus




%%%%% STROFA
\beginverse		%Oppure \beginverse* se non si vuole il numero di fianco
%\memorize 		% <<< DECOMMENTA se si vuole utilizzarne la funzione
%\chordsoff		% <<< DECOMMENTA se vuoi una strofa senza accordi
\[B-]Laudato \[E-]sii, mi Si\[B-]gnore,
\[A]quello che \[B-]porta \[F#]la tua \[B-]pace,
\[B-]e saprà \[E-]perdo\[B-]nare,
\[A]per il tuo a\[B-]more \[F#]saprà a\[B-]mare.
\endverse




%%%%% RITORNELLO
\beginchorus
\textnote{\textbf{Rit.}}
\[G]Lau\[A]dato
\[D*]sii \[F#*]mi Si\[B-]gnore,
\[E-]per la \[B-]terra e \[F#]le tue crea\[B-]ture
\[G]Lau\[A]dato
\[D*]sii \[F#*]mi Si\[B-]gnore,
\[E-]per la \[B-]terra e \[F#]le tue crea\[B-]ture
\endchorus




%%%%% STROFA
\beginverse		%Oppure \beginverse* se non si vuole il numero di fianco
%\memorize 		% <<< DECOMMENTA se si vuole utilizzarne la funzione
%\chordsoff		% <<< DECOMMENTA se vuoi una strofa senza accordi
\[B-]Laudato \[E-]sii, mi Si\[B-]gnore,
\[A]per sora \[B-]morte \[F#]corpo\[B-]rale,
\[B-]dalla quale \[E-]homo vi\[B-]vente
\[A]non potrà mai, \[F#]mai scap\[A]pare.
\endverse




%%%%% RITORNELLO
\beginchorus
\textnote{\textbf{Rit.}}
\[G]Lau\[A]dato
\[D*]sii \[F#*]mi Si\[B-]gnore,
\[E-]per la \[B-]terra e \[F#]le tue crea\[B-]ture
\[G]Lau\[A]dato
\[D*]sii \[F#*]mi Si\[B-]gnore,
\[E-]per la \[B-]terra e \[F#]le tue crea\[B-]ture
\endchorus





%%%%% STROFA
\beginverse		%Oppure \beginverse* se non si vuole il numero di fianco
%\memorize 		% <<< DECOMMENTA se si vuole utilizzarne la funzione
%\chordsoff		% <<< DECOMMENTA se vuoi una strofa senza accordi
\[B-]Laudate \[E-]e bene\[A]dite,
\[A]ringrazi\[B-]ate \[F#]e ser\[B-]vite,
\[B-]il Signore \[E-]con humil\[B-]tate,
\[A]ringra\[B-]ziate \[F#]e ser\[B-]vite.
\endverse




%%%%% RITORNELLO
\beginchorus
\textnote{\textbf{Rit.}}
\[G]Lau\[A]dato
\[D*]sii \[F#*]mi Si\[B-]gnore,
\[E-]per la \[B-]terra e \[F#]le tue crea\[B-]ture
\[G]Lau\[A]dato
\[D*]sii \[F#*]mi Si\[B-]gnore,
\[E-]per la \[B-]terra e \[F#]le tue crea\[B-]ture
\endchorus




\endsong
%------------------------------------------------------------
%			FINE CANZONE
%------------------------------------------------------------



%-------------------------------------------------------------
%			INIZIO	CANZONE
%-------------------------------------------------------------


%titolo: 	Canto degli umili
%autore: 	D. Machetta
%tonalita: 	Re



%%%%%% TITOLO E IMPOSTAZONI
\beginsong{Canto degli umili}[by={Cantico di Anna — D. Machetta}] 	% <<< MODIFICA TITOLO E AUTORE
\transpose{0} 						% <<< TRASPOSIZIONE #TONI (0 nullo)
%\preferflats  %SE VOGLIO FORZARE i bemolle come alterazioni
%\prefersharps %SE VOGLIO FORZARE i # come alterazioni
\momenti{}							% <<< INSERISCI MOMENTI	
% momenti vanno separati da ; e vanno scelti tra:
% Ingresso; Atto penitenziale; Acclamazione al Vangelo; Dopo il Vangelo; Offertorio; Comunione; Ringraziamento; Fine; Santi; Pasqua; Avvento; Natale; Quaresima; Canti Mariani; Battesimo; Prima Comunione; Cresima; Matrimonio; Meditazione; Spezzare del pane;
\ifchorded
	%\textnote{Tonalità migliore }	% <<< EV COMMENTI (tonalità originale/migliore)
\fi




%%%%%% INTRODUZIONE
\ifchorded
\vspace*{\versesep}
\musicnote{
\begin{minipage}{0.48\textwidth}
\textbf{Intro}
\hfill 
%( \eighthnote \, 80)   % <<  MODIFICA IL TEMPO
% Metronomo: \eighthnote (ottavo) \quarternote (quarto) \halfnote (due quarti)
\end{minipage}
} 	
\vspace*{-\versesep}
\beginverse*


\nolyrics

%---- Prima riga -----------------------------
\vspace*{-\versesep}
\[D] \[G] \[D]	 % \[*D] per indicare le pennate, \rep{2} le ripetizioni

%---- Ogni riga successiva -------------------
%\vspace*{-\versesep}
%\[G] \[C]  \[D]	

%---- Ev Indicazioni -------------------------			
%\textnote{\textit{(Oppure tutta la strofa)} }	

\endverse
\fi









%%%%% STROFA
\beginverse		%Oppure \beginverse* se non si vuole il numero di fianco
%\memorize 		% <<< DECOMMENTA se si vuole utilizzarne la funzione
%\chordsoff		% <<< DECOMMENTA se vuoi una strofa senza accordi

\[D]L'arco dei \[G]forti s'è spez\[D]zato,
\[B-]gli umili si \[F#-]vestono \[B-]della tua \[D]forza.
\[G]Grande è il nostro \[A7]Dio!

\endverse




%%%%% RITORNELLO
\beginchorus
\textnote{\textbf{Rit.}}

\[D]Non potrò ta\[F#-]cere, mio \[B-]Signore,
i bene\[G]fici del tuo a\[A7]mo\[D]re.

\endchorus



%%%%% STROFA
\beginverse		%Oppure \beginverse* se non si vuole il numero di fianco
%\memorize 		% <<< DECOMMENTA se si vuole utilizzarne la funzione
%\chordsoff		% <<< DECOMMENTA se vuoi una strofa senza accordi

\[D]Dio solleva il \[G]misero dal \[D]fango,
\[B-]libera il \[F#-]povero \[B-]dall'ingius\[D]tizia.
\[G]Grande è il nostro \[A7]Dio!

\endverse




%%%%% STROFA
\beginverse		%Oppure \beginverse* se non si vuole il numero di fianco
%\memorize 		% <<< DECOMMENTA se si vuole utilizzarne la funzione
%\chordsoff		% <<< DECOMMENTA se vuoi una strofa senza accordi


\[D]Dio tiene i \[G]cardini del \[D]mondo,
\[B-]veglia sui \[F#-]giusti, \[B-]guida i loro \[D]passi.
\[G]Grande è il nostro \[A7]Dio!

\endverse



\endsong
%------------------------------------------------------------
%			FINE CANZONE
%------------------------------------------------------------




%-------------------------------------------------------------
%			INIZIO	CANZONE
%-------------------------------------------------------------


%titolo: 	Custodiscimi
%autore: 	C. Neuf
%tonalita: 	Mi-



%%%%%% TITOLO E IMPOSTAZONI
\beginsong{Custodiscimi}[by={Chemin\ Neuf}]
\transpose{0} 						% <<< TRASPOSIZIONE #TONI (0 nullo)
%\preferflats  %SE VOGLIO FORZARE i bemolle come alterazioni
%\prefersharps %SE VOGLIO FORZARE i # come alterazioni
\momenti{Ringraziamento}							% <<< INSERISCI MOMENTI	
% momenti vanno separati da ; e vanno scelti tra:
% Ingresso; Atto penitenziale; Acclamazione al Vangelo; Dopo il Vangelo; Offertorio; Comunione; Ringraziamento; Fine; Santi; Pasqua; Avvento; Natale; Quaresima; Canti Mariani; Battesimo; Prima Comunione; Cresima; Matrimonio; Meditazione; Spezzare del pane;
\ifchorded
	%\textnote{Tonalità migliore }	% <<< EV COMMENTI (tonalità originale/migliore)
\fi







%%%%%% INTRODUZIONE
\ifchorded
\vspace*{\versesep}
\musicnote{
\begin{minipage}{0.48\textwidth}
\textbf{Intro}
\hfill 
%( \eighthnote \, 80)   % <<  MODIFICA IL TEMPO
% Metronomo: \eighthnote (ottavo) \quarternote (quarto) \halfnote (due quarti)
\end{minipage}
} 	
\vspace*{-\versesep}
\beginverse*
\nolyrics

%---- Prima riga -----------------------------
\vspace*{-\versesep}
\[E-] \[E-] \[A-*] \[B7*] \[E-]	 % \[*D] per indicare le pennate, \rep{2} le ripetizioni

%---- Ogni riga successiva -------------------
%\vspace*{-\versesep}
%\[G] \[C]  \[D]	

%---- Ev Indicazioni -------------------------			
%\textnote{\textit{(Oppure tutta la strofa)} }	

\endverse
\fi

\beginverse
\[E-]Ho detto a Dio, \[A-*]senza \[B7*]di \[E-]te,
alcun \[D]bene non \[G]ho, custo\[A-]disci\[B7]mi.
\[E-]Magnifica è la \[A-*]mia e\[B7*]redi\[E-]tà,
bene\[D]detto sei \[G]Tu, sempre \[A-*]sei \[B-7*]con \[E-]me.
\endverse

\beginchorus
\textnote{\textbf{Rit.}}
Custo\[E-*]di\[A-*]sci\[D]mi, mia \[G]forza sei \[A-]Tu,
Custo\[E-*]di\[A-*]sci\[D]mi, mia \[C]gioia Ge\[E-]sù. \rep{2}
\endchorus

\beginverse
\chordsoff
Ti pongo sempre innanzi a me
al sicuro sarò, mai vacillerò!
Via, verità e vita sei;
Mio Dio credo che Tu mi guiderai.
\endverse



\endsong






%-------------------------------------------------------------
%			INIZIO	CANZONE
%-------------------------------------------------------------


%titolo: 	E camminava con loro
%autore: 	???
%tonalita: 	Fa 



%%%%%% TITOLO E IMPOSTAZONI
\beginsong{E camminava con loro}[by={}] 	% <<< MODIFICA TITOLO E AUTORE
\transpose{0} 						% <<< TRASPOSIZIONE #TONI (0 nullo)
%\preferflats  %SE VOGLIO FORZARE i bemolle come alterazioni
%\prefersharps %SE VOGLIO FORZARE i # come alterazioni
\momenti{}							% <<< INSERISCI MOMENTI	
% momenti vanno separati da ; e vanno scelti tra:
% Ingresso; Atto penitenziale; Acclamazione al Vangelo; Dopo il Vangelo; Offertorio; Comunione; Ringraziamento; Fine; Santi; Pasqua; Avvento; Natale; Quaresima; Canti Mariani; Battesimo; Prima Comunione; Cresima; Matrimonio; Meditazione; Spezzare del pane;
\ifchorded
	%\textnote{Tonalità migliore }	% <<< EV COMMENTI (tonalità originale\migliore)
\fi


%%%%%% INTRODUZIONE
\ifchorded
\vspace*{\versesep}
\musicnote{
\begin{minipage}{0.48\textwidth}
\textbf{Intro}
\hfill 
%( \eighthnote \, 80)   % <<  MODIFICA IL TEMPO
% Metronomo: \eighthnote (ottavo) \quarternote (quarto) \halfnote (due quarti)
\end{minipage}
} 	
\vspace*{-\versesep}
\beginverse*


\nolyrics

%---- Prima riga -----------------------------
\vspace*{-\versesep}
\[F]\[C]\[D-] % \[*D] per indicare le pennate, \rep{2} le ripetizioni

%---- Ogni riga successiva -------------------
\vspace*{-\versesep}
\[G7]\[C7]		

%---- Ev Indicazioni -------------------------			
%\textnote{\textit{(Oppure tutta la strofa)} }	

\endverse
\fi




%%%%% STROFA
\beginverse		%Oppure \beginverse* se non si vuole il numero di fianco
\memorize 		% <<< DECOMMENTA se si vuole utilizzarne la funzione
%\chordsoff		% <<< DECOMMENTA se vuoi una strofa senza accordi

\[F]Ecco il tempo nuovo in c\[C]ui
il Padre mio \[D-]darà
la Grazia ad ogni \[G7]uo\[C7]mo.
\[F]Già lo Spirito che è in \[C]me
annuncia libe\[D-]rtà,
ai poveri la \[B&-]gio\[C7]ia.
\endverse
\beginverse*	
\[D-]Croce fu e si \[C]spense in noi
la Pa\[B&]rola che dava \[C]luce agli occhi.
\[D-]"Stolti e tardi nel \[C]credere";
la sua \[SI&]voce scaldò il \[G7]cuo\[C]re.
\endverse


%%%%% RITORNELLO
\beginchorus
\textnote{\textbf{Rit.}}

\[F]Come un volto \[C]amico, \[D-]verità \[A-]inattesa:
\[B&]è Parola \[F]eterna, \[B&]Pane \[C]vi\[C7]vo.
\[F]Corre nuovo i\[C]l passo, \[D-]carico d'\[A-]annuncio:
\[B&]è risorto, \[F]vive \[B&]e cam-\[F]mi-\[C]na con \[F]noi.

\endchorus




%%%%%% EV. INTERMEZZO
\beginverse*
\vspace*{1.3\versesep}
{
	\nolyrics
	\textnote{Intermezzo strumentale}
	
	\ifchorded

	%---- Prima riga -----------------------------
	\vspace*{-\versesep}
	\[F]\[C]\[D-]
	%---- Ogni riga successiva -------------------
	\vspace*{-\versesep}
    \[G7]\[C7]


	\fi
	%---- Ev Indicazioni -------------------------			
	%\textnote{\textit{(ripetizione della strofa)}} 
	 
}
\vspace*{\versesep}
\endverse


%%%%% STROFA
\beginverse		%Oppure \beginverse* se non si vuole il numero di fianco
\memorize 		% <<< DECOMMENTA se si vuole utilizzarne la funzione
%\chordsoff		% <<< DECOMMENTA se vuoi una strofa senza accordi
\[F]Questo pane che vi \[C]do
è il corpo mio per \[D-]voi;
sia fatto in mia me\[G7]mor\[C7]ia.
\[F]Questo calice sa\[C]rà,
nel sangue mio per \[D-]voi,
un'alleanza \[B&-]nuo\[C7]va.
\endverse
\beginverse*	
\[D-]Croce fu e fug\[C]gimmo noi,
rinne\[B&]gando chi era am\[C]ore eterno.
\[D-]"Resta qui, si fa \[C]sera ormai";
e di\[B&]vise ancora il \[G7]pa\[C]ne.
\endverse




%%%%% STROFA
\beginverse		%Oppure \beginverse* se non si vuole il numero di fianco
%\memorize 		% <<< DECOMMENTA se si vuole utilizzarne la funzione
%\chordsoff		% <<< DECOMMENTA se vuoi una strofa senza accordi
\[F]"Io vi mando ad annun\[C]ciar
la pace a chi non \[D-]sa
che il Regno si avv\[G7]ici\[C7]na.
\[F]Chi vi accoglie, in veri\[C]tà,
accoglie anche \[D-]me
e chi mi ha \[B&-]manda\[C7]to".
\endverse
\beginverse*	
\[D-]Croce fu ed in\[C]creduli
fummo so\[B&]rdi a chi lo di\[C]ceva vivo.
\[D-]"Era Lui per la \[C]via con noi";
ripa\[B&]rtimmo senza \[G7]indu\[C]gio.
\endverse





%%%%% RITORNELLO
\beginchorus
\textnote{\textbf{Rit.}}
\[F]Come un volto \[C]amico, \[D-]verità \[A-]inattesa:
\[B&]è Parola \[F]eterna, \[B&]Pane \[C]vi\[C7]vo.
\[F]Corre nuovo i\[C]l passo, \[D-]carico d'\[A-]annuncio:
\[B&]è risorto, \[F]vive \[B&]e 
\endchorus


%%%%%% EV. FINALE

\beginchorus %oppure \beginverse*
\vspace*{1.3\versesep}
\textnote{\textbf{Finale} \textit{(rallentando)}} %<<< EV. INDICAZIONI
 cam-\[F]mi-\[C]na con \[F]noi. \[F*]
\endchorus  %oppure \endverse







\endsong
%------------------------------------------------------------
%			FINE CANZONE
%------------------------------------------------------------




%-------------------------------------------------------------
%			INIZIO	CANZONE
%-------------------------------------------------------------


%titolo: 	E sei rimasto qui
%autore: 	Gen\ Rosso
%tonalita: 	Fa 



%%%%%% TITOLO E IMPOSTAZONI
\beginsong{E sei rimasto qui}[by={Gen\ Rosso}]
\transpose{0} 						% <<< TRASPOSIZIONE #TONI (0 nullo)
%\preferflats  %SE VOGLIO FORZARE i bemolle come alterazioni
%\prefersharps %SE VOGLIO FORZARE i # come alterazioni
\momenti{Comunione; Cresima}							% <<< INSERISCI MOMENTI	
% momenti vanno separati da ; e vanno scelti tra:
% Ingresso; Atto penitenziale; Acclamazione al Vangelo; Dopo il Vangelo; Offertorio; Comunione; Ringraziamento; Fine; Santi; Pasqua; Avvento; Natale; Quaresima; Canti Mariani; Battesimo; Prima Comunione; Cresima; Matrimonio; Meditazione; Spezzare del pane;
\ifchorded
	%\textnote{Tonalità migliore }	% <<< EV COMMENTI (tonalità originale/migliore)
\fi



%%%%%% INTRODUZIONE
\ifchorded
\vspace*{\versesep}
\musicnote{
\begin{minipage}{0.48\textwidth}
\textbf{Intro}
\hfill 
%( \eighthnote \, 80)   % <<  MODIFICA IL TEMPO
% Metronomo: \eighthnote (ottavo) \quarternote (quarto) \halfnote (due quarti)
\end{minipage}
} 	
\vspace*{-\versesep}
\beginverse*


\nolyrics

%---- Prima riga -----------------------------
\vspace*{-\versesep}
 \[F] \[C] \[F] 	 % \[*D] per indicare le pennate, \rep{2} le ripetizioni

%---- Ogni riga successiva -------------------
\vspace*{-\versesep}
\[F] \[C] \[B&]	

%---- Ev Indicazioni -------------------------			
%\textnote{\textit{(Oppure tutta la strofa)} }	

\endverse
\fi



%%%%% STROFA
\beginverse		%Oppure \beginverse* se non si vuole il numero di fianco
\memorize 		% <<< DECOMMENTA se si vuole utilizzarne la funzione
%\chordsoff		% <<< DECOMMENTA se vuoi una strofa senza accordi
\[F] Perché la sete d'infi\[B&/F]nito? \brk \[G-] Perché la fame d'immor\[D-]tali\[C]tà?
\[F] Sei Tu che hai messo dentro \[B&/F]l'uomo  \brk \[G-] il desiderio dell'e\[F/C]terni\[C]tà!
Ma \[G-]Tu sapevi \[F/A]che quel vuoto \[B&]lo colmavi \[F/A]Tu,
per \[G-]questo sei ve\[F/A]nuto in mezzo a \[C]noi.
\endverse




%%%%% RITORNELLO
\beginchorus
\textnote{\textbf{Rit.}}
E \[F]sei rimasto qui, \[B&/F]visibile mistero.
E \[F]sei rimasto qui, \[D-]cuore del mondo in\[C]tero.
E \[B&]rimarrai con noi fin\[A-]ché quest'uni\[D-]verso gire\[G]rà.
Sal\[F]vezza dell'u\[C]mani\[F]tà. \[C] 
\endchorus





%%%%% STROFA
\beginverse		%Oppure \beginverse* se non si vuole il numero di fianco
%\memorize 		% <<< DECOMMENTA se si vuole utilizzarne la funzione
%\chordsoff		% <<< DECOMMENTA se vuoi una strofa senza accordi
\[D] Si apre il cielo del fu\[G/D]turo,   \brk \[E-] il muro della morte or\[B-]mai non \[A]c'è.
\[D]Tu, Pane vivo, ci fai \[G/B]Uno:\brk \[E-]  richiami tutti i figli at\[D]torno a \[A]Te.
E \[E-]doni il tuo \[D/F#]Spirito che \[G]lascia dentro \[D/F#]noi
il \[E-]germe della  \[D/F#]sua immortali\[A]tà.
\endverse






%%%%% RITORNELLO
\beginchorus
\textnote{\textbf{Rit.}}
\[D]Sei rimasto qui, \[G/D]visibile mistero.
\[D]Sei rimasto qui, \[B-]cuore del mondo in\[A]tero.
E \[G]rimarrai con noi fin\[F#-]ché quest'uni\[B-]verso gire\[E]rà.
Sal\[D]vezza dell'u\[A]mani\[D]tà. \[C] 
\endchorus




%%%%% STROFA
\beginverse		%Oppure \beginverse* se non si vuole il numero di fianco
%\memorize 		% <<< DECOMMENTA se si vuole utilizzarne la funzione
%\chordsoff		% <<< DECOMMENTA se vuoi una strofa senza accordi
^ Presenza vera nel mi^stero, \brk ^  ma più reale di ogni ^realtà, ^
^ da te ogni cosa prende ^vita  \brk ^ e tutto un giorno a te ri^torne^rà.
Var^cando l'infi^nito tutti ^troveremo in ^Te
un ^Sole immenso ^di felici\[G-]tà. \[A-]\[B&]\[C]
\endverse




%%%%% STROFA
\beginverse*		%Oppure \beginverse* se non si vuole il numero di fianco
%\memorize 		% <<< DECOMMENTA se si vuole utilizzarne la funzione
%\chordsoff		% <<< DECOMMENTA se vuoi una strofa senza accordi
\[F/A]Noi,  \[B&/D]trasformati in \[C]Te, sa\[F]remo il seme \[B&]che
fa\[G-]rà fiorire l'\[F/A]universo \[B&]nella Trini\[C]tà.
\[F/A]Noi,  \[B&/D]trasformati in \[C]Te, sa\[F]remo il seme \[B&]che
fa\[G-]rà fiorire \[F/A]tutto l'uni\[B&]verso insieme a \[C]Te.
\endverse


%%%%% RITORNELLO
\beginchorus
\textnote{\textbf{Rit.}}
E \[G]sei rimasto qui, vi\[C]sibile mistero.
\[G]Sei rimasto qui, \[E-]cuore del mondo in\[D]tero.
E \[C]rimarrai con noi \brk fin\[B-]ché quest'uni\[E-]verso gire\[A]rà.  \[D] 
\[G]Sei rimasto qui, vi\[C]sibile mistero.
\[G]Sei rimasto qui, \[E-]cuore del mondo in\[D]tero.
E \[C]rimarrai con noi fin\[B-]ché quest'uni\[E-]verso gire\[A]rà.
\[G/D]Ieri oggi e sempre. \[A-] \[G/B] 
\[C]Sal\[C7+/D]vezza dell'u\[D]manità. \[G] 

\endchorus



%%%%%% EV. CHIUSURA SOLO STRUMENTALE
\ifchorded
\beginchorus %oppure \beginverse*
\vspace*{1.3\versesep}
\textnote{Chiusura strumentale} %<<< EV. INDICAZIONI

\[D] \[G] \[G/B] \[D] \[G] \[C] \[G] 

\endchorus  %oppure \endverse
\fi


\endsong
%------------------------------------------------------------
%			FINE CANZONE
%------------------------------------------------------------




%-------------------------------------------------------------
%			INIZIO	CANZONE
%-------------------------------------------------------------


%titolo: 	Ecco l'uomo
%autore: 	P. Sequeri
%tonalita:  Mi- 



%%%%%% TITOLO E IMPOSTAZONI
\beginsong{Ecco l'uomo}[by={P. Sequeri}]  % <<< MODIFICA TITOLO E AUTORE
\transpose{0} 						% <<< TRASPOSIZIONE #TONI (0 nullo)
%\preferflats  %SE VOGLIO FORZARE i bemolle come alterazioni
%\prefersharps %SE VOGLIO FORZARE i # come alterazioni
\momenti{}							% <<< INSERISCI MOMENTI	
% momenti vanno separati da ; e vanno scelti tra:
% Ingresso; Atto penitenziale; Acclamazione al Vangelo; Dopo il Vangelo; Offertorio; Comunione; Ringraziamento; Fine; Santi; Pasqua; Avvento; Natale; Quaresima; Canti Mariani; Battesimo; Prima Comunione; Cresima; Matrimonio; Meditazione; Spezzare del pane;
\ifchorded
	%\textnote{Tonalità migliore }	% <<< EV COMMENTI (tonalità originale/migliore)
\fi

\beginverse
\[E-7]Nella memoria di \[A-7]questa Passione
\[D7]noi ti chiediamo per\[G]dono, Si\[B7]gnore
\[E-7]per ogni volta che \[A-7]abbiamo lasciato
\[D7]il tuo fratello so\ch{B}{f}{f}{ff}rire da \[B7]solo.
\endverse

\beginchorus
\textnote{\textbf{Rit.}}
\[E-]Noi ti pre\[A-]ghiamo \[D7]Uomo della \[G]Croce;
\[E-]Figlio e Fratel\[F#-dim]lo, \brk \[B7/9]noi speriamo in \[E-]te! \rep{2}
\endchorus

\beginverse
\chordsoff
Nella memoria di questa tua Morte
noi ti chiediamo coraggio, Signore
per ogni volta che il dono d'amore
ci chiederà di soffrire da soli.
\endverse

\beginverse
\chordsoff
Nella memoria dell'Ultima Cena
noi spezzeremo di nuovo il tuo Pane
ed ogni volta il tuo Corpo donato
sarà la nostra speranza di vita.
\endverse
\endsong


%-------------------------------------------------------------
%			INIZIO	CANZONE
%-------------------------------------------------------------


%titolo: 	Francesco vai
%autore: 	M. C. Bizzeti
%tonalita: 	Mi-



%%%%%% TITOLO E IMPOSTAZONI
\beginsong{Francesco vai}[by={M. C. Bizzeti}]
\transpose{0} 						% <<< TRASPOSIZIONE #TONI (0 nullo)
%\preferflats  %SE VOGLIO FORZARE i bemolle come alterazioni
%\prefersharps %SE VOGLIO FORZARE i # come alterazioni
\momenti{}							% <<< INSERISCI MOMENTI	
% momenti vanno separati da ; e vanno scelti tra:
% Ingresso; Atto penitenziale; Acclamazione al Vangelo; Dopo il Vangelo; Offertorio; Comunione; Ringraziamento; Fine; Santi; Pasqua; Avvento; Natale; Quaresima; Canti Mariani; Battesimo; Prima Comunione; Cresima; Matrimonio; Meditazione; Spezzare del pane;
\ifchorded
	%\textnote{Tonalità migliore }	% <<< EV COMMENTI (tonalità originale/migliore)
\fi




%%%%%% INTRODUZIONE
\ifchorded
\vspace*{\versesep}
\textnote{Intro: \qquad \qquad  }%(\eighthnote 116) % <<  MODIFICA IL TEMPO
% Metronomo: \eighthnote (ottavo) \quarternote (quarto) \halfnote (due quarti)
\vspace*{-\versesep}
\beginverse*

\nolyrics

%---- Prima riga -----------------------------
\vspace*{-\versesep}
\[E-] \[D] \[E-]	 % \[*D] per indicare le pennate, \rep{2} le ripetizioni

%---- Ogni riga successiva -------------------
%\vspace*{-\versesep}
%\[G] \[C]  \[D]	

%---- Ev Indicazioni -------------------------			
%\textnote{\textit{(Oppure tutta la strofa)} }	

\endverse
\fi





%%%%% STROFA
\beginverse		%Oppure \beginverse* se non si vuole il numero di fianco
\memorize 		% <<< DECOMMENTA se si vuole utilizzarne la funzione
%\chordsoff		% <<< DECOMMENTA se vuoi una strofa senza accordi
\[E-]Quello che io vivo non mi \[D]basta \[E-]più,
tutto quel che avevo non mi \[D]serve \[E-]più:
io cerche\[B]rò quello che dav\[E-]vero vale,
e non più il \[A-]servo, ma il pa\[C]drone segui\[B]rò!
\endverse




%%%%% RITORNELLO
\beginchorus
\textnote{\textbf{Rit.}}
Francesco, \[E-]vai, ri\[D]para la mia \[E-]casa!
Fran\[D]cesco, \[E-]vai, non \[D]vedi che è in ro\[G]vina?
E non te\[A-]mere: \[C]io sarò con \[G*]te do\[B*]vunque an\[E-]drai.
\[D]Francesco, \[E-]vai!
\endchorus



%%%%% STROFA
\beginverse		%Oppure \beginverse* se non si vuole il numero di fianco
%\memorize 		% <<< DECOMMENTA se si vuole utilizzarne la funzione
%\chordsoff		% <<< DECOMMENTA se vuoi una strofa senza accordi
Nel ^buio e nel silenzio ti ho cer^cato, ^Dio;
dal fondo della notte ho alzato il ^grido ^mio
e gride^rò finché non a^vrò risposta
per co^noscere la ^tua volon^tà.
\endverse


%%%%% STROFA
\beginverse		%Oppure \beginverse* se non si vuole il numero di fianco
%\memorize 		% <<< DECOMMENTA se si vuole utilizzarne la funzione
\chordsoff		% <<< DECOMMENTA se vuoi una strofa senza accordi
Al^tissimo Signore, cosa ^vuoi da ^me?
Tutto quel che avevo l'ho do^nato a ^te.
Ti segui^rò nella gioia e ^nel dolore
e della ^vita mia una ^lode a te fa^rò.
\endverse

%%%%% STROFA
\beginverse		%Oppure \beginverse* se non si vuole il numero di fianco
%\memorize 		% <<< DECOMMENTA se si vuole utilizzarne la funzione
\chordsoff		% <<< DECOMMENTA se vuoi una strofa senza accordi
^Quello che cercavo l'ho tro^vato ^qui:
ora ho riscoperto nel mio ^dirti ^sì
la liber^tà di essere ^figlio tuo,
fratello e ^sposo di Ma^donna pover^tà.
\endverse



\endsong
%------------------------------------------------------------
%			FINE CANZONE
%------------------------------------------------------------




%-------------------------------------------------------------
%			INIZIO	CANZONE
%-------------------------------------------------------------


%titolo: 	Il cuore nuovo
%autore: 	D. MAchetta
%tonalita: 	Do-



%%%%%% TITOLO E IMPOSTAZONI
\beginsong{Il cuore nuovo}[by={D. Machetta}] 	% <<< MODIFICA TITOLO E AUTORE
%#ALIAS Io verrò a salvarvi
\transpose{2} 						% <<< TRASPOSIZIONE #TONI (0 nullo)
\preferflats  %SE VOGLIO FORZARE i bemolle come alterazioni
%\prefersharps %SE VOGLIO FORZARE i # come alterazioni
\momenti{Atto penitenziale; }							% <<< INSERISCI MOMENTI	
% momenti vanno separati da ; e vanno scelti tra:
% Ingresso; Atto penitenziale; Acclamazione al Vangelo; Dopo il Vangelo; Offertorio; Comunione; Ringraziamento; Fine; Santi; Pasqua; Avvento; Natale; Quaresima; Canti Mariani; Battesimo; Prima Comunione; Cresima; Matrimonio; Meditazione; Spezzare del pane;
\ifchorded
	%\textnote{Tonalità migliore }	% <<< EV COMMENTI (tonalità originale\migliore)
\fi


%%%%%% INTRODUZIONE
\ifchorded
\vspace*{\versesep}
\musicnote{
\begin{minipage}{0.48\textwidth}
\textbf{Intro}
\hfill 
%( \eighthnote \, 80)   % <<  MODIFICA IL TEMPO
% Metronomo: \eighthnote (ottavo) \quarternote (quarto) \halfnote (due quarti)
\end{minipage}
} 	
\vspace*{-\versesep}
\beginverse*
\nolyrics

%---- Prima riga -----------------------------
\vspace*{-\versesep}
\[C-] \[G-] \[C-]	 % \[*D] per indicare le pennate, \rep{2} le ripetizioni

%---- Ogni riga successiva -------------------
%\vspace*{-\versesep}
%\[G] \[C]  \[D]	

%---- Ev Indicazioni -------------------------			
%\textnote{\textit{(Oppure tutta la strofa)} }	

\endverse
\fi




%%%%% STROFA
\beginverse		%Oppure \beginverse* se non si vuole il numero di fianco
\memorize 		% <<< DECOMMENTA se si vuole utilizzarne la funzione
%\chordsoff		% <<< DECOMMENTA se vuoi una strofa senza accordi

\[C-]Io verrò a salvarvi tra le \[G-]genti,
vi condur\[A&]rò nella vostra di\[G-]mora.
\[C-]Spargerò su voi torrenti d'\[G-]acque:
da ogni \[A&]colpa sarete la\[C]vati.

\endverse




%%%%% RITORNELLO
\beginchorus
\textnote{\textbf{Rit.}}

\[E&]Dio ci da\[F]rà un cuore \[B&]nuovo,
\[E&]porrà in \[F]noi uno spirito \[G]nuovo.

\endchorus



%%%%% STROFA
\beginverse		%Oppure \beginverse* se non si vuole il numero di fianco
%\memorize 		% <<< DECOMMENTA se si vuole utilizzarne la funzione
%\chordsoff		% <<< DECOMMENTA se vuoi una strofa senza accordi

\[C-]Voglio liberarvi dai \[G-]peccati,
abbatte\[A&]rò ogni falso di\[G-]o.
\[C-]Tolgo il vostro cuore di pi\[G-]etra
per rega\[A&]larvi un cuore di \[C]carne.

\endverse



%%%%% STROFA
\beginverse		%Oppure \beginverse* se non si vuole il numero di fianco
%\memorize 		% <<< DECOMMENTA se si vuole utilizzarne la funzione
%\chordsoff		% <<< DECOMMENTA se vuoi una strofa senza accordi

\[C-]Voi osserverete la mia \[G-]legge
e abite\[A&]rete la terra dei \[G-]padri.
\[C-]Voi sarete il popolo fe\[G-]dele
e io sa\[A&]rò il vostro Dio per \[C]sempre.

\endverse




\endsong
%------------------------------------------------------------
%			FINE CANZONE
%------------------------------------------------------------



%-------------------------------------------------------------
%			INIZIO	CANZONE
%-------------------------------------------------------------


%titolo: 	Il Signore ci ha salvato
%autore: 	L. Capello, A. Roncari
%tonalita: 	RE-



%%%%%% TITOLO E IMPOSTAZONI
\beginsong{Il Signore ci ha salvato}[by={L. Capello, A. Roncari}] 	% <<< MODIFICA TITOLO E AUTORE
\transpose{0} 						% <<< TRASPOSIZIONE #TONI (0 nullo)
%\preferflats  %SE VOGLIO FORZARE i bemolle come alterazioni
%\prefersharps %SE VOGLIO FORZARE i # come alterazioni
\momenti{Quaresima}							% <<< INSERISCI MOMENTI	
% momenti vanno separati da ; e vanno scelti tra:
% Ingresso; Atto penitenziale; Acclamazione al Vangelo; Dopo il Vangelo; Offertorio; Comunione; Ringraziamento; Fine; Santi; Pasqua; Avvento; Natale; Quaresima; Canti Mariani; Battesimo; Prima Comunione; Cresima; Matrimonio; Meditazione; Spezzare del pane;
\ifchorded
	%\textnote{Tonalità migliore }	% <<< EV COMMENTI (tonalità originale\migliore)
\fi


%%%%%% INTRODUZIONE
\ifchorded
\vspace*{\versesep}
\musicnote{
\begin{minipage}{0.48\textwidth}
\textbf{Intro}
\hfill 
%( \eighthnote \, 80)   % <<  MODIFICA IL TEMPO
% Metronomo: \eighthnote (ottavo) \quarternote (quarto) \halfnote (due quarti)
\end{minipage}
} 	
\vspace*{-\versesep}
\beginverse*

\nolyrics

%---- Prima riga -----------------------------
\vspace*{-\versesep}
\[D-] \[G-] \[D-]	 % \[*D] per indicare le pennate, \rep{2} le ripetizioni

%---- Ogni riga successiva -------------------
%\vspace*{-\versesep}
%\[G] \[C]  \[D]	

%---- Ev Indicazioni -------------------------			
%\textnote{\textit{(Oppure tutta la strofa)} }	

\endverse
\fi




%%%%% STROFA
\beginverse		%Oppure \beginverse* se non si vuole il numero di fianco
\memorize 		% <<< DECOMMENTA se si vuole utilizzarne la funzione
%\chordsoff		% <<< DECOMMENTA se vuoi una strofa senza accordi

Il Sign\[D-]ore ci ha sal\[G-]vato dai ne\[D-]mici
nel pass\[G-]aggio \[C7]del Mar \[F]Rosso:
\[G-]l'acqua che ha travolto gli Egi\[D-]ziani
fu per \[G-]noi \[A7]la sal\[D-]vezza.\[D7]
\endverse




%%%%% RITORNELLO
\beginchorus
\textnote{\textbf{Rit.}}

"Se cono\[G-]scessi il \[C]dono di \[F]Dio
e chi è co\[G-]lui che ti chi\[A7]ede da \[D-]bere,
lo preghe\[B&]resti tu \[C]stesso di \[F]darti
quell'acqua \[G-]viva che ti salve\[A7]rà".

\endchorus



%%%%% STROFA
\beginverse		%Oppure \beginverse* se non si vuole il numero di fianco
%\memorize 		% <<< DECOMMENTA se si vuole utilizzarne la funzione
%\chordsoff		% <<< DECOMMENTA se vuoi una strofa senza accordi

Era\[D-]vamo pro\[G-]strati nel de\[D-]serto,
consu\[G-]mati \[C7]dalla se\[F]te:
qu\[G-]ando fu percossa la \[D-]roccia,
zampi\[G-]llò \[A7]una so\[D-]rgente.\[D7]

\endverse




%%%%% STROFA
\beginverse		%Oppure \beginverse* se non si vuole il numero di fianco
%\memorize 		% <<< DECOMMENTA se si vuole utilizzarne la funzione
%\chordsoff		% <<< DECOMMENTA se vuoi una strofa senza accordi
\chordsoff
Dal\[D-]le mura del te\[G-]mpio di \[D-]Dio
sgorga \[G-]un fi\[C7]ume d'acqua \[F]viva:
\[G-]tutto quel che l'acqua \[D-]toccherà
na\[G-]scerà \[A7]a nuova vi\[D-]ta.\[D7]
\endverse

%%%%% STROFA
\beginverse		%Oppure \beginverse* se non si vuole il numero di fianco
%\memorize 		% <<< DECOMMENTA se si vuole utilizzarne la funzione
%\chordsoff		% <<< DECOMMENTA se vuoi una strofa senza accordi
\chordsoff
Ve\[D-]nga a me chi ha se\[G-]te e chi mi \[D-]cerca,
si di\[G-]sseti co\[C7]lui che in me cr\[F]ede:
\[G-]fiumi d'acqua viva scorre\[D-]ranno
dal mio \[G-]cuo\[A7]re tra\[D-]fitto.\[D7]
\endverse

%%%%% STROFA
\beginverse		%Oppure \beginverse* se non si vuole il numero di fianco
%\memorize 		% <<< DECOMMENTA se si vuole utilizzarne la funzione
%\chordsoff		% <<< DECOMMENTA se vuoi una strofa senza accordi
\chordsoff
Sul\[D-]la croce il Fi\[G-]glio di \[D-]Dio
fu tra\[G-]fitto \[C7]da una la\[F]ncia:
dal cuore dell'Agnello imm\[D-]olato
sca\[G-]turà sa\[A7]ngue ed a\[D-]cqua.\[D7]
\endverse


%%%%% STROFA
\beginverse		%Oppure \beginverse* se non si vuole il numero di fianco
%\memorize 		% <<< DECOMMENTA se si vuole utilizzarne la funzione
%\chordsoff		% <<< DECOMMENTA se vuoi una strofa senza accordi
\chordsoff
Chi \[D-]berrà l'acqu\[G-]a viva che io \[D-]dono
non a\[G-]vrà mai più \[C7]sete in e\[F]terno:
\[G-]in lui diventerà una sor\[D-]gente
za\[G-]mpilla\[A7]nte per se\[D-]mpre.\[D7]
\endverse




\endsong
%------------------------------------------------------------
%			FINE CANZONE
%------------------------------------------------------------




%titolo{Ora è tempo di gioia}
%autore{Gen Rosso}
%album{Se siamo uniti}
%tonalita{Re}
%famiglia{Liturgica}
%gruppo{}
%momenti{}
%identificatore{ora_e_tempo_di_gioia}
%data_revisione{2014_09_30}
%trascrittore{Francesco Endrici}
\beginsong{Ora è tempo di gioia}[by={Gen\ Rosso}]
\beginverse
L'\[D]eco \[E-]torna d'an\[D]tiche \[G]val\[A]li
\[D]la sua \[E-]voce \[D7+]non porta \[C7+]più,
\[B-]ricordo \[F#-]di som\[G]messe \[E7]lacri\[A]me
\[D]di e\[E-7]si\[D]li in terre \[A4]lonta\[D]ne.
\endverse
\beginchorus
\[G]Ora è \[D]tempo di \[C]gio\[D]ia, \brk \[G]non \[A-7]ve \[G]ne ac\[C]cor\[D4]ge\[D]te
\[G]ecco \[D]faccio una \[A-]cosa \[E-]nuova
\[B7]nel de\[C7+]serto una \[B-]strada apri\[E-]rò.
\endchorus
\beginverse
%\chordsoff
^Come l'^onda che ^sulla ^sab^bia
^copre le ^orme e ^poi passa e ^va,
^così nel ^tempo ^si can^cella^no
^le ombre ^scure ^del lungo in^ver^no.
\endverse
\beginverse
%\chordsoff
^Tra i sen^tieri dei ^boschi il ^ven^to
^con i ^rami ^ricompor^rà
^nuove armo^nie ^che tra^sforma^no
^i la^menti ^in canti di ^fe^sta.
\endverse
\endsong
%titolo{Re di gloria}
%autore{Marranzino, De Luca}
%album{Cantiamo con gioia}
%tonalita{Sol}
%famiglia{Liturgica}
%gruppo{}
%momenti{Comunione}
%identificatore{re_di_gloria}
%data_revisione{2011_12_31}
%trascrittore{Francesco Endrici - Manuel Toniato}
\beginsong{Re di gloria}[by={Marranzino, De\ Luca}]

\ifchorded
\beginverse*
\vspace*{-0.8\versesep}
{\nolyrics \[A-7] \[C] \[D] \[G] }
\vspace*{-\versesep}
\endverse
\fi
\beginverse
\[G]Ho incontrato Te Gesù \brk e ogni \[D]cosa in me è cambiata
\[A-]tutta la mia \[A-7]vita ora ti \[D]appar\[A-7]tie\[D]ne
\[G]tutto il mio passato io lo a\ch{C}{f}{f}{ff}i\[B7]do a \[E-]Te \[E-7] 
Ge\[A-]sù Re di \[C7+]gloria mio Si\[D4]gnor\[D7].
\endverse

\beginverse
\chordsoff
Tutto in Te riposa, la mia mente il mio cuore
trovo pace in Te Signor,Tu mi dai la gioia
voglio stare insieme a Te,non lasciarti mai
Gesù Re di gloria mio Signor.
\endverse

\beginchorus
\[D7]Dal Tuo a\[G]more chi \[C7+]mi separe\[D4]rà \[D] 
sulla \[A-7]croce hai \[C7+]dato la \[D4]vita per \[D]me
u\[D7]na co\[G]rona di \[C7+]gloria mi da\[D4]rai \[D] 
quando un \[A-7]gior\[C]no \[D]ti ve\[G]drò.
\endchorus

\beginverse
\chordsoff
Tutto in Te riposa,la mia mente il mio cuore
trovo pace in Te Signor, Tu mi dai la gioia vera
voglio stare insieme a Te, non lasciarti mai
Gesù Re di gloria mio Signor.
\endverse

\beginchorus
\[D7]Dal Tuo a\[G]more chi \[C7+]mi separe\[D4]rà \[D] 
sulla \[A-7]croce hai \[C7+]dato la vita per \[D]me
u\[D7]na co\[G]rona di \[C7+]gloria mi da\[D4]rai \[D] 
quando un \[A-7]gior\[C]no \[D]ti ve\[E&]drò. \[(Sol)]
\[E&7]Dal Tuo a\[Ab]more chi \[D&7+]mi separe\[E&]rà \ldots
\endchorus
\ifchorded
\beginverse*
\vspace*{-\versesep}
{\nolyrics \[B&-7] \[D&7+] \[E&]  \[E&7] \[Ab] \[D&7+] \[E&4] \[E&] }
\endverse
\fi
\beginverse*\bfseries
Io ti a\[B&-7]spet\[D&7+]to \[E&]mio Si\[F-]gnor \[Ab] 
io ti a\[B&-7]spet\[Ab]to \[E&]mio \[C]Si\[F-]gnor \[Ab] 
io ti a\[B&-7]spet\[Ab]to \[E&]mio \[Ab4]Re! \[Ab] 
\endverse
\textnote{Nel caso non si cambi tonalità, riportiamo il finale in Sol maggiore:}
\beginverse*\bfseries
\[D7]Dal Tuo a\[G]more chi \[C7+]mi separe\[D]rà \ldots
\endverse
\ifchorded
\beginverse*
\vspace*{-\versesep}
{\nolyrics \[A-7]  \[C7+]  \[D]  \[D7]  \[G]  \[C7+]  \[D4]  \[D] }
\endverse
\fi
\beginverse*\bfseries
Io ti a\[A-7]spet\[C7+]to \[D]mio Si\[E-]gnor \[G] 
io ti a\[A-7]spet\[G]to \[D]mio \[B]Si\[E-]gnor \[G] 
io ti a\[A-7]spet\[G]to \[D]mio \[G4]Re! \[G] 
\endverse

\endsong



%-------------------------------------------------------------
%			INIZIO	CANZONE
%-------------------------------------------------------------


%titolo: 	Ti seguirò
%autore: 	M. Frisina
%tonalita: 	Do



%%%%%% TITOLO E IMPOSTAZONI
\beginsong{Ti seguirò}[by={M. Frisina}] 	% <<< MODIFICA TITOLO E AUTORE
\transpose{0} 						% <<< TRASPOSIZIONE #TONI (0 nullo)
%\preferflats  %SE VOGLIO FORZARE i bemolle come alterazioni
%\prefersharps %SE VOGLIO FORZARE i # come alterazioni
\momenti{Quaresima}							% <<< INSERISCI MOMENTI	
% momenti vanno separati da ; e vanno scelti tra:
% Ingresso; Atto penitenziale; Acclamazione al Vangelo; Dopo il Vangelo; Offertorio; Comunione; Ringraziamento; Fine; Santi; Pasqua; Avvento; Natale; Quaresima; Canti Mariani; Battesimo; Prima Comunione; Cresima; Matrimonio; Meditazione; Spezzare del pane;
\ifchorded
	%\textnote{Tonalità migliore }	% <<< EV COMMENTI (tonalità originale/migliore)
\fi

%%%%%% INTRODUZIONE
\ifchorded
\vspace*{\versesep}
\musicnote{
\begin{minipage}{0.48\textwidth}
\textbf{Intro}
\hfill 
%( \eighthnote \, 80)   % <<  MODIFICA IL TEMPO
% Metronomo: \eighthnote (ottavo) \quarternote (quarto) \halfnote (due quarti)
\end{minipage}
} 	
\vspace*{-\versesep}
\beginverse*

\nolyrics

%---- Prima riga -----------------------------
\vspace*{-\versesep}
\[C] 	 % \[*D] per indicare le pennate, \rep{2} le ripetizioni

%---- Ogni riga successiva -------------------
%\vspace*{-\versesep}
%\[G] \[C]  \[D]	

%---- Ev Indicazioni -------------------------			
\textnote{\textit{(oppure tutto il ritornello)} }	

\endverse
\fi










%%%%% RITORNELLO
\beginchorus
\textnote{\textbf{Rit.}}

\[C]Ti segui\[G]rò, ti \[A-]seguirò, o Si\[F]gnore
\[C]e \[G]nella \[E]tua \[A-]strada \[F]cammine\[C]rò.

\endchorus




%%%%% STROFA
\beginverse		%Oppure \beginverse* se non si vuole il numero di fianco
%\memorize 		% <<< DECOMMENTA se si vuole utilizzarne la funzione
%\chordsoff		% <<< DECOMMENTA se vuoi una strofa senza accordi

\[C]Ti segui\[G]rò nella \[A-]via dell'a\[F]more
\[C]e \[G]done\[E]rò al \[A-]mondo \[F]la vi\[C]ta.

\endverse




%%%%% STROFA
\beginverse		%Oppure \beginverse* se non si vuole il numero di fianco
%\memorize 		% <<< DECOMMENTA se si vuole utilizzarne la funzione
%\chordsoff		% <<< DECOMMENTA se vuoi una strofa senza accordi

\[C]Ti segui\[G]rò nella \[A-]via del do\[F]lore
\[C]e \[G]la \[E]tua \[A-]croce \[F]ci salve\[C]rà.

\endverse





%%%%% STROFA
\beginverse		%Oppure \beginverse* se non si vuole il numero di fianco
%\memorize 		% <<< DECOMMENTA se si vuole utilizzarne la funzione
%\chordsoff		% <<< DECOMMENTA se vuoi una strofa senza accordi

\[C]Ti segui\[G]rò nella \[A-]via della \[F]gioia
\[C]e \[G]la \[E]tua \[A-]luce \[F]ci guide\[C]rà.

\endverse




\endsong
%------------------------------------------------------------
%			FINE CANZONE
%------------------------------------------------------------


%-------------------------------------------------------------
%			INIZIO	CANZONE
%-------------------------------------------------------------


%titolo: 	Tu sarai profeta
%autore: 	M. Frisina
%tonalita: 	Re 



%%%%%% TITOLO E IMPOSTAZONI
\beginsong{Tu sarai profeta}[by={M. Frisina}] 	% <<< MODIFICA TITOLO E AUTORE
\transpose{0} 						% <<< TRASPOSIZIONE #TONI (0 nullo)
%\preferflats  %SE VOGLIO FORZARE i bemolle come alterazioni
%\prefersharps %SE VOGLIO FORZARE i # come alterazioni
\momenti{}							% <<< INSERISCI MOMENTI	
% momenti vanno separati da ; e vanno scelti tra:
% Ingresso; Atto penitenziale; Acclamazione al Vangelo; Dopo il Vangelo; Offertorio; Comunione; Ringraziamento; Fine; Santi; Pasqua; Avvento; Natale; Quaresima; Canti Mariani; Battesimo; Prima Comunione; Cresima; Matrimonio; Meditazione; Spezzare del pane;
\ifchorded
	%\textnote{Tonalità migliore }	% <<< EV COMMENTI (tonalità originale\migliore)
\fi


%%%%%% INTRODUZIONE
\ifchorded
\vspace*{\versesep}
\textnote{Intro: \qquad \qquad  }%(\eighthnote 116) % <<  MODIFICA IL TEMPO
% Metronomo: \eighthnote (ottavo) \quarternote (quarto) \halfnote (due quarti)
\vspace*{-\versesep}
\beginverse*

\nolyrics

%---- Prima riga -----------------------------
\vspace*{-\versesep}
\[D] \[A] \[B-]	 \[F#-] % \[*D] per indicare le pennate, \rep{2} le ripetizioni

%---- Ogni riga successiva -------------------
\vspace*{-\versesep}
\[G] \[D]  \[A] \[A]	

%---- Ev Indicazioni -------------------------			
%\textnote{\textit{(Oppure tutta la strofa)} }	

\endverse
\fi




%%%%% STROFA
\beginverse		%Oppure \beginverse* se non si vuole il numero di fianco
\memorize 		% <<< DECOMMENTA se si vuole utilizzarne la funzione
%\chordsoff		% <<< DECOMMENTA se vuoi una strofa senza accordi

\[D]Una \[A]luce che ri\[B-]schia\[F#-]ra,
\[D]una \[G]lampada che \[D]ar\[A]de,
\[B-*]u\[G*]na \[G]voce che pro\[A]cla\[B-]ma
\[E-]la Pa\[G]rola di sal\[A4]vezza.
\[D]Precur\[A]sore nella \[B-]gio\[F#-]ia,
\[D]precu\[G]rsore nel \[D]dolo\[A]re,
\[B-*]tu \[G*]che \[G]sveli nel \[A]perdo\[B-]no
\[E-]l'annunzio \[G]di miseri\[A]cord\[D]ia.

\endverse




%%%%% RITORNELLO
\beginchorus
\textnote{\textbf{Rit.}}

Tu sa\[G]ra\[D]i pro\[E-]feta di salve\[B-]zza
\[G]fino ai con\[D]fini della \[A4]terra,
porte\[G]ra\[D]i la \[E-]mia Pa\[B-]rola,
\[G]risplende\[D]rai della mia \[A]lu\[D]ce.

\endchorus



%%%%% STROFA
\beginverse		%Oppure \beginverse* se non si vuole il numero di fianco
%\memorize 		% <<< DECOMMENTA se si vuole utilizzarne la funzione
%\chordsoff		% <<< DECOMMENTA se vuoi una strofa senza accordi

\[D]Forte \[A]amico dello \[B-]Spo\[F#-]so,
\[D]che \[G]gioisci alla sua \[D]vo\[A]ce,
\[B-*]tu \[G*]cam\[G]mini per il \[A]mon\[B-]do
\[E-]per pre\[G]cedere il \[A4]Signore.
\[D]Stende\[A]rò la mia \[B-]ma\[F#-]no
\[D]e por\[G]rò sulla tua \[D]boc\[A]ca
\[B-*]la \[G*]po\[G]tente mia Pa\[A]ro\[B-]la
\[E-]che con\[G]vertirà il \[A]mon\[D]do.

\endverse



\endsong
%------------------------------------------------------------
%			FINE CANZONE
%------------------------------------------------------------



%-------------------------------------------------------------
%			INIZIO	CANZONE
%-------------------------------------------------------------


%titolo: 	Volto dell'Uomo
%autore: 	C. Davide, D. Machetta
%tonalita: 	Mi-



%%%%%% TITOLO E IMPOSTAZONI
\beginsong{Volto dell'Uomo}[by={C. Davide, D. Machetta}] 	% <<< MODIFICA TITOLO E AUTORE
\transpose{0} 						% <<< TRASPOSIZIONE #TONI (0 nullo)
%\preferflats  %SE VOGLIO FORZARE i bemolle come alterazioni
%\prefersharps %SE VOGLIO FORZARE i # come alterazioni
\momenti{Quaresima}							% <<< INSERISCI MOMENTI	
% momenti vanno separati da ; e vanno scelti tra:
% Ingresso; Atto penitenziale; Acclamazione al Vangelo; Dopo il Vangelo; Offertorio; Comunione; Ringraziamento; Fine; Santi; Pasqua; Avvento; Natale; Quaresima; Canti Mariani; Battesimo; Prima Comunione; Cresima; Matrimonio; Meditazione; Spezzare del pane;
\ifchorded
	%\textnote{Tonalità migliore }	% <<< EV COMMENTI (tonalità originale/migliore)
\fi


%%%%%% INTRODUZIONE
\ifchorded
\vspace*{\versesep}
\musicnote{
\begin{minipage}{0.48\textwidth}
\textbf{Intro}
\hfill 
( \halfnote \, 52)   % <<  MODIFICA IL TEMPO
% Metronomo: \eighthnote (ottavo) \quarternote (quarto) \halfnote (due quarti)
\end{minipage}
} 	
\vspace*{-\versesep}
\beginverse*

\nolyrics

%---- Prima riga -----------------------------
\vspace*{-\versesep}
\[E-] \[B-] % \[*D] per indicare le pennate, \rep{2} le ripetizioni

%---- Ogni riga successiva -------------------
\vspace*{-\versesep}
\[D] \[A]  \[B-]	

%---- Ev Indicazioni -------------------------			
%\textnote{\textit{(Oppure tutta la strofa)} }	

\endverse
\fi




%%%%% STROFA
\beginverse		%Oppure \beginverse* se non si vuole il numero di fianco
%\memorize 		% <<< DECOMMENTA se si vuole utilizzarne la funzione
%\chordsoff		% <<< DECOMMENTA se vuoi una strofa senza accordi

\[E-]Volto dell'\[B-]uomo
pene\[D]trato \[A]dal do\[B-]lore,
\[G]volto di \[D]Dio
pene\[E-]trato di \[A]umil\[B]tà,
\[G]scandalo dei \[D]grandi
che con\[A-]fidano nel \[B]mondo,
\[A-]uomo dei do\[E-]lori, pie\[B-7]tà di \[E]noi.

\endverse





%%%%% STROFA
\beginverse		%Oppure \beginverse* se non si vuole il numero di fianco
%\memorize 		% <<< DECOMMENTA se si vuole utilizzarne la funzione
%\chordsoff		% <<< DECOMMENTA se vuoi una strofa senza accordi

\[E-]Volto di \[B-]pace,
di per\[D]dono e \[A]di bon\[B-]tà,
\[G]tu che in si\[D]lenzio
hai \[E-]pagato i \[A]nostri \[B]errori,
\[G]scandalo dei \[D]forti,
di chi ha \[A-]sete di vio\[B]lenza,
\[A-]Cristo Salva\[E-]tore, pie\[B-7]tà di \[E]noi.

\endverse


%%%%% STROFA
\beginverse		%Oppure \beginverse* se non si vuole il numero di fianco
%\memorize 		% <<< DECOMMENTA se si vuole utilizzarne la funzione
%\chordsoff		% <<< DECOMMENTA se vuoi una strofa senza accordi

\[E-]Volto di \[B-]luce,
di vit\[D]toria e \[A]liber\[B-]tà,
\[G]tu hai trac\[D]ciato
i senti\[E-]eri \[A]della \[B]vita;
\[G]spezzi con la \[D]Croce
le bar\[A-]riere della \[B]morte:
\[A-]Figlio di \[E-]Dio, pie\[B-7]tà di \[E]noi.

\endverse



\endsong
%------------------------------------------------------------
%			FINE CANZONE
%------------------------------------------------------------




%*****************************************************
% *  *  *  *  *  *  *  *  *  *  *  *  *  *  *  *  *  *






\end{songs}




%\ifcanzsingole
%	\relax
%\else
%	\iftitleindex
%		\ifxetex
%		\printindex[alfabetico]
%		\else
%		\printindex
%		\fi
%	\else
%	\fi
%	\ifauthorsindex
%	\printindex[autori]
%	\else
%	\fi
%	\iftematicindex
%	\printindex[tematico]
%	\else
%	\fi
%	\ifcover
%		\relax
%	\else
%		\colophon
%	\fi
%\fi
\end{document}
