%-------------------------------------------------------------
%			INIZIO	CANZONE
%-------------------------------------------------------------


%titolo: 	Danza la vita
%autore: 	
%tonalita: 	Re 



%%%%%% TITOLO E IMPOSTAZONI
\beginsong{Danza la vita}[by={Canto Scout}] 	% <<< MODIFICA TITOLO E AUTORE
\transpose{0} 						% <<< TRASPOSIZIONE #TONI (0 nullo)
\momenti{Cresima; Fine; Battesimo; Matrimonio; Ringraziamento; Comunione}							% <<< INSERISCI MOMENTI	
% momenti vanno separati da ; e vanno scelti tra:
% Ingresso; Atto penitenziale; Acclamazione al Vangelo; Dopo il Vangelo; Offertorio; Comunione; Ringraziamento; Fine; Santi; Pasqua; Avvento; Natale; Quaresima; Canti Mariani; Battesimo; Prima Comunione; Cresima; Matrimonio; Meditazione;
\ifchorded
	%\textnote{Tonalità originale }	% <<< EV COMMENTI (tonalità originale/migliore)
\fi





%%%%%% INTRODUZIONE
\ifchorded
\vspace*{\versesep}
\textnote{Intro: \qquad \qquad  }%(\eighthnote 116) % << MODIFICA IL TEMPO
% Metronomo: \eighthnote (ottavo) \quarternote (quarto) \halfnote (due quarti)
\vspace*{-\versesep}
\beginverse*

\nolyrics

%---- Prima riga -----------------------------
\vspace*{-\versesep}
\[D] \[G]  \[D]	\[G] % \[*D] per indicare le pennate, \rep{2} le ripetizioni

%---- Ogni riga successiva -------------------
%\vspace*{-\versesep}
%\[G] \[C]  \[D]	

%---- Ev Indicazioni -------------------------			
\textnote{\textit{(a ripetizione)} }	

\endverse
\fi




%%%%% STROFA
\beginverse
\memorize
\[D]Canta con la \[G]voce e con il \[D]cuore, \[G]
\[D]con la bocca e \[G]con la vita, \[D] \[G]
\[D]canta senza \[G]stonature, \[D] \[G]
la \[D]verità \[*G] del \[D]cuore. \[G]
\[D]Canta come \[G]cantano i viandanti 
\echo{\[D]canta come \[G]cantano i viandanti}
non \[D]solo per riem\[G]pire il tempo, 
\echo{non \[D]solo per \[G]riempire il tempo,}
\[D]Ma per soste\[G]nere lo sforzo 
\echo{\[D]Ma per soste\[G]nere lo sforzo.}
\[D]Canta \[G] e cam\[D]mina \[G]
\[D]Canta \[G] e cam\[D]mina \[G]
Se \[A]poi, credi non possa ba\[B-]stare
segui il \[E]tempo, stai \[G]pron\[A]to e
\endverse





%%%%% RITORNELLO

\beginchorus
\textnote{\textbf{Rit.}}
\[D]Danza la \[G]vita, al \[A]ritmo dello \[D]Spirito. 
\qquad \quad \echo{Spirito che riempi i nostri }
\[B-]Danza, \[G]danza al \[A]ritmo che c'è in \[D]te. 
\echo{cuor, danza assieme a noi. Danza}
\[G]Spirito \[A]che \[D]riempi i nostri 
\echo{la vita al ritmo dello Spirito}
\[B-]cuor. \[G]Danza assieme a \[A]no\[D]i. 
\echo{Danza, danza al ritmo che c'è in te.}
\endchorus



%%%%%% EV. INTERMEZZO
\beginverse*
\vspace*{1.3\versesep}
{
	\nolyrics
	\textnote{Breve intermezzo}
	
	\ifchorded

	%---- Prima riga -----------------------------
	\vspace*{-\versesep}
	\[G] \[D]  \[G]	 




	\fi
	%---- Ev Indicazioni -------------------------			
	%\textnote{\textit{(ripetizione della strofa)}} 
	 
}
\vspace*{\versesep}
\endverse


%%%%% STROFA
\beginverse
Cam^mina sulle ^orme del Si^gnore, ^
non ^solo con i ^piedi ^ma ^ 
^usa soprat^tutto il cuore.^^
^Ama ^ chi è con ^te. ^
Cam^mina con lo ^zaino sulle spalle 
\echo{Cam^mina con lo ^zaino sulle spalle}
^la fatica a^iuta a crescere 
\echo{^la fatica a^iuta a crescere}
^nella con^divisione 
\echo{^nella con^divisione.}
^Canta ^ e cam^mina, ^
^canta  ^ e cam^mina. ^
Se ^poi, credi non possa ba^stare
segui il ^tempo, stai ^pron^to e
\endverse



%%%%%% EV. INTERMEZZO
\beginverse*
\vspace*{1.3\versesep}
{
	\nolyrics
	\musicnote{Chiusura}
	
	\ifchorded

	%---- Prima riga -----------------------------
	\vspace*{-\versesep}
	\[G] \[D]  \[G]	 \[*D]




	\fi
	%---- Ev Indicazioni -------------------------			
	%\textnote{\textit{(ripetizione della strofa)}} 
	 
}
\vspace*{\versesep}
\endverse


\endsong
%------------------------------------------------------------
%			FINE CANZONE
%------------------------------------------------------------

