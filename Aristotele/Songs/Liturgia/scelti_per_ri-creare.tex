%-------------------------------------------------------------
%			INIZIO	CANZONE
%-------------------------------------------------------------


%titolo: 	Scelti per ri-creare
%autore: 	M.A. Chiodaroli
%tonalita: 	Do



%%%%%% TITOLO E IMPOSTAZONI
\beginsong{Scelti per ri-creare}[by={M.A. Chiodaroli}] 	% <<< MODIFICA TITOLO E AUTORE
\transpose{0} 						% <<< TRASPOSIZIONE #TONI (0 nullo)
\momenti{Meditazione; Ringraziamento; Comunione}			% <<< INSERISCI MOMENTI	
% momenti vanno separati da ; e vanno scelti tra:
% Ingresso; Atto penitenziale; Acclamazione al Vangelo; Dopo il Vangelo; Offertorio; Comunione; Ringraziamento; Fine; Santi; Pasqua; Avvento; Natale; Quaresima; Canti Mariani; Battesimo; Prima Comunione; Cresima; Matrimonio; Meditazione; Spezzare del pane;
\ifchorded
	%\textnote{Tonalità migliore}	% <<< EV COMMENTI (tonalità originale/migliore)
\fi


%%%%%% INTRODUZIONE
\ifchorded
\vspace*{\versesep}
\textnote{Intro: \qquad \qquad }% (\eighthnote 120)} % <<  MODIFICA IL TEMPO
% Metronomo: \eighthnote (ottavo) \quarternote (quarto) \halfnote (due quarti)
\vspace*{-\versesep}
\beginverse*

\nolyrics

%---- Prima riga -----------------------------
\vspace*{-\versesep}
\[C] \[A-] \[F]	 \[G] \rep{2} % \[*D] per indicare le pennate, \rep{2} le ripetizioni

%---- Ogni riga successiva -------------------
%\vspace*{-\versesep}
%\[G] \[C]  \[D]	

%---- Ev Indicazioni -------------------------			
%\textnote{\textit{(Oppure tutta la strofa)} }	

\endverse
\fi




%%%%% STROFA
\beginverse		%Oppure \beginverse* se non si vuole il numero di fianco
\memorize 		% <<< DECOMMENTA se si vuole utilizzarne la funzione
%\chordsoff		% <<< DECOMMENTA se vuoi una strofa senza accordi

Ho cer\[C]cato di impa\[G]rare \brk la bel\[F]lezza dell’a\[G]gire
Ho ca\[C]pito che ci v\[G]oglion \brk molti \[F]mesi giorni e \[G]ore
Non ci vu\[C]ole più la \[G]fretta  \brk di chi \[F]corre nel suo \[G]tempo 
devo at\[C]tendere pa\[G]ziente, \brk contin\[F]uando a costr\[G]uire.

\endverse




%%%%% RITORNELLO
\beginchorus
%\textnote{\textbf{Rit.}}

\[C]Tu Sign\[A-]ore hai scelto \[F]me come att\[G]ore,
non per \[C]finge\[A-]re ma per \[F]ri-cre\[G]are
\[C]Tu hai \[A-]scelto noi per \[F]ricostru\[G]ire
con pa\[C]zienza \[A-]la bellezza \[F]che c’è per\[G]chè…
Ci hai scelti \[F]per ri-cre\[G]are con \[C]Te!


\endchorus



%%%%%% EV. INTERMEZZO
\beginverse*
\vspace*{1.3\versesep}
{
	\nolyrics
	\textnote{Intermezzo strumentale}
	
	\ifchorded

	%---- Prima riga -----------------------------
	\vspace*{-\versesep}
	\[C] \[A-] \[F]	 \[G] \rep{2} 


	\fi
	%---- Ev Indicazioni -------------------------			
	%\textnote{\textit{(ripetizione della strofa)}} 
	 
}
\vspace*{\versesep}
\endverse



%%%%% STROFA
\beginverse		%Oppure \beginverse* se non si vuole il numero di fianco
%\memorize 		% <<< DECOMMENTA se si vuole utilizzarne la funzione
%\chordsoff		% <<< DECOMMENTA se vuoi una strofa senza accordi

Ogni ^volta che poi ^provo  \brk a costru^ire con le ^mani
Scopro ^che non c’è crea^zione \brk che non ^porti scritta in ^sé
La pass^ione e poi la ^cura  \brk di chi ^non si è mai stan^cato
E ha pro^vato a model^lare \brk tutto ^ciò che gli hai do^nato



\endverse


%%%%% RITORNELLO
\beginchorus
%\textnote{\textbf{Rit.}}

\[C]Tu Sign\[A-]ore hai scelto \[F]me come att\[G]ore,
non per \[C]finge\[A-]re ma per \[F]ri-cre\[G]are
\[C]Tu hai \[A-]scelto noi per \[F]ricostru\[G]ire
con pa\[C]zienza \[A-]la bellezza \[F]che c’è per\[G]chè…
Ci hai scelti \[F]per ri-cre\[G]are con \[C]Te!


\endchorus


%%%%% STROFA
\beginverse*		%Oppure \beginverse* se non si vuole il numero di fianco
%\memorize 		% <<< DECOMMENTA se si vuole utilizzarne la funzione
%\chordsoff		% <<< DECOMMENTA se vuoi una strofa senza accordi
\vspace*{1.3\versesep}
\textnote{Bridge}
Ho apprez\[F]zato quanto è \[C]bello \brk lavo\[G]rare fianco a f\[A-]ianco 
con chi \[F]sa come il mio \[C]gesto \brk con il \[G]proprio comple\[A-]tare,
\[F]Non è \[C]solo \brk l’amic\[G]izia che ci \[A-]lega,
c’è qual\[F*]cosa di più Grande  \brk per cui \[G*]noi collaboriamo


\endverse


%%%%% RITORNELLO
\beginchorus
%\textnote{\textbf{Rit.}}

\[C]Tu Sign\[A-]ore hai scelto \[F]noi come att\[G]ori,
non per \[C]finge\[A-]re ma per \[F]ri-cre\[G]are
\[C]Tu hai \[A-]scelto noi per \[F]ricostru\[G]ire
con pa\[C]zienza \[A-]la bellezza \[F]che c’è per\[G]chè…
Ci hai scelti \[F]per ri-cre\[G]are con \[C]Te!	

Ci hai scelti \[F]per ri-cre\[G]are con \[C]Te!
Ci hai scelti \[F]per ri-cre\[G]are con \[C]Te! \[C*]
\endchorus



\endsong
%------------------------------------------------------------
%			FINE CANZONE
%------------------------------------------------------------


