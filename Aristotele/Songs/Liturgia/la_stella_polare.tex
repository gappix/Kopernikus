%-------------------------------------------------------------
%			INIZIO	CANZONE
%-------------------------------------------------------------


%titolo: 	La stella polare (Tu al centro del mio cuore)
%autore: 	Gen Verde
%tonalita: 	Sol 



%%%%%% TITOLO E IMPOSTAZONI
\beginsong{La stella polare}[ititle={Tu al centro del mio cuore }, by={ Tu al centro del mio cuore — Gen Verde}] 	% <<< MODIFICA TITOLO E AUTORE
\transpose{0} 						% <<< TRASPOSIZIONE #TONI (0 nullo)
\momenti{Ringraziamento}							% <<< INSERISCI MOMENTI	
% momenti vanno separati da ; e vanno scelti tra:
% Ingresso; Atto penitenziale; Acclamazione al Vangelo; Dopo il Vangelo; Offertorio; Comunione; Ringraziamento; Fine; Santi; Pasqua; Avvento; Natale; Quaresima; Canti Mariani; Battesimo; Prima Comunione; Cresima; Matrimonio; Meditazione; Spezzare del pane;
\ifchorded
	%\textnote{Tonalità originale }	% <<< EV COMMENTI (tonalità originale/migliore)
\fi

%%%%%% INTRODUZIONE
\ifchorded
\vspace*{\versesep}
\musicnote{
\begin{minipage}{0.48\textwidth}
\textbf{Intro}
\hfill 
%( \eighthnote \, 80)   % <<  MODIFICA IL TEMPO
% Metronomo: \eighthnote (ottavo) \quarternote (quarto) \halfnote (due quarti)
\end{minipage}
} 	
\vspace*{-\versesep}
\beginverse*

\nolyrics

%---- Prima riga -----------------------------
\vspace*{-\versesep}
\[E-] \[G] \[B-] \[G]	 % \[*D] per indicare le pennate, \rep{2} le ripetizioni

%---- Ogni riga successiva -------------------
%\vspace*{-\versesep}
%\[G] \[C]  \[D]	

%---- Ev Indicazioni -------------------------			
%\textnote{\textit{(Oppure tutta la strofa)} }	

\endverse
\fi

%%%%% STROFA
\beginverse		%Oppure \beginverse* se non si vuole il numero di fianco
\memorize 		% <<< DECOMMENTA se si vuole utilizzarne la funzione
%\chordsoff		% <<< DECOMMENTA se vuoi una strofa senza accordi

\[E-]Ho bisogno di incontrarti nel mio \[G]cuore
\[B-]di trovare Te di stare insieme a \[C]Te
\[A-]unico riferimento del mio an\[E-]dare 
\[C]unica ragione \[D]Tu, \[B-]unico sostegno \[E-]Tu
\[C]al centro del mio cuore \[D]ci sei solo \[G]Tu 

\endverse

%%%%% STROFA
\beginverse*		%Oppure \beginverse* se non si vuole il numero di fianco
\memorize 		% <<< DECOMMENTA se si vuole utilizzarne la funzione
%\chordsoff		% <<< DECOMMENTA se vuoi una strofa senza accordi

\[E-]Anche il cielo gira intorno e non ha \[G]pace
\[B-]ma c'è un punto fermo è quella stella \[C]là
\[A-]la stella polare è fissa ed è la \[E-]sola 
\[C]la stella polare \[D]Tu , \[B-]la stella sicura \[E-]Tu
\[C]al centro del mio cuore \[D]ci sei solo \[G]Tu

\endverse

%%%%% RITORNELLO
\beginchorus
\textnote{\textbf{Rit.}}

Tutto \[B-]ruota intorno a \[C]Te , in funzione di \[B-]Te-\[E-]e
e poi \[B-]non importa il \[C]come, il dove , il \[D]se.

\endchorus

%%%%% STROFA
\beginverse		%Oppure \beginverse* se non si vuole il numero di fianco
%\memorize 		% <<< DECOMMENTA se si vuole utilizzarne la funzione
%\chordsoff		% <<< DECOMMENTA se vuoi una strofa senza accordi

\[E-]Che Tu splenda sempre al centro del mio \[G]cuore
\[B-]il significato allora sarai \[C]Tu
\[A-]quello che farò sarà soltanto \[E-]amore
\[C]unica ragione \[D]Tu , \[B-]la stella polare \[E-]Tu
\[C]al centro del mio cuore \[D]ci sei solo \[G]Tu.

\endverse

%%%%% RITORNELLO
\beginchorus
\textnote{\textbf{Rit.}}

Tutto \[B-]ruota intorno a \[C]Te , in funzione di \[B-]Te-\[E-]e
e poi \[B-]non importa il \[C]come, il dove , il \[D]se.

\endchorus

%%%%%% EV. FINALE

\beginverse* %oppure \beginchorus
\vspace*{1.3\versesep}
\textnote{\textbf{Finale} \textit{(humming)}} %<<< EV. INDICAZIONI

\[E-]Uhmmm... \[G] \[B-] \[G]

\endverse  %oppure \endchorus




\endsong
%------------------------------------------------------------
%			FINE CANZONE
%------------------------------------------------------------