%-------------------------------------------------------------
%			INIZIO	CANZONE
%-------------------------------------------------------------


%titolo: 	Lodate il Signore
%autore: 	Daniele Ricci
%tonalita: 	Sol 



%%%%%% TITOLO E IMPOSTAZONI
\beginsong{Lodate il Signore}[by={RnS}] 	% <<< MODIFICA TITOLO E AUTORE
\transpose{0} 						% <<< TRASPOSIZIONE #TONI (0 nullo)
%\preferflats  %SE VOGLIO FORZARE i bemolle come alterazioni
%\prefersharps %SE VOGLIO FORZARE i # come alterazioni
\momenti{}							% <<< INSERISCI MOMENTI	
% momenti vanno separati da ; e vanno scelti tra:
% Ingresso; Atto penitenziale; Acclamazione al Vangelo; Dopo il Vangelo; Offertorio; Comunione; Ringraziamento; Fine; Santi; Pasqua; Avvento; Natale; Quaresima; Canti Mariani; Battesimo; Prima Comunione; Cresima; Matrimonio; Meditazione; Spezzare del pane;
\ifchorded
	%\textnote{Tonalità migliore }	% <<< EV COMMENTI (tonalità originale/migliore)
\fi


%%%%%% INTRODUZIONE
\ifchorded
\vspace*{\versesep}
\musicnote{
\begin{minipage}{0.48\textwidth}
\textbf{Intro}
\hfill 
%( \eighthnote \, 80)   % <<  MODIFICA IL TEMPO
% Metronomo: \eighthnote (ottavo) \quarternote (quarto) \halfnote (due quarti)
\end{minipage}
} 	
\vspace*{-\versesep}
\beginverse*

\nolyrics

%---- Prima riga -----------------------------
\vspace*{-\versesep}
\[A] \[E] \[F#-] \[A]


%---- Ogni riga successiva -------------------
\vspace*{-\versesep}
\[D] \[A] \[B-] \[E]

%---- Ogni riga successiva -------------------
\vspace*{-\versesep}
\[A]



%---- Ev Indicazioni -------------------------			
\textnote{\textit{[come il ritornello]} }	

\endverse
\fi




%%%%% STROFA
\beginverse		%Oppure \beginverse* se non si vuole il numero di fianco
\memorize 		% <<< DECOMMENTA se si vuole utilizzarne la funzione
%\chordsoff		% <<< DECOMMENTA se vuoi una strofa senza accordi

Lo\[A]date il Si\[D]gnore nei c\[E]ieli,
lo\[A]date \[D]angeli \[E]suoi.
Lo\[C]date voi \[F]tutte sue schiere 
la \[E]Sua mae\[E]stà.
\endverse
\beginverse*	
Lo^date da ^tutta la ^terra,
lo^date ^popoli s^uoi.
Lo^date ^giovani e vecchi 
la ^Sua bon^tà.
\endverse

%%%%% RITORNELLO
\beginchorus
\textnote{\textbf{Rit.}}

Cantate al Si\[A]gnore, \[E] 
un cantico \[F#-]nuovo. \[A]
Sia onore al \[D]Re, 
sia gloria al \[A]Dio,
che siede sul \[B-]tro\[E7]no.
Risuoni la \[A]lode, \[E]
la benedi\[F#-]zione al Si\[C#-]gnor,
che era e che \[D]è, che sempre sa\[E]rà.
Allelu\[A]ja! \[E] Allelu\[F#-]ja! \[A]
\endchorus

%%%%%% EV. INTERMEZZO
\beginverse*
\vspace*{1.3\versesep}
{
	\nolyrics
	\textnote{\textbf{Rit.}}
    \textnote{Intermezzo strumentale}
	
	\ifchorded

	%---- Prima riga -----------------------------
	\vspace*{-\versesep}
	\[D] \[A] \[B-] \[E] 

    
	%---- Ogni riga successiva -------------------
	\vspace*{-\versesep}
   \[A]

	\fi
	%---- Ev Indicazioni -------------------------			
	%\textnote{\textit{(ripetizione della strofa)}} 
	 
}
\vspace*{\versesep}
\endverse

%%%%% STROFA
\beginverse		%Oppure \beginverse* se non si vuole il numero di fianco
\memorize 		% <<< DECOMMENTA se si vuole utilizzarne la funzione
%\chordsoff		% <<< DECOMMENTA se vuoi una strofa senza accordi

Gio^iscano ^nel crea^tore,
e^sultino i ^figli di ^Sion.
^danzino, ^facciano festa 
al ^loro ^Dio.

\endverse
\beginverse*	

^Lodino il ^nome del ^Padre,
con ^timpani in^neggino a ^Lui.
C^ieli e ^terra si prostrino 
al ^Re dei ^re. 

\endverse



%%%%% RITORNELLO
\beginchorus
\textnote{\textbf{Rit.}}

Cantate al Si\[A]gnore, \[E] 
un cantico \[F#-]nuovo. \[A]
Sia onore al \[D]Re, 
sia gloria al \[A]Dio,
che siede sul \[B-]tro\[E7]no.
Risuoni la \[A]lode, \[E]
la benedi\[F#-]zione al Si\[C#-]gnor,
che era e che \[D]è, che sempre sa\[E]rà.
Allelu\[A]ja! \[E] Allelu\[F#-]ja! \[A]
\endchorus



%%%%%% EV. INTERMEZZO
\beginverse*
\vspace*{1.3\versesep}
{
	\nolyrics
    \textnote{Intermezzo strumentale}
	
	\ifchorded

	%---- Prima riga -----------------------------
	\vspace*{-\versesep}
	\[D] \[A] \[B-] \[E] 
	\fi
	%---- Ev Indicazioni -------------------------			
	%\textnote{\textit{(ripetizione della strofa)}} 
	 
}
\vspace*{\versesep}
\endverse

%%%%% RITORNELLO
\beginchorus
\vspace*{-\versesep}
Risuoni la \[A]lode, \[E]
la benedi\[F#-]zione al Si\[C#-]gnor,
che era e che \[D]è, che sempre sa\[E]rà.
\textnote{\textit{[si alza la tonalità]}}
Allelu\[A]ja! \[F]
\endchorus



\transpose{1}
\preferflats 



%%%%% RITORNELLO
\beginchorus
\textnote{\textbf{Rit.}}

Cantate al Si\[A]gnore, \[E]
un cantico \[F#-]nuovo. \[A]
Sia onore al \[D]Re, 
sia gloria al \[A]Dio,
che siede sul \[B-]tro\[E7]no.
Risuoni la \[A]lode, \[E]
la benedi\[F#-]zione al Si\[C#-]gnor,
che era e che \[D]è, che sempre sa\[E]rà.
\endchorus




%%%%%% EV. CHIUSURA SOLO STRUMENTALE
\ifchorded
\beginchorus %oppure \beginverse*
\vspace*{1.3\versesep}
\textnote{Chiusura } %<<< EV. INDICAZIONI

\[F]A-\[G4]\[G]a-\[A]alleluja! \[A]Alleluja!   \rep{5} \quad \[A*]
\endchorus  %oppure \endverse
\fi


\endsong
%------------------------------------------------------------
%			FINE CANZONE
%------------------------------------------------------------



