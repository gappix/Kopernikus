%-------------------------------------------------------------
%			INIZIO	CANZONE
%-------------------------------------------------------------


%titolo: 	Segni del tuo amore
%autore: 	Gen Verde, Gen rosso
%tonalita: 	Do

\beginsong{Segni del Tuo amore}[by={Gen\ Verde, Gen\ Rosso}]
\transpose{0} 						% <<< TRASPOSIZIONE #TONI (0 nullo)
\momenti{Offertorio; Prima Comunione}							% <<< INSERISCI MOMENTI	
% momenti vanno separati da ; e vanno scelti tra:
% Ingresso; Atto penitenziale; Acclamazione al Vangelo; Dopo il Vangelo; Offertorio; Comunione; Ringraziamento; Fine; Santi; Pasqua; Avvento; Natale; Quaresima; Canti Mariani; Battesimo; Prima Comunione; Cresima; Matrimonio; Meditazione;
\ifchorded
	%\textnote{Tonalità originale }	% <<< EV COMMENTI (tonalità originale/migliore)
\fi



%%%%%% INTRODUZIONE
\ifchorded
\vspace*{\versesep}
\textnote{Intro: \qquad \qquad  (\quarternote 118)}% % << MODIFICA IL TEMPO
% Metronomo: \eighthnote (ottavo) \quarternote (quarto) \halfnote (due quarti)
\vspace*{-\versesep}
\beginverse*

\nolyrics

%---- Prima riga -----------------------------
\vspace*{-\versesep}
\[C] \[C]  \[C] \[D-]	 % \[*D] per indicare le pennate, \rep{2} le ripetizioni

%---- Ogni riga successiva -------------------
\vspace*{-\versesep}
\[C] \[C]  \[*D-] \[*C] \[D-]	

%---- Ev Indicazioni -------------------------			
%\textnote{\textit{(Oppure tutta la strofa)} }	

\endverse
\fi


%%%%% STROFA
\beginverse
\memorize

\[C]Mille e mille grani nelle 
spighe \[D-]d'o\[C]ro \quad \[*D-] \[*C] \[D-]
\[C]mandano fragranza e danno 
gioia al \[D-]cuo\[C]re, \quad \[*D-] \[*C] \[D-]
\[C]quando, macinati, fanno un 
pane \[D-]so\[C]lo, \quad \[*D-] \[*C] \[D-]
\[C]pane quotidiano, dono tuo, 
Si\[D-]gno\[C]re. \quad \[*D-] \[*C] \[C]

\endverse



%%%%% RITORNELLO
\beginchorus
\textnote{\textbf{Rit.}}

\[G]Ecco il pane e il vino, segni del tuo a\[F]mo\[C]re.
\[G]Ecco questa offerta, accoglila Si\[F]gno\[C]re,
\[F]tu di mille e mille \[G]cuori fai un cuore \[C]solo,
un corpo solo in \[G]te
e il \[F]Figlio tuo verrà, vi\[G]vrà 
ancora in mezzo a \[C]noi.

\endchorus




%%%%%% EV. INTERMEZZO
\beginverse*
\vspace*{1.3\versesep}
{
	\nolyrics
	\textnote{Breve intermezzo strumentale}
	
	\ifchorded

	%---- Prima riga -----------------------------
	\vspace*{-\versesep}
	\[(C)] \[C]  \[C] \[D-]	 % \[*D] per indicare le pennate, \rep{2} le ripetizioni

	%---- Ogni riga successiva -------------------
	\vspace*{-\versesep}
	\[C] \[C]  \[*D-] \[*C] \[D-]	


	\fi
	%---- Ev Indicazioni -------------------------			
	%\textnote{\textit{(ripetizione della strofa)}} 
	 
}
\vspace*{\versesep}
\endverse




%%%%% STROFA
\beginverse

^Mille grappoli maturi 
sotto il ^so^le, \quad ^ ^ ^
^festa della terra donano 
vi^go^re, \quad ^ ^ ^
^quando da ogni perla stilla 
il vino ^nuo^vo, \quad ^ ^ ^
^vino della gioia, dono tuo, 
Si^gno^re. \quad ^ ^ ^

\endverse


%%%%% RITORNELLO
\beginchorus
\textnote{\textbf{Rit.}}

\[G]Ecco il pane e il vino, segni del tuo a\[F]mo\[C]re.
\[G]Ecco questa offerta, accoglila Si\[F]gno\[C]re,
\[F]tu di mille e mille \[G]cuori fai un cuore \[C]solo,
un corpo solo in \[G]te
e il \[F]Figlio tuo verrà, vi\[G]vrà 
ancora in mezzo a \[F]noi.		\quad \[C]  \iflyric \rep{2} \fi
\ifchorded
\vspace*{2\versesep}
\[G]Ecco il pane e il vino, segni del tuo a\[F]mo\[C]re.
\[G]Ecco questa offerta, accoglila Si\[F]gno\[C]re,
\[F]tu di mille e mille \[G]cuori fai un cuore \[C]solo,
un corpo solo in \[G]te
e il \[F]Figlio tuo verrà, vi\[G]vrà 
ancora in mezzo a \[C]noi.
\fi
\endchorus



%%%%%% EV. INTERMEZZO
\beginverse*
\vspace*{1.3\versesep}
{
	\nolyrics
	\musicnote{Finale}
	
	\ifchorded

	%---- Prima riga -----------------------------
	\vspace*{-\versesep}
	\[(C)] \[C]  \[C] \[D-]	 % \[*D] per indicare le pennate, \rep{2} le ripetizioni

	%---- Ogni riga successiva -------------------
	\vspace*{-\versesep}
	\[C] \[C]  \[*D-] \[*C] \[D-]	\quad \[C]


	\fi
	%---- Ev Indicazioni -------------------------			
	%\textnote{\textit{(ripetizione della strofa)}} 
	 
}
\vspace*{\versesep}
\endverse



\endsong
%------------------------------------------------------------
%			FINE CANZONE
%------------------------------------------------------------

