%-------------------------------------------------------------
%			INIZIO	CANZONE
%-------------------------------------------------------------


%titolo: 	Beati quelli 
%autore: 	Sequeri
%tonalita: 	Fa >Re



%%%%%% TITOLO E IMPOSTAZONI
\beginsong{Beati quelli}[by={P. Sequeri}]% <<< MODIFICA TITOLO E AUTORE
\transpose{-3} 						% <<< TRASPOSIZIONE #TONI (0 nullo)
\momenti{Ingresso}							% <<< INSERISCI MOMENTI	
% momenti vanno separati da ; e vanno scelti tra:
% Ingresso; Atto penitenziale; Acclamazione al Vangelo; Dopo il Vangelo; Offertorio; Comunione; Ringraziamento; Fine; Santi; Pasqua; Avvento; Natale; Quaresima; Canti Mariani; Battesimo; Prima Comunione; Cresima; Matrimonio; Meditazione; Spezzare del pane;
\ifchorded
	%\textnote{Tonalità migliore }	% <<< EV COMMENTI (tonalità originale/migliore)
\fi



%%%%%% INTRODUZIONE
\ifchorded
\vspace*{\versesep}
\musicnote{
\begin{minipage}{0.48\textwidth}
\textbf{Intro}
\hfill 
(\eighthnote \, 116)
%( \eighthnote \, 80)   % <<  MODIFICA IL TEMPO
% Metronomo: \eighthnote (ottavo) \quarternote (quarto) \halfnote (due quarti)
\end{minipage}
} 	
\vspace*{-\versesep}
\beginverse*


\nolyrics

%---- Prima riga -----------------------------
\vspace*{-\versesep}
\[F]\[B&]\[C7]	 % \[*D] per indicare le pennate, \rep{2} le ripetizioni

%---- Ogni riga successiva -------------------
%\vspace*{-\versesep}
%\[G] \[C]  \[D]	

%---- Ev Indicazioni -------------------------			
%\textnote{\textit{(Oppure tutta la strofa)} }	

\endverse
\fi

\beginverse
\memorize
Beati \[F]quelli che \[B&]poveri \[F]sono, \[A7]
beati \[D-]quanti son \[C7]puri di \[F]cuore. \[C7]
Beati \[F]quelli che \[B&]vivono in \[F]pena \[A7]
nell'at\[D-]tesa d'un \[G-]nuovo mat\[C7]tino.
\endverse



%%%%% RITORNELLO
\beginchorus
\textnote{\textbf{Rit.}}
Saran be\[F]ati, vi \[B&]dico, be\[F*]\[A7*]a\[D-]ti
perché di \[B&]essi è il \[B&-]regno dei \[C7]cieli.
Saran be\[F]ati, vi \[B&]dico, be\[F*]\[A7*]a\[D-]ti
perché di \[B&]essi è il \[G7]regno dei \[F*]\[C9*]cie\[F]li.
\endchorus

\beginverse
Beati ^quelli che ^fanno la ^pace, ^
beati ^quelli che in^segnano l'a^more. ^
Beati ^quelli che ^hanno la ^fame ^
e la ^sete di ^vera giu^stizia.
\endverse


\beginverse
Beati ^quelli che un ^giorno sa^ranno ^
persegui^tati per ^causa di ^Cristo, ^
perché nel ^cuore non ^hanno vio^lenza, ^
ma la ^forza di ^questo Van^gelo.
\endverse

\endsong
%------------------------------------------------------------
%			FINE CANZONE
%------------------------------------------------------------




