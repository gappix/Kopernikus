%-------------------------------------------------------------
%			INIZIO	CANZONE
%-------------------------------------------------------------


%titolo: 	Proteggimi o Dio
%autore: 	Gallotta
%tonalita: 	Re 



%%%%%% TITOLO E IMPOSTAZONI
\beginsong{Proteggimi o Dio}[by={A. Gallotta}] 	% <<< MODIFICA TITOLO E AUTORE
\transpose{0} 						% <<< TRASPOSIZIONE #TONI (0 nullo)
\momenti{Ingresso}							% <<< INSERISCI MOMENTI	
% momenti vanno separati da ; e vanno scelti tra:
% Ingresso; Atto penitenziale; Acclamazione al Vangelo; Dopo il Vangelo; Offertorio; Comunione; Ringraziamento; Fine; Santi; Pasqua; Avvento; Natale; Quaresima; Canti Mariani; Battesimo; Prima Comunione; Cresima; Matrimonio; Meditazione; Spezzare del pane;
\ifchorded
	%\textnote{Tonalità originale }	% <<< EV COMMENTI (tonalità originale/migliore)
\fi



%%%%%% INTRODUZIONE
\ifchorded
\vspace*{\versesep}
\textnote{Intro: \qquad \qquad  }%(\eighthnote 116) % <<  MODIFICA IL TEMPO
% Metronomo: \eighthnote (ottavo) \quarternote (quarto) \halfnote (due quarti)
\vspace*{-\versesep}
\beginverse*

\nolyrics

%---- Prima riga -----------------------------
\vspace*{-\versesep}
\[D]  \[D]	 % \[*D] per indicare le pennate, \rep{2} le ripetizioni

%---- Ogni riga successiva -------------------
%\vspace*{-\versesep}
%\[G] \[C]  \[D]	

%---- Ev Indicazioni -------------------------			
%\textnote{\textit{(Oppure tutta la strofa)} }	

\endverse
\fi



%%%%% STROFA
\beginverse		%Oppure \beginverse* se non si vuole il numero di fianco
\memorize 		% <<< DECOMMENTA se si vuole utilizzarne la funzione
%\chordsoff		% <<< DECOMMENTA se vuoi una strofa senza accordi
\[D]Proteggimi, o \[A]Dio: in \[*G]te io \[*E]mi \[A]rifugio.
\[B-]Ho detto a \[D]lui: “Sei \[*G]tu il mi\[*E-]o Signo\[A]re,
\[D]senza di \[A]te \[G]non ho bene al\[A]cuno”.
\[B-]Nelle tue \[D]mani, Si\[*G]gnore, è \[*E]la mia \[A]vita!
\endverse




%%%%% RITORNELLO
\beginchorus
\textnote{\textbf{Rit.}}
\[F]Tu mi \[B&]indicherai il sen\[C]tiero della \[F]vita
\[B&]gioia piena nella \[F]tua presenza,
\[G-]dolcezza senza \[C]fine.
\[F]Tu mi indiche\[B&]rai il sen\[C]tiero della \[F]vita
\[B&]gioia \[C]piena nella \[D-]tua presenza,
\[G-]dolcezza senza \[A]fine.
\endchorus


%%%%% STROFA
\beginverse		%Oppure \beginverse* se non si vuole il numero di fianco
%\memorize 		% <<< DECOMMENTA se si vuole utilizzarne la funzione
%\chordsoff		% <<< DECOMMENTA se vuoi una strofa senza accordi

^Benedico ^Dio che ^m'ha da^to con^siglio;
^anche di ^notte il mio ^cuore m'^istru^isce.
^Innanzi a ^me ^sempre il Si^gnore,
^sta alla mia ^destra, non ^posso ^vacil^lare.

\endverse


%%%%% RITORNELLO
\beginchorus
\textnote{\textbf{Rit.}}
\[F]Tu mi \[B&]indicherai il sen\[C]tiero della \[F]vita
\[B&]gioia piena nella \[F]tua presenza,
\[G-]dolcezza senza \[C]fine.
\[F]Tu mi indiche\[B&]rai il sen\[C]tiero della \[F]vita
\[B&]gioia \[C]piena nella \[D-]tua presenza,
\[G-]dolcezza senza \[A]fine.
\endchorus


%%%%% STROFA
\beginverse		%Oppure \beginverse* se non si vuole il numero di fianco
%\memorize 		% <<< DECOMMENTA se si vuole utilizzarne la funzione
%\chordsoff		% <<< DECOMMENTA se vuoi una strofa senza accordi

^Mia eredi^tà, mio ^calice^ è il Si^gnore,
^per me la ^sorte è su ^luoghi ^deli^ziosi.
^Lieto e se^reno ^è il cuore ^mio,
^luce e spe^ranza ai miei ^passi ^tu da^rai.

\endverse

%%%%% RITORNELLO
\beginchorus
\textnote{\textbf{Rit.}}
\[F]Tu mi \[B&]indicherai il sen\[C]tiero della \[F]vita
\[B&]gioia piena nella \[F]tua presenza,
\[G-]dolcezza senza \[C]fine.
\[F]Tu mi indiche\[B&]rai il sen\[C]tiero della \[F]vita
\[B&]gioia \[C]piena nella \[D-]tua presenza,
\[G-]dolcezza senza \[A]fi-\[A]i-\[D]ne.
\endchorus

\endsong







