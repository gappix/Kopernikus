%-------------------------------------------------------------
%			INIZIO	CANZONE
%-------------------------------------------------------------


%titolo: 	Camminiamo incontro al Signore
%autore: 	Galliano, Semprini
%tonalita: 	Re



%%%%%% TITOLO E IMPOSTAZONI
\beginsong{Camminiamo incontro al Signore}[by={A. M. Galliano, D. Semprini}] 	% <<< MODIFICA TITOLO E AUTORE
\transpose{0} 						% <<< TRASPOSIZIONE #TONI (0 nullo)
\momenti{}							% <<< INSERISCI MOMENTI	
% momenti vanno separati da ; e vanno scelti tra:
% Ingresso; Atto penitenziale; Acclamazione al Vangelo; Dopo il Vangelo; Offertorio; Comunione; Ringraziamento; Fine; Santi; Pasqua; Avvento; Natale; Quaresima; Canti Mariani; Battesimo; Prima Comunione; Cresima; Matrimonio; Meditazione; Spezzare del pane;
\ifchorded
	%\textnote{Tonalità migliore }	% <<< EV COMMENTI (tonalità originale/migliore)
\fi


%%%%%% INTRODUZIONE
\ifchorded
\vspace*{\versesep}
\musicnote{
\begin{minipage}{0.48\textwidth}
\textbf{Intro}
\hfill 
%( \eighthnote \, 80)   % <<  MODIFICA IL TEMPO
% Metronomo: \eighthnote (ottavo) \quarternote (quarto) \halfnote (due quarti)
\end{minipage}
} 	
\vspace*{-\versesep}
\beginverse*

\nolyrics

%---- Prima riga -----------------------------
\vspace*{-\versesep}
\[D] \[A] \[B-]\[F#-]  % \[*D] per indicare le pennate, \rep{2} le ripetizioni

%---- Ogni riga successiva -------------------
\vspace*{-\versesep}
 \[G] \[D] \[A] \[A]


%---- Ev Indicazioni -------------------------			
%\textnote{\textit{(Oppure tutta la strofa)} }	

\endverse
\fi






%%%%% RITORNELLO
\beginchorus
\textnote{\textbf{Rit.}}
\[D]Cammini\[A]amo in\[B-]contro al Si\[F#-]gnore,
\[G]cammin\[D]iamo con \[A]gio\[B-]ia:
\[G]egli \[D]viene non \[A]tarde\[B-]rà,
\[G]egli \[D]viene, ci \[A]salve\[B-]ra-\[A7]à.
\endchorus



%%%%% STROFA
\beginverse		%Oppure \beginverse* se non si vuole il numero di fianco
\memorize 		% <<< DECOMMENTA se si vuole utilizzarne la funzione
%\chordsoff		% <<< DECOMMENTA se vuoi una strofa senza accordi
\[D]Egli \[A]viene il \[B-]giorno è vi\[F#-]cino
\[G]E la \[D]notte va \[A]verso l’au\[B-]rora
\[G]Elev\[D]iamo a \[A]Lui-l’anima-\[B-]nostra
\[G]Non sa\[D]remo de\[A]lu\[B-]si
\[G]Non sa\[D]remo de\[A]lu\[B-]si –\[A7]i
\endverse




%%%%% STROFA
\beginverse		%Oppure \beginverse* se non si vuole il numero di fianco
%\memorize 		% <<< DECOMMENTA se si vuole utilizzarne la funzione
%\chordsoff		% <<< DECOMMENTA se vuoi una strofa senza accordi
^Egli v^iene vegl^iamo in at^tesa
^Ricor^dando la ^sua pa^rola
^Rives^tiamo la ^forza di ^Dio
^Per re^sistere al ^ma^le
^Per re^sistere al ^ma^le-^e
\endverse




%%%%% STROFA
\beginverse		%Oppure \beginverse* se non si vuole il numero di fianco
%\memorize 		% <<< DECOMMENTA se si vuole utilizzarne la funzione
%\chordsoff		% <<< DECOMMENTA se vuoi una strofa senza accordi
^Egli v^iene and^iamogli in^contro
^Ritor^nando sui ^retti sen^tieri
^Mostre^rà la ^sua miseri^cordia
^Ci da^rà la sua ^gra^zia
^Ci da^rà la sua ^gra^zia-^a.
\endverse




%%%%% STROFA
\beginverse		%Oppure \beginverse* se non si vuole il numero di fianco
%\memorize 		% <<< DECOMMENTA se si vuole utilizzarne la funzione
%\chordsoff		% <<< DECOMMENTA se vuoi una strofa senza accordi
^Egli v^iene è il ^Dio fe^dele
^Che ci ^chiama alla ^sua comu^nione
^Il Si^gnore sa^rà-il nostro ^bene
^Noi la ^terra fe^con^da
^Noi la ^terra fe^con^da-^a.
\endverse





\endsong
%------------------------------------------------------------
%			FINE CANZONE
%------------------------------------------------------------


