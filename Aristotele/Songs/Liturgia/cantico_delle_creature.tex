%-------------------------------------------------------------
%			INIZIO	CANZONE
%-------------------------------------------------------------


%titolo: 	Cantico delle creature
%autore: 	Varnavà, Mancinoni
%tonalita: 	Mi-



%%%%%% TITOLO E IMPOSTAZONI
\beginsong{Cantico delle creature}[by={Per la terra e le tue creature — S. Varnavà, R. Mancinoni}] 	% <<< MODIFICA TITOLO E AUTORE
\transpose{0} 						% <<< TRASPOSIZIONE #TONI (0 nullo)
%\preferflats  %SE VOGLIO FORZARE i bemolle come alterazioni
%\prefersharps %SE VOGLIO FORZARE i # come alterazioni
\momenti{Salmi; Santi}							% <<< INSERISCI MOMENTI	
% momenti vanno separati da ; e vanno scelti tra:
% Ingresso; Atto penitenziale; Acclamazione al Vangelo; Dopo il Vangelo; Offertorio; Comunione; Ringraziamento; Fine; Santi; Pasqua; Avvento; Natale; Quaresima; Canti Mariani; Battesimo; Prima Comunione; Cresima; Matrimonio; Meditazione; Spezzare del pane;
\ifchorded
	%\textnote{Tonalità migliore }	% <<< EV COMMENTI (tonalità originale/migliore)
\fi




%%%%%% INTRODUZIONE
\ifchorded
\vspace*{\versesep}
\musicnote{
\begin{minipage}{0.48\textwidth}
\textbf{Intro}
\hfill 
%( \eighthnote \, 80)   % <<  MODIFICA IL TEMPO
% Metronomo: \eighthnote (ottavo) \quarternote (quarto) \halfnote (due quarti)
\end{minipage}
} 	
\vspace*{-\versesep}
\beginverse*


\nolyrics

%---- Prima riga -----------------------------
\vspace*{-\versesep}
\[B-] \[E-] \[B-]	 % \[*D] per indicare le pennate, \rep{2} le ripetizioni

%---- Ogni riga successiva -------------------
%\vspace*{-\versesep}
%\[G] \[C]  \[D]	

%---- Ev Indicazioni -------------------------			
%\textnote{\textit{(Oppure tutta la strofa)} }	

\endverse
\fi






%%%%% STROFA
\beginverse		%Oppure \beginverse* se non si vuole il numero di fianco
\memorize 		% <<< DECOMMENTA se si vuole utilizzarne la funzione
%\chordsoff		% <<< DECOMMENTA se vuoi una strofa senza accordi
\[B-]Laudato \[E-]sii mi Si\[B-]gnore,
\[A]per frate \[B-]sole, \[F#]sora \[B-]luna,
\[B-]frate vento, il \[E-]cielo, \[B-]le stelle,
\[A]per sora \[B-]acqua, \[F#]frate \[B-]focu.
\endverse



%%%%% RITORNELLO
\beginchorus
\textnote{\textbf{Rit.}}
\[G]Lau\[A]dato
\[D*]sii \[F#*]mi Si\[B-]gnore,
\[E-]per la \[B-]terra e \[F#]le tue crea\[B-]ture
\[G]Lau\[A]dato
\[D*]sii \[F#*]mi Si\[B-]gnore,
\[E-]per la \[B-]terra e \[F#]le tue crea\[B-]ture
\endchorus




%%%%% STROFA
\beginverse		%Oppure \beginverse* se non si vuole il numero di fianco
%\memorize 		% <<< DECOMMENTA se si vuole utilizzarne la funzione
%\chordsoff		% <<< DECOMMENTA se vuoi una strofa senza accordi
\[B-]Laudato \[E-]sii, mi Si\[B-]gnore,
\[A]quello che \[B-]porta \[F#]la tua \[B-]pace,
\[B-]e saprà \[E-]perdo\[B-]nare,
\[A]per il tuo a\[B-]more \[F#]saprà a\[B-]mare.
\endverse




%%%%% RITORNELLO
\beginchorus
\textnote{\textbf{Rit.}}
\[G]Lau\[A]dato
\[D*]sii \[F#*]mi Si\[B-]gnore,
\[E-]per la \[B-]terra e \[F#]le tue crea\[B-]ture
\[G]Lau\[A]dato
\[D*]sii \[F#*]mi Si\[B-]gnore,
\[E-]per la \[B-]terra e \[F#]le tue crea\[B-]ture
\endchorus




%%%%% STROFA
\beginverse		%Oppure \beginverse* se non si vuole il numero di fianco
%\memorize 		% <<< DECOMMENTA se si vuole utilizzarne la funzione
%\chordsoff		% <<< DECOMMENTA se vuoi una strofa senza accordi
\[B-]Laudato \[E-]sii, mi Si\[B-]gnore,
\[A]per sora \[B-]morte \[F#]corpo\[B-]rale,
\[B-]dalla quale \[E-]homo vi\[B-]vente
\[A]non potrà mai, \[F#]mai scap\[A]pare.
\endverse




%%%%% RITORNELLO
\beginchorus
\textnote{\textbf{Rit.}}
\[G]Lau\[A]dato
\[D*]sii \[F#*]mi Si\[B-]gnore,
\[E-]per la \[B-]terra e \[F#]le tue crea\[B-]ture
\[G]Lau\[A]dato
\[D*]sii \[F#*]mi Si\[B-]gnore,
\[E-]per la \[B-]terra e \[F#]le tue crea\[B-]ture
\endchorus





%%%%% STROFA
\beginverse		%Oppure \beginverse* se non si vuole il numero di fianco
%\memorize 		% <<< DECOMMENTA se si vuole utilizzarne la funzione
%\chordsoff		% <<< DECOMMENTA se vuoi una strofa senza accordi
\[B-]Laudate \[E-]e bene\[A]dite,
\[A]ringrazi\[B-]ate \[F#]e ser\[B-]vite,
\[B-]il Signore \[E-]con humil\[B-]tate,
\[A]ringra\[B-]ziate \[F#]e ser\[B-]vite.
\endverse




%%%%% RITORNELLO
\beginchorus
\textnote{\textbf{Rit.}}
\[G]Lau\[A]dato
\[D*]sii \[F#*]mi Si\[B-]gnore,
\[E-]per la \[B-]terra e \[F#]le tue crea\[B-]ture
\[G]Lau\[A]dato
\[D*]sii \[F#*]mi Si\[B-]gnore,
\[E-]per la \[B-]terra e \[F#]le tue crea\[B-]ture
\endchorus




\endsong
%------------------------------------------------------------
%			FINE CANZONE
%------------------------------------------------------------


