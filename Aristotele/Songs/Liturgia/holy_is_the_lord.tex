%-------------------------------------------------------------
%			INIZIO	CANZONE
%-------------------------------------------------------------


%titolo: 	Holy is the Lord
%autore: 	Chris Tomlin
%tonalita: 	Sol 



%%%%%% TITOLO E IMPOSTAZONI
\beginsong{Holy is the Lord}[by={C. Tomlin, L. Giglio}] 	% <<< MODIFICA TITOLO E AUTORE
\transpose{-3} 						% <<< TRASPOSIZIONE #TONI (0 nullo)
\momenti{Natale}							% <<< INSERISCI MOMENTI
\ifchorded
	\textnote{Tonalità migliore per le bambine }
\fi


%%%%%% INTRODUZIONE
\ifchorded
\vspace*{\versesep}
\textnote{
\begin{minipage}{0.48\textwidth}
Intro:
\hfill 
%(\quarternote 80)   % <<  MODIFICA IL TEMPO
% Metronomo: \eighthnote (ottavo) \quarternote (quarto) \halfnote (due quarti)
\end{minipage}
} 	
\vspace*{-\versesep}
\beginverse*

\nolyrics

%---- Prima riga -----------------------------
\vspace*{-\versesep}
\[G] \[C]  \[D]	 \rep{2}

%---- Ogni riga successiva -------------------
%\vspace*{-\versesep}
%\[G] \[C]  \[D]	

%---- Ev Indicazioni -------------------------			
%\textnote{\textit{(Oppure tutta la strofa)} }	

\endverse
\fi



%%%%% STROFA
\beginverse		%Oppure \beginverse* se non si vuole il numero di fianco
%\memorize 		% <<< DECOMMENTA se si vuole utilizzarne la funzione
%\chordsoff		& <<< DECOMMENTA se vuoi una strofa senza accordi

We \[G]stand and \[C]lift up our \[D]hands,
for the \[E-]joy of the \[C]Lord is our str\[D]enght.
\[G]We bow do\[C]wn and wor\[D]ship Him now,
how \[E-]great how \[C]awesome is \[D]He.

\endverse


%%%%% RITORNELLO
\textnote{\textbf{Rit.}}
\beginchorus

Holy is the \[G]Lord
\[C]God al\[D]mighty.
The \[E-]Earth is \[C]filled
with His \[D]glory. \rep{2} 

\endchorus



%%%%%% EV. INTERMEZZO
\beginverse*
\vspace*{1.3\versesep}
{
	\nolyrics
	\textnote{Intermezzo strumentale}
	
	\ifchorded

	%---- Prima riga -----------------------------
	\vspace*{-\versesep}
	\[G] \[C]  \[D]	 \rep{2}




	\fi
	%---- Ev Indicazioni -------------------------			
	\musicnote{e si riprende dalla strofa} 
	 
}
\vspace*{\versesep}
\endverse



%%%%%% FINALE

\beginchorus
\vspace*{1.3\versesep}
\textnote{Finale:}
The \[E-]Earth is \[C]filled
with His \[D]glory.
\textnote{\textit{rallentando}}
The \[E-]Earth is \[C]filled
with His \[D]glory.
The \[E-]Earth is \[C]filled
with His \[D]glo-o-ory.
\endchorus



\endsong
%------------------------------------------------------------
%			FINE CANZONE
%------------------------------------------------------------


%++++++++++++++++++++++++++++++++++++++++++++++++++++++++++++
%			CANZONE TRASPOSTA
%++++++++++++++++++++++++++++++++++++++++++++++++++++++++++++
\ifchorded
%decremento contatore per avere stesso numero
\addtocounter{songnum}{-1} 
\beginsong{Holy is the Lord}[by={C. Tomlin, L. Giglio}] 	% <<< MODIFICA TITOLO E AUTORE
\transpose{0} 						% <<< TRASPOSIZIONE #TONI + - (0 nullo)
\ifchorded
	\textnote{Tonalità originale}	% <<< EV COMMENTI (tonalità originale/migliore)
\fi



%%%%%% INTRODUZIONE
\ifchorded
\vspace*{\versesep}
\textnote{
\begin{minipage}{0.48\textwidth}
Intro:
\hfill 
%(\quarternote 80)   % <<  MODIFICA IL TEMPO
% Metronomo: \eighthnote (ottavo) \quarternote (quarto) \halfnote (due quarti)
\end{minipage}
} 	
\vspace*{-\versesep}
\beginverse*

\nolyrics

%---- Prima riga -----------------------------
\vspace*{-\versesep}
\[G] \[C]  \[D]	 \rep{2}

%---- Ogni riga successiva -------------------
%\vspace*{-\versesep}
%\[G] \[C]  \[D]	

%---- Ev Indicazioni -------------------------			
%\textnote{\textit{(Oppure tutta la strofa)} }	

\endverse
\fi



%%%%% STROFA
\beginverse		%Oppure \beginverse* se non si vuole il numero di fianco
%\memorize 		% <<< DECOMMENTA se si vuole utilizzarne la funzione
%\chordsoff		& <<< DECOMMENTA se vuoi una strofa senza accordi

We \[G]stand and \[C]lift up our \[D]hands,
for the \[E-]joy of the \[C]Lord is our str\[D]enght.
\[G]We bow do\[C]wn and wor\[D]ship Him now,
how \[E-]great how \[C]awesome is \[D]He.

\endverse


%%%%% RITORNELLO
\textnote{\textbf{Rit.}}
\beginchorus

Holy is the \[G]Lord
\[C]God al\[D]mighty.
The \[E-]Earth is \[C]filled
with His \[D]glory. \rep{2} 

\endchorus



%%%%%% EV. INTERMEZZO
\beginverse*
\vspace*{1.3\versesep}
{
	\nolyrics
	\textnote{Intermezzo strumentale}
	
	\ifchorded

	%---- Prima riga -----------------------------
	\vspace*{-\versesep}
	\[G] \[C]  \[D]	 \rep{2}




	\fi
	%---- Ev Indicazioni -------------------------			
	\musicnote{e si riprende dalla strofa} 
	 
}
\vspace*{\versesep}
\endverse



%%%%%% FINALE

\beginchorus
\vspace*{1.3\versesep}
\textnote{Finale:}
The \[E-]Earth is \[C]filled
with His \[D]glory.
\textnote{\textit{rallentando}}
The \[E-]Earth is \[C]filled
with His \[D]glory.
The \[E-]Earth is \[C]filled
with His \[D]glo-o-ory.
\endchorus

\endsong
\fi
%++++++++++++++++++++++++++++++++++++++++++++++++++++++++++++
%			FINE CANZONE TRASPOSTA
%++++++++++++++++++++++++++++++++++++++++++++++++++++++++++++


