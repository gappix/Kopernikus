%-------------------------------------------------------------
%			INIZIO	CANZONE
%-------------------------------------------------------------


%titolo: 	Luce del mondo
%autore: 	F. Pesarese, RnS
%tonalita: 	Fa 



%%%%%% TITOLO E IMPOSTAZONI
\beginsong{Luce del mondo}[by={F. Pesarese, RnS}] 	% <<< MODIFICA TITOLO E AUTORE
\transpose{0} 						% <<< TRASPOSIZIONE #TONI (0 nullo)
%\preferflats  %SE VOGLIO FORZARE i bemolle come alterazioni
%\prefersharps %SE VOGLIO FORZARE i # come alterazioni
\momenti{}							% <<< INSERISCI MOMENTI	
% momenti vanno separati da ; e vanno scelti tra:
% Ingresso; Atto penitenziale; Acclamazione al Vangelo; Dopo il Vangelo; Offertorio; Comunione; Ringraziamento; Fine; Santi; Pasqua; Avvento; Natale; Quaresima; Canti Mariani; Battesimo; Prima Comunione; Cresima; Matrimonio; Meditazione; Spezzare del pane;
\ifchorded
	%\textnote{Tonalità migliore }	% <<< EV COMMENTI (tonalità originale/migliore)
\fi


%%%%%% INTRODUZIONE
\ifchorded
\vspace*{\versesep}
\textnote{Intro: \qquad \qquad  }%(\eighthnote 116) % <<  MODIFICA IL TEMPO
% Metronomo: \eighthnote (ottavo) \quarternote (quarto) \halfnote (due quarti)
\vspace*{-\versesep}
\beginverse*

\nolyrics

%---- Prima riga -----------------------------
\vspace*{-\versesep}
\[F] \[B&]  \[F] \[B&]  \[C*]	 % \[*D] per indicare le pennate, \rep{2} le ripetizioni

%---- Ogni riga successiva -------------------
%\vspace*{-\versesep}
%\[G] \[C]  \[D]	

%---- Ev Indicazioni -------------------------			
%\textnote{\textit{(Oppure tutta la strofa)} }	

\endverse
\fi

\newchords{verse}
\newchords{chorus}


%%%%% STROFA
\beginverse		%Oppure \beginverse* se non si vuole il numero di fianco
\memorize[verse]% <<< DECOMMENTA se si vuole utilizzarne la funzione
%\chordsoff		% <<< DECOMMENTA se vuoi una strofa senza accordi

\[F]Luce del mondo sei, Signor,
\[F] il tuo Amore ci illumina
e le \[D-]tenebre che avvolgono il nostro cuor
\[B&]con la tua Luce sva\[C*]nisco\[F]no.
\[F]Luce del mondo sei, Signor,
\[F] il tuo Amore ci illumina
e le \[D-]tenebre che avvolgono il nostro cuor
\[B&]con la tua Luce sva\[C*]nisco\[D-]no,
\[B&]con la tua Luce sva\[C*]nisco\[F]no.
\endverse




%%%%% RITORNELLO
\beginchorus
\textnote{\textbf{Rit.}}
\memorize[chorus]
\[F]Luce del mondo sei,
luce che illumi\[A-]na,
\[D-]luce del mondo sei.
\[B&]Il tuo Amore, Si\[C]gnor,
\[A]mai si spegne\[D-]rà,
perché \[B&*]luce del \[C*]mondo sei,
\[B&*]luce che il\[G-*]lumi\[C]na. \[C]

\endchorus





%%%%% STROFA
\beginverse		%Oppure \beginverse* se non si vuole il numero di fianco
\replay[verse]		% <<< DECOMMENTA se si vuole utilizzarne la funzione
%\chordsoff		% <<< DECOMMENTA se vuoi una strofa senza accordi

^Gioia del mondo sei, Signor,
^ il tuo Amore ci fa cantar,
le tris^tezze che sempre ci opprimono,
^con la tua gioia sva^nisco^no.
^Gioia del mondo sei, Signor,
^ il tuo Amore ci fa cantar,
le tris^tezze che sempre ci opprimono,
^con la tua gioia sva^nisco^no,
^con la tua gioia sva^nisco^no.

\endverse




%%%%% RITORNELLO
\beginchorus
\textnote{\textbf{Rit.}}
\replay[chorus]

^Gioia del mondo sei,
gioia che fa can^tar,
^gioia del mondo sei!
^Il tuo amore, Si^gnor,
^mai si spegne^rà,
perché ^gioia del ^mondo sei,
\vspace*{\versesep}
\textnote{Si alza di tonalità}
^gioia che ^fa can^ta-\[D]re.
\endchorus


\transpose{2}


%%%%% STROFA
\beginverse		%Oppure \beginverse* se non si vuole il numero di fianco
\replay[verse]		% <<< DECOMMENTA se si vuole utilizzarne la funzione
%\chordsoff		% <<< DECOMMENTA se vuoi una strofa senza accordi


^Forza del mondo sei, Signor,
^ il tuo amore ci libera,
le ca^tene che ci legano,
^con la tua forza sva^nisco^no.
^Forza del mondo sei, Signor,
^ il tuo amore ci libera,
le ca^tene che ci legano,
^con la tua forza sva^nisco^no,
^con la tua forza sva^nisco^no.

\endverse




%%%%% RITORNELLO
\beginchorus
\textnote{\textbf{Rit.}}
\replay[chorus]


^Forza del mondo sei,
forza che ^libera,
^forza del mondo sei!
^Il tuo amore, Si^gnor,
^mai si spegne^rà,
perché ^forza del ^mondo sei,
^forza che ^libe^ra. ^
\endchorus

%%%%%% EV. INTERMEZZO
\beginverse*
\vspace*{1.3\versesep}
{
	\nolyrics
	\textnote{Intermezzo strumentale}
	
	\ifchorded

	%---- Prima riga -----------------------------
	\vspace*{-\versesep}
	\[F] \[F] 
    
	%---- Ogni riga successiva -------------------
	\vspace*{-\versesep}
	\[F] \[A-] 

	%---- Ogni riga successiva -------------------
	\vspace*{-\versesep}
	\[D-] \[D-]
 

    %---- Ogni riga successiva -------------------
	\vspace*{-\versesep}
	\[B&] \[C*] \[D-] \[B&] \[C*] \[F] 

	\fi
	%---- Ev Indicazioni -------------------------			
	%\textnote{\textit{(ripetizione della strofa)}} 
	 
}
\vspace*{\versesep}
\endverse



%%%%% STROFA
\beginverse		%Oppure \beginverse* se non si vuole il numero di fianco
%\memorize[verse]% <<< DECOMMENTA se si vuole utilizzarne la funzione
%\chordsoff		% <<< DECOMMENTA se vuoi una strofa senza accordi

\[F]Luce del mondo sei, Signor,
\[F] il tuo Amore ci illumina
e le \[D-]tenebre che avvolgono il nostro cuor
\[B&]con la tua Luce sva\[C*]nisco\[F]no.
\[F]Luce del mondo sei, Signor,
\[F] il tuo Amore ci illumina
e le \[D-]tenebre che avvolgono il nostro cuor
\[B&]con la tua Luce sva\[C*]nisco\[D-]no,
\[B&]con la tua Luce sva\[C*]nisco\[F]no.
\endverse




%%%%% RITORNELLO
\beginchorus
\textnote{\textbf{Rit.}}
%\memorize[chorus]

\[F]Luce del mondo sei,
gioia che fa can\[A-]tar,
\[D-]forza che libera.
\[B&]Il tuo Amore, Si\[C]gnor,
\[A]mai si spegne\[D-]rà,
perché \[B&*]luce del \[C*]mondo sei,
\[B&*]luce che il\[G-*]lumi\[C]na. \[C]

\endchorus

%%%%%% EV. FINALE

\beginchorus %oppure \beginverse*
\vspace*{1.3\versesep}
\textnote{Finale \textit{(rallentando)}} %<<< EV. INDICAZIONI

\[C/A-]Luce che illumi\[F]na. \[F*]

\endchorus  %oppure \endverse




\endsong
%------------------------------------------------------------
%			FINE CANZONE
%------------------------------------------------------------


