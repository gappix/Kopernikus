%-------------------------------------------------------------
%			INIZIO	CANZONE
%-------------------------------------------------------------


%titolo: 	Madonna nera
%autore: 	Bagniewski
%tonalita: 	Sol



%%%%%% TITOLO E IMPOSTAZONI
\beginsong{Madonna nera}[by={Madonna di Czestochowa - Bagniewski}] 	% <<< MODIFICA TITOLO E AUTORE
\transpose{0} 						% <<< TRASPOSIZIONE #TONI (0 nullo)
\momenti{Canti Mariani}							% <<< INSERISCI MOMENTI	
% momenti vanno separati da ; e vanno scelti tra:
% Ingresso; Atto penitenziale; Acclamazione al Vangelo; Dopo il Vangelo; Offertorio; Comunione; Ringraziamento; Fine; Santi; Pasqua; Avvento; Natale; Quaresima; Canti Mariani; Battesimo; Prima Comunione; Cresima; Matrimonio; Meditazione; Spezzare del pane;
\ifchorded
	%\textnote{Tonalità migliore }	% <<< EV COMMENTI (tonalità originale/migliore)
\fi


%%%%%% INTRODUZIONE
\ifchorded
\vspace*{\versesep}
\textnote{Intro: \qquad \qquad  }%(\eighthnote 116) % <<  MODIFICA IL TEMPO
% Metronomo: \eighthnote (ottavo) \quarternote (quarto) \halfnote (due quarti)
\vspace*{-\versesep}
\beginverse*

\nolyrics

%---- Prima riga -----------------------------
\vspace*{-\versesep}
\[G]  % \[*D] per indicare le pennate, \rep{2} le ripetizioni


%---- Ev Indicazioni -------------------------			
%\textnote{\textit{(Oppure tutta la strofa)} }	

\endverse
\fi



%%%%% STROFA
\beginverse		%Oppure \beginverse* se non si vuole il numero di fianco
\memorize 		% <<< DECOMMENTA se si vuole utilizzarne la funzione
C'è una \[G]terra silen\[G7]ziosa dove o\[C]gnuno vuol tor\[E]nare
una \[A-]terra un dolce \[A7]volto
con due \[D]segni di vio\[D7]lenza
Sguardo in\[G]tenso e premu\[G7]roso
che ti \[C]chiede di affi\[A-]dare
la tua \[D]vita e il tuo \[D7]mondo in \[C*]mano a \[G]Lei. 
\endverse


%%%%% RITORNELLO
\beginchorus
\textnote{\textbf{Rit.}}
Ma\[G]donna, Madonna \[C]Nera, 
è \[D]dolce esser tuo \[G]figlio! \[D7]
Oh \[G]lascia, Madonna \[C]Nera, 
ch'io \[D]viva vicino a \[G]Te.
\endchorus



%%%%% STROFA
\beginverse		%Oppure \beginverse* se non si vuole il numero di fianco
%\memorize 		% <<< DECOMMENTA se si vuole utilizzarne la funzione
Lei ti ^calma e rasse^rena, Lei ti ^libera dal ^male,
perché ^sempre ha un cuore ^grande \brk per cia^scuno dei suoi ^figli.
Lei t'il^lumina il cam^mino \brk se le ^offri un po' d'a^more,
se ogni ^giorno parle^rai a ^Lei co^sì.
\endverse



%%%%% STROFA
\beginverse		%Oppure \beginverse* se non si vuole il numero di fianco
%\memorize 		% <<< DECOMMENTA se si vuole utilizzarne la funzione
Questo ^mondo in sub^buglio \brk cosa all'^uomo potrà of^frire?
Solo il ^volto di una ^Madre \brk pace ^vera può do^nare.
Nel suo ^sguardo noi cer^chiamo \brk quel sor^riso del Si^gnore
che ri^desta un po' di ^bene in ^fondo al ^cuor.
\endverse



\endsong
%------------------------------------------------------------
%			FINE CANZONE
%------------------------------------------------------------