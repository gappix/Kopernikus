%-------------------------------------------------------------
%			INIZIO	CANZONE
%-------------------------------------------------------------


%titolo: 	Luce di verità
%autore: 	Becchimanzi, Scordari, Giordano
%tonalita: 	Mi



%%%%%% TITOLO E IMPOSTAZONI
\beginsong{Luce di verità}[by={Becchimanzi, Scordari, Giordano}] 	% <<< MODIFICA TITOLO E AUTORE
\transpose{0} 						% <<< TRASPOSIZIONE #TONI (0 nullo)
\momenti{Ingresso; Comunione; Dopo il Vangelo; Cresima}							% <<< INSERISCI MOMENTI	
% momenti vanno separati da ; e vanno scelti tra:
% Ingresso; Atto penitenziale; Acclamazione al Vangelo; Dopo il Vangelo; Offertorio; Comunione; Ringraziamento; Fine; Santi; Pasqua; Avvento; Natale; Quaresima; Canti Mariani; Battesimo; Prima Comunione; Cresima; Matrimonio; Meditazione; Spezzare del pane;
\ifchorded
	%\textnote{Tonalità originale }	% <<< EV COMMENTI (tonalità originale/migliore)
\fi


%%%%%% INTRODUZIONE
\ifchorded
\vspace*{\versesep}
\textnote{Intro: \qquad \qquad  }%(\eighthnote 116) % <<  MODIFICA IL TEMPO
% Metronomo: \eighthnote (ottavo) \quarternote (quarto) \halfnote (due quarti)
\vspace*{-\versesep}
\beginverse*

\nolyrics

%---- Prima riga -----------------------------
\vspace*{-\versesep}
\[E] \[A] 	\[E] \[G#-]  % \[*D] per indicare le pennate, \rep{2} le ripetizioni

%---- Ogni riga successiva -------------------
\vspace*{-\versesep}
\[A] \[E] \[(C#-7*)] \[F#7] \[A] \[B]

%---- Ev Indicazioni -------------------------			
%\textnote{\textit{(Oppure tutta la strofa)} }	

\endverse
\fi


%%%%% RITORNELLO
\beginchorus
\textnote{\textbf{Rit.}}
\[E]Luce di veri\[A]tà, \[E]fiamma di cari\[G#-]tà,
\[A]vincolo di uni\[E]tà, \[(C#-7*)]Spirito \[F#7]Santo A\[A]mo\[B]re.
\[E]Dona la liber\[A]tà, \[E]dona la santi\[G#-]tà,
\[A]fa' dell'umani\[E]tà \[(C#-7*)]il tuo \[D]canto di \[A]lo\[B]de.
\endchorus

\musicnote{\textit{Dolce, leggero}}

%%%%% STROFA
\beginverse		%Oppure \beginverse* se non si vuole il numero di fianco
\memorize 		% <<< DECOMMENTA se si vuole utilizzarne la funzione
%\chordsoff		% <<< DECOMMENTA se vuoi una strofa senza accordi
\[C#-] Ci poni come \[B]luce sopra un \[E]mon\[A]te:
\[F#-] in noi l'umani\[E]tà vedrà il tuo \[B4]vol\[B]to. 
\[A] Ti testimonie\[B]remo fra le \[E]gen\[A]ti:
\[F#-] in noi l'umani\[E]tà vedrà il tuo \[B4]volto,  
\[B]\textbf{Spirito, vieni!}
\endverse

%%%%% STROFA
\beginverse		%Oppure \beginverse* se non si vuole il numero di fianco
%\memorize 		% <<< DECOMMENTA se si vuole utilizzarne la funzione
%\chordsoff		% <<< DECOMMENTA se vuoi una strofa senza accordi
^ Cammini accanto a ^noi lungo la ^stra^da,
^ si realizzi in ^noi la tua missi^one.^
^ Attingeremo ^forza dal tuo ^cuor^e,
^ si realizzi in ^noi la tua missi^one, 
^\textbf{Spirito, vieni!}
\endverse

%%%%% STROFA
\beginverse		%Oppure \beginverse* se non si vuole il numero di fianco
%\memorize 		% <<< DECOMMENTA se si vuole utilizzarne la funzione
%\chordsoff		% <<< DECOMMENTA se vuoi una strofa senza accordi
^ Come sigillo ^posto sul tuo ^cuo^re,
^ ci custodisci, ^Dio, nel tuo a^more.^
^ Hai dato la tua ^vita per sal^var^ci,
^ ci custodisci, ^Dio, nel tuo a^more, 
^\textbf{Spirito, vieni!}
\endverse

%%%%% STROFA
\beginverse		%Oppure \beginverse* se non si vuole il numero di fianco
%\memorize 		% <<< DECOMMENTA se si vuole utilizzarne la funzione
%\chordsoff		% <<< DECOMMENTA se vuoi una strofa senza accordi
^ Dissiperai le ^tenebre del ^ma^le,
^ esulterà in ^te la crea^zione.^
^ Vivremo al tuo cos^petto in e^te^rno,
^ esulterà in ^te la crea^zione, 
^\textbf{Spirito, vieni!}
\endverse

%%%%% STROFA
\beginverse		%Oppure \beginverse* se non si vuole il numero di fianco
%\memorize 		% <<< DECOMMENTA se si vuole utilizzarne la funzione
%\chordsoff		% <<< DECOMMENTA se vuoi una strofa senza accordi
^ Vergine del si^lenzio e della ^fe^de
^ l'Eterno ha posto in ^te la sua di^mora.^
^ Il tuo "sì" ri^suonerà per ^sem^pre:
^ l'Eterno ha posto in ^te la sua di^mora, 
^\textbf{Spirito, vieni!}
\endverse

%%%%% STROFA
\beginverse		%Oppure \beginverse* se non si vuole il numero di fianco
%\memorize 		% <<< DECOMMENTA se si vuole utilizzarne la funzione
%\chordsoff		% <<< DECOMMENTA se vuoi una strofa senza accordi
^ Tu nella Santa ^Casa accogli il ^do^no,
^ sei tu la porta ^che ci apre il C^ielo^
^ Con te la Chiesa ^canta la sua ^lo^de,
^ sei tu la porta ^che ci apre il Ci^elo, 
^\textbf{Spirito, vieni!}
\endverse

%%%%% STROFA
\beginverse		%Oppure \beginverse* se non si vuole il numero di fianco
%\memorize 		% <<< DECOMMENTA se si vuole utilizzarne la funzione
%\chordsoff		% <<< DECOMMENTA se vuoi una strofa senza accordi
^ Tu nella brezza ^parli al nostro ^cuo^re:
^ ascolteremo, ^Dio, la tua pa^rola;^
^ ci chiami a condi^videre il tuo a^mo^re:
^ ascolteremo, ^Dio, la tua pa^rola, 
^\textbf{Spirito, vieni!}
\endverse

\ifchorded
%%%%% RITORNELLO
\beginchorus
\textnote{\textbf{Finale }}
\[E]Luce di veri\[A]tà, \[E]fiamma di cari\[G#-]tà,
\[A]vincolo di uni\[E]tà, \[(C#-7*)]Spirito \[F#7]Santo A\[A]mo\[B]re.
\[E]Dona la liber\[A]tà, \[E]dona la santi\[G#-]tà,
\[A]fa' dell'umani\[E]tà \[(C#-7*)]il tuo \[D]canto di \[A]lo-\[B]o\[E]de. \[E*]
\endchorus
\fi

\endsong
%------------------------------------------------------------
%			FINE CANZONE
%------------------------------------------------------------




