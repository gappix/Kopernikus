%-------------------------------------------------------------
%			INIZIO	CANZONE
%-------------------------------------------------------------


%titolo: 	Ti chiedo perdono
%autore: 	P. Sequeri
%tonalita: 	Sol 



%%%%%% TITOLO E IMPOSTAZONI
\beginsong{Ti chiedo perdono}[by={P. Sequeri}] 	% <<< MODIFICA TITOLO E AUTORE
\transpose{0} 						% <<< TRASPOSIZIONE #TONI (0 nullo)
\momenti{Atto penitenziale}							% <<< INSERISCI MOMENTI	
% momenti vanno separati da ; e vanno scelti tra:
% Ingresso; Atto penitenziale; Acclamazione al Vangelo; Dopo il Vangelo; Offertorio; Comunione; Ringraziamento; Fine; Santi; Pasqua; Avvento; Natale; Quaresima; Canti Mariani; Battesimo; Prima Comunione; Cresima; Matrimonio; Meditazione; Spezzare del pane;
\ifchorded
	%\textnote{Tonalità originale }	% <<< EV COMMENTI (tonalità originale/migliore)
\fi


%%%%%% INTRODUZIONE
\ifchorded
\vspace*{\versesep}
\textnote{Intro: \qquad \qquad  }%(\eighthnote 116) % <<  MODIFICA IL TEMPO
% Metronomo: \eighthnote (ottavo) \quarternote (quarto) \halfnote (due quarti)
\vspace*{-\versesep}
\beginverse*

\nolyrics

%---- Prima riga -----------------------------
\vspace*{-\versesep}
\[B-] \[C#dim/(A)]	 \[B-] \rep{2} % \[*D] per indicare le pennate, \rep{2} le ripetizioni

%---- Ogni riga successiva -------------------
%\vspace*{-\versesep}
%\[G] \[C]  \[D]	

%---- Ev Indicazioni -------------------------			
%\textnote{\textit{(Oppure tutta la strofa)} }	

\endverse
\fi




%%%%% STROFA
\beginverse*		%Oppure \beginverse* se non si vuole il numero di fianco
%\memorize 		% <<< DECOMMENTA se si vuole utilizzarne la funzione
%\chordsoff		% <<< DECOMMENTA se vuoi una strofa senza accordi


Ti \[B-]chiedo per\[C#dim/(A)]dono, Padre \[B-]buono, \[D]
per \[G]ogni man\[F#]canza d'a\[B-]more, \[G*] \[F#*]
\[B-]per la mia \[C#dim/(A)]debole spe\[B-]ranza \[D]
e \[G]per la mia \[F#]fragile \[B-]fede. \[G*] \[A*]

\endverse


%%%%% STROFA
\beginverse*		%Oppure \beginverse* se non si vuole il numero di fianco
%\memorize 		% <<< DECOMMENTA se si vuole utilizzarne la funzione
%\chordsoff		% <<< DECOMMENTA se vuoi una strofa senza accordi

Do\[D]mando a \[G]Te, Si\[D]gnore, \[A]
che il\[G]lumi\[A]ni i miei \[D]passi, \[F#-]
la \[G]forza di \[G-]vivere,
con \[D]tutti i miei fra\[G]telli, \[G-]
nuova\[D]mente fe\[A9]dele al Tuo van\[D*]gelo.\[G*] \[D]

\endverse

\endsong
%------------------------------------------------------------
%			FINE CANZONE
%------------------------------------------------------------

