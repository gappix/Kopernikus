%-------------------------------------------------------------
%			INIZIO	CANZONE
%-------------------------------------------------------------


%titolo: 	Come Maria
%autore: 	Gen\ Rosso
%tonalita: 	La-



%%%%%% TITOLO E IMPOSTAZONI
\beginsong{Come Maria}[by={Gen\ Rosso}]	% <<< MODIFICA TITOLO E AUTORE
\transpose{0} 						% <<< TRASPOSIZIONE #TONI (0 nullo)
\momenti{Canti Mariani; Offertorio}							% <<< INSERISCI MOMENTI	
% momenti vanno separati da ; e vanno scelti tra:
% Ingresso; Atto penitenziale; Acclamazione al Vangelo; Dopo il Vangelo; Offertorio; Comunione; Ringraziamento; Fine; Santi; Pasqua; Avvento; Natale; Quaresima; Canti Mariani; Battesimo; Prima Comunione; Cresima; Matrimonio; Meditazione; Spezzare del pane;
\ifchorded
	%\textnote{Tonalità migliore }	% <<< EV COMMENTI (tonalità originale/migliore)
\fi


%%%%%% INTRODUZIONE
\ifchorded
\vspace*{\versesep}
\textnote{Intro: \qquad \qquad  }%(\eighthnote 116) % <<  MODIFICA IL TEMPO
% Metronomo: \eighthnote (ottavo) \quarternote (quarto) \halfnote (due quarti)
\vspace*{-\versesep}
\beginverse*

\nolyrics

%---- Prima riga -----------------------------
\vspace*{-\versesep}
\[A-] \[C] \[E-] \[A-]	 % \[*D] per indicare le pennate, \rep{2} le ripetizioni

%---- Ogni riga successiva -------------------
%\vspace*{-\versesep}
%\[G] \[C]  \[D]	

%---- Ev Indicazioni -------------------------			
%\textnote{\textit{(Oppure tutta la strofa)} }	

\endverse
\fi



%%%%% STROFA
\beginverse		%Oppure \beginverse* se non si vuole il numero di fianco
\memorize 		% <<< DECOMMENTA se si vuole utilizzarne la funzione
%\chordsoff		% <<< DECOMMENTA se vuoi una strofa senza accordi
\[A-] Vogliamo vivere Si\[C]gnore, 
\[A-] offrendo a te la nostra \[E-]vita, 
\[F] Con questo pane e questo \[D]vino 
\[A-] accetta quello che noi \[E]siamo. 
\[A-] Vogliamo vivere Si\[C]gnore, 
\[A-] abbandonati alla tua \[E-]voce, 
\[F] staccati dalle cose \[D]vane, 
\[A-] fissati nella vita \[E4]vera. \[E]
\endverse


%%%%% RITORNELLO
\beginchorus
\textnote{\textbf{Rit.}}
Vo\[A]gliamo \[D]vivere \[E] come Ma\[F#-]ria 
\[D] l'irraggiun\[E]gibile, \[D] la madre a\[E]mata, 
\[D] che vince il \[E]mondo con l'a\[C#-]more 
\[D ]e offrire s\[B-]empre \brk la Tua \[C#]vita che \[D]viene dal \[A]cielo.
\endchorus




%%%%% STROFA
\beginverse		%Oppure \beginverse* se non si vuole il numero di fianco
%\memorize 		% <<< DECOMMENTA se si vuole utilizzarne la funzione
%\chordsoff		% <<< DECOMMENTA se vuoi una strofa senza accordi
^ Accetta dalle nostre ^mani 
^ come un'offerta a te gra^dita 
^ i desideri di ogni ^cuore, 
^ le ansie della nostra ^vita. 
^ Vogliamo vivere Si^gnore, 
^ accesi dalle Tue Pa^role, 
^ per riportare in ogni ^uomo 
^ la fiamma viva del Tuo a^more. ^
\endverse




%%%%%% EV. FINALE
\beginchorus %oppure \beginverse*
\vspace*{1.3\versesep}
\textnote{Finale} %<<< EV. INDICAZIONI
\[D] e offrire \[B-]sempre la Tua \[C#]vita che \[D]viene dal \[A]cielo.
\endchorus




\endsong
%------------------------------------------------------------
%			FINE CANZONE
%------------------------------------------------------------

