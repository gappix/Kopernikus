%-------------------------------------------------------------
%			INIZIO	CANZONE
%-------------------------------------------------------------


%titolo: 	Amare questa vita
%autore: 	Meregalli
%tonalita: 	Re



%%%%%% TITOLO E IMPOSTAZONI
\beginsong{Amare questa vita}[ititle={Erano uomini senza paura}, by={G. Meregalli}] 	% <<< MODIFICA TITOLO E AUTORE
\transpose{-3} 						% <<< TRASPOSIZIONE #TONI (0 nullo)
\momenti{Comunione; Cresima}							% <<< INSERISCI MOMENTI	
% momenti vanno separati da ; e vanno scelti tra:
% Ingresso; Atto penitenziale; Acclamazione al Vangelo; Dopo il Vangelo; Offertorio; Comunione; Ringraziamento; Fine; Santi; Pasqua; Avvento; Natale; Quaresima; Canti Mariani; Battesimo; Prima Comunione; Cresima; Matrimonio; Meditazione;
\ifchorded
	%\textnote{Tonalità originale }	% <<< EV COMMENTI (tonalità originale/migliore)
\fi

%%%%%% INTRODUZIONE
\ifchorded
\vspace*{\versesep}
\musicnote{
\begin{minipage}{0.48\textwidth}
\textbf{Intro}
\hfill 
%( \eighthnote \, 80)   % <<  MODIFICA IL TEMPO
% Metronomo: \eighthnote (ottavo) \quarternote (quarto) \halfnote (due quarti)
\end{minipage}
} 	
\vspace*{-\versesep}
\beginverse*

\nolyrics

%---- Prima riga -----------------------------
\vspace*{-\versesep}
\[F] \[B&] \[C]	 % \[*D] per indicare le pennate, \rep{2} le ripetizioni

%---- Ogni riga successiva -------------------
%\vspace*{-\versesep}
%\[G] \[C]  \[D]	

%---- Ev Indicazioni -------------------------			
%\textnote{\textit{(Oppure tutta la strofa)} }	

\endverse
\fi

%%%%% STROFA
\beginverse		%Oppure \beginverse* se non si vuole il numero di fianco
\memorize 		% <<< DECOMMENTA se si vuole utilizzarne la funzione
%\chordsoff		& <<< DECOMMENTA se vuoi una strofa senza accordi

\[F]Erano \[B&]uomini \[C]senza pa\[7]ura,
di \[D-]solcare il \[7]mare pen\[A-]sando alla \[7]riva,
\[B&]barche sotto il \[C]cielo, \[F] tra montagne e si\[C]lenzio,
\[B&]davano le \[C]reti al \[F]ma\[D-]re, \brk \[B&]vita dalle \[G-]mani di \[C4]Dio \[C]

\endverse

%%%%% STROFA
\beginverse		%Oppure \beginverse* se non si vuole il numero di fianco
%\memorize 		% <<< DECOMMENTA se si vuole utilizzarne la funzione
%\chordsoff		& <<< DECOMMENTA se vuoi una strofa senza accordi

^Venne nell'^ora più ^lenta del ^giorno,
^quando le ^reti si ^sdraiano a ^riva,
^l'aria senza ^vento, ^ si riempì di una ^voce,
^mani cari^che di ^sa^le, ^sale nelle ^mani di ^Dio. ^

\endverse

%%%%% RITORNELLO
\beginchorus
\textnote{\textbf{Rit.}}

Lo se\[F]guimmo fi\[G-7]dandoci degli \[F]occhi \[7]
gli cre\[B&]demmo a\[A7]mando le pa\[D-]role. \[D7]
Fu il \[G-]sole caldo a \[C7]riva
o fu il \[F]vento sulla \[C]vela
o il \[B&]gusto e la fa\[F]tica di ri\[G-7]schiare
e accettare quella \[C]sfida.
\endchorus

\beginverse
^Prima che un ^sole più ^alto vi ^insidi, 
^prima che il ^giorno vi ^lasci de^lusi,
^riprendete il ^largo ^ e gettate le ^reti, 
^barche cari^che di ^pe^sci, \brk ^vita dalle ^mani di ^Dio. ^
\endverse
\beginchorus
Lo se\[F]guimmo fi\[G-7]dandoci degli \[F]occhi \[7]
gli cre\[B&]demmo a\[A7]mando le pa\[D-]role. \[D7]
Lui \[G-]voce lui no\[C7]tizia
lui \[F]strada e lui sua \[C]meta
lui \[B&]gioia impreve\[F]dibile e sin\[G-7]cera
di amare questa \[C]vita.
\endchorus
\beginverse
^Erano ^uomini ^senza pa^ura
di s^olcare il ^mare pen^sando alla ^riva,
^anche quella ^sera, ^ senza dire pa^role,
^misero le ^barche in ^ma^re, ^vita dalle ^mani di ^Dio, ^
\[B&]misero le \[C]barche in \[F]ma\[D-]re, \brk \[B&]vita dalle \[C]mani di \[F]Dio.
\endverse
\endsong

