%-------------------------------------------------------------
%			INIZIO	CANZONE
%-------------------------------------------------------------

%titolo: A volte le parole
%autore: Colombo, Racz
%tonalita: Sol 


%%%%%% TITOLO E IMPOSTAZONI
\beginsong{A volte le parole}[ititle={La tua parola è  vita per noi}, by={Colombo, Racz}] 	% <<< MODIFICA TITOLO E AUTORE
\transpose{0} 						% <<< TRASPOSIZIONE #TONI (0 nullo)
\momenti{ Dopo il Vangelo}							% <<< INSERISCI MOMENTI	
% momenti vanno separati da ; e vanno scelti tra:
% Ingresso; Atto penitenziale; Acclamazione al Vangelo; Dopo il Vangelo; Offertorio; Comunione; Ringraziamento; Fine; Santi; Pasqua; Avvento; Natale; Quaresima; Canti Mariani; Battesimo; Prima Comunione; Cresima; Matrimonio; Meditazione;


%%%%%% INTRODUZIONE
\ifchorded
\vspace*{\versesep}
\textnote{Intro: \qquad \qquad  }%(\eighthnote 116) % << MODIFICA IL TEMPO(\eighthnote 88) 
% Metronomo: \eighthnote (ottavo) \quarternote (quarto) \halfnote (due quarti)
\vspace*{-\versesep}
\beginverse*

\nolyrics

%---- Prima riga -----------------------------
\vspace*{-\versesep}
\[G] \[D] \[G] 	 % \[*D] per indicare le pennate, \rep{2} le ripetizioni

%---- Ogni riga successiva -------------------
%\vspace*{-\versesep}
%\[G] \[C]  \[D]	

%---- Ev Indicazioni -------------------------			
%\textnote{\textit{(Oppure tutta la strofa)} }	

\endverse
\fi




%%%%% STROFA
\beginverse*
\[G]La tua pa\[D]rola è \[E-]vita per \[C]noi.
\[D] Luce del cam\[B-]mi\[E-]no,
\[A-]forza che il cuore non \[D]ha.
\[G]La tua pa\[D]rola la \[E-]pace ci \[C]da.
\[D] Vince ogni \[B-]ma\[E-]le,
\[A-] crea uni\[D]tà,
\[C] crea \[D]uni\[G]tà. 
\endverse



\endsong
%------------------------------------------------------------
%			FINE CANZONE
%------------------------------------------------------------