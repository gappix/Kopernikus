%-------------------------------------------------------------
%			INIZIO	CANZONE
%-------------------------------------------------------------


%titolo: 	Ora è tempoo di gioia
%autore: 	Gen Rosso
%tonalita: 	Re



%%%%%% TITOLO E IMPOSTAZONI
\beginsong{Ora è tempo di gioia}[by={Gen\ Rosso}]
\transpose{0} 						% <<< TRASPOSIZIONE #TONI (0 nullo)
%\preferflats  %SE VOGLIO FORZARE i bemolle come alterazioni
%\prefersharps %SE VOGLIO FORZARE i # come alterazioni
\momenti{}							% <<< INSERISCI MOMENTI	
% momenti vanno separati da ; e vanno scelti tra:
% Ingresso; Atto penitenziale; Acclamazione al Vangelo; Dopo il Vangelo; Offertorio; Comunione; Ringraziamento; Fine; Santi; Pasqua; Avvento; Natale; Quaresima; Canti Mariani; Battesimo; Prima Comunione; Cresima; Matrimonio; Meditazione; Spezzare del pane;
\ifchorded
	%\textnote{Tonalità migliore }	% <<< EV COMMENTI (tonalità originale/migliore)
\fi



%%%%%% INTRODUZIONE
\ifchorded
\vspace*{\versesep}
\musicnote{
\begin{minipage}{0.48\textwidth}
\textbf{Intro}
\hfill 
%( \eighthnote \, 80)   % <<  MODIFICA IL TEMPO
% Metronomo: \eighthnote (ottavo) \quarternote (quarto) \halfnote (due quarti)
\end{minipage}
} 	
\vspace*{-\versesep}
\beginverse*

\nolyrics

%---- Prima riga -----------------------------
\vspace*{-\versesep}
\[D] \[E-] \[D]	 % \[*D] per indicare le pennate, \rep{2} le ripetizioni

%---- Ogni riga successiva -------------------
%\vspace*{-\versesep}
%\[G] \[C]  \[D]	

%---- Ev Indicazioni -------------------------			
%\textnote{\textit{(Oppure tutta la strofa)} }	

\endverse
\fi




%%%%% STROFA
\beginverse		%Oppure \beginverse* se non si vuole il numero di fianco
\memorize 		% <<< DECOMMENTA se si vuole utilizzarne la funzione
%\chordsoff		% <<< DECOMMENTA se vuoi una strofa senza accordi
L'\[D]eco \[E-]torna d'an\[D]tiche \[G*]val\[A]li
\[D]la sua \[E-]voce \[D7+]non porta \[C7+]più,
\[B-]ricordo \[F#-]di som\[G*]messe \[E7*]lacri\[A]me
\[D]di e\[E-7]sili in terre lon\[A4*]ta\[D]ne.
\endverse



%%%%% RITORNELLO
\beginchorus
\textnote{\textbf{Rit.}}
\[G]Ora è \[D]tempo di \[C]gio\[D]ia,  
\[G*]non \[A-7*]ve \[G*]ne ac\[C*]cor\[D4]ge\[D]te
\[G]ecco \[D]faccio una \[A-]cosa \[E-]nuova
\[B7]nel de\[C7+]serto una \[B-]strada apri\[E-]rò.
\endchorus





%%%%% STROFA
\beginverse		%Oppure \beginverse* se non si vuole il numero di fianco
%\memorize 		% <<< DECOMMENTA se si vuole utilizzarne la funzione
%\chordsoff		% <<< DECOMMENTA se vuoi una strofa senza accordi
^Come l'^onda che ^sulla ^sab^bia
^copre le ^orme e ^poi passa e ^va,
^così nel ^tempo ^si can^cella^no
^le ombre ^scure del lungo in^ver^no.
\endverse


%%%%% STROFA
\beginverse		%Oppure \beginverse* se non si vuole il numero di fianco
%\memorize 		% <<< DECOMMENTA se si vuole utilizzarne la funzione
\chordsoff		% <<< DECOMMENTA se vuoi una strofa senza accordi
^Tra i sen^tieri dei ^boschi il ^ven^to
^con i ^rami ^ricompor^rà
^nuove armo^nie ^che tra^sforma^no
^i la^menti in canti di ^fe^sta.
\endverse

\endsong
%------------------------------------------------------------
%			FINE CANZONE
%------------------------------------------------------------
