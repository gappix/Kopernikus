%-------------------------------------------------------------
%			INIZIO	CANZONE
%-------------------------------------------------------------


%titolo: 	E camminava con loro
%autore: 	???
%tonalita: 	Fa 



%%%%%% TITOLO E IMPOSTAZONI
\beginsong{E camminava con loro}[by={}] 	% <<< MODIFICA TITOLO E AUTORE
\transpose{0} 						% <<< TRASPOSIZIONE #TONI (0 nullo)
%\preferflats  %SE VOGLIO FORZARE i bemolle come alterazioni
%\prefersharps %SE VOGLIO FORZARE i # come alterazioni
\momenti{}							% <<< INSERISCI MOMENTI	
% momenti vanno separati da ; e vanno scelti tra:
% Ingresso; Atto penitenziale; Acclamazione al Vangelo; Dopo il Vangelo; Offertorio; Comunione; Ringraziamento; Fine; Santi; Pasqua; Avvento; Natale; Quaresima; Canti Mariani; Battesimo; Prima Comunione; Cresima; Matrimonio; Meditazione; Spezzare del pane;
\ifchorded
	%\textnote{Tonalità migliore }	% <<< EV COMMENTI (tonalità originale\migliore)
\fi


%%%%%% INTRODUZIONE
\ifchorded
\vspace*{\versesep}
\musicnote{
\begin{minipage}{0.48\textwidth}
\textbf{Intro}
\hfill 
%( \eighthnote \, 80)   % <<  MODIFICA IL TEMPO
% Metronomo: \eighthnote (ottavo) \quarternote (quarto) \halfnote (due quarti)
\end{minipage}
} 	
\vspace*{-\versesep}
\beginverse*


\nolyrics

%---- Prima riga -----------------------------
\vspace*{-\versesep}
\[F]\[C]\[D-] % \[*D] per indicare le pennate, \rep{2} le ripetizioni

%---- Ogni riga successiva -------------------
\vspace*{-\versesep}
\[G7]\[C7]		

%---- Ev Indicazioni -------------------------			
%\textnote{\textit{(Oppure tutta la strofa)} }	

\endverse
\fi




%%%%% STROFA
\beginverse		%Oppure \beginverse* se non si vuole il numero di fianco
\memorize 		% <<< DECOMMENTA se si vuole utilizzarne la funzione
%\chordsoff		% <<< DECOMMENTA se vuoi una strofa senza accordi

\[F]Ecco il tempo nuovo in c\[C]ui
il Padre mio \[D-]darà
la Grazia ad ogni \[G7]uo\[C7]mo.
\[F]Già lo Spirito che è in \[C]me
annuncia libe\[D-]rtà,
ai poveri la \[B&-]gio\[C7]ia.
\endverse
\beginverse*	
\[D-]Croce fu e si \[C]spense in noi
la Pa\[B&]rola che dava \[C]luce agli occhi.
\[D-]"Stolti e tardi nel \[C]credere";
la sua \[SI&]voce scaldò il \[G7]cuo\[C]re.
\endverse


%%%%% RITORNELLO
\beginchorus
\textnote{\textbf{Rit.}}

\[F]Come un volto \[C]amico, \[D-]verità \[A-]inattesa:
\[B&]è Parola \[F]eterna, \[B&]Pane \[C]vi\[C7]vo.
\[F]Corre nuovo i\[C]l passo, \[D-]carico d'\[A-]annuncio:
\[B&]è risorto, \[F]vive \[B&]e cam-\[F]mi-\[C]na con \[F]noi.

\endchorus






%%%%% STROFA
\beginverse		%Oppure \beginverse* se non si vuole il numero di fianco
\memorize 		% <<< DECOMMENTA se si vuole utilizzarne la funzione
%\chordsoff		% <<< DECOMMENTA se vuoi una strofa senza accordi
\[F]Questo pane che vi \[C]do
è il corpo mio per \[D-]voi;
sia fatto in mia me\[G7]mor\[C7]ia.
\[F]Questo calice sa\[C]rà,
nel sangue mio per \[D-]voi,
un'alleanza \[B&-]nuo\[C7]va.
\endverse
\beginverse*	
\[D-]Croce fu e fug\[C]gimmo noi,
rinne\[B&]gando chi era am\[C]ore eterno.
\[D-]"Resta qui, si fa \[C]sera ormai";
e di\[B&]vise ancora il \[G7]pa\[C]ne.
\endverse




%%%%% STROFA
\beginverse		%Oppure \beginverse* se non si vuole il numero di fianco
%\memorize 		% <<< DECOMMENTA se si vuole utilizzarne la funzione
%\chordsoff		% <<< DECOMMENTA se vuoi una strofa senza accordi
\[F]"Io vi mando ad annun\[C]ciar
la pace a chi non \[D-]sa
che il Regno si avv\[G7]ici\[C7]na.
\[F]Chi vi accoglie, in veri\[C]tà,
accoglie anche \[D-]me
e chi mi ha \[B&-]manda\[C7]to".
\endverse
\beginverse*	
\[D-]Croce fu ed in\[C]creduli
fummo so\[B&]rdi a chi lo di\[C]ceva vivo.
\[D-]"Era Lui per la \[C]via con noi";
ripa\[B&]rtimmo senza \[G7]indu\[C]gio.
\endverse





%%%%% RITORNELLO
\beginchorus
\textnote{\textbf{Rit.}}
\[F]Come un volto \[C]amico, \[D-]verità \[A-]inattesa:
\[B&]è Parola \[F]eterna, \[B&]Pane \[C]vi\[C7]vo.
\[F]Corre nuovo i\[C]l passo, \[D-]carico d'\[A-]annuncio:
\[B&]è risorto, \[F]vive \[B&]e 
\endchorus


%%%%%% EV. FINALE

\beginchorus %oppure \beginverse*
\vspace*{1.3\versesep}
\textnote{\textbf{Finale} \textit{(rallentando)}} %<<< EV. INDICAZIONI
 cam-\[F]mi-\[C]na con \[F]noi. \[F*]
\endchorus  %oppure \endverse







\endsong
%------------------------------------------------------------
%			FINE CANZONE
%------------------------------------------------------------



