%-------------------------------------------------------------
%			INIZIO	CANZONE
%-------------------------------------------------------------


%titolo: 	Terra buona per te
%autore: 	Diliberto, Arzuffi, Bodega, Ruaro
%tonalita: 	Re



%%%%%% TITOLO E IMPOSTAZONI
\beginsong{Terra buona per te}[by={Diliberto, Arzuffi, Bodega, Ruaro}] 	% <<< MODIFICA TITOLO E AUTORE
\transpose{0} 						% <<< TRASPOSIZIONE #TONI (0 nullo)
\momenti{Comunione}			% <<< INSERISCI MOMENTI	
% momenti vanno separati da ; e vanno scelti tra:
% Ingresso; Atto penitenziale; Acclamazione al Vangelo; Dopo il Vangelo; Offertorio; Comunione; Ringraziamento; Fine; Santi; Pasqua; Avvento; Natale; Quaresima; Canti Mariani; Battesimo; Prima Comunione; Cresima; Matrimonio; Meditazione; Spezzare del pane;
\ifchorded
	%\textnote{Tonalità migliore}	% <<< EV COMMENTI (tonalità originale/migliore)
\fi


%%%%%% INTRODUZIONE
\ifchorded
\vspace*{\versesep}
\textnote{Intro: \qquad \qquad }% (\eighthnote 120)} % <<  MODIFICA IL TEMPO
% Metronomo: \eighthnote (ottavo) \quarternote (quarto) \halfnote (due quarti)
\vspace*{-\versesep}
\beginverse*

\nolyrics

%---- Prima riga -----------------------------
\vspace*{-\versesep}
\[D] \[D] \[A]	 \[B-]  % \[*D] per indicare le pennate, \rep{2} le ripetizioni

%---- Ogni riga successiva -------------------
\vspace*{-\versesep}
\[G] \[D] \[A]	 	

%---- Ev Indicazioni -------------------------			
%\textnote{\textit{(Oppure tutta la strofa)} }	

\endverse
\fi




%%%%% RITORNELLO
\beginchorus
%\textnote{\textbf{Rit.}}

\[D]Vieni, vieni nel mio \[A]campo
\[G]c’è terra \[D]buona per \[G]te. \[A*]
\[D]Vieni e getta ancora il \[A]se\[B-]me
\[G]che cerche\[D]rà la \[A]lu\[B-]ce, 
\[G]germoglie\[D]rà e darà \[A]frut\[(D*)]to.
\endchorus


%%%%%% EV. INTERMEZZO
\beginverse*
\vspace*{1.3\versesep}
{
	\nolyrics
	\textnote{Breve intermezzo}
	
	\ifchorded

	%---- Prima riga -----------------------------
	\vspace*{-\versesep}
	\[D] \[D4] \[D] \rep{2} 


	\fi
	%---- Ev Indicazioni -------------------------			
	%\textnote{\textit{(ripetizione della strofa)}} 
	 
}
\vspace*{\versesep}
\endverse


%%%%% STROFA
\beginverse		%Oppure \beginverse* se non si vuole il numero di fianco
\memorize 		% <<< DECOMMENTA se si vuole utilizzarne la funzione
%\chordsoff		% <<< DECOMMENTA se vuoi una strofa senza accordi

Non \[B-]l’ansia, \[D]non la volon\[G]tà di a\[A]vere \[A]
potranno \[D]mai riem\[G]pire questa \[A]vita.
Ac\[G]colgo la pa\[A]rola che mi in\[F#-]vita \[B-]
a \[E-]vivere per gli \[G*]altri o\[D*]gni mio \[A]giorno,
\[F#*]co\[B-]sì potrò sco\[B-]prire vera\[G]mente,
\[G]moltipli\[D]cati, \[G]i doni \[E]rice\[A]vuti.

\endverse

%%%%% RITORNELLO
\beginchorus
%\textnote{\textbf{Rit.}}

\[D]Vieni, vieni nel mio \[A]campo
\[G]c’è terra \[D]buona per \[G]te. \[A*]
\[D]Vieni e getta ancora il \[A]se\[B-]me
\[G]che cerche\[D]rà la \[A]lu\[B-]ce, 
\[G]germoglie\[D]rà e darà \[A]frut\[(D*)]to.
\endchorus


%%%%%% EV. INTERMEZZO
\beginverse*
\vspace*{1.3\versesep}
{
	\nolyrics
	\textnote{Breve intermezzo}
	
	\ifchorded

	%---- Prima riga -----------------------------
	\vspace*{-\versesep}
	\[D] \[D4] \[D] \rep{2} 


	\fi
	%---- Ev Indicazioni -------------------------			
	%\textnote{\textit{(ripetizione della strofa)}} 
	 
}
\vspace*{\versesep}
\endverse


%%%%% STROFA
\beginverse		%Oppure \beginverse* se non si vuole il numero di fianco
%\memorize 		% <<< DECOMMENTA se si vuole utilizzarne la funzione
%\chordsoff		% <<< DECOMMENTA se vuoi una strofa senza accordi

E ^forte, ^da radici or^mai pro^fonde ^
si allarghe^rà la ^pianta nel ter^reno.
Con^templo nel tuo ^volto l’infi^nito, ^
ma ^so che quoti^diana è ^la fa^tica. 
^Di^fendimi, Si^gnore, e il mio cam^mino
^incontro al ^mondo ^sarà di ^nuovo ^festa.

\endverse



%%%%% RITORNELLO
\beginchorus
%\textnote{\textbf{Rit.}}

\[D]Vieni, vieni nel mio \[A]campo
\[G]c’è terra \[D]buona per \[G]te. \[A*]
\[D]Vieni e getta ancora il \[A]se\[B-]me
\[G]che cerche\[D]rà la \[A]lu\[B-]ce, 
\[G]germoglie\[D]rà e darà \[A]frut\[(D*)]to.
\endchorus


%%%%%% EV. INTERMEZZO
\beginverse*
\vspace*{1.3\versesep}
{
	\nolyrics
	\textnote{Breve intermezzo}
	
	\ifchorded

	%---- Prima riga -----------------------------
	\vspace*{-\versesep}
	\[D] \[D4] \[D] \rep{2} 


	\fi
	%---- Ev Indicazioni -------------------------			
	%\textnote{\textit{(ripetizione della strofa)}} 
	 
}
\vspace*{\versesep}
\endverse



%%%%% STROFA
\beginverse		%Oppure \beginverse* se non si vuole il numero di fianco
%\memorize 		% <<< DECOMMENTA se si vuole utilizzarne la funzione
%\chordsoff		% <<< DECOMMENTA se vuoi una strofa senza accordi

Per ^sempre ^questo tuo res^tarmi ac^canto ^
dilate^rà la ^gioia e la fi^ducia.
Com’^è davvero ^grande la pa^zienza ^
di ^Dio che semi^nando ^non si ^stanca!
^E ^tu, compagno ^nel mio andare in^torno,
^spalanca il ^cielo ^a questa ^voce at^tesa.

\endverse


%%%%% RITORNELLO
\beginchorus
%\textnote{\textbf{Rit.}}

\[D]Vieni, vieni nel mio \[A]campo
\[G]c’è terra \[D]buona per \[G]te. \[A*]
\[D]Vieni e getta ancora il \[A]se\[B-]me
\[G]che cerche\[D]rà la \[A]lu\[B-]ce, 
\[G]germoglie\[D]rà e darà \[A]frut \textit{(to)}
\[D]Vieni, vieni nel mio \[A]campo
\[G]c’è terra \[D]buona per \[G]te. \[A*]
\[D]Vieni e getta ancora il \[A]se\[B-]me
\[G]che cerche\[D]rà la \[A]lu\[B-]ce, 
\[G]germoglie\[D]rà e darà \[A]frut\[(D*)]to.
\endchorus


%%%%%% EV. INTERMEZZO
\beginverse*
\vspace*{1.3\versesep}
{
	\nolyrics
	\textnote{Finale strumentale}
	
	\ifchorded

	%---- Prima riga -----------------------------
	\vspace*{-\versesep}
	\[D] \[D4] \[D] 

	%---- Ogni riga successiva -------------------
	\vspace*{-\versesep}
	\[D] \[D4] \[D*] 	
	\fi
	%---- Ev Indicazioni -------------------------			
	%\textnote{\textit{(ripetizione della strofa)}} 
	 
}
\vspace*{\versesep}
\endverse


\endsong
%------------------------------------------------------------
%			FINE CANZONE
%------------------------------------------------------------


