%-------------------------------------------------------------
%			INIZIO	CANZONE
%-------------------------------------------------------------


%titolo: 	Ecco l'uomo
%autore: 	P. Sequeri
%tonalita:  Mi- 



%%%%%% TITOLO E IMPOSTAZONI
\beginsong{Ecco l'uomo}[by={P. Sequeri}]  % <<< MODIFICA TITOLO E AUTORE
\transpose{0} 						% <<< TRASPOSIZIONE #TONI (0 nullo)
%\preferflats  %SE VOGLIO FORZARE i bemolle come alterazioni
%\prefersharps %SE VOGLIO FORZARE i # come alterazioni
\momenti{}							% <<< INSERISCI MOMENTI	
% momenti vanno separati da ; e vanno scelti tra:
% Ingresso; Atto penitenziale; Acclamazione al Vangelo; Dopo il Vangelo; Offertorio; Comunione; Ringraziamento; Fine; Santi; Pasqua; Avvento; Natale; Quaresima; Canti Mariani; Battesimo; Prima Comunione; Cresima; Matrimonio; Meditazione; Spezzare del pane;
\ifchorded
	%\textnote{Tonalità migliore }	% <<< EV COMMENTI (tonalità originale/migliore)
\fi

\beginverse
\[E-7]Nella memoria di \[A-7]questa Passione
\[D7]noi ti chiediamo per\[G]dono, Si\[B7]gnore
\[E-7]per ogni volta che \[A-7]abbiamo lasciato
\[D7]il tuo fratello so\ch{B}{f}{f}{ff}rire da \[B7]solo.
\endverse

\beginchorus
\[E-]Noi ti pre\[A-]ghiamo \[D7]Uomo della \[G]Croce;
\[E-]Figlio e Fratel\[F#-dim]lo, \brk \[B7/9]noi speriamo in \[E-]te! \rep{2}
\endchorus

\beginverse
\chordsoff
Nella memoria di questa tua Morte
noi ti chiediamo coraggio, Signore
per ogni volta che il dono d'amore
ci chiederà di soffrire da soli.
\endverse

\beginverse
\chordsoff
Nella memoria dell'Ultima Cena
noi spezzeremo di nuovo il tuo Pane
ed ogni volta il tuo Corpo donato
sarà la nostra speranza di vita.
\endverse
\endsong

