%-------------------------------------------------------------
%			INIZIO	CANZONE
%-------------------------------------------------------------


%titolo: 	Accendi la vita
%autore: 	Bertoglio, Testa
%tonalita: 	Sol 



%%%%%% TITOLO E IMPOSTAZONI
\beginsong{Accendi la vita}[by={E. Bertoglio, A. Testa}] 	% <<< MODIFICA TITOLO E AUTORE
\transpose{0} 						% <<< TRASPOSIZIONE #TONI (0 nullo)
\momenti{Ingresso; Cresima; Prima Comunione; Fine; Santi; Matrimonio;}							% <<< INSERISCI MOMENTI	
% momenti vanno separati da ; e vanno scelti tra:
% Ingresso; Atto penitenziale; Acclamazione al Vangelo; Dopo il Vangelo; Offertorio; Comunione; Ringraziamento; Fine; Santi; Pasqua; Avvento; Natale; Quaresima; Canti Mariani; Battesimo; Prima Comunione; Cresima; Matrimonio; Meditazione;
\ifchorded
	%\textnote{Tonalità originale }	% <<< EV COMMENTI (tonalità originale/migliore)
\fi


%%%%%% INTRODUZIONE
\ifchorded
\vspace*{\versesep}
\musicnote{
\begin{minipage}{0.48\textwidth}
\textbf{Intro}
\hfill 
%( \eighthnote \, 80)   % <<  MODIFICA IL TEMPO
% Metronomo: \eighthnote (ottavo) \quarternote (quarto) \halfnote (due quarti)
\end{minipage}
} 	
\vspace*{-\versesep}
\beginverse*

\nolyrics

%---- Prima riga -----------------------------
\vspace*{-\versesep}
\[D] \[A]  \[D]	 % \[*D] per indicare le pennate, \rep{2} le ripetizioni

%---- Ogni riga successiva -------------------
%\vspace*{-\versesep}
%\[G] \[C]  \[D]	

%---- Ev Indicazioni -------------------------			
%\textnote{\textit{(Oppure tutta la strofa)} }	

\endverse
\fi




%%%%% STROFA
\beginverse		%Oppure \beginverse* se non si vuole il numero di fianco
\memorize 		% <<< DECOMMENTA se si vuole utilizzarne la funzione
%\chordsoff		& <<< DECOMMENTA se vuoi una strofa senza accordi

\[D]Come il vento in\[A]frange il mare a \[B-]riva \[B-]
\[G]così il tempo \[D]agita il mio cu\[A]ore \[A]
\[D]cerca il senso \[A]della sua esis\[B-]tenza \[B-]
\[G]cerca il volto \[D]mite del Si\[A]gnore. \[A]
Ed ho cer\[G]cato \echo{ed ho cercato}
per ogni \[D]via \echo{per ogni via}
su ogni \[F#-]vetta i miei \[G]piedi
han cammi\[A]nato
e nel do\[G]lore \echo{e nel dolore} 
mi son pie\[D]gato \echo{mi son piegato}
nella fa\[F#-]tica il tuo \[G]nome io ne\[A]gai
ma \[B-]poi ...

\endverse



%%%%% STROFA
\beginverse*		%Oppure \beginverse* se non si vuole il numero di fianco
%\memorize 		% <<< DECOMMENTA se si vuole utilizzarne la funzione
%\chordsoff		% <<< DECOMMENTA se vuoi una strofa senza accordi

^Ancora ho prepa^rato le mie ^cose ^
^pronto per un vi^aggio che ver^rà ^
^affidando al ^sonno della ^notte ^
^sogni di un in^contro che sa^rà ^
finché una ^voce \echo{finché una voce} 
mi ha des^tato \echo{mi ha destato}
finché il mio ^nome nel ^vento ha risuo^nato
è come un ^fuoco \echo{è come un fuoco}
che incendia il ^cuore \echo{che incendia il cuore}
un fuoco ^che caldo ^si libere^rà.

\endverse



%%%%% RITORNELLO
\beginchorus
\textnote{\textbf{Rit.}}

Accendi la \[D]vita che Dio ti \[G]dà
brucia d’a\[F#-]mo\[B-]re \[A] non perderti \[D]mai \[A]
accendi la \[D]vita perché ora \[G]sai
che il nostro \[F#-]viag\[B-]gio \[A] porta al Si\[D]gnor.

\endchorus



%%%%% STROFA
\beginverse		%Oppure \beginverse* se non si vuole il numero di fianco
%\memorize 		% <<< DECOMMENTA se si vuole utilizzarne la funzione
%\chordsoff		% <<< DECOMMENTA se vuoi una strofa senza accordi

^Come il sole ^dona il suo ca^lore ^
^tu o Signore ^doni veri^tà ^
^luce di una f^iamma senza ^fine ^
^alba di una ^nuova umani^tà. ^
Ed ho ascol^tato \echo{Ed ho ascoltato}
le tue pa^role \echo{le tue parole}
mi son nu^trito di ^nuovo del tuo a^more
ho aperto gli ^occhi \echo{ho aperto gli occhi}
alla mia ^gente \echo{alla mia gente}
con te vi^cino la ^vita esplode^rà.

\endverse


%%%%% RITORNELLO
\beginchorus
\textnote{\textbf{Rit.}}

Accendi la \[D]vita ... \quad \quad \rep{3}

\endchorus



\endsong
%------------------------------------------------------------
%			FINE CANZONE
%------------------------------------------------------------
