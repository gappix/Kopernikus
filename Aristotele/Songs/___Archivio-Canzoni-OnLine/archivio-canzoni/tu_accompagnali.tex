%titolo{Tu accompagnali}
%autore{}
%album{}
%tonalita{Mi}
%famiglia{Liturgica}
%gruppo{}
%momenti{Matrimonio}
%identificatore{tu_accompagnali}
%data_revisione{2014_10_01}
%trascrittore{Francesco Endrici}
\beginsong{Tu accompagnali}[ititle={Per questi nostri amici}]
\beginverse
Per \[E]questi nostri a\[B]mici ti pre\[C#-]ghiamo:
a\[A]iutali a \[B]cammi\[E]nare, a ripen\[B]sare ogni \[C#-]giorno
a \[A]questo giorno di \[E]fe\[C#-]sta, \[A]che Tu doni a \[D]\[B]noi.
\endverse
\beginchorus
\[E]Tu \[B]accompagna\[A]li, \[E]Dio \[C#-] dell'a\[D]more! \[B]
\[E]Tu \[B]accompagna\[A]li, \[E]Dio \[C#-] dell'a\[D]more! \[B]
\endchorus
\beginverse
%\chordsoff
Per ^questi nostri ^amici ti pre^ghiamo:
siano ^segno ^del tuo a^more \brk in un ^mondo senza ^sogni
siano ^sale, siano ^lu^ce, ^nostalgia di ^Te. ^
\endverse
\beginverse
%\chordsoff
Per ^questi nostri ^amici ti pre^ghiamo:
^copri^li di ^gioia come ^solo Tu sai ^fare,
per ac^cogliere la ^vi^ta e ^ringraziare ^Te. ^
\endverse
\beginverse
%\chordsoff
^C'è qualcosa in ^noi, che Tu ci hai ^dato,
che ^solo ^Tu, Si^gnore, hai po^tuto inven^tare,
^grande come il ^cie^lo e ^forse anche di ^più. ^
\endverse
\endsong