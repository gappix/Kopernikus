%titolo{L'olimpo}
%autore{Andrea Brandalise}
%album{VdB 2011}
%tonalita{A-}
%famiglia{Scout}
%gruppo{}
%momenti{}
%identificatore{l_olimpo}
%data_revisione{2013_12_30}
%trascrittore{Francesco Endrici}
\beginsong{L'Olimpo}[by={Brandalise}]
\transpose{4}
\beginchorus
O\[A-]limpo \[A-] \[F E]
un \[A-]monte \[A-] \[F E]
di\[A-]vino \[A-] \[F E]
su ogni \[A-]fronte \[A-] \[F E]
con \[A-]mille  \[A-] \[F E]
crea\[A-]ture \[A-] \[F E]
im\[A-]magina  \[A-] \[F E]
le avven\[A-]ture \[A-] \[F E]
\endchorus
\beginchorus
O\[A-]lim\[A-]po un \[F]mon\[E]te
di\[A-]vino \[A-]su ogni \[F]fron\[E]te
con \[A-]mil\[A-]le crea\[F]tu\[E]re
im\[A-]magina \[A-]le avven\[F]tu\[E]re  
\endchorus
\beginverse
Il \[F]vaso di Pan\[C]dora
si è \[G]rotto in mille \[A-]pezzi
met\[F]tiamoci alla \[C]prova
re\[G]cuperiamo i \[A-]cocci.
\endverse
\beginverse*
%\chordsoff
Da un ^giorno sol^tanto
di^venta non so ^quanto
una ^festa a sor^presa
di^venta un'im^presa.
\endverse
\beginverse*
I ^mali nel ^mondo
non ^ci stanno ^bene
^se ci organiz^ziamo
li ri^portiamo \[E]qui, sull'\dots
\endverse

\beginverse
Gli ^dei sono ^tanti
al^cuni più impor^tanti
e ^qui li presen^tiamo
e a ^loro ci inchi^niamo!
\endverse
\beginverse*
%\chordsoff
Ar^temide per la ^caccia
Posei^done pensa all'^acqua   
^Hermes ai mes^saggi
^Zeus insegna ai ^saggi.
\endverse
\beginverse*
Afro^dite ai senti^menti
^Ade ai pati^menti
^sono gli ^dei 
che ^vivon tutti in\[E]sieme  sull’…
\endverse

\beginverse
^O grande ^Zeus
^guidaci ^tu!
Al^lenaci quas^sù,
per ^far del bene ^giù!
\endverse
\beginverse*
\chordsoff
Tor^nei in grande ^stile
^gare a non fi^nire
al^loro ai mi^gliori
e ^pene a chi rimane ^fuori.
\endverse
\beginverse*
Ari^stotele per ^studiare
il ^sole da le^vare
man^giare, dor^mire
non ^ci si ferma \[E]mai!!  
Sull’O\[A-]lim\[A-]po.
\endverse
\endsong