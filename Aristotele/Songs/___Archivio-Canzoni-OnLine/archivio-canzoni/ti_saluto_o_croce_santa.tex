%titolo{Ti saluto o croce santa}
%autore{Gazzera, Damilano}
%album{}
%tonalita{Re-}
%famiglia{Liturgica}
%gruppo{}
%momenti{Quaresima}
%identificatore{ti_saluto_o_croce_santa}
%data_revisione{2014_10_01}
%trascrittore{Francesco Endrici}
\beginsong{Ti saluto o croce santa}[by={Gazzera, Damilano}]
\beginchorus
Ti sa\[D-]luto, o Croce \[F]santa
che por\[D-]tasti il Reden\[A7]tor;
gloria, \[F]lode, o\[G-]nor ti \[F]canta
ogni \[G-]lingua ed \[A7]ogni \[D-]cuor.
\endchorus
\beginverse
Sei ves\[F]sillo glorioso di \[C]Cristo,
sei sal\[F]vezza del \[G-]popol fe\[F]del.
Grondi \[G-]sangue inno\[A7]cente sul \[D-]tristo
che ti \[A7]volle martirio cru\[D-]del.
\endverse
\beginverse
%\chordsoff
Tu na^scesti fra braccia amo^rose
d'una ^Vergine ^Madre, o Ge^sù.
Tu mo^risti fra ^braccia pie^tose
d'una ^croce che data ti ^fu.
\endverse
\beginverse
%\chordsoff
O A^gnello divino immo^lato
sull'al^tar della ^croce, pie^tà!
Tu, che ^togli dal ^mondo il pec^cato,
salva l'^uomo che pace non ^ha.
\endverse
\beginverse
%\chordsoff
Del giu^dizio nel giorno tre^mendo,
sulle ^nubi del ^cielo ver^rai.
Piange^ranno le ^genti ve^dendo
qual tro^feo di gloria sa^rai.
\endverse
\endsong