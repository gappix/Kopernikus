%titolo{La vita è per volare}
%autore{}
%album{}
%tonalita{Re}
%famiglia{Scout}
%gruppo{}
%momenti{}
%identificatore{la_vita_e_per_volare}
%data_revisione{2012_12_05}
%trascrittore{Francesco Endrici}
\beginsong{La vita è per volare}
\beginverse
Tra \[D]la scogliera e il \[F#-]mondo sen\[G]tieri non ce \[D]n'è
c'è il \[B-]mare scuro e \[F#-]fondo e \[G]noi tra il mare e il \[D]ciel.
Con le \[F#]ali per pie\[B-]gare in \[F#]volo fin lag\[B-]giù
e \[G]poi rialzarsi in \[D]volo nell'\[E]universo \[A]blu. \[D]
\endverse
\beginchorus
La vita è per volare, \[G]per inven\[D]tare
\[G]scegli di \[D]vivere \[F#]da primo at\[G]tore.
In\[D]sieme per provare, \[G]per navi\[D]gare,
\[G]sopra le \[D]nuvole dal \[A]blu nel \[D]blu.
\endchorus
\chordsoff
\beginverse
Quel pellicano grasso che abita laggiù
mi ha detto: “Qui è uno spasso e io non mi muovo più”.
Ma è già un bel po' che giro ed ho il sospetto che
dalla mia ala a tiro ancor di meglio c'è.
\endverse
\beginverse
Sono arrivato in cima a quelle rocce sai,
e ho visto che se prima sembravan più che mai
lontane ed irreali le nuvole lassù
col vento tra le ali puoi giungerci anche tu.
\endverse
\beginverse
Se un acquazzone fitto ti prende in volo un dì
prova a pensare al giorno in cui splendeva il sol.
Con l'acqua tra le ali che sembri un baccalà
il che per un gabbiano non è gran dignità.
\endverse
\beginverse
Volare non è facile, ci son quei giorni che
diresti: “Amico caro, basta che voli te”.
Ma un nido non è il cielo ed è pur vero che
il pesce è fresco solo se vai a pescarlo te!
\endverse
\endsong
