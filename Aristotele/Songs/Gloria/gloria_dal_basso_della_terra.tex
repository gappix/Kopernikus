%-------------------------------------------------------------
%			INIZIO	CANZONE
%-------------------------------------------------------------


%titolo: 	Gloria dal basso della terra
%autore: 	Sermig
%tonalita: 	Fa



%%%%%% TITOLO E IMPOSTAZONI
\beginsong{Gloria dal basso della terra}[by={Sermig}]% <<< MODIFICA TITOLO E AUTORE
\transpose{0} 						% <<< TRASPOSIZIONE #TONI (0 nullo)
\momenti{}							% <<< INSERISCI MOMENTI	
% momenti vanno separati da ; e vanno scelti tra:
% Ingresso; Atto penitenziale; Acclamazione al Vangelo; Dopo il Vangelo; Offertorio; Comunione; Ringraziamento; Fine; Santi; Pasqua; Avvento; Natale; Quaresima; Canti Mariani; Battesimo; Prima Comunione; Cresima; Matrimonio; Meditazione; Spezzare del pane;
\ifchorded
	%\textnote{Tonalità migliore }	% <<< EV COMMENTI (tonalità originale/migliore)
\fi


%%%%%% INTRODUZIONE
\ifchorded
\vspace*{\versesep}
\textnote{Intro: \qquad \qquad  }%(\eighthnote 116) % <<  MODIFICA IL TEMPO
% Metronomo: \eighthnote (ottavo) \quarternote (quarto) \halfnote (due quarti)
\vspace*{-\versesep}
\beginverse*

\nolyrics

%---- Prima riga -----------------------------
\vspace*{-\versesep}
 |\[G]\[D7] | \[G]\[D7] | \[E-]\[D] | \[C] \[D] |
  % \[*D] per indicare le pennate, \rep{2} le ripetizioni

%---- Ogni riga successiva -------------------
\vspace*{-\versesep}
\[G]\[D7] | \[G]\[D7] | \[E-]\[D]| \[C]  \[D] |	

%---- Ev Indicazioni -------------------------			
%\textnote{\textit{(Oppure tutta la strofa)} }	

\endverse
\fi





\beginverse*
\memorize
|\[G]Gloria dal basso \[D7]della |\[G]terra,
gloria \[D7]dal più in|\[E-]fame degli ster\[D]mi-i\[C]ni.  \[D]
|\[G]Gloria nella \[D7]care|\[G]stia,
gloria \[D7]nella |\[E-]guerra più a\[D]tro-o\[C]ce. \[D]
\endverse



\beginverse*
|^Gloria, gloria, ^gloria,
|^solo tu hai la ^forza |^con la tua ^glori|^a ^
|^di-asciugare-le ^lacri|^me,
di por^tare |^nella tua ^glori|^a  ^
|\[E-]nell'alto dei \[D]cie-e|\[C]li  
i |\[C]vinti della \[D]te-er|\[E-]ra, 
i |\[E-]vinti della \[D]te-er|\[G]ra, 
i |\[C]vinti della \[D]te-er|\[G]ra. \[D7]
\endverse



%%%%%% EV. INTERMEZZO
\beginverse*
\vspace*{1.3\versesep}
{
	\nolyrics
	\textnote{Intermezzo strumentale}
	
	\ifchorded

	%---- Prima riga -----------------------------
	\vspace*{-\versesep}
	\[G]\[D7] | \[E-]\[D] | \[C]\[D]

	%---- Ogni riga successiva -------------------
	\vspace*{-\versesep}
	 \[G]\[D7] | \[G]\[D7] | \[E-]\[D] | \[C]\[D]


	\fi
	%---- Ev Indicazioni -------------------------			
	%\textnote{\textit{(ripetizione della strofa)}} 
	 
}
\vspace*{\versesep}
\endverse




\beginverse*
|\[G]Gloria dal basso \[D7]della |\[G]terra,
gloria \[D7]dal più in|\[E-]fame degli ster\[D]mi-i\[C]ni. \[D]
|\[G]Gloria nella \[D7]care|\[G]stia,
gloria \[D7]nella |\[E-]guerra più a\[D]tro-o|\[C]ce. \[D]
\endverse


\beginverse*
|^Gloria, gloria, ^gloria,
|^solo tu hai la ^forza |^con la tua ^glori|^a ^
|^di-asciugare-le ^lacri|^me,
di por^tare |^nella tua ^glori|^a ^
|\[E-]nell'alto dei \[D]cie-e|\[C]li  
i |\[C]vinti della \[D]te-er|\[E-]ra, 
i |\[E-]vinti della \[D]te-er|\[G]ra, 
i |\[C]vinti della \[D]te-er|\[G]ra. \[D7]
\endverse


%%%%%% EV. INTERMEZZO
\beginverse*
\vspace*{1.3\versesep}
{
	\nolyrics
	\textnote{Chiusura strumentale}
	
	\ifchorded

	%---- Prima riga -----------------------------
	\vspace*{-\versesep}
	\[G]\[D7] | \[E-]\[D] | \[C]\[D]

	%---- Ogni riga successiva -------------------
	\vspace*{-\versesep}
	 \[G]\[D7] | \[G]\[D7] | \[E-]\[D] | \[G*]


	\fi
	%---- Ev Indicazioni -------------------------			
	%\textnote{\textit{(ripetizione della strofa)}} 
	 
}
\vspace*{\versesep}
\endverse



\endsong
%------------------------------------------------------------
%			FINE CANZONE
%------------------------------------------------------------


