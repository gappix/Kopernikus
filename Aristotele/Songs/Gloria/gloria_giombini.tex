%-------------------------------------------------------------
%			INIZIO	CANZONE
%-------------------------------------------------------------


%titolo: 	Gloria
%autore: 	M. Giombini
%tonalita: 	Do



%%%%%% TITOLO E IMPOSTAZONI
\beginsong{Gloria (Giombini)}[by={M. Giombini}] 	% <<< MODIFICA TITOLO E AUTORE
\transpose{0} 						% <<< TRASPOSIZIONE #TONI (0 nullo)
\momenti{}							% <<< INSERISCI MOMENTI	
% momenti vanno separati da ; e vanno scelti tra:
% Ingresso; Atto penitenziale; Acclamazione al Vangelo; Dopo il Vangelo; Offertorio; Comunione; Ringraziamento; Fine; Santi; Pasqua; Avvento; Natale; Quaresima; Canti Mariani; Battesimo; Prima Comunione; Cresima; Matrimonio; Meditazione; Spezzare del pane;
\ifchorded
	%\textnote{Tonalità migliore }	% <<< EV COMMENTI (tonalità originale/migliore)
\fi


%%%%%% INTRODUZIONE
\ifchorded
\vspace*{\versesep}
\textnote{Intro: \qquad \qquad  }%(\eighthnote 116) % <<  MODIFICA IL TEMPO
% Metronomo: \eighthnote (ottavo) \quarternote (quarto) \halfnote (due quarti)
\vspace*{-\versesep}
\beginverse*

\nolyrics

%---- Prima riga -----------------------------
\vspace*{-\versesep}
\[C*] \[G]  \[A-] \[F] \quad \[C] \[G] \[C] \quad \[F*] \[G*]



%---- Ev Indicazioni -------------------------			
\textnote{\textit{(Come la seconda parte della prima strofa)} }	

\endverse
\fi


\beginverse*		%Oppure \beginverse* se non si vuole il numero di fianco
\memorize 		% <<< DECOMMENTA se si vuole utilizzarne la funzione
%\chordsoff		% <<< DECOMMENTA se vuoi una strofa senza accordi
\[C]Glo-\[G]o-o-\[C]ria, \quad \[F*] \[G*]
\[C]Glo-\[G]o-o-\[C]ria \quad \[F*] \[G*]
\[C] a \[G]Dio nell'\[A-]alto dei \[F]Cieli, 
\[C]Glo-\[G]o-o-\[C]ria! \quad \[F*] \[G*]
\endverse

\beginverse*
^E ^pa-a-a-^ce, \quad ^ ^
^e ^pa-a-a-^ce \quad ^ ^
^ in ^terra agli ^uomi^ni
di ^buona ^volon^tà. \quad \[C7]
\endverse

\beginverse*
Noi \[F]ti lodiamo \echo{noi ti lodiamo}
\[C]ti benediciamo \echo{ti benediciamo}
ti \[F]adoriamo \echo{ti adoriamo}
\[G]ti glorifichiamo \echo{ti glorifichiamo}.
\endverse

\beginverse*
^Ti rendi^a-a-a-^mo  \quad ^ ^
^gra-^a-a-^zie  \quad ^ ^
^ per ^la tua ^Glori^a
im^me-^e-en^sa.  \quad \[E7]
\endverse

\beginverse*
Si\[A-]gnore Figlio Uni\[E-]genito
Gesù \[F]Cristo, Si\[G]gnore \[C]Dio \quad \[E7]
\[A-] Agnello \[G] di Dio,
\[F] Figlio del \[E7]Padre.
\endverse

\beginverse*
\[A-] Tu che togli i pec\[E-]cati, \echo{tu che togli i peccati}
\[A-] i peccati del \[E]mondo, \echo{i peccati del mondo}
\[F] abbi pie\[C]tà  di noi,
\[D7] abbi pie\[G]tà  di noi!
\[A-] Tu che togli i pecca\[E-]ti,
\[A-] i peccati del \[E]mondo,
\[F] accogli, \[C] accogli,
\[D7] la nostra \[G]supplica!
\endverse

\beginverse*
\[C] Tu che siedi alla \[G]destra \[A-]
\[(A-)] Alla destra del \[E-]Padre \[F]
\[(F)] Abbi pie\[C]tà  di noi, \[D7] abbi pie\[G]tà  di noi
\endverse 


%%%%%% INTERMEZZO
\beginverse*
\vspace*{1.3\versesep}
{
	\nolyrics
	\textnote{Intermezzo strumentale}
	
	\ifchorded

	%---- Prima riga -----------------------------
	\vspace*{-\versesep}
	\[C] \[G] \[C] \quad \[F*] \[G*]	
	%---- Ogni riga successiva -------------------
	\vspace*{-\versesep}
	\[C] \[G] \[C] \quad \[F*] \[G*]
	%---- Ogni riga successiva -------------------
	\vspace*{-\versesep}
	\[C] \[G]  \[A-] \[F] \quad \[C] \[G] \[C] \quad \[C7]
	
	\fi
	%---- Ev Indicazioni -------------------------			
	%\textnote{\textit{(ripetizione della strofa)}} 
	 
}
\vspace*{\versesep}
\endverse


\beginverse*
Per\[F]chè tu solo il Santo \echo{perchè tu solo il Santo}
tu \[C]solo il Signore \echo{tu solo il Signore}
tu \[F]solo l'Altissimo \echo{tu solo l'Altissimo}
\[G]Gesù Cristo \echo{Gesù Cristo}.
\endverse

\beginverse*
^Con lo ^Spiri^to  \quad ^ ^
^Sa-^a-an^to  \quad ^ ^
^ nella ^gloria ^di Dio ^Padre 
^A-^a-a^men \quad \[F*] \[G*]
\endverse

\beginverse*
^Con lo ^Spiri^to  \echo{con lo ^Spiri^to Santo} 
^Sa-^a-an^to   \echo{nella ^glo^ria di Dio Padre}
^ nella ^gloria ^di Dio ^Padre 
^A-^a-a^men! \echo{nella \[F*]glo\[G*]ria di Dio Padre} 
\endverse



\beginverse*
\[C] nella \[G]gloria \[A-]di Dio \[F]Padre 
\[C]A-\[G7]a-a\[C]men! \quad \[G] \quad \[C*]
\endverse

\endsong
%------------------------------------------------------------
%			FINE CANZONE
%------------------------------------------------------------


