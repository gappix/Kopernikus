%-------------------------------------------------------------
%			INIZIO	CANZONE
%-------------------------------------------------------------


%titolo: 	Gloria (Buttazzo)
%autore: 	Buttazzo
%tonalita: 	Sol 



%%%%%% TITOLO E IMPOSTAZONI
\beginsong{Gloria (Buttazzo)}[by={F. Buttazzo}]	% <<< MODIFICA TITOLO E AUTORE
\transpose{0} 						% <<< TRASPOSIZIONE #TONI (0 nullo)
\momenti{}							% <<< INSERISCI MOMENTI	
% momenti vanno separati da ; e vanno scelti tra:
% Ingresso; Atto penite0nziale; Acclamazione al Vangelo; Dopo il Vangelo; Offertorio; Comunione; Ringraziamento; Fine; Santi; Pasqua; Avvento; Natale; Quaresima; Canti Mariani; Battesimo; Prima Comunione; Cresima; Matrimonio; Meditazione; Spezzare del pane;
\ifchorded
	%\textnote{Tonalità migliore }	% <<< EV COMMENTI (tonalità originale/migliore)
\fi


%%%%%% INTRODUZIONE
\ifchorded
\vspace*{\versesep}
\textnote{Intro: \qquad \qquad  }%(\eighthnote 116) % <<  MODIFICA IL TEMPO
% Metronomo: \eighthnote (ottavo) \quarternote (quarto) \halfnote (due quarti)
\vspace*{-\versesep}
\beginverse*

\nolyrics

%---- Prima riga -----------------------------
\vspace*{-\versesep}
\[G] \[D]  \[C] \[G]	% \[*D] per indicare le pennate, \rep{2} le ripetizioni

%---- Ogni riga successiva -------------------
\vspace*{-\versesep}
\[E-] \[C]  \[D4]	\[D]	

%---- Ev Indicazioni -------------------------			
\textnote{\textit{(Oppure tutto il ritornello)} }	

\endverse
\fi



%%%%% RITORNELLO
\beginchorus
\[G]Gloria a \[D]Dio nell'\[C]alto dei \[G]cieli 
e \[E-]pace in \[C]terra agli \[D4]uomi\[D]ni.
\[G]Gloria a \[F]Dio nell'\[C]alto dei \[G]cieli 
e \[E-]pace in \[C]terra agli \[D4]uomi\[G]ni. \[F] \[C] \[B7] 
\endchorus




%%%%% STROFA
\beginverse		%Oppure \beginverse* se non si vuole il numero di fianco
\memorize
\[E-]Noi Ti lo\[C]diamo, Ti \[D]benedi\[E-]ciamo.
\[E-]Ti ado\[C]riamo, Ti \[D]glorifi\[G]chiamo.
\[B]Ti ren\[E-]diamo \[C]gra\[G]zie 
\[C]per la Tua \[A-]Gloria im\[D4]men\[D]sa.
\endverse



%%%%% STROFA
\beginverse		%Oppure \beginverse* se non si vuole il numero di fianco
%\memorize 		% <<< DECOMMENTA se si vuole utilizzarne la funzione
%\chordsoff		% <<< DECOMMENTA se vuoi una strofa senza accordi
^Signore ^Dio, ^Re del c^ielo,
^Dio ^Padre ^Onnipo^tente.
^Gesù ^Cristo, Ag^nello di ^Dio, 
^Tu, ^Figlio del ^Pa-^dre.
\endverse



%%%%% STROFA
\beginverse		%Oppure \beginverse* se non si vuole il numero di fianco
%\memorize 		% <<< DECOMMENTA se si vuole utilizzarne la funzione
%\chordsoff		% <<< DECOMMENTA se vuoi una strofa senza accordi
^Tu che ^togli i pec^cati del ^mondo,  
^la nostra ^supplica as^colta, Sig^nore.
^Tu che s^iedi alla ^destra del ^Padre, 
^abbi pie^tà di  ^no-^i.
\endverse


%%%%% STROFA
\beginverse		%Oppure \beginverse* se non si vuole il numero di fianco
%\memorize 		% <<< DECOMMENTA se si vuole utilizzarne la funzione
%\chordsoff		% <<< DECOMMENTA se vuoi una strofa senza accordi
^Tu solo il ^Santo, Tu ^solo il Si^gnore, 
^Tu, l’Al^tissimo ^Gesù ^Cristo,
^con lo ^Spirito ^San-^to    
^nella ^Gloria del ^Pa-^dre
\endverse



%%%%% RITORNELLO
\beginchorus
\[G]Gloria a \[D]Dio nell'\[C]alto dei \[G]cieli 
e \[E-]pace in \[C]terra agli \[D4]uomi\[D]ni.
\[G]Gloria a \[F]Dio nell'\[C]alto dei \[G]cieli 
e \[E-]pace in \[C]terra agli \[D4]uomi\[G]ni. 
\endchorus





%%%%%% EV. INTERMEZZO
\beginverse*
\vspace*{1.3\versesep}
{
	\nolyrics
	\musicnote{Chiusura strumentale}
	
	\ifchorded

    %---- Prima riga -----------------------------
    \vspace*{-\versesep}
    \[F] \[C] \[D]  	% \[*D] per indicare le pennate, \rep{2} le ripetizioni

    %---- Ogni riga successiva -------------------
    \vspace*{-\versesep}
    \[G] \[F] \[C] \[D]  \[G*]		

	\fi
	%---- Ev Indicazioni -------------------------			
	%\textnote{\textit{(ripetizione della strofa)}} 
	 
}
\vspace*{\versesep}
\endverse


\endsong
%------------------------------------------------------------
%			FINE CANZONE
%------------------------------------------------------------


