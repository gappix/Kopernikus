%-------------------------------------------------------------
%			INIZIO	CANZONE
%-------------------------------------------------------------


%titolo: 	Gloria a te Cristo Gesù
%autore: 	Lecot
%tonalita: 	Sol 



%%%%%% TITOLO E IMPOSTAZONI
\beginsong{Gloria a te Cristo Gesù}[by={Inno del Giubileo — Lecot}]	% <<< MODIFICA TITOLO E AUTORE
\transpose{0} 						% <<< TRASPOSIZIONE #TONI (0 nullo)
\momenti{}							% <<< INSERISCI MOMENTI	
% momenti vanno separati da ; e vanno scelti tra:
% Ingresso; Atto penite0nziale; Acclamazione al Vangelo; Dopo il Vangelo; Offertorio; Comunione; Ringraziamento; Fine; Santi; Pasqua; Avvento; Natale; Quaresima; Canti Mariani; Battesimo; Prima Comunione; Cresima; Matrimonio; Meditazione; Spezzare del pane;
\ifchorded
	%\textnote{Tonalità migliore }	% <<< EV COMMENTI (tonalità originale/migliore)
\fi


%%%%%% INTRODUZIONE
\ifchorded
\vspace*{\versesep}
\textnote{Intro: \qquad \qquad  }%(\eighthnote 116) % <<  MODIFICA IL TEMPO
% Metronomo: \eighthnote (ottavo) \quarternote (quarto) \halfnote (due quarti)
\vspace*{-\versesep}
\beginverse*

\nolyrics

%---- Prima riga -----------------------------
\vspace*{-\versesep}
\[G] \[C*] \[G]	 \rep{2} % \[*D] per indicare le pennate, \rep{2} le ripetizioni

%---- Ogni riga successiva -------------------
%\vspace*{-\versesep}
%\[G] \[C]  \[D]	

%---- Ev Indicazioni -------------------------			
%\textnote{\textit{(Oppure tutta la strofa)} }	

\endverse
\fi



%%%%% RITORNELLO
\beginchorus
%\textnote{\textbf{Rit.}}
\[G]Glo\[C*]ria a \[G]te, Cristo \[C*]Ge\[B-]sù, \brk \[E-]og\[D*]gi e \[G]sempre tu \[A-]regne\[B]rai.
\[D]Gloria a \[C]te! \[G]Presto ver\[D]rai: \brk \[E-]sei spe\[C]ran\[C]za \[A-]so\[D*]lo \[G]tu.
\endchorus


%%%%% STROFA
\beginverse		%Oppure \beginverse* se non si vuole il numero di fianco
\memorize 		% <<< DECOMMENTA se si vuole utilizzarne la funzione
%\chordsoff		% <<< DECOMMENTA se vuoi una strofa senza accordi
\[G]Sia lode a \[D]te! \[E-]Cristo Si\[B-]gnore, \brk \[C]offri per\[F]dono, chiedi giu\[C]stizia:
l'anno di \[A-]grazia \[F]apre le \[E-]porte. \brk \[A-]Solo in \[G]te \[D]pace e uni\[B-]tà! 
\[G]Amen! Al\[C]le\[G*]lu\[D]ia!
\endverse


%%%%% STROFA
\beginverse		%Oppure \beginverse* se non si vuole il numero di fianco
%\memorize 		% <<< DECOMMENTA se si vuole utilizzarne la funzione
%\chordsoff		% <<< DECOMMENTA se vuoi una strofa senza accordi
^Sia lode a ^te! ^Prega con ^noi \brk ^la bene^detta Vergine ^Madre:
tu l'esau^disci, ^tu la co^roni. \brk ^Solo in ^te ^pace e uni^tà 
^Amen! Al^le^lu^ia!
\endverse


%%%%% STROFA
\beginverse		%Oppure \beginverse* se non si vuole il numero di fianco
%\memorize 		% <<< DECOMMENTA se si vuole utilizzarne la funzione
%\chordsoff		% <<< DECOMMENTA se vuoi una strofa senza accordi
^Sia lode a ^te! ^Tutta la ^Chiesa \brk ^celebra il ^Padre con la tua ^voce
e nello ^Spirito ^canta di ^gioia. \brk ^Solo in ^te ^pace e uni^tà 
^Amen! Al^le^lu^ia!
\endverse




\endsong
%------------------------------------------------------------
%			FINE CANZONE
%------------------------------------------------------------


