%-------------------------------------------------------------
%			INIZIO	CANZONE
%-------------------------------------------------------------


%titolo: 	Gloria (esme)
%autore: 	Gen Verde
%tonalita: 	FA e RE 



%%%%%% TITOLO E IMPOSTAZONI
\beginsong{Gloria nell'alto dei cieli}[by={Gen Verde, Esme}] 	% <<< MODIFICA TITOLO E AUTORE
\transpose{0} 						% <<< TRASPOSIZIONE #TONI (0 nullo)
\momenti{}							% <<< INSERISCI MOMENTI	
% momenti vanno separati da ; e vanno scelti tra:
% Ingresso; Atto penitenziale; Acclamazione al Vangelo; Dopo il Vangelo; Offertorio; Comunione; Ringraziamento; Fine; Santi; Pasqua; Avvento; Natale; Quaresima; Canti Mariani; Battesimo; Prima Comunione; Cresima; Matrimonio; Meditazione;
\ifchorded
	%\textnote{Tonalità originale }	% <<< EV COMMENTI (tonalità originale/migliore)
\fi




%%%%%% INTRODUZIONE
\ifchorded
\vspace*{\versesep}
\textnote{Intro: \qquad \qquad  }%(\eighthnote 116) % << MODIFICA IL TEMPO
% Metronomo: \eighthnote (ottavo) \quarternote (quarto) \halfnote (due quarti)
\vspace*{-\versesep}
\beginverse*

\nolyrics

%---- Prima riga -----------------------------
\vspace*{-\versesep}
\[F]   \[B&]  \[C]  \[C]	 \rep{2} % \[*D] per indicare le pennate, \rep{2} le ripetizioni

%---- Ogni riga successiva -------------------
%\vspace*{-\versesep}
%\[G] \[C]  \[D]	

%---- Ev Indicazioni -------------------------			
%\textnote{\textit{(Oppure tutta la strofa)} }	

\endverse
\fi





%%%%% RITORNELLO
\beginchorus
%\textnote{\textbf{Rit.}}

\[F]Gloria, \[B&]gloria a \[D-]Dio. \[C]
Gloria, \[F]gloria nel\[B&]l'alto dei \[D-]cie\[C]li.
\[F]Pace in \[B&]terra agli \[D-]uomi\[C]ni
di \[F]buona \[B&]volon\[F]tà. \[B&] 
\[F]Gl\[B&]o\[F]ria!
\endchorus







%%%%% STROFA
\beginverse*		%Oppure \beginverse* se non si vuole il numero di fianco
\memorize 		% <<< DECOMMENTA se si vuole utilizzarne la funzione
%\chordsoff		& <<< DECOMMENTA se vuoi una strofa senza accordi
Noi \[B&]ti lo\[F]diamo, \[G-]ti benedi\[F]ciamo,
ti \[B&]ador\[F]iamo, glo\[E&]rifichiamo \[F]te,
\[B&]ti ren\[F]diamo \[G-7]grazie per la \[F]tua immensa
\[E&]glor\[C4]ia. \[C]

\endverse


%%%%% STROFA
\beginverse*
Si^gnore ^Dio, ^glor^ia!  ^Re del ci^elo, ^glor^ia!
^Dio ^Padre, ^Dio onnipo^tente, ^glor\[C]ia! \[G-] \[E&] \[C]
\endverse



%%%%% RITORNELLO
\beginchorus
%\textnote{\textbf{Rit.}}

\[F]Gloria, \[B&]gloria a \[D-]Dio. \[C]
Gloria, \[F]gloria nel\[B&]l'alto dei \[D-]cie\[C]li.
\[F]Pace in \[B&]terra agli \[D-]uomi\[C]ni
di \[F]buona \[B&]volon\[F]tà. \[B&] 
\[F]Gl\[B&]o\[F]ria!
\endchorus



%%%%% STROFA
\beginverse*
Si\[F]gnore, Figlio uni\[E&]genito, \[B&]Gesù Cri\[F]sto,
Si\[F]gnore, Agnello di \[E&]Dio, \[B&]Figlio del Pad\[F]re.
\[F]Tu che togli i pec\[E&]cati del mondo,
a\[B&]bbi pietà  di no\[F]i;
\[F]tu che togli i pec\[E&]cati del mondo,
a\[B&]ccogli la nostra su\[F]pplica;
\[F]tu che siedi alla \[E&]destra del Padre,
\[B&]abbi pietà  di n\[C4]oi. \[C]
\endverse




%%%%% RITORNELLO
\beginchorus
%\textnote{\textbf{Rit.}}

\[F]Gloria, \[B&]gloria a \[D-]Dio. \[C]
Gloria, \[F]gloria nel\[B&]l'alto dei \[D-]cie\[C]li.
\[F]Pace in \[B&]terra agli \[D-]uomi\[C]ni
di \[F]buona \[B&]volon\[F]tà. \[B&] 
\[F]Gl\[B&]o\[F]ria!
\endchorus


%%%%% STROFA
\beginverse*
Per^chè tu ^solo il ^Santo, il Si^gnore,
tu ^solo l'Al^tissimo, ^Cristo G^esù
^con lo ^Spirito ^Santo nella ^gloria
del ^Pad\[C]re. \[G-] \[E&] \[C]
\endverse



%%%%% RITORNELLO
\beginchorus
%\textnote{\textbf{Rit.}}

\[F]Gloria, \[B&]gloria a \[D-]Dio. \[C]
Gloria, \[F]gloria nel\[B&]l'alto dei \[D-]cie\[C]li.
\[F]Pace in \[B&]terra agli \[D-]uomi\[C]ni
di \[F]buona \[B&]volon\[F]tà. \[B&] 
\[F]Gl\[B&]o\[F]ria! \[F] \[*F]
\endchorus




\endsong
%------------------------------------------------------------
%			FINE CANZONE
%------------------------------------------------------------
%++++++++++++++++++++++++++++++++++++++++++++++++++++++++++++
%			CANZONE TRASPOSTA
%++++++++++++++++++++++++++++++++++++++++++++++++++++++++++++
\ifchorded
%decremento contatore per avere stesso numero
\addtocounter{songnum}{-1} 
\beginsong{Gloria nell'alto dei cieli}[by={Gen Verde, Esme}] 	% <<< COPIA TITOLO E AUTORE
\transpose{-3} 						% <<< TRASPOSIZIONE #TONI + - (0 nullo)
\ifchorded
	\textnote{Tonalità migliore per le chitarre}	% <<< EV COMMENTI (tonalità originale/migliore)
\fi




%%%%%% INTRODUZIONE
\ifchorded
\vspace*{\versesep}
\textnote{Intro: \qquad \qquad  }%(\eighthnote 116) % << MODIFICA IL TEMPO
% Metronomo: \eighthnote (ottavo) \quarternote (quarto) \halfnote (due quarti)
\vspace*{-\versesep}
\beginverse*

\nolyrics

%---- Prima riga -----------------------------
\vspace*{-\versesep}
\[F]   \[B&]  \[C]  \[C]	 \rep{2} % \[*D] per indicare le pennate, \rep{2} le ripetizioni

%---- Ogni riga successiva -------------------
%\vspace*{-\versesep}
%\[G] \[C]  \[D]	

%---- Ev Indicazioni -------------------------			
%\textnote{\textit{(Oppure tutta la strofa)} }	

\endverse
\fi





%%%%% RITORNELLO
\beginchorus
%\textnote{\textbf{Rit.}}

\[F]Gloria, \[B&]gloria a \[D-]Dio. \[C]
Gloria, \[F]gloria nel\[B&]l'alto dei \[D-]cie\[C]li.
\[F]Pace in \[B&]terra agli \[D-]uomi\[C]ni
di \[F]buona \[B&]volon\[F]tà. \[B&] 
\[F]Gl\[B&]o\[F]ria!
\endchorus







%%%%% STROFA
\beginverse*		%Oppure \beginverse* se non si vuole il numero di fianco
\memorize 		% <<< DECOMMENTA se si vuole utilizzarne la funzione
%\chordsoff		& <<< DECOMMENTA se vuoi una strofa senza accordi
Noi \[B&]ti lo\[F]diamo, \[G-]ti benedi\[F]ciamo,
ti \[B&]ador\[F]iamo, glo\[E&]rifichiamo \[F]te,
\[B&]ti ren\[F]diamo \[G-7]grazie per la \[F]tua immensa
\[E&]glor\[C4]ia. \[C]

\endverse


%%%%% STROFA
\beginverse*
Si^gnore ^Dio, ^glor^ia!  ^Re del ci^elo, ^glor^ia!
^Dio ^Padre, ^Dio onnipo^tente, ^glor\[C]ia! \[G-] \[E&] \[C]
\endverse



%%%%% RITORNELLO
\beginchorus
%\textnote{\textbf{Rit.}}

\[F]Gloria, \[B&]gloria a \[D-]Dio. \[C]
Gloria, \[F]gloria nel\[B&]l'alto dei \[D-]cie\[C]li.
\[F]Pace in \[B&]terra agli \[D-]uomi\[C]ni
di \[F]buona \[B&]volon\[F]tà. \[B&] 
\[F]Gl\[B&]o\[F]ria!
\endchorus



%%%%% STROFA
\beginverse*
Si\[F]gnore, Figlio uni\[E&]genito, \[B&]Gesù Cri\[F]sto,
Si\[F]gnore, Agnello di \[E&]Dio, \[B&]Figlio del Pad\[F]re.
\[F]Tu che togli i pec\[E&]cati del mondo,
a\[B&]bbi pietà  di no\[F]i;
\[F]tu che togli i pec\[E&]cati del mondo,
a\[B&]ccogli la nostra su\[F]pplica;
\[F]tu che siedi alla \[E&]destra del Padre,
\[B&]abbi pietà  di n\[C4]oi. \[C]
\endverse




%%%%% RITORNELLO
\beginchorus
%\textnote{\textbf{Rit.}}

\[F]Gloria, \[B&]gloria a \[D-]Dio. \[C]
Gloria, \[F]gloria nel\[B&]l'alto dei \[D-]cie\[C]li.
\[F]Pace in \[B&]terra agli \[D-]uomi\[C]ni
di \[F]buona \[B&]volon\[F]tà. \[B&] 
\[F]Gl\[B&]o\[F]ria!
\endchorus


%%%%% STROFA
\beginverse*
Per^chè tu ^solo il ^Santo, il Si^gnore,
tu ^solo l'Al^tissimo, ^Cristo G^esù
^con lo ^Spirito ^Santo nella ^gloria
del ^Pad\[C]re. \[G-] \[E&] \[C]
\endverse



%%%%% RITORNELLO
\beginchorus
%\textnote{\textbf{Rit.}}

\[F]Gloria, \[B&]gloria a \[D-]Dio. \[C]
Gloria, \[F]gloria nel\[B&]l'alto dei \[D-]cie\[C]li.
\[F]Pace in \[B&]terra agli \[D-]uomi\[C]ni
di \[F]buona \[B&]volon\[F]tà. \[B&] 
\[F]Gl\[B&]o\[F]ria! \[F] \[*F]
\endchorus




\endsong

\fi
%++++++++++++++++++++++++++++++++++++++++++++++++++++++++++++
%			FINE CANZONE TRASPOSTA
%++++++++++++++++++++++++++++++++++++++++++++++++++++++++++++

