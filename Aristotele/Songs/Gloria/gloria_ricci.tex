%-------------------------------------------------------------
%			INIZIO	CANZONE
%-------------------------------------------------------------


%titolo: 	Gloria (Ricci)
%autore: 	D. Ricci
%tonalita: 	re



%%%%%% TITOLO E IMPOSTAZONI
\beginsong{Gloria (Ricci)}[by={D. Ricci}] 	% <<< MODIFICA TITOLO E AUTORE
\transpose{0} 						% <<< TRASPOSIZIONE #TONI (0 nullo)
\momenti{}							% <<< INSERISCI MOMENTI	
% momenti vanno separati da ; e vanno scelti tra:
% Ingresso; Atto penitenziale; Acclamazione al Vangelo; Dopo il Vangelo; Offertorio; Comunione; Ringraziamento; Fine; Santi; Pasqua; Avvento; Natale; Quaresima; Canti Mariani; Battesimo; Prima Comunione; Cresima; Matrimonio; Meditazione;
\ifchorded
	%\textnote{Tonalità migliore per le chitarre }	% <<< EV COMMENTI (tonalità originale/migliore)
\fi




%%%%%% INTRODUZIONE
\ifchorded
\vspace*{\versesep}
\textnote{Intro: \qquad \qquad  }%(\eighthnote 116) % << MODIFICA IL TEMPO
% Metronomo: \eighthnote (ottavo) \quarternote (quarto) \halfnote (due quarti)
\vspace*{-\versesep}
\beginverse*

\nolyrics

%---- Prima riga -----------------------------
\vspace*{-\versesep}
\[D]    \[C] \[D]  \[C]	 \rep{2} % \[*D] per indicare le pennate, \rep{2} le ripetizioni

%---- Ogni riga successiva -------------------
%\vspace*{-\versesep}
%\[G] \[C]  \[D]	

%---- Ev Indicazioni -------------------------			
%\textnote{\textit{(Oppure tutta la strofa)} }	

\endverse
\fi





%%%%% RITORNELLO
\beginchorus
%\textnote{\textbf{Rit.}}

\[D]Gloria \echo{nell'\[C]alto dei cieli},
\[D]Gloria \echo{nell'\[C]alto dei cieli},
\[D]Gloria \echo{nell'\[C]alto dei cieli},
\[G]Gloria Gloria a \[D]Dio!
\[D]Pace-e \echo{agli \[C]uomini in terra},
\[D]Pace-e \echo{agli \[C]uomini in terra},
\[D]Pace-e \echo{agli \[C]uomini di},
\[G]buona volon\[D]tà !
\endchorus







%%%%% STROFA
\beginverse*		%Oppure \beginverse* se non si vuole il numero di fianco
\memorize 		% <<< DECOMMENTA se si vuole utilizzarne la funzione
%\chordsoff		& <<< DECOMMENTA se vuoi una strofa senza accordi

\[F] Noi ti lodiamo, \[G]ti benediciamo,
\[D] ti adoriamo, ti glorifichiamo.
\[F] Ti rendiamo grazie \[G]per la tua gloria 
im\[D]mensa, gloria immensa.
\[F] Signore Dio, \[G]Re del Cielo, 
 Dio \[D]Padre Onnipotente.
\[F] Signore Dio, \[G]Re del Cielo, 
Dio \[D]Padre Onnipo\[A]tente.

\endverse






%%%%% RITORNELLO
\beginchorus
%\textnote{\textbf{Rit.}}

\[D]Gloria \echo{nell'\[C]alto dei cieli},
\[D]Gloria \echo{nell'\[C]alto dei cieli},
\[D]Gloria \echo{nell'\[C]alto dei cieli},
\[G]Gloria Gloria a \[D]Dio!
\[D]Pace-e \echo{agli \[C]uomini in terra},
\[D]Pace-e \echo{agli \[C]uomini in terra},
\[D]Pace-e \echo{agli \[C]uomini di},
\[G]buona volon\[D]tà !
\endchorus





%%%%% STROFA
\beginverse*
^ Signore Figlio uni^genito Gesù Cristo,
Sig^nore Dio agnello di Dio, figlio del Padre.
^ Tu che togli i pec^cati del mondo, 
ab^bi pietà di noi,
^ Tu che togli i pec^cati del mondo, 
ac^cogli la nostra supplica,
^ Tu che siedi alla ^destra di Dio Padre, 
^abbi pietà di ^noi.
\endverse







%%%%% RITORNELLO
\beginchorus
%\textnote{\textbf{Rit.}}

\[D]Gloria \echo{nell'\[C]alto dei cieli},
\[D]Gloria \echo{nell'\[C]alto dei cieli},
\[D]Gloria \echo{nell'\[C]alto dei cieli},
\[G]Gloria Gloria a \[D]Dio!
\[D]Pace-e \echo{agli \[C]uomini in terra},
\[D]Pace-e \echo{agli \[C]uomini in terra},
\[D]Pace-e \echo{agli \[C]uomini di},
\[G]buona volon\[D]tà !
\endchorus





%%%%% STROFA
\beginverse*
^ Perchè tu solo il Santo, ^tu solo il Signore,
^tu solo l’Altissimo 
Gesù Cristo,
^ con lo Spirito ^Santo nella Gloria 
di Dio ^Padre, 
\[A]A-a-\[D]men. \[C] \[D] \[C] \[D] \[C] \[D*]
\endverse




\endsong
%------------------------------------------------------------
%			FINE CANZONE
%------------------------------------------------------------
