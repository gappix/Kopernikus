%-------------------------------------------------------------
%			INIZIO	CANZONE
%-------------------------------------------------------------


%titolo: 	Santo Bonfitto
%autore: 	Michele Bonfitto
%tonalita: 	Sol 



%%%%%% TITOLO E IMPOSTAZONI
\beginsong{Santo Bonfitto}[by={M. Bonfitto}] 	% <<< MODIFICA TITOLO E AUTORE
\transpose{-2} 						% <<< TRASPOSIZIONE #TONI (0 nullo)
%\preferflats  %SE VOGLIO FORZARE i bemolle come alterazioni
%\prefersharps %SE VOGLIO FORZARE i # come alterazioni
\momenti{}							% <<< INSERISCI MOMENTI	
% momenti vanno separati da ; e vanno scelti tra:
% Ingresso; Atto penitenziale; Acclamazione al Vangelo; Dopo il Vangelo; Offertorio; Comunione; Ringraziamento; Fine; Santi; Pasqua; Avvento; Natale; Quaresima; Canti Mariani; Battesimo; Prima Comunione; Cresima; Matrimonio; Meditazione; Spezzare del pane;
\ifchorded
	%\textnote{Tonalità migliore }	% <<< EV COMMENTI (tonalità originale/migliore)
\fi


%%%%%% INTRODUZIONE
\ifchorded
\vspace*{\versesep}
\textnote{Intro: \qquad \qquad  }%(\eighthnote 116) % <<  MODIFICA IL TEMPO
% Metronomo: \eighthnote (ottavo) \quarternote (quarto) \halfnote (due quarti)
\vspace*{-\versesep}
\beginverse*

\nolyrics

%---- Prima riga -----------------------------
\vspace*{-\versesep}
\[G] \[G]	 % \[*D] per indicare le pennate, \rep{2} le ripetizioni

%---- Ogni riga successiva -------------------
%\vspace*{-\versesep}
%\[G] \[C]  \[D]	

%---- Ev Indicazioni -------------------------			
%\textnote{\textit{(Oppure tutta la strofa)} }	

\endverse
\fi

%%%%% STROFA
\beginverse	*	%Oppure \beginverse* se non si vuole il numero di fianco
\memorize 		% <<< DECOMMENTA se si vuole utilizzarne la funzione
%\chordsoff		% <<< DECOMMENTA se vuoi una strofa senza accordi

\[G]San\[D]to, \[E-]santo, \[A-]santo il Si\[D]gnore 
\[G]Dio dell'uni\[D*]\[C*]ver\[D]so. \[(A-*)] \[(D)]

\endverse

%%%%% STROFA
\beginverse*		%Oppure \beginverse* se non si vuole il numero di fianco
\memorize 		% <<< DECOMMENTA se si vuole utilizzarne la funzione
%\chordsoff		% <<< DECOMMENTA se vuoi una strofa senza accordi

I \[G]cieli e la \[C]terra sono \[A-]pieni della tua \[D]gloria.

\endverse

%%%%% RITORNELLO
\beginchorus

O\[G]sanna, o\[E-]sanna, \brk  o\[B-]sa\[E-]nna nell'\[A-]alto dei \[D]cieli. \[D]

\endchorus

%%%%% STROFA
\beginverse*		%Oppure \beginverse* se non si vuole il numero di fianco
\memorize 		% <<< DECOMMENTA se si vuole utilizzarne la funzione
%\chordsoff		% <<< DECOMMENTA se vuoi una strofa senza accordi

Bene^detto colui che ^viene nel ^nome del Si^gnore.

\endverse

%%%%% RITORNELLO
\beginchorus

O\[G]sanna, o\[E-]sanna, \brk o\[B-]sa\[E-]nna nell'\[A-]alto dei \[D]cieli. \[D]

\endchorus

%%%%% STROFA
\beginverse*		%Oppure \beginverse* se non si vuole il numero di fianco
\memorize 		% <<< DECOMMENTA se si vuole utilizzarne la funzione
%\chordsoff		% <<< DECOMMENTA se vuoi una strofa senza accordi

I ^cieli  \quad \echo{Benedetto!}
e la ^terra  \quad \echo{colui che viene!}
sono ^pieni della tua ^gloria. \echo{nel nome del Signore!}


\endverse

%%%%% RITORNELLO
\beginchorus

O\[G]sanna, o\[E-]sanna, \brk o\[B-]sa\[E-]nna nell'\[A-]alto dei \[D]cieli. \[D*]

\endchorus

\endsong
%------------------------------------------------------------
%			FINE CANZONE
%------------------------------------------------------------