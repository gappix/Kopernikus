%-------------------------------------------------------------
%			INIZIO	CANZONE
%-------------------------------------------------------------


%titolo: 	Santo Milan
%autore: 	Gen Verde
%tonalita: 	Sol 



%%%%%% TITOLO E IMPOSTAZONI
\beginsong{Santo Milan}[by={Gen Verde}] 	% <<< MODIFICA TITOLO E AUTORE
\transpose{0} 						% <<< TRASPOSIZIONE #TONI (0 nullo)
\momenti{Santo}							% <<< INSERISCI MOMENTI	
% momenti vanno separati da ; e vanno scelti tra:
% Ingresso; Atto penitenziale; Acclamazione al Vangelo; Dopo il Vangelo; Offertorio; Comunione; Ringraziamento; Fine; Santi; Pasqua; Avvento; Natale; Quaresima; Canti Mariani; Battesimo; Prima Comunione; Cresima; Matrimonio; Meditazione;
\ifchorded
	\textnote{$\bigstar$ Tonalità originale }	% <<< EV COMMENTI (tonalità originale/migliore)
\fi


%%%%%% INTRODUZIONE
\ifchorded
\vspace*{\versesep}
\musicnote{
\begin{minipage}{0.48\textwidth}
\textbf{Intro}
\hfill 
%( \eighthnote \, 80)   % <<  MODIFICA IL TEMPO
% Metronomo: \eighthnote (ottavo) \quarternote (quarto) \halfnote (due quarti)
\end{minipage}
} 	
\vspace*{-\versesep}
\beginverse*

\nolyrics

%---- Prima riga -----------------------------
\vspace*{-\versesep}
\[A] \[E]  \[C#-] \[B]	 % \[*D] per indicare le pennate, \rep{2} le ripetizioni

%---- Ogni riga successiva -------------------
\vspace*{-\versesep}
\[F#-] \[E]  \[A]  \[B]	

%---- Ev Indicazioni -------------------------			
\textnote{\textit{(come le prime due righe)} }	

\endverse
\fi








%%%%% RITORNELLO
\beginchorus

\[A]San\[E]to, \[C#-]San\[B]to,
\[F#-]Santo il Si\[E]gnore, \[A]Dio dell'uni\[B]verso.
\[A]San\[E]to, \[C#-]San\[B]to.
I \[F#-]cieli e la \[E]terra 
sono \[A]pieni della tua \[(F#-)]glo\[E]ria.

\endchorus



%%%%% STROFA
\beginverse*		%Oppure \beginverse* se non si vuole il numero di fianco
%\memorize 		% <<< DECOMMENTA se si vuole utilizzarne la funzione
%\chordsoff		% <<< DECOMMENTA se vuoi una strofa senza accordi

O\[A]sanna nel\[B]l'alto dei \[A]cie\[B]li.
O\[F#-]sanna nell'alto dei \[A]cieli.

\endverse



%%%%% RITORNELLO
\beginchorus

\[A]San\[E]to, \[C#-]San\[B]to,
\[F#-]Santo il Si\[E]gnore, \[A]Dio dell'uni\[B]verso.
\[A]San\[E]to, \[C#-]San\[B]to.
I \[F#-]cieli e la \[E]terra 
sono \[A]pieni della tua \[(F#-)]glo\[E]ria.

\endchorus




%%%%% STROFA
\beginverse*		%Oppure \beginverse* se non si vuole il numero di fianco
%\memorize 		% <<< DECOMMENTA se si vuole utilizzarne la funzione
%\chordsoff		& <<< DECOMMENTA se vuoi una strofa senza accordi

\[B]Benedetto co\[A]lui che viene
nel \[E]nome del Sig\[B]nore.
O\[A]sanna nel\[B]l'alto dei \[A]cie\[B]li.
O\[F#-]sanna nell'alto dei \[A]cieli.

\endverse


%%%%% RITORNELLO
\beginchorus

\[A]San\[E]to, \[C#-]San\[B]to,
\[F#-]Sa-\[A]a-n\[E]to.  \[*E] 

\endchorus











\endsong
%------------------------------------------------------------
%			FINE CANZONE
%------------------------------------------------------------




%++++++++++++++++++++++++++++++++++++++++++++++++++++++++++++
%			CANZONE TRASPOSTA
%++++++++++++++++++++++++++++++++++++++++++++++++++++++++++++
\ifchorded
%decremento contatore per avere stesso numero
\addtocounter{songnum}{-1} 
\beginsong{Santo Milan}[by={Gen Verde}] 	% <<< COPIA TITOLO E AUTORE
\transpose{-2} 						% <<< TRASPOSIZIONE #TONI + - (0 nullo)
\ifchorded
	\textnote{$\triangle$ Tonalità più facile per le chitarre}	% <<< EV COMMENTI (tonalità originale/migliore)
\fi


%%%%%% INTRODUZIONE
\ifchorded
\vspace*{\versesep}
\musicnote{
\begin{minipage}{0.48\textwidth}
\textbf{Intro}
\hfill 
%( \eighthnote \, 80)   % <<  MODIFICA IL TEMPO
% Metronomo: \eighthnote (ottavo) \quarternote (quarto) \halfnote (due quarti)
\end{minipage}
} 	
\vspace*{-\versesep}
\beginverse*

\nolyrics

%---- Prima riga -----------------------------
\vspace*{-\versesep}
\[A] \[E]  \[C#-] \[B]	 % \[*D] per indicare le pennate, \rep{2} le ripetizioni

%---- Ogni riga successiva -------------------
\vspace*{-\versesep}
\[F#-] \[E]  \[A]  \[B]	

%---- Ev Indicazioni -------------------------			
\textnote{\textit{(come le prime due righe)} }	

\endverse
\fi








%%%%% RITORNELLO
\beginchorus

\[A]San\[E]to, \[C#-]San\[B]to,
\[F#-]Santo il Si\[E]gnore, \[A]Dio dell'uni\[B]verso.
\[A]San\[E]to, \[C#-]San\[B]to.
I \[F#-]cieli e la \[E]terra 
sono \[A]pieni della tua \[(F#-)]glo\[E]ria.

\endchorus



%%%%% STROFA
\beginverse*		%Oppure \beginverse* se non si vuole il numero di fianco
%\memorize 		% <<< DECOMMENTA se si vuole utilizzarne la funzione
%\chordsoff		% <<< DECOMMENTA se vuoi una strofa senza accordi

O\[A]sanna nel\[B]l'alto dei \[A]cie\[B]li.
O\[F#-]sanna nell'alto dei \[A]cieli.

\endverse



%%%%% RITORNELLO
\beginchorus

\[A]San\[E]to, \[C#-]San\[B]to,
\[F#-]Santo il Si\[E]gnore, \[A]Dio dell'uni\[B]verso.
\[A]San\[E]to, \[C#-]San\[B]to.
I \[F#-]cieli e la \[E]terra 
sono \[A]pieni della tua \[(F#-)]glo\[E]ria.

\endchorus




%%%%% STROFA
\beginverse*		%Oppure \beginverse* se non si vuole il numero di fianco
%\memorize 		% <<< DECOMMENTA se si vuole utilizzarne la funzione
%\chordsoff		& <<< DECOMMENTA se vuoi una strofa senza accordi

\[B]Benedetto co\[A]lui che viene
nel \[E]nome del Sig\[B]nore.
O\[A]sanna nel\[B]l'alto dei \[A]cie\[B]li.
O\[F#-]sanna nell'alto dei \[A]cieli.

\endverse


%%%%% RITORNELLO
\beginchorus

\[A]San\[E]to, \[C#-]San\[B]to,
\[F#-]Sa-\[A]a-n\[E]to.  \[*E] 

\endchorus






\endsong

\fi
%++++++++++++++++++++++++++++++++++++++++++++++++++++++++++++
%			FINE CANZONE TRASPOSTA
%++++++++++++++++++++++++++++++++++++++++++++++++++++++++++++
