%-------------------------------------------------------------
%			INIZIO	CANZONE
%-------------------------------------------------------------


%titolo: 	Santo Gen Verde
%autore: 	Gen Verde
%tonalita: 	Mi 



%%%%%% TITOLO E IMPOSTAZONI
\beginsong{Santo Gen Verde}[by={Gen Verde}] 	% <<< MODIFICA TITOLO E AUTORE
\transpose{0} 						% <<< TRASPOSIZIONE #TONI (0 nullo)
\momenti{}							% <<< INSERISCI MOMENTI	
% momenti vanno separati da ; e vanno scelti tra:
% Ingresso; Atto penitenziale; Acclamazione al Vangelo; Dopo il Vangelo; Offertorio; Comunione; Ringraziamento; Fine; Santi; Pasqua; Avvento; Natale; Quaresima; Canti Mariani; Battesimo; Prima Comunione; Cresima; Matrimonio; Meditazione;
\ifchorded
	%\textnote{Tonalità originale }	% <<< EV COMMENTI (tonalità originale/migliore)
\fi


%%%%%% INTRODUZIONE
\ifchorded
\vspace*{\versesep}
\textnote{Intro: \qquad \qquad  }%(\eighthnote 116) % <<  MODIFICA IL TEMPO
% Metronomo: \eighthnote (ottavo) \quarternote (quarto) \halfnote (due quarti)
\vspace*{-\versesep}
\beginverse*

\nolyrics

%---- Prima riga -----------------------------
\vspace*{-\versesep}
\[E] \[*A] \[E] \[*A] 	 % \[*D] per indicare le pennate, \rep{2} le ripetizioni

%---- Ogni riga successiva -------------------
\vspace*{-\versesep}
\[E] \[B] \[E] \rep{2}

%---- Ev Indicazioni -------------------------			
%\textnote{\textit{(Oppure tutta la strofa)} }	

\endverse
\fi

%%%%% STROFA
\beginverse*		%Oppure \beginverse* se non si vuole il numero di fianco
\memorize 		% <<< DECOMMENTA se si vuole utilizzarne la funzione
%\chordsoff		% <<< DECOMMENTA se vuoi una strofa senza accordi
\[E]San\[*A]to, \[E]san\[*A]to, \[E]\[B]san\[E]to,
\[E]san\[*A]to, \[E]san\[*A]to, \[E]\[B]san\[E]to.
Il Si\[A]gnore Dio dell'uni\[E]verso,
il Si\[A]gnore Dio dell'uni\[E]verso,
i \[A]cieli e la \[E]terra \brk sono \[B]pieni della tua \[E]gloria.
\endverse


%%%%% RITORNELLO
\beginchorus
O\[A]sanna o\[E]sanna nell'\[B]alto dei \[E]cieli.
O\[A]sanna o\[E]sanna nell'\[B]alto dei \[E]cieli.
\endchorus

%%%%% STROFA
\beginverse*
^San^to, ^san^to, ^^san^to.
^San^to, ^san^to, ^^san^to.
Bene^detto colui che ^viene nel ^nome del Si^gnore. 
Bene^detto colui che ^viene nel ^nome del Si^gnore. 
\endverse


%%%%% RITORNELLO
\beginchorus
O\[A]sanna o\[E]sanna nell'\[B]alto dei \[E]cieli.
O\[A]sanna o\[E]sanna nell'\[B]alto dei \[E]cieli.
\endchorus



%%%%% STROFA
\beginverse*
^San^to, ^san^to, ^^san^to.
^San^to, ^san^to, ^^san^to.
\endverse

\endsong
%------------------------------------------------------------
%			FINE CANZONE
%------------------------------------------------------------


