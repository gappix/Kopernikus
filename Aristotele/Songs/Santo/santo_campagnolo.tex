%-------------------------------------------------------------
%			INIZIO	CANZONE
%-------------------------------------------------------------


%titolo: 	Santo Campagnolo
%autore: 	
%tonalita: 	Do Re 



%%%%%% TITOLO E IMPOSTAZONI
\beginsong{Santo Campagnolo}[by={}] 	% <<< MODIFICA TITOLO E AUTORE
\transpose{0} 						% <<< TRASPOSIZIONE #TONI (0 nullo)
\momenti{}							% <<< INSERISCI MOMENTI	
% momenti vanno separati da ; e vanno scelti tra:
% Ingresso; Atto penitenziale; Acclamazione al Vangelo; Dopo il Vangelo; Offertorio; Comunione; Ringraziamento; Fine; Santi; Pasqua; Avvento; Natale; Quaresima; Canti Mariani; Battesimo; Prima Comunione; Cresima; Matrimonio; Meditazione;
\ifchorded
	%\textnote{Tonalità originale }	% <<< EV COMMENTI (tonalità originale/migliore)
\fi


%%%%%% INTRODUZIONE
\ifchorded
\vspace*{\versesep}
\textnote{Intro: \qquad \qquad  }%(\eighthnote 116) % << MODIFICA IL TEMPO
% Metronomo: \eighthnote (ottavo) \quarternote (quarto) \halfnote (due quarti)
\vspace*{-\versesep}
\beginverse*

\nolyrics

%---- Prima riga -----------------------------
\vspace*{-\versesep}
\[C] \[A-]  \[D-7]	\[G] % \[*D] per indicare le pennate, \rep{2} le ripetizioni

%---- Ogni riga successiva -------------------
%\vspace*{-\versesep}
%\[G] \[C]  \[D]	

%---- Ev Indicazioni -------------------------			
%\textnote{\textit{(Oppure tutta la strofa)} }	

\endverse
\fi






%%%%% RITORNELLO
\beginchorus


\[C]Santo \[A-]santo \[D-7]santo il Si\[G]gnore 
Dio, dell'universo,
i cieli e la terra sono pieni della Tua 
gloria!
Osanna nell’alto dei cieli, 
osanna nell’alto dei cieli.
Benedetto colui che viene
nel nome del Signore! 
Osanna nell’alto dei cieli,
\endchorus







\endsong
%------------------------------------------------------------
%			FINE CANZONE
%------------------------------------------------------------




%++++++++++++++++++++++++++++++++++++++++++++++++++++++++++++
%			CANZONE TRASPOSTA
%++++++++++++++++++++++++++++++++++++++++++++++++++++++++++++
\ifchorded
%decremento contatore per avere stesso numero
\addtocounter{songnum}{-1} 
\beginsong{Santo Campagnolo}[by={}] 	% <<< COPIA TITOLO E AUTORE
\transpose{2} 						% <<< TRASPOSIZIONE #TONI + - (0 nullo)
\ifchorded
	\textnote{Tonalità alternativa}	% <<< EV COMMENTI (tonalità originale/migliore)
\fi


%%%%%% INTRODUZIONE
\ifchorded
\vspace*{\versesep}
\textnote{Intro: \qquad \qquad  }%(\eighthnote 116) % << MODIFICA IL TEMPO
% Metronomo: \eighthnote (ottavo) \quarternote (quarto) \halfnote (due quarti)
\vspace*{-\versesep}
\beginverse*

\nolyrics

%---- Prima riga -----------------------------
\vspace*{-\versesep}
\[C] \[A-]  \[D-7]	\[G] % \[*D] per indicare le pennate, \rep{2} le ripetizioni

%---- Ogni riga successiva -------------------
%\vspace*{-\versesep}
%\[G] \[C]  \[D]	

%---- Ev Indicazioni -------------------------			
%\textnote{\textit{(Oppure tutta la strofa)} }	

\endverse
\fi






%%%%% RITORNELLO
\beginchorus


\[C]Santo \[A-]santo \[D-7]santo il Si\[G]gnore 
Dio, dell'universo,
i cieli e la terra sono pieni della Tua 
gloria!
Osanna nell’alto dei cieli, 
osanna nell’alto dei cieli.
Benedetto colui che viene
nel nome del Signore! 
Osanna nell’alto dei cieli,
\endchorus







\endsong

\fi
%++++++++++++++++++++++++++++++++++++++++++++++++++++++++++++
%			FINE CANZONE TRASPOSTA
%++++++++++++++++++++++++++++++++++++++++++++++++++++++++++++
