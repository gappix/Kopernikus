%-------------------------------------------------------------
%			INIZIO	CANZONE
%-------------------------------------------------------------


%titolo: 	Come Te
%autore: 	Gen Rosso
%tonalita: 	Do



%%%%%% TITOLO E IMPOSTAZONI
\beginsong{Come Te}[by={Gen Rosso}] 	% <<< MODIFICA TITOLO E AUTORE
\transpose{0} 						% <<< TRASPOSIZIONE #TONI (0 nullo)
\momenti{Comunione; Ringraziamento; Natale; Avvento}							% <<< INSERISCI MOMENTI	
% momenti vanno separati da ; e vanno scelti tra:
% Ingresso; Atto penitenziale; Acclamazione al Vangelo; Dopo il Vangelo; Offertorio; Comunione; Ringraziamento; Fine; Santi; Pasqua; Avvento; Natale; Quaresima; Canti Mariani; Battesimo; Prima Comunione; Cresima; Matrimonio; Meditazione;
\ifchorded
	%\textnote{Tonalità originale }	% <<< EV COMMENTI (tonalità originale/migliore)
\fi


%%%%%% INTRODUZIONE
\ifchorded
\vspace*{\versesep}
\textnote{Intro: \qquad \qquad  }%(\eighthnote 116) % << MODIFICA IL TEMPO
% Metronomo: \eighthnote (ottavo) \quarternote (quarto) \halfnote (due quarti)
\vspace*{-\versesep}
\beginverse*

\nolyrics

%---- Prima riga -----------------------------
\vspace*{-\versesep}
\[C] \[G]  \[C]	 \[G] % \[*D] per indicare le pennate, \rep{2} le ripetizioni

%---- Ogni riga successiva -------------------
%\vspace*{-\versesep}
%\[G] \[C]  \[D]	

%---- Ev Indicazioni -------------------------			
%\textnote{\textit{(Oppure tutta la strofa)} }	

\endverse
\fi




%%%%% STROFA
\beginverse		%Oppure \beginverse* se non si vuole il numero di fianco
\memorize 		% <<< DECOMMENTA se si vuole utilizzarne la funzione
%\chordsoff		& <<< DECOMMENTA se vuoi una strofa senza accordi

\[C]Come \[G]Te, che sei \[C]sceso dal \[G]cielo \[C] \[G]
ad inse\[D-]gnarci l’a\[F]more di \[G]Dio
\[D-7] e hai preso su di \[A-]Te
la \[(*G)]nostra \[C]povera e \[F]fragile \[D-]umani\[G]tà.

\endverse




\beginverse*

^Come ^Te, che non ^ti sei te^nuto ^  ^
come se^greto l’a^more di ^Dio,
^ ma sei venuto ^qui
a ^rinno^vare la ^vita dell’^umani^tà.

\endverse




\beginverse*

\[G]Io non mi tirerò ind\[F]ietro
io non avrò più p\[C]aura
di dare tutto di \[G]me.  \[*F] \[*G]

\endverse


%%%%% RITORNELLO
\beginchorus
\textnote{\textbf{Rit.}}

\[C]Per am\[F]ore dell’\[C]uomo,\[F]
d’\[D-]ogni u\[F]omo come \[G]me
\[C]mi son \[F]fatto si\[C]lenzio\[F]
\[D-] per diven\[F]tare come \[G]Te.
\[C] Per \[G]amore \[A-]tuo 
\[G] mi farò \[F]servo d’ogni \[G]uomo che \[C]vive\[G]
\[A-] servo \[C]d’ogni u\[F]omo
\[C] per a\[G]mo\[F]re. \[G]

\endchorus






%%%%%% EV. INTERMEZZO
\beginverse*
\vspace*{1.3\versesep}
{
	\nolyrics
	\ifchorded

	\textnote{Intermezzo }

	%---- Prima riga -----------------------------
	\vspace*{-\versesep}
	\[C] \[G]  \[C]	 \[G]


	\fi
	%---- Ev Indicazioni -------------------------			
	%\textnote{\textit{(ripetizione della strofa)}} 
	 
}
\vspace*{\versesep}
\endverse




%%%%% STROFA
\beginverse

^Come ^Te che hai la^sciato le ^stelle ^ ^
per farti ^proprio come ^uno di ^noi,
^ senza tenere ^niente
hai ^dato ^anche la ^vita, hai pa^gato per ^noi.

\endverse




\beginverse*

\[G]Davanti a questo mis\[F]tero
come potrò ricam\[C]biare,
che cosa mai potrò f\[G]are?  \[*F]\[*G]

\endverse









\endsong
%------------------------------------------------------------
%			FINE CANZONE
%------------------------------------------------------------
