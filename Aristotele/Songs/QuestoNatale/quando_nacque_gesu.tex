%-------------------------------------------------------------
%			INIZIO	CANZONE
%-------------------------------------------------------------

%titolo: Quando Nacque Gesù
%autore: Canto popolare
%tonalita: Re- (e La-)


%%%%%% TITOLO E IMPOSTAZONI
\beginsong{Quando nacque Gesuuuuuuuù}[by={Greensleeves — Canto popolare natalizio}] 	% <<< MODIFICA TITOLO E AUTORE
\transpose{-5} 			% <<< TRASPOSIZIONE #TONI (0 nullo)
\momenti{Natale}		% <<< INSERISCI MOMENTI



%%%%%% INTRODUZIONE
\ifchorded
\vspace*{\versesep}
\textnote{Intro: \qquad \qquad  }%(\eighthnote 116) % << MODIFICA IL TEMPO
\vspace*{-\versesep}
\beginverse*

\nolyrics

%---- Prima riga -----------------------------
\vspace*{-\versesep}
\[D-] 

%---- Ev Indicazioni -------------------------			
\textnote{\textit{(Oppure tutta la strofa)} }	
	 
\endverse
\fi



%%%%% STROFA
\beginverse		%Oppure \beginverse* se non si vuole il numero di fianco
\memorize 		% <<< DECOMMENTA se si vuole utilizzarne la funzione

Un \[D-]bimbo è \[F]nato a Be\[C]tlem\[A-]me,
un bam\[D-]bino è nato per \[A]noi!
Ri\[D-]posa qu\[F]ieto su \[C]paglia e \[A-]fien
e Ma\[D-]ria lo \[A]culla se\[D-]ren.


\endverse




%%%%% RITORNELLO

\beginchorus

\[F]Gloria, gloria, Alle\[C]lu\[A-]ia!
Un bam\[D-]bino è nato per \[A]noi.
\[F]Gloria, gloria, Alle\[C]lu\[A-]ia! 
Oggi è \[D-]nato il \[A]Cristo \[D-]Gesù!

\endchorus





%%%%%% INTERMEZZO

\beginverse*
\vspace*{1.3\versesep}
{
	\textnote{Ev. intermezzo strumentale}
	\textnote{\textit{(ripetizione della strofa)}} 
	 
}
\endverse



%%%%% STROFA
\beginverse		%Oppure \beginverse* se non si vuole il numero di fianco
%\memorize 		% <<< DECOMMENTA se si vuole utilizzarne la funzione

Nel ^cielo gli ^angeli ^can^tano:
"su cor^rete tutti a Bet^lemme,
vi ^trove^rete su ^paglia e ^fien
il Si^gnore il ^Cristo Ge^sù".


\endverse




\endsong
%------------------------------------------------------------
%			FINE CANZONE
%------------------------------------------------------------