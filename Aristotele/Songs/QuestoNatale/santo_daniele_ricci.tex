%-------------------------------------------------------------
%			INIZIO	CANZONE
%-------------------------------------------------------------

%titolo: 	Santo Ricci
%autore: 	Daniele Ricci
%tonalita: 	Sol 


%%%%%% TITOLO E IMPOSTAZONI
\beginsong{Santo Ricci}[by={Daniele Ricci}] 	% <<< MODIFICA TITOLO E AUTORE
\transpose{0} 						% <<< TRASPOSIZIONE #TONI (0 nullo)
\momenti{}							% <<< INSERISCI MOMENTI




%%%%%% INTRODUZIONE
\ifchorded
\beginverse*
\vspace*{-0.5\versesep}
{
	\nolyrics
	\textbf{Intro:} \qquad \qquad % (\eighthnote 116) % << MODIFICA IL TEMPO
	\vspace*{-\versesep}

	%---- Prima riga -----------------------------
	\[G] \[D]  \[A-]	%\textbf{x2}

	%---- Ogni riga successiva -------------------
	\vspace*{-\versesep}
	\[G] \[D]  \[A-]	%\textbf{x2}


}
\vspace*{-0.3\versesep}
\endverse
\fi



%%%%% STROFA
\beginverse*	%Oppure \beginverse* se non si vuole il numero di fianco
\memorize 		% <<< DECOMMENTA se si vuole utilizzarne la funzione
%\chordsoff		& <<< DECOMMENTA se vuoi una strofa senza accordi

\[G]Santo santo \[D]santo - \[A-]o
\echo{è il Signore Dio dell'universo}
^Santo santo ^santo - ^o 
\echo{è il Signore Dio dell'universo}
^Santo santo ^santo - ^o 
\echo{i cieli e la terra}
\[G]Santo santo \[D]santo - \[A-]o
\echo{sono pieni della tua gloria}
\[G]Osanna osanna \[D]osanna - \[A-]a
\echo{nell'alto dei cieli}


\endverse


\beginverse*
\[G]Benedetto colui che vi\[D]ene
nel \[C]nome \[D]del Si\[C]gno\[D]re
\endverse

\beginverse*	

^Osanna osanna ^osanna - ^a
\echo{nell'alto dei cieli}
^Osanna osanna ^osanna - ^a
\echo{nell'alto dei cieli}
^Osanna osanna ^osanna - ^a
\echo{nell'alto dei cieli}
\[G]Osanna-a...

\endverse

\endsong
%------------------------------------------------------------
%			FINE CANZONE
%------------------------------------------------------------
