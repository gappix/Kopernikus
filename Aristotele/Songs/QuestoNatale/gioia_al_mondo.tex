%-------------------------------------------------------------
%			INIZIO	CANZONE
%-------------------------------------------------------------


%titolo: 	Gioia al mondo
%autore: 	G. F. Handel
%tonalita: 	Do 



%%%%%% TITOLO E IMPOSTAZONI
\beginsong{Gioia al mondo}[by={Joy to the world! — G. F. Handel}] 	% <<< MODIFICA TITOLO E AUTORE
\transpose{0} 						% <<< TRASPOSIZIONE #TONI (0 nullo)
\momenti{Natale}					% <<< INSERISCI MOMENTI




%%%%%% INTRODUZIONE
\ifchorded
\beginverse*
\vspace*{-0.5\versesep}
{
	\nolyrics
	Intro: \qquad \qquad % (\eighthnote 116) % << MODIFICA IL TEMPO
	\vspace*{-\versesep}

	%---- Prima riga -----------------------------
	\[C] \[C] \textoverline{\[F]} \textoverline{\[G]} \[C]	%\textbf{x2}

	%---- Ogni riga successiva -------------------
	\vspace*{-\versesep}
	\[F] \[G]  \[C]	%\textbf{x2}

	%---- Ev Indicazioni -------------------------			
	\vspace{-\versesep}\textit{(Prime due righe)} 	
	 
}
\vspace*{-0.3\versesep}
\endverse
\fi



%%%%% STROFA
\beginverse		%Oppure \beginverse* se non si vuole il numero di fianco
\memorize 		% <<< DECOMMENTA se si vuole utilizzarne la funzione
%\chordsoff		& <<< DECOMMENTA se vuoi una strofa senza accordi

\[C]Gioia al mondo il \[F]Cre\[G]a\[C]tor,
è \[F]nato in \[G]mezzo a \[C]noi!
I \[C]cuori allor si aprono
con \[C]gioia alla sua luce
e i \[C]cieli e terra sa\[G]ran 
una lode all'unico re,
che go\[F]verna ogni \[C]popolo nei \[F]se\[G]co\[C]li.

\endverse



%%%%% STROFA
\beginverse		%Oppure \beginverse* se non si vuole il numero di fianco
%\memorize 		% <<< DECOMMENTA se si vuole utilizzarne la funzione
%\chordsoff		% <<< DECOMMENTA se vuoi una strofa senza accordi

^Lui porta a noi la ^ve^ri^tà, 
il ^mondo ^salve^rà! 
La^sciate che ogni uomo 
in^vochi il salvatore 
il ^cielo splende^rà 
e la terra esulterà 
della ^grazia di^vina del ^Sal^va^tor!

\endverse



%%%%% STROFA
\beginverse		%Oppure \beginverse* se non si vuole il numero di fianco
%\memorize 		% <<< DECOMMENTA se si vuole utilizzarne la funzione
%\chordsoff		% <<< DECOMMENTA se vuoi una strofa senza accordi

^Oggi è nato il ^re^den^tor, 
la ^terra e^sulte^rà! 
La^sciate che ogni cuore 
gli ^faccia un po' di posto
il ^cielo splende^rà, 
e la terra gioirà 
della ^grazia di^vina del ^Sal^va^tor!

\endverse


Finale:

%%%%% STROFA
\beginverse*	%Oppure \beginverse* se non si vuole il numero di fianco
%\memorize 		% <<< DECOMMENTA se si vuole utilizzarne la funzione
%\chordsoff		% <<< DECOMMENTA se vuoi una strofa senza accordi

della ^grazia di^vina del ^Sal^va^tor!

\endverse



\endsong
%------------------------------------------------------------
%			FINE CANZONE
%------------------------------------------------------------