%-------------------------------------------------------------
%			INIZIO	CANZONE
%-------------------------------------------------------------


%titolo: 	Gioia al mondo
%autore: 	G. F. Handel
%tonalita: 	Do 



%%%%%% TITOLO E IMPOSTAZONI
\beginsong{Gioia al mondo}[by={Joy to the world! — G. F. Handel}] 	% <<< MODIFICA TITOLO E AUTORE
\transpose{0} 						% <<< TRASPOSIZIONE #TONI (0 nullo)
\momenti{Natale}					% <<< INSERISCI MOMENTI



%%%%%% INTRODUZIONE
\ifchorded
\vspace*{\versesep}
\textnote{Intro: \qquad \qquad  }%(\eighthnote 116) % << MODIFICA IL TEMPO
% Metronomo: \eighthnote (ottavo) \quarternote (quarto) \halfnote (due quarti)
\vspace*{-\versesep}
\beginverse*

\nolyrics

%---- Prima riga -----------------------------
\vspace*{-\versesep}
\[C] \[C] \[*F] \[*G] \[*C] % \[*D] per indicare le pennate, \rep{2} le ripetizioni

%---- Ogni riga successiva -------------------
\vspace*{-\versesep}
\[F] \[G]  \[C]	

%---- Ev Indicazioni -------------------------			
\textnote{\textit{(Come le prime due righe)} }	

\endverse
\fi




%%%%% STROFA
\beginverse		%Oppure \beginverse* se non si vuole il numero di fianco
\memorize 		% <<< DECOMMENTA se si vuole utilizzarne la funzione
%\chordsoff		& <<< DECOMMENTA se vuoi una strofa senza accordi

\[C]Gioia al mondo il \[*F]Cre\[*G]a\[*C]tor,
è \[F]nato in \[G]mezzo a \[C]noi!
I \[C]cuori allor si aprono
con \[C]gioia alla sua luce
e i \[C]cieli e terra sa\[C]ran 
una \[G]lode all'unico \[G]re,
che go\[F]verna ogni \[C]popolo nei \[*F]se\[*G]co\[*C]li.

\endverse



%%%%%% EV. INTERMEZZO
\ifchorded
\beginverse*
\vspace*{1.3\versesep}
{	

	\nolyrics
	\textnote{Intermezzo strumentale}
	

	%---- Prima riga -----------------------------
	\vspace*{-\versesep}
	\[C] \[C] \[*F] \[*G] \[*C] 

	%---- Ogni riga successiva -------------------
	\vspace*{-\versesep}
	\[F] \[G]  \[C]	
	 
}
\vspace*{\versesep}
\endverse
\fi


%%%%% STROFA
\beginverse		%Oppure \beginverse* se non si vuole il numero di fianco
%\memorize 		% <<< DECOMMENTA se si vuole utilizzarne la funzione
%\chordsoff		% <<< DECOMMENTA se vuoi una strofa senza accordi

^Lui porta a noi la ^ve^ri^tà, 
il ^mondo ^salve^rà! 
La^sciate che ogni uomo 
in^vochi il salvatore 
il ^cielo splende^rà 
e la ^terra esulte^rà 
della ^grazia di^vi - i - na del ^Sal^va^tor!

\endverse



%%%%%% EV. INTERMEZZO
\ifchorded
\beginverse*
\vspace*{1.3\versesep}
{	

	\nolyrics
	\textnote{Intermezzo strumentale}
	

	%---- Prima riga -----------------------------
	\vspace*{-\versesep}
	\[C] \[C] \[*F] \[*G] \[*C] 

	%---- Ogni riga successiva -------------------
	\vspace*{-\versesep}
	\[F] \[G]  \[C]	
	 
}
\vspace*{\versesep}
\endverse
\fi



%%%%% STROFA
\beginverse		%Oppure \beginverse* se non si vuole il numero di fianco
%\memorize 		% <<< DECOMMENTA se si vuole utilizzarne la funzione
%\chordsoff		% <<< DECOMMENTA se vuoi una strofa senza accordi

^Oggi è nato il ^re^den^tor, 
la ^terra e^sulte^rà! 
La^sciate che ogni cuore 
gli ^faccia un po' di posto
il ^cielo splende^rà, 
e la ^terra gioi^rà 
della ^grazia di^vi - i - na del ^Sal^va^tor!

\endverse



%%%%%%FINALE

\beginchorus %oppure \beginverse*
\vspace*{1.3\versesep}
\textnote{Finale \textit{(rallentando)}} %<<< EV. INDICAZIONI

della \[F]grazia di\[C]vi - i - na del \[*F]Sal\[*G]va\[*C]tor!

\endchorus  %oppure \endverse






\endsong
%------------------------------------------------------------
%			FINE CANZONE
%------------------------------------------------------------