%-------------------------------------------------------------
%			INIZIO	CANZONE
%-------------------------------------------------------------


%titolo: 	È bello lodart
%autore: 	Gen Verde
%tonalita: 	Sol 



%%%%%% TITOLO E IMPOSTAZONI
\beginsong{È bello lodarti}[by={Gen Verde}] 	% <<< MODIFICA TITOLO E AUTORE
\transpose{0} 						% <<< TRASPOSIZIONE #TONI (0 nullo)
\momenti{Ingresso; Fine}							% <<< INSERISCI MOMENTI	
% momenti vanno separati da ; e vanno scelti tra:
% Ingresso; Atto penitenziale; Acclamazione al Vangelo; Dopo il Vangelo; Offertorio; Comunione; Ringraziamento; Fine; Santi; Pasqua; Avvento; Natale; Quaresima; Canti Mariani; Battesimo; Prima Comunione; Cresima; Matrimonio; Meditazione;
\ifchorded
	%\textnote{Tonalità originale }	% <<< EV COMMENTI (tonalità originale/migliore)
\fi



%%%%%% INTRODUZIONE
\ifchorded
\vspace*{\versesep}
\textnote{Intro: \qquad \qquad  }%(\eighthnote 116) % << MODIFICA IL TEMPO
% Metronomo: \eighthnote (ottavo) \quarternote (quarto) \halfnote (due quarti)
\vspace*{-\versesep}
\beginverse*

\nolyrics

%---- Prima riga -----------------------------
\vspace*{-\versesep}
\[G] \[D]  \[C] \[C]	 \rep{2} % \[*D] per indicare le pennate, \rep{2} le ripetizioni

%---- Ogni riga successiva -------------------
%\vspace*{-\versesep}
%\[G] \[C]  \[D]	

%---- Ev Indicazioni -------------------------			
%\textnote{\textit{(Oppure tutta la strofa)} }	

\endverse
\fi





%%%%% RITORNELLO
\beginchorus
\textnote{\textbf{Rit.}}
\[G] È \[D]bello can\[C]tare il tuo a\[G]more,
\[A-7] è bello lo\[G]dare il tuo \[D]nome.
\[G] È bello can\[B4]tare il tuo a\[C]more,
è \[G]bello lo\[D]darti, Si\[C]gnore,
è \[G]bello can\[D]tare a \[C]te!  \rep{2}
\endchorus



%%%%% STROFA
\beginverse		%Oppure \beginverse* se non si vuole il numero di fianco
\memorize 		% <<< DECOMMENTA se si vuole utilizzarne la funzione
%\chordsoff		% <<< DECOMMENTA se vuoi una strofa senza accordi
\[E-]Tu che sei l'amore infi\[B-6]nito
che nep\[C]pure il cielo può contenere,
ti \[A-]sei fatto \[7]uomo, \[D6]Tu sei venuto qui
ad \[B7]abitare in mezzo a \[C]noi, allora\dots  
\endverse




%%%%% STROFA
\beginverse		%Oppure \beginverse* se non si vuole il numero di fianco
%\memorize 		% <<< DECOMMENTA se si vuole utilizzarne la funzione
%\chordsoff		% <<< DECOMMENTA se vuoi una strofa senza accordi
^Tu che conti tutte le ^stelle
e le ^chiami una ad una per nome,
da ^mille sen^tieri ^ci hai radunati qui,
^ci hai chiamati figli ^tuoi, allora\dots 
\endverse


%%%%%% EV. FINALE

\beginchorus %oppure \beginverse*
\vspace*{1.3\versesep}
\textnote{Finale \textit{(rallentando un poco)}} %<<< EV. INDICAZIONI

è \[G]bello can\[D]tare a \[C]te!  \[C] \[G]

\endchorus  %oppure \endverse

\endsong
%------------------------------------------------------------
%			FINE CANZONE
%------------------------------------------------------------