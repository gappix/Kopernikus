%-------------------------------------------------------------
%			INIZIO	CANZONE
%-------------------------------------------------------------


%titolo: 	Alleluia Venite a me
%autore: 	Orlandini, Herrera
%tonalita: 	Re



%%%%%% TITOLO E IMPOSTAZONI
\beginsong{Alleluia Venite a me}[by={G. Orlandini, J. Herrera}] 	% <<< MODIFICA TITOLO E AUTORE
\transpose{0} 						% <<< TRASPOSIZIONE #TONI (0 nullo)
\momenti{Acclamazione al Vangelo;}							% <<< INSERISCI MOMENTI	
% momenti vanno separati da ; e vanno scelti tra:
% Ingresso; Atto penitenziale; Acclamazione al Vangelo; Dopo il Vangelo; Offertorio; Comunione; Ringraziamento; Fine; Santi; Pasqua; Avvento; Natale; Quaresima; Canti Mariani; Battesimo; Prima Comunione; Cresima; Matrimonio; Meditazione; Spezzare del pane;
\ifchorded
	%\textnote{Tonalità migliore }	% <<< EV COMMENTI (tonalità originale/migliore)
\fi


%%%%%% INTRODUZIONE
\ifchorded
\vspace*{\versesep}
\musicnote{
\begin{minipage}{0.48\textwidth}
\textbf{Intro}
\hfill 
%( \eighthnote \, 80)   % <<  MODIFICA IL TEMPO
% Metronomo: \eighthnote (ottavo) \quarternote (quarto) \halfnote (due quarti)
\end{minipage}
} 	
\vspace*{-\versesep}
\beginverse*

\nolyrics

%---- Prima riga -----------------------------
\vspace*{-\versesep}
\[D] \[G*] \[D]	 \rep{2} % \[*D] per indicare le pennate, \rep{2} le ripetizioni

%---- Ogni riga successiva -------------------
%\vspace*{-\versesep}
%\[G] \[C]  \[D]	

%---- Ev Indicazioni -------------------------			
%\textnote{\textit{(Oppure tutta la strofa)} }	

\endverse
\fi



%%%%% STROFA
\beginverse		%Oppure \beginverse* se non si vuole il numero di fianco
\memorize 		% <<< DECOMMENTA se si vuole utilizzarne la funzione
%\chordsoff		% <<< DECOMMENTA se vuoi una strofa senza accordi
Ve\[D]nite \[G*]a \[D]me: allel\[A]u\[D]ia!
Cre\[D]dete \[G*]in \[D]me: allel\[A]u\[D]ia!
Io \[G]sono la \[A]via, la \[F#-]veri\[B-]tà: 
allel\[D]u\[G]ia, alle\[A7]lu\[D]ia! \quad  \[G*] \[D] \[G*] \[D]
\endverse

\beginverse

Res^tate ^in ^me: alle^lu^ia! 
Vi^vete ^in ^me: alle^lu^ia!
Io ^sono la ^vita, ^la santi^tà, 
alle^lu^ia, alle^lu^ia. \quad  \[G*] \[D] \[G*] \[D]
\endverse

\beginverse

Can^tate ^con ^me: alle^lu^ia! 
Dan^zate ^con ^me: alle^lu^ia!
Io ^sono la ^gioia, ^la liber^tà: 
alle^lu^ia, alle^lu^ia. \quad  \[G*] \[D] \[G*] \[D]

\endverse


\endsong
%------------------------------------------------------------
%			FINE CANZONE
%------------------------------------------------------------

