%-------------------------------------------------------------
%			INIZIO	CANZONE
%-------------------------------------------------------------


%titolo: 	Alleluia lodate il Signore
%autore: 	Frisina
%tonalita: 	RE



%%%%%% TITOLO E IMPOSTAZONI
\beginsong{Alleluia Lodate il Signore}[by={Frisina}] 	% <<< MODIFICA TITOLO E AUTORE
\transpose{0} 						% <<< TRASPOSIZIONE #TONI (0 nullo)
\momenti{Acclamazione al Vangelo;}							% <<< INSERISCI MOMENTI	
% momenti vanno separati da ; e vanno scelti tra:
% Ingresso; Atto penitenziale; Acclamazione al Vangelo; Dopo il Vangelo; Offertorio; Comunione; Ringraziamento; Fine; Santi; Pasqua; Avvento; Natale; Quaresima; Canti Mariani; Battesimo; Prima Comunione; Cresima; Matrimonio; Meditazione;
\ifchorded
	%\textnote{Tonalità originale }	% <<< EV COMMENTI (tonalità originale/migliore)
\fi




%%%%%% INTRODUZIONE
\ifchorded
\vspace*{\versesep}
\textnote{Intro: \qquad \qquad  }%(\eighthnote 116) % << MODIFICA IL TEMPO
% Metronomo: \eighthnote (ottavo) \quarternote (quarto) \halfnote (due quarti)
\vspace*{-\versesep}
\beginverse*

\nolyrics

%---- Prima riga -----------------------------
\vspace*{-\versesep}
\[A] \[G] \[D] \[E-] \[B-] % \[*D] per indicare le pennate, \rep{2} le ripetizioni

%---- Ogni riga successiva -------------------
\vspace*{-\versesep}
\[G] \[A]  \[D]	 \[A]  

%---- Ev Indicazioni -------------------------			
\textnote{\textit{(Come le prime due righe)} }	

\endverse
\fi





%%%%% RITORNELLO
\beginchorus
\textnote{\textbf{Rit.}}
\[A]Alle\[G]lu\[D]ia alle\[E-]lu\[B-]ia, 
lo\[G]date il Si\[A]gno\[D]re \[A]
alle\[G]lu\[D]ia alle\[E-]lu\[B-]ia, 
lo\[G]date il Si\[A]gno\[D]re.
\endchorus


%%%%% STROFA
\beginverse
Lo\[B&]date il Si\[F]gnore nel \[B&]suo tempio \[F]santo
lo\[G-]datelo nell'\[D-]alto firma\[C]mento
lo\[B&]datelo nei \[F]grandi pro\[B&]digi del suo a\[F]more
lo\[G]datene l'ec\[A-]celsa sua \[E]mae\[A]stà.
\endverse


%%%%% STROFA
\beginverse
\chordsoff
Lodatelo col suono gioioso delle trombe,
lodatelo sull'arpa e sulla cetra.
Lodatelo col suono dei timpani e dei sistri,
lodatelo coi flauti e sulle corde. 
\endverse


%%%%% STROFA
\beginverse
\chordsoff
Lodatelo col suono dei cimbali sonori,
lodatelo con cimbali squillanti.
Lodate il Signore voi tutte creature,
Lodate e cantate al Signore. 
\endverse


%%%%% STROFA
\beginverse
\chordsoff
Lodate voi tutti suoi angeli dei cieli,
Lodatelo voi tutte sue schiere.
Lodatelo voi cieli, voi astri e voi stelle,
lodate il Signore Onnipotente. 
\endverse


%%%%% STROFA
\beginverse
\chordsoff
Voi tutti governanti e genti della terra,
lodate il nome santo del Signore.
Perché solo la sua gloria risplende sulla terra,
lodate e benedite il Signore. 
\endverse
\endsong
%------------------------------------------------------------
%			FINE CANZONE
%------------------------------------------------------------


