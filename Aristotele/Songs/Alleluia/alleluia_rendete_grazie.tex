%-------------------------------------------------------------
%			INIZIO	CANZONE
%-------------------------------------------------------------


%titolo: 	Alleluia Rendete grazie
%autore: 	Gen Verde
%tonalita: 	Sib


%%%%%% TITOLO E IMPOSTAZONI
\beginsong{Alleluia Rendete grazie}[by={Gen Verde}] 	% <<< MODIFICA TITOLO E AUTORE
\transpose{0} 						% <<< TRASPOSIZIONE #TONI (0 nullo)
\momenti{Acclamazione al Vangelo;}							% <<< INSERISCI MOMENTI	
% momenti vanno separati da ; e vanno scelti tra:
% Ingresso; Atto penitenziale; Acclamazione al Vangelo; Dopo il Vangelo; Offertorio; Comunione; Ringraziamento; Fine; Santi; Pasqua; Avvento; Natale; Quaresima; Canti Mariani; Battesimo; Prima Comunione; Cresima; Matrimonio; Meditazione;
\ifchorded
	%\textnote{Tonalità originale }	% <<< EV COMMENTI (tonalità originale/migliore)
\fi


%%%%%% INTRODUZIONE
\ifchorded
\vspace*{\versesep}
\musicnote{
\begin{minipage}{0.48\textwidth}
\textbf{Intro}
\hfill 
%( \eighthnote \, 80)   % <<  MODIFICA IL TEMPO
% Metronomo: \eighthnote (ottavo) \quarternote (quarto) \halfnote (due quarti)
\end{minipage}
} 	
\vspace*{-\versesep}
\beginverse*

\nolyrics

%---- Prima riga -----------------------------
\vspace*{-\versesep}
\[B&] \[C] \[D-] \[A-]	 % \[*D] per indicare le pennate, \rep{2} le ripetizioni

%---- Ogni riga successiva -------------------
\vspace*{-\versesep}
\[B&] \[C] \[F] \[F]

%---- Ev Indicazioni -------------------------			
%\textnote{\textit{(Con stop e attacco solo voce)} }	

\endverse
\fi





%%%%% RITORNELLO
\beginchorus
\textnote{\textbf{Rit.}}
Alle\[B&]luia, alle\[C]luia, alle\[D-]luia, \[A-]
alle\[B&]luia, alle\[C4]l\[C]u\[F]ia. \[A-]
Alle\[B&]luia, alle\[C]luia, alle\[D-]luia, \[A-]
alle\[B&]luia, alle\[C4]l\[C]u\[F]ia. \[F]

\endchorus




%%%%% STROFA
\beginverse		%Oppure \beginverse* se non si vuole il numero di fianco
\memorize 		% <<< DECOMMENTA se si vuole utilizzarne la funzione
%\chordsoff		& <<< DECOMMENTA se vuoi una strofa senza accordi
Ren\[B&]dete grazie a \[G-]Dio, Egli è bu\[C]ono,
e\[A7]terno e fedele è il suo a\[D-]more.
\[B&]Sì, è così: lo \[F]dica Israele,
\[G-]dica che il suo a\[F]more è per \[E&]sempre. \[C]
\endverse





%%%%% STROFA
\beginverse		%Oppure \beginverse* se non si vuole il numero di fianco
%\memorize 		% <<< DECOMMENTA se si vuole utilizzarne la funzione
\chordsoff		% <<< DECOMMENTA se vuoi una strofa senza accordi
La destra del Signore si è innalzata
a compiere grandiose meraviglie.
Non morirò, ma resterò in vita
e annuncerò i prodigi del Signore.
\endverse




%%%%% STROFA
\beginverse		%Oppure \beginverse* se non si vuole il numero di fianco
%\memorize 		% <<< DECOMMENTA se si vuole utilizzarne la funzione
\chordsoff		% <<< DECOMMENTA se vuoi una strofa senza accordi
La pietra che avevano scartato
è divenuta pietra angolare.
Questo prodigio ha fatto il Signore,
una meraviglia ai nostri occhi.
\endverse




\endsong
%------------------------------------------------------------
%			FINE CANZONE
%------------------------------------------------------------


