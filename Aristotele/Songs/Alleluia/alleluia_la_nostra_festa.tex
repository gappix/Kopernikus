%-------------------------------------------------------------
%			INIZIO	CANZONE
%-------------------------------------------------------------


%titolo: 	Alleluia la nostra festa
%autore: 	Varnavà
%tonalita: 	Do

\beginsong{Alleluia La nostra festa }[ititle={Alleluia delle lampadine}, by={Alleluia delle lampadine — S. Varnavà}]
\transpose{0} 						% <<< TRASPOSIZIONE #TONI (0 nullo)
\momenti{Acclamazione al Vangelo}							% <<< INSERISCI MOMENTI	
% momenti vanno separati da ; e vanno scelti tra:
% Ingresso; Atto penitenziale; Acclamazione al Vangelo; Dopo il Vangelo; Offertorio; Comunione; Ringraziamento; Fine; Santi; Pasqua; Avvento; Natale; Quaresima; Canti Mariani; Battesimo; Prima Comunione; Cresima; Matrimonio; Meditazione;
\ifchorded
	%\textnote{Tonalità originale }	% <<< EV COMMENTI (tonalità originale/migliore)
\fi


%%%%%% INTRODUZIONE
\ifchorded
\vspace*{\versesep}
\textnote{Intro: \qquad \qquad  }%(\eighthnote 116) % << MODIFICA IL TEMPO
% Metronomo: \eighthnote (ottavo) \quarternote (quarto) \halfnote (due quarti)
\vspace*{-\versesep}
\beginverse*

\nolyrics

%---- Prima riga -----------------------------
\vspace*{-\versesep}
\[C] \[F]  \[C] \[G]	 % \[*D] per indicare le pennate, \rep{2} le ripetizioni

%---- Ogni riga successiva -------------------
\vspace*{-\versesep}
\[C]\[F] \[*C] \[*G] \[C]  

%---- Ev Indicazioni -------------------------			
\textnote{\textit{(Come il ritornello)} }	

\endverse
\fi




%%%%% RITORNELLO
\beginchorus

\[C]Alle\[F]luia, alleluia, \[C]alle\[G]luia, alleluia,
\[C]alle\[F]luia, alleluia, \[*C]al\[*G]lelu\[C]ia. \rep{2}

\endchorus


%%%%% STROFA
\beginverse*

\[A-]La nostra \[D-]festa non \[G7]deve fi\[C]nire
non \[A-]deve fi\[D-]nire e \[G7]non fini\[C]rà. \rep{2}

\endverse
\beginverse*

Per\[F]ché la \[G]festa \[A-]siamo \[E-]noi
che \[F]cammi\[G]niamo verso \[C]Te, \[C7]
per\[F]ché la \[G]festa \[A-]siamo \[E-]noi
can\[D]tando in\[7]sieme co\[G7]sì:

\endverse


%%%%% RITORNELLO
\beginchorus

\[C]Alle\[F]luia, alleluia, \[C]alle\[G]luia, alleluia,
\[C]alle\[F]luia, alleluia, \[*C]al\[*G]lelu\[C]ia. \rep{2}

\endchorus


\endsong
%------------------------------------------------------------
%			FINE CANZONE
%------------------------------------------------------------


