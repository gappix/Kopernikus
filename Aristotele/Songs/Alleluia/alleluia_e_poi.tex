%-------------------------------------------------------------
%			INIZIO	CANZONE
%-------------------------------------------------------------


%titolo: 	Alleluia e poi
%autore: 	Luca Diliberto, Giuliana Monti
%tonalita: 	Do 


%%%%%% TITOLO E IMPOSTAZONI
\beginsong{Alleluia E poi}[by={L. Diliberto, G. Monti}]
\transpose{0} 						% <<< TRASPOSIZIONE #TONI (0 nullo)
\momenti{Acclamazione al Vangelo}							% <<< INSERISCI MOMENTI	
% momenti vanno separati da ; e vanno scelti tra:
% Ingresso; Atto penitenziale; Acclamazione al Vangelo; Dopo il Vangelo; Offertorio; Comunione; Ringraziamento; Fine; Santi; Pasqua; Avvento; Natale; Quaresima; Canti Mariani; Battesimo; Prima Comunione; Cresima; Matrimonio; Meditazione;
\ifchorded
	%\textnote{Tonalità originale }	% <<< EV COMMENTI (tonalità originale/migliore)
\fi




%%%%%% INTRODUZIONE
\ifchorded
\vspace*{\versesep}
\musicnote{
\begin{minipage}{0.48\textwidth}
\textbf{Intro:}
\hfill 
%( \eighthnote \, 80)   % <<  MODIFICA IL TEMPO
% Metronomo: \eighthnote (ottavo) \quarternote (quarto) \halfnote (due quarti)
\end{minipage}
} 	
\vspace*{-\versesep}
\beginverse*

\nolyrics

%---- Prima riga -----------------------------
\vspace*{-\versesep}
\[C] \[G]  \[C]	 % \[*D] per indicare le pennate, \rep{2} le ripetizioni

%---- Ogni riga successiva -------------------
%\vspace*{-\versesep}
%\[G] \[C]  \[D]	

%---- Ev Indicazioni -------------------------			
%\textnote{\textit{(Oppure tutta la strofa)} }	

\endverse
\fi



%%%%% STROFA
\beginverse		%Oppure \beginverse* se non si vuole il numero di fianco
\memorize 		% <<< DECOMMENTA se si vuole utilizzarne la funzione
%\chordsoff		& <<< DECOMMENTA se vuoi una strofa senza accordi
\[C]Chiama, ed \[G]io ver\[A-]rò da \[E-]Te:
\[F]Figlio, nel si\[C]lenzio, mi \[D]accoglie\[G]rai.
\[C]Voce e \[G]{poi\dots} la \[A-]liber\[E-]tà,
\[F]nella Tua Pa\[C]rola cam\[D7]mine\[G]rò.
\endverse



%%%%% RITORNELLO
\beginchorus

\[C]Alleluia, \[G]alleluia, \[A-]allelu\[E-]ia,
\[F]alleluia, \[C]alle\[D7]lu\[G]ia,
\[C]Alleluia, \[G]alleluia, \[A-]allelu\[E-]ia,
\[F]alleluia, \[C]alle\[G]lu\[C]ia.

\endchorus




%%%%% STROFA
\beginverse		%Oppure \beginverse* se non si vuole il numero di fianco
%\memorize 		% <<< DECOMMENTA se si vuole utilizzarne la funzione
%\chordsoff		& <<< DECOMMENTA se vuoi una strofa senza accordi

^Danza, ed ^io ver^rò con ^Te:
^Figlio, la Tua ^strada com^prende^rò.
^Luce, e ^poi, nel ^tempo ^tuo,
^oltre il desi^derio ri^pose^rò. 
\endverse



\endsong
%------------------------------------------------------------
%			FINE CANZONE
%------------------------------------------------------------
