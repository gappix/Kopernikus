%-------------------------------------------------------------
%			INIZIO	CANZONE
%-------------------------------------------------------------


%titolo: 	Alleluia Passeranno i cieli
%autore: 	Costa, Varnavà
%tonalita: Fa 
\beginsong{Alleluia Passeranno i cieli}[by={E. Costa, S. Varnavà}]
\transpose{3} 						% <<< TRASPOSIZIONE #TONI (0 nullo)
\momenti{Acclamazione al Vangelo}							% <<< INSERISCI MOMENTI	
% momenti vanno separati da ; e vanno scelti tra:
% Ingresso; Atto penitenziale; Acclamazione al Vangelo; Dopo il Vangelo; Offertorio; Comunione; Ringraziamento; Fine; Santi; Pasqua; Avvento; Natale; Quaresima; Canti Mariani; Battesimo; Prima Comunione; Cresima; Matrimonio; Meditazione;
\ifchorded
	%\textnote{Tonalità originale }	% <<< EV COMMENTI (tonalità originale/migliore)
\fi





%%%%%% INTRODUZIONE
\ifchorded
\vspace*{\versesep}
\musicnote{
\begin{minipage}{0.48\textwidth}
\textbf{Intro:}
\hfill 
( \eighthnote \, 120)   % <<  MODIFICA IL TEMPO
% Metronomo: \eighthnote (ottavo) \quarternote (quarto) \halfnote (due quarti)
\end{minipage}
} 	
\vspace*{-\versesep}
\beginverse*

\nolyrics

%---- Prima riga -----------------------------
\vspace*{-\versesep}
\[D] \[A]  \[(D)]	 % \[*D] per indicare le pennate, \rep{2} le ripetizioni

%---- Ogni riga successiva -------------------
%\vspace*{-\versesep}
%\[G] \[C]  \[D]	

%---- Ev Indicazioni -------------------------			
%\textnote{\textit{(Oppure tutta la strofa)} }	

\endverse
\fi






\beginchorus
\[D]Alle\[A]luia, \[B-]alleluia, \[F#-]alleluia,
\[G]alleluia, \[D]allelu\[E-]ia, 
\[A7*]a-\[D]alle\[G]luia, al\[A7]lelu\[D]ia.
\endchorus


\beginverse*
\[D]Passeranno i \[A]cieli e \[B-]passerà la \[F#-]terra,
\[G]la Tua parola \[D]non passe\[E-]rà. 
\[A7*]a-\[D]alle\[G]luia, al\[A7]lelu\[D]ia.
\endverse




\endsong

