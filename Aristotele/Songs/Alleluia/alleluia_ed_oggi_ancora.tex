%-------------------------------------------------------------
%			INIZIO	CANZONE
%-------------------------------------------------------------


%titolo: 	Alleluia Ed oggi ancra
%autore: 	P. Sequeri
%tonalita: 	Re-



%%%%%% TITOLO E IMPOSTAZONI
\beginsong{Alleluia Ed oggi ancora}[by={P. Sequeri}]% <<< MODIFICA TITOLO E AUTORE
\transpose{0} 						% <<< TRASPOSIZIONE #TONI (0 nullo)
\momenti{Acclamazione al Vangelo;}							% <<< INSERISCI MOMENTI	
% momenti vanno separati da ; e vanno scelti tra:
% Ingresso; Atto penitenziale; Acclamazione al Vangelo; Dopo il Vangelo; Offertorio; Comunione; Ringraziamento; Fine; Santi; Pasqua; Avvento; Natale; Quaresima; Canti Mariani; Battesimo; Prima Comunione; Cresima; Matrimonio; Meditazione; Spezzare del pane;
\ifchorded
	%\textnote{Tonalità migliore }	% <<< EV COMMENTI (tonalità originale/migliore)
\fi



%%%%%% INTRODUZIONE
\ifchorded
\vspace*{\versesep}
\musicnote{
\begin{minipage}{0.48\textwidth}
\textbf{Intro:}
\hfill 
( \eighthnote  \, 132)
\end{minipage}
} 
\vspace*{-\versesep}
\beginverse*

\nolyrics

%---- Prima riga -----------------------------
\vspace*{-\versesep}
\[(D-)] \[G-] \[C7] \[F]   % \[*D] per indicare le pennate, \rep{2} le ripetizioni

%---- Ogni riga successiva -------------------
\vspace*{-\versesep}
\[B&6] \[G-] \[A7]\[D-]

%---- Ev Indicazioni -------------------------			
%\textnote{\textit{(Oppure tutta la strofa)} }	

\endverse
\fi



\beginchorus
\textnote{\textbf{Rit.}}
\[(D-)] Alle\[G-]luia, \[C7] allelu\[F]ia, \quad \[B&6] 
alle\[G-]luia, \[A7] alle\[D-]luia,
\[D-7] alle\[G-]luia, \[C7] allelu\[F]ia, \quad \[B&6]
  alle\[G-]luia, \[A7] alle\[D-]luia.
\endchorus

\beginverse*
Ed oggi an\[D-]cora, mio Si\[G-]gnore, \brk ascolte\[C]rò la tua pa\[F]rola,
che mi \[B&]guida nel cam\[G-]mino della \[A4]vita. \[A]
\endverse


\endsong
%------------------------------------------------------------
%			FINE CANZONE
%------------------------------------------------------------



