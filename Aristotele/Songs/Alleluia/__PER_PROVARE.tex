% SETTINGS
%——————————————————————————————————————————————————————
% STILE DOCUMENTO
%-------------------------------------------------------------------------------                                                                             
\documentclass[a4vert, palatino, titleindex, tematicindex, chorded, cover]{canzoniereonline}

%opzioni formato: singoli, standard (A4), a5vert, a5oriz, a6vert;
%opzioni accordi: lyric, chorded {quelli d Songs}
%opzioni font: palatino, libertine
%opzioni segno minore: "minorsign=quel che vuoi"
%opzioni indici: authorsindex, titleindex, tematicindex

%opzioi copertina: cover e nocover

\def\canzsongcolumsnumber{2} %# coolonne lungo cui disporre le canzoni




% PACCHETTI DA IMPORTARE
%-------------------------------------------------------------------------------                                                                             
\usepackage[T1]{fontenc}
\usepackage[utf8]{inputenc}
\usepackage[italian]{babel}
\usepackage{pdfpages}
\usepackage{hyperref}
\usepackage{wasysym}



\newcommand\textbox[1]{%
  \parbox{.42\textwidth}{#1}%
}






% NUOVI COMANDI E VARIABILI GLOBALI
%--------------------------------------------------------------------------------
%Coomando per la suddivisione in capitoli
\renewcommand{\songchapter}{\chapter*}


%Counter globale per tenere traccia di una numerazione progressiva
%Si affianca a un altro counter già utilizzato nella classe CanzoniereOnLine "songnum" 
%che, tuttavia, si riazzera ognivolta viene definito un nuovo ambiente \beginsongs{}
\newcounter{GlobalSongCounter} 

%Ciascun capitolo contiene già tutta la logica di gestione della 
%numerazione progressiva, del DB locale da cui attingere le canzoni
%e la creazione/chiusura dell'ambiente in cui vengono importate 
%tutte le canzoni relative
\addtocounter{GlobalSongCounter}{1} %set starting song counter to 1 (0 otherwise)


%------------
\makeatletter
\newcommand*{\textoverline}[1]{$\overline{\hbox{#1}}\m@th$}
\makeatother
%-----------

%Starting Document
\begin{document}


\begin{songs}{}
\songcolumns{\canzsongcolumsnumber}
\setcounter{songnum}{\theGlobalSongCounter} %set songnum counter, otherwise would be reset



%  *  *  *  *  *  TEST SONG HERE  *  *  *  *  *  *  * 


%-------------------------------------------------------------
%			INIZIO	CANZONE
%-------------------------------------------------------------


%titolo: 	Alleluia Ed oggi ancra
%autore: 	Sequerii
%tonalita: 	Re-



%%%%%% TITOLO E IMPOSTAZONI
\beginsong{Alleluia Ed oggi ancora}[by={Sequeri}]% <<< MODIFICA TITOLO E AUTORE
\transpose{0} 						% <<< TRASPOSIZIONE #TONI (0 nullo)
\momenti{Acclamazione al Vangelo;}							% <<< INSERISCI MOMENTI	
% momenti vanno separati da ; e vanno scelti tra:
% Ingresso; Atto penitenziale; Acclamazione al Vangelo; Dopo il Vangelo; Offertorio; Comunione; Ringraziamento; Fine; Santi; Pasqua; Avvento; Natale; Quaresima; Canti Mariani; Battesimo; Prima Comunione; Cresima; Matrimonio; Meditazione; Spezzare del pane;
\ifchorded
	%\textnote{Tonalità migliore }	% <<< EV COMMENTI (tonalità originale/migliore)
\fi


%%%%%% INTRODUZIONE
\ifchorded
\vspace*{\versesep}
\textnote{Intro: \qquad \qquad  (\eighthnote 132) }% <<  MODIFICA IL TEMPO
% Metronomo: \eighthnote (ottavo) \quarternote (quarto) \halfnote (due quarti)
\vspace*{-\versesep}
\beginverse*

\nolyrics

%---- Prima riga -----------------------------
\vspace*{-\versesep}
\[(D-)] \[G-] \[C7] \[F]   % \[*D] per indicare le pennate, \rep{2} le ripetizioni

%---- Ogni riga successiva -------------------
\vspace*{-\versesep}
\[B&6] \[G-] \[A7]\[D-]

%---- Ev Indicazioni -------------------------			
%\textnote{\textit{(Oppure tutta la strofa)} }	

\endverse
\fi



\beginchorus
\[(D-)] Alle\[G-]luia, \[C7] allelu\[F]ia, \quad \[B&6] 
alle\[G-]luia, \[A7] alle\[D-]luia,
\[D-7] alle\[G-]luia, \[C7] allelu\[F]ia, \quad \[B&6]
  alle\[G-]luia, \[A7] alle\[D-]luia.
\endchorus

\beginverse*
Ed oggi an\[D-]cora, mio Si\[G-]gnore, \brk ascolte\[C]rò la tua pa\[F]rola,
che mi \[B&]guida nel cam\[G-]mino della \[A4]vi\[A]ta.
\endverse


\endsong
%------------------------------------------------------------
%			FINE CANZONE
%------------------------------------------------------------








% *  *  *  *  *  *  *  *  *  *  *  *  *  *  *  *  *  *






\end{songs}




%\ifcanzsingole
%	\relax
%\else
%	\iftitleindex
%		\ifxetex
%		\printindex[alfabetico]
%		\else
%		\printindex
%		\fi
%	\else
%	\fi
%	\ifauthorsindex
%	\printindex[autori]
%	\else
%	\fi
%	\iftematicindex
%	\printindex[tematico]
%	\else
%	\fi
%	\ifcover
%		\relax
%	\else
%		\colophon
%	\fi
%\fi
\end{document}
