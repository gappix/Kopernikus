%-------------------------------------------------------------
%			INIZIO	CANZONE
%-------------------------------------------------------------


%titolo: 	Alleluia Canto per Cristo
%autore: 	Costa
%tonalita: 	Re 



%%%%%% TITOLO E IMPOSTAZONI
\beginsong{Alleluia Canto per Cristo}[by={E. Costa}] 	% <<< MODIFICA TITOLO E AUTORE
\transpose{0} 						% <<< TRASPOSIZIONE #TONI (0 nullo)
\momenti{Acclamazione al Vangelo;}							% <<< INSERISCI MOMENTI	
% momenti vanno separati da ; e vanno scelti tra:
% Ingresso; Atto penitenziale; Acclamazione al Vangelo;  Dopo il Vangelo; Offertorio; Comunione; Ringraziamento; Fine; Santi; Pasqua; Avvento; Natale; Quaresima; Canti Mariani; Battesimo; Prima Comunione; Cresima; Matrimonio; Meditazione; Spezzare del pane;
\ifchorded
	%\textnote{Tonalità migliore }	% <<< EV COMMENTI (tonalità originale/migliore)
\fi


%%%%%% INTRODUZIONE
\ifchorded
\vspace*{\versesep}
\musicnote{
\begin{minipage}{0.48\textwidth}
\textbf{Intro}
\hfill 
%( \eighthnote \, 80)   % <<  MODIFICA IL TEMPO
% Metronomo: \eighthnote (ottavo) \quarternote (quarto) \halfnote (due quarti)
\end{minipage}
} 	
\vspace*{-\versesep}
\beginverse*

\nolyrics

%---- Prima riga -----------------------------
\vspace*{-\versesep}
\[D] \[F#-] \[G] \[D]\[D]	 % \[*D] per indicare le pennate, \rep{2} le ripetizioni

%---- Ogni riga successiva -------------------
%\vspace*{-\versesep}
%\[G] \[C]  \[D]	

%---- Ev Indicazioni -------------------------			
%\textnote{\textit{(Oppure tutta la strofa)} }	

\endverse
\fi



%%%%% RITORNELLO
\beginchorus
\textnote{\textbf{Rit.}}
\[D]Allelu\[F#-]ia, alle\[G]luia, allelu\[D]ia,
\[G]allelu\[D]ia, alle\[A]lu\[A]ia!
\[D]Allelu\[F#-]ia, alle\[G]luia, allelu\[D]ia,
\[G]allelu\[D]ia, alle\[A]lu\[D]ia!
\endchorus


%%%%% STROFA
\beginverse		%Oppure \beginverse* se non si vuole il numero di fianco
\memorize 		% <<< DECOMMENTA se si vuole utilizzarne la funzione
%\chordsoff		% <<< DECOMMENTA se vuoi una strofa senza accordi
\[D]Canto per \[F#-]Cristo che \[G]mi libere\[D]rà
\[G]quando ver\[D]rà nella \[A]glo\[A]ria,
\[D]quando la \[F#-]vita con \[G]Lui rinasce\[D]rà,
\[G]allelu\[D]ia, alle\[A]lu\[D]ia!
\endverse

%%%%% STROFA
\beginverse		%Oppure \beginverse* se non si vuole il numero di fianco
%\memorize 		% <<< DECOMMENTA se si vuole utilizzarne la funzione
%\chordsoff		% <<< DECOMMENTA se vuoi una strofa senza accordi
^Canto per ^Cristo: in ^Lui rifiori^rà
^ogni spe^ranza per^du^ta,
^ogni crea^tura con ^Lui risorge^rà,
^allelu^ia, alle^lu^ia!
\endverse

%%%%% STROFA
\beginverse		%Oppure \beginverse* se non si vuole il numero di fianco
%\memorize 		% <<< DECOMMENTA se si vuole utilizzarne la funzione
%\chordsoff		% <<< DECOMMENTA se vuoi una strofa senza accordi
^Canto per ^Cristo: un ^giorno torne^rà!
^Festa per ^tutti gli a^mi^ci,
^festa di un ^mondo che ^più non mori^rà,
^allelu^ia, alle^lu^ia!
\endverse


\endsong
%------------------------------------------------------------
%			FINE CANZONE
%------------------------------------------------------------
