%-------------------------------------------------------------
%			INIZIO	CANZONE
%-------------------------------------------------------------


%titolo: 	Alleluia lode cosmica
%autore: 	Puri
%tonalita: 	Lam 


%%%%%% TITOLO E IMPOSTAZONI
\beginsong{Alleluia Lode cosmica}[by={S. Puri}]
\transpose{7}
\momenti{Acclamazione al Vangelo;}							% <<< INSERISCI MOMENTI	
% momenti vanno separati da ; e vanno scelti tra:
% Ingresso; Atto penitenziale; Acclamazione al Vangelo; Dopo il Vangelo; Offertorio; Comunione; Ringraziamento; Fine; Santi; Pasqua; Avvento; Natale; Quaresima; Canti Mariani; Battesimo; Prima Comunione; Cresima; Matrimonio; Meditazione;
\ifchorded
	\textnote{$\bigstar$ Tonalità migliore per le bambine }	% <<< EV COMMENTI (tonalità originale/migliore)
\fi



%%%%%% INTRODUZIONE
\ifchorded
\vspace*{\versesep}
\musicnote{
\begin{minipage}{0.48\textwidth}
\textbf{Intro:}
\hfill 
%( \eighthnote \, 80)   % <<  MODIFICA IL TEMPO
% Metronomo: \eighthnote (ottavo) \quarternote (quarto) \halfnote (due quarti)
\end{minipage}
} 	
\vspace*{-\versesep}
\beginverse*

\nolyrics

%---- Prima riga -----------------------------
\vspace*{-\versesep}
\[D-] \[F]  \[C] \[C]	 % \[*D] per indicare le pennate, \rep{2} le ripetizioni

%---- Ogni riga successiva -------------------
\vspace*{-\versesep}
\[D-] \[F]  \[C] \[C]

%---- Ev Indicazioni -------------------------			
%\textnote{\textit{(Oppure tutta la strofa)} }	

\endverse
\fi





%%%%% RITORNELLO
\beginchorus
\textnote{\textbf{Rit.}}
\[D-]Alleluia, \[F]àllelu\[C]ia! 	\[C]
\[G-]Alleluia, \[B&]àllelu\[D-]ia! \[C6]
Alle\[B&]lu-u\[F]ia, al\[G-]le-elu\[D-]ia!
Alle\[C]lù\[D-]ia,	alle\[A4/7]lù\[A7]ia! \iflyric \rep{2} \fi
\ifchorded
\[D-]Alleluia, \[F]àllelu\[C]ia! 	\[C]
\[G-]Alleluia, \[B&]àllelu\[D-]ia! \[C6]
Alle\[B&]lu-u\[F]ia, al\[G-]le-elu\[D-]ia!
Alle\[C]lù\[D-]ia,	alle\[A4/7]lu\[A4/7]u\[A7]\[A7]ia! 
\fi
\endchorus

%%%%% STROFA
\beginverse
\memorize
\[D-]Lodino il Si\[B&]gnor i \[F]cie\[C]li, 
\[D-]lodino il Si\[B&]gnor i \[F]ma\[C]ri,
gli \[G-]angeli, i \[G-]cieli dei \[B&]cieli:
il Suo \[D-]nome è \[D-]grande e su\[A4/7]bli\[A7]me.

\[D-]Sole, luna e \[B&]stelle ar\[F]den\[C]ti, 
\[D-]neve, pioggia, \[B&]nebbia, e \[F]fuo\[C]co 
\[G-]lodino \[G-] il Suo \[B&]nome \[B&]in e\[D-]ter\[D-]no! 
\[A4/7]Sia \[A7]lode al Si\[B&7+]gnor! \[B&]
\[C6]Sia	\[(B&7+)]lode al Suo \[B&7+]nome
\[A4]Sia \[A]lode al Signor!
\endverse






%%%%% RITORNELLO
\beginchorus
\textnote{\textbf{Rit.}}
\[D-]Alleluia, \[F]àllelu\[C]ia! 	\[C] \iflyric \dots \rep{2}\fi
\ifchorded
\[G-]Alleluia, \[B&]àllelu\[D-]ia! \[C6]
Alle\[B&]lu-u\[F]ia, al\[G-]le-elu\[D-]ia!
Alle\[C]lù\[D-]ia,	alle\[A4/7]lù\[A7]ia!  
\[D-]Alleluia, \[F]àllelu\[C]ia! 	\[C]
\[G-]Alleluia, \[B&]àllelu\[D-]ia! \[C6]
Alle\[B&]lu-u\[F]ia, al\[G-]le-elu\[D-]ia!
Alle\[C]lù\[D-]ia,	alle\[A4/7]lu\[A4/7]u\[A7]\[A7]ia! 
\fi
\endchorus







\beginverse
^Lodino il Si^gnor le ^ter^re, 
^lodino il Si^gnor i ^mon^ti,
il ^vento ^ della tem^pesta
che obbe^disce al^la Sua ^vo^ce,

^giudici, so^vrani ^tut^ti, 
^giovani, fan^ciulle, ^vec^chi 
^lodino ^ la Sua ^Gloria ^in e^ter^no! 
^Sia ^lode al Si^gnor! ^
^Sia ^lode al Suo ^nome!
^Sia ^lode al Signor!
\endverse


%%%%% RITORNELLO
\beginchorus
\textnote{\textbf{Rit.}}
\[D-]Alleluia, \[F]àllelu\[C]ia! 	\[C] \iflyric \dots \fi
\ifchorded
\[G-]Alleluia, \[B&]àllelu\[D-]ia! \[C6]
Alle\[B&]lu-u\[F]ia, al\[G-]le-elu\[D-]ia!
Alle\[C]lù\[D-]ia,	alle\[A4/7]lù\[A7]ia!  
\fi
\endchorus

\beginchorus %oppure \beginverse*
\vspace*{1.3\versesep}
\textnote{\textbf{Finale} \textit{(rallentando)}} %<<< EV. INDICAZIONI
\[D-]Alleluia, \[F]àllelu\[C]ia! 	\[C]
\[G-]Alleluia, \[B&]àllelu\[D-]ia! \[C6]
Alle\[B&]lu-u\[F]ia, al\[G-]le-elu\[D-]ia!
Alle\[C]lu\[D-]ia,	
alle\[A4/7]lu\[A4/7]u\[A7]u\[A7]u\[B&7]\[B&7]ia!
A-lle-\[B&6]lu-\[B&6]u-\[D]ia!

\endchorus

\endsong

