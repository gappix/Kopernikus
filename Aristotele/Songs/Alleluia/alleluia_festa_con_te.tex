%-------------------------------------------------------------
%			INIZIO	CANZONE
%-------------------------------------------------------------


%titolo: 	Alleluia, festa con Te
%autore: 	Fabio Avolio
%tonalita: 	Do 



%%%%%% TITOLO E IMPOSTAZONI
\beginsong{Alleluia, festa con Te}[by={Fabio Avolio}] 	% <<< MODIFICA TITOLO E AUTORE
\transpose{0} 						% <<< TRASPOSIZIONE #TONI (0 nullo)
\momenti{Acclamazione al Vangelo}							% <<< INSERISCI MOMENTI	
% momenti vanno separati da ; e vanno scelti tra:
% Ingresso; Atto penitenziale; Acclamazione al Vangelo; Dopo il Vangelo; Offertorio; Comunione; Ringraziamento; Fine; Santi; Pasqua; Avvento; Natale; Quaresima; Canti Mariani; Battesimo; Prima Comunione; Cresima; Matrimonio; Meditazione; Spezzare del pane;
\ifchorded
	%\textnote{Tonalità originale }	% <<< EV COMMENTI (tonalità originale/migliore)
\fi

%%%%%% INTRODUZIONE
\ifchorded
\vspace*{\versesep}
\textnote{Intro: \qquad \qquad  }%(\eighthnote 116) % <<  MODIFICA IL TEMPO
% Metronomo: \eighthnote (ottavo) \quarternote (quarto) \halfnote (due quarti)
\vspace*{-\versesep}
\beginverse*

\nolyrics

%---- Prima riga -----------------------------
\vspace*{-\versesep}
\[C] \[F] \[G] \[A-] \[F] \[C] \[D-] \[G]	 % \[*D] per indicare le pennate, \rep{2} le ripetizioni

%---- Ogni riga successiva -------------------
%\vspace*{-\versesep}
%\[G] \[C]  \[D]	

%---- Ev Indicazioni -------------------------			
%\textnote{\textit{(Oppure tutta la strofa)} }	

\endverse
\fi

%%%%% RITORNELLO
\beginchorus
\textnote{\textbf{Rit.}}

\[C]Allelu\[F]ia, \[G]allelu\[A-]ia, \[F]oggi è \[C]festa con \[D-]te, Ges\[G]ù.
\[C]Tu sei con \[F]noi, \[G]gioia ci \[A-]dai, \[F]allelu\[C]ia, alle\[G]lu\[C]ia.

\endchorus

%%%%% STROFA
\beginverse		%Oppure \beginverse* se non si vuole il numero di fianco
\memorize 		% <<< DECOMMENTA se si vuole utilizzarne la funzione
%\chordsoff		% <<< DECOMMENTA se vuoi una strofa senza accordi

\[C]Nella tua \[G]casa \[C]siamo ve\[F]nuti \[C]per incon\[D-]trar\[G]ti.
\[C]A te can\[G]tiamo \[A-]la nostra \[F]lode, \[C]gloria al tuo \[G]no\[C]me. \[G]

\endverse

%%%%% STROFA
\beginverse		%Oppure \beginverse* se non si vuole il numero di fianco
%\memorize 		% <<< DECOMMENTA se si vuole utilizzarne la funzione
\chordsoff		% <<< DECOMMENTA se vuoi una strofa senza accordi

^Il pane ^vivo ^che ci hai pro^messo ^dona la ^vi^ta.
^A te can^tiamo ^la nostra ^lode, ^gloria al tuo ^no^me. ^

\endverse

%%%%% STROFA
\beginverse		%Oppure \beginverse* se non si vuole il numero di fianco
%\memorize 		% <<< DECOMMENTA se si vuole utilizzarne la funzione
\chordsoff		% <<< DECOMMENTA se vuoi una strofa senza accordi

^Tu sei l'^mico ^che ci accom^pagna ^lungo il cam^mi^no.
^A te can^tiamo ^la nostra ^lode, ^gloria al tuo ^no^me. ^

\endverse

\endsong
%------------------------------------------------------------
%			FINE CANZONE
%------------------------------------------------------------