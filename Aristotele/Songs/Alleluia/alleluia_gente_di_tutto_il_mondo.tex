%-------------------------------------------------------------
%			INIZIO	CANZONE
%-------------------------------------------------------------


%titolo: 	Alleluia, gente di tutto il mondo
%autore: 	Pierangelo Sequeri
%tonalita: 	Fa



%%%%%% TITOLO E IMPOSTAZONI
\beginsong{Alleluia Gente di tutto il mondo}[by={Pierangelo Sequeri}] 	% <<< MODIFICA TITOLO E AUTORE
\transpose{0} 						% <<< TRASPOSIZIONE #TONI (0 nullo)
\momenti{Acclamazione al Vangelo}							% <<< INSERISCI MOMENTI	
% momenti vanno separati da ; e vanno scelti tra:
% Ingresso; Atto penitenziale; Acclamazione al Vangelo; Dopo il Vangelo; Offertorio; Comunione; Ringraziamento; Fine; Santi; Pasqua; Avvento; Natale; Quaresima; Canti Mariani; Battesimo; Prima Comunione; Cresima; Matrimonio; Meditazione; Spezzare del pane;
\ifchorded
	%\textnote{Tonalità migliore }	% <<< EV COMMENTI (tonalità originale/migliore)
\fi

%%%%%% INTRODUZIONE
\ifchorded
\vspace*{\versesep}
\textnote{Intro: \qquad \qquad  }%(\eighthnote 116) % <<  MODIFICA IL TEMPO
% Metronomo: \eighthnote (ottavo) \quarternote (quarto) \halfnote (due quarti)
\vspace*{-\versesep}
\beginverse*

\nolyrics

%---- Prima riga -----------------------------
\vspace*{-\versesep}
\[F] \[B&] \[F]	 % \[*D] per indicare le pennate, \rep{2} le ripetizioni

%---- Ogni riga successiva -------------------
%\vspace*{-\versesep}
%\[G] \[C]  \[D]	

%---- Ev Indicazioni -------------------------			
%\textnote{\textit{(Oppure tutta la strofa)} }	

\endverse
\fi

%%%%% STROFA
\beginverse		%Oppure \beginverse* se non si vuole il numero di fianco
\memorize 		% <<< DECOMMENTA se si vuole utilizzarne la funzione
%\chordsoff		% <<< DECOMMENTA se vuoi una strofa senza accordi

\[F]Gente di \[A7]tutto il \[G-]mondo
ascol\[C7]tate il nostro \[F]can\[F7]to,
\[B&]lieti vi annun\[F]ciamo:
il Si\[B&-]gno\[C7]re è ri\[F]sorto!

\endverse

%%%%% RITORNELLO
\beginchorus
\textnote{\textbf{Rit.}}

\[F]Allelu\[D-]ia, \[G-]allelu\[C]ia, 
\[F]allelu, allelu, \[B&]allelu\[C7]ia.
\[F]Allelu\[D-]ia, \[G-]allelu\[C]ia,
\[F]allelu, \[C7]allelu\[F]ia. \[B&] \[F]

\endchorus

%%%%% STROFA
\beginverse		%Oppure \beginverse* se non si vuole il numero di fianco
%\memorize 		% <<< DECOMMENTA se si vuole utilizzarne la funzione
\chordsoff		% <<< DECOMMENTA se vuoi una strofa senza accordi

^Il Figlio ^tanto a^mato
che il Dio ^nostro ci ha do^na^to
^l'ha risusci^tato
per la ^vi^ta del ^mondo!

\endverse

%%%%% STROFA
\beginverse		%Oppure \beginverse* se non si vuole il numero di fianco
%\memorize 		% <<< DECOMMENTA se si vuole utilizzarne la funzione
\chordsoff		% <<< DECOMMENTA se vuoi una strofa senza accordi

^Diede la ^propria ^vita
per a^more dei ^fra^telli.
^Vinta ormai la ^morte
è per ^sem^pre con ^noi.

\endverse

%%%%% STROFA
\beginverse		%Oppure \beginverse* se non si vuole il numero di fianco
%\memorize 		% <<< DECOMMENTA se si vuole utilizzarne la funzione
\chordsoff		% <<< DECOMMENTA se vuoi una strofa senza accordi

^Vivere ^del suo a^more
nell'at^tesa che ri^tor^ni:
^questa è la pa^rola
che ci ^do^na spe^ranza.

\endverse

\endsong
%------------------------------------------------------------
%			FINE CANZONE
%------------------------------------------------------------

%++++++++++++++++++++++++++++++++++++++++++++++++++++++++++++
%			CANZONE TRASPOSTA
%++++++++++++++++++++++++++++++++++++++++++++++++++++++++++++
\ifchorded
%decremento contatore per avere stesso numero
\addtocounter{songnum}{-1} 
\beginsong{Alleluia, gente di tutto il mondo}[by={Pierangelo Sequeri}] 	% <<< COPIA TITOLO E AUTORE
\transpose{+2} 						% <<< TRASPOSIZIONE #TONI + - (0 nullo)
%\preferflats  %SE VOGLIO FORZARE i bemolle come alterazioni
%\prefersharps %SE VOGLIO FORZARE i # come alterazioni
\ifchorded
	%\textnote{Tonalità originale}	% <<< EV COMMENTI (tonalità originale/migliore)
\fi


%%%%%% INTRODUZIONE
\ifchorded
\vspace*{\versesep}
\textnote{Intro: \qquad \qquad  }%(\eighthnote 116) % <<  MODIFICA IL TEMPO
% Metronomo: \eighthnote (ottavo) \quarternote (quarto) \halfnote (due quarti)
\vspace*{-\versesep}
\beginverse*

\nolyrics

%---- Prima riga -----------------------------
\vspace*{-\versesep}
\[F] \[B&] \[F]	 % \[*D] per indicare le pennate, \rep{2} le ripetizioni

%---- Ogni riga successiva -------------------
%\vspace*{-\versesep}
%\[G] \[C]  \[D]	

%---- Ev Indicazioni -------------------------			
%\textnote{\textit{(Oppure tutta la strofa)} }	

\endverse
\fi

%%%%% STROFA
\beginverse		%Oppure \beginverse* se non si vuole il numero di fianco
\memorize 		% <<< DECOMMENTA se si vuole utilizzarne la funzione
%\chordsoff		% <<< DECOMMENTA se vuoi una strofa senza accordi

\[F]Gente di \[A7]tutto il \[G-]mondo
ascol\[C7]tate il nostro \[F]can\[F7]to,
\[Bb]lieti vi annun\[F]ciamo:
il Si\[B&-]gno\[C7]re è ri\[F]sorto!

\endverse

%%%%% RITORNELLO
\beginchorus
\textnote{\textbf{Rit.}}

\[F]Allelu\[D-]ia, \[G-]allelu\[C]ia, 
\[F]allelu, allelu, \[B&]allelu\[C7]ia.
\[F]Allelu\[D-]ia, \[G-]allelu\[C]ia,
\[F]allelu, \[C7]allelu\[F]ia. \[B&] \[F]

\endchorus

%%%%% STROFA
\beginverse		%Oppure \beginverse* se non si vuole il numero di fianco
%\memorize 		% <<< DECOMMENTA se si vuole utilizzarne la funzione
\chordsoff		% <<< DECOMMENTA se vuoi una strofa senza accordi

^Il Figlio ^tanto a^mato
che il Dio ^nostro ci ha do^na^to
^l'ha risusci^tato
per la ^vi^ta del ^mondo!

\endverse

%%%%% STROFA
\beginverse		%Oppure \beginverse* se non si vuole il numero di fianco
%\memorize 		% <<< DECOMMENTA se si vuole utilizzarne la funzione
\chordsoff		% <<< DECOMMENTA se vuoi una strofa senza accordi

^Diede la ^propria ^vita
per a^more dei ^fra^telli.
^Vinta ormai la ^morte
è per ^sem^pre con ^noi.

\endverse

%%%%% STROFA
\beginverse		%Oppure \beginverse* se non si vuole il numero di fianco
%\memorize 		% <<< DECOMMENTA se si vuole utilizzarne la funzione
\chordsoff		% <<< DECOMMENTA se vuoi una strofa senza accordi

^Vivere ^del suo a^more
nell'at^tesa che ri^tor^ni:
^questa è la pa^rola
che ci ^do^na spe^ranza.

\endverse

\endsong


\fi
%++++++++++++++++++++++++++++++++++++++++++++++++++++++++++++
%			FINE CANZONE TRASPOSTA
%++++++++++++++++++++++++++++++++++++++++++++++++++++++++++++
