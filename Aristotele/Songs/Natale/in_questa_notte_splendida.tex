%-------------------------------------------------------------
%			INIZIO	CANZONE
%-------------------------------------------------------------

%titolo: In questa notte splendida
%autore: Sequeri
%tonalita: Mi e Sol 


%%%%%% TITOLO E IMPOSTAZONI
\beginsong{In questa notte splendida}[by={M. Chieffo}] 	% <<< MODIFICA TITOLO E AUTORE
\transpose{5} 						% <<< TRASPOSIZIONE #TONI (0 nullo)
\momenti{Natale}							% <<< INSERISCI MOMENTI
\ifchorded
	\textnote{$\bigstar$ Tonalità migliore per le bambine}	% <<< EV COMMENTI (tonalità originale/migliore)
\fi

%%%%%% INTRODUZIONE
\ifchorded
\vspace*{\versesep}
\musicnote{
\begin{minipage}{0.48\textwidth}
\textbf{Intro}
\hfill 
%( \eighthnote \, 80)   % <<  MODIFICA IL TEMPO
% Metronomo: \eighthnote (ottavo) \quarternote (quarto) \halfnote (due quarti)
\end{minipage}
} 
\musicnote{\textit{[Dolcissimo, partendo molto piano]}}
\vspace*{-\versesep}
\beginverse*

\nolyrics

%---- Prima riga -----------------------------
\vspace*{-\versesep}
\[E] \[A]  \[E]  \[B] 

%---- Ogni riga successiva -------------------
\vspace*{-\versesep}
\[E] \[A]  \[E]  \[B] \[E]



\endverse
\fi






%%%%% STROFA
\beginverse		%Oppure \beginverse* se non si vuole il numero di fianco
\memorize 		% <<< DECOMMENTA se si vuole utilizzarne la funzione
%\chordsoff		% <<< DECOMMENTA se vuoi una strofa senza accordi

In \[E]questa notte \[A]splendida
di \[E]luce e di chia\[B]ror
il \[E]nostro cuore \[A]trepida,
è \[E]nato il \[B]Salva\[E]tor.
Un \[A]bimbo picco\[E]lissimo
le \[A]porte ci apri\[E]rà
del \[E]cielo dell'Al\[A]tissimo
nel\[E]la sua \[B]veri\[E]tà.

\endverse

%%%%%% EV. INTERMEZZO
\beginverse*
\vspace*{1.3\versesep}
{
	\nolyrics
	\textnote{Intermezzo strumentale}
	\musicnote{\textit{[Crescendo di intensità]}} 
	 
	\ifchorded

	%---- Prima riga -----------------------------
	\vspace*{-\versesep}
	\[E] \[A]  \[E]  \[B] 

	%---- Ogni riga successiva -------------------
	\vspace*{-\versesep}
	\[E] \[A]  \[E]  \[B] \[E]

	\fi

}
\vspace*{\versesep}
\endverse

%%%%% STROFA
\beginverse		%Oppure \beginverse* se non si vuole il numero di fianco
%\memorize 		% <<< DECOMMENTA se si vuole utilizzarne la funzione
%\chordsoff		% <<< DECOMMENTA se vuoi una strofa senza accordi

Svegli^atevi dal ^sonno,
corr^ete coi pas^tor,
è ^notte di mi^racoli,
di ^grazia e ^di stu^por.
A^sciuga le tue ^lacrime,
non ^piangere per^chè
Ge^sù nostro ca^rissimo 
è ^nato an^che per ^te.

\endverse

\transpose{3}



%%%%%% EV. INTERMEZZO
\beginverse*
\vspace*{1.3\versesep}
{
	\nolyrics
	\textnote{Intermezzo strumentale}
	\textnote{\textit{[Cambia la tonalità]}} 
	 
	\ifchorded

	%---- Prima riga -----------------------------
	\vspace*{-\versesep}
	\[E] \[A]  \[E]  \[B] 

	%---- Ogni riga successiva -------------------
	\vspace*{-\versesep}
	\[E] \[A]  \[E]  \[B] \[E]

	\fi

}
\vspace*{\versesep}
\endverse


%%%%% STROFA
\beginverse		%Oppure \beginverse* se non si vuole il numero di fianco
%\memorize 		% <<< DECOMMENTA se si vuole utilizzarne la funzione
%\chordsoff		% <<< DECOMMENTA se vuoi una strofa senza accordi

In ^questa notte ^limpida
di ^gloria e di splen^dor,
il ^nostro cuore ^trepida
è ^nato il ^Salva^tor.
Ge^sù nostro ca^rissimo
le ^porte ci apri^rà,

\musicnote{\textit{[Rallentando]}}
il ^figlio dell'Al^tissimo
con ^no\[E]i sem^pr\[B]e sa^r\[E]à.

\endverse





\endsong
%------------------------------------------------------------
%			FINE CANZONE
%------------------------------------------------------------

%++++++++++++++++++++++++++++++++++++++++++++++++++++++++++++
%			CANZONE TRASPOSTA
%++++++++++++++++++++++++++++++++++++++++++++++++++++++++++++
\ifchorded
%decremento contatore per avere stesso numero
\addtocounter{songnum}{-1} 
\beginsong{In questa notte splendida}[by={M. Chieffo}] 	% <<< MODIFICA TITOLO E AUTORE
\transpose{0} 						% <<< TRASPOSIZIONE #TONI + - (0 nullo)
\ifchorded
	\textnote{$\lozenge$ Tonalità originale}	% <<< EV COMMENTI (tonalità originale/migliore)
\fi




%%%%%% INTRODUZIONE
\ifchorded
\vspace*{\versesep}
\musicnote{
\begin{minipage}{0.48\textwidth}
\textbf{Intro}
\hfill 
%( \eighthnote \, 80)   % <<  MODIFICA IL TEMPO
% Metronomo: \eighthnote (ottavo) \quarternote (quarto) \halfnote (due quarti)
\end{minipage}
} 
\musicnote{\textit{[Dolcissimo, partendo molto piano]}}
\vspace*{-\versesep}
\beginverse*

\nolyrics

%---- Prima riga -----------------------------
\vspace*{-\versesep}
\[E] \[A]  \[E]  \[B] 

%---- Ogni riga successiva -------------------
\vspace*{-\versesep}
\[E] \[A]  \[E]  \[B] \[E]



\endverse
\fi






%%%%% STROFA
\beginverse		%Oppure \beginverse* se non si vuole il numero di fianco
\memorize 		% <<< DECOMMENTA se si vuole utilizzarne la funzione
%\chordsoff		% <<< DECOMMENTA se vuoi una strofa senza accordi

In \[E]questa notte \[A]splendida
di \[E]luce e di chia\[B]ror
il \[E]nostro cuore \[A]trepida,
è \[E]nato il \[B]Salva\[E]tor.
Un \[A]bimbo picco\[E]lissimo
le \[A]porte ci apri\[E]rà
del \[E]cielo dell'Al\[A]tissimo
nel\[E]la sua \[B]veri\[E]tà.

\endverse

%%%%%% EV. INTERMEZZO
\beginverse*
\vspace*{1.3\versesep}
{
	\nolyrics
	\textnote{Intermezzo strumentale}
	\musicnote{\textit{[Crescendo di intensità]}} 
	 
	\ifchorded

	%---- Prima riga -----------------------------
	\vspace*{-\versesep}
	\[E] \[A]  \[E]  \[B] 

	%---- Ogni riga successiva -------------------
	\vspace*{-\versesep}
	\[E] \[A]  \[E]  \[B] \[E]

	\fi

}
\vspace*{\versesep}
\endverse

%%%%% STROFA
\beginverse		%Oppure \beginverse* se non si vuole il numero di fianco
%\memorize 		% <<< DECOMMENTA se si vuole utilizzarne la funzione
%\chordsoff		% <<< DECOMMENTA se vuoi una strofa senza accordi

Svegli^atevi dal ^sonno,
corr^ete coi pas^tor,
è ^notte di mi^racoli,
di ^grazia e ^di stu^por.
A^sciuga le tue ^lacrime,
non ^piangere per^chè
Ge^sù nostro ca^rissimo 
è ^nato an^che per ^te.

\endverse

\transpose{3}



%%%%%% EV. INTERMEZZO
\beginverse*
\vspace*{1.3\versesep}
{
	\nolyrics
	\textnote{Intermezzo strumentale}
	\musicnote{\textit{[Cambia la tonalità]}} 
	 
	\ifchorded

	%---- Prima riga -----------------------------
	\vspace*{-\versesep}
	\[E] \[A]  \[E]  \[B] 

	%---- Ogni riga successiva -------------------
	\vspace*{-\versesep}
	\[E] \[A]  \[E]  \[B] \[E]

	\fi

}
\vspace*{\versesep}
\endverse


%%%%% STROFA
\beginverse		%Oppure \beginverse* se non si vuole il numero di fianco
%\memorize 		% <<< DECOMMENTA se si vuole utilizzarne la funzione
%\chordsoff		% <<< DECOMMENTA se vuoi una strofa senza accordi

In ^questa notte ^limpida
di ^gloria e di splen^dor,
il ^nostro cuore ^trepida
è ^nato il ^Salva^tor.
Ge^sù nostro ca^rissimo
le ^porte ci apri^rà,

\musicnote{\textit{[Rallentando]}}
il ^figlio dell'Al^tissimo
con ^no\[E]i sem^pr\[B]e sa^r\[E]à.

\endverse





\endsong
\fi
%++++++++++++++++++++++++++++++++++++++++++++++++++++++++++++
%			FINE CANZONE TRASPOSTA
%++++++++++++++++++++++++++++++++++++++++++++++++++++++++++++