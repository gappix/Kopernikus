%-------------------------------------------------------------
%			INIZIO	CANZONE
%-------------------------------------------------------------


%titolo: 	Dio è qui
%autore: 	J. Williams
%tonalita: 	Do



%%%%%% TITOLO E IMPOSTAZONI
\beginsong{Dio è qui}[by={Somewhere in my memory — J. Williams}] 	% <<< MODIFICA TITOLO E AUTORE
\transpose{0} 						% <<< TRASPOSIZIONE #TONI (0 nullo)
%\preferflats  %SE VOGLIO FORZARE i bemolle come alterazioni
%\prefersharps %SE VOGLIO FORZARE i # come alterazioni
\momenti{Natale}							% <<< INSERISCI MOMENTI	
% momenti vanno separati da ; e vanno scelti tra:
% Ingresso; Atto penitenziale; Acclamazione al Vangelo; 
%Dopo il Vangelo; Offertorio; Comunione; Ringraziamento; 
%Fine; Santi; Pasqua; Avvento; Natale; Quaresima; Canti Mariani; Battesimo; Prima Comunione; Cresima; Matrimonio; Meditazione; Spezzare del pane;
\ifchorded
	%\textnote{Tonalità migliore }	% <<< EV COMMENTI (tonalità originale/migliore)
\fi




%%%%%% INTRODUZIONE
\ifchorded
\vspace*{\versesep}
\musicnote{
\begin{minipage}{0.48\textwidth}
\textbf{Intro}
\hfill 
%( \eighthnote \, 80)   % <<  MODIFICA IL TEMPO
% Metronomo: \eighthnote (ottavo) \quarternote (quarto) \halfnote (due quarti)
\end{minipage}
} 	
\vspace*{-\versesep}
\beginverse*

\nolyrics

%---- Prima riga -----------------------------
\vspace*{-\versesep}
\[C] \[F] \[C]	 % \[*D] per indicare le pennate, \rep{2} le ripetizioni

%---- Ogni riga successiva -------------------
\vspace*{-\versesep}
\[G] \[C]  \[F] \[G]	

%---- Ev Indicazioni -------------------------			
%\textnote{\textit{(Oppure tutta la strofa)} }	

\endverse
\fi






%%%%% STROFA
\beginverse*		%Oppure \beginverse* se non si vuole il numero di fianco
\memorize 		% <<< DECOMMENTA se si vuole utilizzarne la funzione
%\chordsoff		% <<< DECOMMENTA se vuoi una strofa senza accordi

\[C]Stella della \[F]not\[C]te
\[G]guida il \[C]nostro cam\[F]mi\[G]no
\[C]dentro una \[C7]ca\[F]pan\[E-]na
\[F]dove c’è un \[E-]bimbo che \[D-/F]dor\[G]me.
\endverse
\beginverse*
\[C]Dove il \[D-]cuore \[D]si ride\[G]sta
\[E]e festo\[A-]so \[F]gioi\[G7]rà.
\endverse
\beginverse*
\[C]Una luce \[F]nuo\[C]va
\[G]viene al \[C]mondo a \[F]Betlem\[G]me
\[C]ogni \[C7]uomo at\[F]tra\[C]e
\[F] col suo can\[C]dore
\[F] dona la \[C]vita
\[F] nuova spe\[C]ranza
\[F]Dio \[G]è \[C]qui.

\endverse




\endsong
%------------------------------------------------------------
%			FINE CANZONE
%------------------------------------------------------------

