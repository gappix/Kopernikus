%-------------------------------------------------------------
%			INIZIO	CANZONE
%-------------------------------------------------------------


%titolo: 	Tu Scendi Dalle Stelle
%autore: 	ALfonso de' Liguori
%tonalita: 	Do (abbassata) 



%%%%%% TITOLO E IMPOSTAZONI
\beginsong{Tu scendi dalle stelle}[by={A. De'\ Liguori}]	% <<< MODIFICA TITOLO E AUTORE
\transpose{-2} 						% <<< TRASPOSIZIONE #TONI (0 nullo)
\momenti{Natale}							% <<< INSERISCI MOMENTI	
% momenti vanno separati da ; e vanno scelti tra:
% Ingresso; Atto penitenziale; Acclamazione al Vangelo; Dopo il Vangelo; Offertorio; Comunione; Ringraziamento; Fine; Santi; Pasqua; Avvento; Natale; Quaresima; Canti Mariani; Battesimo; Prima Comunione; Cresima; Matrimonio; Meditazione;
\ifchorded
	%\textnote{Tonalità originale }	% <<< EV COMMENTI (tonalità originale/migliore)
\fi


%%%%%% INTRODUZIONE
\ifchorded
\vspace*{\versesep}
\textnote{Intro: \qquad \qquad  }%(\eighthnote 116) % << MODIFICA IL TEMPO
% Metronomo: \eighthnote (ottavo) \quarternote (quarto) \halfnote (due quarti)
\vspace*{-\versesep}
\beginverse*

\nolyrics

%---- Prima riga -----------------------------
\vspace*{-\versesep}
\[D] 	 % \[*D] per indicare le pennate, \rep{2} le ripetizioni

%---- Ogni riga successiva -------------------
%\vspace*{-\versesep}
%\[G] \[C]  \[D]	

%---- Ev Indicazioni -------------------------			
%\textnote{\textit{(Oppure tutta la strofa)} }	

\endverse
\fi



%%%%% STROFA
\beginverse
\memorize
Tu \[D]scendi dalle stelle, o Re del cie\[A]lo
e vieni in una \[A]grot\[G]ta 
al \[D]freddo e al \[A]ge\[D]lo,
e \[A]vieni in una \[A]grot\[G]ta 
al \[D]freddo e al \[A]ge\[D]lo.
O Bam\[A]bino, mio Di\[D]vino,
io Ti \[A]vedo qui a tre\[D]mar. O Dio be\[A]ato!
Ah quanto Ti cos\[A]tò \[G]l'a\[D]vermi a\[A]ma\[D]to!
Ah \[A]quanto Ti cos\[A]tò \[G]l'a\[D]vermi a\[A]ma\[D]to!
\endverse



%%%%% STROFA
\beginverse
%\chordsoff
A ^Te che sei del mondo il Creato^re,
mancano panni e ^fuo^co 
o ^mio Si^gno^re,
manc^ano panni e ^fuo^co 
o ^mio Si^gno^re.
Caro e^letto Pargo^letto,
quanto ^questa pover^tà 
più m'inna^mora.
Giacchè Ti fece am^or ^po^vero an^co^ra!
Giac^chè Ti fece am^or ^po^vero an^co^ra!
\endverse



%%%%% STROFA
\beginverse
%\chordsoff
Tu ^lasci il bel gioire del divin se^no,
per giungere a tre^ma^re su ^questo ^fie^no;
per ^giungere a tre^ma^re su ^questo ^fie^no.
Dolce a^more del mio ^cuore, 
dove a^mor ti traspo^rtò! O Gesù ^mio, 
perchè tanto pat^ir, ^^per amor ^mi^o.
Per^chè tanto pat^ir, ^^per amor ^mi^o.
\endverse


\endsong
%------------------------------------------------------------
%			FINE CANZONE
%------------------------------------------------------------