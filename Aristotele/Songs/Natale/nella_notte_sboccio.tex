%-------------------------------------------------------------
%			INIZIO	CANZONE
%-------------------------------------------------------------


%titolo: 	Nella notte sbocciò
%autore: 	The first Nowell — Sir D. V. Willcocks, F. Filisetti.
%tonalita: 	Do 



%%%%%% TITOLO E IMPOSTAZONI
\beginsong{Nella notte sbocciò}[by={The first Nowell — Sir D. V. Willcocks\ifacinquevert \iftwocolumns \else , F. Filisetti\fi\fi}] 	% <<< MODIFICA TITOLO E AUTORE
\transpose{0} 						% <<< TRASPOSIZIONE #TONI (0 nullo)
%\preferflats  %SE VOGLIO FORZARE i bemolle come alterazioni
%\prefersharps %SE VOGLIO FORZARE i # come alterazioni
\momenti{Natale}							% <<< INSERISCI MOMENTI	
% momenti vanno separati da ; e vanno scelti tra:
% Ingresso; Atto penitenziale; Acclamazione al Vangelo; Dopo il Vangelo; Offertorio; Comunione; Ringraziamento; Fine; Santi; Pasqua; Avvento; Natale; Quaresima; Canti Mariani; Battesimo; Prima Comunione; Cresima; Matrimonio; Meditazione; Spezzare del pane;
\ifchorded
	%\textnote{Tonalità migliore }	% <<< EV COMMENTI (tonalità originale/migliore)
\fi


%%%%%% INTRODUZIONE
\ifchorded
\vspace*{\versesep}
\musicnote{
\begin{minipage}{0.48\textwidth}
\textbf{Intro}
\hfill 
%( \eighthnote \, 80)   % <<  MODIFICA IL TEMPO
% Metronomo: \eighthnote (ottavo) \quarternote (quarto) \halfnote (due quarti)
\end{minipage}
} 	
\vspace*{-\versesep}
\beginverse*

\nolyrics

%---- Prima riga -----------------------------
\vspace*{-\versesep}
\[C] \[E-] \[F] \[C]


%---- Ev Indicazioni -------------------------			
\textnote{\textit{(oppure tutta la strofa)} }	

\endverse
\fi




%%%%% STROFA
\beginverse		%Oppure \beginverse* se non si vuole il numero di fianco
\memorize 		% <<< DECOMMENTA se si vuole utilizzarne la funzione
%\chordsoff		% <<< DECOMMENTA se vuoi una strofa senza accordi

Nella \[C]notte sboc\[E-]ciò tutto il \[F]Cielo las\[C]sù;
una \[A-]festa \[F*]di \[C]luce \[G*]il \[C]buio \[F*]squar\[C]ciò.
\endverse
\beginverse*
Una ^voce annun^ciò ai pa^stori quag^giù: 
“Oggi è ^nato a ^Be^tlemme ^per ^noi il ^Si^gnor”.
\endverse




%%%%% RITORNELLO
\beginchorus
\textnote{\textbf{Rit.}}

O-o^sanna al Si^gnor che per ^noi si incar^nò, 
che ci ^dona ^la ^pace, ^la ^gioia, ^l’a^mor.

\endchorus



%%%%% STROFA
\beginverse		%Oppure \beginverse* se non si vuole il numero di fianco
%\memorize 		% <<< DECOMMENTA se si vuole utilizzarne la funzione
%\chordsoff		% <<< DECOMMENTA se vuoi una strofa senza accordi

Come ^l’alba si a^prì, come il ^giorno spun^tò
sul de^serto o^riz^zonte l’^at^teso ^Ge^sù.
\endverse
\beginverse*
Come ^miele stil^lò, come ^manna fio^rì:
all’at^tesa ^del ^mondo ^il ^Verbo ^si of^frì.

\endverse



%%%%% STROFA
\beginverse		%Oppure \beginverse* se non si vuole il numero di fianco
%\memorize 		% <<< DECOMMENTA se si vuole utilizzarne la funzione
%\chordsoff		% <<< DECOMMENTA se vuoi una strofa senza accordi
Ogni ^giorno per ^te il Na^tale sa^rà,
se nell’^uomo ^che ^soffre ^ri^trovi ^Ge^sù.
\endverse
\beginverse*
Tra la ^gente che ^va, che cam^mina con ^te,
sulle ^strade ^del ^mondo ^ri^nasce ^Ge^sù. 
\endverse




\endsong
%------------------------------------------------------------
%			FINE CANZONE
%------------------------------------------------------------


