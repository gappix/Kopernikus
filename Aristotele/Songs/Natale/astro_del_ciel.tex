%-------------------------------------------------------------
%			INIZIO	CANZONE
%-------------------------------------------------------------


%titolo: 	Astro del Ciel
%autore: 	Gruber
%tonalita: 	La



%%%%%% TITOLO E IMPOSTAZONI
\beginsong{Astro del ciel}[by={Gruber}]
\transpose{0} 						% <<< TRASPOSIZIONE #TONI (0 nullo)
\momenti{Natale}							% <<< INSERISCI MOMENTI	
% momenti vanno separati da ; e vanno scelti tra:
% Ingresso; Atto penitenziale; Acclamazione al Vangelo; Dopo il Vangelo; Offertorio; Comunione; Ringraziamento; Fine; Santi; Pasqua; Avvento; Natale; Quaresima; Canti Mariani; Battesimo; Prima Comunione; Cresima; Matrimonio; Meditazione;
\ifchorded
	%\textnote{Tonalità originale }	% <<< EV COMMENTI (tonalità originale/migliore)
\fi




%%%%%% INTRODUZIONE
\ifchorded
\vspace*{\versesep}
\textnote{Intro: \qquad \qquad  }%(\eighthnote 116) % << MODIFICA IL TEMPO
% Metronomo: \eighthnote (ottavo) \quarternote (quarto) \halfnote (due quarti)
\vspace*{-\versesep}
\beginverse*

\nolyrics

%---- Prima riga -----------------------------
\vspace*{-\versesep}
\[A] 	 % \[*D] per indicare le pennate, \rep{2} le ripetizioni

%---- Ogni riga successiva -------------------
%\vspace*{-\versesep}
%\[G] \[C]  \[D]	

%---- Ev Indicazioni -------------------------			
%\textnote{\textit{(Oppure tutta la strofa)} }	

\endverse
\fi


%%%%% STROFA
\beginverse		%Oppure \beginverse* se non si vuole il numero di fianco
\memorize 		% <<< DECOMMENTA se si vuole utilizzarne la funzione
%\chordsoff		& <<< DECOMMENTA se vuoi una strofa senza accordi
\[A]Astro del ciel, Pargol divin, \[E]mite agnello \[A]redentor!
\[D]Tu che i Vati da \[A]lungi sognar, \brk \[D]tu che angeliche \[A]voci annunziar,
\endverse


%%%%% RITORNELLO
\beginchorus
\textnote{\textbf{Rit.}}
\[E]Luce dona alle \[A]men\[F#-]ti, \brk \[A]pace in\[E]fondi nei \[A]cuor. \rep{2}
\endchorus


%%%%% STROFA
\beginverse		%Oppure \beginverse* se non si vuole il numero di fianco
%\memorize 		% <<< DECOMMENTA se si vuole utilizzarne la funzione
%\chordsoff		& <<< DECOMMENTA se vuoi una strofa senza accordi
^Astro del ciel, Pargol divin, ^mite agnello ^redentor,
^tu di stirpe re^gale decor, \brk ^tu virgineo ^mistico fior.
\endverse


%%%%% STROFA
\beginverse		%Oppure \beginverse* se non si vuole il numero di fianco
%\memorize 		% <<< DECOMMENTA se si vuole utilizzarne la funzione
%\chordsoff		& <<< DECOMMENTA se vuoi una strofa senza accordi
^Astro del ciel, Pargol divin, ^mite agnello ^redentor
^tu disceso a scon^tare l'error, \brk ^tu sol nato a par^lare d'amor.
\endverse
\endsong



