%-------------------------------------------------------------
%			INIZIO	CANZONE
%-------------------------------------------------------------


%titolo: 	Gli angeli delle campagne
%autore: 	
%tonalita: 	Do


%%%%%% TITOLO E IMPOSTAZONI
\beginsong{ Gli angeli delle campagne }[by={Les anges dans nos campagnes }] 	% <<< MODIFICA TITOLO E AUTORE
\transpose{0} 						% <<< TRASPOSIZIONE #TONI (0 nullo)
\momenti{Natale}							% <<< INSERISCI MOMENTI	
% momenti vanno separati da ; e vanno scelti tra:
% Ingresso; Atto penitenziale; Acclamazione al Vangelo; Dopo il Vangelo; Offertorio; Comunione; Ringraziamento; Fine; Santi; Pasqua; Avvento; Natale; Quaresima; Canti Mariani; Battesimo; Prima Comunione; Cresima; Matrimonio; Meditazione; Spezzare del pane;
\ifchorded
	\textnote{$\bigstar$ Tonalità singola }	% <<< EV COMMENTI (tonalità originale/migliore)
\fi

%%%%%% INTRODUZIONE
\ifchorded
\vspace*{\versesep}
\musicnote{
\begin{minipage}{0.48\textwidth}
\textbf{Intro}
\hfill 
%( \eighthnote \, 80)   % <<  MODIFICA IL TEMPO
% Metronomo: \eighthnote (ottavo) \quarternote (quarto) \halfnote (due quarti)
\end{minipage}
} 	
\vspace*{-\versesep}
\beginverse*
\nolyrics

%---- Prima riga -----------------------------
\vspace*{-\versesep}
\[C] \[G] \[C]  \rep{2} % \[*D] per indicare le pennate, \rep{2} le ripetizioni

%---- Ogni riga successiva -------------------
%\vspace*{-\versesep}
%\[G] \[C]  \[D]	

%---- Ev Indicazioni -------------------------			
%\textnote{\textit{(Oppure tutta la strofa)} }	

\endverse
\fi

%%%%% STROFA
\beginverse		%Oppure \beginverse* se non si vuole il numero di fianco
\memorize 		% <<< DECOMMENTA se si vuole utilizzarne la funzione
%\chordsoff		% <<< DECOMMENTA se vuoi una strofa senza accordi

\[C]Gli angeli del\[G]le cam\[C]pagne 
\[C]cantano l’inno "\[G]Gloria in ci\[C]el !"
\[C]e l’eco del\[G]le mon\[C]tagne 
\[C]ripete il canto \[G]dei fe\[C]del.
\endverse

%%%%% RITORNELLO
\beginchorus
\textnote{\textbf{Rit.}}

\[C]Glo  \[A]o  o  \[D-]o  \[G]o  o \[C]o \[F]o o \[D]o \[G7]ria         
\[C*]in \[G*]ex \[C*]cel \[F*]sis \[C]De \[G]o
\[C]Glo  \[A]o  o  \[D-]o  \[G]o  o \[C]o \[F]o o \[D]o \[G7]ria         
\[C*]in \[G*]ex \[C*]cel \[F*]sis \[C]De-\[G]e-\[C]o
\endchorus

%%%%% STROFA
\beginverse		%Oppure \beginverse* se non si vuole il numero di fianco
%\memorize 		% <<< DECOMMENTA se si vuole utilizzarne la funzione
%\chordsoff		% <<< DECOMMENTA se vuoi una strofa senza accordi

^O pastori ^che can^tate
^dite il perché di ^tanto o^nor.
^Qual Signor, qua^le pro^feta
^merita questo ^gran splen^dor.

\endverse

%%%%% STROFA
\beginverse		%Oppure \beginverse* se non si vuole il numero di fianco
%\memorize 		% <<< DECOMMENTA se si vuole utilizzarne la funzione
%\chordsoff		% <<< DECOMMENTA se vuoi una strofa senza accordi

^Oggi è nato in ^una s^talla
^nella notturna os^curi^tà.
^Egli è il Verbo, ^s’è incar^nato
^e venne in questa ^pover^tà.

\endverse

\endsong
%------------------------------------------------------------
%			FINE CANZONE
%------------------------------------------------------------






%++++++++++++++++++++++++++++++++++++++++++++++++++++++++++++
%			CANZONE TRASPOSTA
%++++++++++++++++++++++++++++++++++++++++++++++++++++++++++++
\ifchorded
%decremento contatore per avere stesso numero
\addtocounter{songnum}{-1} 
\beginsong{Gli angeli delle campagne}[by={Les anges dans nos campagnes — Autore ignoto}] 	% <<< \transpose{0} 						% <<< TRASPOSIZIONE #TONI + - (0 nullo)
%\preferflats  %SE VOGLIO FORZARE i bemolle come alterazioni
%\prefersharps %SE VOGLIO FORZARE i # come alterazioni
\ifchorded
	\textnote{$\triangle$ Tonalità crescente}	% <<< EV COMMENTI (tonalità originale/migliore)
\fi

%%%%%% INTRODUZIONE
\ifchorded
\vspace*{\versesep}
\musicnote{
\begin{minipage}{0.48\textwidth}
\textbf{Intro}
\hfill 
%( \eighthnote \, 80)   % <<  MODIFICA IL TEMPO
% Metronomo: \eighthnote (ottavo) \quarternote (quarto) \halfnote (due quarti)
\end{minipage}
} 	
\vspace*{-\versesep}
\beginverse*
\nolyrics

%---- Prima riga -----------------------------
\vspace*{-\versesep}
\[C] \[G] \[C]  \rep{2} % \[*D] per indicare le pennate, \rep{2} le ripetizioni

%---- Ogni riga successiva -------------------
%\vspace*{-\versesep}
%\[G] \[C]  \[D]	

%---- Ev Indicazioni -------------------------			
%\textnote{\textit{(Oppure tutta la strofa)} }	

\endverse
\fi

%%%%% STROFA
\beginverse		%Oppure \beginverse* se non si vuole il numero di fianco
\memorize 		% <<< DECOMMENTA se si vuole utilizzarne la funzione
%\chordsoff		% <<< DECOMMENTA se vuoi una strofa senza accordi

\[C]Gli angeli del\[G]le cam\[C]pagne 
\[C]cantano l’inno "\[G]Gloria in ci\[C]el !"
\[C]e l’eco del\[G]le mon\[C]tagne 
\[C]ripete il canto \[G]dei fe\[C]del.
\endverse

%%%%% RITORNELLO
\beginchorus
\textnote{\textbf{Rit.}}

\[C]Glo  \[A]o  o  \[D-]o  \[G]o  o \[C]o \[F]o o \[D]o \[G7]ria         
\[C*]in \[G*]ex \[C*]cel \[F*]sis \[C]De \[G]o
\[C]Glo  \[A]o  o  \[D-]o  \[G]o  o \[C]o \[F]o o \[D]o \[G7]ria         
\[C*]in \[G*]ex \[C*]cel \[F*]sis \[C]De-\[G]e-\[C]o \qquad \[A]
\endchorus


%%%%%% EV. INTERMEZZO
\beginverse*
\vspace*{1.3\versesep}
{
	\nolyrics
	\textnote{Intermezzo strumentale}
	
	\ifchorded

	%---- Prima riga -----------------------------
	\vspace*{-\versesep}
	\[D] \[A]  \[D]	 \rep{2}

	%---- Ogni riga successiva -------------------
	%\vspace*{-\versesep}
	%\[G] \[C]  \[D]	


	\fi
	%---- Ev Indicazioni -------------------------			
	%\textnote{\textit{(ripetizione della strofa)}} 
	 
}
\vspace*{\versesep}
\endverse
%%%%% STROFA
\beginverse		%Oppure \beginverse* se non si vuole il numero di fianco
%\memorize 		% <<< DECOMMENTA se si vuole utilizzarne la funzione
%\chordsoff		% <<< DECOMMENTA se vuoi una strofa senza accordi
\transpose{2}

^O pastori ^che can^tate
^dite il perché di ^tanto o^nor.
^Qual Signor, qua^le pro^feta
^merita questo ^gran splen^dor.

\endverse


%%%%% RITORNELLO
\beginchorus
\textnote{\textbf{Rit.}}
\transpose{2}
\[C]Glo  \[A]o  o  \[D-]o  \[G]o  o \[C]o \[F]o o \[D]o \[G7]ria         
\[C*]in \[G*]ex \[C*]cel \[F*]sis \[C]De \[G]o
\[C]Glo  \[A]o  o  \[D-]o  \[G]o  o \[C]o \[F]o o \[D]o \[G7]ria         
\[C*]in \[G*]ex \[C*]cel \[F*]sis \[C]De-\[G]e-\[C]o \qquad \[A]
\endchorus


%%%%%% EV. INTERMEZZO
\beginverse*
\vspace*{1.3\versesep}
{
	\nolyrics
	\textnote{Intermezzo strumentale}
	
	\ifchorded

	%---- Prima riga -----------------------------
	\vspace*{-\versesep}
	\[E] \[B]  \[E]	 \rep{2}

	%---- Ogni riga successiva -------------------
	%\vspace*{-\versesep}
	%\[G] \[C]  \[D]	


	\fi
	%---- Ev Indicazioni -------------------------			
	%\textnote{\textit{(ripetizione della strofa)}} 
	 
}
\vspace*{\versesep}
\endverse

%%%%% STROFA
\beginverse		%Oppure \beginverse* se non si vuole il numero di fianco
%\memorize 		% <<< DECOMMENTA se si vuole utilizzarne la funzione
%\chordsoff		% <<< DECOMMENTA se vuoi una strofa senza accordi
\transpose{4}
^Oggi è nato in ^una s^talla
^nella notturna os^curi^tà.
^Egli è il Verbo, ^s’è incar^nato
^e venne in questa ^pover^tà.

\endverse

%%%%% RITORNELLO
\beginchorus
\textnote{\textbf{Rit.}}
\transpose{4}
\[C]Glo  \[A]o  o  \[D-]o  \[G]o  o \[C]o \[F]o o \[D]o \[G7]ria         
\[C*]in \[G*]ex \[C*]cel \[F*]sis \[C]De \[G]o
\[C]Glo  \[A]o  o  \[D-]o  \[G]o  o \[C]o \[F]o o \[D]o \[G7]ria         
\[C*]in \[G*]ex \[C*]cel \[F*]sis \[C]De-\[G]e-\[C]o!
\endchorus

\endsong
\fi
%++++++++++++++++++++++++++++++++++++++++++++++++++++++++++++
%			FINE CANZONE TRASPOSTA
%++++++++++++++++++++++++++++++++++++++++++++++++++++++++++++
