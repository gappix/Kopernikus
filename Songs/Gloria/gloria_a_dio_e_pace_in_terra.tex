%-------------------------------------------------------------
%			INIZIO	CANZONE
%-------------------------------------------------------------


%titolo: 	GGloria a Dio e pace in terra
%autore: 	A. Monti
%tonalita: 	Fa



%%%%%% TITOLO E IMPOSTAZONI
\beginsong{Gloria a Dio e pace in terra}[by={A. Monti}] 	% <<< MODIFICA TITOLO E AUTORE
\transpose{0} 						% <<< TRASPOSIZIONE #TONI (0 nullo)
\momenti{Gloria}							% <<< INSERISCI MOMENTI	
% momenti vanno separati da ; e vanno scelti tra:
% Ingresso; Atto penitenziale; Acclamazione al Vangelo; Dopo il Vangelo; Offertorio; Comunione; Ringraziamento; Fine; Santi; Pasqua; Avvento; Natale; Quaresima; Canti Mariani; Battesimo; Prima Comunione; Cresima; Matrimonio; Meditazione; Spezzare del pane;
\ifchorded
	\textnote{$\bigstar$ Tonalità migliore  }	% <<< EV COMMENTI (tonalità originale/migliore)
\fi


%%%%%% INTRODUZIONE
\ifchorded
\vspace*{\versesep}
\musicnote{
\begin{minipage}{0.48\textwidth}
\textbf{Intro}
\hfill 
%( \eighthnote \, 80)   % <<  MODIFICA IL TEMPO
% Metronomo: \eighthnote (ottavo) \quarternote (quarto) \halfnote (due quarti)
\end{minipage}
} 	
\vspace*{-\versesep}
\beginverse*

\nolyrics

%---- Prima riga -----------------------------
\vspace*{-\versesep}
 \[F] \[C] \[F] \[F*]



%---- Ev Indicazioni -------------------------			
%\textnote{\textit{(Come la seconda parte della prima strofa)} }	

\endverse
\fi


%%%%% RITORNELLO
\beginchorus
\textnote{\textbf{Rit.}}
\[F]Gloria a Dio e \[C]pace in terra
a \[D-]chi ha riconosc\[A-]iuto
il \[B&]grande A\[C]more di un \[F]Dio \[D]
\[G-]che \[C]non ha a\[F]vuto pa\[B&]ura
 \[G-]a farsi \[C]uomo per \[F]noi, \[D]
\[G-]e che ch\[C]iede ad \[F]ogni crea\[B&]tura
\[G-]di farsi n\[C]uova ogni \[F]giorno. \[A]
\endchorus



%%%%% STROFA
\beginverse
\memorize 
\[D-]Un amore i\[A-]naspet\[D-]tato 
\[D-]silenzioso \[A-]si ri\[F]vela
\[C]spirito di \[F]vita \[C]e di veri\[D-]tà;
\endverse
\beginverse*
^è il progetto ^per il ^mondo, 
^è il disegno ^del cre^ato,
^segno di un mis^tero ^vivo in mezzo a ^noi. \[C]
\endverse



%%%%% STROFA
\beginverse
^Voce lieve ^mai ud^ita 
^porta all’uomo ^l’uomo ^vero:
^chi vorrà accet^tare ^con lui reste^rà.
\endverse
\beginverse*
^Corre l’eco ^tra le ^valli 
^trasportando ^la no^tizia:
^lode a chi ha do^nato ^gioia a tutti ^noi!  \[C]
\endverse


%%%%% STROFA
\beginverse
^Luce chiara ^nella ^notte, 
^alba senza ^più tra^monto,
^tenda del per^dono ^e di liber^tà.
\endverse
\beginverse*
^Trova forza ^lo scon^fitto, 
^trova senso ^la na^tura,
^nella via di ^Cristo ^vivo in mezzo a ^noi. \[C]
\endverse


\ifchorded
%%%%% RITORNELLO
\beginchorus
\textnote{\textbf{Rit.}}
\[F]Gloria a Dio e \[C]pace in terra
a \[D-]chi ha riconosc\[A-]iuto
il \[B&]grande A\[C]more di un \[F]Dio \[D]
\[G-]che \[C]non ha a\[F]vuto pa\[B&]ura
 \[G-]a farsi \[C]uomo per \[F]noi, \[D]
\[G-]e che ch\[C]iede ad \[F]ogni crea\[B&]tura
\[G-]di farsi n\[C]uova ogni \[F]giorno. \[F*]
\endchorus
\fi
\endsong
%------------------------------------------------------------
%			FINE CANZONE
%------------------------------------------------------------


