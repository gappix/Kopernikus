%-------------------------------------------------------------
%			INIZIO	CANZONE
%-------------------------------------------------------------


%titolo: 	Francesco vai
%autore: 	M. C. Bizzeti
%tonalita: 	Mi-



%%%%%% TITOLO E IMPOSTAZONI
\beginsong{Francesco vai}[by={M. C. Bizzeti}]
\transpose{0} 						% <<< TRASPOSIZIONE #TONI (0 nullo)
%\preferflats  %SE VOGLIO FORZARE i bemolle come alterazioni
%\prefersharps %SE VOGLIO FORZARE i # come alterazioni
\momenti{Santi}							% <<< INSERISCI MOMENTI	
% momenti vanno separati da ; e vanno scelti tra:
% Ingresso; Atto penitenziale; Acclamazione al Vangelo; Dopo il Vangelo; Offertorio; Comunione; Ringraziamento; Fine; Santi; Pasqua; Avvento; Natale; Quaresima; Canti Mariani; Battesimo; Prima Comunione; Cresima; Matrimonio; Meditazione; Spezzare del pane;
\ifchorded
	%\textnote{Tonalità migliore }	% <<< EV COMMENTI (tonalità originale/migliore)
\fi





%%%%%% INTRODUZIONE
\ifchorded
\vspace*{\versesep}
\musicnote{
\begin{minipage}{0.48\textwidth}
\textbf{Intro}
\hfill 
%( \eighthnote \, 80)   % <<  MODIFICA IL TEMPO
% Metronomo: \eighthnote (ottavo) \quarternote (quarto) \halfnote (due quarti)
\end{minipage}
} 	
\vspace*{-\versesep}
\beginverse*

\nolyrics

%---- Prima riga -----------------------------
\vspace*{-\versesep}
\[E-] \[D] \[E-]	 % \[*D] per indicare le pennate, \rep{2} le ripetizioni

%---- Ogni riga successiva -------------------
%\vspace*{-\versesep}
%\[G] \[C]  \[D]	

%---- Ev Indicazioni -------------------------			
%\textnote{\textit{(Oppure tutta la strofa)} }	

\endverse
\fi





%%%%% STROFA
\beginverse		%Oppure \beginverse* se non si vuole il numero di fianco
\memorize 		% <<< DECOMMENTA se si vuole utilizzarne la funzione
%\chordsoff		% <<< DECOMMENTA se vuoi una strofa senza accordi
\[E-]Quello che io vivo non mi \[D]basta \[E-]più,
tutto quel che avevo non mi \[D]serve \[E-]più:
io cerche\[B]rò quello che dav\[E-]vero vale,
e non più il \[A-]servo, ma il pa\[C]drone segui\[B]rò!
\endverse




%%%%% RITORNELLO
\beginchorus
\textnote{\textbf{Rit.}}
Francesco, \[E-]vai, ri\[D]para la mia \[E-]casa!
Fran\[D]cesco, \[E-]vai, non \[D]vedi che è in ro\[G]vina?
E non te\[A-]mere: \[C]io sarò con \[G*]te do\[B*]vunque an\[E-]drai.
\[D]Francesco, \[E-]vai!
\endchorus



%%%%% STROFA
\beginverse		%Oppure \beginverse* se non si vuole il numero di fianco
%\memorize 		% <<< DECOMMENTA se si vuole utilizzarne la funzione
%\chordsoff		% <<< DECOMMENTA se vuoi una strofa senza accordi
Nel ^buio e nel silenzio ti ho cer^cato, ^Dio;
dal fondo della notte ho alzato il ^grido ^mio
e gride^rò finché non a^vrò risposta
per co^noscere la ^tua volon^tà.
\endverse


%%%%% STROFA
\beginverse		%Oppure \beginverse* se non si vuole il numero di fianco
%\memorize 		% <<< DECOMMENTA se si vuole utilizzarne la funzione
\chordsoff		% <<< DECOMMENTA se vuoi una strofa senza accordi
Al^tissimo Signore, cosa ^vuoi da ^me?
Tutto quel che avevo l'ho do^nato a ^te.
Ti segui^rò nella gioia e ^nel dolore
e della ^vita mia una ^lode a te fa^rò.
\endverse

%%%%% STROFA
\beginverse		%Oppure \beginverse* se non si vuole il numero di fianco
%\memorize 		% <<< DECOMMENTA se si vuole utilizzarne la funzione
\chordsoff		% <<< DECOMMENTA se vuoi una strofa senza accordi
^Quello che cercavo l'ho tro^vato ^qui:
ora ho riscoperto nel mio ^dirti ^sì
la liber^tà di essere ^figlio tuo,
fratello e ^sposo di Ma^donna pover^tà.
\endverse



\endsong
%------------------------------------------------------------
%			FINE CANZONE
%------------------------------------------------------------


