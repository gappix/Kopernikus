%-------------------------------------------------------------
%			INIZIO	CANZONE
%-------------------------------------------------------------


%titolo: 	Fratello Sole, sorella Luna
%autore: 	J. M. Benjamin, R. Ortolani
%tonalita: 	Do



%%%%%% TITOLO E IMPOSTAZONI
\beginsong{Fratello Sole, sorella Luna}[by={J. M. Benjamin, R. Ortolani}] 	% <<< MODIFICA TITOLO E AUTORE
\transpose{0} 						% <<< TRASPOSIZIONE #TONI (0 nullo)
\momenti{Comunione; Ringraziamento; Meditazione}							% <<< INSERISCI MOMENTI	
% momenti vanno separati da ; e vanno scelti tra:
% Ingresso; Atto penitenziale; Acclamazione al Vangelo; Dopo il Vangelo; Offertorio; Comunione; Ringraziamento; Fine; Santi; Pasqua; Avvento; Natale; Quaresima; Canti Mariani; Battesimo; Prima Comunione; Cresima; Matrimonio; Meditazione; Spezzare del pane;
\ifchorded
	%\textnote{Tonalità migliore }	% <<< EV COMMENTI (tonalità originale/migliore)
\fi



%%%%%% INTRODUZIONE
\ifchorded
\vspace*{\versesep}
\musicnote{
\begin{minipage}{0.48\textwidth}
\textbf{Intro}
\hfill 
%( \eighthnote \, 80)   % <<  MODIFICA IL TEMPO
% Metronomo: \eighthnote (ottavo) \quarternote (quarto) \halfnote (due quarti)
\end{minipage}
} 	
\vspace*{-\versesep}
\beginverse*


\nolyrics

%---- Prima riga -----------------------------
\vspace*{-\versesep}
\[C]\[A-*] \[F*] \[E-] % \[*D] per indicare le pennate, \rep{2} le ripetizioni

%---- Ogni riga successiva -------------------
\vspace*{-\versesep}
\[F] \[G]  \[C]	

%---- Ev Indicazioni -------------------------			
%\textnote{\textit{(Oppure tutta la strofa)} }	

\endverse
\fi



%%%%% STROFA
\beginverse*		%Oppure \beginverse* se non si vuole il numero di fianco
\memorize 		% <<< DECOMMENTA se si vuole utilizzarne la funzione
\[C]Dol\[A-*]ce \[F*]sen\[E-]tire \brk \[F]come \[G]nel mio \[C]cuore,
\[A-]o\[E-*]ra u\[F*]mil\[E-]men\[A-]te, \brk \[D-]sta nas\[7]cendo a\[G]more.
\endverse


%%%%% STROFA
\beginverse*		%Oppure \beginverse* se non si vuole il numero di fianco
%\memorize 		% <<< DECOMMENTA se si vuole utilizzarne la funzione
%\vspace*{-\versesep}
^Dol^ce ^ca^pire \brk ^che non ^son piu’ ^solo
^ma ^che ^son ^par^te \brk ^di una im\[G]mensa \[C]vita,
\endverse





%%%%% STROFA
\beginverse*	%Oppure \beginverse* se non si vuole il numero di fianco
%\memorize 		% <<< DECOMMENTA se si vuole utilizzarne la funzione
\[A-] che \[F-]gene\[C]rosa \brk ris\[F6]plende in\[G9]torno a \[C]me:
\[A-] do\[F-]no di \[C]Lui \brk del \[F]suo im\[F6/G]menso a\[C]more.

\endverse



%%%%% STROFA
\beginverse*		%Oppure \beginverse* se non si vuole il numero di fianco
%\memorize 		% <<< DECOMMENTA se si vuole utilizzarne la funzione
^Ci ha ^da^to il ^cielo \brk ^e le ^chiare ^stelle
^fra^tel^lo ^so^le \brk ^e so^rella ^luna;
\endverse


%%%%% STROFA
\beginverse*		%Oppure \beginverse* se non si vuole il numero di fianco
%\memorize 		% <<< DECOMMENTA se si vuole utilizzarne la funzione
%\vspace*{-\versesep}
^la ^ma^dre ^terra \brk con ^frutti, ^prati e ^fiori
^il ^fuo^co, il ^ven^to, \brk ^l’aria e ^l’acqua ^pura
\[A-]fon\[E-*]te \[F*]di \[E-]vi\[A-]ta, \brk \[D-]per le \[G]sue crea\[C]ture
\endverse



%%%%% STROFA
\beginverse*		%Oppure \beginverse* se non si vuole il numero di fianco
%\memorize 		% <<< DECOMMENTA se si vuole utilizzarne la funzione
\[A-] do\[F-]no di \[C]Lui \brk del \[F6]suo im\[G9]menso a\[C]more
\[A-] do\[F-]no di \[C]Lui \brk del \[F]suo im\[F6/G]menso a\[C]more.
\endverse



\endsong
%------------------------------------------------------------
%			FINE CANZONE
%------------------------------------------------------------