%-------------------------------------------------------------
%			INIZIO	CANZONE
%-------------------------------------------------------------


%titolo: 	Conte faremo cose grandi
%autore: 	Meregalli
%tonalita: 	Fa



%%%%%% TITOLO E IMPOSTAZONI
\beginsong{Con Te faremo cose grandi}[by={G. Meregalli}]	% <<< MODIFICA TITOLO E AUTORE
\transpose{0} 						% <<< TRASPOSIZIONE #TONI (0 nullo)
\momenti{Ingresso; Avvento}							% <<< INSERISCI MOMENTI	
% momenti vanno separati da ; e vanno scelti tra:
% Ingresso; Atto penitenziale; Acclamazione al Vangelo; Dopo il Vangelo; Offertorio; Comunione; Ringraziamento; Fine; Santi; Pasqua; Avvento; Natale; Quaresima; Canti Mariani; Battesimo; Prima Comunione; Cresima; Matrimonio; Meditazione; Spezzare del pane;
\ifchorded
	%\textnote{Tonalità migliore }	% <<< EV COMMENTI (tonalità originale/migliore)
\fi



%%%%%% INTRODUZIONE
\ifchorded
\vspace*{\versesep}
\musicnote{
\begin{minipage}{0.48\textwidth}
\textbf{Intro}
\hfill 
%( \eighthnote \, 80)   % <<  MODIFICA IL TEMPO
% Metronomo: \eighthnote (ottavo) \quarternote (quarto) \halfnote (due quarti)
\end{minipage}
} 	
\vspace*{-\versesep}
\beginverse*

\nolyrics

%---- Prima riga -----------------------------
\vspace*{-\versesep}
\[F] \[C] \[F]		 % \[*D] per indicare le pennate, \rep{2} le ripetizioni

%---- Ogni riga successiva -------------------
%\vspace*{-\versesep}
%\[G] \[C]  \[D]	

%---- Ev Indicazioni -------------------------			
%\textnote{\textit{(Oppure tutta la strofa)} }	

\endverse
\fi



%%%%% RITORNELLO
\beginchorus
\textnote{\textbf{Rit.}}
Con \[F]Te fa\[C]remo cose \[D-]gran\[G]di
il cam\[B&]mino che per\[C]correremo in\[A-]sie\[C]me
\[C7]di \[F]Te si \[C]riempiranno \[D-]sguar\[G]di
la spe\[B&]ranza che ri\[C]splenderà nei \[A-]vol\[C]ti.
\endchorus


\beginchorus
Tu la \[B&]luce che ri\[C]schiara,
Tu la \[B&]voce che ci \[C]chiama,
Tu la \[B&]gioia che dà \[C]vita ai nostri \[A-]so\[C]gni.
\endchorus


\beginverse
\memorize
\[F]Parlaci Si\[C]gnore come \[B&]sai,
sei pre\[C]sente nel mi\[C7]stero in mezzo a \[F]noi. \[C]
\[F]Chiamaci col \[C]nome che vor\[B&]rai
e sia \[C]fatto il tuo di\[C7]segno su di \[F]noi. \[C]
\endverse

\beginchorus
Tu la \[B&]luce che ri\[C]schiara,
Tu la \[B&]voce che ci \[C]chiama,
Tu la \[B&]gioia che dà \[C]vita ai nostri \[A-]so\[C]gni.
\endchorus

%%%%% RITORNELLO
\beginchorus
\textnote{\textbf{Rit.}}
Con \[F]Te fa\[C]remo cose \[D-]gran\[G]di
il cam\[B&]mino che per\[C]correremo in\[A-]sie\[C]me
\[C7]di \[F]Te si \[C]riempiranno \[D-]sguar\[G]di
la spe\[B&]ranza che ri\[C]splenderà nei \[A-]vol\[C]ti.
\endchorus


\beginchorus
Tu l'a\[B&]more che dà \[C]vita,
Tu il sor\[B&]riso che ci al\[C]lieta,
Tu la \[B&]forza che ra\[C]duna i nostri \[A-]gior\[C]ni.
\endchorus

\beginverse
^Guidaci Si^gnore dove ^sai
da chi ^soffre chi è più ^piccolo di ^noi ^
stru^menti di quel ^regno che Tu ^fai,
di quel ^regno che ora ^vive in mezzo a ^noi. ^
\endverse


\beginchorus
Tu l'a\[B&]more che dà \[C]vita,
Tu il sor\[B&]riso che ci al\[C]lieta,
Tu la \[B&]forza che ra\[C]duna i nostri \[A-]gior\[C]ni.
\endchorus

%%%%% RITORNELLO
\beginchorus
\textnote{\textbf{Rit.}}
Con \[F]Te fa\[C]remo cose \[D-]gran\[G]di
il cam\[B&]mino che per\[C]correremo in\[A-]sie\[C]me
\[C7]di \[F]Te si \[C]riempiranno \[D-]sguar\[G]di
la spe\[B&]ranza che ri\[C]splenderà nei \[A-]vol\[C]ti.
\endchorus

\beginchorus %oppure \beginverse*
\vspace*{1.3\versesep}
\textnote{\textbf{Finale} } %<<< EV. INDICAZIONI
Tu l'am\[B&]ore che dà \[C]vita,
Tu il sor\[B&]riso che ci al\[C]lieta,
Tu la \[B&]forza che ra\[C]duna i nostri \[B&]gio-o-\[B&]o-or\[F]ni.
\endchorus



\endsong
%------------------------------------------------------------
%			FINE CANZONE
%------------------------------------------------------------




