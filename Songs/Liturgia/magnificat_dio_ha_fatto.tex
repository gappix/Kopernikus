%-------------------------------------------------------------
%			INIZIO	CANZONE
%-------------------------------------------------------------


%titolo: 	Magnificat (Dio ha fatto)
%autore: 	Gragnani, Casini, Ciardella
%tonalita: 	Sol 



%%%%%% TITOLO E IMPOSTAZONI
\beginsong{Magnificat Dio ha fatto}[by={M. Gragnani, G. Casini, P. Ciardella}]	% <<< MODIFICA TITOLO E AUTORE
\transpose{0} 						% <<< TRASPOSIZIONE #TONI (0 nullo)
\momenti{Canti Mariani;}							% <<< INSERISCI MOMENTI	
% momenti vanno separati da ; e vanno scelti tra:
% Ingresso; Atto penitenziale; Acclamazione al Vangelo; Dopo il Vangelo; Offertorio; 
%Comunione; Ringraziamento; Fine; Santi; Pasqua; Avvento; Natale; Quaresima; Canti Mariani; Battesimo; Prima Comunione; Cresima; Matrimonio; Meditazione; Spezzare del pane;
\ifchorded
	%\textnote{Tonalità migliore }	% <<< EV COMMENTI (tonalità originale/migliore)
\fi


%%%%%% INTRODUZIONE
\ifchorded
\vspace*{\versesep}
\musicnote{
\begin{minipage}{0.48\textwidth}
\textbf{Intro}
\hfill 
%( \eighthnote \, 80)   % <<  MODIFICA IL TEMPO
% Metronomo: \eighthnote (ottavo) \quarternote (quarto) \halfnote (due quarti)
\end{minipage}
} 	
\vspace*{-\versesep}
\beginverse*
\nolyrics

%---- Prima riga -----------------------------
\vspace*{-\versesep}
\[G] \[C] \[G]	 % \[*D] per indicare le pennate, \rep{2} le ripetizioni

%---- Ogni riga successiva -------------------
%\vspace*{-\versesep}
%\[G] \[C]  \[D]	

%---- Ev Indicazioni -------------------------			
%\textnote{\textit{(Oppure tutta la strofa)} }	

\endverse
\fi




%%%%% STROFA
\beginverse		%Oppure \beginverse* se non si vuole il numero di fianco
\memorize 		% <<< DECOMMENTA se si vuole utilizzarne la funzione
%\chordsoff		% <<< DECOMMENTA se vuoi una strofa senza accordi
\[G]Dio \[C7+] ha fatto in \[B-7]me cose \[E-7]grandi.
\[C7+]Lui \[A-7] che guarda l'\[D]umile \[G7+]servo
e di\[C7+]sperde i su\[A-6]perbi
nell'or\[B7]goglio del \[E-]cuore-\[E]e.
\endverse



%%%%% RITORNELLO
\beginchorus
\textnote{\textbf{Rit.}}
\[A-]L'a-\[D7]anima \[G7+]mia \[C7+] esulta in \[A-]Dio 
\[B7] mio salva\[E-]tore-\[E]e.
\[A-]L'a-\[D7]anima \[G7+]mia \[C7+] esulta in \[A-]Dio 
\[B7] mio salva\[E-]tore. 
\[C] La sua sal\[A-]vezza \[D]cante\[G]rò.
\endchorus



%%%%% STROFA
\beginverse		%Oppure \beginverse* se non si vuole il numero di fianco
%\memorize 		% <<< DECOMMENTA se si vuole utilizzarne la funzione
%\chordsoff		% <<< DECOMMENTA se vuoi una strofa senza accordi
^Lui, ^ Onnipo^tente e ^santo.
^Lui ^ abbatte i ^grandi dai ^troni 
e sol^leva dal ^fango il suo ^umile ^servo-^o. 
\endverse





%%%%% STROFA
\beginverse		%Oppure \beginverse* se non si vuole il numero di fianco
%\memorize 		% <<< DECOMMENTA se si vuole utilizzarne la funzione
%\chordsoff		% <<< DECOMMENTA se vuoi una strofa senza accordi
^Lui ^ miseri^cordia infi^nita,
^Lui ^ che rende ^povero il ^ricco 
e ri^colma di ^beni chi si af^fida al suo a^more-^e. 
\endverse




%%%%% STROFA
\beginverse		%Oppure \beginverse* se non si vuole il numero di fianco
%\memorize 		% <<< DECOMMENTA se si vuole utilizzarne la funzione
%\chordsoff		% <<< DECOMMENTA se vuoi una strofa senza accordi
^Lui, ^ Amore ^sempre fe^dele.
^Lui ^ guida il suo ^servo Isra^ele 
e ri^corda il suo ^patto stabi^lito per ^sempre-^e. 
\endverse



\endsong
%------------------------------------------------------------
%			FINE CANZONE
%------------------------------------------------------------



