%-------------------------------------------------------------
%			INIZIO	CANZONE
%-------------------------------------------------------------


%titolo: 	Il mattino di Pasqua
%autore: 	Sequeri
%tonalita: 	Do



%%%%%% TITOLO E IMPOSTAZONI
\beginsong{Il mattino di Pasqua}[by={P. Sequeri}] 	% <<< MODIFICA TITOLO E AUTORE
\transpose{-2} 						% <<< TRASPOSIZIONE #TONI (0 nullo)
\momenti{Pasqua}							% <<< INSERISCI MOMENTI	
% momenti vanno separati da ; e vanno scelti tra:
% Ingresso; Atto penitenziale; Acclamazione al Vangelo; Dopo il Vangelo; Offertorio; Comunione; Ringraziamento; Fine; Santi; Pasqua; Avvento; Natale; Quaresima; Canti Mariani; Battesimo; Prima Comunione; Cresima; Matrimonio; Meditazione; Spezzare del pane;
\ifchorded
	%\textnote{Tonalità migliore }	% <<< EV COMMENTI (tonalità originale/migliore)
\fi


%%%%%% INTRODUZIONE
\ifchorded
\vspace*{\versesep}
\musicnote{
\begin{minipage}{0.48\textwidth}
\textbf{Intro}
\hfill 
%( \eighthnote \, 80)   % <<  MODIFICA IL TEMPO
% Metronomo: \eighthnote (ottavo) \quarternote (quarto) \halfnote (due quarti)
\end{minipage}
} 	
\vspace*{-\versesep}
\beginverse*

\nolyrics

%---- Prima riga -----------------------------
\vspace*{-\versesep}
\[D] \[F#-] \[B-]\[E]  % \[*D] per indicare le pennate, \rep{2} le ripetizioni

%---- Ogni riga successiva -------------------
\vspace*{-\versesep}
 \[A*] \[E*] \[A]

%---- Ev Indicazioni -------------------------			
%\textnote{\textit{(Oppure tutta la strofa)} }	

\endverse
\fi



%%%%% RITORNELLO
\beginchorus
\textnote{\textbf{Rit.}}
\[D] II Signore è ri\[F#-]sorto: cantate con \[B-]noi!
egli ha vinto la \[E]morte, 
al\[A*]le\[E*]lu\[A]ia! \quad \[A]
\[D] Allelu-u\[F#-]ia! Allelu-u\[B-]ia!
Allelu-u\[E]ia!
Al\[A*]le\[E*]lu\[A]ia! \quad \[A]
\endchorus



%%%%% STROFA
\beginverse		%Oppure \beginverse* se non si vuole il numero di fianco
\memorize 		% <<< DECOMMENTA se si vuole utilizzarne la funzione
%\chordsoff		% <<< DECOMMENTA se vuoi una strofa senza accordi
\textnote{\textit{(cambia la velocità)}}
\[D] II mattino di \[F#-]Pasqua, \brk nel ricordo di \[B-]lui,
siamo andate al se\[E]polcro: \brk non era più \[A7]là! \quad \[A7]
\[D] Senza nulla spe\[F#-]rare, \brk con il cuore sos\[B-]peso,
siamo andati al se\[E]polcro: \brk non era più \[A7]là! \quad \[A7]
\endverse






%%%%% STROFA
\beginverse		%Oppure \beginverse* se non si vuole il numero di fianco
%\memorize 		% <<< DECOMMENTA se si vuole utilizzarne la funzione
%\chordsoff		% <<< DECOMMENTA se vuoi una strofa senza accordi
^ Sulla strada di ^casa \brk parlavamo di ^lui
e l’abbiamo incon^trato: \brk ha mangiato con ^noi! \quad ^
^ Sulle rive del ^lago \brk pensavamo a quei ^giorni
e l’abbiamo incon^trato: \brk ha mangiato con ^noi! \quad ^
\endverse




%%%%% STROFA
\beginverse		%Oppure \beginverse* se non si vuole il numero di fianco
%\memorize 		% <<< DECOMMENTA se si vuole utilizzarne la funzione
\chordsoff		% <<< DECOMMENTA se vuoi una strofa senza accordi
Oggi ancora fratelli, \brk ricordando quei giorni,
ascoltiamo la voce \brk del Signore tra noi!
E, spezzando il suo pane \brk con la gioia nel cuore
noi cantiamo alla vita \brk nell’attesa di lui!
\endverse



\endsong
%------------------------------------------------------------
%			FINE CANZONE
%------------------------------------------------------------


