%-------------------------------------------------------------
%			INIZIO	CANZONE
%-------------------------------------------------------------


%titolo: 	Saldo è il mio cuore
%autore: 	M. Frisina
%tonalita:  Re 
%youtube: https://www.youtube.com/watch?v=cG_-iab1xcI&feature=youtu.be



%%%%%% TITOLO E IMPOSTAZONI
\beginsong{Saldo è il mio cuore}[by={M. Frisina}] 	% <<< MODIFICA TITOLO E AUTORE
\transpose{0} 						% <<< TRASPOSIZIONE #TONI (0 nullo)
%\preferflats  %SE VOGLIO FORZARE i bemolle come alterazioni
%\prefersharps %SE VOGLIO FORZARE i # come alterazioni
\momenti{Ringraziamento}							% <<< INSERISCI MOMENTI	
% momenti vanno separati da ; e vanno scelti tra:
% Ingresso; Atto penitenziale; Acclamazione al Vangelo; Dopo il Vangelo; Offertorio; Comunione; Ringraziamento; Fine; Santi; Pasqua; Avvento; Natale; Quaresima; Canti Mariani; Battesimo; Prima Comunione; Cresima; Matrimonio; Meditazione; Spezzare del pane;
\ifchorded
	%\textnote{Tonalità migliore }	% <<< EV COMMENTI (tonalità originale/migliore)
\fi


%%%%%% INTRODUZIONE
\ifchorded
\vspace*{\versesep}
\musicnote{
\begin{minipage}{0.48\textwidth}
\textbf{Intro}
\hfill 
%( \eighthnote \, 80)   % <<  MODIFICA IL TEMPO
% Metronomo: \eighthnote (ottavo) \quarternote (quarto) \halfnote (due quarti)
\end{minipage}
} 	
\vspace*{-\versesep}
\beginverse*

\nolyrics

%---- Prima riga -----------------------------
\vspace*{-\versesep}
\[D] \[G] \quad \[D]\[A]\[D]	 % \[*D] per indicare le pennate, \rep{2} le ripetizioni

%---- Ogni riga successiva -------------------
%\vspace*{-\versesep}
%\[G] \[C]  \[D]	

%---- Ev Indicazioni -------------------------			
%\textnote{\textit{(Oppure tutta la strofa)} }	

\endverse
\fi




%%%%% STROFA
\beginverse		%Oppure \beginverse* se non si vuole il numero di fianco
\memorize 		% <<< DECOMMENTA se si vuole utilizzarne la funzione
%\chordsoff		% <<< DECOMMENTA se vuoi una strofa senza accordi


\[D]Saldo è il mio \[G]cuore, Dio \[D]\[A]mi\[D]o.
A \[D]te cante\[G]rà l'anima \[A]\[G]mi\[A]a.
De\[D]statevi \[G]arpa e \[D]\[A]ce\[B-]tra,
\[G]vo\[A]glio sve\[B-]glia\[E-]re l'au\[D]\[A]ro\[D]ra. \quad \[A]

\endverse
\beginverse	
\textbf{A ^te la mia ^lode tra le ^^gen^ti,
per^chè fino ai ^cieli \brk è il tuo ^^amo^re.
^Sorgi ed in^nalzati, o ^^Di^o,
^splen^da sul ^mondo ^la tua ^^glo^ria.\quad ^
}
\endverse
\beginverse

Con ^te noi fa^remo cose ^^gran^di.
Con ^te noi con^vertiremo il ^^mon^do.
Tu ^sei nostra ^luce e con^^for^to,
^for^za, ri^fugio, ^o Si^^gno^re. \quad ^

\endverse
\beginverse
\textbf{^Per te noi an^dremo per il ^^mon^do,
^inni cante^remo alla tua ^^glo^ria.
^Donaci la ^grazia, Si^^gno^re,
^an^nunce^remo ^il tuo a^^mo^re. \quad \[D*]
}
\endverse





\endsong
%------------------------------------------------------------
%			FINE CANZONE
%------------------------------------------------------------


