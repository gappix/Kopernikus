%-------------------------------------------------------------
%			INIZIO	CANZONE
%-------------------------------------------------------------


%titolo: 	Come ti ama Dio
%autore: 	Anonimo
%tonalita:  DO



%%%%%% TITOLO E IMPOSTAZONI
\beginsong{Come ti ama Dio}[by={Anonimo}] 	% <<< MODIFICA TITOLO E AUTORE
\transpose{0} 						% <<< TRASPOSIZIONE #TONI (0 nullo)
%\preferflats  %SE VOGLIO FORZARE i bemolle come alterazioni
%\prefersharps %SE VOGLIO FORZARE i # come alterazioni
\momenti{Matrimonio}							% <<< INSERISCI MOMENTI	
% momenti vanno separati da ; e vanno scelti tra:
% Ingresso; Atto penitenziale; Acclamazione al Vangelo; Dopo il Vangelo; Offertorio; Comunione; Ringraziamento; Fine; Santi; Pasqua; Avvento; Natale; Quaresima; Canti Mariani; Battesimo; Prima Comunione; Cresima; Matrimonio; Meditazione; Spezzare del pane;
\ifchorded
	%\textnote{$\bigstar$ Tonalità migliore }	% <<< EV COMMENTI (tonalità originale\migliore)
\fi


%%%%%% INTRODUZIONE
\ifchorded
\vspace*{\versesep}
\musicnote{
\begin{minipage}{0.48\textwidth}
\textbf{Intro}
\hfill 
%( \eighthnote \, 80)   % <<  MODIFICA IL TEMPO
% Metronomo: \eighthnote (ottavo) \quarternote (quarto) \halfnote (due quarti)
\end{minipage}
} 	
\vspace*{-\versesep}
\beginverse*

\nolyrics

%---- Prima riga -----------------------------
\vspace*{-\versesep}
\[C] \[A-] \[F]	 \[G] \rep{2} % \[*D] per indicare le pennate, \rep{2} le ripetizioni

%---- Ogni riga successiva -------------------
%\vspace*{-\versesep}
%\[G] \[C]  \[D]	

%---- Ev Indicazioni -------------------------			
%\textnote{\textit{[oppure tutta la strofa]} }	

\endverse
\fi




%%%%% STROFA
\beginverse		%Oppure \beginverse* se non si vuole il numero di fianco
\memorize 		% <<< DECOMMENTA se si vuole utilizzarne la funzione
%\chordsoff		% <<< DECOMMENTA se vuoi una strofa senza accordi

\[C]Io vorrei sa\[A-]perti amare \[F]come Dio
\[G]che ti prende per \[C]mano ma ti \[A-]lascia anche \[F]andare.
\[G]Vorrei saperti \[C]amare senza \[A-]farti mai \[F]domande,
\[G]felice perchè \[C]esisti e co\[A-]sì io posso \[F]darti \brk il \[G]meglio di \[C]me.

\endverse




%%%%% RITORNELLO
\beginchorus
\textnote{\textbf{Rit.}}

Con la \[G]forza del \[A-]mare, \brk l'e\[F]ternità dei \[C]giorni,
la \[G]gioia dei \[A-]voli, \brk la \[F]pace della \[C]sera,
l'im\[G]mensità del \[A-]cielo: \brk \[F]come ti ama \[C]Dio.

\endchorus



%%%%% STROFA
\beginverse		%Oppure \beginverse* se non si vuole il numero di fianco
%\memorize 		% <<< DECOMMENTA se si vuole utilizzarne la funzione
%\chordsoff		% <<< DECOMMENTA se vuoi una strofa senza accordi

\[C]Io vorrei sa\[A-]perti amare \brk \[F]come ti ama \[G]Dio
che ti cono\[C]sce e ti \[A-]accetta \brk come \[F]sei.
\[G]Tenerti fra le \[C]mani \brk come \[A-]voli nell'\[F]azzurro,
\[G]felice perchè \[C]esisti e \[A-]così io posso \[F]darti \brk il \[G]meglio di \[C]me.
 

\endverse



%%%%% RITORNELLO
\beginchorus
\textnote{\textbf{Rit.}}

Con la \[G]forza del \[A-]mare, \brk l'e\[F]ternità dei \[C]giorni,
la \[G]gioia dei \[A-]voli, \brk la \[F]pace della \[C]sera,
l'im\[G]mensità del \[A-]cielo: \brk \[F]come ti ama \[C]Dio.

\endchorus



%%%%% STROFA
\beginverse		%Oppure \beginverse* se non si vuole il numero di fianco
%\memorize 		% <<< DECOMMENTA se si vuole utilizzarne la funzione
%\chordsoff		% <<< DECOMMENTA se vuoi una strofa senza accordi

\[C]Io vorrei sa\[A-]perti amare \brk \[F]come ti ama \[G]Dio
che ti fa mi\[C]gliore con l'a\[A-]more \brk che ti \[F]dona.
\[G]Seguirti fra la \[C]gente \brk con la \[A-]gioia che hai \[F]dentro,
\[G]felice perchè \[C]esisti e \[A-]così io posso \[F]darti \brk il \[G]meglio di \[C]me.
 

\endverse


%%%%% RITORNELLO
\beginchorus
\textnote{\textbf{Rit.}}

Con la \[G]forza del \[A-]mare, \brk l'e\[F]ternità dei \[C]giorni,
la \[G]gioia dei \[A-]voli,\brk  la \[F]pace della \[C]sera,
l'im\[G]mensità del \[A-]cielo: \brk \[F]come ti ama \[C]Dio.

\endchorus



\endsong
%------------------------------------------------------------
%			FINE CANZONE
%------------------------------------------------------------


