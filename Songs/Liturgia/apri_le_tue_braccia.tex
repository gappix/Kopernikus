%-------------------------------------------------------------
%			INIZIO	CANZONE
%-------------------------------------------------------------


%titolo: 	Apri le tue braccia
%autore: 	D. Machetta
%tonalita: 	Re-



%%%%%% TITOLO E IMPOSTAZONI
\beginsong{Apri le tue braccia}[by={D. Machetta}]
\transpose{2} 						% <<< TRASPOSIZIONE #TONI (0 nullo)
%\preferflats  %SE VOGLIO FORZARE i bemolle come alterazioni
%\prefersharps %SE VOGLIO FORZARE i # come alterazioni
\momenti{Atto penitenziale;}							% <<< INSERISCI MOMENTI	
% momenti vanno separati da ; e vanno scelti tra:
% Ingresso; Atto penitenziale; Acclamazione al Vangelo; Dopo il Vangelo; Offertorio; Comunione; Ringraziamento; Fine; Santi; Pasqua; Avvento; Natale; Quaresima; Canti Mariani; Battesimo; Prima Comunione; Cresima; Matrimonio; Meditazione; Spezzare del pane;
\ifchorded
	%\textnote{Tonalità migliore }	% <<< EV COMMENTI (tonalità originale/migliore)
\fi


%%%%%% INTRODUZIONE
\ifchorded
\vspace*{\versesep}
\musicnote{
\begin{minipage}{0.48\textwidth}
\textbf{Intro}
\hfill 
%( \eighthnote \, 80)   % <<  MODIFICA IL TEMPO
% Metronomo: \eighthnote (ottavo) \quarternote (quarto) \halfnote (due quarti)
\end{minipage}
} 	
\vspace*{-\versesep}
\beginverse*

\nolyrics

%---- Prima riga -----------------------------
\vspace*{-\versesep}
\[F] \[C] \[D-] \[A-]	 % \[*D] per indicare le pennate, \rep{2} le ripetizioni

%---- Ogni riga successiva -------------------
\vspace*{-\versesep}
\[B&] \[C]  \[F]	

%---- Ev Indicazioni -------------------------			
%\textnote{\textit{(Oppure tutta la strofa)} }	

\endverse
\fi



%%%%% STROFA
\beginverse		%Oppure \beginverse* se non si vuole il numero di fianco
\memorize 		% <<< DECOMMENTA se si vuole utilizzarne la funzione
%\chordsoff		% <<< DECOMMENTA se vuoi una strofa senza accordi
Hai cer\[D-]cato la \[C]libertà lon\[D-]tano,
hai tro\[A-]vato la \[E&]noia e le ca\[B&]tene;
hai va\[G-]ga\[D-]to \[G-]senza \[D-]via, \[G-]solo, \brk  \[E&] con la tua \[C]fame.
\endverse



%%%%% RITORNELLO
\beginchorus
\textnote{\textbf{Rit.}}
\[F]A\[C]pri le tue \[D-]brac\[A-]cia, \brk \[B&]corri in\[G-]contro al \[C]Padre;
\[F]oggi \[D] la sua \[G-]ca\[B&]sa \brk sarà in \[F]fe\[C7]sta per \[F]te.
\endchorus


%%%%% STROFA
\beginverse		%Oppure \beginverse* se non si vuole il numero di fianco
%\memorize 		% <<< DECOMMENTA se si vuole utilizzarne la funzione
%\chordsoff		% <<< DECOMMENTA se vuoi una strofa senza accordi
Se vor^rai spez^zare le ca^tene
trove^rai la ^strada dell'a^more;
la tua ^gio^ia ^cante^rai: \brk  ^questa ^ è liber^tà.
\endverse


\beginverse
\chordsoff
I tuoi ^occhi ri^cercano l'az^zurro;
c'è una ^casa che a^spetta il tuo ri^torno,
e la ^pa^ce ^torne^rà: \brk  ^questa ^ è liber^tà.
\endverse



\endsong
%------------------------------------------------------------
%			FINE CANZONE
%------------------------------------------------------------
