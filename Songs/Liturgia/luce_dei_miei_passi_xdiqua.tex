%-------------------------------------------------------------
%			INIZIO	CANZONE
%-------------------------------------------------------------


%titolo: 	Luce dei miei passi (XdiQua)
%autore: 	N. Cermenati, A. Motti, E. Porro, M. Violato, S. Tremolada
%tonalita: 	Re-



%%%%%% TITOLO E IMPOSTAZONI
\beginsong{Luce dei miei passi - XdiQua}[by={N. Cermenati, A. Motti, E. Porro, M. Violato, S. Tremolada}] 	% <<< MODIFICA TITOLO E AUTORE
\transpose{-2} 						% <<< TRASPOSIZIONE #TONI (0 nullo)
%\preferflats  %SE VOGLIO FORZARE i bemolle come alterazioni
%\prefersharps %SE VOGLIO FORZARE i # come alterazioni
\momenti{Ringraziamento; Congedo; Comunione}							% <<< INSERISCI MOMENTI	
% momenti vanno separati da ; e vanno scelti tra:
% Ingresso; Atto penitenziale; Acclamazione al Vangelo; Dopo il Vangelo; Offertorio; Comunione; Ringraziamento; Fine; Santi; Pasqua; Avvento; Natale; Quaresima; Canti Mariani; Battesimo; Prima Comunione; Cresima; Matrimonio; Meditazione; Spezzare del pane;
\ifchorded
	%\textnote{Tonalità migliore }	% <<< EV COMMENTI (tonalità originale/migliore)
\fi


%%%%%% INTRODUZIONE
\ifchorded
\vspace*{\versesep}
\musicnote{
\begin{minipage}{0.48\textwidth}
\textbf{Intro}
\hfill 
%( \eighthnote \, 80)   % <<  MODIFICA IL TEMPO
% Metronomo: \eighthnote (ottavo) \quarternote (quarto) \halfnote (due quarti)
\end{minipage}
} 	
\vspace*{-\versesep}
\beginverse*
\nolyrics

%---- Prima riga -----------------------------
\vspace*{-\versesep}
\[E] \[A] \[E] \[B]	 % \[*D] per indicare le pennate, \rep{2} le ripetizioni

%---- Ogni riga successiva -------------------
\vspace*{-\versesep}
\[E] \[A] \[B] \[B]

%---- Ev Indicazioni -------------------------			
%\textnote{\textit{(Oppure tutta la strofa)} }	

\endverse
\fi




%%%%% STROFA
\beginverse		%Oppure \beginverse* se non si vuole il numero di fianco
\memorize 		% <<< DECOMMENTA se si vuole utilizzarne la funzione
%\chordsoff		% <<< DECOMMENTA se vuoi una strofa senza accordi

\[E]Nel cammino del\[A]la mia vita \brk \[E]cerco quel che \[B4]sono, \[B]
\[E]ogni passo in \[A]questo mondo \brk \[E]sogno la mia liber\[B]tà.
\[E]Ma la strada a \[A]volte è buia, \brk \[E]perdo la fi\[B4]ducia. \[B]
\[E]Ho bisogno di un \[A]po' di luce, \brk il \[E]sole chi sa\[B]rà?

\endverse
\beginverse*		

Ma se guardo il \[F#-]volto tuo, \brk io l'Amore \[A]vedo in Te.
Guardo la tua \[F#-]croce e 
Tu speranza \[A]sei.. per \[B]me. \[B]

\endverse


%%%%% RITORNELLO
\beginchorus
\textnote{\textbf{Rit.}}

Sono \[E]qui, Signore, sono \[A]qui! \[A]
\[F#-]Luce dei miei passi è la \[A]Tua pa\[B]rola!
\echo{Sono} \[E]qui, Signore, sono \[A]qui! \[A]
\[F#-]Lungo la mia strada ogni \[A]giorno \brk tu sei \[B]guida per \[E]me!
\endchorus







%%%%%% EV. INTERMEZZO
\beginverse*
\vspace*{1.3\versesep}
{
	\nolyrics
	\textnote{Intermezzo strumentale}
	
	\ifchorded

	%---- Prima riga -----------------------------
	\vspace*{-\versesep}
	\[A] \[E]  \[B]	 

	%---- Ogni riga successiva -------------------
	\vspace*{-\versesep}
	\[E] \[A]  \[B]	 \[B]


	\fi
	%---- Ev Indicazioni -------------------------			
	%\textnote{\textit{(ripetizione della strofa)}} 
	 
}
\vspace*{\versesep}
\endverse



%%%%% STROFA
\beginverse		%Oppure \beginverse* se non si vuole il numero di fianco
\memorize 		% <<< DECOMMENTA se si vuole utilizzarne la funzione
%\chordsoff		% <<< DECOMMENTA se vuoi una strofa senza accordi

^Nel cammino del^la mia vita \brk in^contro il mio fra^tello.^
^Condivido la ^mia strada, \brk la ^meta è PerDi^Quà!
^Ma la strada a ^volte è dura, \brk ^perdo la spe^ranza.^
^Cerco solo un ^punto fermo, \brk in ^chi lo trove^rò?

\endverse
\beginverse*		

Ma se guardo il \[F#-]volto tuo, \brk io l'Amore \[A]vedo in Te.
Guardo la tua \[F#-]croce e 
Tu speranza \[A]sei.. per \[B]me. \[B]

\endverse



%%%%% RITORNELLO
\beginchorus
\textnote{\textbf{Rit.}}

Sono \[E]qui, Signore, sono \[A]qui! \[A]
\[F#-]Luce dei miei passi è la \[A]Tua pa\[B]rola!
\echo{Sono} \[E]qui, Signore, sono \[A]qui! \[A]
\[F#-]Lungo la mia strada ogni \[A]giorno \brk tu sei \[B]guida per \[E]me! \[B] 
\endchorus






%%%%% BRIDGE
\beginverse*		%Oppure \beginverse* se non si vuole il numero di fianco
%\memorize 		% <<< DECOMMENTA se si vuole utilizzarne la funzione
%\chordsoff		% <<< DECOMMENTA se vuoi una strofa senza accordi

Ma se guardo il \[F#-]volto tuo, \brk io l'Amore \[A]vedo in Te.
Guardo la tua \[F#-]croce e 
Tu speranza \[A]sei.. per \[B]me. \[C#]

\endverse




%%%%% RITORNELLO
\beginchorus

\textnote{\textbf{Rit.}}
\textnote{\textit{(si alza la tonalità)}}
\transpose{2}
Sono \[E]qui, Signore, sono \[A]qui! \[A]
\[F#-]Luce dei miei passi è la \[A]Tua pa\[B]rola!
\echo{Sono} \[E]qui, Signore, sono \[A]qui! \[A]
\[F#-]Lungo la mia strada ogni \[A]giorno \brk tu sei \[B]guida per \[E]me! \[A]
\endchorus




%%%%%% EV. FINALE

\beginchorus %oppure \beginverse*
\vspace*{1.3\versesep}
\textnote{\textbf{Finale} \textit{(rallentando)}} %<<< EV. INDICAZIONI

\[F#*] Tu sei \[C#*]guida per \[B]me!  \[B]
\[F#*] Tu sei \[C#*]guida per \[B]me! \[B] \[F#*]
\endchorus  %oppure \endverse


\endsong
%------------------------------------------------------------
%			FINE CANZONE
%------------------------------------------------------------


