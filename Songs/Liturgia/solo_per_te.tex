%-------------------------------------------------------------
%			INIZIO	CANZONE
%-------------------------------------------------------------


%titolo: 	Solo per Te
%autore: 	Sermig
%tonalita: 	Re



%%%%%% TITOLO E IMPOSTAZONI
\beginsong{Solo per Te}[by={Sermig}] 	% <<< MODIFICA TITOLO E AUTORE
\transpose{0} 						% <<< TRASPOSIZIONE #TONI (0 nullo)
%\preferflats  %SE VOGLIO FORZARE i bemolle come alterazioni
%\prefersharps %SE VOGLIO FORZARE i # come alterazioni
\momenti{Ringraziamento; Prima Comunione}							% <<< INSERISCI MOMENTI	
% momenti vanno separati da ; e vanno scelti tra:
% Ingresso; Atto penitenziale; Acclamazione al Vangelo; Dopo il Vangelo; Offertorio; Comunione; Ringraziamento; Fine; Santi; Pasqua; Avvento; Natale; Quaresima; Canti Mariani; Battesimo; Prima Comunione; Cresima; Matrimonio; Meditazione; Spezzare del pane;
\ifchorded
	%\textnote{$\bigstar$ Tonalità migliore }	% <<< EV COMMENTI (tonalità originale\migliore)
\fi


%%%%%% INTRODUZIONE
\ifchorded
\vspace*{\versesep}
\musicnote{
\begin{minipage}{0.48\textwidth}
\textbf{Intro}
\hfill 
%( \eighthnote \, 80)   % <<  MODIFICA IL TEMPO
% Metronomo: \eighthnote (ottavo) \quarternote (quarto) \halfnote (due quarti)
\end{minipage}
} 	
\vspace*{-\versesep}
\beginverse*

\nolyrics

%---- Prima riga -----------------------------
\vspace*{-\versesep}
\[D] \[A] \[G] \[D] \[A] \[G] 	 % \[*D] per indicare le pennate, \rep{2} le ripetizioni

%---- Ogni riga successiva -------------------
%\vspace*{-\versesep}
%\[G] \[C]  \[D]	

%---- Ev Indicazioni -------------------------			
%\textnote{\textit{[oppure tutta la strofa]} }	

\endverse
\fi




%%%%% STROFA
\beginverse*		%Oppure \beginverse* se non si vuole il numero di fianco
\memorize 		% <<< DECOMMENTA se si vuole utilizzarne la funzione
%\chordsoff		% <<< DECOMMENTA se vuoi una strofa senza accordi

\[D]Solo l'a\[A]more\[B-] \brk \[A]rende il \[D]sacrificio un \[A]dono\[G] \[A]
\[D]solo l'a\[A]more\[B-]  \brk fa \[A]diven\[D]tare giorno la \[A]notte\[G] \[A]
\[B-]solo l'a\[A]mo\[D]re  \brk \[A]fa del \[B-]tempo una preghie\[A]\[D]ra\[A]
\[B-]solo l'a\[A]mo\[D]re   \brk dà \[A]la cer\[B-]tezza che per t\[A]e
vale la \[G]pena correre \brk  \[D]cercare\[A], senza sost\[D]a.

\endverse



%%%%%% EV. INTERMEZZO
\beginverse*
\vspace*{1.3\versesep}
{
	\nolyrics
	\textnote{Intermezzo strumentale}
	
	\ifchorded

	%---- Prima riga -----------------------------
	\vspace*{-\versesep}
	\[A] \[G] \[A] \[D]

	%---- Ogni riga successiva -------------------
	\vspace*{-\versesep}
	\[A] \[G] \[A]


	\fi
	%---- Ev Indicazioni -------------------------			
	%\textnote{\textit{(ripetizione della strofa)}} 
	 
}
\vspace*{\versesep}
\endverse


%%%%% STROFA
\beginverse*		%Oppure \beginverse* se non si vuole il numero di fianco
%\memorize 		% <<< DECOMMENTA se si vuole utilizzarne la funzione
%\chordsoff		% <<< DECOMMENTA se vuoi una strofa senza accordi

\[D]Solo l'a\[A]more\[B-]  \brk \[A]apre il \[D]cuore alla sag\[A]ge\[G]zza\[A]
\[D]solo l'a\[A]more\[B-]  \brk \[A]rende \[D]liberi e si\[A]cu\[G]ri\[A]
\[B-]solo l'a\[A]mo\[D]re  \brk a\[A]sciuga il \[B-]viso di chi pia\[A]n\[D]ge\[A]
\[B-]solo l'a\[A]mo\[D]re  \brk \[A]dà un \[B-]volto alla bont\[A]à
divide il \[G]pane  \brk senza chie\[D]dere per sé.\[A] \[F#]

\endverse



%%%%% STROFA
\beginverse*		%Oppure \beginverse* se non si vuole il numero di fianco
%\memorize 		% <<< DECOMMENTA se si vuole utilizzarne la funzione
%\chordsoff		% <<< DECOMMENTA se vuoi una strofa senza accordi

\[B-]Solo per te, \[A] \[D]
\[G]solo con te, \[D] \[A] \[F#7]
\[B-]solo in te \[A]che \[D]sei l'amo\[A]re.\[F#7]
\[B-]Solo per te, \[A] \[D]
\[G]solo con te, \[D] \[A] \[F#7]
\[B-]solo in te \[A]che \[D]sei l'amo\[A]re.

\endverse








%%%%%% EV. CHIUSURA SOLO STRUMENTALE
\ifchorded
\beginchorus %oppure \beginverse*
\vspace*{1.3\versesep}
\textnote{Chiusura strumentale } %<<< EV. INDICAZIONI

\nolyrics

	%---- Prima riga -----------------------------
	\vspace*{-\versesep}
	\[D] \[A] \[G] \[A]

	%---- Ogni riga successiva -------------------
	\vspace*{-\versesep}
	\[D] \[A] \[G] \[A]	\rep{2}

	%---- Ogni riga successiva -------------------
	\vspace*{-\versesep}
	\[D*] 	


	%---- Ev Indicazioni -------------------------			
	%\textnote{\textit{(ripetizione della strofa)}} 

\endchorus  %oppure \endverse
\fi


\endsong
%------------------------------------------------------------
%			FINE CANZONE
%------------------------------------------------------------


