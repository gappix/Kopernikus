%-------------------------------------------------------------
%			INIZIO	CANZONE
%-------------------------------------------------------------


%titolo: 	E sono solo un uomo
%autore: 	Sequeri
%tonalita: 	Re



%%%%%% TITOLO E IMPOSTAZONI
\beginsong{E sono solo un uomo}[by={P. Sequeri}]	% <<< MODIFICA TITOLO E AUTORE
\transpose{0} 						% <<< TRASPOSIZIONE #TONI (0 nullo)
\momenti{Comunione}							% <<< INSERISCI MOMENTI	
% momenti vanno separati da ; e vanno scelti tra:
% Ingresso; Atto penitenziale; Acclamazione al Vangelo; Dopo il Vangelo; Offertorio; Comunione; Ringraziamento; Fine; Santi; Pasqua; Avvento; Natale; Quaresima; Canti Mariani; Battesimo; Prima Comunione; Cresima; Matrimonio; Meditazione; Spezzare del pane;
\ifchorded
	%\textnote{Tonalità migliore }	% <<< EV COMMENTI (tonalità originale/migliore)
\fi


%%%%%% INTRODUZIONE
\ifchorded
\vspace*{\versesep}
\musicnote{
\begin{minipage}{0.48\textwidth}
\textbf{Intro}
\hfill 
%( \eighthnote \, 80)   % <<  MODIFICA IL TEMPO
% Metronomo: \eighthnote (ottavo) \quarternote (quarto) \halfnote (due quarti)
\end{minipage}
} 	
\vspace*{-\versesep}
\beginverse*


\nolyrics

%---- Prima riga -----------------------------
\vspace*{-\versesep}
\[D] \[F#-] \[G] \[D]	 % \[*D] per indicare le pennate, \rep{2} le ripetizioni

%---- Ogni riga successiva -------------------
%\vspace*{-\versesep}
%\[G] \[C]  \[D]	

%---- Ev Indicazioni -------------------------			
%\textnote{\textit{(Oppure tutta la strofa)} }	

\endverse
\fi



%%%%% STROFA
\beginverse		%Oppure \beginverse* se non si vuole il numero di fianco
\memorize 		% <<< DECOMMENTA se si vuole utilizzarne la funzione
%\chordsoff		% <<< DECOMMENTA se vuoi una strofa senza accordi
\[D]Io lo so Si\[F#-]gnore che \[G]vengo da lon\[D]tano
\[D]prima nel pen\[F#-]siero e \[G]poi nella tua \[A7]mano
\[D]io mi rendo \[A]conto che \[G]Tu sei la mia \[D]vita
e \[G]non mi sembra \[E-]vero di pre\[E7]garti co\[A4/7]sì.
\endverse

\beginverse*
^Padre di ogni ^uomo e ^non ti ho visto ^mai.
^Spirito di ^Vita e ^nacqui da una ^donna
^Figlio mio fra^tello e ^sono solo un ^uomo
ep^pure io ca^pisco che ^Tu sei veri\[A7]tà.
\endverse



%%%%% RITORNELLO
\beginchorus
\textnote{\textbf{Rit.}}

E im\[D]parerò a guar\[G]dare tutto il \[A]mondo \[D7]
con gli \[G]occhi traspa\[A]renti di un bam\[D]bino \[D7]
e in\[G]segnerò a chia\[A]marti Padre \[D]nostro \[B7]
ad \[E-]ogni figlio \[E7 ]che diventa \[A7 ]uomo. \[A7 ] \ifchorded
E im\[D]parerò a guar\[G]dare tutto il \[A]mondo \[D7]
con gli \[G]occhi traspa\[A]renti di un bam\[D]bino \[D7]
e in\[G]segnerò a chia\[A]marti Padre \[D]nostro \[B7]
ad \[E-]ogni figlio \[A7]che diventa \[D]uomo. \[G]  \[D]

\else \rep{2}
\fi
\endchorus



%%%%% STROFA
\beginverse		%Oppure \beginverse* se non si vuole il numero di fianco
%\memorize 		% <<< DECOMMENTA se si vuole utilizzarne la funzione
%\chordsoff		% <<< DECOMMENTA se vuoi una strofa senza accordi
^Io lo so Si^gnore che ^Tu mi sei vi^cino,
^luce alla mia ^mente ^guida al mio cam^mino,
^mano che sor^regge ^sguardo che per^dona
e ^non mi sembra ^vero che Tu e^sista co^sì.
\endverse

\beginverse*
^Dove nasce a^more ^Tu sei la sor^gente.
^Dove c'è una ^croce ^Tu sei la spe^ranza
^dove il tempo ha ^fine ^Tu sei vita e^terna
e ^so che posso ^sempre con^tare su di \[A7]Te.
\endverse



%%%%% RITORNELLO
\beginchorus
\textnote{\textbf{Rit.}}

E ac\[D]coglierò la \[G]vita come un \[A]dono \[D7]
e a\[G]vrò il coraggio \[A]di morire an\[D]ch'io \[D7]
e in\[G]contro a Te ver\[A]rò col mio fra\[D]tello \[B7]
che \[E-]non si sente a\[E7 ]mato da nes\[A7]suno. \[A7] \ifchorded 
E ac\[D]coglierò la \[G]vita come un \[A]dono \[D7]
e a\[G]vrò il coraggio \[A]di morire an\[D]ch'io \[D7]
e in\[G]contro a Te ver\[A]rò col mio fra\[D]tello \[B7]
che \[E-]non si sente a\[A7]mato da nes\[D]suno.   \[G]  \[D*]
\else \rep{2}
\fi 
\endchorus


\endsong
%------------------------------------------------------------
%			FINE CANZONE
%------------------------------------------------------------



