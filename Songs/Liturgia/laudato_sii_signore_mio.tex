%-------------------------------------------------------------
%			INIZIO	CANZONE
%-------------------------------------------------------------


%titolo: 	Laudato sii, Signore mio
%autore: 	Cento
%tonalita: 	Mi 



%%%%%% TITOLO E IMPOSTAZONI
\beginsong{Laudato sii, Signore mio}[by={Il canto della creazione — G. Cento}]	% <<< MODIFICA TITOLO E AUTORE
\transpose{-2} 						% <<< TRASPOSIZIONE #TONI (0 nullo)
\momenti{Santi}							% <<< INSERISCI MOMENTI	
% momenti vanno separati da ; e vanno scelti tra:
% Ingresso; Atto penitenziale; Acclamazione al Vangelo; Dopo il Vangelo; Offertorio; Comunione; Ringraziamento; Fine; Santi; Pasqua; Avvento; Natale; Quaresima; Canti Mariani; Battesimo; Prima Comunione; Cresima; Matrimonio; Meditazione; Spezzare del pane;
\ifchorded
	%\textnote{Tonalità migliore }	% <<< EV COMMENTI (tonalità originale/migliore)
\fi


%%%%%% INTRODUZIONE
\ifchorded
\vspace*{\versesep}
\musicnote{
\begin{minipage}{0.48\textwidth}
\textbf{Intro}
\hfill 
%( \eighthnote \, 80)   % <<  MODIFICA IL TEMPO
% Metronomo: \eighthnote (ottavo) \quarternote (quarto) \halfnote (due quarti)
\end{minipage}
} 	
\vspace*{-\versesep}
\beginverse*

\nolyrics

%---- Prima riga -----------------------------
\vspace*{-\versesep}
\[E]	 % \[*D] per indicare le pennate, \rep{2} le ripetizioni

%---- Ogni riga successiva -------------------
%\vspace*{-\versesep}
%\[G] \[C]  \[D]	

%---- Ev Indicazioni -------------------------			
%\textnote{\textit{(Oppure tutta la strofa)} }	

\endverse
\fi


\beginchorus
\[E]Laudato sii Signore \[F#-]mio 
\[B] Laudato sii Signore \[C#-]mio 
\[A] Laudato sii Signore \[B]mio 
\[A] Laudato \[F#-*]sii Si\[B*]gnore \[E]mio.
\endchorus
\beginverse
\memorize
\[E] Per il sole d'ogni \[F#-]giorno, 
\[B] che riscalda e dona \[C#-]vita. 
\[A] Egli illumina il cam\[B]mino
\[A] di chi \[F#-*]cerca te Si\[B7]gnore. 
\[E] Per la luna e per le \[F#-]stelle, 
\[B] io le sento mie so\[C#-]relle 
\[A] le hai formate su nel \[B]cielo 
\[A] e le \[F#-]doni a chi è nel \[B]buio. 
\endverse
\beginverse

^ Per la nostra madre ^terra,  
^ che ci dona fiori ed ^erba, 
^ su di lei noi fati^chiamo,  
^ per il ^pane d'ogni ^giorno. 
^ Per chi soffre con co^raggio,  
^ e perdona nel tuo a^more, 
^ Tu gli dai la pace ^tua,  
^ alla ^sera della ^vita.
\endverse
\beginverse

^ Per la morte che è di ^tutti,  
^ io la sento ogni i^stante, 
^ ma se vivo nel tuo a^more, 
^ dona un ^senso alla mia ^vita. 
^ Per l'amore che è nel ^mondo,  
^ tra una donna e l'uomo ^suo, 
^ per la vita dei bam^bini  
^ che il mio ^mondo fanno ^nuovo.
\endverse
\beginverse

^ Io ti canto mio Si^gnore  
^ e con me la crea^zione 
^ ti ringrazia umil^mente  
^ perché ^tu sei il Si^gnore. ^\rep{2}
\endverse


\endsong
%------------------------------------------------------------
%			FINE CANZONE
%------------------------------------------------------------
