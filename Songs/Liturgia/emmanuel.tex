%-------------------------------------------------------------
%			INIZIO	CANZONE
%-------------------------------------------------------------


%titolo: 	Emmanuel
%autore: 	M. Mammoli
%tonalita: 	Mi 



%%%%%% TITOLO E IMPOSTAZONI
\beginsong{Emmanuel }[by={Inno Giubileo 2000 — M. Mammoli}] 	% <<< MODIFICA TITOLO E AUTORE
\transpose{-2} 						% <<< TRASPOSIZIONE #TONI (0 nullo)
\momenti{Ingresso; Comunione; Congedo}							% <<< INSERISCI MOMENTI	
% momenti vanno separati da ; e vanno scelti tra:
% Ingresso; Atto penitenziale; Acclamazione al Vangelo; Dopo il Vangelo; Offertorio; Comunione; Ringraziamento; Fine; Santi; Pasqua; Avvento; Natale; Quaresima; Canti Mariani; Battesimo; Prima Comunione; Cresima; Matrimonio; Meditazione; Spezzare del pane;
\ifchorded
	%\textnote{$\bigstar$ Tonalità migliore }	% <<< EV COMMENTI (tonalità originale/migliore)
\fi




%%%%%% INTRODUZIONE
\ifchorded
\vspace*{\versesep}
\musicnote{
\begin{minipage}{0.48\textwidth}
\textbf{Intro}
\hfill 
%( \eighthnote \, 80)   % <<  MODIFICA IL TEMPO
% Metronomo: \eighthnote (ottavo) \quarternote (quarto) \halfnote (due quarti)
\end{minipage}
} 	
\vspace*{-\versesep}
\beginverse*

\nolyrics

%---- Prima riga -----------------------------
\vspace*{-\versesep}
\[E] \[B] \[E]	\[B] % \[*D] per indicare le pennate, \rep{2} le ripetizioni

%---- Ogni riga successiva -------------------
%\vspace*{-\versesep}
%\[G] \[C]  \[D]	

%---- Ev Indicazioni -------------------------			
%\textnote{\textit{(Oppure tutta la strofa)} }	

\endverse
\fi




%%%%% STROFA
\beginverse		%Oppure \beginverse* se non si vuole il numero di fianco
\memorize 		% <<< DECOMMENTA se si vuole utilizzarne la funzione
%\chordsoff		% <<< DECOMMENTA se vuoi una strofa senza accordi
\[E]Dall'orizzonte una grande luce
v\[B]iaggia nella storia
e \[A]lungo gli anni ha vinto il buio
fa\[B]cendosi Memoria,
e il\[E]luminando la nostra vita
ch\[B]iaro ci rivela
che \[A]non si vive se non si cerca
\[F#-]la Veri\[B]tà...
\endverse

\beginverse*	%Oppure \beginverse* se non si vuole il numero di fianco
...a-a-\[E]ah ...a-a-\[B]ah.. \quad \[A] \quad \[B]
\endverse

%%%%% STROFA
\beginverse*		%Oppure \beginverse* se non si vuole il numero di fianco
%\memorize 		% <<< DECOMMENTA se si vuole utilizzarne la funzione
%\chordsoff		% <<< DECOMMENTA se vuoi una strofa senza accordi



Dal^la città di chi ha versato
il ^sangue per amore
ed ^ha cambiato il vecchio mondo
vogli^amo ripartire.
Se^guendo Cristo, insieme a Pietro,
ri^nasce in noi la fede,
Pa^rola viva che ci rinnova
e ^cresce in ^noi...







\endverse


%%%%% RITORNELLO
\beginchorus
\textnote{\textbf{Rit.}}

Siamo \[G#-]qui
\[A] sotto la stessa luce
\[F#-] sotto la sua croce
\[D] cantando ad \[B]una voce.
\[E]È l'Emmanu\[B]el,
l'Emmanu\[A]el, Emmanu\[B]el.
\echo{Cantando ad una voce}
\[C#-]È l'Emmanu\[B]el, Emmanu\[A]el. \quad\[B]

\endchorus





%%%%% STROFA
\beginverse		%Oppure \beginverse* se non si vuole il numero di fianco
%\memorize 		% <<< DECOMMENTA se si vuole utilizzarne la funzione
\chordsoff		% <<< DECOMMENTA se vuoi una strofa senza accordi
Un grande dono che Dio ci ha fatto
è il Cristo suo Figlio,
l'umanità è rinnovata,
è in lui salvata.
È vero uomo, è vero Dio,
è il Pane della Vita,
che ad ogni uomo ai suoi fratelli
ridonerà.
\endverse

%%%%% STROFA
\beginverse		%Oppure \beginverse* se non si vuole il numero di fianco
%\memorize 		% <<< DECOMMENTA se si vuole utilizzarne la funzione
\chordsoff		% <<< DECOMMENTA se vuoi una strofa senza accordi
La morte è uccisa,
la vita ha vinto,
è Pasqua in tutto il mondo,
un vento soffia in ogni uomo
lo Spirito fecondo,
che porta avanti nella storia
la Chiesa sua sposa,
sotto lo sguardo di Maria,
comunità.
\endverse

%%%%% STROFA
\beginverse		%Oppure \beginverse* se non si vuole il numero di fianco
%\memorize 		% <<< DECOMMENTA se si vuole utilizzarne la funzione
\chordsoff		% <<< DECOMMENTA se vuoi una strofa senza accordi
Noi debitori del passato
di secoli di storia,
di vite date per amore,.
di santi che han creduto,
di uomini che ad alta quota
insegnano a volare,
di chi la storia sa cambiare,
come Gesù. 
\endverse

%%%%% STROFA
\beginverse		%Oppure \beginverse* se non si vuole il numero di fianco
%\memorize 		% <<< DECOMMENTA se si vuole utilizzarne la funzione
\chordsoff		% <<< DECOMMENTA se vuoi una strofa senza accordi
È giunta un'era di primavera,
è tempo di cambiare.
È oggi il tempo sempre nuovo
per ricominciare, per dare svolte, parole nuove
e convertire il cuore,
per dire al mondo, ad ogni uomo:
Signore Gesù. 
\endverse

%%%%% STROFA
\beginverse		%Oppure \beginverse* se non si vuole il numero di fianco
%\memorize 		% <<< DECOMMENTA se si vuole utilizzarne la funzione
\chordsoff		% <<< DECOMMENTA se vuoi una strofa senza accordi
Da ^mille strade arriviamo a Roma
sui ^passi della fede,
senti^amo l'eco della Parola
^che risuona ancora
da ^queste mura, da questo cielo
^per il mondo intero:
è v^ivo oggi, è l'Uomo Vero
^Cristo tra ^noi...
\endverse

%%%%% RITORNELLO
\ifchorded
\beginchorus
\textnote{\textbf{Rit.}}

Siamo \[G#-]qui
\[A] sotto la stessa luce
\[F#-] sotto la sua croce
\[D] cantando ad \[B]una voce.
\[E]È l'Emmanu\[B]el,
l'Emmanu\[A]el, Emmanu\[B]el.
\echo{Cantando ad una voce}
\[C#-]È l'Emmanu\[B]el, Emmanu\[A]el. \quad  \[B] \quad \[E*]

\endchorus
\fi


\endsong
%------------------------------------------------------------
%			FINE CANZONE
%------------------------------------------------------------

