%-------------------------------------------------------------
%			INIZIO	CANZONE
%-------------------------------------------------------------


%titolo: 	Luce
%autore: 	Comunità del Cenacolo
%tonalita: 	Sol



%%%%%% TITOLO E IMPOSTAZONI
\beginsong{Luce}[by={Comunità\ del\ Cenacolo}]% <<< MODIFICA TITOLO E AUTORE
\transpose{0} 						% <<< TRASPOSIZIONE #TONI (0 nullo)
%\preferflats  %SE VOGLIO FORZARE i bemolle come alterazioni
%\prefersharps %SE VOGLIO FORZARE i # come alterazioni
\momenti{Comunione}							% <<< INSERISCI MOMENTI	
% momenti vanno separati da ; e vanno scelti tra:
% Ingresso; Atto penitenziale; Acclamazione al Vangelo; Dopo il Vangelo; Offertorio; Comunione; Ringraziamento; Fine; Santi; Pasqua; Avvento; Natale; Quaresima; Canti Mariani; Battesimo; Prima Comunione; Cresima; Matrimonio; Meditazione; Spezzare del pane;
\ifchorded
	%\textnote{Tonalità migliore }	% <<< EV COMMENTI (tonalità originale/migliore)
\fi


%%%%%% INTRODUZIONE
\ifchorded
\vspace*{\versesep}
\musicnote{
\begin{minipage}{0.48\textwidth}
\textbf{Intro}
\hfill 
%( \eighthnote \, 80)   % <<  MODIFICA IL TEMPO
% Metronomo: \eighthnote (ottavo) \quarternote (quarto) \halfnote (due quarti)
\end{minipage}
} 	
\vspace*{-\versesep}
\beginverse*

\nolyrics

%---- Prima riga -----------------------------
\vspace*{-\versesep}
\[G] \[D] \[C] \[C]	 \rep{2}% \[*D] per indicare le pennate, \rep{2} le ripetizioni

%---- Ogni riga successiva -------------------
%\vspace*{-\versesep}
%\[G] \[C]  \[D]	

%---- Ev Indicazioni -------------------------			
%\textnote{\textit{(Oppure tutta la strofa)} }	

\endverse
\fi




%%%%% STROFA
\beginverse		%Oppure \beginverse* se non si vuole il numero di fianco
\memorize 		% <<< DECOMMENTA se si vuole utilizzarne la funzione
%\chordsoff		% <<< DECOMMENTA se vuoi una strofa senza accordi
\[G]C'è il segreto della liber\[D]tà 
quella \[E-]vera 
nasce \[C]dentro di \[D]te.
È \[G]come risvegliarsi un mat\[D]tino 
con il \[E-]sole 
dopo un \[C]lungo in\[D]verno
\endverse
\beginverse*
\[G] nel soffrire 
\[D] mio Signore 
\[E-] ho incontrato \[C]Te-e  Dio a\[D]more
\[G] nel perdono 
\[D] nel gioire 
\[E-] ho capito \[C]che... sei \[D]luce per me!
\endverse





%%%%% RITORNELLO
\beginchorus
\textnote{\textbf{Rit.}}
\[G] Signore sono \[D]qui 
per dirti ancora \[E-]sì \[C]lu\[D]ce
\[G] fammi scoppiare \[D]di 
gioia di vive\[E-]re \[C]lu\[D]ce.
\[G] Fammi strumento \[D]per 
portare attorno a \[E-]me \[C]lu\[D]ce
\[G] e chi è vicino a \[D]me 
sappia che tutto in \[E-]te è \[C]lu\[D]ce.
\endchorus




%%%%%% EV. INTERMEZZO
\beginverse*
\vspace*{1.3\versesep}
{
	\nolyrics
	\textnote{Intermezzo strumentale}
	
	\ifchorded

	%---- Prima riga -----------------------------
	\vspace*{-\versesep}
	\[G] \[D] \[C]  \[C]	 

	%---- Ogni riga successiva -------------------
	%\vspace*{-\versesep}
	%\[G] \[C]  \[D]	


	\fi
	%---- Ev Indicazioni -------------------------			
	%\textnote{\textit{(ripetizione della strofa)}} 
	 
}
\vspace*{\versesep}
\endverse


%%%%% STROFA
\beginverse		%Oppure \beginverse* se non si vuole il numero di fianco
%\memorize 		% <<< DECOMMENTA se si vuole utilizzarne la funzione
%\chordsoff		% <<< DECOMMENTA se vuoi una strofa senza accordi
^Voglio ringraziarti Si^gnore 
per la ^Vita 
che mi ^hai rido^nato
^so che sei nell'a^more 
degli a^mici 
che ora ^ho incont^rato
\endverse
\beginverse*
\[G] nel soffrire 
\[D] mio Signore 
\[E-] ho incontrato \[C]Te-e  Dio a\[D]more
\[G] nel perdono 
\[D] nel gioire 
\[E-] ho capito \[C]che... sei \[D]luce per me!
\endverse



%%%%% RITORNELLO
\beginchorus
\textnote{\textbf{Rit.}}
\[G] Signore sono \[D]qui 
per dirti ancora \[E-]sì \[C]lu\[D]ce
\[G] fammi scoppiare \[D]di 
gioia di vive\[E-]re \[C]lu\[D]ce.
\[G] Fammi strumento \[D]per 
portare attorno a \[E-]me \[C]lu\[D]ce
\[G] e chi è vicino a \[D]me 
sappia che tutto in \[E-]te è \[C]lu\[D]ce.
\endchorus



\beginverse*
\[E-] E con le lacrime agli \[C]occhi \brk e le mie mani al\[G]zate verso te Ge\[D]sù
\[E-] con la speranza nel \[C]cuore \brk e la tua luce in \[D4]me paura non ho \[C]più.
\endverse



%%%%% RITORNELLO
\beginchorus
\textnote{\textbf{Rit.}}
\[G] Signore sono \[D]qui 
per dirti ancora \[E-]sì \[C]lu\[D]ce
\[G] fammi scoppiare \[D]di 
gioia di vive\[E-]re \[C]lu\[D]ce.
\[G] Fammi strumento \[D]per 
portare attorno a \[E-]me \[C]lu\[D]ce
\[G] e chi è vicino a \[D]me 
sappia che tutto in \[E-]te è \[C]lu\[D]ce. \quad \[G*]
\endchorus


\endsong
%------------------------------------------------------------
%			FINE CANZONE
%------------------------------------------------------------