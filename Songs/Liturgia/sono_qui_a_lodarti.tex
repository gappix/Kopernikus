%-------------------------------------------------------------
%			INIZIO	CANZONE
%-------------------------------------------------------------


%titolo: 	Sono qui a lodarti
%autore: 	T. Hughes
%tonalita: 	Mi > Re 



%%%%%% TITOLO E IMPOSTAZONI
\beginsong{Sono qui a lodarti}[by={T. Hughes}] 	% <<< MODIFICA TITOLO E AUTORE
\transpose{-4} 						% <<< TRASPOSIZIONE #TONI (0 nullo)
%\preferflats  %SE VOGLIO FORZARE i bemolle come alterazioni
%\prefersharps %SE VOGLIO FORZARE i # come alterazioni
\momenti{Ringraziamento; Meditazione; Ingresso; Pasqua}							% <<< INSERISCI MOMENTI	
% momenti vanno separati da ; e vanno scelti tra:
% Ingresso; Atto penitenziale; Acclamazione al Vangelo; Dopo il Vangelo; Offertorio; Comunione; Ringraziamento; Fine; Santi; Pasqua; Avvento; Natale; Quaresima; Canti Mariani; Battesimo; Prima Comunione; Cresima; Matrimonio; Meditazione; Spezzare del pane;
\ifchorded
	\textnote{$\bigstar$ Tonalità migliore }	% <<< EV COMMENTI (tonalità originale/migliore)
\fi


%%%%%% INTRODUZIONE
\ifchorded
\vspace*{\versesep}
\musicnote{
\begin{minipage}{0.48\textwidth}
\textbf{Intro}
\hfill 
( \eighthnote \, 70)   % <<  MODIFICA IL TEMPO
% Metronomo: \eighthnote (ottavo) \quarternote (quarto) \halfnote (due quarti)
\end{minipage}
} 	
\vspace*{-\versesep}
\beginverse*

\nolyrics

%---- Prima riga -----------------------------
\vspace*{-\versesep}
\[E] \[B] \[F#-] \[A]	 % \[*D] per indicare le pennate, \rep{2} le ripetizioni

%---- Ogni riga successiva -------------------
\vspace*{-\versesep}
\[E] \[B]\[A] \[A]

%---- Ev Indicazioni -------------------------			
%\textnote{\textit{(Oppure tutta la strofa)} }	

\endverse
\fi




%%%%% STROFA
\beginverse		%Oppure \beginverse* se non si vuole il numero di fianco
\memorize 		% <<< DECOMMENTA se si vuole utilizzarne la funzione
%\chordsoff		% <<< DECOMMENTA se vuoi una strofa senza accordi

\[E]Luce del \[B]mondo, nel \[F#-]buio del \[A]cuore
\[E]Vieni ed il\[B]lumina\[A]mi. \[A]
\[E]Tu mia \[B]sola spe\[F#-]ranza di \[A]vita,
\[E]Resta per \[B]sempre con \[A]me.

\endverse




%%%%% RITORNELLO
\beginchorus
\textnote{\textbf{Rit.}}

\[(A)]Sono qui a lo\[E]darti, qui per ado\[B]rarti
qui per dirti \[E]che Tu sei il mio \[A]Dio.
E solo Tu sei \[E]santo, sei meravigl\[B]ioso
degno e glor\[E]ioso sei per \[A]me.

\endchorus



%%%%% STROFA
\beginverse		%Oppure \beginverse* se non si vuole il numero di fianco
%\memorize 		% <<< DECOMMENTA se si vuole utilizzarne la funzione
%\chordsoff		% <<< DECOMMENTA se vuoi una strofa senza accordi

^Re della ^storia e ^Re nella ^gloria,
^sei sceso in ^terra fra ^noi. ^
^Con umil^tà il Tuo ^trono hai lasc^iato,
^per dimos^trarci il Tuo a^mor.

\endverse



%%%%% RITORNELLO
\beginchorus
\textnote{\textbf{Rit.}}

\[(A)]Sono qui a lo\[E]darti, qui per ado\[B]rarti
qui per dirti \[E]che Tu sei il mio \[A]Dio.
E solo Tu sei \[E]santo, sei meravigl\[B]ioso
degno e glor\[E]ioso sei per \[A]me.

\endchorus


%%%%% STROFA
\beginverse*		%Oppure \beginverse* se non si vuole il numero di fianco
%\memorize 		% <<< DECOMMENTA se si vuole utilizzarne la funzione
%\chordsoff		% <<< DECOMMENTA se vuoi una strofa senza accordi
\vspace*{1.3\versesep}
\textnote{\textbf{Finale} \textit{(a ripetizione)}}

Io \[B]mai sa\[E]prò quanto t\[A]i costò,
lì \[B]sulla \[E]croce mor\[A]ir per me...
\endverse





\endsong
%------------------------------------------------------------
%			FINE CANZONE
%------------------------------------------------------------




%++++++++++++++++++++++++++++++++++++++++++++++++++++++++++++
%			CANZONE TRASPOSTA
%++++++++++++++++++++++++++++++++++++++++++++++++++++++++++++
\ifchorded
%decremento contatore per avere stesso numero
\addtocounter{songnum}{-1} 
\beginsong{Sono qui a lodarti}[by={T. Hughes}]  	% <<< COPIA TITOLO E AUTORE
\transpose{0} 						% <<< TRASPOSIZIONE #TONI + - (0 nullo)
%\preferflats  %SE VOGLIO FORZARE i bemolle come alterazioni
%\prefersharps %SE VOGLIO FORZARE i # come alterazioni
\ifchorded
	\textnote{$\lozenge$ Tonalità originale}	% <<< EV COMMENTI (tonalità originale/migliore)
\fi



%%%%%% INTRODUZIONE
\ifchorded
\vspace*{\versesep}
\musicnote{
\begin{minipage}{0.48\textwidth}
\textbf{Intro}
\hfill 
( \eighthnote \, 70)   % <<  MODIFICA IL TEMPO
% Metronomo: \eighthnote (ottavo) \quarternote (quarto) \halfnote (due quarti)
\end{minipage}
} 	
\vspace*{-\versesep}
\beginverse*

\nolyrics

%---- Prima riga -----------------------------
\vspace*{-\versesep}
\[E] \[B] \[F#-] \[A]	 % \[*D] per indicare le pennate, \rep{2} le ripetizioni

%---- Ogni riga successiva -------------------
\vspace*{-\versesep}
\[E] \[B]\[A] \[A]

%---- Ev Indicazioni -------------------------			
%\textnote{\textit{(Oppure tutta la strofa)} }	

\endverse
\fi




%%%%% STROFA
\beginverse		%Oppure \beginverse* se non si vuole il numero di fianco
\memorize 		% <<< DECOMMENTA se si vuole utilizzarne la funzione
%\chordsoff		% <<< DECOMMENTA se vuoi una strofa senza accordi

\[E]Luce del \[B]mondo, nel \[F#-]buio del \[A]cuore
\[E]Vieni ed il\[B]lumina\[A]mi. \[A]
\[E]Tu mia \[B]sola spe\[F#-]ranza di \[A]vita,
\[E]Resta per \[B]sempre con \[A]me.

\endverse




%%%%% RITORNELLO
\beginchorus
\textnote{\textbf{Rit.}}

\[(A)]Sono qui a lo\[E]darti, qui per ado\[B]rarti
qui per dirti \[E]che Tu sei il mio \[A]Dio.
E solo Tu sei \[E]santo, sei meravigl\[B]ioso
degno e glor\[E]ioso sei per \[A]me.

\endchorus



%%%%% STROFA
\beginverse		%Oppure \beginverse* se non si vuole il numero di fianco
%\memorize 		% <<< DECOMMENTA se si vuole utilizzarne la funzione
%\chordsoff		% <<< DECOMMENTA se vuoi una strofa senza accordi

^Re della ^storia e ^Re nella ^gloria,
^sei sceso in ^terra fra ^noi. ^
^Con umil^tà il Tuo ^trono hai lasc^iato,
^per dimos^trarci il Tuo a^mor.

\endverse



%%%%% RITORNELLO
\beginchorus
\textnote{\textbf{Rit.}}

\[(A)]Sono qui a lo\[E]darti, qui per ado\[B]rarti
qui per dirti \[E]che Tu sei il mio \[A]Dio.
E solo Tu sei \[E]santo, sei meravigl\[B]ioso
degno e glor\[E]ioso sei per \[A]me.

\endchorus


%%%%% STROFA
\beginverse*		%Oppure \beginverse* se non si vuole il numero di fianco
%\memorize 		% <<< DECOMMENTA se si vuole utilizzarne la funzione
%\chordsoff		% <<< DECOMMENTA se vuoi una strofa senza accordi
\vspace*{1.3\versesep}
\textnote{\textbf{Finale} \textit{(a ripetizione)}}

Io \[B]mai sa\[E]prò quanto t\[A]i costò,
lì \[B]sulla \[E]croce mor\[A]ir per me...
\endverse




\endsong


\fi
%++++++++++++++++++++++++++++++++++++++++++++++++++++++++++++
%			FINE CANZONE TRASPOSTA
%++++++++++++++++++++++++++++++++++++++++++++++++++++++++++++
