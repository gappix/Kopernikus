%-------------------------------------------------------------
%			INIZIO	CANZONE
%-------------------------------------------------------------


%titolo: 	Volto dell'Uomo
%autore: 	C. Davide, D. Machetta
%tonalita: 	Mi-



%%%%%% TITOLO E IMPOSTAZONI
\beginsong{Volto dell'Uomo}[by={C. Davide, D. Machetta}] 	% <<< MODIFICA TITOLO E AUTORE
\transpose{0} 						% <<< TRASPOSIZIONE #TONI (0 nullo)
%\preferflats  %SE VOGLIO FORZARE i bemolle come alterazioni
%\prefersharps %SE VOGLIO FORZARE i # come alterazioni
\momenti{Quaresima}							% <<< INSERISCI MOMENTI	
% momenti vanno separati da ; e vanno scelti tra:
% Ingresso; Atto penitenziale; Acclamazione al Vangelo; Dopo il Vangelo; Offertorio; Comunione; Ringraziamento; Fine; Santi; Pasqua; Avvento; Natale; Quaresima; Canti Mariani; Battesimo; Prima Comunione; Cresima; Matrimonio; Meditazione; Spezzare del pane;
\ifchorded
	%\textnote{Tonalità migliore }	% <<< EV COMMENTI (tonalità originale/migliore)
\fi


%%%%%% INTRODUZIONE
\ifchorded
\vspace*{\versesep}
\musicnote{
\begin{minipage}{0.48\textwidth}
\textbf{Intro}
\hfill 
( \halfnote \, 52)   % <<  MODIFICA IL TEMPO
% Metronomo: \eighthnote (ottavo) \quarternote (quarto) \halfnote (due quarti)
\end{minipage}
} 	
\vspace*{-\versesep}
\beginverse*

\nolyrics

%---- Prima riga -----------------------------
\vspace*{-\versesep}
\[E-] \[B-] % \[*D] per indicare le pennate, \rep{2} le ripetizioni

%---- Ogni riga successiva -------------------
\vspace*{-\versesep}
\[D] \[A]  \[B-]	

%---- Ev Indicazioni -------------------------			
%\textnote{\textit{(Oppure tutta la strofa)} }	

\endverse
\fi




%%%%% STROFA
\beginverse		%Oppure \beginverse* se non si vuole il numero di fianco
%\memorize 		% <<< DECOMMENTA se si vuole utilizzarne la funzione
%\chordsoff		% <<< DECOMMENTA se vuoi una strofa senza accordi

\[E-]Volto dell'\[B-]uomo
pene\[D]trato \[A]dal do\[B-]lore,
\[G]volto di \[D]Dio
pene\[E-]trato di \[A]umil\[B]tà,
\[G]scandalo dei \[D]grandi
che con\[A-]fidano nel \[B]mondo,
\[A-]uomo dei do\[E-]lori, pie\[B-7]tà di \[E]noi.

\endverse





%%%%% STROFA
\beginverse		%Oppure \beginverse* se non si vuole il numero di fianco
%\memorize 		% <<< DECOMMENTA se si vuole utilizzarne la funzione
%\chordsoff		% <<< DECOMMENTA se vuoi una strofa senza accordi

\[E-]Volto di \[B-]pace,
di per\[D]dono e \[A]di bon\[B-]tà,
\[G]tu che in si\[D]lenzio
hai \[E-]pagato i \[A]nostri \[B]errori,
\[G]scandalo dei \[D]forti,
di chi ha \[A-]sete di vio\[B]lenza,
\[A-]Cristo Salva\[E-]tore, pie\[B-7]tà di \[E]noi.

\endverse


%%%%% STROFA
\beginverse		%Oppure \beginverse* se non si vuole il numero di fianco
%\memorize 		% <<< DECOMMENTA se si vuole utilizzarne la funzione
%\chordsoff		% <<< DECOMMENTA se vuoi una strofa senza accordi

\[E-]Volto di \[B-]luce,
di vit\[D]toria e \[A]liber\[B-]tà,
\[G]tu hai trac\[D]ciato
i senti\[E-]eri \[A]della \[B]vita;
\[G]spezzi con la \[D]Croce
le bar\[A-]riere della \[B]morte:
\[A-]Figlio di \[E-]Dio, pie\[B-7]tà di \[E]noi.

\endverse



\endsong
%------------------------------------------------------------
%			FINE CANZONE
%------------------------------------------------------------


