%-------------------------------------------------------------
%			INIZIO	CANZONE
%-------------------------------------------------------------


%titolo: 	Scatenate la gioia!
%autore: 	L. D'Amico
%tonalita: 	Re



%%%%%% TITOLO E IMPOSTAZONI
\beginsong{Scatenate la gioia}[by={L. D'Amico}] 	% <<< MODIFICA TITOLO E AUTORE
\transpose{-2} 						% <<< TRASPOSIZIONE #TONI (0 nullo)
%\preferflats  %SE VOGLIO FORZARE i bemolle come alterazioni
%\prefersharps %SE VOGLIO FORZARE i # come alterazioni
\momenti{Congedo; Matrimonio; Ringraziamento}							% <<< INSERISCI MOMENTI	
% momenti vanno separati da ; e vanno scelti tra:
% Ingresso; Atto penitenziale; Acclamazione al Vangelo; Dopo il Vangelo; Offertorio; Comunione; Ringraziamento; Fine; Santi; Pasqua; Avvento; Natale; Quaresima; Canti Mariani; Battesimo; Prima Comunione; Cresima; Matrimonio; Meditazione; Spezzare del pane;
\ifchorded
	%\textnote{Tonalità migliore }	% <<< EV COMMENTI (tonalità originale/migliore)
\fi


%%%%%% INTRODUZIONE
\ifchorded
\vspace*{\versesep}
\musicnote{
\begin{minipage}{0.48\textwidth}
\textbf{Intro}
\hfill 
%( \eighthnote \, 80)   % <<  MODIFICA IL TEMPO
% Metronomo: \eighthnote (ottavo) \quarternote (quarto) \halfnote (due quarti)
\end{minipage}
} 	
\vspace*{-\versesep}
\beginverse*

\nolyrics

%---- Prima riga -----------------------------
\vspace*{-\versesep}
\[D] \[G] \[D] \[G]	 % \[*D] per indicare le pennate, \rep{2} le ripetizioni

%---- Ogni riga successiva -------------------
\vspace*{-\versesep}
\[D] \[G]  \[D]  \[A] \[A]	

%---- Ev Indicazioni -------------------------			
%\textnote{\textit{(Oppure tutta la strofa)} }	

\endverse
\fi




%%%%% STROFA
\beginverse		%Oppure \beginverse* se non si vuole il numero di fianco
\memorize 		% <<< DECOMMENTA se si vuole utilizzarne la funzione
%\chordsoff		% <<< DECOMMENTA se vuoi una strofa senza accordi

\[D]Uscite dalle case voi, \[G]che siete chiusi dentro
\[D]Venite qui tra noi, \[G]qualcosa sta accadendo.
\[D]Qui non piove, \[G]qui c’è solo il sole
\[D]Fate in fretta, man\[A4]cate solo \[A]voi. 


\endverse

%%%%% STROFA
\beginverse*		%Oppure \beginverse* se non si vuole il numero di fianco
%\memorize 		% <<< DECOMMENTA se si vuole utilizzarne la funzione
%\chordsoff		% <<< DECOMMENTA se vuoi una strofa senza accordi

^Muovi i piedi tu, ^tu che stai ascoltando,
^e le mani alza tu, ^per tenere il tempo.
^Segui il ritmo di ^questa canzone,
t^utti pronti, poss^iamo dare la via. 


\endverse



%%%%% RITORNELLO
\beginchorus
\textnote{\textbf{Rit.}}

\[D]Scate\[A]nate la \[B-]gioia, \[G]oggi \[D]qui si fa \[A]festa!
\[B-]Dai, cantate con \[G]noi. 
\[D]Qui la \[A]festa siamo \[G]noi!  \rep{2}


\endchorus




%%%%%% EV. INTERMEZZO
\beginverse*
\vspace*{1.3\versesep}
{
	\nolyrics
	\textnote{Intermezzo strumentale}
	
	\ifchorded

	%---- Prima riga -----------------------------
	\vspace*{-\versesep}
	\[D] \[G]  \[D] \[G]	 

	%---- Ogni riga successiva -------------------
	%\vspace*{-\versesep}
	%\[G] \[C]  \[D]	


	\fi
	%---- Ev Indicazioni -------------------------			
	%\textnote{\textit{(ripetizione della strofa)}} 
	 
}
\vspace*{\versesep}
\endverse



%%%%% STROFA
\beginverse		%Oppure \beginverse* se non si vuole il numero di fianco
%\memorize 		% <<< DECOMMENTA se si vuole utilizzarne la funzione
%\chordsoff		% <<< DECOMMENTA se vuoi una strofa senza accordi

^Non si sente bene qui, ^qualcuno canta piano,
^fatti trascinare tu, che ^non puoi farne a meno.
^Se qui non canti, ^togli il tuo colore
^all’arcobaleno di ^questa canzone. ^

\endverse




%%%%% STROFA
\beginverse		%Oppure \beginverse* se non si vuole il numero di fianco
%\memorize 		% <<< DECOMMENTA se si vuole utilizzarne la funzione
%\chordsoff		% <<< DECOMMENTA se vuoi una strofa senza accordi

^Siamo in tanti qui, ^a cantare forte
^che la gioia entrerà, ^basta aprire le porte.
^Qui nell’aria si ^sente un buon profumo,
^se ci stai, non ^manca più nessuno! ^ 

\endverse






\endsong
%------------------------------------------------------------
%			FINE CANZONE
%------------------------------------------------------------


