%-------------------------------------------------------------
%			INIZIO	CANZONE
%-------------------------------------------------------------


%titolo: 	Giovane donna
%autore: 	Scaglianti, Bancolini
%tonalita: 	Re 



%%%%%% TITOLO E IMPOSTAZONI
\beginsong{Giovane donna}[by={L. Scaglianti, L. Bancolini}] 	% <<< MODIFICA TITOLO E AUTORE
\transpose{0} 						% <<< TRASPOSIZIONE #TONI (0 nullo)
\momenti{Canti Mariani; }							% <<< INSERISCI MOMENTI	
% momenti vanno separati da ; e vanno scelti tra:
% Ingresso; Atto penitenziale; Acclamazione al Vangelo; Dopo il Vangelo; Offertorio; Comunione; Ringraziamento; Fine; Santi; Pasqua; Avvento; Natale; Quaresima; Canti Mariani; Battesimo; Prima Comunione; Cresima; Matrimonio; Meditazione; Spezzare del pane;
\ifchorded
	%\textnote{Tonalità migliore }	% <<< EV COMMENTI (tonalità originale/migliore)
\fi


%%%%%% INTRODUZIONE
\ifchorded
\vspace*{\versesep}
\musicnote{
\begin{minipage}{0.48\textwidth}
\textbf{Intro}
\hfill 
( \eighthnote \, 104)   % <<  MODIFICA IL TEMPO
% Metronomo: \eighthnote (ottavo) \quarternote (quarto) \halfnote (due quarti)
\end{minipage}
} 	
\vspace*{-\versesep}
\beginverse*


\nolyrics

%---- Prima riga -----------------------------
\vspace*{-\versesep}
\[D] \[E-]\[G]  \[D]	 % \[*D] per indicare le pennate, \rep{2} le ripetizioni

%---- Ogni riga successiva -------------------
%\vspace*{-\versesep}
%\[G] \[C]  \[D]	

%---- Ev Indicazioni -------------------------			
%\textnote{\textit{(Oppure tutta la strofa)} }	

\endverse
\fi





%%%%% STROFA
\beginverse		%Oppure \beginverse* se non si vuole il numero di fianco
\memorize 		% <<< DECOMMENTA se si vuole utilizzarne la funzione
%\chordsoff		% <<< DECOMMENTA se vuoi una strofa senza accordi
\[D]Giovane \[E-]donna, at\[G]tesa dell'umani\[D]tà,
un desi\[E-]derio d'a\[G]more e pura liber\[D]tà.
Il Dio lon\[F#-]tano è \[G]qui, vicino a \[A]te,
\[D]voce e si\[F#-]lenzio, an\[G]nuncio di novi\[A]tà. \[A4/3]
\endverse

\beginchorus
\[D]\[F#7]A\[B-]ve, Ma\[G]\[G-]ri\[D]a. 
 \[D]\[F#7]A\[B-]ve, Ma\[E-]\[A7]ri\[D]a.
\endchorus

\beginverse
%\chordsoff
^Dio t'ha pre^scelta qual ^madre piena di bel^lezza, 
ed il suo a^more ti av^volgerà con la sua ^ombra.
Grembo per ^Dio ve^nuto sulla ^terra,
^tu sarai ^madre di un ^uomo nuo^vo. ^
\endverse

\beginverse
%\chordsoff
^Ecco l'an^cella che ^vive della tua Pa^rola
libero il ^cuore per^ché l'amore trovi ^casa.
Ora l'at^tesa è ^densa di pre^ghiera,
^e l'uomo ^nuovo è ^qui, in mezzo a ^noi. ^
\endverse
\endsong
%------------------------------------------------------------
%			FINE CANZONE
%------------------------------------------------------------

