%-------------------------------------------------------------
%			INIZIO	CANZONE
%-------------------------------------------------------------


%titolo: 	Osanna eh
%autore: 	
%tonalita: 	Mi 



%%%%%% TITOLO E IMPOSTAZONI
\beginsong{Santo Zaire}[by={Osanna eh! — Autore ignoto}] 	% <<< MODIFICA TITOLO E AUTORE
\transpose{0} 						% <<< TRASPOSIZIONE #TONI (0 nullo)
\momenti{Santo}							% <<< INSERISCI MOMENTI	
% momenti vanno separati da ; e vanno scelti tra:
% Ingresso; Atto penitenziale; Acclamazione al Vangelo; Dopo il Vangelo; Offertorio; Comunione; Ringraziamento; Fine; Santi; Pasqua; Avvento; Natale; Quaresima; Canti Mariani; Battesimo; Prima Comunione; Cresima; Matrimonio; Meditazione; Spezzare del pane;
\ifchorded
	%\textnote{Tonalità originale }	% <<< EV COMMENTI (tonalità originale/migliore)
\fi


%%%%%% INTRODUZIONE
\ifchorded
\vspace*{\versesep}
\musicnote{
\begin{minipage}{0.48\textwidth}
\textbf{Intro}
\hfill 
%( \eighthnote \, 80)   % <<  MODIFICA IL TEMPO
% Metronomo: \eighthnote (ottavo) \quarternote (quarto) \halfnote (due quarti)
\end{minipage}
} 	
\vspace*{-\versesep}
\beginverse*

\nolyrics

%---- Prima riga -----------------------------
\vspace*{-\versesep}
\[E] \[A] \[E] \[B]  \[E]	 % \[*D] per indicare le pennate, \rep{2} le ripetizioni

%---- Ogni riga successiva -------------------
%\vspace*{-\versesep}
%\[G] \[C]  \[D]	

%---- Ev Indicazioni -------------------------			
%\textnote{\textit{(Oppure tutta la strofa)} }	

\endverse
\fi

%%%%% STROFA
\beginverse*		%Oppure \beginverse* se non si vuole il numero di fianco
\memorize 		% <<< DECOMMENTA se si vuole utilizzarne la funzione
%\chordsoff		% <<< DECOMMENTA se vuoi una strofa senza accordi

\[E]Santo, \[A7]santo os\[E]a\[B]nn\[E]a.
\[E]Santo, \[A7]santo os\[E]a\[B]nn\[E]a.

\endverse

%%%%% RITORNELLO
\beginchorus

O\[E]sanna eh! O\[A]sanna \[E]eh!
O\[A]sanna a \[B]Cristo Si\[E]gnor!  \rep{2}

\endchorus

%%%%% STROFA
\beginverse*		%Oppure \beginverse* se non si vuole il numero di fianco
%\memorize 		% <<< DECOMMENTA se si vuole utilizzarne la funzione
%\chordsoff		% <<< DECOMMENTA se vuoi una strofa senza accordi

^I cieli e la terra o Si^gnore \brk sono ^pie^ni di ^te.
^I cieli e la terra o Si^gnore \brk sono ^pie^ni di ^te.

\endverse

%%%%% RITORNELLO
\beginchorus

O\[E]sanna eh! O\[A]sanna \[E]eh!
O\[A]sanna a \[B]Cristo Si\[E]gnor!  \rep{2}

\endchorus

%%%%% STROFA
\beginverse*	%Oppure \beginverse* se non si vuole il numero di fianco
%\memorize 		% <<< DECOMMENTA se si vuole utilizzarne la funzione
%\chordsoff		% <<< DECOMMENTA se vuoi una strofa senza accordi

Bene^detto colui che ^viene \brk nel ^nome ^tuo Si^gnor.
Bene^detto colui che ^viene \brk nel ^nome ^tuo Si^gnor.

\endverse

%%%%% RITORNELLO
\beginchorus

O\[E]sanna eh! O\[A]sanna \[E]eh!
O\[A]sanna a \[B]Cristo Si\[E]gnor!  \rep{2} \[E*]

\endchorus

\endsong
%------------------------------------------------------------
%			FINE CANZONE
%------------------------------------------------------------