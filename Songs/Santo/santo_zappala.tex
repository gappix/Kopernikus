%-------------------------------------------------------------
%			INIZIO	CANZONE
%-------------------------------------------------------------


%titolo: 	Santo Zappalà
%autore: 	
%tonalita: 	Sol 



%%%%%% TITOLO E IMPOSTAZONI
\beginsong{Santo Zappalà}[by={G. Zappalà, A. Mancuso}] 	% <<< MODIFICA TITOLO E AUTORE
\transpose{0} 						% <<< TRASPOSIZIONE #TONI (0 nullo)
\momenti{Santo}							% <<< INSERISCI MOMENTI	
% momenti vanno separati da ; e vanno scelti tra:
% Ingresso; Atto penitenziale; Acclamazione al Vangelo; Dopo il Vangelo; Offertorio; Comunione; Ringraziamento; Fine; Santi; Pasqua; Avvento; Natale; Quaresima; Canti Mariani; Battesimo; Prima Comunione; Cresima; Matrimonio; Meditazione;
\ifchorded
	%\textnote{$\bigstar$ Tonalità originale }	% <<< EV COMMENTI (tonalità originale\migliore)
\fi


%%%%%% INTRODUZIONE
\ifchorded
\vspace*{\versesep}
\musicnote{
\begin{minipage}{0.48\textwidth}
\textbf{Intro}
\hfill 
%( \eighthnote \, 80)   % <<  MODIFICA IL TEMPO
% Metronomo: \eighthnote (ottavo) \quarternote (quarto) \halfnote (due quarti)
\end{minipage}
} 	
\vspace*{-\versesep}
\beginverse*

\nolyrics

%---- Prima riga -----------------------------
\vspace*{-\versesep}
\[G] \[E-] \[B-] \[D7] 	 % \[*D] per indicare le pennate, \rep{2} le ripetizioni


\endverse
\fi








%%%%% RITORNELLO
\beginchorus

\[G]Santo, \[E-]Santo, \[B-]Santo, \brk il Sign\[C]ore Dio dell’uni\[D]verso.
I \[C]cie\[B-]li e la \[C]ter\[E-]ra \brk sono \[C]pieni \[A7]della tua \[D]gloria.

\endchorus




%%%%% RITORNELLO
\beginchorus

Os\[G]an\[D]na, Os\[G]an\[D]na, \brk Os\[C]anna nell’\[D]alto dei ci\[G]eli. \[C*]\[D*] 
Os\[G]an\[D]na, Os\[G]an\[D]na, \brk Os\[C]anna nell’\[D]alto dei ci\[G]eli.
\endchorus





%%%%% STROFA
\beginverse*		%Oppure \beginverse* se non si vuole il numero di fianco
%\memorize 		% <<< DECOMMENTA se si vuole utilizzarne la funzione
%\chordsoff		% <<< DECOMMENTA se vuoi una strofa senza accordi

Benede\[E-]tto co\[C]lui che \[D]viene nel nome del Si\[G]gnore. \[C*] \[D*]

\endverse


%%%%% RITORNELLO
\beginchorus

Os\[G]an\[D]na, Os\[G]an\[D]na, \brk Os\[C]anna nell’\[D]alto dei ci\[G]eli. \[C*]\[D*] 
Os\[G]an\[D]na, Os\[G]an\[D]na, \brk Os\[C]anna nell’\[D]alto dei ci\[G]eli.
\endchorus








\endsong
%------------------------------------------------------------
%			FINE CANZONE
%------------------------------------------------------------


