%-------------------------------------------------------------
%			INIZIO	CANZONE
%-------------------------------------------------------------


%titolo: 	In notte placida
%autore: 	M. Frisina
%tonalita: 	Sol 



%%%%%% TITOLO E IMPOSTAZONI
\beginsong{In notte placida}[by={M. Frisina}] 	% <<< MODIFICA TITOLO E AUTORE
\transpose{0} 					% <<< TRASPOSIZIONE #TONI (0 nullo)
%\preferflats  %SE VOGLIO FORZARE i bemolle come alterazioni
%\prefersharps %SE VOGLIO FORZARE i # come alterazioni
\momenti{Natale}							% <<< INSERISCI MOMENTI	
% momenti vanno separati da ; e vanno scelti tra:
% Ingresso; Atto penitenziale; Acclamazione al Vangelo; Dopo il Vangelo; Offertorio; Comunione; Ringraziamento; Fine; Santi; Pasqua; Avvento; Natale; Quaresima; Canti Mariani; Battesimo; Prima Comunione; Cresima; Matrimonio; Meditazione; Spezzare del pane;
\ifchorded
	%\textnote{Tonalità migliore }	% <<< EV COMMENTI (tonalità originale/migliore)
\fi


%%%%%% INTRODUZIONE
\ifchorded
\vspace*{\versesep}
\musicnote{
\begin{minipage}{0.48\textwidth}
\textbf{Intro}
\hfill 
%( \eighthnote \, 80)   % <<  MODIFICA IL TEMPO
% Metronomo: \eighthnote (ottavo) \quarternote (quarto) \halfnote (due quarti)
\end{minipage}
} 	
\vspace*{-\versesep}
\beginverse*

\nolyrics

%---- Prima riga -----------------------------
\vspace*{-\versesep}
\[C] \[G] \[C] \rep{2}	 % \[*D] per indicare le pennate, \rep{2} le ripetizioni

%---- Ogni riga successiva -------------------
%\vspace*{-\versesep}
%\[G] \[C]  \[D]	

%---- Ev Indicazioni -------------------------			
%\textnote{\textit{(Oppure tutta la strofa)} }	

\endverse
\fi




%%%%% STROFA
\beginverse		%Oppure \beginverse* se non si vuole il numero di fianco
\memorize 		% <<< DECOMMENTA se si vuole utilizzarne la funzione
%\chordsoff		% <<< DECOMMENTA se vuoi una strofa senza accordi
\[C]In \[G]notte \[C]placida, per \[C]mu\[G]to sen\[C]tier,
dai campi del \[F]ciel è discese l’A\[C]mor,
all’alme fe\[F]deli il Reden\[G]tor!
\endverse
\beginverse*
^Nell’^aura è il ^palpito d’un ^gran^de mis^ter:
del nuovo Isra^el è nato il Si^gnor,
il fiore più ^bello ^dei nostri \[C]fior!
del nuovo Isra\[F]el è nato il Si\[C]gnor,
il fiore più \[F]bello \[G]dei nostri \[C]fior!
\endverse



%%%%% RITORNELLO
\beginchorus
\textnote{\textbf{Rit.}}

\[C-]Cantate, o \[G]popoli, \[C-]gloria all’Al\[G]tissimo
\[C-]l’animo ap\[G]rite a spe\[C-]ranza ed a\[G]mor! \rep{2}

\endchorus



%%%%% STROFA
\beginverse		%Oppure \beginverse* se non si vuole il numero di fianco
%\memorize 		% <<< DECOMMENTA se si vuole utilizzarne la funzione
%\chordsoff		% <<< DECOMMENTA se vuoi una strofa senza accordi

^Se l’^aura è ^gelida, se ^fos^co è il c^iel,
oh, vieni al mio ^cuore, vieni a po^sar,
ti vò col mio a^more riscal^dar.
\endverse
\beginverse*
^Se il ^fieno è ^rigido, se il ^ven^to è cru^del,
un cuore che t’^ama voglio a Te ^dar,
un cuor che Te ^brama, ^Gesù cul\[C]lar.
un cuore che t’\[F]ama voglio a Te \[C]dar,
un cuor che Te \[F]brama, \[G]Gesù cul\[C]lar.
\endverse




\endsong
%------------------------------------------------------------
%			FINE CANZONE
%------------------------------------------------------------