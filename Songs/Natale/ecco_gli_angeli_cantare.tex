%-------------------------------------------------------------
%			INIZIO	CANZONE
%-------------------------------------------------------------


%titolo: 	Ecco gli angeli cantare
%autore:  C. Wesley
%tonalita: Re 


%%%%%% TITOLO E IMPOSTAZONI
\beginsong{Ecco gli angeli cantare}[by={Hark! The Herald Angels Sing — C. Wesley}] 	% <<< MODIFICA TITOLO E AUTORE
\transpose{0} 						% <<< TRASPOSIZIONE #TONI (0 nullo)
%\preferflats  %SE VOGLIO FORZARE i bemolle come alterazioni
%\prefersharps %SE VOGLIO FORZARE i # come alterazioni
\momenti{Natale}							% <<< INSERISCI MOMENTI	
% momenti vanno separati da ; e vanno scelti tra:
% Ingresso; Atto penitenziale; Acclamazione al Vangelo; Dopo il Vangelo; Offertorio; Comunione; Ringraziamento; Fine; Santi; Pasqua; Avvento; Natale; Quaresima; Canti Mariani; Battesimo; Prima Comunione; Cresima; Matrimonio; Meditazione; Spezzare del pane;
\ifchorded
	%\textnote{$\bigstar$ Tonalità migliore }	% <<< EV COMMENTI (tonalità originale/migliore)
\fi



%%%%%% INTRODUZIONE
\ifchorded
\vspace*{\versesep}
\musicnote{
\begin{minipage}{0.48\textwidth}
\textbf{Intro}
\hfill 
%( \eighthnote \, 80)   % <<  MODIFICA IL TEMPO
% Metronomo: \eighthnote (ottavo) \quarternote (quarto) \halfnote (due quarti)
\end{minipage}
} 	
\vspace*{-\versesep}
\beginverse*

\nolyrics

%---- Prima riga -----------------------------
\vspace*{-\versesep}
\[G]  \[E-]	\[B] \[E-]

%---- Ogni riga successiva -------------------
\vspace*{-\versesep}
\[A] \[D]  \[D*]	\[A*] \[D*]		

%---- Ev Indicazioni -------------------------			
\textnote{\textit{(melodia del ritornello)} }	

\endverse
\fi




%%%%% STROFA
\beginverse		%Oppure \beginverse* se non si vuole il numero di fianco
\memorize 		% <<< DECOMMENTA se si vuole utilizzarne la funzione
%\chordsoff		& <<< DECOMMENTA se vuoi una strofa senza accordi

\[D]Senti l'angelo che \[D]can\[A]ta:
\[D]"Gloria al \[G]nato \[D*]re \[A*]dei \[D*]re!"
\[D]Pace \[B-]vera e vero a\[E-]more
\[A]ha portato al \[A]mon\[(E)]do in\[A]ter.
\[D]Sveglia dunque \[G]le na\[D*]zio\[A*]ni
\[D]alla gioia \[G]del cre\[D*]a\[A*]to
\[G]e con gli \[E-]angeli \[B7]gri\[E-]diam:
\[A]"Cristo è \[D]nato a Be\[A7]tle\[D]hem!"

\endverse




%%%%% RITORNELLO
\beginchorus

\[G]Senti \[E-]l'ange\[E-*]lo \[B7*]can\[E-*]tar:
\[A]"Gloria al \[D]nato \[D*]Re \[A*]dei \[D*]Re!"

\endchorus



%%%%% STROFA
\beginverse		%Oppure \beginverse* se non si vuole il numero di fianco
%\memorize 		% <<< DECOMMENTA se si vuole utilizzarne la funzione
%\chordsoff		& <<< DECOMMENTA se vuoi una strofa senza accordi

^Cristo è l'unico Si^gno^re
^e per ^sempre ^re^gne^rà,
^desi^derio delle ^genti
^vieni presto ^in me^zzo a ^noi.
^Dio velato in ^carne u^ma^na
^hai voluto ^vera^men^te
^abi^tare ^qui con ^noi
^"Salve o ^nostro Emma^nu^el!"

\endverse




%%%%% RITORNELLO
\beginchorus

\[G]Senti \[E-]l'ange\[E-*]lo \[B7*]can\[E-*]tar:
\[A]"Gloria al \[D]nato \[D*]Re \[A*]dei \[D*]Re!"

\endchorus



%%%%% STROFA
\beginverse		%Oppure \beginverse* se non si vuole il numero di fianco
%\memorize 		% <<< DECOMMENTA se si vuole utilizzarne la funzione
%\chordsoff		& <<< DECOMMENTA se vuoi una strofa senza accordi

^Vero Principe di ^pa^ce
^dona al ^mondo ^li^ber^tà,
^vero s^ole di gius^tizia
^sana Tu l'u^ma^ni^tà.
^Tu sei nato ^perchè ^l'uo^mo
^non possa mai ^più mo^ri^re,
^perchè ^l'uomo ^abbia in ^Te
^una sec^onda ^nasci^ta

\endverse




%%%%% RITORNELLO
\beginchorus

\[G]Senti \[E-]l'ange\[E-*]lo \[B7*]can\[E-*]tar:
\[A]"Gloria al \[D]nato \[D*]Re \[A*]dei \[D*]Re!"

\endchorus





\endsong
%------------------------------------------------------------
%			FINE CANZONE
%------------------------------------------------------------



