%titolo{Il Signore della danza}
%autore{}
%album{}
%tonalita{Re}
%famiglia{Liturgica}
%gruppo{}
%momenti{}
%identificatore{il_signore_della_danza}
%data_revisione{2013_12_31}
%trascrittore{Antonio Badan}
\beginsong{Il Signore della danza}
\beginchorus
|\[D]Danza al\[B-]lor o\[G]vunque tu sa\[A]rai!
Io \[D]sono il Si\[B-]gnore della \[G]danza, \[A]sai!
\[D]E ti condur\[D7]rò do\[G]vunque tu vorrai 
e per \[D]sempre nell’anima tu \[A]danze\[D]rai!
\endchorus
\chordsoff
\beginverse
Danzai al mattino quando tutto incominciò,
nel sole e nella luna il mio spirito danzò!
Son sceso dal cielo per portar la verità
e perciò chi mi segue sempre danzerà!
\endverse
\beginverse
Danzai allora per gli Scribi e i Farisei,
ma erano incapaci e non sapevano imparar.
Quando ai pescatori io proposi di danzar
subito impararono e si misero a danzar!
\endverse
\beginverse
Di sabato volevano impedirmi di danzar,
ad uno stolto a vivere, a sorridere a cantar!
Poi mi inchiodarono al legno di una croce
ma no, non riuscirono a togliermi la voce!
\endverse
\beginverse
Il cielo si oscurò quando danzai il venerdì,
ma è difficile danzar così!
"Nella tomba” pensano “più non danzerà”
ma io sono la danza che mai finirà!
\endverse
\beginverse
Sì, sono vivo e continuo a danzar,
a soffrire, a morire e a ogni dì resuscitar!
Se vivrai in me, io vivrò in te,
e allora vieni e danza insieme a me!
\endverse
\beginverse
Se mi presti il tuo corpo io danzerò in te,
perché la mia gioia è gioire in te!
Quassù in cielo non si suda più,
ma io voglio stancarmi e vengo ancora giù!
\endverse
\endsong
