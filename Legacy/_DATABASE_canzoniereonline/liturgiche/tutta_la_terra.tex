%titolo{Tutta la terra}
%autore{Ricci}
%album{Sei venuto dal cielo}
%tonalita{Re}
%famiglia{Liturgica}
%gruppo{}
%momenti{Natale;Salmi}
%identificatore{tutta_la_terra}
%data_revisione{2014_10_01}
%trascrittore{Francesco Endrici - Manuel Toniato}
\beginsong{Tutta la terra}[by={Ricci}]
\ifchorded
\beginverse*
\vspace*{-0.8\versesep}
{\nolyrics \[D] \[A] \[B-] \[G] }
\vspace*{-\versesep}
\endverse
\fi
\beginchorus
\[D]Tutta la \[A]terra ha ve\[B-]duto la sal\[G]vezza \brk del Si\[D]gnore \[A] \[B-] \[G] \rep{2}
\endchorus

%\chordsoff
\beginverse\memorize
Can\[E-]tate al Signore un canto \[B-]nuovo,
un canto \[A]nuovo perché ha compiuto pro\[D]digi.
Gli ha \[F#7]dato vittoria la sua \[B-]destra,
la sua \[F#7]destra e il suo braccio \[B-]santo. \[A] 
\endverse
\beginchorus
\[D]Tutta la \[A]terra ha ve\[B-]duto la sal\[G]vezza \brk del Si\[D]gnore \[A] \[B-] \[G] 
\endchorus
\beginverse
Il Si^gnore ha mostrato la sal^vezza,
a tutti i ^popoli qual è la giu^stizia.
^Lui si è ricordato del suo a^more
verso il ^popolo di Isra^ele. ^
\endverse
\beginchorus
\[D]Tutta la \[A]terra ha ve\[B-]duto la sal\[G]vezza \brk del Si\[D]gnore \[A] \[B-] \[G] 
\endchorus
\beginverse
^Tutti i confini della ^terra
hanno ve^duto la salvezza di ^Dio.
Ac^clami al Signore tutto il ^mondo,
esul^tate con canti di ^gioia. ^
\endverse
\beginchorus
\[D]Tutta la \[A]terra ha ve\[B-]duto la sal\[G]vezza \brk del Si\[D]gnore \echo{La sal\[A]vezza del Si\[B-]gnore.} 
{\nolyrics \[G] \[D]\[A]\[B-]\[G]\[D]\[A]\[B-]\[G]}
\endchorus
\beginverse
Can^tate inni al Si^gnore
Con l'arpa e con ^suoni, suoni melo^diosi,
con ^trombe e con suoni di ^corno
accla^mate al Re, il Si^gnore. ^
\endverse
\beginchorus
\[D]Tutta la \[A]terra ha ve\[B-]duto la sal\[G]vezza \brk del Si\[D]gnore \[A] \[B-] \[G]
\[D]Tutta la \[A]terra ha ve\[B-]duto la sal\[G]vezza \brk del Si\[D]gnore \echo{La sal\[A]vezza del Si\[B-]gnore.} 
{\nolyrics \[G] \[D]\[A]\[B-]\[G]\[D]\[A]\[B-]\[G]\[D]}
\endchorus
\endsong
