%titolo{Ti esalto Dio mio re}
%autore{Cavalieri, Uva}
%album{Sempre canterò}
%tonalita{Sol}
%famiglia{Liturgica}
%gruppo{}
%momenti{Ingresso}
%identificatore{ti_esalto_dio_mio_re}
%data_revisione{2011_12_31}
%trascrittore{Francesco Endrici - Manuel Toniato}
\beginsong{Ti esalto Dio mio re}[by={Cavalieri, Uva}]

\beginchorus
\[D7]Ti e\[G]salto \[D7]Dio mio \[G]re, \[E-]  
cante\[A-]rò in e\[G]terno a \[D]{te} \[D7] 
io \[G]voglio lo\[E-]darti Si\[A-]gnor \[D7]  
e bene\[G]dirti, \[D7]allelu\[G]ia! \[C] \[G] 
\endchorus

\beginverse
Il Si\[C]gnore è degno di ogni \[G]lode,
non si \[C]può misurar la sua gran\[G]dezza;
ogni vi\[E-]vente pro\[D7]clami la sua \[G]gloria\[C],
la sua \[G]opera è giu\[D]stizia e veri\[G]tà.
\endverse

\beginverse
%\chordsoff
Il Si^gnore è paziente e pie^toso.
Lento all'^ira e ricco di ^Grazia.
Tene^rezza per ^ogni crea^tura, ^
il Si^gnore è ^buono verso ^tutti.
\endverse

\beginverse
%\chordsoff
Il Si^gnore sostiene chi va^cilla
e ^rialza chiunque è ca^duto.
Gli occhi di ^tutti ri^cercano il suo ^volto, ^
la sua ^mano prov^vede loro il ^cibo.
\endverse
\endsong
