%titolo{Guarda che bianca luna}
%autore{}
%album{}
%tonalita{Mi-}
%famiglia{Scout}
%gruppo{}
%momenti{}
%identificatore{guardate_bene}
%data_revisione{2012_12_04}
%trascrittore{Francesco Endrici}
\beginsong{Guarda che bianca luna}
\beginverse
\[E-]Guarda che bianca luna nel \[A-7]cielo \[E-]vedi,
somiglia a te, fratel\[B]lino.
\[E-]Guarda, par che sorrida, non \[A-7]ha pa\[E-]ura,
vorrei an\[B7]darle vi\[E-]cin!
\[G]Son boccioli di \[D]rosso fior
\[E-]quei punti accesi \[B]che stan lassù?
\[E-]No, ma, fratello, corri,
il \[A-7]Branco a\[E-]desso non \[B]vedo \[E-]più.
\endverse
\beginverse
^Un momentino ancora, Fra^tello ^Bigio, 
fammi guardare nel ^cielo. 
^Tutta la Rupe è bianca, la ^bianca ^luna 
l'ha ricoperta ^d'un ^vel.
^Nella Waingunga ^vedo brillar ^la roccia azzurra, ^bella, laggiù. 
^Sì, ma, fratello, corri, 
il ^Branco a^desso non ^vedo ^più.
\endverse
\beginverse*
Lalla la la la la la\dots 
\[E-]ecco una nube nera 
o \[A-7]bianca luna, \[E-]non ci sei \[B]più. \[E-]
\endverse
\endsong
