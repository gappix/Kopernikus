% !TEX encoding = UTF-8 Unicode
%....................................................................................
%
% ██╗  ██╗ ██████╗ ██████╗ ███████╗██████╗ ███╗   ██╗██╗██╗  ██╗██╗   ██╗███████╗
% ██║ ██╔╝██╔═══██╗██╔══██╗██╔════╝██╔══██╗████╗  ██║██║██║ ██╔╝██║   ██║██╔════╝
% █████╔╝ ██║   ██║██████╔╝█████╗  ██████╔╝██╔██╗ ██║██║█████╔╝ ██║   ██║███████╗
% ██╔═██╗ ██║   ██║██╔═══╝ ██╔══╝  ██╔══██╗██║╚██╗██║██║██╔═██╗ ██║   ██║╚════██║
% ██║  ██╗╚██████╔╝██║     ███████╗██║  ██║██║ ╚████║██║██║  ██╗╚██████╔╝███████║
% ╚═╝  ╚═╝ ╚═════╝ ╚═╝     ╚══════╝╚═╝  ╚═╝╚═╝  ╚═══╝╚═╝╚═╝  ╚═╝ ╚═════╝ ╚══════╝
% Font: Ansi Shadow
%....................................................................................


%% CHORDED Settings
%>>>> A4vert + palatino + titleindex + tematicindex + chorded
%\documentclass[a4vert,twocolumns, titleindex,tematicindex,chorded,palatino]{canzoniereonline}
%>>>> a5vert,twocolumns, titleindex,tematicindex,lyric, palatino
\documentclass[a5vert,twocolumns, titleindex,tematicindex,lyric,libertine]{canzoniereonline}%

%opzioni formato: singoli, standard (A4), a5vert, b5vert, a5oriz, a6vert;
%opzioni accordi: lyric, chorded {quelli d Songs}
%opzioni font: palatino, libertine
%opzioni segno minore: "minorsign=quel che vuoi"
%opzioni indici: authorsindex, titleindex, tematicindex
%opzioi copertina: cover e nocover
%opzioni colonne: twocolumns, onecolumn;
\def\canzsongcolumsnumber{2}







% PACCHETTI DA IMPORTARE
%-------------------------------------------------------------------------------                   
\usepackage[italian]{babel}
\usepackage[hidelinks]{hyperref}
\usepackage{wasysym} %per \eightnote e \quarternote



% FRONTESPIZIO
%-------------------------------------------------------------------------------


% RIGHE PER LA COPERTINA
%������������������������������������������������������������������������


%titoletto
\renewcommand{\titolettocop}{BETA 2.0} 	


%TITOLO					
\renewcommand{\titolocop}{KOPERNIKUS}  			


%Sottotitolo
\renewcommand{\sottotitolocop}{La rivoluzione dei canzonieri} 	


%fondo pagina
\renewcommand{\piede}{\today}							



% NUOVI COMANDI E VARIABILI GLOBALI
%--------------------------------------------------------------------------------
%Command to chapter subsectioning
\renewcommand{\songchapter}{\chapter*}

%New command to make colophon appear on even (closing) page
\newcommand*\cleartoleftpage{%
  \clearpage
  \ifodd\value{page}\hbox{}\newpage\fi
}



%Counter globale per tenere traccia di una numerazione progressiva
%Si affianca a un altro counter gia utilizzato nella classe CanzoniereOnLine "songnum" 
%che, tuttavia, si riazzera ognivolta viene definito un nuovo ambiente \beginsongs{}
\newcounter{GlobalSongCounter} 







\begin{document}
\ifcover
	\firstpage
\else
	\relax
\fi





% CAPITOLI DEL CANZONIERE
%-------------------------------------------------------------------------------
% Ciascun capitolo contiene già tutta la logica di gestione della 
% numerazione progressiva, del DB locale da cui attingere le canzoni
% e la creazione/chiusura dell'ambiente in cui vengono importate 
% tutte le canzoni relative
\addtocounter{GlobalSongCounter}{1} %set starting song counter to 1 (0 otherwise)


\songchapter{Alleluia}
%...............................................................................
%
%   █████╗ ██╗     ██╗     ███████╗██╗     ██╗   ██╗██╗ █████╗     
%  ██╔══██╗██║     ██║     ██╔════╝██║     ██║   ██║██║██╔══██╗    
%  ███████║██║     ██║     █████╗  ██║     ██║   ██║██║███████║    
%  ██╔══██║██║     ██║     ██╔══╝  ██║     ██║   ██║██║██╔══██║    
%  ██║  ██║███████╗███████╗███████╗███████╗╚██████╔╝██║██║  ██║    
%  ╚═╝  ╚═╝╚══════╝╚══════╝╚══════╝╚══════╝ ╚═════╝ ╚═╝╚═╝  ╚═╝
% Font: ANSI Shadow
%...............................................................................
\begin{songs}{}
\songcolumns{\canzsongcolumsnumber}
\setcounter{songnum}{\theGlobalSongCounter} %set songnum counter, otherwise would be reset

%set the default path inside current folder
\makeatletter
\def\input@path{{Songs/Alleluia/}}
\makeatother


%***** INSERT SONGS HERE ******

%-------------------------------------------------------------
%			INIZIO	CANZONE
%-------------------------------------------------------------


%titolo: 	Alleluia a te canto
%autore: 	Giombini
%tonalita: 	Do 



%%%%%% TITOLO E IMPOSTAZONI
\beginsong{Alleluia A Te canto}[by={Giombini}] 	% <<< MODIFICA TITOLO E AUTORE
\transpose{-3} 						% <<< TRASPOSIZIONE #TONI (0 nullo)
\momenti{Acclamazione al Vangelo;}							% <<< INSERISCI MOMENTI	
% momenti vanno separati da ; e vanno scelti tra:
% Ingresso; Atto penitenziale; Acclamazione al Vangelo; Dopo il Vangelo; Offertorio; Comunione; Ringraziamento; Fine; Santi; Pasqua; Avvento; Natale; Quaresima; Canti Mariani; Battesimo; Prima Comunione; Cresima; Matrimonio; Meditazione;
\ifchorded
	%\textnote{Tonalità originale }	% <<< EV COMMENTI (tonalità originale/migliore)
\fi


%%%%%% INTRODUZIONE
\ifchorded
\vspace*{\versesep}
\textnote{Intro: \qquad \qquad  }%(\eighthnote 116) % << MODIFICA IL TEMPO
% Metronomo: \eighthnote (ottavo) \quarternote (quarto) \halfnote (due quarti)
\vspace*{-\versesep}
\beginverse*

\nolyrics

%---- Prima riga -----------------------------
\vspace*{-\versesep}
\[C] \[G] \[C*]	 % \[*D] per indicare le pennate, \rep{2} le ripetizioni

%---- Ogni riga successiva -------------------
%\vspace*{-\versesep}
%\[G] \[C]  \[D]	

%---- Ev Indicazioni -------------------------			
\textnote{\textit{(Con stop e attacco solo voce)} }	

\endverse
\fi




%%%%% STROFA
\beginverse		%Oppure \beginverse* se non si vuole il numero di fianco
\memorize 		% <<< DECOMMENTA se si vuole utilizzarne la funzione
%\chordsoff		& <<< DECOMMENTA se vuoi una strofa senza accordi

\[(C)] A te canto alle\[G]luia! 
\[A-] A te dono la mia \[E-]gioia,
\[F] a te grido mio Si\[C]gnore,
\[D7] a te offro ogni do\[G]lo\[7]re!

\endverse




%%%%% RITORNELLO
\beginchorus
\textnote{\textbf{Rit.}}

\[C]Al-\[G]le \[A-]lu ia! \[F]ah! \[C] 
Alle\[E-]luia! \[F]Alle\[G]luia!
\[C]Al-\[G]le \[A-]lu ia! \[F]ah! \[C] 
Alle\[E-]luia! Al\[F]le\[G]lu\[C]ia!  \[C] \[C*]

\endchorus



%%%%% STROFA
\beginverse		%Oppure \beginverse* se non si vuole il numero di fianco
%\memorize 		% <<< DECOMMENTA se si vuole utilizzarne la funzione
%\chordsoff		% <<< DECOMMENTA se vuoi una strofa senza accordi

^ A te dico io ti ^amo,
^ a te dedico la ^vita
^ a te chiedo dammi ^pace,
^ a te grido la mia ^fe^de!


\endverse




\endsong
%------------------------------------------------------------
%			FINE CANZONE
%------------------------------------------------------------



%-------------------------------------------------------------
%			INIZIO	CANZONE
%-------------------------------------------------------------


%titolo: 	Alleluia Canto per Cristo
%autore: 	Costa
%tonalita: 	Re 



%%%%%% TITOLO E IMPOSTAZONI
\beginsong{Alleluia Canto per Cristo}[by={E. Costa}] 	% <<< MODIFICA TITOLO E AUTORE
\transpose{0} 						% <<< TRASPOSIZIONE #TONI (0 nullo)
\momenti{Acclamazione al Vangelo;}							% <<< INSERISCI MOMENTI	
% momenti vanno separati da ; e vanno scelti tra:
% Ingresso; Atto penitenziale; Acclamazione al Vangelo;  Dopo il Vangelo; Offertorio; Comunione; Ringraziamento; Fine; Santi; Pasqua; Avvento; Natale; Quaresima; Canti Mariani; Battesimo; Prima Comunione; Cresima; Matrimonio; Meditazione; Spezzare del pane;
\ifchorded
	%\textnote{Tonalità migliore }	% <<< EV COMMENTI (tonalità originale/migliore)
\fi


%%%%%% INTRODUZIONE
\ifchorded
\vspace*{\versesep}
\musicnote{
\begin{minipage}{0.48\textwidth}
\textbf{Intro:}
\hfill 
%( \eighthnote \, 80)   % <<  MODIFICA IL TEMPO
% Metronomo: \eighthnote (ottavo) \quarternote (quarto) \halfnote (due quarti)
\end{minipage}
} 	
\vspace*{-\versesep}
\beginverse*

\nolyrics

%---- Prima riga -----------------------------
\vspace*{-\versesep}
\[D] \[F#-] \[G] \[D]\[D]	 % \[*D] per indicare le pennate, \rep{2} le ripetizioni

%---- Ogni riga successiva -------------------
%\vspace*{-\versesep}
%\[G] \[C]  \[D]	

%---- Ev Indicazioni -------------------------			
%\textnote{\textit{(Oppure tutta la strofa)} }	

\endverse
\fi



%%%%% RITORNELLO
\beginchorus
\[D]Allelu\[F#-]ia, alle\[G]luia, allelu\[D]ia,
\[G]allelu\[D]ia, alle\[A]lu\[A]ia!
\[D]Allelu\[F#-]ia, alle\[G]luia, allelu\[D]ia,
\[G]allelu\[D]ia, alle\[A]lu\[D]ia!
\endchorus


%%%%% STROFA
\beginverse		%Oppure \beginverse* se non si vuole il numero di fianco
\memorize 		% <<< DECOMMENTA se si vuole utilizzarne la funzione
%\chordsoff		% <<< DECOMMENTA se vuoi una strofa senza accordi
\[D]Canto per \[F#-]Cristo che \[G]mi libere\[D]rà
\[G]quando ver\[D]rà nella \[A]glo\[A]ria,
\[D]quando la \[F#-]vita con \[G]Lui rinasce\[D]rà,
\[G]allelu\[D]ia, alle\[A]lu\[D]ia!
\endverse

%%%%% STROFA
\beginverse		%Oppure \beginverse* se non si vuole il numero di fianco
%\memorize 		% <<< DECOMMENTA se si vuole utilizzarne la funzione
%\chordsoff		% <<< DECOMMENTA se vuoi una strofa senza accordi
^Canto per ^Cristo: in ^Lui rifiori^rà
^ogni spe^ranza per^du^ta,
^ogni crea^tura con ^Lui risorge^rà,
^allelu^ia, alle^lu^ia!
\endverse

%%%%% STROFA
\beginverse		%Oppure \beginverse* se non si vuole il numero di fianco
%\memorize 		% <<< DECOMMENTA se si vuole utilizzarne la funzione
%\chordsoff		% <<< DECOMMENTA se vuoi una strofa senza accordi
^Canto per ^Cristo: un ^giorno torne^rà!
^Festa per ^tutti gli a^mi^ci,
^festa di un ^mondo che ^più non mori^rà,
^allelu^ia, alle^lu^ia!
\endverse


\endsong
%------------------------------------------------------------
%			FINE CANZONE
%------------------------------------------------------------

%-------------------------------------------------------------
%			INIZIO	CANZONE
%-------------------------------------------------------------


%titolo: 	Alleluia e poi
%autore: 	Luca Diliberto, Giuliana Monti
%tonalita: 	Do 


%%%%%% TITOLO E IMPOSTAZONI
\beginsong{Alleluia E poi}[by={L. Diliberto, G. Monti}]
\transpose{0} 						% <<< TRASPOSIZIONE #TONI (0 nullo)
\momenti{Acclamazione al Vangelo}							% <<< INSERISCI MOMENTI	
% momenti vanno separati da ; e vanno scelti tra:
% Ingresso; Atto penitenziale; Acclamazione al Vangelo; Dopo il Vangelo; Offertorio; Comunione; Ringraziamento; Fine; Santi; Pasqua; Avvento; Natale; Quaresima; Canti Mariani; Battesimo; Prima Comunione; Cresima; Matrimonio; Meditazione;
\ifchorded
	%\textnote{Tonalità originale }	% <<< EV COMMENTI (tonalità originale/migliore)
\fi




%%%%%% INTRODUZIONE
\ifchorded
\vspace*{\versesep}
\musicnote{
\begin{minipage}{0.48\textwidth}
\textbf{Intro}
\hfill 
%( \eighthnote \, 80)   % <<  MODIFICA IL TEMPO
% Metronomo: \eighthnote (ottavo) \quarternote (quarto) \halfnote (due quarti)
\end{minipage}
} 	
\vspace*{-\versesep}
\beginverse*

\nolyrics

%---- Prima riga -----------------------------
\vspace*{-\versesep}
\[C] \[G]  \[C]	 % \[*D] per indicare le pennate, \rep{2} le ripetizioni

%---- Ogni riga successiva -------------------
%\vspace*{-\versesep}
%\[G] \[C]  \[D]	

%---- Ev Indicazioni -------------------------			
%\textnote{\textit{(Oppure tutta la strofa)} }	

\endverse
\fi



%%%%% STROFA
\beginverse		%Oppure \beginverse* se non si vuole il numero di fianco
\memorize 		% <<< DECOMMENTA se si vuole utilizzarne la funzione
%\chordsoff		& <<< DECOMMENTA se vuoi una strofa senza accordi
\[C]Chiama, ed \[G]io ver\[A-]rò da \[E-]Te:
\[F]Figlio, nel si\[C]lenzio, mi \[D]accoglie\[G]rai.
\[C]Voce e \[G]{poi\dots} la \[A-]liber\[E-]tà,
\[F]nella Tua Pa\[C]rola cam\[D7]mine\[G]rò.
\endverse



%%%%% RITORNELLO
\beginchorus
\textnote{\textbf{Rit.}}
\[C]Alleluia, \[G]alleluia, \[A-]allelu\[E-]ia,
\[F]alleluia, \[C]alle\[D7]lu\[G]ia,
\[C]Alleluia, \[G]alleluia, \[A-]allelu\[E-]ia,
\[F]alleluia, \[C]alle\[G]lu\[C]ia.

\endchorus




%%%%% STROFA
\beginverse		%Oppure \beginverse* se non si vuole il numero di fianco
%\memorize 		% <<< DECOMMENTA se si vuole utilizzarne la funzione
%\chordsoff		& <<< DECOMMENTA se vuoi una strofa senza accordi

^Danza, ed ^io ver^rò con ^Te:
^Figlio, la Tua ^strada com^prende^rò.
^Luce, e ^poi, nel ^tempo ^tuo,
^oltre il desi^derio ri^pose^rò. 
\endverse



\endsong
%------------------------------------------------------------
%			FINE CANZONE
%------------------------------------------------------------

%-------------------------------------------------------------
%			INIZIO	CANZONE
%-------------------------------------------------------------


%titolo: 	Alleluia Ed oggi ancra
%autore: 	P. Sequeri
%tonalita: 	Re-



%%%%%% TITOLO E IMPOSTAZONI
\beginsong{Alleluia Ed oggi ancora}[by={P. Sequeri}]% <<< MODIFICA TITOLO E AUTORE
\transpose{0} 						% <<< TRASPOSIZIONE #TONI (0 nullo)
\momenti{Acclamazione al Vangelo}							% <<< INSERISCI MOMENTI	
% momenti vanno separati da ; e vanno scelti tra:
% Ingresso; Atto penitenziale; Acclamazione al Vangelo; Dopo il Vangelo; Offertorio; Comunione; Ringraziamento; Fine; Santi; Pasqua; Avvento; Natale; Quaresima; Canti Mariani; Battesimo; Prima Comunione; Cresima; Matrimonio; Meditazione; Spezzare del pane;
\ifchorded
	%\textnote{Tonalità migliore }	% <<< EV COMMENTI (tonalità originale/migliore)
\fi



%%%%%% INTRODUZIONE
\ifchorded
\vspace*{\versesep}
\musicnote{
\begin{minipage}{0.48\textwidth}
\textbf{Intro}
\hfill 
( \eighthnote  \, 132)
\end{minipage}
} 
\vspace*{-\versesep}
\beginverse*

\nolyrics

%---- Prima riga -----------------------------
\vspace*{-\versesep}
\[(D-)] \[G-] \[C7] \[F]   % \[*D] per indicare le pennate, \rep{2} le ripetizioni

%---- Ogni riga successiva -------------------
\vspace*{-\versesep}
\[B&6] \[G-] \[A7]\[D-]

%---- Ev Indicazioni -------------------------			
%\textnote{\textit{(Oppure tutta la strofa)} }	

\endverse
\fi



\beginchorus
\textnote{\textbf{Rit.}}
\[(D-)] Alle\[G-]luia, \[C7] allelu\[F]ia, \quad \[B&6] 
alle\[G-]luia, \[A7] alle\[D-]luia,
\[D-7] alle\[G-]luia, \[C7] allelu\[F]ia, \quad \[B&6]
  alle\[G-]luia, \[A7] alle\[D-]luia.
\endchorus

\beginverse*
Ed oggi an\[D-]cora, mio Si\[G-]gnore, \brk ascolte\[C]rò la tua pa\[F]rola,
che mi \[B&]guida nel cam\[G-]mino della \[A4]vita. \[A]
\endverse


\endsong
%------------------------------------------------------------
%			FINE CANZONE
%------------------------------------------------------------




%-------------------------------------------------------------
%			INIZIO	CANZONE
%-------------------------------------------------------------


%titolo: 	Alleluia, festa con Te
%autore: 	Fabio Avolio
%tonalita: 	Do 



%%%%%% TITOLO E IMPOSTAZONI
\beginsong{Alleluia Festa con Te}[by={F. Avolio}] 	% <<< MODIFICA TITOLO E AUTORE
\transpose{0} 						% <<< TRASPOSIZIONE #TONI (0 nullo)
\momenti{Acclamazione al Vangelo}							% <<< INSERISCI MOMENTI	
% momenti vanno separati da ; e vanno scelti tra:
% Ingresso; Atto penitenziale; Acclamazione al Vangelo; Dopo il Vangelo; Offertorio; Comunione; Ringraziamento; Fine; Santi; Pasqua; Avvento; Natale; Quaresima; Canti Mariani; Battesimo; Prima Comunione; Cresima; Matrimonio; Meditazione; Spezzare del pane;
\ifchorded
	%\textnote{Tonalità originale }	% <<< EV COMMENTI (tonalità originale/migliore)
\fi

%%%%%% INTRODUZIONE
\ifchorded
\vspace*{\versesep}
\textnote{Intro: \qquad \qquad  }%(\eighthnote 116) % <<  MODIFICA IL TEMPO
% Metronomo: \eighthnote (ottavo) \quarternote (quarto) \halfnote (due quarti)
\vspace*{-\versesep}
\beginverse*

\nolyrics

%---- Prima riga -----------------------------
\vspace*{-\versesep}
\[C] \[F] \[G] \[A-] 	 % \[*D] per indicare le pennate, \rep{2} le ripetizioni

%---- Ogni riga successiva -------------------
\vspace*{-\versesep}
\[F] \[C] \[D-] \[G]

%---- Ev Indicazioni -------------------------			
%\textnote{\textit{(Oppure tutta la strofa)} }	

\endverse
\fi

%%%%% RITORNELLO
\beginchorus
\textnote{\textbf{Rit.}}

\[C]Allelu\[F]ia, \[G]allelu\[A-]ia, \[F]oggi è \[C]festa con \[D-]te, Ges\[G]ù.
\[C]Tu sei con \[F]noi, \[G]gioia ci \[A-]dai, \[F]allelu\[C]ia, alle\[G]lu\[C]ia.

\endchorus

%%%%% STROFA
\beginverse		%Oppure \beginverse* se non si vuole il numero di fianco
\memorize 		% <<< DECOMMENTA se si vuole utilizzarne la funzione
%\chordsoff		% <<< DECOMMENTA se vuoi una strofa senza accordi

\[C]Nella Tua \[G]casa \[C]siamo ve\[F]nuti \brk \[C]per incon\[D-]trar\[G]ti.
\[C]A Te can\[G]tiamo \[A-]la nostra \[F]lode,  \brk  \[C]gloria al Tuo \[G]no\[C]me. \[G]

\endverse

%%%%% STROFA
\beginverse		%Oppure \beginverse* se non si vuole il numero di fianco
%\memorize 		% <<< DECOMMENTA se si vuole utilizzarne la funzione
%\chordsoff		% <<< DECOMMENTA se vuoi una strofa senza accordi

^Il pane ^vivo ^che ci hai pro^messo  \brk  ^dona la ^vi^ta.
^A Te can^tiamo ^la nostra ^lode,  \brk  ^gloria al Tuo ^no^me. ^

\endverse

%%%%% STROFA
\beginverse		%Oppure \beginverse* se non si vuole il numero di fianco
%\memorize 		% <<< DECOMMENTA se si vuole utilizzarne la funzione
%\chordsoff		% <<< DECOMMENTA se vuoi una strofa senza accordi

^Tu sei l'^amico ^che ci accom^pagna  \brk  ^lungo il cam^mi^no.
^A Te can^tiamo ^la nostra ^lode,  \brk  ^gloria al Tuo ^no^me. ^

\endverse

\endsong
%------------------------------------------------------------
%			FINE CANZONE
%------------------------------------------------------------
%-------------------------------------------------------------
%			INIZIO	CANZONE
%-------------------------------------------------------------


%titolo: 	Alleluia, gente di tutto il mondo
%autore: 	Pierangelo Sequeri
%tonalita: 	Fa



%%%%%% TITOLO E IMPOSTAZONI
\beginsong{Alleluia, gente di tutto il mondo}[by={Pierangelo Sequeri}] 	% <<< MODIFICA TITOLO E AUTORE
\transpose{0} 						% <<< TRASPOSIZIONE #TONI (0 nullo)
\momenti{Acclamazione al Vangelo}							% <<< INSERISCI MOMENTI	
% momenti vanno separati da ; e vanno scelti tra:
% Ingresso; Atto penitenziale; Acclamazione al Vangelo; Dopo il Vangelo; Offertorio; Comunione; Ringraziamento; Fine; Santi; Pasqua; Avvento; Natale; Quaresima; Canti Mariani; Battesimo; Prima Comunione; Cresima; Matrimonio; Meditazione; Spezzare del pane;
\ifchorded
	%\textnote{Tonalità migliore }	% <<< EV COMMENTI (tonalità originale/migliore)
\fi

%%%%%% INTRODUZIONE
\ifchorded
\vspace*{\versesep}
\textnote{Intro: \qquad \qquad  }%(\eighthnote 116) % <<  MODIFICA IL TEMPO
% Metronomo: \eighthnote (ottavo) \quarternote (quarto) \halfnote (due quarti)
\vspace*{-\versesep}
\beginverse*

\nolyrics

%---- Prima riga -----------------------------
\vspace*{-\versesep}
\[F] \[B&] \[F]	 % \[*D] per indicare le pennate, \rep{2} le ripetizioni

%---- Ogni riga successiva -------------------
%\vspace*{-\versesep}
%\[G] \[C]  \[D]	

%---- Ev Indicazioni -------------------------			
%\textnote{\textit{(Oppure tutta la strofa)} }	

\endverse
\fi

%%%%% STROFA
\beginverse		%Oppure \beginverse* se non si vuole il numero di fianco
\memorize 		% <<< DECOMMENTA se si vuole utilizzarne la funzione
%\chordsoff		% <<< DECOMMENTA se vuoi una strofa senza accordi

\[F]Gente di \[A7]tutto il \[G-]mondo
ascol\[C7]tate il nostro \[F]can\[F7]to,
\[B&]lieti vi annun\[F]ciamo:
il Si\[B&-]gno\[C7]re è ri\[F]sorto!

\endverse

%%%%% RITORNELLO
\beginchorus
\textnote{\textbf{Rit.}}

\[F]Allelu\[D-]ia, \[G-]allelu\[C]ia, 
\[F]allelu, allelu, \[B&]allelu\[C7]ia.
\[F]Allelu\[D-]ia, \[G-]allelu\[C]ia,
\[F]allelu, \[C7]allelu\[F]ia. \[B&] \[F]

\endchorus

%%%%% STROFA
\beginverse		%Oppure \beginverse* se non si vuole il numero di fianco
%\memorize 		% <<< DECOMMENTA se si vuole utilizzarne la funzione
\chordsoff		% <<< DECOMMENTA se vuoi una strofa senza accordi

^Il Figlio ^tanto a^mato
che il Dio ^nostro ci ha do^na^to
^l'ha risusci^tato
per la ^vi^ta del ^mondo!

\endverse

%%%%% STROFA
\beginverse		%Oppure \beginverse* se non si vuole il numero di fianco
%\memorize 		% <<< DECOMMENTA se si vuole utilizzarne la funzione
\chordsoff		% <<< DECOMMENTA se vuoi una strofa senza accordi

^Diede la ^propria ^vita
per a^more dei ^fra^telli.
^Vinta ormai la ^morte
è per ^sem^pre con ^noi.

\endverse

%%%%% STROFA
\beginverse		%Oppure \beginverse* se non si vuole il numero di fianco
%\memorize 		% <<< DECOMMENTA se si vuole utilizzarne la funzione
\chordsoff		% <<< DECOMMENTA se vuoi una strofa senza accordi

^Vivere ^del suo a^more
nell'at^tesa che ri^tor^ni:
^questa è la pa^rola
che ci ^do^na spe^ranza.

\endverse

\endsong
%------------------------------------------------------------
%			FINE CANZONE
%------------------------------------------------------------

%++++++++++++++++++++++++++++++++++++++++++++++++++++++++++++
%			CANZONE TRASPOSTA
%++++++++++++++++++++++++++++++++++++++++++++++++++++++++++++
\ifchorded
%decremento contatore per avere stesso numero
\addtocounter{songnum}{-1} 
\beginsong{Alleluia, gente di tutto il mondo}[by={Pierangelo Sequeri}] 	% <<< COPIA TITOLO E AUTORE
\transpose{+2} 						% <<< TRASPOSIZIONE #TONI + - (0 nullo)
%\preferflats  %SE VOGLIO FORZARE i bemolle come alterazioni
%\prefersharps %SE VOGLIO FORZARE i # come alterazioni
\ifchorded
	%\textnote{Tonalità originale}	% <<< EV COMMENTI (tonalità originale/migliore)
\fi


%%%%%% INTRODUZIONE
\ifchorded
\vspace*{\versesep}
\textnote{Intro: \qquad \qquad  }%(\eighthnote 116) % <<  MODIFICA IL TEMPO
% Metronomo: \eighthnote (ottavo) \quarternote (quarto) \halfnote (due quarti)
\vspace*{-\versesep}
\beginverse*

\nolyrics

%---- Prima riga -----------------------------
\vspace*{-\versesep}
\[F] \[B&] \[F]	 % \[*D] per indicare le pennate, \rep{2} le ripetizioni

%---- Ogni riga successiva -------------------
%\vspace*{-\versesep}
%\[G] \[C]  \[D]	

%---- Ev Indicazioni -------------------------			
%\textnote{\textit{(Oppure tutta la strofa)} }	

\endverse
\fi

%%%%% STROFA
\beginverse		%Oppure \beginverse* se non si vuole il numero di fianco
\memorize 		% <<< DECOMMENTA se si vuole utilizzarne la funzione
%\chordsoff		% <<< DECOMMENTA se vuoi una strofa senza accordi

\[F]Gente di \[A7]tutto il \[G-]mondo
ascol\[C7]tate il nostro \[F]can\[F7]to,
\[Bb]lieti vi annun\[F]ciamo:
il Si\[B&-]gno\[C7]re è ri\[F]sorto!

\endverse

%%%%% RITORNELLO
\beginchorus
\textnote{\textbf{Rit.}}

\[F]Allelu\[D-]ia, \[G-]allelu\[C]ia, 
\[F]allelu, allelu, \[B&]allelu\[C7]ia.
\[F]Allelu\[D-]ia, \[G-]allelu\[C]ia,
\[F]allelu, \[C7]allelu\[F]ia. \[B&] \[F]

\endchorus

%%%%% STROFA
\beginverse		%Oppure \beginverse* se non si vuole il numero di fianco
%\memorize 		% <<< DECOMMENTA se si vuole utilizzarne la funzione
\chordsoff		% <<< DECOMMENTA se vuoi una strofa senza accordi

^Il Figlio ^tanto a^mato
che il Dio ^nostro ci ha do^na^to
^l'ha risusci^tato
per la ^vi^ta del ^mondo!

\endverse

%%%%% STROFA
\beginverse		%Oppure \beginverse* se non si vuole il numero di fianco
%\memorize 		% <<< DECOMMENTA se si vuole utilizzarne la funzione
\chordsoff		% <<< DECOMMENTA se vuoi una strofa senza accordi

^Diede la ^propria ^vita
per a^more dei ^fra^telli.
^Vinta ormai la ^morte
è per ^sem^pre con ^noi.

\endverse

%%%%% STROFA
\beginverse		%Oppure \beginverse* se non si vuole il numero di fianco
%\memorize 		% <<< DECOMMENTA se si vuole utilizzarne la funzione
\chordsoff		% <<< DECOMMENTA se vuoi una strofa senza accordi

^Vivere ^del suo a^more
nell'at^tesa che ri^tor^ni:
^questa è la pa^rola
che ci ^do^na spe^ranza.

\endverse

\endsong


\fi
%++++++++++++++++++++++++++++++++++++++++++++++++++++++++++++
%			FINE CANZONE TRASPOSTA
%++++++++++++++++++++++++++++++++++++++++++++++++++++++++++++

%-------------------------------------------------------------
%			INIZIO	CANZONE
%-------------------------------------------------------------


%titolo: 	Alleluia la nostra festa
%autore: 	Varnavà
%tonalita: 	Do

\beginsong{Alleluia La nostra festa }[ititle={Alleluia delle lampadine}, by={Alleluia delle lampadine — S. Varnavà}]
\transpose{0} 						% <<< TRASPOSIZIONE #TONI (0 nullo)
\momenti{Acclamazione al Vangelo}							% <<< INSERISCI MOMENTI	
% momenti vanno separati da ; e vanno scelti tra:
% Ingresso; Atto penitenziale; Acclamazione al Vangelo; Dopo il Vangelo; Offertorio; Comunione; Ringraziamento; Fine; Santi; Pasqua; Avvento; Natale; Quaresima; Canti Mariani; Battesimo; Prima Comunione; Cresima; Matrimonio; Meditazione;
\ifchorded
	%\textnote{Tonalità originale }	% <<< EV COMMENTI (tonalità originale/migliore)
\fi


%%%%%% INTRODUZIONE
\ifchorded
\vspace*{\versesep}
\textnote{Intro: \qquad \qquad  }%(\eighthnote 116) % << MODIFICA IL TEMPO
% Metronomo: \eighthnote (ottavo) \quarternote (quarto) \halfnote (due quarti)
\vspace*{-\versesep}
\beginverse*

\nolyrics

%---- Prima riga -----------------------------
\vspace*{-\versesep}
\[C] \[F]  \[C] \[G]	 % \[*D] per indicare le pennate, \rep{2} le ripetizioni

%---- Ogni riga successiva -------------------
\vspace*{-\versesep}
\[C]\[F] \[*C] \[*G] \[C]  

%---- Ev Indicazioni -------------------------			
\textnote{\textit{(Come il ritornello)} }	

\endverse
\fi




%%%%% RITORNELLO
\beginchorus

\[C]Alle\[F]luia, alleluia, \[C]alle\[G]luia, alleluia,
\[C]alle\[F]luia, alleluia, \[*C]al\[*G]lelu\[C]ia. \rep{2}

\endchorus


%%%%% STROFA
\beginverse*

\[A-]La nostra \[D-]festa non \[G7]deve fi\[C]nire
non \[A-]deve fi\[D-]nire e \[G7]non fini\[C]rà. \rep{2}

\endverse
\beginverse*

Per\[F]ché la \[G]festa \[A-]siamo \[E-]noi
che \[F]cammi\[G]niamo verso \[C]Te, \[C7]
per\[F]ché la \[G]festa \[A-]siamo \[E-]noi
can\[D]tando in\[7]sieme co\[G7]sì:

\endverse


%%%%% RITORNELLO
\beginchorus

\[C]Alle\[F]luia, alleluia, \[C]alle\[G]luia, alleluia,
\[C]alle\[F]luia, alleluia, \[*C]al\[*G]lelu\[C]ia. \rep{2}

\endchorus


\endsong
%------------------------------------------------------------
%			FINE CANZONE
%------------------------------------------------------------



%-------------------------------------------------------------
%			INIZIO	CANZONE
%-------------------------------------------------------------


%titolo: 	Alleluia lodate il Signore
%autore: 	Frisina
%tonalita: 	RE



%%%%%% TITOLO E IMPOSTAZONI
\beginsong{Alleluia Lodate il Signore}[by={Frisina}] 	% <<< MODIFICA TITOLO E AUTORE
\transpose{0} 						% <<< TRASPOSIZIONE #TONI (0 nullo)
\momenti{Acclamazione al Vangelo;}							% <<< INSERISCI MOMENTI	
% momenti vanno separati da ; e vanno scelti tra:
% Ingresso; Atto penitenziale; Acclamazione al Vangelo; Dopo il Vangelo; Offertorio; Comunione; Ringraziamento; Fine; Santi; Pasqua; Avvento; Natale; Quaresima; Canti Mariani; Battesimo; Prima Comunione; Cresima; Matrimonio; Meditazione;
\ifchorded
	%\textnote{Tonalità originale }	% <<< EV COMMENTI (tonalità originale/migliore)
\fi




%%%%%% INTRODUZIONE
\ifchorded
\vspace*{\versesep}
\textnote{Intro: \qquad \qquad  }%(\eighthnote 116) % << MODIFICA IL TEMPO
% Metronomo: \eighthnote (ottavo) \quarternote (quarto) \halfnote (due quarti)
\vspace*{-\versesep}
\beginverse*

\nolyrics

%---- Prima riga -----------------------------
\vspace*{-\versesep}
\[A] \[G] \[D] \[E-] \[B-] % \[*D] per indicare le pennate, \rep{2} le ripetizioni

%---- Ogni riga successiva -------------------
\vspace*{-\versesep}
\[G] \[A]  \[D]	 \[A]  

%---- Ev Indicazioni -------------------------			
\textnote{\textit{(Come le prime due righe)} }	

\endverse
\fi





%%%%% RITORNELLO
\beginchorus
\textnote{\textbf{Rit.}}
\[A]Alle\[G]lu\[D]ia alle\[E-]lu\[B-]ia, 
lo\[G]date il Si\[A]gno\[D]re \[A]
alle\[G]lu\[D]ia alle\[E-]lu\[B-]ia, 
lo\[G]date il Si\[A]gno\[D]re.
\endchorus


%%%%% STROFA
\beginverse
Lo\[B&]date il Si\[F]gnore nel \[B&]suo tempio \[F]santo
lo\[G-]datelo nell'\[D-]alto firma\[C]mento
lo\[B&]datelo nei \[F]grandi pro\[B&]digi del suo a\[F]more
lo\[G]datene l'ec\[A-]celsa sua \[E]mae\[A]stà.
\endverse


%%%%% STROFA
\beginverse
\chordsoff
Lodatelo col suono gioioso delle trombe,
lodatelo sull'arpa e sulla cetra.
Lodatelo col suono dei timpani e dei sistri,
lodatelo coi flauti e sulle corde. 
\endverse


%%%%% STROFA
\beginverse
\chordsoff
Lodatelo col suono dei cimbali sonori,
lodatelo con cimbali squillanti.
Lodate il Signore voi tutte creature,
Lodate e cantate al Signore. 
\endverse


%%%%% STROFA
\beginverse
\chordsoff
Lodate voi tutti suoi angeli dei cieli,
Lodatelo voi tutte sue schiere.
Lodatelo voi cieli, voi astri e voi stelle,
lodate il Signore Onnipotente. 
\endverse


%%%%% STROFA
\beginverse
\chordsoff
Voi tutti governanti e genti della terra,
lodate il nome santo del Signore.
Perché solo la sua gloria risplende sulla terra,
lodate e benedite il Signore. 
\endverse
\endsong
%------------------------------------------------------------
%			FINE CANZONE
%------------------------------------------------------------



%-------------------------------------------------------------
%			INIZIO	CANZONE
%-------------------------------------------------------------


%titolo: 	Alleluia lode cosmica
%autore: 	Puri
%tonalita: 	Lam 


%%%%%% TITOLO E IMPOSTAZONI
\beginsong{Alleluia Lode cosmica}[by={S. Puri}]
\transpose{7}
\momenti{Acclamazione al Vangelo;}							% <<< INSERISCI MOMENTI	
% momenti vanno separati da ; e vanno scelti tra:
% Ingresso; Atto penitenziale; Acclamazione al Vangelo; Dopo il Vangelo; Offertorio; Comunione; Ringraziamento; Fine; Santi; Pasqua; Avvento; Natale; Quaresima; Canti Mariani; Battesimo; Prima Comunione; Cresima; Matrimonio; Meditazione;
\ifchorded
	\textnote{$\bigstar$ Tonalità migliore per le bambine }	% <<< EV COMMENTI (tonalità originale/migliore)
\fi



%%%%%% INTRODUZIONE
\ifchorded
\vspace*{\versesep}
\musicnote{
\begin{minipage}{0.48\textwidth}
\textbf{Intro:}
\hfill 
%( \eighthnote \, 80)   % <<  MODIFICA IL TEMPO
% Metronomo: \eighthnote (ottavo) \quarternote (quarto) \halfnote (due quarti)
\end{minipage}
} 	
\vspace*{-\versesep}
\beginverse*

\nolyrics

%---- Prima riga -----------------------------
\vspace*{-\versesep}
\[D-] \[F]  \[C] \[C]	 % \[*D] per indicare le pennate, \rep{2} le ripetizioni

%---- Ogni riga successiva -------------------
\vspace*{-\versesep}
\[D-] \[F]  \[C] \[C]

%---- Ev Indicazioni -------------------------			
%\textnote{\textit{(Oppure tutta la strofa)} }	

\endverse
\fi





%%%%% RITORNELLO
\beginchorus
\textnote{\textbf{Rit.}}
\[D-]Alleluia, \[F]àllelu\[C]ia! 	\[C]
\[G-]Alleluia, \[B&]àllelu\[D-]ia! \[C6]
Alle\[B&]lu-u\[F]ia, al\[G-]le-elu\[D-]ia!
Alle\[C]lù\[D-]ia,	alle\[A4/7]lù\[A7]ia! \iflyric \rep{2} \fi
\ifchorded
\[D-]Alleluia, \[F]àllelu\[C]ia! 	\[C]
\[G-]Alleluia, \[B&]àllelu\[D-]ia! \[C6]
Alle\[B&]lu-u\[F]ia, al\[G-]le-elu\[D-]ia!
Alle\[C]lù\[D-]ia,	alle\[A4/7]lu\[A4/7]u\[A7]\[A7]ia! 
\fi
\endchorus

%%%%% STROFA
\beginverse
\memorize
\[D-]Lodino il Si\[B&]gnor i \[F]cie\[C]li, 
\[D-]lodino il Si\[B&]gnor i \[F]ma\[C]ri,
gli \[G-]angeli, i \[G-]cieli dei \[B&]cieli:
il Suo \[D-]nome è \[D-]grande e su\[A4/7]bli\[A7]me.

\[D-]Sole, luna e \[B&]stelle ar\[F]den\[C]ti, 
\[D-]neve, pioggia, \[B&]nebbia, e \[F]fuo\[C]co 
\[G-]lodino \[G-] il Suo \[B&]nome \[B&]in e\[D-]ter\[D-]no! 
\[A4/7]Sia \[A7]lode al Si\[B&7+]gnor! \[B&]
\[C6]Sia	\[(B&7+)]lode al Suo \[B&7+]nome
\[A4]Sia \[A]lode al Signor!
\endverse






%%%%% RITORNELLO
\beginchorus
\textnote{\textbf{Rit.}}
\[D-]Alleluia, \[F]àllelu\[C]ia! 	\[C] \iflyric \dots \rep{2}\fi
\ifchorded
\[G-]Alleluia, \[B&]àllelu\[D-]ia! \[C6]
Alle\[B&]lu-u\[F]ia, al\[G-]le-elu\[D-]ia!
Alle\[C]lù\[D-]ia,	alle\[A4/7]lù\[A7]ia!  
\[D-]Alleluia, \[F]àllelu\[C]ia! 	\[C]
\[G-]Alleluia, \[B&]àllelu\[D-]ia! \[C6]
Alle\[B&]lu-u\[F]ia, al\[G-]le-elu\[D-]ia!
Alle\[C]lù\[D-]ia,	alle\[A4/7]lu\[A4/7]u\[A7]\[A7]ia! 
\fi
\endchorus







\beginverse
^Lodino il Si^gnor le ^ter^re, 
^lodino il Si^gnor i ^mon^ti,
il ^vento ^ della tem^pesta
che obbe^disce al^la Sua ^vo^ce,

^giudici, so^vrani ^tut^ti, 
^giovani, fan^ciulle, ^vec^chi 
^lodino ^ la Sua ^Gloria ^in e^ter^no! 
^Sia ^lode al Si^gnor! ^
^Sia ^lode al Suo ^nome!
^Sia ^lode al Signor!
\endverse


%%%%% RITORNELLO
\beginchorus
\textnote{\textbf{Rit.}}
\[D-]Alleluia, \[F]àllelu\[C]ia! 	\[C] \iflyric \dots \fi
\ifchorded
\[G-]Alleluia, \[B&]àllelu\[D-]ia! \[C6]
Alle\[B&]lu-u\[F]ia, al\[G-]le-elu\[D-]ia!
Alle\[C]lù\[D-]ia,	alle\[A4/7]lù\[A7]ia!  
\fi
\endchorus

\beginchorus %oppure \beginverse*
\vspace*{1.3\versesep}
\textnote{\textbf{Finale} \textit{(rallentando)}} %<<< EV. INDICAZIONI
\[D-]Alleluia, \[F]àllelu\[C]ia! 	\[C]
\[G-]Alleluia, \[B&]àllelu\[D-]ia! \[C6]
Alle\[B&]lu-u\[F]ia, al\[G-]le-elu\[D-]ia!
Alle\[C]lu\[D-]ia,	
alle\[A4/7]lu\[A4/7]u\[A7]u\[A7]u\[B&7]\[B&7]ia!
A-lle-\[B&6]lu-\[B&6]u-\[D]ia!

\endchorus

\endsong


%-------------------------------------------------------------
%			INIZIO	CANZONE
%-------------------------------------------------------------


%titolo: 	Alleluia Passeranno i cieli
%autore: 	Costa, Varnavà
%tonalita: Fa 
\beginsong{Alleluia Passeranno i cieli}[by={E. Costa, S. Varnavà}]
\transpose{3} 						% <<< TRASPOSIZIONE #TONI (0 nullo)
\momenti{Acclamazione al Vangelo}							% <<< INSERISCI MOMENTI	
% momenti vanno separati da ; e vanno scelti tra:
% Ingresso; Atto penitenziale; Acclamazione al Vangelo; Dopo il Vangelo; Offertorio; Comunione; Ringraziamento; Fine; Santi; Pasqua; Avvento; Natale; Quaresima; Canti Mariani; Battesimo; Prima Comunione; Cresima; Matrimonio; Meditazione;
\ifchorded
	%\textnote{Tonalità originale }	% <<< EV COMMENTI (tonalità originale/migliore)
\fi





%%%%%% INTRODUZIONE
\ifchorded
\vspace*{\versesep}
\musicnote{
\begin{minipage}{0.48\textwidth}
\textbf{Intro:}
\hfill 
( \eighthnote \, 120)   % <<  MODIFICA IL TEMPO
% Metronomo: \eighthnote (ottavo) \quarternote (quarto) \halfnote (due quarti)
\end{minipage}
} 	
\vspace*{-\versesep}
\beginverse*

\nolyrics

%---- Prima riga -----------------------------
\vspace*{-\versesep}
\[D] \[A]  \[(D)]	 % \[*D] per indicare le pennate, \rep{2} le ripetizioni

%---- Ogni riga successiva -------------------
%\vspace*{-\versesep}
%\[G] \[C]  \[D]	

%---- Ev Indicazioni -------------------------			
%\textnote{\textit{(Oppure tutta la strofa)} }	

\endverse
\fi






\beginchorus
\[D]Alle\[A]luia, \[B-]alleluia, \[F#-]alleluia,
\[G]alleluia, \[D]allelu\[E-]ia, 
\[A7*]a-\[D]alle\[G]luia, al\[A7]lelu\[D]ia.
\endchorus


\beginverse*
\[D]Passeranno i \[A]cieli e \[B-]passerà la \[F#-]terra,
\[G]la Tua parola \[D]non passe\[E-]rà. 
\[A7*]a-\[D]alle\[G]luia, al\[A7]lelu\[D]ia.
\endverse




\endsong


%-------------------------------------------------------------
%			INIZIO	CANZONE
%-------------------------------------------------------------


%titolo: 	Alleluia Rendete grazie
%autore: 	Gen Verde
%tonalita: 	Sib


%%%%%% TITOLO E IMPOSTAZONI
\beginsong{Alleluia Rendete grazie}[by={Gen Verde}] 	% <<< MODIFICA TITOLO E AUTORE
\transpose{0} 						% <<< TRASPOSIZIONE #TONI (0 nullo)
\momenti{Acclamazione al Vangelo;}							% <<< INSERISCI MOMENTI	
% momenti vanno separati da ; e vanno scelti tra:
% Ingresso; Atto penitenziale; Acclamazione al Vangelo; Dopo il Vangelo; Offertorio; Comunione; Ringraziamento; Fine; Santi; Pasqua; Avvento; Natale; Quaresima; Canti Mariani; Battesimo; Prima Comunione; Cresima; Matrimonio; Meditazione;
\ifchorded
	%\textnote{Tonalità originale }	% <<< EV COMMENTI (tonalità originale/migliore)
\fi


%%%%%% INTRODUZIONE
\ifchorded
\vspace*{\versesep}
\musicnote{
\begin{minipage}{0.48\textwidth}
\textbf{Intro:}
\hfill 
%( \eighthnote \, 80)   % <<  MODIFICA IL TEMPO
% Metronomo: \eighthnote (ottavo) \quarternote (quarto) \halfnote (due quarti)
\end{minipage}
} 	
\vspace*{-\versesep}
\beginverse*

\nolyrics

%---- Prima riga -----------------------------
\vspace*{-\versesep}
\[B&] \[C] \[D-] \[A-]	 % \[*D] per indicare le pennate, \rep{2} le ripetizioni

%---- Ogni riga successiva -------------------
\vspace*{-\versesep}
\[B&] \[C] \[F] \[F]

%---- Ev Indicazioni -------------------------			
%\textnote{\textit{(Con stop e attacco solo voce)} }	

\endverse
\fi





%%%%% RITORNELLO
\beginchorus
\textnote{\textbf{Rit.}}
Alle\[B&]luia, alle\[C]luia, alle\[D-]luia, \[A-]
alle\[B&]luia, alle\[C4]l\[C]u\[F]ia. \[A-]
Alle\[B&]luia, alle\[C]luia, alle\[D-]luia, \[A-]
alle\[B&]luia, alle\[C4]l\[C]u\[F]ia. \[F]

\endchorus




%%%%% STROFA
\beginverse		%Oppure \beginverse* se non si vuole il numero di fianco
\memorize 		% <<< DECOMMENTA se si vuole utilizzarne la funzione
%\chordsoff		& <<< DECOMMENTA se vuoi una strofa senza accordi
Ren\[B&]dete grazie a \[G-]Dio, Egli è bu\[C]ono,
e\[A7]terno e fedele è il suo a\[D-]more.
\[B&]Sì, è così: lo \[F]dica Israele,
\[G-]dica che il suo a\[F]more è per \[E&]sempre. \[C]
\endverse





%%%%% STROFA
\beginverse		%Oppure \beginverse* se non si vuole il numero di fianco
%\memorize 		% <<< DECOMMENTA se si vuole utilizzarne la funzione
\chordsoff		% <<< DECOMMENTA se vuoi una strofa senza accordi
La destra del Signore si è innalzata
a compiere grandiose meraviglie.
Non morirò, ma resterò in vita
e annuncerò i prodigi del Signore.
\endverse




%%%%% STROFA
\beginverse		%Oppure \beginverse* se non si vuole il numero di fianco
%\memorize 		% <<< DECOMMENTA se si vuole utilizzarne la funzione
\chordsoff		% <<< DECOMMENTA se vuoi una strofa senza accordi
La pietra che avevano scartato
è divenuta pietra angolare.
Questo prodigio ha fatto il Signore,
una meraviglia ai nostri occhi.
\endverse




\endsong
%------------------------------------------------------------
%			FINE CANZONE
%------------------------------------------------------------



%-------------------------------------------------------------
%			INIZIO	CANZONE
%-------------------------------------------------------------


%titolo: 	Alleluia Signore Sei Venuto
%autore: 	Scaglianti
%tonalita: 	LA



%%%%%% TITOLO E IMPOSTAZONI
\beginsong{Alleluia Signore sei venuto}[by={L. Scaglianti}]
	% <<< MODIFICA TITOLO E AUTORE
\transpose{0} 						% <<< TRASPOSIZIONE #TONI (0 nullo)
\momenti{Acclamazione al Vangelo;}							% <<< INSERISCI MOMENTI	
% momenti vanno separati da ; e vanno scelti tra:
% Ingresso; Atto penitenziale; Acclamazione al Vangelo; Dopo il Vangelo; Offertorio; Comunione; Ringraziamento; Fine; Santi; Pasqua; Avvento; Natale; Quaresima; Canti Mariani; Battesimo; Prima Comunione; Cresima; Matrimonio; Meditazione;
\ifchorded
	%\textnote{Tonalità originale }	% <<< EV COMMENTI (tonalità originale/migliore)
\fi


%%%%%% INTRODUZIONE
\ifchorded
\vspace*{\versesep}
\musicnote{
\begin{minipage}{0.48\textwidth}
\textbf{Intro}
\hfill 
%( \eighthnote \, 80)   % <<  MODIFICA IL TEMPO
% Metronomo: \eighthnote (ottavo) \quarternote (quarto) \halfnote (due quarti)
\end{minipage}
} 	
\vspace*{-\versesep}
\beginverse*

\nolyrics

%---- Prima riga -----------------------------
\vspace*{-\versesep}
\[A] \[A]  \[D]	 \[A] % \[*D] per indicare le pennate, \rep{2} le ripetizioni

%---- Ogni riga successiva -------------------
%\vspace*{-\versesep}
%\[G] \[C]  \[D]	

%---- Ev Indicazioni -------------------------			
\textnote{\textit{(con il ritmo della prima riga)} }	

\endverse
\fi





\beginverse
Si\[A]gnore, sei ve\[C#-]nuto fra\[D]tello in mezzo a \[E]noi,
Si\[F#-]gnore, hai por\[D]tato a\[B7]more e liber\[E]tà.
Si\[F#-]gnore sei vis\[F#-]suto \[E]nella \[D]pover\[C#-]tà:
\[A]noi ti ringra\[D]ziamo, Ge\[E4/7]sù.
\endverse

\beginchorus
\textnote{\textbf{Rit.}}
\[A]Alle\[F#-]luia, \[D] allelu\[E]ia, \[A]alle\[F#-]luia, \[B7] allelu\[E]ia.
\[A]Alle\[F#-]luia, \[D] allelu\[E]ia, \[C#-]Alle\[D]lu\[*A]ia. \[*D]  \[*A] 
\endchorus

\beginverse
\chordsoff
Signore sei venuto, fratello nel dolore.
Signore, hai parlato del regno dell'amore.
Signore, hai donato la tua vita a noi.
Noi ti ringraziamo Gesù.
\endverse

\beginverse
\chordsoff
Sei qui con noi Signore, fratello in mezzo a noi.
Tu parli al nostro cuore d'amore e libertà
tu vuoi che ti cerchiamo nella povertà.
Noi ti ringraziamo Gesù.
\endverse
\endsong
%------------------------------------------------------------
%			FINE CANZONE
%------------------------------------------------------------




%-------------------------------------------------------------
%			INIZIO	CANZONE
%-------------------------------------------------------------


%titolo: 	Alleluia Venite a me
%autore: 	Orlandini, Herrera
%tonalita: 	Re



%%%%%% TITOLO E IMPOSTAZONI
\beginsong{Alleluia Venite a me}[by={G. Orlandini, J. Herrera}] 	% <<< MODIFICA TITOLO E AUTORE
\transpose{0} 						% <<< TRASPOSIZIONE #TONI (0 nullo)
\momenti{Acclamazione al Vangelo}							% <<< INSERISCI MOMENTI	
% momenti vanno separati da ; e vanno scelti tra:
% Ingresso; Atto penitenziale; Acclamazione al Vangelo; Dopo il Vangelo; Offertorio; Comunione; Ringraziamento; Fine; Santi; Pasqua; Avvento; Natale; Quaresima; Canti Mariani; Battesimo; Prima Comunione; Cresima; Matrimonio; Meditazione; Spezzare del pane;
\ifchorded
	%\textnote{Tonalità migliore }	% <<< EV COMMENTI (tonalità originale/migliore)
\fi


%%%%%% INTRODUZIONE
\ifchorded
\vspace*{\versesep}
\musicnote{
\begin{minipage}{0.48\textwidth}
\textbf{Intro}
\hfill 
%( \eighthnote \, 80)   % <<  MODIFICA IL TEMPO
% Metronomo: \eighthnote (ottavo) \quarternote (quarto) \halfnote (due quarti)
\end{minipage}
} 	
\vspace*{-\versesep}
\beginverse*

\nolyrics

%---- Prima riga -----------------------------
\vspace*{-\versesep}
\[D] \[G*] \[D]	 \rep{2} % \[*D] per indicare le pennate, \rep{2} le ripetizioni

%---- Ogni riga successiva -------------------
%\vspace*{-\versesep}
%\[G] \[C]  \[D]	

%---- Ev Indicazioni -------------------------			
%\textnote{\textit{(Oppure tutta la strofa)} }	

\endverse
\fi



%%%%% STROFA
\beginverse		%Oppure \beginverse* se non si vuole il numero di fianco
\memorize 		% <<< DECOMMENTA se si vuole utilizzarne la funzione
%\chordsoff		% <<< DECOMMENTA se vuoi una strofa senza accordi
Ve\[D]nite \[G*]a \[D]me: allel\[A]u\[D]ia!
Cre\[D]dete \[G*]in \[D]me: allel\[A]u\[D]ia!
Io \[G]sono la \[A]via, la \[F#-]veri\[B-]tà: 
allel\[D]u\[G]ia, alle\[A7]lu\[D]ia! \quad  \[G*] \[D] \[G*] \[D]
\endverse

\beginverse

Res^tate ^in ^me: alle^lu^ia! 
Vi^vete ^in ^me: alle^lu^ia!
Io ^sono la ^vita, ^la santi^tà, 
alle^lu^ia, alle^lu^ia. \quad  \[G*] \[D] \[G*] \[D]
\endverse

\beginverse

Can^tate ^con ^me: alle^lu^ia! 
Dan^zate ^con ^me: alle^lu^ia!
Io ^sono la ^gioia, ^la liber^tà: 
alle^lu^ia, alle^lu^ia. \quad  \[G*] \[D] \[G*] \[D]

\endverse


\endsong
%------------------------------------------------------------
%			FINE CANZONE
%------------------------------------------------------------






%******* END SONGS ENVIRONMENT ******
\setcounter{GlobalSongCounter}{\thesongnum}
\end{songs}


\songchapter{Gloria}
%...............................................................................
%
%  ██████╗ ██╗      ██████╗ ██████╗ ██╗ █████╗     
% ██╔════╝ ██║     ██╔═══██╗██╔══██╗██║██╔══██╗    
% ██║  ███╗██║     ██║   ██║██████╔╝██║███████║    
% ██║   ██║██║     ██║   ██║██╔══██╗██║██╔══██║    
% ╚██████╔╝███████╗╚██████╔╝██║  ██║██║██║  ██║    
%  ╚═════╝ ╚══════╝ ╚═════╝ ╚═╝  ╚═╝╚═╝╚═╝  ╚═╝
% Font ANSI Shadow                                                          
%...............................................................................
\begin{songs}{}
\songcolumns{\canzsongcolumsnumber}
\setcounter{songnum}{\theGlobalSongCounter} %set songnum counter, otherwise would be reset

%set the default path inside current folder
\makeatletter
\def\input@path{{Songs/Gloria/}}
\makeatother


%***** INSERT SONGS HERE ******


%-------------------------------------------------------------
%			INIZIO	CANZONE
%-------------------------------------------------------------


%titolo: 	Gloria a te Cristo Gesù
%autore: 	Lecot
%tonalita: 	Sol 



%%%%%% TITOLO E IMPOSTAZONI
\beginsong{Gloria a te Cristo Gesù}[by={Inno del Giubileo 2000 — J. P. Lecot}]	% <<< MODIFICA TITOLO E AUTORE
\transpose{0} 						% <<< TRASPOSIZIONE #TONI (0 nullo)
\momenti{Gloria}							% <<< INSERISCI MOMENTI	
% momenti vanno separati da ; e vanno scelti tra:
% Ingresso; Atto penite0nziale; Acclamazione al Vangelo; Dopo il Vangelo; Offertorio; Comunione; Ringraziamento; Fine; Santi; Pasqua; Avvento; Natale; Quaresima; Canti Mariani; Battesimo; Prima Comunione; Cresima; Matrimonio; Meditazione; Spezzare del pane;
\ifchorded
	%\textnote{Tonalità migliore }	% <<< EV COMMENTI (tonalità originale/migliore)
\fi

%%%%%% INTRODUZIONE
\ifchorded
\vspace*{\versesep}
\musicnote{
\begin{minipage}{0.48\textwidth}
\textbf{Intro}
\hfill 
%( \eighthnote \, 80)   % <<  MODIFICA IL TEMPO
% Metronomo: \eighthnote (ottavo) \quarternote (quarto) \halfnote (due quarti)
\end{minipage}
} 	
\vspace*{-\versesep}
\beginverse*

\nolyrics

%---- Prima riga -----------------------------
\vspace*{-\versesep}
\[G] \[C*] \[G]	 \rep{2} % \[*D] per indicare le pennate, \rep{2} le ripetizioni

%---- Ogni riga successiva -------------------
%\vspace*{-\versesep}
%\[G] \[C]  \[D]	

%---- Ev Indicazioni -------------------------			
%\textnote{\textit{(Oppure tutta la strofa)} }	

\endverse
\fi



%%%%% RITORNELLO
\beginchorus
\textnote{\textbf{Rit.}}
\[G]Glo\[C*]ria a \[G]te, Cristo \[C*]Ge\[B-]sù, \brk \[E-]og\[D*]gi e \[G]sempre tu \[A-]regne\[B]rai.
\[D]Gloria a \[C]te! \[G]Presto ver\[D]rai: \brk \[E-]sei spe\[C]ran\[C]za \[A-]so\[D*]lo \[G]tu.
\endchorus


%%%%% STROFA
\beginverse		%Oppure \beginverse* se non si vuole il numero di fianco
\memorize 		% <<< DECOMMENTA se si vuole utilizzarne la funzione
%\chordsoff		% <<< DECOMMENTA se vuoi una strofa senza accordi
\[G]Sia lode a \[D]te! \[E-]Cristo Si\[B-]gnore, \brk \[C]offri per\[F]dono, chiedi giu\[C]stizia:
l'anno di \[A-]grazia \[F]apre le \[E-]porte. \brk \[A-]Solo in \[G]te \[D]pace e uni\[B-]tà! 
\[G]Amen! Al\[C]le\[G*]lu\[D]ia!
\endverse


%%%%% STROFA
\beginverse		%Oppure \beginverse* se non si vuole il numero di fianco
%\memorize 		% <<< DECOMMENTA se si vuole utilizzarne la funzione
%\chordsoff		% <<< DECOMMENTA se vuoi una strofa senza accordi
^Sia lode a ^te! ^Prega con ^noi \brk ^la bene^detta Vergine ^Madre:
tu l'esau^disci, ^tu la co^roni. \brk ^Solo in ^te ^pace e uni^tà 
^Amen! Al^le^lu^ia!
\endverse


%%%%% STROFA
\beginverse		%Oppure \beginverse* se non si vuole il numero di fianco
%\memorize 		% <<< DECOMMENTA se si vuole utilizzarne la funzione
%\chordsoff		% <<< DECOMMENTA se vuoi una strofa senza accordi
^Sia lode a ^te! ^Tutta la ^Chiesa \brk ^celebra il ^Padre con la tua ^voce
e nello ^Spirito ^canta di ^gioia. \brk ^Solo in ^te ^pace e uni^tà 
^Amen! Al^le^lu^ia!
\endverse




\endsong
%------------------------------------------------------------
%			FINE CANZONE
%------------------------------------------------------------



%-------------------------------------------------------------
%			INIZIO	CANZONE
%-------------------------------------------------------------


%titolo: 	Gloria (esme)
%autore: 	Gen Verde
%tonalita: 	FA e RE 



%%%%%% TITOLO E IMPOSTAZONI
\beginsong{Gloria nell'alto dei cieli}[by={Gen Verde, Esme}] 	% <<< MODIFICA TITOLO E AUTORE
\transpose{0} 						% <<< TRASPOSIZIONE #TONI (0 nullo)
\momenti{}							% <<< INSERISCI MOMENTI	
% momenti vanno separati da ; e vanno scelti tra:
% Ingresso; Atto penitenziale; Acclamazione al Vangelo; Dopo il Vangelo; Offertorio; Comunione; Ringraziamento; Fine; Santi; Pasqua; Avvento; Natale; Quaresima; Canti Mariani; Battesimo; Prima Comunione; Cresima; Matrimonio; Meditazione;
\ifchorded
	%\textnote{Tonalità originale }	% <<< EV COMMENTI (tonalità originale/migliore)
\fi




%%%%%% INTRODUZIONE
\ifchorded
\vspace*{\versesep}
\textnote{Intro: \qquad \qquad  }%(\eighthnote 116) % << MODIFICA IL TEMPO
% Metronomo: \eighthnote (ottavo) \quarternote (quarto) \halfnote (due quarti)
\vspace*{-\versesep}
\beginverse*

\nolyrics

%---- Prima riga -----------------------------
\vspace*{-\versesep}
\[F]   \[B&]  \[C]  \[C]	 \rep{2} % \[*D] per indicare le pennate, \rep{2} le ripetizioni

%---- Ogni riga successiva -------------------
%\vspace*{-\versesep}
%\[G] \[C]  \[D]	

%---- Ev Indicazioni -------------------------			
%\textnote{\textit{(Oppure tutta la strofa)} }	

\endverse
\fi





%%%%% RITORNELLO
\beginchorus
%\textnote{\textbf{Rit.}}

\[F]Gloria, \[B&]gloria a \[D-]Dio. \[C]
Gloria, \[F]gloria nel\[B&]l'alto dei \[D-]cie\[C]li.
\[F]Pace in \[B&]terra agli \[D-]uomi\[C]ni
di \[F]buona \[B&]volon\[F]tà. \[B&] 
\[F]Gl\[B&]o\[F]ria!
\endchorus







%%%%% STROFA
\beginverse*		%Oppure \beginverse* se non si vuole il numero di fianco
\memorize 		% <<< DECOMMENTA se si vuole utilizzarne la funzione
%\chordsoff		& <<< DECOMMENTA se vuoi una strofa senza accordi
Noi \[B&]ti lo\[F]diamo, \[G-]ti benedi\[F]ciamo,
ti \[B&]ador\[F]iamo, glo\[E&]rifichiamo \[F]te,
\[B&]ti ren\[F]diamo \[G-7]grazie per la \[F]tua immensa
\[E&]glor\[C4]ia. \[C]

\endverse


%%%%% STROFA
\beginverse*
Si^gnore ^Dio, ^glor^ia!  ^Re del ci^elo, ^glor^ia!
^Dio ^Padre, ^Dio onnipo^tente, ^glor\[C]ia! \[G-] \[E&] \[C]
\endverse



%%%%% RITORNELLO
\beginchorus
%\textnote{\textbf{Rit.}}

\[F]Gloria, \[B&]gloria a \[D-]Dio. \[C]
Gloria, \[F]gloria nel\[B&]l'alto dei \[D-]cie\[C]li.
\[F]Pace in \[B&]terra agli \[D-]uomi\[C]ni
di \[F]buona \[B&]volon\[F]tà. \[B&] 
\[F]Gl\[B&]o\[F]ria!
\endchorus



%%%%% STROFA
\beginverse*
Si\[F]gnore, Figlio uni\[E&]genito, \[B&]Gesù Cri\[F]sto,
Si\[F]gnore, Agnello di \[E&]Dio, \[B&]Figlio del Pad\[F]re.
\[F]Tu che togli i pec\[E&]cati del mondo,
a\[B&]bbi pietà  di no\[F]i;
\[F]tu che togli i pec\[E&]cati del mondo,
a\[B&]ccogli la nostra su\[F]pplica;
\[F]tu che siedi alla \[E&]destra del Padre,
\[B&]abbi pietà  di n\[C4]oi. \[C]
\endverse




%%%%% RITORNELLO
\beginchorus
%\textnote{\textbf{Rit.}}

\[F]Gloria, \[B&]gloria a \[D-]Dio. \[C]
Gloria, \[F]gloria nel\[B&]l'alto dei \[D-]cie\[C]li.
\[F]Pace in \[B&]terra agli \[D-]uomi\[C]ni
di \[F]buona \[B&]volon\[F]tà. \[B&] 
\[F]Gl\[B&]o\[F]ria!
\endchorus


%%%%% STROFA
\beginverse*
Per^chè tu ^solo il ^Santo, il Si^gnore,
tu ^solo l'Al^tissimo, ^Cristo G^esù
^con lo ^Spirito ^Santo nella ^gloria
del ^Pad\[C]re. \[G-] \[E&] \[C]
\endverse



%%%%% RITORNELLO
\beginchorus
%\textnote{\textbf{Rit.}}

\[F]Gloria, \[B&]gloria a \[D-]Dio. \[C]
Gloria, \[F]gloria nel\[B&]l'alto dei \[D-]cie\[C]li.
\[F]Pace in \[B&]terra agli \[D-]uomi\[C]ni
di \[F]buona \[B&]volon\[F]tà. \[B&] 
\[F]Gl\[B&]o\[F]ria! \[F] \[*F]
\endchorus




\endsong
%------------------------------------------------------------
%			FINE CANZONE
%------------------------------------------------------------
%++++++++++++++++++++++++++++++++++++++++++++++++++++++++++++
%			CANZONE TRASPOSTA
%++++++++++++++++++++++++++++++++++++++++++++++++++++++++++++
\ifchorded
%decremento contatore per avere stesso numero
\addtocounter{songnum}{-1} 
\beginsong{Gloria nell'alto dei cieli}[by={Gen Verde, Esme}] 	% <<< COPIA TITOLO E AUTORE
\transpose{-3} 						% <<< TRASPOSIZIONE #TONI + - (0 nullo)
\ifchorded
	\textnote{Tonalità migliore per le chitarre}	% <<< EV COMMENTI (tonalità originale/migliore)
\fi




%%%%%% INTRODUZIONE
\ifchorded
\vspace*{\versesep}
\textnote{Intro: \qquad \qquad  }%(\eighthnote 116) % << MODIFICA IL TEMPO
% Metronomo: \eighthnote (ottavo) \quarternote (quarto) \halfnote (due quarti)
\vspace*{-\versesep}
\beginverse*

\nolyrics

%---- Prima riga -----------------------------
\vspace*{-\versesep}
\[F]   \[B&]  \[C]  \[C]	 \rep{2} % \[*D] per indicare le pennate, \rep{2} le ripetizioni

%---- Ogni riga successiva -------------------
%\vspace*{-\versesep}
%\[G] \[C]  \[D]	

%---- Ev Indicazioni -------------------------			
%\textnote{\textit{(Oppure tutta la strofa)} }	

\endverse
\fi





%%%%% RITORNELLO
\beginchorus
%\textnote{\textbf{Rit.}}

\[F]Gloria, \[B&]gloria a \[D-]Dio. \[C]
Gloria, \[F]gloria nel\[B&]l'alto dei \[D-]cie\[C]li.
\[F]Pace in \[B&]terra agli \[D-]uomi\[C]ni
di \[F]buona \[B&]volon\[F]tà. \[B&] 
\[F]Gl\[B&]o\[F]ria!
\endchorus







%%%%% STROFA
\beginverse*		%Oppure \beginverse* se non si vuole il numero di fianco
\memorize 		% <<< DECOMMENTA se si vuole utilizzarne la funzione
%\chordsoff		& <<< DECOMMENTA se vuoi una strofa senza accordi
Noi \[B&]ti lo\[F]diamo, \[G-]ti benedi\[F]ciamo,
ti \[B&]ador\[F]iamo, glo\[E&]rifichiamo \[F]te,
\[B&]ti ren\[F]diamo \[G-7]grazie per la \[F]tua immensa
\[E&]glor\[C4]ia. \[C]

\endverse


%%%%% STROFA
\beginverse*
Si^gnore ^Dio, ^glor^ia!  ^Re del ci^elo, ^glor^ia!
^Dio ^Padre, ^Dio onnipo^tente, ^glor\[C]ia! \[G-] \[E&] \[C]
\endverse



%%%%% RITORNELLO
\beginchorus
%\textnote{\textbf{Rit.}}

\[F]Gloria, \[B&]gloria a \[D-]Dio. \[C]
Gloria, \[F]gloria nel\[B&]l'alto dei \[D-]cie\[C]li.
\[F]Pace in \[B&]terra agli \[D-]uomi\[C]ni
di \[F]buona \[B&]volon\[F]tà. \[B&] 
\[F]Gl\[B&]o\[F]ria!
\endchorus



%%%%% STROFA
\beginverse*
Si\[F]gnore, Figlio uni\[E&]genito, \[B&]Gesù Cri\[F]sto,
Si\[F]gnore, Agnello di \[E&]Dio, \[B&]Figlio del Pad\[F]re.
\[F]Tu che togli i pec\[E&]cati del mondo,
a\[B&]bbi pietà  di no\[F]i;
\[F]tu che togli i pec\[E&]cati del mondo,
a\[B&]ccogli la nostra su\[F]pplica;
\[F]tu che siedi alla \[E&]destra del Padre,
\[B&]abbi pietà  di n\[C4]oi. \[C]
\endverse




%%%%% RITORNELLO
\beginchorus
%\textnote{\textbf{Rit.}}

\[F]Gloria, \[B&]gloria a \[D-]Dio. \[C]
Gloria, \[F]gloria nel\[B&]l'alto dei \[D-]cie\[C]li.
\[F]Pace in \[B&]terra agli \[D-]uomi\[C]ni
di \[F]buona \[B&]volon\[F]tà. \[B&] 
\[F]Gl\[B&]o\[F]ria!
\endchorus


%%%%% STROFA
\beginverse*
Per^chè tu ^solo il ^Santo, il Si^gnore,
tu ^solo l'Al^tissimo, ^Cristo G^esù
^con lo ^Spirito ^Santo nella ^gloria
del ^Pad\[C]re. \[G-] \[E&] \[C]
\endverse



%%%%% RITORNELLO
\beginchorus
%\textnote{\textbf{Rit.}}

\[F]Gloria, \[B&]gloria a \[D-]Dio. \[C]
Gloria, \[F]gloria nel\[B&]l'alto dei \[D-]cie\[C]li.
\[F]Pace in \[B&]terra agli \[D-]uomi\[C]ni
di \[F]buona \[B&]volon\[F]tà. \[B&] 
\[F]Gl\[B&]o\[F]ria! \[F] \[*F]
\endchorus




\endsong

\fi
%++++++++++++++++++++++++++++++++++++++++++++++++++++++++++++
%			FINE CANZONE TRASPOSTA
%++++++++++++++++++++++++++++++++++++++++++++++++++++++++++++


%-------------------------------------------------------------
%			INIZIO	CANZONE
%-------------------------------------------------------------


%titolo: 	Gloria
%autore: 	M. Giombini
%tonalita: 	Do



%%%%%% TITOLO E IMPOSTAZONI
\beginsong{Gloria Giombini}[by={M. Giombini}] 	% <<< MODIFICA TITOLO E AUTORE
\transpose{0} 						% <<< TRASPOSIZIONE #TONI (0 nullo)
\momenti{Gloria}							% <<< INSERISCI MOMENTI	
% momenti vanno separati da ; e vanno scelti tra:
% Ingresso; Atto penitenziale; Acclamazione al Vangelo; Dopo il Vangelo; Offertorio; Comunione; Ringraziamento; Fine; Santi; Pasqua; Avvento; Natale; Quaresima; Canti Mariani; Battesimo; Prima Comunione; Cresima; Matrimonio; Meditazione; Spezzare del pane;
\ifchorded
	%\textnote{Tonalità migliore }	% <<< EV COMMENTI (tonalità originale/migliore)
\fi

%%%%%% INTRODUZIONE
\ifchorded
\vspace*{\versesep}
\musicnote{
\begin{minipage}{0.48\textwidth}
\textbf{Intro}
\hfill 
%( \eighthnote \, 80)   % <<  MODIFICA IL TEMPO
% Metronomo: \eighthnote (ottavo) \quarternote (quarto) \halfnote (due quarti)
\end{minipage}
} 	
\vspace*{-\versesep}
\beginverse*


\nolyrics

%---- Prima riga -----------------------------
\vspace*{-\versesep}
\[C*] \[G]  \[A-] \[F] \quad \[C] \[G] \[C] \quad \[F*] \[G*]



%---- Ev Indicazioni -------------------------			
\textnote{\textit{(come la seconda parte della prima strofa)} }	

\endverse
\fi


\beginverse*		%Oppure \beginverse* se non si vuole il numero di fianco
\memorize 		% <<< DECOMMENTA se si vuole utilizzarne la funzione
%\chordsoff		% <<< DECOMMENTA se vuoi una strofa senza accordi
\[C]Glo-\[G]o-o-\[C]ria, \quad \[F*] \[G*]
\[C]Glo-\[G]o-o-\[C]ria \quad \[F*] \[G*]
\[C] a \[G]Dio nell'\[A-]alto dei \[F]Cieli, 
\[C]Glo-\[G]o-o-\[C]ria! \quad \[F*] \[G*]
\endverse

\beginverse*
^E ^pa-a-a-^ce, \quad ^ ^
^e ^pa-a-a-^ce \quad ^ ^
^ in ^terra agli ^uomi^ni
di ^buona ^volon^tà. \quad \[C7]
\endverse

\beginverse*
Noi \[F]ti lodiamo \echo{noi ti lodiamo}
\[C]ti benediciamo \echo{ti benediciamo}
ti \[F]adoriamo \echo{ti adoriamo}
\[G]ti glorifichiamo \echo{ti glorifichiamo}.
\endverse

\beginverse*
^Ti rendi^a-a-a-^mo  \quad ^ ^
^gra-^a-a-^zie  \quad ^ ^
^ per ^la tua ^Glori^a
im^me-^e-en^sa.  \quad \[E7]
\endverse

\beginverse*
Si\[A-]gnore Figlio Uni\[E-]genito
Gesù \[F]Cristo, Si\[G]gnore \[C]Dio \quad \[E7]
\[A-] Agnello \[G] di Dio,
\[F] Figlio del \[E7]Padre.
\endverse

\beginverse*
\[A-] Tu che togli i pec\[E-]cati, \echo{tu che togli i peccati}
\[A-] i peccati del \[E]mondo, \echo{i peccati del mondo}
\[F] abbi pie\[C]tà  di noi,
\[D7] abbi pie\[G]tà  di noi!
\[A-] Tu che togli i pecca\[E-]ti,
\[A-] i peccati del \[E]mondo,
\[F] accogli, \[C] accogli,
\[D7] la nostra \[G]supplica!
\endverse

\beginverse*
\[C] Tu che siedi alla \[G]destra \[A-]
\[(A-)] Alla destra del \[E-]Padre \[F]
\[(F)] Abbi pie\[C]tà  di noi, \[D7] abbi pie\[G]tà  di noi
\endverse 


%%%%%% INTERMEZZO
\beginverse*
\vspace*{1.3\versesep}
{
	\nolyrics
	\textnote{Intermezzo strumentale}
	
	\ifchorded

	%---- Prima riga -----------------------------
	\vspace*{-\versesep}
	\[C] \[G] \[C] \quad \[F*] \[G*]	
	%---- Ogni riga successiva -------------------
	\vspace*{-\versesep}
	\[C] \[G] \[C] \quad \[F*] \[G*]
	%---- Ogni riga successiva -------------------
	\vspace*{-\versesep}
	\[C] \[G]  \[A-] \[F] \quad \[C] \[G] \[C] \quad \[C7]
	
	\fi
	%---- Ev Indicazioni -------------------------			
	%\textnote{\textit{(ripetizione della strofa)}} 
	 
}
\vspace*{\versesep}
\endverse


\beginverse*
Per\[F]chè tu solo il Santo \echo{perchè tu solo il Santo}
tu \[C]solo il Signore \echo{tu solo il Signore}
tu \[F]solo l'Altissimo \echo{tu solo l'Altissimo}
\[G]Gesù Cristo \echo{Gesù Cristo}.
\endverse

\beginverse*
^Con lo ^Spiri^to  \quad ^ ^
^Sa-^a-an^to  \quad ^ ^
^ nella ^gloria ^di Dio ^Padre 
^A-^a-a^men \quad \[F*] \[G*]
\endverse

\beginverse*
^Con lo ^Spiri^to  \echo{con lo ^Spiri^to Santo} 
^Sa-^a-an^to   \echo{nella ^glo^ria di Dio Padre}
^ nella ^gloria ^di Dio ^Padre 
^A-^a-a^men! \echo{nella \[F*]glo\[G*]ria di Dio Padre} 
\endverse



\beginverse*
\[C] nella \[G]gloria \[A-]di Dio \[F]Padre 
\[C]A-\[G7]a-a\[C]men! \quad \[G] \quad \[C*]
\endverse

\endsong
%------------------------------------------------------------
%			FINE CANZONE
%------------------------------------------------------------








%******* END SONGS ENVIRONMENT ******
\setcounter{GlobalSongCounter}{\thesongnum}
\end{songs}


\songchapter{Liturgia}
%...............................................................................
%
% ██╗     ██╗████████╗██╗   ██╗██████╗  ██████╗ ██╗ █████╗ 
% ██║     ██║╚══██╔══╝██║   ██║██╔══██╗██╔════╝ ██║██╔══██╗
% ██║     ██║   ██║   ██║   ██║██████╔╝██║  ███╗██║███████║
% ██║     ██║   ██║   ██║   ██║██╔══██╗██║   ██║██║██╔══██║
% ███████╗██║   ██║   ╚██████╔╝██║  ██║╚██████╔╝██║██║  ██║
% ╚══════╝╚═╝   ╚═╝    ╚═════╝ ╚═╝  ╚═╝ ╚═════╝ ╚═╝╚═╝  ╚═╝
% Font: ANSI Shadow                                                                               
%...............................................................................
\begin{songs}{}
\songcolumns{\canzsongcolumsnumber}
\setcounter{songnum}{\theGlobalSongCounter} %set songnum counter, otherwise would be reset

%set the default path inside current folder
\makeatletter
\def\input@path{{Songs/Liturgia/}}
\makeatother


%***** INSERT SONGS HERE ******

%AAA
%-------------------------------------------------------------
%			INIZIO	CANZONE
%-------------------------------------------------------------


%titolo: 	Accendi la vita
%autore: 	Bertoglio, Testa
%tonalita: 	Sol 



%%%%%% TITOLO E IMPOSTAZONI
\beginsong{Accendi la vita}[by={Bertoglio, Testa}] 	% <<< MODIFICA TITOLO E AUTORE
\transpose{0} 						% <<< TRASPOSIZIONE #TONI (0 nullo)
\momenti{Ingresso; Cresima; Prima Comunione; Fine; Santi; Matrimonio;}							% <<< INSERISCI MOMENTI	
% momenti vanno separati da ; e vanno scelti tra:
% Ingresso; Atto penitenziale; Acclamazione al Vangelo; Dopo il Vangelo; Offertorio; Comunione; Ringraziamento; Fine; Santi; Pasqua; Avvento; Natale; Quaresima; Canti Mariani; Battesimo; Prima Comunione; Cresima; Matrimonio; Meditazione;
\ifchorded
	%\textnote{Tonalità originale }	% <<< EV COMMENTI (tonalità originale/migliore)
\fi


%%%%%% INTRODUZIONE
\ifchorded
\vspace*{\versesep}
\textnote{Intro: \qquad \qquad  }%(\eighthnote 116) % << MODIFICA IL TEMPO
% Metronomo: \eighthnote (ottavo) \quarternote (quarto) \halfnote (due quarti)
\vspace*{-\versesep}
\beginverse*

\nolyrics

%---- Prima riga -----------------------------
\vspace*{-\versesep}
\[D] \[A]  \[D]	 % \[*D] per indicare le pennate, \rep{2} le ripetizioni

%---- Ogni riga successiva -------------------
%\vspace*{-\versesep}
%\[G] \[C]  \[D]	

%---- Ev Indicazioni -------------------------			
%\textnote{\textit{(Oppure tutta la strofa)} }	

\endverse
\fi




%%%%% STROFA
\beginverse		%Oppure \beginverse* se non si vuole il numero di fianco
\memorize 		% <<< DECOMMENTA se si vuole utilizzarne la funzione
%\chordsoff		& <<< DECOMMENTA se vuoi una strofa senza accordi

\[D]Come il vento in\[A]frange il mare a \[B-]riva \[B-]
\[G]così il tempo \[D]agita il mio cu\[A]ore \[A]
\[D]cerca il senso \[A]della sua esis\[B-]tenza \[B-]
\[G]cerca il volto \[D]mite del Si\[A]gnore. \[A]
Ed ho cer\[G]cato \echo{ed ho cercato}
per ogni \[D]via \echo{per ogni via}
su ogni \[F#-]vetta i miei \[G]piedi
han cammi\[A]nato
e nel do\[G]lore \echo{e nel dolore} 
mi son pie\[D]gato \echo{mi son piegato}
nella fa\[F#-]tica il tuo \[G]nome io ne\[A]gai
ma \[B-]poi ...

\endverse



%%%%% STROFA
\beginverse		%Oppure \beginverse* se non si vuole il numero di fianco
%\memorize 		% <<< DECOMMENTA se si vuole utilizzarne la funzione
%\chordsoff		% <<< DECOMMENTA se vuoi una strofa senza accordi

^Ancora ho prepa^rato le mie ^cose ^
^pronto per un vi^aggio che ver^rà ^
^affidando al ^sonno della ^notte ^
^sogni di un in^contro che sa^rà ^
finché una ^voce \echo{finché una voce} 
mi ha des^tato \echo{mi ha destato}
finché il mio ^nome nel ^vento ha risuo^nato
è come un ^fuoco \echo{è come un fuoco}
che incendia il ^cuore \echo{che incendia il cuore}
un fuoco ^che caldo ^si libere^rà.

\endverse



%%%%% RITORNELLO
\beginchorus
\textnote{\textbf{Rit.}}

Accendi la \[D]vita che Dio ti \[G]dà
brucia d’a\[F#-]mo\[B-]re \[A] non perderti \[D]mai \[A]
accendi la \[D]vita perché ora \[G]sai
che il nostro \[F#-]viag\[B-]gio \[A] porta al Si\[D]gnor.

\endchorus



%%%%% STROFA
\beginverse		%Oppure \beginverse* se non si vuole il numero di fianco
%\memorize 		% <<< DECOMMENTA se si vuole utilizzarne la funzione
%\chordsoff		% <<< DECOMMENTA se vuoi una strofa senza accordi

^Come il sole ^dona il suo ca^lore ^
^tu o Signore ^doni veri^tà ^
^luce di una f^iamma senza ^fine ^
^alba di una ^nuova umani^tà. ^
Ed ho ascol^tato \echo{Ed ho ascoltato}
le tue pa^role \echo{le tue parole}
mi son nu^trito di ^nuovo del tuo a^more
ho aperto gli ^occhi \echo{ho aperto gli occhi}
alla mia ^gente \echo{alla mia gente}
con te vi^cino la ^vita esplode^rà.

\endverse


%%%%% RITORNELLO
\beginchorus
\textnote{\textbf{Rit.}}

Accendi la \[D]vita ... \quad \quad \rep{3}

\endchorus



\endsong
%------------------------------------------------------------
%			FINE CANZONE
%------------------------------------------------------------

%-------------------------------------------------------------
%			INIZIO	CANZONE
%-------------------------------------------------------------


%titolo: 	Acqua siamo noi
%autore: 	Cento
%tonalita: 	Re



%%%%%% TITOLO E IMPOSTAZONI
\beginsong{Acqua siamo noi}[by={Cento}] 	% <<< MODIFICA TITOLO E AUTORE
\transpose{0} 						% <<< TRASPOSIZIONE #TONI (0 nullo)
\momenti{Ingresso; Battesimo}							% <<< INSERISCI MOMENTI	
% momenti vanno separati da ; e vanno scelti tra:
% Ingresso; Atto penitenziale; Acclamazione al Vangelo; Dopo il Vangelo; Offertorio; Comunione; Ringraziamento; Fine; Santi; Pasqua; Avvento; Natale; Quaresima; Canti Mariani; Battesimo; Prima Comunione; Cresima; Matrimonio; Meditazione;
\ifchorded
	%\textnote{Tonalità originale }	% <<< EV COMMENTI (tonalità originale/migliore)
\fi


%%%%%% INTRODUZIONE
\ifchorded
\vspace*{\versesep}
\textnote{Intro: \qquad \qquad  }%(\eighthnote 116) % << MODIFICA IL TEMPO
% Metronomo: \eighthnote (ottavo) \quarternote (quarto) \halfnote (due quarti)
\vspace*{-\versesep}
\beginverse*

\nolyrics

%---- Prima riga -----------------------------
\vspace*{-\versesep}
\[D] \[A]  \[D]	\[G] \[A] \[D] % \[*D] per indicare le pennate, \rep{2} le ripetizioni

%---- Ogni riga successiva -------------------
%\vspace*{-\versesep}
%\[G] \[C]  \[D]	

%---- Ev Indicazioni -------------------------			
\textnote{\textit{(Come la prima riga)} }	

\endverse
\fi



%%%%% STROFA
\beginverse		%Oppure \beginverse* se non si vuole il numero di fianco
\memorize 		% <<< DECOMMENTA se si vuole utilizzarne la funzione
%\chordsoff		& <<< DECOMMENTA se vuoi una strofa senza accordi

|\[D]Acqua \[A]siamo \[D]noi, \brk dall'an|\[G]tica sor\[A]gente ve\[D]niamo,
\[D]fiumi \[A]siamo \[D]noi \brk se i ru\[G]scelli si \[A]mettono in\[D]sieme,
\[D]mari \[G]siamo \[D]noi \brk se i tor\[G]renti si \[A]danno la \[D]mano,
\[D]vita nuova \[B-7]c'è \brk se Ge\[G]sù è in \[A]mezzo a \[D]noi.
\endverse



%%%%% RITORNELLO
\textnote{\textbf{Rit.}}
\beginchorus

E allora \[F#-7]diamoci la \[B-7]mano
e tutti in\[F#-7]sieme cammi\[G7+]niamo
ed un o\[F#-7]ceano di \[B-7]pace nasce\[A4]rà. \[A]
E l'ego\[E-]ismo cancel\[C7+]liamo
e un cuore \[E-]limpido sen\[C]tiamo
è Dio che \[C7+]bagna del suo a\[A7]mor l'umani\[D]tà. \[A4]\[A]
\endchorus


\beginverse
%\chodsoff
^Su nel ^cielo ^c'è \brk Dio ^Padre che ^vive per ^l'uomo
^crea ^tutti ^noi \brk e ci ^ama di a^more infi^nito,
^figli ^siamo ^noi \brk e fra^telli di ^Cristo Si^gnore,
^vita nuova ^c'è \brk quando ^Lui è in ^mezzo a ^noi.
\endverse


\beginverse
%\chordsoff
^Nuova u^mani^tà \brk oggi ^nasce da ^chi crede in ^Lui,
^nuovi ^siamo ^noi \brk se l'a^more è la ^legge di ^vita,
^figli ^siamo ^noi \brk se non ^siamo di^visi da ^niente,
^vita eterna ^c'è \brk quando ^Lui è ^dentro ^noi.
\endverse
\endsong

%-------------------------------------------------------------
%			INIZIO	CANZONE
%-------------------------------------------------------------


%titolo: 	Amare questa vita
%autore: 	Meregalli
%tonalita: 	Re



%%%%%% TITOLO E IMPOSTAZONI
\beginsong{Amare questa vita}[ititle={Erano uomini senza paura}, by={G. Meregalli}] 	% <<< MODIFICA TITOLO E AUTORE
\transpose{-3} 						% <<< TRASPOSIZIONE #TONI (0 nullo)
\momenti{Comunione; Cresima}							% <<< INSERISCI MOMENTI	
% momenti vanno separati da ; e vanno scelti tra:
% Ingresso; Atto penitenziale; Acclamazione al Vangelo; Dopo il Vangelo; Offertorio; Comunione; Ringraziamento; Fine; Santi; Pasqua; Avvento; Natale; Quaresima; Canti Mariani; Battesimo; Prima Comunione; Cresima; Matrimonio; Meditazione;
\ifchorded
	%\textnote{Tonalità originale }	% <<< EV COMMENTI (tonalità originale/migliore)
\fi

%%%%%% INTRODUZIONE
\ifchorded
\vspace*{\versesep}
\musicnote{
\begin{minipage}{0.48\textwidth}
\textbf{Intro}
\hfill 
%( \eighthnote \, 80)   % <<  MODIFICA IL TEMPO
% Metronomo: \eighthnote (ottavo) \quarternote (quarto) \halfnote (due quarti)
\end{minipage}
} 	
\vspace*{-\versesep}
\beginverse*

\nolyrics

%---- Prima riga -----------------------------
\vspace*{-\versesep}
\[F] \[B&] \[C]	 % \[*D] per indicare le pennate, \rep{2} le ripetizioni

%---- Ogni riga successiva -------------------
%\vspace*{-\versesep}
%\[G] \[C]  \[D]	

%---- Ev Indicazioni -------------------------			
%\textnote{\textit{(Oppure tutta la strofa)} }	

\endverse
\fi

%%%%% STROFA
\beginverse		%Oppure \beginverse* se non si vuole il numero di fianco
\memorize 		% <<< DECOMMENTA se si vuole utilizzarne la funzione
%\chordsoff		& <<< DECOMMENTA se vuoi una strofa senza accordi

\[F]Erano \[B&]uomini \[C]senza pa\[7]ura,
di \[D-]solcare il \[7]mare pen\[A-]sando alla \[7]riva,
\[B&]barche sotto il \[C]cielo, \[F] tra montagne e si\[C]lenzio,
\[B&]davano le \[C]reti al \[F]ma\[D-]re, \brk \[B&]vita dalle \[G-]mani di \[C4]Dio \[C]

\endverse

%%%%% STROFA
\beginverse		%Oppure \beginverse* se non si vuole il numero di fianco
%\memorize 		% <<< DECOMMENTA se si vuole utilizzarne la funzione
%\chordsoff		& <<< DECOMMENTA se vuoi una strofa senza accordi

^Venne nell'^ora più ^lenta del ^giorno,
^quando le ^reti si ^sdraiano a ^riva,
^l'aria senza ^vento, ^ si riempì di una ^voce,
^mani cari^che di ^sa^le, ^sale nelle ^mani di ^Dio. ^

\endverse

%%%%% RITORNELLO
\beginchorus
\textnote{\textbf{Rit.}}

Lo se\[F]guimmo fi\[G-7]dandoci degli \[F]occhi \[7]
gli cre\[B&]demmo a\[A7]mando le pa\[D-]role. \[D7]
Fu il \[G-]sole caldo a \[C7]riva
o fu il \[F]vento sulla \[C]vela
o il \[B&]gusto e la fa\[F]tica di ri\[G-7]schiare
e accettare quella \[C]sfida.
\endchorus

\beginverse
^Prima che un ^sole più ^alto vi ^insidi, 
^prima che il ^giorno vi ^lasci de^lusi,
^riprendete il ^largo ^ e gettate le ^reti, 
^barche cari^che di ^pe^sci, \brk ^vita dalle ^mani di ^Dio. ^
\endverse
\beginchorus
Lo se\[F]guimmo fi\[G-7]dandoci degli \[F]occhi \[7]
gli cre\[B&]demmo a\[A7]mando le pa\[D-]role. \[D7]
Lui \[G-]voce lui no\[C7]tizia
lui \[F]strada e lui sua \[C]meta
lui \[B&]gioia impreve\[F]dibile e sin\[G-7]cera
di amare questa \[C]vita.
\endchorus
\beginverse
^Erano ^uomini ^senza pa^ura
di s^olcare il ^mare pen^sando alla ^riva,
^anche quella ^sera, ^ senza dire pa^role,
^misero le ^barche in ^ma^re, ^vita dalle ^mani di ^Dio, ^
\[B&]misero le \[C]barche in \[F]ma\[D-]re, \brk \[B&]vita dalle \[C]mani di \[F]Dio.
\endverse
\endsong


%-------------------------------------------------------------
%			INIZIO	CANZONE
%-------------------------------------------------------------


%titolo: 	Andate per le strade
%autore: 	Roncari, Capello
%tonalita: 	Si-



%%%%%% TITOLO E IMPOSTAZONI
\beginsong{Andate per le strade}[by={Roncari, Capello}] 	% <<< MODIFICA TITOLO E AUTORE
\transpose{0} 						% <<< TRASPOSIZIONE #TONI (0 nullo)
\momenti{Fine}							% <<< INSERISCI MOMENTI	
% momenti vanno separati da ; e vanno scelti tra:
% Ingresso; Atto penitenziale; Acclamazione al Vangelo; Dopo il Vangelo; Offertorio; Comunione; Ringraziamento; Fine; Santi; Pasqua; Avvento; Natale; Quaresima; Canti Mariani; Battesimo; Prima Comunione; Cresima; Matrimonio; Meditazione;
\ifchorded
	%\textnote{Tonalità originale }	% <<< EV COMMENTI (tonalità originale/migliore)
\fi


%%%%%% INTRODUZIONE
\ifchorded
\vspace*{\versesep}
\textnote{Intro: \qquad \qquad  }%(\eighthnote 116) % << MODIFICA IL TEMPO
% Metronomo: \eighthnote (ottavo) \quarternote (quarto) \halfnote (due quarti)
\vspace*{-\versesep}
\beginverse*

\nolyrics

%---- Prima riga -----------------------------
\vspace*{-\versesep}
\[B-] \[A] \[*B-]	 % \[*D] per indicare le pennate, \rep{2} le ripetizioni

%---- Ogni riga successiva -------------------
%\vspace*{-\versesep}
%\[G] \[C]  \[D]	

%---- Ev Indicazioni -------------------------			
%\textnote{\textit{(Oppure tutta la strofa)} }

\endverse
\fi

%%%%% RITORNELLO
\beginchorus
\textnote{\textbf{Rit.}}

An\[(B-)]date per le \[D]strade in \[G]tutto il \[A]mondo,
chia\[F#]mate i miei a\[B-]mici \[A]per far \[D]festa,
c'è un \[F#]posto per cia\[G]scuno \[A]alla mia \[B-]mensa.  \[*B-]

\endchorus

%%%%% STROFA
\beginverse		%Oppure \beginverse* se non si vuole il numero di fianco
\memorize 		% <<< DECOMMENTA se si vuole utilizzarne la funzione
%\chordsoff		& <<< DECOMMENTA se vuoi una strofa senza accordi

Nel \[(D)]vostro cam\[G]mino annun\[A]ciate il Van\[D]gelo 
di\[B-]cendo “È vi\[E-]cino il \[F#]Regno dei \[B-]cieli!”.
Gua\[D7]rite i ma\[G]lati, mon\[A]date i leb\[D]brosi,
ren\[B-]dete la \[F#-]vita a \[C#]chi l'ha per\[F#]duta. \[F#] \quad \[*B-] 

\endverse

\beginverse
%\chordsoff
^Vi è stato do^nato con a^more gra^tuito: 
u^gualmente do^nate con ^gioia e per a^more. 
Con ^voi non pren^dete ^né oro né ar^gento, 
per^ché l'ope^raio ha di^ritto al suo ^cibo. ^ \quad ^
\endverse
\beginverse
%\chordsoff
En^trando in una ^casa do^natele la ^pace: 
se ^c'è chi vi ri^fiuta e ^non accoglie il ^dono, 
la ^pace torni a ^voi e u^scite dalla ^casa, 
^scuotendo la ^polvere dai ^vostri cal^zari.  ^ \quad ^
\endverse
\beginverse
%\chordsoff
^Ecco, io vi ^mando a^gnelli in mezzo ai ^lupi: 
siate ^dunque avve^duti come ^sono i ser^penti, 
ma ^liberi e ^chiari co^me le co^lombe; 
do^vrete soppor^tare pri^gioni e tribu^nali. ^ \quad ^
\endverse
\beginverse
%\chordsoff
Nes^suno è più ^grande del ^proprio ma^estro, 
né il ^servo è più im^portante ^del suo pa^drone. 
Se ^hanno odiato ^me, odie^ranno anche ^voi; 
ma ^voi non te^mete, io non ^vi lascio ^soli! ^ \quad ^
\endverse

\endsong
%------------------------------------------------------------
%			FINE CANZONE
%------------------------------------------------------------
%-------------------------------------------------------------
%			INIZIO	CANZONE
%-------------------------------------------------------------


%titolo: 	Antica eterna danza
%autore: 	Gen Verde
%tonalita: 	Sol 



%%%%%% TITOLO E IMPOSTAZONI
\beginsong{Antica eterna danza}[by={Gen Verde}] 	% <<< MODIFICA TITOLO E AUTORE
\transpose{0} 						% <<< TRASPOSIZIONE #TONI (0 nullo)
\momenti{Offertorio; Prima Comunione}							% <<< INSERISCI MOMENTI	
% momenti vanno separati da ; e vanno scelti tra:
% Ingresso; Atto penitenziale; Acclamazione al Vangelo; Dopo il Vangelo; Offertorio; Comunione; Ringraziamento; Fine; Santi; Pasqua; Avvento; Natale; Quaresima; Canti Mariani; Battesimo; Prima Comunione; Cresima; Matrimonio; Meditazione;
\ifchorded
	%\textnote{Tonalità originale }	% <<< EV COMMENTI (tonalità originale/migliore)
\fi





%%%%%% INTRODUZIONE
\ifchorded
\vspace*{\versesep}
\textnote{Intro: \qquad \qquad  }%(\eighthnote 116) % << MODIFICA IL TEMPO
% Metronomo: \eighthnote (ottavo) \quarternote (quarto) \halfnote (due quarti)
\vspace*{-\versesep}
\beginverse*

\nolyrics

%---- Prima riga -----------------------------
\vspace*{-\versesep}
\[*G] \[*D]  \[*C]	 % \[*D] per indicare le pennate, \rep{2} le ripetizioni

%---- Ogni riga successiva -------------------
\vspace*{-\versesep}
\[G] \[C]  \[G]	

%---- Ev Indicazioni -------------------------			
%\textnote{\textit{(Oppure tutta la strofa)} }	

\endverse
\fi



%%%%% STROFA
\beginverse
\memorize

\[G]Spighe \[D]d'oro al \[E-]vento, 
an\[E-]tica, e\[D]terna \[C]danza
per \[A-]fare un \[D]solo \[E-]pa\[D]ne
spez\[C]zato \[A-]sulla m\[B]ensa.
\[G]Grappoli \[D]dei \[E-]colli, 
pro\[E-]fumo \[D]di le\[C]tizia
per \[A-]fare un \[D]solo \[E-]vi\[D]no 
be\[A-]vanda \[B-]della \[E-]grazia.  

\endverse



%%%%%% EV. INTERMEZZO
\beginverse*
\vspace*{1.3\versesep}
{
	\nolyrics
	
	\ifchorded
	\textnote{Intermezzo strumentale}

		
	%---- Prima riga -----------------------------
	\vspace*{-\versesep}
	\[*G] \[*D]  \[*C]	 % \[*D] per indicare le pennate, \rep{2} le ripetizioni

	%---- Ogni riga successiva -------------------
	\vspace*{-\versesep}
	\[G] \[C]  \[G] 


	\fi
	%---- Ev Indicazioni -------------------------			
	%\textnote{\textit{(ripetizione della strofa)}} 
	 
}
\vspace*{\versesep}
\endverse




%%%%% STROFA
\beginverse

^Con il pa^ne e il vi^no 
Si^gnore ^ti doni^amo
le ^nostre gi^oie ^pu^re, 
le at^tese e ^le pa^ure.
^Frutti ^del la^voro 
e ^fede n^el fu^turo,
la ^voglia ^di cam^bia^re 
e ^di ri^cominci^are.  

\endverse



%%%%%% EV. INTERMEZZO
\beginverse*
\vspace*{1.3\versesep}
{
	\nolyrics
	
	\ifchorded
	\textnote{Intermezzo strumentale}

		
	%---- Prima riga -----------------------------
	\vspace*{-\versesep}
	\[*G] \[*D]  \[*C]	 % \[*D] per indicare le pennate, \rep{2} le ripetizioni

	%---- Ogni riga successiva -------------------
	\vspace*{-\versesep}
	\[G] \[C]  \[G]	


	\fi
	%---- Ev Indicazioni -------------------------			
	%\textnote{\textit{(ripetizione della strofa)}} 
	 
}
\vspace*{\versesep}
\endverse




%%%%% STROFA
\beginverse

^Dio del^la spe^ranza, 
sor^gente ^d'ogni ^dono
ac^cogli q^uesta o^ffer^ta 
che in^sieme ^Ti porti^amo.
^Dio dell'^uni^verso 
rac^cogli ^chi è dis^perso
 

\vspace*{1.3\versesep}
\textnote{\textit{(rallentando)}}
e ^facci ^tutti ^Chie^sa,
u^na ^cosa in ^Te.
\endverse
\endsong



%-------------------------------------------------------------
%			INIZIO	CANZONE
%-------------------------------------------------------------


%titolo: 	Ave Maria
%autore: 	Casucci, Balduzzi
%tonalita: 	Re 


%%%%%% TITOLO E IMPOSTAZONI
\beginsong{Ave Maria}[by={Casucci, Balduzzi}] 	% <<< MODIFICA TITOLO E AUTORE
\transpose{0} 						% <<< TRASPOSIZIONE #TONI (0 nullo)
\momenti{Canti Mariani; Ringraziamento}							% <<< INSERISCI MOMENTI	
% momenti vanno separati da ; e vanno scelti tra:
% Ingresso; Atto penitenziale; Acclamazione al Vangelo; Dopo il Vangelo; Offertorio; Comunione; Ringraziamento; Fine; Santi; Pasqua; Avvento; Natale; Quaresima; Canti Mariani; Battesimo; Prima Comunione; Cresima; Matrimonio; Meditazione;
\ifchorded
	%\textnote{Tonalità originale }	% <<< EV COMMENTI (tonalità originale/migliore)
\fi



%%%%%% INTRODUZIONE
\ifchorded
\vspace*{\versesep}
\textnote{Intro: \qquad \qquad  (\quarternote  72)}%(\eighthnote 116) % << MODIFICA IL TEMPO
% Metronomo: \eighthnote (ottavo) \quarternote (quarto) \halfnote (due quarti)
\vspace*{-\versesep}
\beginverse*

\nolyrics

%---- Prima riga -----------------------------
\vspace*{-\versesep}
\[D]\[A]\[B-]\[G]	 % \[*D] per indicare le pennate, \rep{2} le ripetizioni

%---- Ogni riga successiva -------------------
\vspace*{-\versesep}
\[D]\[A]\[E-] \[G]

%---- Ev Indicazioni -------------------------			
\textnote{\textit{(Come mezzo ritornello)} }	

\endverse
\fi




%%%%% RITORNELLO
\textnote{\textbf{Rit.}}
\beginchorus

\[D]A\[A]ve Ma\[B-]ria, \[G] 
\[D]\[A]a\[E-]ve, \[G]
\[D]a\[A]ve Ma\[B-]ria, \[G] 
\[D]\[A]a\[D4]ve. \[D]

\endchorus



%%%%% STROFA
\beginverse
\memorize
\[D]Donna dell'at\[D]tesa e \[B-]madre di spe\[B-]ranza
\[A]ora pro no\[G]bis.
\[D]Donna del sor\[D]riso e \[B-]madre del si\[B-]lenzio
\[A]ora pro no\[G]bis.
\[D]Donna di fron\[D]tiera e \[A]madre dell'ar\[A]dore
\[B-]ora pro no\[G]bis.
\[D]Donna del ri\[D]poso e \[A]madre del sen\[A]tiero
\[G]ora pro no\[A]bis.
\endverse




%%%%% STROFA
\beginverse
^Donna del de^serto e ^madre del re^spiro
^ora pro no^bis.
^Donna della ^sera e ^madre del ri^cordo
^ora pro no^bis.
^Donna del pre^sente e ^madre del ri^torno
^ora pro no^bis.
^Donna della ^terra e ^madre dell'a^more
^ora pro no^bis.
\endverse


\endsong
%------------------------------------------------------------
%			FINE CANZONE
%------------------------------------------------------------




%BBB
%-------------------------------------------------------------
%			INIZIO	CANZONE
%-------------------------------------------------------------


%titolo: 	Beato il cuore GMG 2016
%autore: 	Casucci, Balduzzi
%tonalita: 	Fa 



%%%%%% TITOLO E IMPOSTAZONI
\beginsong{Beato il cuore (Inno GMG 2016)}[by={Jakub Blycharz, GMG Cracovia 2016}]
\transpose{-2} 						% <<< TRASPOSIZIONE #TONI (0 nullo)
\momenti{Comunione; Fine; Ingresso}							% <<< INSERISCI MOMENTI	
% momenti vanno separati da ; e vanno scelti tra:
% Ingresso; Atto penitenziale; Acclamazione al Vangelo; Dopo il Vangelo; Offertorio; Comunione; Ringraziamento; Fine; Santi; Pasqua; Avvento; Natale; Quaresima; Canti Mariani; Battesimo; Prima Comunione; Cresima; Matrimonio; Meditazione;
\ifchorded
	\textnote{Tonalità migliore}	% <<< EV COMMENTI (tonalità originale/migliore)
\fi


%%%%%% INTRODUZIONE
\ifchorded
\vspace*{\versesep}
\textnote{Intro: \qquad \qquad  }%(\eighthnote 116) % << MODIFICA IL TEMPO
% Metronomo: \eighthnote (ottavo) \quarternote (quarto) \halfnote (due quarti)
\vspace*{-\versesep}
\beginverse*

\nolyrics

%---- Prima riga -----------------------------
\vspace*{-\versesep}
\[C#-]  \[A]	\[E]  % \[*D] per indicare le pennate, \rep{2} le ripetizioni

%---- Ogni riga successiva -------------------
\vspace*{-\versesep}
\[B] \[(B)] \[(B)]  \[C#-]

%---- Ev Indicazioni -------------------------			
%\textnote{\textit{(Oppure tutta la strofa)} }	

\endverse
\fi




%%%%% STROFA
\beginverse		%Oppure \beginverse* se non si vuole il numero di fianco
\memorize 		% <<< DECOMMENTA se si vuole utilizzarne la funzione
%\chordsoff		& <<< DECOMMENTA se vuoi una strofa senza accordi

\[C#-]Sei sceso \[A]dalla tua immensi\[E]tà
\[D]in nostro a\[A]iu\[E]to.
Miseri\[B]cordia  scorre  da \[F#]te
\[A]sopra \[B]tutti \[C#]noi.


\endverse


%%%%% STROFA
\beginverse*	%Oppure \beginverse* se non si vuole il numero di fianco
%\memorize 		% <<< DECOMMENTA se si vuole utilizzarne la funzione
%\chordsoff		& <<< DECOMMENTA se vuoi una strofa senza accordi

^ Persi in un ^mondo d’oscuri^tà
^lì Tu ci ^tro^vi.
Nelle tue ^braccia ci stringi e ^poi
^dai la ^vita per ^noi.


\endverse



%%%%% RITORNELLO
\beginchorus
\textnote{\textbf{Rit.}}

Beato è il \[E]cuo\[B]re che per\[C#-]do\[A]na!
Miseri\[E]cordia riceve\[B]rà da Dio in ci\[F#]elo! \rep{2}

\endchorus



%%%%% STROFA
\beginverse		%Oppure \beginverse* se non si vuole il numero di fianco
%\memorize 		% <<< DECOMMENTA se si vuole utilizzarne la funzione
%\chordsoff		% <<< DECOMMENTA se vuoi una strofa senza accordi

^ Solo il per^dono riporte^rà
^pace nel ^mon^do.
Solo il per^dono ci svele^rà
^come f^igli t^uoi.

\endverse



%%%%% RITORNELLO
\beginchorus
\textnote{\textbf{Rit.}}

Beato è il \[E]cuo\[B]re che per\[C#-]do\[A]na!
Miseri\[E]cordia riceve\[B]rà da Dio in ci\[F#]elo! \rep{2}

\endchorus




%%%%% STROFA
\beginverse		%Oppure \beginverse* se non si vuole il numero di fianco
%\memorize 		% <<< DECOMMENTA se si vuole utilizzarne la funzione
%\chordsoff		% <<< DECOMMENTA se vuoi una strofa senza accordi

^ Col sangue in ^croce hai pagato ^Tu
^le nostre ^pover^tà.
Se noi ci am^iamo e restiamo in^ te
^il mondo ^crede^rà!

\endverse



%%%%% RITORNELLO
\beginchorus
\textnote{\textbf{Rit.}}

Beato è il \[E]cuo\[B]re che per\[C#-]do\[A]na!
Miseri\[E]cordia riceve\[B]rà da Dio in ci\[F#]elo! \rep{2}

\endchorus




%%%%% BRIDGE
\beginverse*		%Oppure \beginverse* se non si vuole il numero di fianco
%\memorize 		% <<< DECOMMENTA se si vuole utilizzarne la funzione
%\chordsoff		% <<< DECOMMENTA se vuoi una strofa senza accordi
\textnote{Bridge}
\[A]Le nostre an\[B]gosce ed ansie\[C#-]tà
get\[A]tiamo ogni \[B]attimo in \[A]te.
Amore \[B]che non abbandona \[C#-]mai,
\[A]vivi in \[B]mezzo a \[C#]noi!

\endverse



%%%%% RITORNELLO
\beginchorus
\textnote{\textbf{Rit.}}

Beato è il \[A]cuo\[E]re che per\[F#-]do\[D]na!
Miseri\[A]cordia riceve\[E]rà da Dio in ci\[B]elo! \rep{4}

\endchorus



%%%%%% EV. INTERMEZZO
\beginverse*
\vspace*{1.3\versesep}
{
	\nolyrics
	\musicnote{Chiusura}
	
	\ifchorded

	%---- Prima riga -----------------------------
	\vspace*{-\versesep}
	\[C#-]  \[A]	\[E]  % \[*D] per indicare le pennate, \rep{2} le ripetizioni


	%---- Ogni riga successiva -------------------
	\vspace*{-\versesep}
	\[B] \[(B)] \[(B)]  \[C#-]


	\fi
	%---- Ev Indicazioni -------------------------			
	%\textnote{\textit{(ripetizione della strofa)}} 
	 
}
\vspace*{\versesep}
\endverse


\endsong
%------------------------------------------------------------
%			FINE CANZONE
%------------------------------------------------------------




%++++++++++++++++++++++++++++++++++++++++++++++++++++++++++++
%			CANZONE TRASPOSTA
%++++++++++++++++++++++++++++++++++++++++++++++++++++++++++++
\ifchorded
%decremento contatore per avere stesso numero
\addtocounter{songnum}{-1} 
\beginsong{Beato il cuore (Inno GMG 2016)}[by={Jakub Blycharz, GMG Cracovia 2016}]	% <<< COPIA TITOLO E AUTORE
\transpose{0} 						% <<< TRASPOSIZIONE #TONI + - (0 nullo)
%\preferflats SE VOGLIO FORZARE i bemolle come alterazioni
%\prefersharps SE VOGLIO FORZARE i # come alterazioni
\ifchorded
	\textnote{Tonalità originale}	% <<< EV COMMENTI (tonalità originale/migliore)
\fi


%%%%%% INTRODUZIONE
\ifchorded
\vspace*{\versesep}
\textnote{Intro: \qquad \qquad  }%(\eighthnote 116) % << MODIFICA IL TEMPO
% Metronomo: \eighthnote (ottavo) \quarternote (quarto) \halfnote (due quarti)
\vspace*{-\versesep}
\beginverse*

\nolyrics

%---- Prima riga -----------------------------
\vspace*{-\versesep}
\[C#-]  \[A]	\[E]  % \[*D] per indicare le pennate, \rep{2} le ripetizioni

%---- Ogni riga successiva -------------------
\vspace*{-\versesep}
\[B] \[(B)] \[(B)]  \[C#-]

%---- Ev Indicazioni -------------------------			
%\textnote{\textit{(Oppure tutta la strofa)} }	

\endverse
\fi




%%%%% STROFA
\beginverse		%Oppure \beginverse* se non si vuole il numero di fianco
\memorize 		% <<< DECOMMENTA se si vuole utilizzarne la funzione
%\chordsoff		& <<< DECOMMENTA se vuoi una strofa senza accordi

\[C#-]Sei sceso \[A]dalla tua immensi\[E]tà
\[D]in nostro a\[A]iu\[E]to.
Miseri\[B]cordia  scorre  da \[F#]te
\[A]sopra \[B]tutti \[C#]noi.


\endverse


%%%%% STROFA
\beginverse*	%Oppure \beginverse* se non si vuole il numero di fianco
%\memorize 		% <<< DECOMMENTA se si vuole utilizzarne la funzione
%\chordsoff		& <<< DECOMMENTA se vuoi una strofa senza accordi

^ Persi in un ^mondo d’oscuri^tà
^lì Tu ci ^tro^vi.
Nelle tue ^braccia ci stringi e ^poi
^dai la ^vita per ^noi.


\endverse



%%%%% RITORNELLO
\beginchorus
\textnote{\textbf{Rit.}}

Beato è il \[E]cuo\[B]re che per\[C#-]do\[A]na!
Miseri\[E]cordia riceve\[B]rà da Dio in ci\[F#]elo! \rep{2}

\endchorus



%%%%% STROFA
\beginverse		%Oppure \beginverse* se non si vuole il numero di fianco
%\memorize 		% <<< DECOMMENTA se si vuole utilizzarne la funzione
%\chordsoff		% <<< DECOMMENTA se vuoi una strofa senza accordi

^ Solo il per^dono riporte^rà
^pace nel ^mon^do.
Solo il per^dono ci svele^rà
^come f^igli t^uoi.

\endverse



%%%%% RITORNELLO
\beginchorus
\textnote{\textbf{Rit.}}

Beato è il \[E]cuo\[B]re che per\[C#-]do\[A]na!
Miseri\[E]cordia riceve\[B]rà da Dio in ci\[F#]elo! \rep{2}

\endchorus




%%%%% STROFA
\beginverse		%Oppure \beginverse* se non si vuole il numero di fianco
%\memorize 		% <<< DECOMMENTA se si vuole utilizzarne la funzione
%\chordsoff		% <<< DECOMMENTA se vuoi una strofa senza accordi

^ Col sangue in ^croce hai pagato ^Tu
^le nostre ^pover^tà.
Se noi ci am^iamo e restiamo in^ te
^il mondo ^crede^rà!

\endverse



%%%%% RITORNELLO
\beginchorus
\textnote{\textbf{Rit.}}

Beato è il \[E]cuo\[B]re che per\[C#-]do\[A]na!
Miseri\[E]cordia riceve\[B]rà da Dio in ci\[F#]elo! \rep{2}

\endchorus




%%%%% BRIDGE
\beginverse*		%Oppure \beginverse* se non si vuole il numero di fianco
%\memorize 		% <<< DECOMMENTA se si vuole utilizzarne la funzione
%\chordsoff		% <<< DECOMMENTA se vuoi una strofa senza accordi
\textnote{Bridge}
\[A]Le nostre an\[B]gosce ed ansie\[C#-]tà
get\[A]tiamo ogni \[B]attimo in \[A]te.
Amore \[B]che non abbandona \[C#-]mai,
\[A]vivi in \[B]mezzo a \[C#]noi!

\endverse



%%%%% RITORNELLO
\beginchorus
\textnote{\textbf{Rit.}}

Beato è il \[A]cuo\[E]re che per\[F#-]do\[D]na!
Miseri\[A]cordia riceve\[E]rà da Dio in ci\[B]elo! \rep{4}

\endchorus



%%%%%% EV. INTERMEZZO
\beginverse*
\vspace*{1.3\versesep}
{
	\nolyrics
	\musicnote{Chiusura}
	
	\ifchorded

	%---- Prima riga -----------------------------
	\vspace*{-\versesep}
	\[C#-]  \[A]	\[E]  % \[*D] per indicare le pennate, \rep{2} le ripetizioni


	%---- Ogni riga successiva -------------------
	\vspace*{-\versesep}
	\[B] \[(B)] \[(B)]  \[C#-]


	\fi
	%---- Ev Indicazioni -------------------------			
	%\textnote{\textit{(ripetizione della strofa)}} 
	 
}
\vspace*{\versesep}
\endverse


\endsong


\fi
%++++++++++++++++++++++++++++++++++++++++++++++++++++++++++++
%			FINE CANZONE TRASPOSTA
%++++++++++++++++++++++++++++++++++++++++++++++++++++++++++++

%CCC
%-------------------------------------------------------------
%			INIZIO	CANZONE
%-------------------------------------------------------------


%titolo: 	Come Te
%autore: 	Gen Rosso
%tonalita: 	Do



%%%%%% TITOLO E IMPOSTAZONI
\beginsong{Come Te}[by={Gen Rosso}] 	% <<< MODIFICA TITOLO E AUTORE
\transpose{0} 						% <<< TRASPOSIZIONE #TONI (0 nullo)
\momenti{Comunione; Ringraziamento; Natale; Avvento}							% <<< INSERISCI MOMENTI	
% momenti vanno separati da ; e vanno scelti tra:
% Ingresso; Atto penitenziale; Acclamazione al Vangelo; Dopo il Vangelo; Offertorio; Comunione; Ringraziamento; Fine; Santi; Pasqua; Avvento; Natale; Quaresima; Canti Mariani; Battesimo; Prima Comunione; Cresima; Matrimonio; Meditazione;
\ifchorded
	%\textnote{Tonalità originale }	% <<< EV COMMENTI (tonalità originale/migliore)
\fi


%%%%%% INTRODUZIONE
\ifchorded
\vspace*{\versesep}
\textnote{Intro: \qquad \qquad  }%(\eighthnote 116) % << MODIFICA IL TEMPO
% Metronomo: \eighthnote (ottavo) \quarternote (quarto) \halfnote (due quarti)
\vspace*{-\versesep}
\beginverse*

\nolyrics

%---- Prima riga -----------------------------
\vspace*{-\versesep}
\[C] \[G]  \[C]	 \[G] % \[*D] per indicare le pennate, \rep{2} le ripetizioni

%---- Ogni riga successiva -------------------
%\vspace*{-\versesep}
%\[G] \[C]  \[D]	

%---- Ev Indicazioni -------------------------			
%\textnote{\textit{(Oppure tutta la strofa)} }	

\endverse
\fi




%%%%% STROFA
\beginverse		%Oppure \beginverse* se non si vuole il numero di fianco
\memorize 		% <<< DECOMMENTA se si vuole utilizzarne la funzione
%\chordsoff		& <<< DECOMMENTA se vuoi una strofa senza accordi

\[C]Come \[G]Te, che sei \[C]sceso dal \[G]cielo \[C] \[G]
ad inse\[D-]gnarci l’a\[F]more di \[G]Dio
\[D-7] e hai preso su di \[A-]Te
la \[(*G)]nostra \[C]povera e \[F]fragile \[D-]umani\[G]tà.

\endverse




\beginverse*

^Come ^Te, che non ^ti sei te^nuto ^  ^
come se^greto l’a^more di ^Dio,
^ ma sei venuto ^qui
a ^rinno^vare la ^vita dell’^umani^tà.

\endverse




\beginverse*

\[G]Io non mi tirerò ind\[F]ietro
io non avrò più p\[C]aura
di dare tutto di \[G]me.  \[*F] \[*G]

\endverse


%%%%% RITORNELLO
\beginchorus
\textnote{\textbf{Rit.}}

\[C]Per am\[F]ore dell’\[C]uomo,\[F]
d’\[D-]ogni u\[F]omo come \[G]me
\[C]mi son \[F]fatto si\[C]lenzio\[F]
\[D-] per diven\[F]tare come \[G]Te.
\[C] Per \[G]amore \[A-]tuo 
\[G] mi farò \[F]servo d’ogni \[G]uomo che \[C]vive\[G]
\[A-] servo \[C]d’ogni u\[F]omo
\[C] per a\[G]mo\[F]re. \[G]

\endchorus






%%%%%% EV. INTERMEZZO
\beginverse*
\vspace*{1.3\versesep}
{
	\nolyrics
	\textnote{Intermezzo musicale }
	\ifchorded
	

	%---- Prima riga -----------------------------
	\vspace*{-\versesep}
	\[C] \[G]  \[C]	 \[G]


	\fi
	%---- Ev Indicazioni -------------------------			
	%\textnote{\textit{(ripetizione della strofa)}} 
	 
}
\vspace*{\versesep}
\endverse




%%%%% STROFA
\beginverse

^Come ^Te che hai la^sciato le ^stelle ^ ^
per farti ^proprio come ^uno di ^noi,
^ senza tenere ^niente
hai ^dato ^anche la ^vita, hai pa^gato per ^noi.

\endverse




\beginverse*

\[G]Davanti a questo mis\[F]tero
come potrò ricam\[C]biare,
che cosa mai potrò f\[G]are?  \[*F]\[*G]

\endverse









\endsong
%------------------------------------------------------------
%			FINE CANZONE
%------------------------------------------------------------

%-------------------------------------------------------------
%			INIZIO	CANZONE
%-------------------------------------------------------------


%titolo: 	Come un fiume
%autore: 	Paci, Preti
%tonalita: 	Do 



%%%%%% TITOLO E IMPOSTAZONI
\beginsong{Come un fiume}[by={D. Paci, P. Preti}] 	% <<< MODIFICA TITOLO E AUTORE
\transpose{0} 						% <<< TRASPOSIZIONE #TONI (0 nullo)
\momenti{Comunione; Avvento}							% <<< INSERISCI MOMENTI	
% momenti vanno separati da ; e vanno scelti tra:
% Ingresso; Atto penitenziale; Acclamazione al Vangelo; Dopo il Vangelo; Offertorio; Comunione; Ringraziamento; Fine; Santi; Pasqua; Avvento; Natale; Quaresima; Canti Mariani; Battesimo; Prima Comunione; Cresima; Matrimonio; Meditazione;
\ifchorded
	%\textnote{Tonalità originale }	% <<< EV COMMENTI (tonalità originale/migliore)
\fi

%%%%%% INTRODUZIONE
\ifchorded
\vspace*{\versesep}
\textnote{Intro: \qquad \qquad  }%(\eighthnote 116) % << MODIFICA IL TEMPO
% Metronomo: \eighthnote (ottavo) \quarternote (quarto) \halfnote (due quarti)
\vspace*{-\versesep}
\beginverse*

\nolyrics

%---- Prima riga -----------------------------
\vspace*{-\versesep}
\[C] \[G]  \[C]	\[C] % \[*D] per indicare le pennate, \rep{2} le ripetizioni

%---- Ogni riga successiva -------------------
%\vspace*{-\versesep}
%\[G] \[C]  \[D]	

%---- Ev Indicazioni -------------------------			
%\textnote{\textit{(Oppure tutta la strofa)} }	

\endverse
\fi

%%%%% RITORNELLO
\beginchorus
\textnote{\textbf{Rit.}}

Come un \[C]fiume in piena che \brk la sabbia \[G]non può arrestare
come l'\[C7]onda che dal mare \brk si di\[F]stende sulla riva
ti pre\[F-]ghiamo Padre \brk che così si \[C]sciolga il nostro amore
e l'a\[D]more dove ar\[D7]riva \brk sciolga il \[G]dubbio e la \[G7]paura. 

\endchorus

%%%%% STROFA
\beginverse		%Oppure \beginverse* se non si vuole il numero di fianco
\memorize 		% <<< DECOMMENTA se si vuole utilizzarne la funzione
%\chordsoff		& <<< DECOMMENTA se vuoi una strofa senza accordi

Come un \[C]pesce che risale a nuoto \[ \brk G]fino alla sorgente
va a sco\[C7]prire dove nasce \brk  e si di\ch{F}{f}{f}{ff}onde la sua vita
ti pre\[F-]ghiamo Padre che  \brk noi risa\[C]liamo la corrente
fino ad \[G]arrivare alla vita \brk  \[F]nell'a\[C]more.  \[(G7)]

\endverse

%%%%% RITORNELLO
\beginchorus
\textnote{\textbf{Rit.}}

Come un \[C]fiume in piena che \brk la sabbia \[G]non può arrestare
come l'\[C7]onda  \brk che dal mare si di\[F]stende sulla riva
ti pre\[F-]ghiamo Padre \brk che così si \[C]sciolga il nostro amore
e l'a\[D]more dove ar\[D7]riva \brk  sciolga il \[G]dubbio e la \[G7]paura. 

\endchorus

%%%%% STROFA
\beginverse		%Oppure \beginverse* se non si vuole il numero di fianco
%\memorize 		% <<< DECOMMENTA se si vuole utilizzarne la funzione
%\chordsoff		& <<< DECOMMENTA se vuoi una strofa senza accordi

Come l'^erba che germoglia  \brk cresce ^senza far rumore
ama il ^giorno della pioggia \brk  si addor^menta sotto il sole
ti pre^ghiamo Padre che  \brk così in un ^giorno di silenzio
anche in ^noi germogli \brk  questa vita ^nell'a^more. \[C7] \[A7]

\endverse

\textnote{\textit{Si alza la tonalità}}
\transpose{2}

%%%%% RITORNELLO
\beginchorus
\textnote{\textbf{Rit.}}

Come un \[C]fiume in piena che \brk la sabbia \[G]non può arrestare
come l'\[C7]onda che dal mare  \brk si di\[F]stende sulla riva
ti pre\[F-]ghiamo Padre \brk che così si \[C]sciolga il nostro amore
e l'a\[D]more dove ar\[D7]riva \brk  sciolga il \[G]dubbio e la \[G7]paura. 

\endchorus

%%%%% STROFA
\beginverse		%Oppure \beginverse* se non si vuole il numero di fianco
%\memorize 		% <<< DECOMMENTA se si vuole utilizzarne la funzione
%\chordsoff		& <<< DECOMMENTA se vuoi una strofa senza accordi

Come un ^albero che affonda  \brk le ra^dici nella terra
e su ^quella terra un uomo  \brk costru^isce la sua casa
ti pre^ghiamo Padre buono  \brk di por^tarci alla tua casa
dove ^vivere una vita piena \brk  ^nell'a^more. 
\endverse

%%%%%% EV. CHIUSURA SOLO STRUMENTALE
\ifchorded
\beginchorus %oppure \beginverse*
\vspace*{1.3\versesep}
\textnote{Chiusura } %<<< EV. INDICAZIONI

\[C*]

\endchorus  %oppure \endverse
\fi


\endsong
%------------------------------------------------------------
%			FINE CANZONE
%------------------------------------------------------------

%DDD
%-------------------------------------------------------------
%			INIZIO	CANZONE
%-------------------------------------------------------------


%titolo: 	Dall'aurora al tramonto
%autore: 	Casucci, Balduzzi
%tonalita: 	Si- 



%%%%%% TITOLO E IMPOSTAZONI
\beginsong{Dall'aurora al tramonto}[by={Casucci, Balduzzi}] 	% <<< MODIFICA TITOLO E AUTORE
\transpose{-2} 						% <<< TRASPOSIZIONE #TONI (0 nullo)
\momenti{Comunione; Meditazione; Ringraziamento}							% <<< INSERISCI MOMENTI	
% momenti vanno separati da ; e vanno scelti tra:
% Ingresso; Atto penitenziale; Acclamazione al Vangelo; Dopo il Vangelo; Offertorio; Comunione; Ringraziamento; Fine; Santi; Pasqua; Avvento; Natale; Quaresima; Canti Mariani; Battesimo; Prima Comunione; Cresima; Matrimonio; Meditazione;
\ifchorded
	%\textnote{Tonalità originale }	% <<< EV COMMENTI (tonalità originale/migliore)
\fi





%%%%%% INTRODUZIONE
\ifchorded
\vspace*{\versesep}
\textnote{Intro: \qquad \qquad  }%(\eighthnote 116) % << MODIFICA IL TEMPO
% Metronomo: \eighthnote (ottavo) \quarternote (quarto) \halfnote (due quarti)
\vspace*{-\versesep}
\beginverse*

\nolyrics

%---- Prima riga -----------------------------
\vspace*{-\versesep}
\[C#-]\[E]\[A]\[B] \rep{2}	 % \[*D] per indicare le pennate, \rep{2} le ripetizioni

%---- Ogni riga successiva -------------------
%\vspace*{-\versesep}
%\[G] \[C]  \[D]	

%---- Ev Indicazioni -------------------------			
\textnote{\textit{(Oppure tutto il ritornello)} }	

\endverse
\fi




\ifchorded
\beginverse*
\vspace*{-0.8\versesep}
{ }
\vspace*{-\versesep}
\endverse
\fi


\beginchorus
\memorize
\[C#-]Dall'au\[E]rora io \[F#-]cerco \[B]te,
\[C#-]fino al tra\[E]monto ti \[F#-]chia\[B]mo.
\[C#-]Ha sete \[G#-]solo di \[A]te l'\[B]anima \[C#-]mia
come \[G#-]terra de\[A]ser\[B]ta. \rep{2}
\endchorus




\beginverse
^Non mi ferme^rò un ^solo i^stante
^sempre cante^rò ^la tua ^lode.
^Perché sei il mio ^Dio, ^il mio ri^paro
^mi protegge^rai ^all'ombra delle tue \[B4]ali.
\endverse



\beginverse
^Non mi ferme^rò un ^solo i^stante,
^io racconte^rò ^le tue ^opere
^perché sei il mio ^Dio, ^unico ^bene.
^Nulla mai po^trà ^la notte contro di \[B4]me.
\endverse
\beginchorus
^Dall'au^rora io ^cerco ^te,
^fino al tra^monto ti ^chia^mo.
^Ha sete ^solo di ^te l'^anima ^mia
come ^terra de^ser^ta.

\endchorus


%%%%%% EV. FINALE

\beginchorus %oppure \beginverse*
\vspace*{1.3\versesep}
\textnote{Finale } %<<< EV. INDICAZIONI
\[C#-]Ha sete \[G#-]solo di \[A]te l'\[B]anima \[E]mia
come \[A]terra de\[B]se-\[B]e-r\[E]ta.
\endchorus  %oppure \endverse



\endsong


%-------------------------------------------------------------
%			INIZIO	CANZONE
%-------------------------------------------------------------


%titolo: 	Danza la vita
%autore: 	
%tonalita: 	Re 



%%%%%% TITOLO E IMPOSTAZONI
\beginsong{Danza la vita}[by={Canto Scout}] 	% <<< MODIFICA TITOLO E AUTORE
\transpose{0} 						% <<< TRASPOSIZIONE #TONI (0 nullo)
\momenti{Cresima; Fine; Battesimo; Matrimonio; Ringraziamento; Comunione}							% <<< INSERISCI MOMENTI	
% momenti vanno separati da ; e vanno scelti tra:
% Ingresso; Atto penitenziale; Acclamazione al Vangelo; Dopo il Vangelo; Offertorio; Comunione; Ringraziamento; Fine; Santi; Pasqua; Avvento; Natale; Quaresima; Canti Mariani; Battesimo; Prima Comunione; Cresima; Matrimonio; Meditazione;
\ifchorded
	%\textnote{Tonalità originale }	% <<< EV COMMENTI (tonalità originale/migliore)
\fi





%%%%%% INTRODUZIONE
\ifchorded
\vspace*{\versesep}
\textnote{Intro: \qquad \qquad  }%(\eighthnote 116) % << MODIFICA IL TEMPO
% Metronomo: \eighthnote (ottavo) \quarternote (quarto) \halfnote (due quarti)
\vspace*{-\versesep}
\beginverse*

\nolyrics

%---- Prima riga -----------------------------
\vspace*{-\versesep}
\[D] \[G]  \[D]	\[G] % \[*D] per indicare le pennate, \rep{2} le ripetizioni

%---- Ogni riga successiva -------------------
%\vspace*{-\versesep}
%\[G] \[C]  \[D]	

%---- Ev Indicazioni -------------------------			
\textnote{\textit{(a ripetizione)} }	

\endverse
\fi




%%%%% STROFA
\beginverse
\memorize
\[D]Canta con la \[G]voce e con il \[D]cuore, \[G]
\[D]con la bocca e \[G]con la vita, \[D] \[G]
\[D]canta senza \[G]stonature, \[D] \[G]
la \[D]verità \[*G] del \[D]cuore. \[G]
\[D]Canta come \[G]cantano i viandanti 
\echo{\[D]canta come \[G]cantano i viandanti}
non \[D]solo per riem\[G]pire il tempo, 
\echo{non \[D]solo per \[G]riempire il tempo,}
\[D]Ma per soste\[G]nere lo sforzo 
\echo{\[D]Ma per soste\[G]nere lo sforzo.}
\[D]Canta \[G] e cam\[D]mina \[G]
\[D]Canta \[G] e cam\[D]mina \[G]
Se \[A]poi, credi non possa ba\[B-]stare
segui il \[E]tempo, stai \[G]pron\[A]to e
\endverse





%%%%% RITORNELLO

\beginchorus
\textnote{\textbf{Rit.}}
\[D]Danza la \[G]vita, al \[A]ritmo dello \[D]Spirito. 
\qquad \quad \echo{Spirito che riempi i nostri }
\[B-]Danza, \[G]danza al \[A]ritmo che c'è in \[D]te. 
\echo{cuor, danza assieme a noi. Danza}
\[G]Spirito \[A]che \[D]riempi i nostri 
\echo{la vita al ritmo dello Spirito}
\[B-]cuor. \[G]Danza assieme a \[A]no\[D]i. 
\echo{Danza, danza al ritmo che c'è in te.}
\endchorus




%%%%% STROFA
\beginverse
Cam^mina sulle ^orme del Si^gnore, ^
non ^solo con i ^piedi ^ma ^ 
^usa soprat^tutto il cuore.^^
^Ama ^ chi è con ^te. ^
Cam^mina con lo ^zaino sulle spalle 
\echo{Cam^mina con lo ^zaino sulle spalle}
^la fatica a^iuta a crescere 
\echo{^la fatica a^iuta a crescere}
^nella con^divisione 
\echo{^nella con^divisione.}
^Canta ^ e cam^mina, ^
^canta  ^ e cam^mina. ^
Se ^poi, credi non possa ba^stare
segui il ^tempo, stai ^pron^to e
\endverse






\endsong
%------------------------------------------------------------
%			FINE CANZONE
%------------------------------------------------------------


%EEE
%-------------------------------------------------------------
%			INIZIO	CANZONE
%-------------------------------------------------------------


%titolo: 	È bello lodart
%autore: 	Gen Verde
%tonalita: 	Sol 



%%%%%% TITOLO E IMPOSTAZONI
\beginsong{È bello lodarti}[by={Gen Verde}] 	% <<< MODIFICA TITOLO E AUTORE
\transpose{0} 						% <<< TRASPOSIZIONE #TONI (0 nullo)
\momenti{Ingresso; Congedo}							% <<< INSERISCI MOMENTI	
% momenti vanno separati da ; e vanno scelti tra:
% Ingresso; Atto penitenziale; Acclamazione al Vangelo; Dopo il Vangelo; Offertorio; Comunione; Ringraziamento; Fine; Santi; Pasqua; Avvento; Natale; Quaresima; Canti Mariani; Battesimo; Prima Comunione; Cresima; Matrimonio; Meditazione;
\ifchorded
	%\textnote{Tonalità originale }	% <<< EV COMMENTI (tonalità originale/migliore)
\fi



%%%%%% INTRODUZIONE
\ifchorded
\vspace*{\versesep}
\musicnote{
\begin{minipage}{0.48\textwidth}
\textbf{Intro}
\hfill 
%( \eighthnote \, 80)   % <<  MODIFICA IL TEMPO
% Metronomo: \eighthnote (ottavo) \quarternote (quarto) \halfnote (due quarti)
\end{minipage}
} 	
\vspace*{-\versesep}
\beginverse*


\nolyrics

%---- Prima riga -----------------------------
\vspace*{-\versesep}
\[G] \[D]  \[C] \[C]	 \rep{2} % \[*D] per indicare le pennate, \rep{2} le ripetizioni

%---- Ogni riga successiva -------------------
%\vspace*{-\versesep}
%\[G] \[C]  \[D]	

%---- Ev Indicazioni -------------------------			
%\textnote{\textit{(Oppure tutta la strofa)} }	

\endverse
\fi





%%%%% RITORNELLO
\beginchorus
\textnote{\textbf{Rit.}}
\[G] È \[D]bello can\[C]tare il tuo a\[G]more,
\[A-7] è bello lo\[G]dare il tuo \[D]nome.
\[G] È bello can\[B4]tare il tuo a\[C]more,
è \[G]bello lo\[D]darti, Si\[C]gnore,
è \[G]bello can\[D]tare a \[C]te!  \rep{2}
\endchorus



%%%%% STROFA
\beginverse		%Oppure \beginverse* se non si vuole il numero di fianco
\memorize 		% <<< DECOMMENTA se si vuole utilizzarne la funzione
%\chordsoff		% <<< DECOMMENTA se vuoi una strofa senza accordi
\[E-]Tu che sei l'amore infi\[B-6]nito
che nep\[C]pure il cielo può contenere,
ti \[A-]sei fatto \[7]uomo, \[D6]Tu sei venuto qui
ad \[B7]abitare in mezzo a \[C]noi, allora\dots  
\endverse




%%%%% STROFA
\beginverse		%Oppure \beginverse* se non si vuole il numero di fianco
%\memorize 		% <<< DECOMMENTA se si vuole utilizzarne la funzione
%\chordsoff		% <<< DECOMMENTA se vuoi una strofa senza accordi
^Tu che conti tutte le ^stelle
e le ^chiami una ad una per nome,
da ^mille sen^tieri ^ci hai radunati qui,
^ci hai chiamati figli ^tuoi, allora\dots 
\endverse


%%%%%% EV. FINALE

\beginchorus %oppure \beginverse*
\vspace*{1.3\versesep}
\textnote{\textbf{Finale} \textit{(rallentando)}} %<<< EV. INDICAZIONI

è \[G]bello can\[D]tare a \[C]te!  \[C] \[G]

\endchorus  %oppure \endverse

\endsong
%------------------------------------------------------------
%			FINE CANZONE
%------------------------------------------------------------
%titolo: 	Eccomi qui
%autore: 	Diliberto Monti
%tonalita: 	Do 



%%%%%% TITOLO E IMPOSTAZONI
\beginsong{Eccomi qui}[by={Diliberto Monti}] 	% <<< MODIFICA TITOLO E AUTORE
\transpose{0} 						% <<< TRASPOSIZIONE #TONI (0 nullo)
\momenti{Ingresso; Dopo il Vangelo}							% <<< INSERISCI MOMENTI	
% momenti vanno separati da ; e vanno scelti tra:
% Ingresso; Atto penitenziale; Acclamazione al Vangelo; Dopo il Vangelo; Offertorio; Comunione; Ringraziamento; Fine; Santi; Pasqua; Avvento; Natale; Quaresima; Canti Mariani; Battesimo; Prima Comunione; Cresima; Matrimonio; Meditazione;
\ifchorded
	%\textnote{Tonalità originale }	% <<< EV COMMENTI (tonalità originale/migliore)
\fi


%%%%%% INTRODUZIONE
\ifchorded
\vspace*{\versesep}
\textnote{Intro: \qquad \qquad  }%(\eighthnote 116) % << MODIFICA IL TEMPO
% Metronomo: \eighthnote (ottavo) \quarternote (quarto) \halfnote (due quarti)
\vspace*{-\versesep}
\beginverse*

\nolyrics

%---- Prima riga -----------------------------
\vspace*{-\versesep}
\[*C] \[*E7] \[A-] 	 % \[*D] per indicare le pennate, \rep{2} le ripetizioni

%---- Ogni riga successiva -------------------
\vspace*{-\versesep}
\[F] \[G] \[G] 

%---- Ev Indicazioni -------------------------			
%\textnote{\textit{(Oppure tutta la strofa)} }	

\endverse
\fi


%%%%% RITORNELLO
\beginchorus
\textnote{\textbf{Rit.}}

\[*C] E\[*E7]ccomi \[A-]qui, \brk  \[F] di nuovo a \[G]te Signore, 
\[*C] e\[*E7]ccomi \[A-]qui: \brk  \[F] accetta \[G]la mia vita;
\[C] non dire \[G]no  \brk a \[F]chi si affi da a \[C]te,
\[A-] mi accoglie\[D-]rai \brk \[F] per sempre \[G]nel tuo amore.\[F]\[C]

\endchorus



%%%%% STROFA
\beginverse		%Oppure \beginverse* se non si vuole il numero di fianco
\memorize 		% <<< DECOMMENTA se si vuole utilizzarne la funzione
%\chordsoff		& <<< DECOMMENTA se vuoi una strofa senza accordi

Q\[C]uando hai scelto di \[G]vivere quag\[A-]giù, \[A-7]
q\[F]uando hai voluto che \[G]fossimo figli t\[C]uoi, \[C7]
\[F] ti sei do\[C]nato ad \[D-]una come \[A-]noi 
\[A-7] e hai cammi\[D-]nato sulle st\[F]rade dell'\[G]uomo.

\endverse

%%%%% STROFA
\beginverse		%Oppure \beginverse* se non si vuole il numero di fianco
%\memorize 		% <<< DECOMMENTA se si vuole utilizzarne la funzione
%\chordsoff		% <<< DECOMMENTA se vuoi una strofa senza accordi

P^rima che il Padre ti ^richiamasse a ^sé ^
p^rima del buio che il tuo g^rido spezze^rà ^
^ tu hai pro^messo di ^non lasciarci p^iù
^ di accompag^narci sulle st^rade del ^mondo.

\endverse

%%%%% STROFA
\beginverse		%Oppure \beginverse* se non si vuole il numero di fianco
%\memorize 		% <<< DECOMMENTA se si vuole utilizzarne la funzione
%\chordsoff		% <<< DECOMMENTA se vuoi una strofa senza accordi

^Ora ti prego con^ducimi con ^te ^
^nella fatica di ser^vir la veri^tà ^
^ sarò vi^cino a ^chi ti invoche^rà
^ e mi guide^rai sulle st^rade dell'^uomo.

\endverse

\endsong
%------------------------------------------------------------
%			FINE CANZONE
%------------------------------------------------------------

%-------------------------------------------------------------
%			INIZIO	CANZONE
%-------------------------------------------------------------


%titolo: 	Ed essi si ameranno
%autore: 	Meregalli
%tonalita: 	Re / Do -2 / La -5 / Sol -7 /



%%%%%% TITOLO E IMPOSTAZONI
\beginsong{Ed essi si ameranno}[by={Meregalli}]
\transpose{0} % <<< TRASPOSIZIONE #TONI + - (0 nullo)
%\preferflats %SE VOGLIO FORZARE i bemolle come alterazioni
%\prefersharps %SE VOGLIO FORZARE i # come alterazioni						
\momenti{Fine; Pasqua; Matrimonio; Santi; Comunione;}			% <<< INSERISCI MOMENTI	
% momenti vanno separati da ; e vanno scelti tra:
% Ingresso; Atto penitenziale; Acclamazione al Vangelo; Dopo il Vangelo; Offertorio; Comunione; Ringraziamento; Fine; Santi; Pasqua; Avvento; Natale; Quaresima; Canti Mariani; Battesimo; Prima Comunione; Cresima; Matrimonio; Meditazione;
\ifchorded
	%\textnote{Tonalità originale }	% <<< EV COMMENTI (tonalità originale/migliore)
\fi



%%%%%% INTRODUZIONE
\ifchorded
\vspace*{\versesep}
\textnote{Intro: \qquad \qquad  }%(\eighthnote 116) % << MODIFICA IL TEMPO
% Metronomo: \eighthnote (ottavo) \quarternote (quarto) \halfnote (due quarti)
\vspace*{-\versesep}
\beginverse*

\nolyrics

%---- Prima riga -----------------------------
\vspace*{-\versesep}
\[D] 

%---- Ogni riga successiva -------------------
%\vspace*{-\versesep}
%\[A-] \[E-] \[F] \[C] \[D-] \[A-] \[F] \[*G] \[A-] 

%---- Ev Indicazioni -------------------------			
%\textnote{\textit{(Oppure tutta la strofa)} }	

\endverse



%%%%% STROFA
\beginverse 	%Oppure \beginverse* se non si vuole il numero di fianco
\memorize 		% <<< DECOMMENTA se si vuole utilizzarne la funzione
%\chordsoff		% <<< DECOMMENTA se vuoi una strofa senza accordi

\[D]Vai a dire alla terra 
di sve\[F#]gliare dal sonno le genti;
dì alla \[D7]folgore, al tuono e alla \[B7]voce 
di inon\[E-]dare di luce la \[A]notte;
dì alle \[E-]nuvole bianche del \[A]cielo 
di var\[D]care la soglia del \[A4]tem\[A]po.

\endverse

\beginverse*	%Oppure \beginverse* se non si vuole il numero di fianco
%\memorize 		% <<< DECOMMENTA se si vuole utilizzarne la funzione
%\chordsoff		% <<< DECOMMENTA se vuoi una strofa senza accordi

^Vai a dire alla terra 
di tre^mare al passo tonante
dei messa^ggeri di pace; pro^clama 
la mia ^legge d’amore alle ^genti;
dì che i ^vecchi delitti ho scor^dati... 
e tra ^voi non sia odio né ^guer^ra.

\endverse





%%%%% RITORNELLO
\beginchorus
\textnote{\textbf{Rit.}}
\[G]È finito questo vecchio mondo:
il \[F]cielo antico è lace\[G]rato.  \rep{2}
\endchorus

\beginchorus
\[F] Il mio popolo \[C]si radune\[G]rà. 
\[F] Il mio popolo \[C]si radune\[G]rà.
\[F] Il mio popolo \[C]si radune\[A]rà. \[A]
\endchorus








%%%%% STROFA
\beginverse 	%Oppure \beginverse* se non si vuole il numero di fianco
%\memorize 		% <<< DECOMMENTA se si vuole utilizzarne la funzione
%\chordsoff		% <<< DECOMMENTA se vuoi una strofa senza accordi


^Vai a dire alla terra 
che il Si^gnore l’ha amata da sempre,
che il suo ^servo reietto da ^molti 
si è addos^sato il peccato di ^tutti
e innal^zato ha patito la ^croce 
e, se^polto, ha rivisto la ^luce^.

\endverse

\beginverse*	%Oppure \beginverse* se non si vuole il numero di fianco
%\memorize 		% <<< DECOMMENTA se si vuole utilizzarne la funzione
%\chordsoff		% <<< DECOMMENTA se vuoi una strofa senza accordi

^Vai a dire alle genti 
di invi^tare i fratelli alla mensa;
ogni ^popolo che è sulla ^terra 
la mia ^legge proclami ed os^servi;
messag^geri di pace so^lerti 
testi^moni di pace fe^de^li.

\endverse








%%%%% RITORNELLO
\beginchorus
\textnote{\textbf{Rit.}}
\[G]Ecco che nasce il nuovo mondo:
il \[F]vecchio è termi\[G]nato.   \rep{2}
\endchorus

\beginchorus
\[F] Il mio popolo \[C]si radune\[G]rà. 
\[F] Il mio popolo \[C]si radune\[G]rà.
\[F] Il mio popolo \[C]si radune\[A]rà. \[A]
\endchorus





%%%%%% EV. FINALE
\beginchorus
\textnote{\textbf{Finale.}}
\textnote{\textit{Cambia il tempo, crescendo ad ogni ripetizione.}}
\[D]Ed il giorno an\[*A]co\[*G]ra \[*D]è spun\[*G]tato \[*A]nuo\[*D]vo,
\[D]uomini di \[*A]pa\[*G]ce ed \[*D]essi \[*G]si ame\[*A]ran\[*D]no.   \rep{3}
\endchorus



\endsong
%------------------------------------------------------------
%			FINE CANZONE
%------------------------------------------------------------
%FFF
%titolo{Frutto della nostra terra}
%autore{Buttazzo}
%-------------------------------------------------------------
%			INIZIO	CANZONE
%-------------------------------------------------------------


%titolo: 	Frutto della nostra terra
%autore: 	Buttazzo
%tonalita: 	Sol 

%%%%%% TITOLO E IMPOSTAZONI
\beginsong{Frutto della nostra terra}[by={Buttazzo}]
\transpose{0} 						% <<< TRASPOSIZIONE #TONI (0 nullo)
\momenti{Offertorio}							% <<< INSERISCI MOMENTI	
% momenti vanno separati da ; e vanno scelti tra:
% Ingresso; Atto penitenziale; Acclamazione al Vangelo; Dopo il Vangelo; Offertorio; Comunione; Ringraziamento; Fine; Santi; Pasqua; Avvento; Natale; Quaresima; Canti Mariani; Battesimo; Prima Comunione; Cresima; Matrimonio; Meditazione;
\ifchorded
	%\textnote{Tonalità originale }	% <<< EV COMMENTI (tonalità originale/migliore)
\fi





%%%%%% INTRODUZIONE
\ifchorded
\vspace*{\versesep}
\textnote{Intro: \qquad \qquad  }%(\eighthnote 116) % << MODIFICA IL TEMPO
% Metronomo: \eighthnote (ottavo) \quarternote (quarto) \halfnote (due quarti)
\vspace*{-\versesep}
\beginverse*

\nolyrics

%---- Prima riga -----------------------------
\vspace*{-\versesep}
\[G] \[D]  \[C]	 \[D] \[D]% \[*D] per indicare le pennate, \rep{2} le ripetizioni

%---- Ogni riga successiva -------------------
%\vspace*{-\versesep}
%\[G] \[C]  \[D]	

%---- Ev Indicazioni -------------------------			
\textnote{\textit{(Melodia del pianoforte o clarinetto)} }	

\endverse
\fi



%%%%% STROFA
\beginverse		%Oppure \beginverse* se non si vuole il numero di fianco
\memorize 		% <<< DECOMMENTA se si vuole utilizzarne la funzione
%\chordsoff		& <<< DECOMMENTA se vuoi una strofa senza accordi
\[G]Frutto della nostra \[C]terra 
\[G]del lavoro di ogni \[D]uomo
\[E-]pane della nostra \[B-]vita, 
cibo \[C]della quotidiani\[D]tà. \[D]
\[G]Tu che lo prendevi un \[C]giorno, 
 \[G]lo spezzavi per i \[D]tuoi
\[E-]oggi vieni in questo \[B-]pane, 
 cibo \[C]vero dell'umani\[D]tà.
\endverse




%%%%% RITORNELLO
\beginchorus
\textnote{\textbf{Rit.}}
E sarò \[G]pane, e sarò \[D]vino,
nella mia \[E-]vita, nelle tue \[B-]mani.
Ti accoglie\[C]rò dentro di \[D]me,
farò di \[E-]me un'offerta \[C]viva,
un sacri\[A-]ficio \[D] gradito a \[G]Te.
\endchorus





%%%%%% EV. INTERMEZZO
\beginverse*
\vspace*{1.3\versesep}
{
	\nolyrics
	\textnote{Breve intermezzo strumentale}
	
	\ifchorded

	%---- Prima riga -----------------------------
	\vspace*{-\versesep}
	\[C]  \[D]	 \[C]  

	\fi
	%---- Ev Indicazioni -------------------------			
	%\textnote{\textit{(ripetizione della strofa)}} 
	 
}
\vspace*{\versesep}
\endverse



%%%%% STROFA
\beginverse		%Oppure \beginverse* se non si vuole il numero di fianco
%\memorize 		% <<< DECOMMENTA se si vuole utilizzarne la funzione
%\chordsoff		% <<< DECOMMENTA se vuoi una strofa senza accordi
%\chordsoff
^Frutto della nostra ^terra
 ^del lavoro di ogni ^uomo
^vino delle nostre ^vigne 
 sulla ^mensa dei fratelli ^tuoi.  ^
^Tu che lo prendevi un ^giorno, 
 ^lo bevevi con i ^tuoi
^oggi vieni in questo ^vino 
 e ti ^doni per la vita ^mia.
\endverse





%%%%% RITORNELLO
\beginchorus
\textnote{\textbf{Rit.}}
E sarò \[G]pane, e sarò \[D]vino,
nella mia \[E-]vita, nelle tue \[B-]mani.
Ti accoglie\[C]rò dentro di \[D]me,
farò di \[E-]me un'offerta \[C]viva,
un sacri\[A-]ficio \[D] gradito a \[E-]Te,
\[C] un sacri\[A-]ficio \[D] gradito a \[G]Te.
\endchorus





%%%%%% EV. INTERMEZZO
\beginverse*
\vspace*{1.3\versesep}
{
	\nolyrics
	\musicnote{Chiusura strumentale}
	
	\ifchorded

	%---- Prima riga -----------------------------
	\vspace*{-\versesep}
	\[C] \[G]  \[C]	 \[*G] 


	\fi
	%---- Ev Indicazioni -------------------------			
	%\textnote{\textit{(ripetizione della strofa)}} 
	 
}
\vspace*{\versesep}
\endverse


\endsong
%------------------------------------------------------------
%			FINE CANZONE
%------------------------------------------------------------



%GGG
%-------------------------------------------------------------
%			INIZIO	CANZONE
%-------------------------------------------------------------


%titolo: 	Acqua siamo noi
%autore: 	Cento
%tonalita: 	Re



%%%%%% TITOLO E IMPOSTAZONI
\beginsong{Giullare dei campi}[by={Canto Salesiano a Don Bosco — P. Pignatelli}] 	% <<< MODIFICA TITOLO E AUTORE
\transpose{0} 						% <<< TRASPOSIZIONE #TONI (0 nullo)
\momenti{Congedo; Santi}							% <<< INSERISCI MOMENTI	
% momenti vanno separati da ; e vanno scelti tra:
% Ingresso; Atto penitenziale; Acclamazione al Vangelo; Dopo il Vangelo; Offertorio; Comunione; Ringraziamento; Fine; Santi; Pasqua; Avvento; Natale; Quaresima; Canti Mariani; Battesimo; Prima Comunione; Cresima; Matrimonio; Meditazione;
\ifchorded
	%\textnote{Tonalità originale }	% <<< EV COMMENTI (tonalità originale/migliore)
\fi



%%%%%% INTRODUZIONE
\ifchorded
\vspace*{\versesep}
\musicnote{
\begin{minipage}{0.48\textwidth}
\textbf{Intro}
\hfill 
%( \eighthnote \, 80)   % <<  MODIFICA IL TEMPO
% Metronomo: \eighthnote (ottavo) \quarternote (quarto) \halfnote (due quarti)
\end{minipage}
} 	
\vspace*{-\versesep}
\beginverse*
\nolyrics

%---- Prima riga -----------------------------
\vspace*{-\versesep}
\[C] \[A-]  \[F]	\[C] % \[*D] per indicare le pennate, \rep{2} le ripetizioni

%---- Ogni riga successiva -------------------
%\vspace*{-\versesep}
%\[G] \[C]  \[D]	

%---- Ev Indicazioni -------------------------			
%\textnote{\textit{(Come la prima riga)} }	

\endverse
\fi



%%%%% STROFA
\beginverse		%Oppure \beginverse* se non si vuole il numero di fianco
\memorize 		% <<< DECOMMENTA se si vuole utilizzarne la funzione
%\chordsoff		& <<< DECOMMENTA se vuoi una strofa senza accordi

Cal\[C]zoni colore del \[A-]prato, 
un ginocchio amma\[F]ccato 
per un salto in \[C]più, 
due pi\[G7]ante un filo ti\[A-]rato,
la mela sul \[F]naso e gli amici \[G7]giù. 
Un \[D-]pezzo di pane e una \[G7]fetta di cielo, 
sa\[C]pore di festa e \[A-]tu: 
Gio\[F]vanni dei Becchi giul\[C]lare dei campi 
re\[G7]galo alla gioven\[C]tù.

\endverse



%%%%% RITORNELLO
\textnote{\textbf{Rit.}}
\beginchorus

Siete tutti \[F]ladri ragazzi mi\[G7]ei, 
non ho più il mio \[C]cuore ce l’avete \[A-]voi! 
Ma non m’inte\[F]ressa da quest’oggi in \[G7]poi 
ogni mio res\[F]piro sarà per \[C]voi. \rep{2}

\endchorus



%%%%% STROFA
\beginverse
La ^veste color della st^rada 
forse un pò consu^mata,
qualche acciacco in ^più, 
nei ^prati intorno a Val^docco 
ti chiama don ^Bosco la tua gioven^tù. 
La ^vecchia tettoia e una ^piccola stanza
fra ^spiagge infinite in ^cuor, 
un ^fischio per Corso Re^gina, uno sguardo 
pro^fondo sentono l’a^more!

\endverse



%%%%% STROFA
\beginverse
%\chordsoff
Un ^eco color della ^storia, 
tesoro dei ^campi 
che oggi non è ^più, 
il ^vecchio pilone del ^sogno, 
il ragazzo sul ^filo non esiste ^più. 
L’an^tica fontana del ^grande cortile 
non ^getta più acqua e ^tu... 
as^petti qualcuno che ^ancora racconti 
l’a^more alla gioven^tù. 
\endverse




\endsong
%------------------------------------------------------------
%			FINE CANZONE
%------------------------------------------------------------
%HHH
%-------------------------------------------------------------
%			INIZIO	CANZONE
%-------------------------------------------------------------


%titolo: 	Holy is the Lord
%autore: 	Chris Tomlin
%tonalita: 	Sol 



%%%%%% TITOLO E IMPOSTAZONI
\beginsong{Holy is the Lord}[by={C. Tomlin, L. Giglio}] 	% <<< MODIFICA TITOLO E AUTORE
\transpose{-3} 						% <<< TRASPOSIZIONE #TONI (0 nullo)
\momenti{Natale}							% <<< INSERISCI MOMENTI
\ifchorded
	\textnote{Tonalità migliore per le bambine }
\fi


%%%%%% INTRODUZIONE
\ifchorded
\vspace*{\versesep}
\textnote{
\begin{minipage}{0.48\textwidth}
Intro:
\hfill 
%(\quarternote 80)   % <<  MODIFICA IL TEMPO
% Metronomo: \eighthnote (ottavo) \quarternote (quarto) \halfnote (due quarti)
\end{minipage}
} 	
\vspace*{-\versesep}
\beginverse*

\nolyrics

%---- Prima riga -----------------------------
\vspace*{-\versesep}
\[G] \[C]  \[D]	 \rep{2}

%---- Ogni riga successiva -------------------
%\vspace*{-\versesep}
%\[G] \[C]  \[D]	

%---- Ev Indicazioni -------------------------			
%\textnote{\textit{(Oppure tutta la strofa)} }	

\endverse
\fi



%%%%% STROFA
\beginverse		%Oppure \beginverse* se non si vuole il numero di fianco
%\memorize 		% <<< DECOMMENTA se si vuole utilizzarne la funzione
%\chordsoff		& <<< DECOMMENTA se vuoi una strofa senza accordi

We \[G]stand and \[C]lift up our \[D]hands,
for the \[E-]joy of the \[C]Lord is our str\[D]enght.
\[G]We bow do\[C]wn and wor\[D]ship Him now,
how \[E-]great how \[C]awesome is \[D]He.

\endverse


%%%%% RITORNELLO
\textnote{\textbf{Rit.}}
\beginchorus

Holy is the \[G]Lord
\[C]God al\[D]mighty.
The \[E-]Earth is \[C]filled
with His \[D]glory. \rep{2} 

\endchorus



%%%%%% EV. INTERMEZZO
\beginverse*
\vspace*{1.3\versesep}
{
	\nolyrics
	\textnote{Intermezzo strumentale}
	
	\ifchorded

	%---- Prima riga -----------------------------
	\vspace*{-\versesep}
	\[G] \[C]  \[D]	 \rep{2}




	\fi
	%---- Ev Indicazioni -------------------------			
	\musicnote{e si riprende dalla strofa} 
	 
}
\vspace*{\versesep}
\endverse



%%%%%% FINALE

\beginchorus
\vspace*{1.3\versesep}
\textnote{Finale:}
The \[E-]Earth is \[C]filled
with His \[D]glory.
\textnote{\textit{rallentando}}
The \[E-]Earth is \[C]filled
with His \[D]glory.
The \[E-]Earth is \[C]filled
with His \[D]glo-o-ory.
\endchorus



\endsong
%------------------------------------------------------------
%			FINE CANZONE
%------------------------------------------------------------


%++++++++++++++++++++++++++++++++++++++++++++++++++++++++++++
%			CANZONE TRASPOSTA
%++++++++++++++++++++++++++++++++++++++++++++++++++++++++++++
\ifchorded
%decremento contatore per avere stesso numero
\addtocounter{songnum}{-1} 
\beginsong{Holy is the Lord}[by={C. Tomlin, L. Giglio}] 	% <<< MODIFICA TITOLO E AUTORE
\transpose{0} 						% <<< TRASPOSIZIONE #TONI + - (0 nullo)
\ifchorded
	\textnote{Tonalità originale}	% <<< EV COMMENTI (tonalità originale/migliore)
\fi



%%%%%% INTRODUZIONE
\ifchorded
\vspace*{\versesep}
\textnote{
\begin{minipage}{0.48\textwidth}
Intro:
\hfill 
%(\quarternote 80)   % <<  MODIFICA IL TEMPO
% Metronomo: \eighthnote (ottavo) \quarternote (quarto) \halfnote (due quarti)
\end{minipage}
} 	
\vspace*{-\versesep}
\beginverse*

\nolyrics

%---- Prima riga -----------------------------
\vspace*{-\versesep}
\[G] \[C]  \[D]	 \rep{2}

%---- Ogni riga successiva -------------------
%\vspace*{-\versesep}
%\[G] \[C]  \[D]	

%---- Ev Indicazioni -------------------------			
%\textnote{\textit{(Oppure tutta la strofa)} }	

\endverse
\fi



%%%%% STROFA
\beginverse		%Oppure \beginverse* se non si vuole il numero di fianco
%\memorize 		% <<< DECOMMENTA se si vuole utilizzarne la funzione
%\chordsoff		& <<< DECOMMENTA se vuoi una strofa senza accordi

We \[G]stand and \[C]lift up our \[D]hands,
for the \[E-]joy of the \[C]Lord is our str\[D]enght.
\[G]We bow do\[C]wn and wor\[D]ship Him now,
how \[E-]great how \[C]awesome is \[D]He.

\endverse


%%%%% RITORNELLO
\textnote{\textbf{Rit.}}
\beginchorus

Holy is the \[G]Lord
\[C]God al\[D]mighty.
The \[E-]Earth is \[C]filled
with His \[D]glory. \rep{2} 

\endchorus



%%%%%% EV. INTERMEZZO
\beginverse*
\vspace*{1.3\versesep}
{
	\nolyrics
	\textnote{Intermezzo strumentale}
	
	\ifchorded

	%---- Prima riga -----------------------------
	\vspace*{-\versesep}
	\[G] \[C]  \[D]	 \rep{2}




	\fi
	%---- Ev Indicazioni -------------------------			
	\musicnote{e si riprende dalla strofa} 
	 
}
\vspace*{\versesep}
\endverse



%%%%%% FINALE

\beginchorus
\vspace*{1.3\versesep}
\textnote{Finale:}
The \[E-]Earth is \[C]filled
with His \[D]glory.
\textnote{\textit{rallentando}}
The \[E-]Earth is \[C]filled
with His \[D]glory.
The \[E-]Earth is \[C]filled
with His \[D]glo-o-ory.
\endchorus

\endsong
\fi
%++++++++++++++++++++++++++++++++++++++++++++++++++++++++++++
%			FINE CANZONE TRASPOSTA
%++++++++++++++++++++++++++++++++++++++++++++++++++++++++++++



%III
%JJJ
%KKK
%LLL
%-------------------------------------------------------------
%			INIZIO	CANZONE
%-------------------------------------------------------------


%titolo: 	La Tue meraviglie
%autore: 	Casucci, Balduzzi
%tonalita: 	Fa 



%%%%%% TITOLO E IMPOSTAZONI
\beginsong{Le tue meraviglie}[by={Casucci, Balduzzi}]
\transpose{0} 						% <<< TRASPOSIZIONE #TONI (0 nullo)
\momenti{Fine; Ringraziamento}							% <<< INSERISCI MOMENTI	
% momenti vanno separati da ; e vanno scelti tra:
% Ingresso; Atto penitenziale; Acclamazione al Vangelo; Dopo il Vangelo; Offertorio; Comunione; Ringraziamento; Fine; Santi; Pasqua; Avvento; Natale; Quaresima; Canti Mariani; Battesimo; Prima Comunione; Cresima; Matrimonio; Meditazione;
\ifchorded
	%\textnote{Tonalità originale }	% <<< EV COMMENTI (tonalità originale/migliore)
\fi



%%%%%% INTRODUZIONE
\ifchorded
\vspace*{\versesep}
\textnote{Intro: \qquad \qquad  }%(\eighthnote 116) % << MODIFICA IL TEMPO
% Metronomo: \eighthnote (ottavo) \quarternote (quarto) \halfnote (due quarti)
\vspace*{-\versesep}
\beginverse*

\nolyrics

%---- Prima riga -----------------------------
\vspace*{-\versesep}
\[A-] \[E-] \[F] \[C] \[D-] \[A-] \[F] \[G]	 % \[*D] per indicare le pennate, \rep{2} le ripetizioni

%---- Ogni riga successiva -------------------
\vspace*{-\versesep}
\[A-] \[E-] \[F] \[C] \[D-] \[A-] \[F] \[*G] \[A-] 

%---- Ev Indicazioni -------------------------			
%\textnote{\textit{(Oppure tutta la strofa)} }	

\endverse



%%%%% RITORNELLO
\beginchorus
\textnote{\textbf{Rit.}}
Ora \[F]lascia, o Si\[G]gnore, che io \[E-]vada in pa\[A-]ce,
perché ho \[D-]visto le tue \[C]mera\[B&]vi\[G]glie.
Il tuo \[F]popolo in \[G]festa per le \[E-]strade corre\[A-]rà
a por\[D-]tare le tue \[C]mera\[B&]vi\[G]glie!
\endchorus




%%%%% STROFA
\beginverse		%Oppure \beginverse* se non si vuole il numero di fianco
\memorize 		% <<< DECOMMENTA se si vuole utilizzarne la funzione
%\chordsoff		% <<< DECOMMENTA se vuoi una strofa senza accordi
\musicnote{\textit{Da piano a forte in un crescendo}}
\[A-]La tua pre\[E-]senza ha riem\[F]pito d'a\[C]more
\[A-]le nostre \[E-]vite, le \[F]nostre gior\[C]nate.
\[B&]In te una sola \[F]anima, \[G-]un solo cuore \[F]siamo noi:
\[B&]con te la luce ri\[F]splende, \brk \[G-]splende più chiara che \[C]mai!
\endverse



%%%%% STROFA
\beginverse

^La tua pre^senza ha inon^dato d'a^more
^le nostre ^vite, le ^nostre gior^nate,
^fra la tua gente ^resterai, \brk ^per sempre vivo in ^mezzo a noi
^fino ai confini del ^tempo: ^così ci accompagne^rai.

\endverse




%%%%% RITORNELLO
\beginchorus
\textnote{\textbf{Rit.}}
Ora \[F]lascia, o Si\[G]gnore, che io \[E-]vada in pa\[A-]ce,
perché ho \[D-]visto le tue \[C]mera\[B&]vi\[G]glie.
Il tuo \[F]popolo in \[G]festa per le \[E-]strade corre\[A-]rà
a por\[D-]tare le tue \[C]mera\[B&]vi\[G]glie!
Ora \[F]lascia, o Si\[G]gnore, che io \[E-]vada in pa\[A-]ce,
perché ho \[F]visto le \[G]tue mera\[E-]vi\[A-]glie.
E il tuo \[F]popolo in \[G]festa per le \[E-]strade corre\[A-]rà
a por\[F]tare le \[G]tue mera\[F]vi\[C]glie!
\endchorus





%%%%%% EV. FINALE
\ifchorded
\beginverse*
\vspace*{1.3\versesep}
{
	\nolyrics
	\textnote{Finale strumentale}
	
	
	%---- Prima riga -----------------------------
	\vspace*{-\versesep}
	\[A-] \[E-] \[F] \[C] \[D-] \[A-] \[F] \[G] 

	%---- Ogni riga successiva -------------------
	\vspace*{-\versesep}
	\[A-] \[E-] \[F] \[C] \[D-] \[A-] \[F] \[*G] \[A-]

}	
\vspace*{\versesep}
\endverse
\fi



\endsong
%------------------------------------------------------------
%			FINE CANZONE
%------------------------------------------------------------
%MMM
%-------------------------------------------------------------
%			INIZIO	CANZONE
%-------------------------------------------------------------


%titolo: 	Maranathà
%autore: 	Buttazzo, Ricci
%tonalita: 	Sim



%%%%%% TITOLO E IMPOSTAZONI
\beginsong{Maranathà}[by={F. Buttazzo, D. Ricci}] 	% <<< MODIFICA TITOLO E AUTORE
\transpose{0} 						% <<< TRASPOSIZIONE #TONI (0 nullo)
\momenti{Ingresso; Avvento; Canti Mariani; Congedo}							% <<< INSERISCI MOMENTI	
% momenti vanno separati da ; e vanno scelti tra:
% Ingresso; Atto penitenziale; Acclamazione al Vangelo; Dopo il Vangelo; Offertorio; Comunione; Ringraziamento; Fine; Santi; Pasqua; Avvento; Natale; Quaresima; Canti Mariani; Battesimo; Prima Comunione; Cresima; Matrimonio; Meditazione;
\ifchorded
	%\textnote{Tonalità originale }	% <<< EV COMMENTI (tonalità originale/migliore)
\fi



%%%%%% INTRODUZIONE
\ifchorded
\vspace*{\versesep}
\musicnote{
\begin{minipage}{0.48\textwidth}
\textbf{Intro}
\hfill 
%( \eighthnote \, 80)   % <<  MODIFICA IL TEMPO
% Metronomo: \eighthnote (ottavo) \quarternote (quarto) \halfnote (due quarti)
\end{minipage}
} 	
\vspace*{-\versesep}
\beginverse*

\nolyrics

%---- Prima riga -----------------------------
\vspace*{-\versesep}
\[(D)] \[B-]  \[F#-] \[G] \[A] \[D] % \[*D] per indicare le pennate, \rep{2} le ripetizioni

%---- Ogni riga successiva -------------------
%\vspace*{-\versesep}
%\[G] \[C]  \[D]	

%---- Ev Indicazioni -------------------------			
\textnote{\textit{(come metà ritornello)} }	

\endverse
\fi



%%%%% RITORNELLO
\textnote{\textbf{Rit.}}
\beginchorus
\[D]Marana\[B-]thà, vieni Si\[F#-]gnor!
Verso te, Ge\[G]sù, le mani \[A]noi levi\[D]am.
\[D]Marana\[B-]thà, vieni Si\[F#-]gnor!
Prendici con \[G]te e salva\[A]ci Si\[D]gnor.
\endchorus


%%%%% STROFA
\beginverse		%Oppure \beginverse* se non si vuole il numero di fianco
\memorize 		% <<< DECOMMENTA se si vuole utilizzarne la funzione
%\chordsoff		% <<< DECOMMENTA se vuoi una strofa senza accordi

\[D]Guardo verso le mon\[A]tagne, 
\echo{Guardo verso le mon\[D]tagne}
\[(D)]donde mi verrà il soc\[A]corso, 
\echo{donde mi verrà il soc\[D]corso}
\[(D)]il soccorso vien da \[A]Dio, 
\echo{il soccorso vien da \[D]Dio}
\[(D)]che ha creato il mondo in\[E-]tero. 
\echo{che ha creato il mondo in\[D]tero}
\endverse



%%%%% STROFE
\beginverse	
\chordsoff
Sorgi con il tuo Amore, la Tua luce splenderà,
ogni ombra svanirà, la tua Gloria apparirà.	
\endverse
\beginverse
\chordsoff	
Santo è nostro Signor, il peccato Egli levò,
dalla morte ci salvò, e la vita a noi donò.
\endverse
\beginverse
\chordsoff	
Mio Signor son peccatore, a Te apro il mio cuore,
fa’ di me quello che vuoi e per sempre in Te vivrò.
\endverse
\beginverse
\chordsoff	
La Parola giungerà sino ad ogni estremità,
testimoni noi sarem della tua verità.
\endverse
\beginverse
\chordsoff	
Tu sei la mia libertà, solo in Te potrò sperar,
ho fiducia in te Signor, la mia vita cambierai.	
\endverse
\beginverse
\chordsoff	
Mi consegno a te Signor, vieni dentro il mio cuor,
ti ricevo o Salvator, tu sei il mio liberator.
\endverse
\beginverse
\chordsoff	
Benedicici o Signor, sii custode ai nostri cuor,
giorno e notte veglierai, e con noi sempre sarai.
\endverse
\beginverse
\chordsoff	
Ringraziamo te o Signor, a te Padre Creator,
allo Spirito d’Amor, vieni presto, o Signor.
\endverse



\endsong
%NNN
%-------------------------------------------------------------
%			INIZIO	CANZONE
%-------------------------------------------------------------


%titolo: 	Noi saremo il pane
%autore: 	Fusco
%tonalita: 	Do 



%%%%%% TITOLO E IMPOSTAZONI
\beginsong{Noi saremo il pane}[by={M. G. Fusco}] 	% <<< MODIFICA TITOLO E AUTORE
\transpose{0} 						% <<< TRASPOSIZIONE #TONI (0 nullo)
\momenti{Offertorio; Prima Comunione; Spezzare del Pane;}							% <<< INSERISCI MOMENTI	
% momenti vanno separati da ; e vanno scelti tra:
% Ingresso; Atto penitenziale; Acclamazione al Vangelo; Dopo il Vangelo; Offertorio; Comunione; Ringraziamento; Fine; Spezzare del Pane; Santi; Pasqua; Avvento; Natale; Quaresima; Canti Mariani; Battesimo; Prima Comunione; Cresima; Matrimonio; Meditazione;
\ifchorded
	%\textnote{Tonalità originale }	% <<< EV COMMENTI (tonalità originale/migliore)
\fi


%%%%%% INTRODUZIONE
\ifchorded
\vspace*{\versesep}
\musicnote{
\begin{minipage}{0.48\textwidth}
\textbf{Intro}
\hfill 
%( \eighthnote \, 80)   % <<  MODIFICA IL TEMPO
% Metronomo: \eighthnote (ottavo) \quarternote (quarto) \halfnote (due quarti)
\end{minipage}
} 	
\vspace*{-\versesep}
\beginverse*

\nolyrics

%---- Prima riga -----------------------------
\vspace*{-\versesep}
\[C] \[F]  \[C]	 \rep{2} % \[*D] per indicare le pennate, \rep{2} le ripetizioni

%---- Ogni riga successiva -------------------
%\vspace*{-\versesep}
%\[G] \[C]  \[D]	

%---- Ev Indicazioni -------------------------			
%\textnote{\textit{(Oppure tutta la strofa)} }	

\endverse
\fi




%%%%% STROFA
\beginverse		%Oppure \beginverse* se non si vuole il numero di fianco
\memorize 		% <<< DECOMMENTA se si vuole utilizzarne la funzione
%\chordsoff		& <<< DECOMMENTA se vuoi una strofa senza accordi

Un \[C]chicco da \[F]solo che \[C]fa?
Non fa un \[F]campo di grano né un \[G]pane!
Un \[C]chicco da \[F]solo non po\[C]trà
esser la \[F]gioia di chi ha \[G]fame!
\[D-7]Ma uniti in\[E-]sieme
tanti \[D-7]chicchi un solo \[G]pane!

\endverse




%%%%% RITORNELLO
\beginchorus
\textnote{\textbf{Rit.}}

\[C]Noi saremo il \[E-]pane, \[F]noi sarem l'a\[E-]more
\[F]noi sarem la \[G]gioia per un \[A-]mondo che ha
fame d'infi\[D-]ni\[G]to!
\[C]Noi saremo il \[E-]pane, \[F]noi sarem l'a\[E-]more
\[F]noi sarem la \[G]gioia per un \[A-]mondo che ha
\[G]fame di \[C]Te!

\endchorus



%%%%%% EV. INTERMEZZO
\beginverse*
\vspace*{1.3\versesep}
{
	\nolyrics
	\textnote{Intermezzo strumentale}
	
	\ifchorded

	%---- Prima riga -----------------------------
	\vspace*{-\versesep}
	\[C] \[F]  \[C]	 

	\fi
	%---- Ev Indicazioni -------------------------			
	%\textnote{\textit{(ripetizione della strofa)}} 
	 
}
\vspace*{\versesep}
\endverse




%%%%% STROFA
\beginverse		%Oppure \beginverse* se non si vuole il numero di fianco
%\memorize 		% <<< DECOMMENTA se si vuole utilizzarne la funzione
\chordsoff		% <<< DECOMMENTA se vuoi una strofa senza accordi

Un ^acino ^solo che ^fa?
Non è ^uva che matura sui ^colli!
Un ^uomo ^solo non po^trà
essere ^"segno" dell'a^more
^ma noi invi^tati
tutti in^sieme Chiesa ^viva!

\endverse






\endsong
%------------------------------------------------------------
%			FINE CANZONE
%------------------------------------------------------------


%OOO

%-------------------------------------------------------------
%			INIZIO	CANZONE
%-------------------------------------------------------------


%titolo: 	Ogni mia parola
%autore: 	Gen Verde
%tonalita: 	Do 



%%%%%% TITOLO E IMPOSTAZONI
\beginsong{Ogni mia parola}[ititle={Come la pioggia e la neve},by={Gen Verde}] 	% <<< MODIFICA TITOLO E AUTORE
\transpose{0} 						% <<< TRASPOSIZIONE #TONI (0 nullo)
\momenti{Offertorio; Dopo il Vangelo}							% <<< INSERISCI MOMENTI	
% momenti vanno separati da ; e vanno scelti tra:
% Ingresso; Atto penitenziale; Acclamazione al Vangelo; Dopo il Vangelo; Offertorio; Comunione; Ringraziamento; Fine; Santi; Pasqua; Avvento; Natale; Quaresima; Canti Mariani; Battesimo; Prima Comunione; Cresima; Matrimonio; Meditazione;
\ifchorded
	%\textnote{Tonalità originale }	% <<< EV COMMENTI (tonalità originale/migliore)
\fi

%%%%%% INTRODUZIONE
\ifchorded
\vspace*{\versesep}
\textnote{Intro: \qquad \qquad (\eighthnote 138) }% % << MODIFICA IL TEMPO
% Metronomo: \eighthnote (ottavo) \quarternote (quarto) \halfnote (due quarti)
\vspace*{-\versesep}
\beginverse*

\nolyrics

%---- Prima riga -----------------------------
\vspace*{-\versesep}
\[C] \[G]  \[C]	\[G] \rep{2}% \[*D] per indicare le pennate, \rep{2} le ripetizioni

%---- Ogni riga successiva -------------------
%\vspace*{-\versesep}
%\[G] \[C]  \[D]	

%---- Ev Indicazioni -------------------------			
\textnote{\textit{(Molto dolce, arpeggiato)} }	

\endverse
\fi

%%%%% STROFA
\beginverse*	%Oppure \beginverse* se non si vuole il numero di fianco
%\memorize 		% <<< DECOMMENTA se si vuole utilizzarne la funzione
%\chordsoff		& <<< DECOMMENTA se vuoi una strofa senza accordi

\[C]Come la \[G]pioggia e la \[C]ne\[G]ve
\[C]scendono \[F]giù dal \[G]cielo
e \[A-]non vi ri\[G]tornano \[F]senza irri\[C]gare
e \[F]far germo\[G]gliare la \[F]ter\[G]ra, \quad \[G]
\endverse
\beginverse*
\[C]così ogni mia Pa\[F]rola non ri\[C]tornerà a \[G]me
\[C]senza operare \[F]quanto de\[G]side\[G7]ro,
\[A-]senza aver compiuto \[G]ciò per cui l'a\[F]vevo man\[C]data.
\[F]Ogni mia pa\[G]rola, \[F]ogni mia pa\[G]rola
\[F]Ogni mia pa\[G]rola, \[F]ogni mia pa\[G]rola  \quad \[G]

\endverse



%%%%%% EV. INTERMEZZO
\beginverse*
\vspace*{1.3\versesep}
{
	\nolyrics
	\musicnote{Chiusura strumentale}
	
	
	\ifchorded

	%---- Prima riga -----------------------------
	\vspace*{-\versesep}
	\[C] \[G]  \[C] \[G]

	%---- Ogni riga successiva -------------------
	\vspace*{-\versesep}
	\[C] \[G]  \[C] \[*G] \textit{(Finisce sospeso in SOL)}


	\fi
	%---- Ev Indicazioni -------------------------			
	%\textnote{\textit{(ripetizione della strofa)}} 
	 
}
\vspace*{\versesep}
\endverse

\endsong
%PPP
%-------------------------------------------------------------
%			INIZIO	CANZONE
%-------------------------------------------------------------


%titolo: 	Pace sia, pace a voi
%autore: 	Gen Verde, Gen Rosso
%tonalita: 	Mi 



%%%%%% TITOLO E IMPOSTAZONI
\beginsong{Pace sia, pace a voi}[by={Gen Verde, Gen Rosso}] 	% <<< MODIFICA TITOLO E AUTORE
\transpose{0} 						% <<< TRASPOSIZIONE #TONI (0 nullo)
\momenti{Offertorio}							% <<< INSERISCI MOMENTI	
% momenti vanno separati da ; e vanno scelti tra:
% Ingresso; Atto penitenziale; Acclamazione al Vangelo; Dopo il Vangelo; Offertorio; Comunione; Ringraziamento; Fine; Santi; Pasqua; Avvento; Natale; Quaresima; Canti Mariani; Battesimo; Prima Comunione; Cresima; Matrimonio; Meditazione;
\ifchorded
	%\textnote{Tonalità originale }	% <<< EV COMMENTI (tonalità originale/migliore)
\fi

%%%%%% INTRODUZIONE
\ifchorded
\vspace*{\versesep}
\musicnote{
\begin{minipage}{0.48\textwidth}
\textbf{Intro}
\hfill 
%( \eighthnote \, 80)   % <<  MODIFICA IL TEMPO
% Metronomo: \eighthnote (ottavo) \quarternote (quarto) \halfnote (due quarti)
\end{minipage}
} 	
\vspace*{-\versesep}
\beginverse*

\nolyrics

%---- Prima riga -----------------------------
\vspace*{-\versesep}
\[E] \[A] \[E] \[B] \[E] \[A] \[E]	 % \[*D] per indicare le pennate, \rep{2} le ripetizioni

%---- Ogni riga successiva -------------------
%\vspace*{-\versesep}
%\[G] \[C]  \[D]	

%---- Ev Indicazioni -------------------------			
%\textnote{\textit{(Oppure tutta la strofa)} }	

\endverse
\fi

%%%%% RITORNELLO
\beginchorus
\textnote{\textbf{Rit.}}

“Pace \[E]sia, pace a voi”: la tua \[A]pace sarà
sulla \[C#-]terra com'è nei \[B]cieli.
“Pace \[E]sia, pace a voi”: la tua \[A]pace sarà
gioia \[G]nei nostri \[D]occhi, nei \[A]cuo\[B]ri.
“Pace \[E]sia, pace a voi”: la tua \[A]pace sarà
luce \[C#-]limpida nei pen\[B]sieri.
“Pace \[E]sia, pace a voi”: la tua \[A]pace sarà
una \[E]casa per \[B]tutti. \[E]\[A]\[E]

\endchorus

%%%%% STROFA
\beginverse		%Oppure \beginverse* se non si vuole il numero di fianco
\memorize 		% <<< DECOMMENTA se si vuole utilizzarne la funzione
%\chordsoff		% <<< DECOMMENTA se vuoi una strofa senza accordi

“\[A]Pace a \[E]voi”: sia il tuo \[B]dono vi\[C#-]sibile.
“\[A]Pace a \[E]voi”: la tua e\[B]redi\[C#-]tà.
“\[A]Pace a \[E]voi”: come un \[B]canto all'u\[C#-]nisono
che \[D]sale dalle nostre cit\[B]tà.

\endverse

%%%%% STROFA
\beginverse		%Oppure \beginverse* se non si vuole il numero di fianco
%\memorize 		% <<< DECOMMENTA se si vuole utilizzarne la funzione
%\chordsoff		% <<< DECOMMENTA se vuoi una strofa senza accordi

“^Pace a ^voi”: sia un im^pronta nei ^secoli.
“^Pace a ^voi”: segno d'^uni^tà.
“^Pace a ^voi”: sia l'ab^braccio tra i ^popoli,
la ^tua promessa all'umani^tà.

\endverse

\endsong
%------------------------------------------------------------
%			FINE CANZONE
%------------------------------------------------------------

%QQQ
%RRR
%-------------------------------------------------------------
%			INIZIO	CANZONE
%-------------------------------------------------------------


%titolo: 	Resta accanto a me
%autore: 	Gen Verde
%tonalita: 	Mi 



%%%%%% TITOLO E IMPOSTAZONI
\beginsong{Resta accanto a me}[by={Gen Verde}] 	% <<< MODIFICA TITOLO E AUTORE
\transpose{0} 						% <<< TRASPOSIZIONE #TONI (0 nullo)
\momenti{Ringraziamento; Fine}							% <<< INSERISCI MOMENTI	
% momenti vanno separati da ; e vanno scelti tra:
% Ingresso; Atto penitenziale; Acclamazione al Vangelo; Dopo il Vangelo; Offertorio; Comunione; Ringraziamento; Fine; Santi; Pasqua; Avvento; Natale; Quaresima; Canti Mariani; Battesimo; Prima Comunione; Cresima; Matrimonio; Meditazione;
\ifchorded
	%\textnote{Tonalità originale }	% <<< EV COMMENTI (tonalità originale/migliore)
\fi

%%%%%% INTRODUZIONE
\ifchorded
\vspace*{\versesep}
\textnote{Intro: \qquad \qquad  }%(\eighthnote 116) % << MODIFICA IL TEMPO
% Metronomo: \eighthnote (ottavo) \quarternote (quarto) \halfnote (due quarti)
\vspace*{-\versesep}
\beginverse*

\nolyrics

%---- Prima riga -----------------------------
\vspace*{-\versesep}
\[A] \[E]  \[B]	 % \[*D] per indicare le pennate, \rep{2} le ripetizioni

%---- Ogni riga successiva -------------------
%\vspace*{-\versesep}
%\[G] \[C]  \[D]	

%---- Ev Indicazioni -------------------------			
%\textnote{\textit{(Oppure tutta la strofa)} }	

\endverse
\fi

%%%%% RITORNELLO
\beginchorus
\textnote{\textbf{Rit.}}

\[E]Ora \[B]vado \[A]sulla mia \[E]strada
\[F#-]con l'a\[G#-]more \[A]tuo che mi \[B]guida.
\[E]O Si\[B]gnore, o\[A]vunque io \[E]vada,
\[A]resta ac\[B]canto a \[E]me.
\[E]Io ti \[B]prego, \[A]stammi vi\[E]cino
\[F#-]ogni \[G#-]passo \[A]del mio cam\[B]mino,
\[E]ogni \[B]notte, \[A]ogni mat\[E]tino,
\[A]resta ac\[B]canto a \[E]me.

\endchorus

%%%%% STROFA
\beginverse		%Oppure \beginverse* se non si vuole il numero di fianco
%\memorize 		% <<< DECOMMENTA se si vuole utilizzarne la funzione
%\chordsoff		& <<< DECOMMENTA se vuoi una strofa senza accordi

\[B]Il tuo sguardo \[A]puro sia luce per \[C#-]me
\[B]e la tua Pa\[A]rola sia voce per \[E]me.
\[A]Che io trovi il \[B]senso del mio andare
\[C#-]so\[B]lo in \[E]te, nel \[B]tuo fedele amare il mio per\[E]ché.

\endverse

%%%%% RITORNELLO
\beginchorus
\textnote{\textbf{Rit.}}

\[E]Ora \[B]vado \[A]sulla mia \[E]strada
\[F#-]con l'a\[G#-]more \[A]tuo che mi \[B]guida.
\[E]O Si\[B]gnore, o\[A]vunque io \[E]vada,
\[A]resta ac\[B]canto a \[E]me.
\[E]Io ti \[B]prego, \[A]stammi vi\[E]cino
\[F#-]ogni \[G#-]passo \[A]del mio cam\[B]mino,
\[E]ogni \[B]notte, \[A]ogni mat\[E]tino,
\[A]resta ac\[B]canto a \[E]me.

\endchorus

%%%%% STROFA
\beginverse		%Oppure \beginverse* se non si vuole il numero di fianco
%\memorize 		% <<< DECOMMENTA se si vuole utilizzarne la funzione
%\chordsoff		& <<< DECOMMENTA se vuoi una strofa senza accordi

^Fa' che chi mi ^guarda non veda che ^te.
^Fa' che chi mi a^scolta non senta che ^te,
^e chi pensa a ^me, fa' che nel cuore
^pen^si a ^te e ^trovi quell'amore \brk che hai dato a ^me.
\endverse

%%%%% RITORNELLO
\beginchorus
\textnote{\textbf{Rit.}}

\[E]Ora \[B]vado \[A]sulla mia \[E]strada
\[F#-]con l'a\[G#-]more \[A]tuo che mi \[B]guida.
\[E]O Si\[B]gnore, o\[A]vunque io \[E]vada,
\[A]resta ac\[B]canto a \[E]me.
\[E]Io ti \[B]prego, \[A]stammi vi\[E]cino
\[F#-]ogni \[G#-]passo \[A]del mio cam\[B]mino,
\[E]ogni \[B]notte, \[A]ogni mat\[E]tino,
\[A]resta ac\[B]canto a \[E]me.

%%%%%% EV. FINALE

\beginchorus %oppure \beginverse*
\vspace*{1.3\versesep}
%\textnote{Finale \textit{(rallentando)}} %<<< EV. INDICAZIONI

\[E]Ora \[B]vado \[A]sulla mia \[E]strada
\[A]resta ac\[B]canto a \[E]me.

\endchorus  %oppure \endverse




\endsong
%------------------------------------------------------------
%			FINE CANZONE
%------------------------------------------------------------



%SSS
%-------------------------------------------------------------
%			INIZIO	CANZONE
%-------------------------------------------------------------


%titolo: 	Segni del tuo amore
%autore: 	Gen Verde, Gen rosso
%tonalita: 	Do

\beginsong{Segni del Tuo amore}[by={Gen\ Verde, Gen\ Rosso}]
\transpose{0} 						% <<< TRASPOSIZIONE #TONI (0 nullo)
\momenti{Offertorio; Prima Comunione}							% <<< INSERISCI MOMENTI	
% momenti vanno separati da ; e vanno scelti tra:
% Ingresso; Atto penitenziale; Acclamazione al Vangelo; Dopo il Vangelo; Offertorio; Comunione; Ringraziamento; Fine; Santi; Pasqua; Avvento; Natale; Quaresima; Canti Mariani; Battesimo; Prima Comunione; Cresima; Matrimonio; Meditazione;
\ifchorded
	%\textnote{Tonalità originale }	% <<< EV COMMENTI (tonalità originale/migliore)
\fi



%%%%%% INTRODUZIONE
\ifchorded
\vspace*{\versesep}
\textnote{Intro: \qquad \qquad  (\quarternote 118)}% % << MODIFICA IL TEMPO
% Metronomo: \eighthnote (ottavo) \quarternote (quarto) \halfnote (due quarti)
\vspace*{-\versesep}
\beginverse*

\nolyrics

%---- Prima riga -----------------------------
\vspace*{-\versesep}
\[C] \[C]  \[C] \[D-]	 % \[*D] per indicare le pennate, \rep{2} le ripetizioni

%---- Ogni riga successiva -------------------
\vspace*{-\versesep}
\[C] \[C]  \[*D-] \[*C] \[D-]	

%---- Ev Indicazioni -------------------------			
%\textnote{\textit{(Oppure tutta la strofa)} }	

\endverse
\fi


%%%%% STROFA
\beginverse
\memorize

\[C]Mille e mille grani nelle 
spighe \[D-]d'o\[C]ro \quad \[*D-] \[*C] \[D-]
\[C]mandano fragranza e danno 
gioia al \[D-]cuo\[C]re, \quad \[*D-] \[*C] \[D-]
\[C]quando, macinati, fanno un 
pane \[D-]so\[C]lo, \quad \[*D-] \[*C] \[D-]
\[C]pane quotidiano, dono tuo, 
Si\[D-]gno\[C]re. \quad \[*D-] \[*C] \[C]

\endverse



%%%%% RITORNELLO
\beginchorus
\textnote{\textbf{Rit.}}

\[G]Ecco il pane e il vino, segni del tuo a\[F]mo\[C]re.
\[G]Ecco questa offerta, accoglila Si\[F]gno\[C]re,
\[F]tu di mille e mille \[G]cuori fai un cuore \[C]solo,
un corpo solo in \[G]te
e il \[F]Figlio tuo verrà, vi\[G]vrà 
ancora in mezzo a \[C]noi.

\endchorus




%%%%%% EV. INTERMEZZO
\beginverse*
\vspace*{1.3\versesep}
{
	\nolyrics
	\textnote{Breve intermezzo strumentale}
	
	\ifchorded

	%---- Prima riga -----------------------------
	\vspace*{-\versesep}
	\[(C)] \[C]  \[C] \[D-]	 % \[*D] per indicare le pennate, \rep{2} le ripetizioni

	%---- Ogni riga successiva -------------------
	\vspace*{-\versesep}
	\[C] \[C]  \[*D-] \[*C] \[D-]	


	\fi
	%---- Ev Indicazioni -------------------------			
	%\textnote{\textit{(ripetizione della strofa)}} 
	 
}
\vspace*{\versesep}
\endverse




%%%%% STROFA
\beginverse

^Mille grappoli maturi 
sotto il ^so^le, \quad ^ ^ ^
^festa della terra donano 
vi^go^re, \quad ^ ^ ^
^quando da ogni perla stilla 
il vino ^nuo^vo, \quad ^ ^ ^
^vino della gioia, dono tuo, 
Si^gno^re. \quad ^ ^ ^

\endverse


%%%%% RITORNELLO
\beginchorus
\textnote{\textbf{Rit.}}

\[G]Ecco il pane e il vino, segni del tuo a\[F]mo\[C]re.
\[G]Ecco questa offerta, accoglila Si\[F]gno\[C]re,
\[F]tu di mille e mille \[G]cuori fai un cuore \[C]solo,
un corpo solo in \[G]te
e il \[F]Figlio tuo verrà, vi\[G]vrà 
ancora in mezzo a \[F]noi.		\quad \[C]  \iflyric \rep{2} \fi
\ifchorded
\vspace*{2\versesep}
\[G]Ecco il pane e il vino, segni del tuo a\[F]mo\[C]re.
\[G]Ecco questa offerta, accoglila Si\[F]gno\[C]re,
\[F]tu di mille e mille \[G]cuori fai un cuore \[C]solo,
un corpo solo in \[G]te
e il \[F]Figlio tuo verrà, vi\[G]vrà 
ancora in mezzo a \[C]noi.
\fi
\endchorus



%%%%%% EV. INTERMEZZO
\beginverse*
\vspace*{1.3\versesep}
{
	\nolyrics
	\musicnote{Finale}
	
	\ifchorded

	%---- Prima riga -----------------------------
	\vspace*{-\versesep}
	\[(C)] \[C]  \[C] \[D-]	 % \[*D] per indicare le pennate, \rep{2} le ripetizioni

	%---- Ogni riga successiva -------------------
	\vspace*{-\versesep}
	\[C] \[C]  \[*D-] \[*C] \[D-]	\quad \[C]


	\fi
	%---- Ev Indicazioni -------------------------			
	%\textnote{\textit{(ripetizione della strofa)}} 
	 
}
\vspace*{\versesep}
\endverse



\endsong
%------------------------------------------------------------
%			FINE CANZONE
%------------------------------------------------------------


\input{su_ali_d_acquila}
%TTT
%UUU
%VVV
%-------------------------------------------------------------
%			INIZIO	CANZONE
%-------------------------------------------------------------


%titolo: 	Verbum panis
%autore: 	Casucci, Balduzzi
%tonalita: 	Mim 



%%%%%% TITOLO E IMPOSTAZONI
\beginsong{Verbum panis}[by={Casucci, Balduzzi}] 	% <<< MODIFICA TITOLO E AUTORE
\transpose{0} 						% <<< TRASPOSIZIONE #TONI (0 nullo)
\momenti{Comunione; Prima Comunione; Avvento}							% <<< INSERISCI MOMENTI	
% momenti vanno separati da ; e vanno scelti tra:
% Ingresso; Atto penitenziale; Acclamazione al Vangelo; Dopo il Vangelo; Offertorio; Comunione; Ringraziamento; Fine; Santi; Pasqua; Avvento; Natale; Quaresima; Canti Mariani; Battesimo; Prima Comunione; Cresima; Matrimonio; Meditazione;
\ifchorded
	%\textnote{Tonalità originale }	% <<< EV COMMENTI (tonalità originale/migliore)
\fi







%%%%%% INTRODUZIONE
\ifchorded
\vspace*{\versesep}
\textnote{Intro: \qquad \qquad  }%(\eighthnote 116) % << MODIFICA IL TEMPO
% Metronomo: \eighthnote (ottavo) \quarternote (quarto) \halfnote (due quarti)
\vspace*{-\versesep}
\beginverse*

\nolyrics

%---- Prima riga -----------------------------
\vspace*{-\versesep}
\[E-] \[D]  \[E-] \[D] \[E-] \[D]	 % \[*D] per indicare le pennate, \rep{2} le ripetizioni

%---- Ogni riga successiva -------------------
\vspace*{-\versesep}
\[*C] \[*D]  \[E-]	

%---- Ev Indicazioni -------------------------			
\textnote{\textit{(arpeggiando)} }	

\endverse
\fi


%%%%% STROFA
\beginverse
\memorize
\[E-]Prima del \[D]tempo
prima an\[E-]cora che la \[D]terra
comin\[E-]ciasse a vive\[D]re 
\[E-]il Verbo \[D]era presso \[E-]Dio. \[D]\[E-]\[D]
\[E-]Venne nel \[D]mondo
e per \[E-]non abbando\[D]narci
in questo \[E-]viaggio ci la\[D]sciò
\[E-]tutto sé \[D]stesso come \[E-]pane. \[D] 
\endverse



\beginverse*
\[E-]Verbum \[E-]caro factum est \[E-]
Verbum \[E-]panis factum est \[E-]
Verbum \[E-]caro factum est \[E-]
Verbum \[E-]panis factum \[C7+]est. \[C]\[D]\[D]
\endverse


\beginchorus
\[G]Qui \[D]spezzi ancora il \[C]pane in mezzo a \[D]noi
e chi\[G]unque mange\[D]rà \[C]non avrà più \[D]fame.
\[G]Qui \[D]vive la tua \[C]chiesa intorno a \[D]te
dove o\[G]gnuno trove\[D]rà \brk \[C]la sua vera \[D]casa. 
\endchorus


\beginverse*
\[E-]Verbum \[E-]caro factum est \[E-]
Verbum \[E-]panis factum est \[E-]
Verbum \[E-]caro factum est \[E-]
Verbum \[E-]panis 
\endverse



%%%%% STROFA
\beginverse
^Prima del ^tempo
quando l'^universo ^fu creato
^dall'oscuri^tà
^il Verbo ^era presso ^Dio. ^^^
^Venne nel ^mondo
nella ^sua miseri^cordia
Dio ha man^dato il Figlio ^suo
^tutto sé ^stesso come ^pane. ^ 
\endverse


\beginverse*
\[E-]Verbum \[E-]caro factum est \[E-]
Verbum \[E-]panis factum est \[E-]
Verbum \[E-]caro factum est \[E-]
Verbum \[E-]panis factum \[C7+]est. \[C7+]\[D4]\[D]
\endverse


\beginchorus
\[G]Qui \[D]spezzi ancora il \[C]pane in mezzo a \[D]noi
e chi\[G]unque mange\[D]rà \[C]non avrà più \[D]fame.
\[G]Qui \[D]vive la tua \[C]chiesa intorno a \[D]te
dove o\[G]gnuno trove\[D]rà \brk \[C]la sua vera \[D]casa. \rep{2}
\endchorus



\beginverse*
\[E-]Verbum caro factum est \[E-]
Verbum \[E-]panis factum est \[E-]
Verbum \[E-]caro factum est \[E-]
Verbum \[E-]panis factum \[E-]est. 
\endverse



\endsong
%------------------------------------------------------------
%			FINE CANZONE
%------------------------------------------------------------


%-------------------------------------------------------------
%			INIZIO	CANZONE
%-------------------------------------------------------------


%titolo: 	Voglio esaltare
%autore: 	Giampiero Colombo
%tonalita: 	MIm e REm 



%%%%%% TITOLO E IMPOSTAZONI
\beginsong{Voglio esaltare}[by={Giampiero Colombo}] 	% <<< MODIFICA TITOLO E AUTORE
\transpose{-2} 						% <<< TRASPOSIZIONE #TONI (0 nullo)
\preferflats
\momenti{Ingresso; Avvento}							% <<< INSERISCI MOMENTI	
% momenti vanno separati da ; e vanno scelti tra:
% Ingresso; Atto penitenziale; Acclamazione al Vangelo; Dopo il Vangelo; Offertorio; Comunione; Ringraziamento; Fine; Santi; Pasqua; Avvento; Natale; Quaresima; Canti Mariani; Battesimo; Prima Comunione; Cresima; Matrimonio; Meditazione;
\ifchorded
	\textnote{Tonalità migliore per le bambine }	% <<< EV COMMENTI (tonalità originale/migliore)
\fi


%%%%%% INTRODUZIONE
\ifchorded
\vspace*{\versesep}
\textnote{Intro: \qquad \qquad  }%(\eighthnote 116) % << MODIFICA IL TEMPO
% Metronomo: \eighthnote (ottavo) \quarternote (quarto) \halfnote (due quarti)
\vspace*{-\versesep}
\beginverse*

\nolyrics

%---- Prima riga -----------------------------
\vspace*{-\versesep}
\[E-] \[B-] \[E-] \[B-]  	 % \[*D] per indicare le pennate, \rep{2} le ripetizioni

%---- Ogni riga successiva -------------------
%\vspace*{-\versesep}
%\[G] \[C]  \[D]	

%---- Ev Indicazioni -------------------------			
%\textnote{\textit{(Oppure tutta la strofa)} }	

\endverse
\fi




%%%%% STROFA
\beginverse		%Oppure \beginverse* se non si vuole il numero di fianco
\memorize 		% <<< DECOMMENTA se si vuole utilizzarne la funzione
%\chordsoff		% <<< DECOMMENTA se vuoi una strofa senza accordi

\[E-]Voglio esal\[B-]tare \brk il \[E-]nome del Dio \[B-]nostro: 
\[C]è Lui la mia \[D]liber\[G]tà! \[B7]
\[E-]Ecco il ma\[B-]ttino, \brk \[E-]gioia di sal\[B-]vezza, 
un \[C]canto sta nas\[D]cendo in \[E-]noi.

\endverse




%%%%% RITORNELLO
\beginchorus
\textnote{\textbf{Rit.}}

\[A-]Vieni, o Si\[D7]gnore, \[G7]luce del cam\[C7]mino,
\[A-]fuoco che nel \[B]cuore 
ac\[E-]cen\[D]de il \[G]"sì". \[*E7]
\[A-]Lieto il tuo pas\[D7]saggio, \[G7]ritmi la spe\[C7]ranza,
\[A-]Padre della \[B7]veri\[E-]tà. \[B-] \[E-] \[B-]

\endchorus




%%%%% STROFA
\beginverse		%Oppure \beginverse* se non si vuole il numero di fianco
%\memorize 		% <<< DECOMMENTA se si vuole utilizzarne la funzione
%\chordsoff		% <<< DECOMMENTA se vuoi una strofa senza accordi
^Voglio esal^tare \brk il ^nome del Dio ^nostro
^grande nella ^fedel^tà! ^
^Egli mi ha ^posto \brk ^sull'alto suo ^monte.
^Roccia che non ^crolla ^mai.

\endverse


%%%%% STROFA
\beginverse		%Oppure \beginverse* se non si vuole il numero di fianco
%\memorize 		% <<< DECOMMENTA se si vuole utilizzarne la funzione
\chordsoff		% <<< DECOMMENTA se vuoi una strofa senza accordi

Voglio Annunciare il dono crocifisso
di Cristo, il Dio con noi!
Perchè della morte lui si prende gioco,
Figlio che ci attira a sé!

\endverse


\endsong
%------------------------------------------------------------
%			FINE CANZONE
%------------------------------------------------------------




% %++++++++++++++++++++++++++++++++++++++++++++++++++++++++++++
% %			CANZONE TRASPOSTA
% %++++++++++++++++++++++++++++++++++++++++++++++++++++++++++++
% \ifchorded
% %decremento contatore per avere stesso numero
% \addtocounter{songnum}{-1} 
% \beginsong{Voglio esaltare}[by={Giampiero Colombo}]  	% <<< COPIA TITOLO E AUTORE
% \transpose{0} 						% <<< TRASPOSIZIONE #TONI + - (0 nullo)
% \ifchorded
% 	\textnote{Tonalità alternativa}	% <<< EV COMMENTI (tonalità originale/migliore)
% \fi



% %%%%%% INTRODUZIONE
% \ifchorded
% \vspace*{\versesep}
% \textnote{Intro: \qquad \qquad  }%(\eighthnote 116) % << MODIFICA IL TEMPO
% % Metronomo: \eighthnote (ottavo) \quarternote (quarto) \halfnote (due quarti)
% \vspace*{-\versesep}
% \beginverse*

% \nolyrics

% %---- Prima riga -----------------------------
% \vspace*{-\versesep}
% \[E-] \[B-] \[E-] \[B-]  	 % \[*D] per indicare le pennate, \rep{2} le ripetizioni

% %---- Ogni riga successiva -------------------
% %\vspace*{-\versesep}
% %\[G] \[C]  \[D]	

% %---- Ev Indicazioni -------------------------			
% %\textnote{\textit{(Oppure tutta la strofa)} }	

% \endverse
% \fi




% %%%%% STROFA
% \beginverse		%Oppure \beginverse* se non si vuole il numero di fianco
% \memorize 		% <<< DECOMMENTA se si vuole utilizzarne la funzione
% %\chordsoff		% <<< DECOMMENTA se vuoi una strofa senza accordi

% \[E-]Voglio esal\[B-]tare \brk il \[E-]nome del Dio \[B-]nostro: 
% \[C]è Lui la mia \[D]liber\[G]tà! \[B7]
% \[E-]Ecco il ma\[B-]ttino, \brk \[E-]gioia di sal\[B-]vezza, 
% un \[C]canto sta nas\[D]cendo in \[E-]noi.

% \endverse




% %%%%% RITORNELLO
% \beginchorus
% \textnote{\textbf{Rit.}}

% \[A-]Vieni, o Si\[D7]gnore, \[G7]luce del cam\[C7]mino,
% \[A-]fuoco che nel \[B]cuore 
% ac\[E-]cen\[D]de il \[G]"sì". \[*E7]
% \[A-]Lieto il tuo pas\[D7]saggio, \[G7]ritmi la spe\[C7]ranza,
% \[A-]Padre della \[B7]veri\[E-]tà. \[B-] \[E-] \[B-]

% \endchorus




% %%%%% STROFA
% \beginverse		%Oppure \beginverse* se non si vuole il numero di fianco
% %\memorize 		% <<< DECOMMENTA se si vuole utilizzarne la funzione
% %\chordsoff		% <<< DECOMMENTA se vuoi una strofa senza accordi
% ^Voglio esal^tare \brk il ^nome del Dio ^nostro
% ^grande nella ^fedel^tà! ^
% ^Egli mi ha ^posto \brk ^sull'alto suo ^monte.
% ^Roccia che non ^crolla ^mai.

% \endverse


% %%%%% STROFA
% \beginverse		%Oppure \beginverse* se non si vuole il numero di fianco
% %\memorize 		% <<< DECOMMENTA se si vuole utilizzarne la funzione
% \chordsoff		% <<< DECOMMENTA se vuoi una strofa senza accordi

% Voglio Annunciare il dono crocifisso
% di Cristo, il Dio con noi!
% Perchè della morte lui si prende gioco,
% Figlio che ci attira a sé!

% \endverse


% \endsong

% \fi
% %++++++++++++++++++++++++++++++++++++++++++++++++++++++++++++
% %			FINE CANZONE TRASPOSTA
% %++++++++++++++++++++++++++++++++++++++++++++++++++++++++++++

%WWW
%XXX
%YYY
%ZZZ





%******* END SONGS ENVIRONMENT ******
\setcounter{GlobalSongCounter}{\thesongnum}
\end{songs}


\songchapter{Taizè}
%...............................................................................
%
% ████████╗ █████╗ ██╗███████╗███████╗    
% ╚══██╔══╝██╔══██╗██║╚══███╔╝██╔════╝    
%    ██║   ███████║██║  ███╔╝ █████╗      
%    ██║   ██╔══██║██║ ███╔╝  ██╔══╝      
%    ██║   ██║  ██║██║███████╗███████╗    
%    ╚═╝   ╚═╝  ╚═╝╚═╝╚══════╝╚══════╝ 
% Font ANSI Shadow
%..............................................................................
\begin{songs}{}
\songcolumns{\canzsongcolumsnumber}
\setcounter{songnum}{\theGlobalSongCounter} %set songnum counter, otherwise would be reset

%set the default path inside current folder
\makeatletter
\def\input@path{{Songs/Taize/}}
\makeatother


%***** INSERT SONGS HERE ******

%-------------------------------------------------------------
%			INIZIO	CANZONE
%-------------------------------------------------------------


%titolo: 	Bless the Lord my soul
%autore: 	Taizé
%tonalita: 	Si- 




%titolo{Bless the lord my soul}
%autore{Taizé}
%album{}
%tonalita{Si-}
%famiglia{Liturgica}
%gruppo{Canoni_ritornelli}
%momenti{Ritornelli;Taizé}
%identificatore{bless_the_lord}
%data_revisione{2011_12_31}
%trascrittore{Francesco Endrici - Manuel Toniato}
\beginsong{Bless the lord my soul}[by={Taizé}]
\transpose{0} 						% <<< TRASPOSIZIONE #TONI (0 nullo)
\momenti{Dopo il Vangelo; Meditazione}				% <<< INSERISCI MOMENTI	
\ifchorded
	%\textnote{Tonalità originale }	% <<< EV COMMENTI (tonalità originale/migliore)
\fi


%%%%%% INTRODUZIONE
\ifchorded
\vspace*{\versesep}
\textnote{Intro: \qquad \qquad  }%(\eighthnote 116) % << MODIFICA IL TEMPO
% Metronomo: \eighthnote (ottavo) \quarternote (quarto) \halfnote (due quarti)
\vspace*{-\versesep}
\beginverse*

\nolyrics

%---- Prima riga -----------------------------
\vspace*{-\versesep}
\[B-] \[E]  \[B-]	 % \[*D] per indicare le pennate, \rep{2} le ripetizioni


%---- Ev Indicazioni -------------------------			
\textnote{\textit{(Oppure tutta la strofa)} }	

\endverse
\fi



%%%%% STROFA
\beginverse*
\[B-]Bless the \[E]Lord, my \[B-]soul, \brk and \[G]bless God's \[A]holy \[D]name. \[F#] 
\[B-]Bless the \[E]Lord, my \[B-]soul, \brk who \[G]leads \[A7]me into \[B-]life.
\endverse
\endsong


%-------------------------------------------------------------
%			INIZIO	CANZONE
%-------------------------------------------------------------


%titolo: 	Bonum est confidere
%autore: 	Taizè
%tonalita: 	La-



%%%%%% TITOLO E IMPOSTAZONI
\beginsong{Bonum est confidere}[by={Taizè}]	% <<< MODIFICA TITOLO E AUTORE
%\transpose{-2} 						% <<< TRASPOSIZIONE #TONI (0 nullo)
\momenti{Ringraziamento; Meditazione}							% <<< INSERISCI MOMENTI	
% momenti vanno separati da ; e vanno scelti tra:
% Ingresso; Atto penitenziale; Acclamazione al Vangelo; Dopo il Vangelo; Offertorio; Comunione; Ringraziamento; Fine; Santi; Pasqua; Avvento; Natale; Quaresima; Canti Mariani; Battesimo; Prima Comunione; Cresima; Matrimonio; Meditazione;
\ifchorded
	%\textnote{Tonalità originale }	% <<< EV COMMENTI (tonalità originale/migliore)
\fi


%%%%%% INTRODUZIONE
\ifchorded
\vspace*{\versesep}
\musicnote{
\begin{minipage}{0.48\textwidth}
\textbf{Intro}
\hfill 
%( \eighthnote \, 80)   % <<  MODIFICA IL TEMPO
% Metronomo: \eighthnote (ottavo) \quarternote (quarto) \halfnote (due quarti)
\end{minipage}
} 	
\vspace*{-\versesep}
\beginverse*

\nolyrics

%---- Prima riga -----------------------------
\vspace*{-\versesep}
\[D-]

%---- Ogni riga successiva -------------------
%\vspace*{-\versesep}
%\[G] \[C]  \[D]	

%---- Ev Indicazioni -------------------------			
\textnote{\textit{(oppure tutta la strofa)} }	

\endverse
\fi

\beginverse*
\[D-]Bonum est con\[A]fidere in \[D-]{Do}\[C]{mi}\[F]no
\[G-]Bonum spe\[D-]rare in \[G-]{Do}\[A]{mi}\[D-]no.
\endverse
\endsong


%-------------------------------------------------------------
%			INIZIO	CANZONE
%-------------------------------------------------------------

%titolo: Christe lux mundi
%autore: Taizè
%tonalita: Rem


%%%%%% TITOLO E IMPOSTAZONI
\beginsong{Christe lux mundi}[by={Taizé}] 	% <<< MODIFICA TITOLO E AUTORE
\transpose{0} 						% <<< TRASPOSIZIONE #TONI (0 nullo)
\momenti{Acclamazione al Vangelo; Meditazione}							% <<< INSERISCI MOMENTI	
% momenti vanno separati da ; e vanno scelti tra:
% Ingresso; Atto penitenziale; Acclamazione al Vangelo; Dopo il Vangelo; Offertorio; Comunione; Ringraziamento; Fine; Santi; Pasqua; Avvento; Natale; Quaresima; Canti Mariani; Battesimo; Prima Comunione; Cresima; Matrimonio; Meditazione;


%%%%%% INTRODUZIONE
\ifchorded
\vspace*{\versesep}
\textnote{Intro: \qquad \qquad  }%(\eighthnote 116) % << MODIFICA IL TEMPO(\eighthnote 88) 
% Metronomo: \eighthnote (ottavo) \quarternote (quarto) \halfnote (due quarti)
\vspace*{-\versesep}
\beginverse*

\nolyrics

%---- Prima riga -----------------------------
\vspace*{-\versesep}
\[D-] 	 % \[*D] per indicare le pennate, \rep{2} le ripetizioni

%---- Ogni riga successiva -------------------
%\vspace*{-\versesep}
%\[G] \[C]  \[D]	

%---- Ev Indicazioni -------------------------			
\textnote{\textit{(Oppure tutta la strofa)} }	

\endverse
\fi




%%%%% STROFA
\beginverse*
\[D-]Christe, lux \[A-]mun\[D-]di, 
\[D-]qui sequi\[G-]tur\[A-]te, 
ha\[F]bebit \[(*D-)]lumen \[C]vitae, \quad \[G-]lumen \[D-]vi\[C]tae...
\endverse



\endsong
%------------------------------------------------------------
%			FINE CANZONE
%------------------------------------------------------------
%-------------------------------------------------------------
%			INIZIO	CANZONE
%-------------------------------------------------------------


%titolo: 	Il Signore è la mia forza
%autore: 	Taizè
%tonalita: 	Re 



%%%%%% TITOLO E IMPOSTAZONI
\beginsong{Il Signore è la mia forza}[by={Taizè}] 	% <<< MODIFICA TITOLO E AUTORE
%\transpose{-3} 						% <<< TRASPOSIZIONE #TONI (0 nullo)
\momenti{Dopo il Vangelo; Meditazione}							% <<< INSERISCI MOMENTI	
% momenti vanno separati da ; e vanno scelti tra:
% Ingresso; Atto penitenziale; Acclamazione al Vangelo; Dopo il Vangelo; Offertorio; Comunione; Ringraziamento; Fine; Santi; Pasqua; Avvento; Natale; Quaresima; Canti Mariani; Battesimo; Prima Comunione; Cresima; Matrimonio; Meditazione;
\ifchorded
	%\textnote{Tonalità originale }	% <<< EV COMMENTI (tonalità originale/migliore)
\fi


%%%%%% INTRODUZIONE
\ifchorded
\vspace*{\versesep}
\textnote{Intro: \qquad \qquad  }%(\eighthnote 116) % << MODIFICA IL TEMPO
% Metronomo: \eighthnote (ottavo) \quarternote (quarto) \halfnote (due quarti)
\vspace*{-\versesep}
\beginverse*

\nolyrics

%---- Prima riga -----------------------------
\vspace*{-\versesep}
\[F]

%---- Ogni riga successiva -------------------
%\vspace*{-\versesep}
%\[G] \[C]  \[D]	

%---- Ev Indicazioni -------------------------			
\textnote{\textit{(oppure tutta la strofa)} }	

\endverse
\fi


\beginverse*
\[D-*]Il \[(C*)]Si\[F]gnore è la mia \[B&*]for\[C]za 
 \[D-*]ed \[(C*)]io \[F]spero in \[C]lui.
Il Si\[B&]gnore è il \[A]Salva\[D-]tor.
In \[C]lui con\[F]fido, non \[B&]ho ti\[C]mor, 
in \[A-]lui con\[D-]fido non \[B&*]ho \[(C*)]ti\[F]mor.
\endverse
\endsong
%------------------------------------------------------------
%			FINE CANZONE
%------------------------------------------------------------



%-------------------------------------------------------------
%			INIZIO	CANZONE
%-------------------------------------------------------------

%titolo: Laudate omnes gentes
%autore: Taizè
%tonalita: Re


%%%%%% TITOLO E IMPOSTAZONI
\beginsong{Laudate omnes gentes}[by={Taizè}] 	% <<< MODIFICA TITOLO E AUTORE
\transpose{0} 						% <<< TRASPOSIZIONE #TONI (0 nullo)
\momenti{Meditazione; Ringraziamento}							% <<< INSERISCI MOMENTI	
% momenti vanno separati da ; e vanno scelti tra:
% Ingresso; Atto penitenziale; Acclamazione al Vangelo; Dopo il Vangelo; Offertorio; Comunione; Ringraziamento; Fine; Santi; Pasqua; Avvento; Natale; Quaresima; Canti Mariani; Battesimo; Prima Comunione; Cresima; Matrimonio; Meditazione;


%%%%%% INTRODUZIONE
\ifchorded
\vspace*{\versesep}
\musicnote{
\begin{minipage}{0.48\textwidth}
\textbf{Intro}
\hfill 
%( \eighthnote \, 80)   % <<  MODIFICA IL TEMPO
% Metronomo: \eighthnote (ottavo) \quarternote (quarto) \halfnote (due quarti)
\end{minipage}
} 	
\vspace*{-\versesep}
\beginverse*

\nolyrics

%---- Prima riga -----------------------------
\vspace*{-\versesep}
\[D] 	 % \[*D] per indicare le pennate, \rep{2} le ripetizioni

%---- Ogni riga successiva -------------------
%\vspace*{-\versesep}
%\[G] \[C]  \[D]	

%---- Ev Indicazioni -------------------------			
\textnote{\textit{(oppure tutta la strofa)} }	

\endverse
\fi




%%%%% STROFA
\beginverse*
\[D]Laudate, \[G]omnes \[A]gen\[B-]tes, \quad \[(F#*)]lau\[B-]date \[D]Domi\[A]num.
\[D]Laudate, \[G]omnes \[A]gen\[B-]tes, \quad lau\[E-]date \[A4]Domi\[D]num.
\endverse



\endsong
%------------------------------------------------------------
%			FINE CANZONE
%------------------------------------------------------------
%-------------------------------------------------------------
%			INIZIO	CANZONE
%-------------------------------------------------------------

%titolo: Misericordias DOmini
%autore: Taizè
%tonalita: Rem


%%%%%% TITOLO E IMPOSTAZONI
\beginsong{Misericordias Domini}[by={Taizè}] 	% <<< MODIFICA TITOLO E AUTORE
\transpose{0} 						% <<< TRASPOSIZIONE #TONI (0 nullo)
\momenti{Meditazione}							% <<< INSERISCI MOMENTI	
% momenti vanno separati da ; e vanno scelti tra:
% Ingresso; Atto penitenziale; Acclamazione al Vangelo; Dopo il Vangelo; Offertorio; Comunione; Ringraziamento; Fine; Santi; Pasqua; Avvento; Natale; Quaresima; Canti Mariani; Battesimo; Prima Comunione; Cresima; Matrimonio; Meditazione;


%%%%%% INTRODUZIONE
\ifchorded
\vspace*{\versesep}
\musicnote{
\begin{minipage}{0.48\textwidth}
\textbf{Intro}
\hfill 
%( \eighthnote \, 80)   % <<  MODIFICA IL TEMPO
% Metronomo: \eighthnote (ottavo) \quarternote (quarto) \halfnote (due quarti)
\end{minipage}
} 	
\vspace*{-\versesep}
\beginverse*

\nolyrics

%---- Prima riga -----------------------------
\vspace*{-\versesep}
\[D-] 	 % \[*D] per indicare le pennate, \rep{2} le ripetizioni

%---- Ogni riga successiva -------------------
%\vspace*{-\versesep}
%\[G] \[C]  \[D]	

%---- Ev Indicazioni -------------------------			
\textnote{\textit{(oppure tutta la strofa)} }	

\endverse
\fi




%%%%% STROFA
\beginverse*
\[D-]Miseri\[A]cordias \[D-]Domi\[C]ni, 
\[F]in ae\[C]ternum \[D-]can\[A]ta\[D-]bo.
\endverse



\endsong
%------------------------------------------------------------
%			FINE CANZONE
%------------------------------------------------------------
%-------------------------------------------------------------
%			INIZIO	CANZONE
%-------------------------------------------------------------


%titolo: 	Nada te turbe
%autore: 	Taizè
%tonalita: 	La-



%%%%%% TITOLO E IMPOSTAZONI
\beginsong{Nada te turbe}[by={Taizè}]	% <<< MODIFICA TITOLO E AUTORE
\transpose{-2} 						% <<< TRASPOSIZIONE #TONI (0 nullo)
\momenti{Ringraziamento; Meditazione}							% <<< INSERISCI MOMENTI	
% momenti vanno separati da ; e vanno scelti tra:
% Ingresso; Atto penitenziale; Acclamazione al Vangelo; Dopo il Vangelo; Offertorio; Comunione; Ringraziamento; Fine; Santi; Pasqua; Avvento; Natale; Quaresima; Canti Mariani; Battesimo; Prima Comunione; Cresima; Matrimonio; Meditazione;
\ifchorded
	%\textnote{Tonalità originale }	% <<< EV COMMENTI (tonalità originale/migliore)
\fi


%%%%%% INTRODUZIONE
\ifchorded
\vspace*{\versesep}
\textnote{Intro: \qquad \qquad  }%(\eighthnote 116) % << MODIFICA IL TEMPO
% Metronomo: \eighthnote (ottavo) \quarternote (quarto) \halfnote (due quarti)
\vspace*{-\versesep}
\beginverse*

\nolyrics

%---- Prima riga -----------------------------
\vspace*{-\versesep}
\[A-]

%---- Ogni riga successiva -------------------
%\vspace*{-\versesep}
%\[G] \[C]  \[D]	

%---- Ev Indicazioni -------------------------			
\textnote{\textit{(Oppure tutta la strofa)} }	

\endverse
\fi



%%%%% STROFA
\beginverse
\memorize 
\[A-]Nada te \[D-7]turbe, \[G]nada te e\[C]spante,
\[F]quien a Dios \[D-6]tiene \[E]nada la \[A-]falta.
\[A-]Nada te \[D-7]turbe, \[G]nada te e\[C]spante,
\[F]solo \[D-6]Dios \[E]ba\[A-]sta.
\endverse

%%%%% STROFA
\beginverse		%Oppure \beginverse* se non si vuole il numero di fianco
%\memorize 		% <<< DECOMMENTA se si vuole utilizzarne la funzione
%\chordsoff		% <<< DECOMMENTA se vuoi una strofa senza accordi


^Niente ti ^turbi, ^niente ti spa^venti:
^chi ha ^Dio, ^niente gli ^manca.
^Niente ti ^turbi, ^niente ti spa^venti:
^solo ^Dio ^bas^ta.

\endverse

\endsong
%------------------------------------------------------------
%			FINE CANZONE
%------------------------------------------------------------
%-------------------------------------------------------------
%			INIZIO	CANZONE
%-------------------------------------------------------------

%titolo: Questa Notte
%autore: Taizè
%tonalita: Lam TRASPOSTA IN Sim


%%%%%% TITOLO E IMPOSTAZONI
\beginsong{Questa Notte}[by={Taizè}] 	% <<< MODIFICA TITOLO E AUTORE
\transpose{2} 						% <<< TRASPOSIZIONE #TONI (0 nullo)
\momenti{Meditazione; Ringraziamento}							% <<< INSERISCI MOMENTI	
% momenti vanno separati da ; e vanno scelti tra:
% Ingresso; Atto penitenziale; Acclamazione al Vangelo; Dopo il Vangelo; Offertorio; Comunione; Ringraziamento; Fine; Santi; Pasqua; Avvento; Natale; Quaresima; Canti Mariani; Battesimo; Prima Comunione; Cresima; Matrimonio; Meditazione;


%%%%%% INTRODUZIONE
\ifchorded
\vspace*{\versesep}
\musicnote{
\begin{minipage}{0.48\textwidth}
\textbf{Intro}
\hfill 
( \eighthnote \, 88)   % <<  MODIFICA IL TEMPO
% Metronomo: \eighthnote (ottavo) \quarternote (quarto) \halfnote (due quarti)
\end{minipage}
} 	
\vspace*{-\versesep}
\beginverse*
\nolyrics

%---- Prima riga -----------------------------
\vspace*{-\versesep}
\[A-] 	 % \[*D] per indicare le pennate, \rep{2} le ripetizioni

%---- Ogni riga successiva -------------------
%\vspace*{-\versesep}
%\[G] \[C]  \[D]	

%---- Ev Indicazioni -------------------------			
%\textnote{\textit{(Oppure tutta la strofa)} }	

\endverse
\fi




%%%%% STROFA
\beginverse*
Questa \[A-]notte non \[G]è più \[C]notte da\[D-]vanti \[E]te:
il \[A-*]buio \[G*]come \[C]luce ri\[D-]sple\[E]nde.
\endverse


%%%%%% EV. CHIUSURA SOLO STRUMENTALE
\ifchorded
\beginchorus %oppure \beginverse*
\vspace*{1.3\versesep}
\textnote{Chiusura } %<<< EV. INDICAZIONI

\[A-]

\endchorus  %oppure \endverse
\fi


\endsong
%------------------------------------------------------------
%			FINE CANZONE
%------------------------------------------------------------





%******* END SONGS ENVIRONMENT ******
\setcounter{GlobalSongCounter}{\thesongnum}
\end{songs}


\songchapter{Canti tradizionali di Natale}
%...............................................................................
% ███╗   ██╗ █████╗ ████████╗ █████╗ ██╗     ███████╗
% ████╗  ██║██╔══██╗╚══██╔══╝██╔══██╗██║     ██╔════╝
% ██╔██╗ ██║███████║   ██║   ███████║██║     █████╗  
% ██║╚██╗██║██╔══██║   ██║   ██╔══██║██║     ██╔══╝  
% ██║ ╚████║██║  ██║   ██║   ██║  ██║███████╗███████╗
% ╚═╝  ╚═══╝╚═╝  ╚═╝   ╚═╝   ╚═╝  ╚═╝╚══════╝╚══════╝
% Font: ANSI Shadow                                                                               
%...............................................................................
\begin{songs}{}
\songcolumns{\canzsongcolumsnumber}
\setcounter{songnum}{\theGlobalSongCounter} %set songnum counter, otherwise would be reset

%set the default path inside current folder
\makeatletter
\def\input@path{{Songs/Natale/}}
\makeatother


%***** INSERT SONGS HERE ******


%-------------------------------------------------------------
%			INIZIO	CANZONE
%-------------------------------------------------------------


%titolo: 	Gioia al mondo
%autore: 	G. F. Handel
%tonalita: 	Do 



%%%%%% TITOLO E IMPOSTAZONI
\beginsong{Gioia al mondo}[by={Joy to the world! — G. F. Handel}] 	% <<< MODIFICA TITOLO E AUTORE
\transpose{0} 						% <<< TRASPOSIZIONE #TONI (0 nullo)
\momenti{Natale}					% <<< INSERISCI MOMENTI




%%%%%% INTRODUZIONE
\ifchorded
\beginverse*
\vspace*{-0.5\versesep}
{
	\nolyrics
	Intro: \qquad \qquad % (\eighthnote 116) % << MODIFICA IL TEMPO
	\vspace*{-\versesep}

	%---- Prima riga -----------------------------
	\[C] \[C] \textoverline{\[F]} \textoverline{\[G]} \[C]	%\textbf{x2}

	%---- Ogni riga successiva -------------------
	\vspace*{-\versesep}
	\[F] \[G]  \[C]	%\textbf{x2}

	%---- Ev Indicazioni -------------------------			
	\vspace{-\versesep}\textit{(Prime due righe)} 	
	 
}
\vspace*{-0.3\versesep}
\endverse
\fi



%%%%% STROFA
\beginverse		%Oppure \beginverse* se non si vuole il numero di fianco
\memorize 		% <<< DECOMMENTA se si vuole utilizzarne la funzione
%\chordsoff		& <<< DECOMMENTA se vuoi una strofa senza accordi

\[C]Gioia al mondo il \[F]Cre\[G]a\[C]tor,
è \[F]nato in \[G]mezzo a \[C]noi!
I \[C]cuori allor si aprono
con \[C]gioia alla sua luce
e i \[C]cieli e terra sa\[G]ran 
una lode all'unico re,
che go\[F]verna ogni \[C]popolo nei \[F]se\[G]co\[C]li.

\endverse



%%%%% STROFA
\beginverse		%Oppure \beginverse* se non si vuole il numero di fianco
%\memorize 		% <<< DECOMMENTA se si vuole utilizzarne la funzione
%\chordsoff		% <<< DECOMMENTA se vuoi una strofa senza accordi

^Lui porta a noi la ^ve^ri^tà, 
il ^mondo ^salve^rà! 
La^sciate che ogni uomo 
in^vochi il salvatore 
il ^cielo splende^rà 
e la terra esulterà 
della ^grazia di^vina del ^Sal^va^tor!

\endverse



%%%%% STROFA
\beginverse		%Oppure \beginverse* se non si vuole il numero di fianco
%\memorize 		% <<< DECOMMENTA se si vuole utilizzarne la funzione
%\chordsoff		% <<< DECOMMENTA se vuoi una strofa senza accordi

^Oggi è nato il ^re^den^tor, 
la ^terra e^sulte^rà! 
La^sciate che ogni cuore 
gli ^faccia un po' di posto
il ^cielo splende^rà, 
e la terra gioirà 
della ^grazia di^vina del ^Sal^va^tor!

\endverse


Finale:

%%%%% STROFA
\beginverse*	%Oppure \beginverse* se non si vuole il numero di fianco
%\memorize 		% <<< DECOMMENTA se si vuole utilizzarne la funzione
%\chordsoff		% <<< DECOMMENTA se vuoi una strofa senza accordi

della ^grazia di^vina del ^Sal^va^tor!

\endverse



\endsong
%------------------------------------------------------------
%			FINE CANZONE
%------------------------------------------------------------
%-------------------------------------------------------------
%			INIZIO	CANZONE
%-------------------------------------------------------------

%titolo: In questa notte splendida
%autore: Sequeri
%tonalita: Mi e Sol 


%%%%%% TITOLO E IMPOSTAZONI
\beginsong{In questa notte splendida}[by={Chieffo}] 	% <<< MODIFICA TITOLO E AUTORE
\transpose{0} 						% <<< TRASPOSIZIONE #TONI (0 nullo)
\momenti{Natale}							% <<< INSERISCI MOMENTI




%%%%%% INTRODUZIONE
\ifchorded
\beginverse*
\vspace*{-0.5\versesep}
{
	\nolyrics
	%\textbf{Intro:} \qquad \qquad  (\eighthnote 150) % << MODIFICA IL TEMPO
	\vspace*{-\versesep}

	%---- Prima riga -----------------------------
	\[E] \[A]  \[E]  \[B] \[E]	%\textbf{x2}

	%---- Ev Indicazioni -------------------------			
	\vspace{-\versesep}\textit{(Ultime due righe oppure l'intera strofa)} 	
	 
}
\vspace*{-0.3\versesep}
\endverse
\fi



%%%%% STROFA
\beginverse		%Oppure \beginverse* se non si vuole il numero di fianco
\memorize 		% <<< DECOMMENTA se si vuole utilizzarne la funzione
%\chordsoff		% <<< DECOMMENTA se vuoi una strofa senza accordi

In \[E]questa notte \[A]splendida
di \[E]luce e di chia\[B]ror
il \[E]nostro cuore \[A]trepida,
è \[E]nato il \[B]Salva\[E]tor.
Un \[A]bimbo picco\[E]lissimo
le \[A]porte ci apri\[E]rà
del \[E]cielo dell'Al\[A]tissimo
nel\[E]la sua \[B]veri\[E]tà.

\endverse



%%%%% STROFA
\beginverse		%Oppure \beginverse* se non si vuole il numero di fianco
%\memorize 		% <<< DECOMMENTA se si vuole utilizzarne la funzione
%\chordsoff		% <<< DECOMMENTA se vuoi una strofa senza accordi

Svegli^atevi dal ^sonno,
corr^ete coi pas^tor,
è ^notte di mi^racoli,
di ^grazia e ^di stu^por.
A^sciuga le tue ^lacrime,
non ^piangere per^chè
Ge^sù nostro ca^rissimo 
è ^nato an^che per ^te.

\endverse



%%%%% STROFA
\beginverse		%Oppure \beginverse* se non si vuole il numero di fianco
%\memorize 		% <<< DECOMMENTA se si vuole utilizzarne la funzione
\chordsoff		% <<< DECOMMENTA se vuoi una strofa senza accordi

In questa notte limpida
di gloria e di splendor,
il nostro cuore trepida
è nato il Salvator.
Gesù nostro carissimo
le porte ci aprirà,
il figlio dell'Altissimo
con noi sempre sarà.

\endverse





\endsong
%------------------------------------------------------------
%			FINE CANZONE
%------------------------------------------------------------




%++++++++++++++++++++++++++++++++++++++++++++++++++++++++++++
%			CANZONE TRASPOSTA
%++++++++++++++++++++++++++++++++++++++++++++++++++++++++++++
\ifchorded
%decremento contatore per avere stesso numero
\addtocounter{songnum}{-1} 
\beginsong{In questa notte splendida [SOL]}[by={Chieffo}] 	% <<< MODIFICA TITOLO E AUTORE
\transpose{3} 						% <<< TRASPOSIZIONE #TONI (0 nullo)



%%%%%% INTRODUZIONE
\ifchorded
\beginverse*
\vspace*{-0.5\versesep}
{
	\nolyrics
	%\textbf{Intro:} \qquad \qquad  (\eighthnote 150) % << MODIFICA IL TEMPO
	\vspace*{-\versesep}

	%---- Prima riga -----------------------------
	\[E] \[A]  \[E]  \[B] \[E]	%\textbf{x2}

	%---- Ev Indicazioni -------------------------			
	\vspace{-\versesep}\textit{(Ultime due righe oppure l'intera strofa)} 	
	 
}
\vspace*{-0.3\versesep}
\endverse
\fi



%%%%% STROFA
\beginverse		%Oppure \beginverse* se non si vuole il numero di fianco
\memorize 		% <<< DECOMMENTA se si vuole utilizzarne la funzione
%\chordsoff		% <<< DECOMMENTA se vuoi una strofa senza accordi

In \[E]questa notte \[A]splendida
di \[E]luce e di chia\[B]ror
il \[E]nostro cuore \[A]trepida,
è \[E]nato il \[B]Salva\[E]tor.
Un \[A]bimbo picco\[E]lissimo
le \[A]porte ci apri\[E]rà
del \[E]cielo dell'Al\[A]tissimo
nel\[E]la sua \[B]veri\[E]tà.

\endverse



%%%%% STROFA
\beginverse		%Oppure \beginverse* se non si vuole il numero di fianco
%\memorize 		% <<< DECOMMENTA se si vuole utilizzarne la funzione
%\chordsoff		% <<< DECOMMENTA se vuoi una strofa senza accordi

Svegli^atevi dal ^sonno,
corr^ete coi pas^tor,
è ^notte di mi^racoli,
di ^grazia e ^di stu^por.
A^sciuga le tue ^lacrime,
non ^piangere per^chè
Ge^sù nostro ca^rissimo 
è ^nato an^che per ^te.

\endverse



%%%%% STROFA
\beginverse		%Oppure \beginverse* se non si vuole il numero di fianco
%\memorize 		% <<< DECOMMENTA se si vuole utilizzarne la funzione
\chordsoff		% <<< DECOMMENTA se vuoi una strofa senza accordi

In questa notte limpida
di gloria e di splendor,
il nostro cuore trepida
è nato il Salvator.
Gesù nostro carissimo
le porte ci aprirà,
il figlio dell'Altissimo
con noi sempre sarà.

\endverse





\endsong
\fi
%++++++++++++++++++++++++++++++++++++++++++++++++++++++++++++
%			FINE CANZONE TRASPOSTA
%++++++++++++++++++++++++++++++++++++++++++++++++++++++++++++


%++++++++++++++++++++++++++++++++++++++++++++++++++++++++++++
%			CANZONE TRASPOSTA
%++++++++++++++++++++++++++++++++++++++++++++++++++++++++++++
\ifchorded
%decremento contatore per avere stesso numero
\addtocounter{songnum}{-1} 
\beginsong{In questa notte splendida [MI-SOL-SIb]}[by={Chieffo}] 	% <<< MODIFICA TITOLO E AUTORE


%%%%%% INTRODUZIONE
\ifchorded
\beginverse*
\vspace*{-0.5\versesep}
{
	\nolyrics
	%\textbf{Intro:} \qquad \qquad  (\eighthnote 150) % << MODIFICA IL TEMPO
	\vspace*{-\versesep}

	%---- Prima riga -----------------------------
	\[E] \[A]  \[E]  \[B] \[E]	%\textbf{x2}

	%---- Ev Indicazioni -------------------------			
	\vspace{-\versesep}\textit{(Ultime due righe oppure l'intera strofa)} 	
	 
}
\vspace*{-0.3\versesep}
\endverse
\fi



%%%%% STROFA
\beginverse		%Oppure \beginverse* se non si vuole il numero di fianco
\memorize 		% <<< DECOMMENTA se si vuole utilizzarne la funzione
%\chordsoff		% <<< DECOMMENTA se vuoi una strofa senza accordi

In \[E]questa notte \[A]splendida
di \[E]luce e di chia\[B]ror
il \[E]nostro cuore \[A]trepida,
è \[E]nato il \[B]Salva\[E]tor.
Un \[A]bimbo picco\[E]lissimo
le \[A]porte ci apri\[E]rà
del \[E]cielo dell'Al\[A]tissimo
nel\[E]la sua \[B]veri\[E]tà.

\endverse



%%%%% STROFA
\transpose{3} 						% <<< TRASPOSIZIONE #TONI (0 nullo)
\beginverse		%Oppure \beginverse* se non si vuole il numero di fianco
%\memorize 		% <<< DECOMMENTA se si vuole utilizzarne la funzione
%\chordsoff		% <<< DECOMMENTA se vuoi una strofa senza accordi

Svegli^atevi dal ^sonno,
corr^ete coi pas^tor,
è ^notte di mi^racoli,
di ^grazia e ^di stu^por.
A^sciuga le tue ^lacrime,
non ^piangere per^chè
Ge^sù nostro ca^rissimo 
è ^nato an^che per ^te.

\endverse



%%%%% STROFA
\transpose{3} 						% <<< TRASPOSIZIONE #TONI (0 nullo)
\beginverse		%Oppure \beginverse* se non si vuole il numero di fianco
%\memorize 		% <<< DECOMMENTA se si vuole utilizzarne la funzione
%\chordsoff		% <<< DECOMMENTA se vuoi una strofa senza accordi

In ^questa notte ^limpida
di ^gloria e di sple^ndor,
il ^nostro cuore ^trepida
è ^nato il ^Salva^tor.
Ge^sù nostro ca^rissimo
le ^porte ci apri^rà,
il ^figlio dell'Al^tissimo
con ^noi sem^pre sa^rà.

\endverse





\endsong
\fi
%++++++++++++++++++++++++++++++++++++++++++++++++++++++++++++
%			FINE CANZONE TRASPOSTA
%++++++++++++++++++++++++++++++++++++++++++++++++++++++++++++

%-------------------------------------------------------------
%			INIZIO	CANZONE
%-------------------------------------------------------------


%titolo: 	Là sulla montagna
%autore: 	Daniele Ricci
%tonalita: 	Sol 



%%%%%% TITOLO E IMPOSTAZONI
\beginsong{Là là sulla montagna}[by={Go Tell It on the Mountain, J. W. Work jr.}] 	% <<< MODIFICA TITOLO E AUTORE
\transpose{0} 						% <<< TRASPOSIZIONE #TONI (0 nullo)
\momenti{Natale}							% <<< INSERISCI MOMENTI	
% momenti vanno separati da ; e vanno scelti tra:
% Ingresso; Atto penitenziale; Acclamazione al Vangelo; Dopo il Vangelo; Offertorio; Comunione; Ringraziamento; Fine; Santi; Pasqua; Avvento; Natale; Quaresima; Canti Mariani; Battesimo; Prima Comunione; Cresima; Matrimonio; Meditazione;
\ifchorded
	%\textnote{Tonalità originale }	% <<< EV COMMENTI (tonalità originale/migliore)
\fi


%%%%%% INTRODUZIONE
\ifchorded
\vspace*{\versesep}
\textnote{Intro: \qquad \qquad  }%(\eighthnote 116) % << MODIFICA IL TEMPO
% Metronomo: \eighthnote (ottavo) \quarternote (quarto) \halfnote (due quarti)
\vspace*{-\versesep}
\beginverse*

\nolyrics

%---- Prima riga -----------------------------
\vspace*{-\versesep}
\[F] \[B&]  \[F]	 % \[*D] per indicare le pennate, \rep{2} le ripetizioni

%---- Ogni riga successiva -------------------
%\vspace*{-\versesep}
%\[G] \[C]  \[D]	

%---- Ev Indicazioni -------------------------			
%\textnote{\textit{(Oppure tutta la strofa)} }	

\endverse
\fi


%%%%% RITORNELLO
\beginchorus
\textnote{\textbf{Rit.}}

\[F]Là, \[B&]là sulla mon\[F]ta\[D-]gna,
\[B&]sulle col\[C7]line \[F*]vai ad \[B&*]annun\[F*]ziar \[B&*]
\[F]che \[B&]il Signore è \[F]na\[D-]to, è \[B&]nato, 
\[C7]nato per \[F*]\[B&*]no\[F]i.

\endchorus

%%%%% STROFA
\beginverse		%Oppure \beginverse* se non si vuole il numero di fianco
\memorize 		% <<< DECOMMENTA se si vuole utilizzarne la funzione
%\chordsoff		& <<< DECOMMENTA se vuoi una strofa senza accordi

Pa\[F]stori \[A-]che re\[D-]state 
sui \[B&]monti a \[B&-]vigi\[F]lar
la \[F]luce \[A-]voi ve\[D-]dete, 
la \[G]stella \[7]di Ge\[C]sù.

\endverse








%%%%% STROFA
\beginverse		%Oppure \beginverse* se non si vuole il numero di fianco
%\memorize 		% <<< DECOMMENTA se si vuole utilizzarne la funzione
%\chordsoff		% <<< DECOMMENTA se vuoi una strofa senza accordi

Se il ^nostro ^canto è im^menso, 
pa^store ^non tre^mar
noi ^Ange^li can^tiamo, 
è ^nato il ^Salva^tor.

\endverse

%%%%% STROFA
\beginverse		%Oppure \beginverse* se non si vuole il numero di fianco
%\memorize 		% <<< DECOMMENTA se si vuole utilizzarne la funzione
\chordsoff		% <<< DECOMMENTA se vuoi una strofa senza accordi

In una mangiatoia,
un bimbo aspetterà
che l’uomo ancor ritrovi, 
la strada dell’amor.

\endverse

\endsong
%------------------------------------------------------------
%			FINE CANZONE
%------------------------------------------------------------



%-------------------------------------------------------------
%			INIZIO	CANZONE
%-------------------------------------------------------------

%titolo: Quando Nacque Gesù
%autore: Canto popolare
%tonalita: Re- (e La-)


%%%%%% TITOLO E IMPOSTAZONI
\beginsong{Quando nacque Gesuuuuuuuù}[by={Greensleeves — Canto popolare natalizio}] 	% <<< MODIFICA TITOLO E AUTORE
\transpose{-5} 			% <<< TRASPOSIZIONE #TONI (0 nullo)
\momenti{Natale}		% <<< INSERISCI MOMENTI



%%%%%% INTRODUZIONE
\ifchorded
\vspace*{\versesep}
\textnote{Intro: \qquad \qquad  }%(\eighthnote 116) % << MODIFICA IL TEMPO
\vspace*{-\versesep}
\beginverse*

\nolyrics

%---- Prima riga -----------------------------
\vspace*{-\versesep}
\[D-] 

%---- Ev Indicazioni -------------------------			
\textnote{\textit{(Oppure tutta la strofa)} }	
	 
\endverse
\fi



%%%%% STROFA
\beginverse		%Oppure \beginverse* se non si vuole il numero di fianco
\memorize 		% <<< DECOMMENTA se si vuole utilizzarne la funzione

Un \[D-]bimbo è \[F]nato a Be\[C]tlem\[A-]me,
un bam\[D-]bino è nato per \[A]noi!
Ri\[D-]posa qu\[F]ieto su \[C]paglia e \[A-]fien
e Ma\[D-]ria lo \[A]culla se\[D-]ren.


\endverse




%%%%% RITORNELLO

\beginchorus

\[F]Gloria, gloria, Alle\[C]lu\[A-]ia!
Un bam\[D-]bino è nato per \[A]noi.
\[F]Gloria, gloria, Alle\[C]lu\[A-]ia! 
Oggi è \[D-]nato il \[A]Cristo \[D-]Gesù!

\endchorus





%%%%%% INTERMEZZO

\beginverse*
\vspace*{1.3\versesep}
{
	\textnote{Ev. intermezzo strumentale}
	\textnote{\textit{(ripetizione della strofa)}} 
	 
}
\endverse



%%%%% STROFA
\beginverse		%Oppure \beginverse* se non si vuole il numero di fianco
%\memorize 		% <<< DECOMMENTA se si vuole utilizzarne la funzione

Nel ^cielo gli ^angeli ^can^tano:
"su cor^rete tutti a Bet^lemme,
vi ^trove^rete su ^paglia e ^fien
il Si^gnore il ^Cristo Ge^sù".


\endverse




\endsong
%------------------------------------------------------------
%			FINE CANZONE
%------------------------------------------------------------
%-------------------------------------------------------------
%			INIZIO	CANZONE
%-------------------------------------------------------------

%titolo: Senti l'angelo
%autore: Sequeri
%tonalita: Re 


%%%%%% TITOLO E IMPOSTAZONI
\beginsong{Senti l'Angelo}[by={Hark! the Herald Angels Sing - C. Wesley}] 	% <<< MODIFICA TITOLO E AUTORE
\transpose{0} 						% <<< TRASPOSIZIONE #TONI (0 nullo)
\momenti{Natale}							% <<< INSERISCI MOMENTI




%%%%%% INTRODUZIONE
\ifchorded
\beginverse*
\vspace*{-0.5\versesep}
{
	\nolyrics
	\textbf{Intro:} \qquad \qquad % (\eighthnote 116) % << MODIFICA IL TEMPO
	\vspace*{-\versesep}

	%---- Prima riga -----------------------------
	\[G]  \[E-]	\[B] \[E-]	%\textbf{x2} 
										%oppure \textoverline{} per le pennate

	%---- Ogni riga successiva -------------------
	\vspace*{-\versesep}
	\[A] \[D]  \textoverline{\[D]}	\textoverline{\[A]} \textoverline{\[D]}	%\textbf{x2}

	%---- Ev Indicazioni -------------------------			
	\vspace{-\versesep}\textit{(Come il ritornello)} 	
	 
}
\vspace*{-0.3\versesep}
\endverse
\fi



%%%%% STROFA
\beginverse		%Oppure \beginverse* se non si vuole il numero di fianco
\memorize 		% <<< DECOMMENTA se si vuole utilizzarne la funzione
%\chordsoff		& <<< DECOMMENTA se vuoi una strofa senza accordi

\[D]Senti l'angelo che \[D]can\[A]ta:
\[D]"Gloria al \[G]nato \[D]re \[A]dei \[D]re!"
\[D]Pace \[B-]vera e vero a\[E-]more
\[A]ha portato al \[A]mon\[(E)]do in\[A]ter.
\[D]Sveglia dunque \[G]le na\[D]zio\[A]ni
\[D]alla gioia \[G]del cre\[D]a\[A]to
\[G]e con gli \[E-]angeli \[B7]gri\[E-]diam:
\[A]"Cristo è \[D]nato a Be\[A7]tle\[D]hem!"

\endverse




%%%%% RITORNELLO
\beginchorus

\[G]Senti \[E-]l'ange\[E-]lo \[B7]can\[E-]tar:
\[A]"Gloria al \[D]nato \[D]Re \[A]dei \[D]Re!"

\endchorus



%%%%% STROFA
\beginverse		%Oppure \beginverse* se non si vuole il numero di fianco
%\memorize 		% <<< DECOMMENTA se si vuole utilizzarne la funzione
%\chordsoff		& <<< DECOMMENTA se vuoi una strofa senza accordi

^Cristo è l'unico Si^gno^re
^e per ^sempre ^re^gne^rà,
^desi^derio delle ^genti
^vieni presto ^in me^zzo a ^noi.
^Dio velato in ^carne u^ma^na
^hai voluto ^vera^men^te
^abi^tare ^qui con ^noi
^"Salve o ^nostro Emma^nu^el!"

\endverse




%%%%% RITORNELLO
\beginchorus

\[G]Senti \[E-]l'ange\[E-]lo \[B7]can\[E-]tar:
\[A]"Gloria al \[D]nato \[D]Re \[A]dei \[D]Re!"

\endchorus



%%%%% STROFA
\beginverse		%Oppure \beginverse* se non si vuole il numero di fianco
%\memorize 		% <<< DECOMMENTA se si vuole utilizzarne la funzione
%\chordsoff		& <<< DECOMMENTA se vuoi una strofa senza accordi

^Vero Principe di ^pa^ce
^dona al ^mondo ^li^ber^tà,
^vero s^ole di gius^tizia
^sana Tu l'u^ma^ni^tà.
^Tu sei nato ^perchè ^l'uo^mo
^non possa mai ^più mo^ri^re,
^perchè ^l'uomo ^abbia in ^Te
^una sec^onda ^nasci^ta

\endverse




%%%%% RITORNELLO
\beginchorus

\[G]Senti \[E-]l'ange\[E-]lo \[B7]can\[E-]tar:
\[A]"Gloria al \[D]nato \[D]Re \[A]dei \[D]Re!"

\endchorus





\endsong
%------------------------------------------------------------
%			FINE CANZONE
%------------------------------------------------------------
%-------------------------------------------------------------
%			INIZIO	CANZONE
%-------------------------------------------------------------


%titolo: 	Tu Scendi Dalle Stelle
%autore: 	ALfonso de' Liguori
%tonalita: 	Do (abbassata) 



%%%%%% TITOLO E IMPOSTAZONI
\beginsong{Tu scendi dalle stelle}[by={A. De'\ Liguori}]	% <<< MODIFICA TITOLO E AUTORE
\transpose{-2} 						% <<< TRASPOSIZIONE #TONI (0 nullo)
\momenti{Natale}							% <<< INSERISCI MOMENTI	
% momenti vanno separati da ; e vanno scelti tra:
% Ingresso; Atto penitenziale; Acclamazione al Vangelo; Dopo il Vangelo; Offertorio; Comunione; Ringraziamento; Fine; Santi; Pasqua; Avvento; Natale; Quaresima; Canti Mariani; Battesimo; Prima Comunione; Cresima; Matrimonio; Meditazione;
\ifchorded
	%\textnote{Tonalità originale }	% <<< EV COMMENTI (tonalità originale/migliore)
\fi


%%%%%% INTRODUZIONE
\ifchorded
\vspace*{\versesep}
\musicnote{
\begin{minipage}{0.48\textwidth}
\textbf{Intro}
\hfill 
%( \eighthnote \, 80)   % <<  MODIFICA IL TEMPO
% Metronomo: \eighthnote (ottavo) \quarternote (quarto) \halfnote (due quarti)
\end{minipage}
} 	
\vspace*{-\versesep}
\beginverse*

\nolyrics

%---- Prima riga -----------------------------
\vspace*{-\versesep}
\[D] 	 % \[*D] per indicare le pennate, \rep{2} le ripetizioni

%---- Ogni riga successiva -------------------
%\vspace*{-\versesep}
%\[G] \[C]  \[D]	

%---- Ev Indicazioni -------------------------			
%\textnote{\textit{(Oppure tutta la strofa)} }	

\endverse
\fi



%%%%% STROFA
\beginverse
\memorize
Tu \[D]scendi dalle stelle, o Re del cie\[A]lo
e vieni in una \[A]grot\[G]ta 
al \[D]freddo e al \[A]ge\[D]lo,
e \[A]vieni in una \[A]grot\[G]ta 
al \[D]freddo e al \[A]ge\[D]lo.
O Bam\[A]bino, mio Di\[D]vino,
io Ti \[A]vedo qui a tre\[D]mar. O Dio be\[A]ato!
Ah quanto Ti cos\[A]tò \[G]l'a\[D]vermi a\[A]ma\[D]to!
Ah \[A]quanto Ti cos\[A]tò \[G]l'a\[D]vermi a\[A]ma\[D]to!
\endverse



%%%%% STROFA
\beginverse
%\chordsoff
A ^Te che sei del mondo il Creato^re,
mancano panni e ^fuo^co 
o ^mio Si^gno^re,
manc^ano panni e ^fuo^co 
o ^mio Si^gno^re.
Caro e^letto Pargo^letto,
quanto ^questa pover^tà 
più m'inna^mora.
Giacchè Ti fece am^or ^po^vero an^co^ra!
Giac^chè Ti fece am^or ^po^vero an^co^ra!
\endverse



%%%%% STROFA
\beginverse
%\chordsoff
Tu ^lasci il bel gioire del divin se^no,
per giungere a tre^ma^re su ^questo ^fie^no;
per ^giungere a tre^ma^re su ^questo ^fie^no.
Dolce a^more del mio ^cuore, 
dove a^mor ti traspo^rtò! O Gesù ^mio, 
perchè tanto pat^ir, ^^per amor ^mi^o.
Per^chè tanto pat^ir, ^^per amor ^mi^o.
\endverse


\endsong
%------------------------------------------------------------
%			FINE CANZONE
%------------------------------------------------------------
%-------------------------------------------------------------
%			INIZIO	CANZONE
%-------------------------------------------------------------


%titolo: 	Venite Fedeli
%autore: 	Stefani, Wade
%tonalita: 	Fa 



%%%%%% TITOLO E IMPOSTAZONI
\beginsong{Venite fedeli}[by={sir J. F. Wade}]	% <<< MODIFICA TITOLO E AUTORE
\transpose{0} 						% <<< TRASPOSIZIONE #TONI (0 nullo)
\momenti{Natale}							% <<< INSERISCI MOMENTI	
% momenti vanno separati da ; e vanno scelti tra:
% Ingresso; Atto penitenziale; Acclamazione al Vangelo; Dopo il Vangelo; Offertorio; Comunione; Ringraziamento; Fine; Santi; Pasqua; Avvento; Natale; Quaresima; Canti Mariani; Battesimo; Prima Comunione; Cresima; Matrimonio; Meditazione;
\ifchorded
	%\textnote{Tonalità originale }	% <<< EV COMMENTI (tonalità originale/migliore)
\fi


%%%%%% INTRODUZIONE
\ifchorded
\vspace*{\versesep}
\musicnote{
\begin{minipage}{0.48\textwidth}
\textbf{Intro}
\hfill 
%( \eighthnote \, 80)   % <<  MODIFICA IL TEMPO
% Metronomo: \eighthnote (ottavo) \quarternote (quarto) \halfnote (due quarti)
\end{minipage}
} 	
\vspace*{-\versesep}
\beginverse*

\nolyrics

%---- Prima riga -----------------------------
\vspace*{-\versesep}
\[(F*)] \[B&] \[G-] \[C]  \[F*]  \[B&*] \[F] \[C] \[F]	 % \[*D] per indicare le pennate, \rep{2} le ripetizioni

%---- Ogni riga successiva -------------------
%\vspace*{-\versesep}
%\[G] \[C]  \[D]	

%---- Ev Indicazioni -------------------------			
\textnote{\textit{(come l'ultima riga del ritornello)} }	

\endverse
\fi



%%%%% STROFA
\beginverse
\memorize
Ve\[F]nite, fe\[C]deli, 
\[F*]l'an\[C*]ge\[F*]lo \[B&*]ci in\[F]vi\[C]ta, 
v\[D-]eni\[C*]te, \[G*]v\[C*]eni-\[(D-*)]i-\[C*]te 
\[F*]a Be\[C]tle-\[G7]em-\[C]me.
\endverse




%%%%% RITORNELLO
\beginchorus
\textnote{\textbf{Rit.}}
\[F]Na\[B&*]sce \[F*]per \[B&]no\[F]i 
\[C*]Cri\[A-*]sto \[D-*]Sal\[(G*-)]va\[C*]to-\[G*]o-\[C]re.
Ve\[F]nite, ado\[F]ria\[C]mo, \brk\[F]venite, \[F*]a\[B&*]dori\[F]a\[C]mo, 
v\[B&]enite, a\[(G-)]do\[C]ria\[D-*]mo \brk \[B&*]il Si\[F]gno\[C7]re Ge\[F]sù!
\endchorus




%%%%% STROFA
\beginverse
La ^luce del ^mo-o-ndo 
^bril^la in ^u^na ^grot^ta: 
la ^fe^de ^ci ^gui-^i-^da 
^a Be^tle-^em-^me.
\endverse


\beginverse
\chordsoff
La ^notte ri^splende, ^tut^to il ^mon^do at^ten^de; 
se^guia^mo i ^pa^^sto^ri ^a Be^^tle^me.
\endverse


\beginverse
\chordsoff
Il ^Figlio di ^Dio, ^Re ^dell'^u^ni^ver^so, 
si é ^fat^to ^bam^^bi^no ^a Be^^tlem^me.
\endverse


\beginverse
\chordsoff
«Sia ^gloria nei ^cieli, ^pa^ce ^sul^la ^ter^ra» 
un ^an^ge^lo an^^nun^cia ^a Be^^tlem^me.
\endverse



\endsong
%------------------------------------------------------------
%			FINE CANZONE
%------------------------------------------------------------


% %++++++++++++++++++++++++++++++++++++++++++++++++++++++++++++
% %			CANZONE TRASPOSTA
% %++++++++++++++++++++++++++++++++++++++++++++++++++++++++++++
% \ifchorded
% %decremento contatore per avere stesso numero
% \addtocounter{songnum}{-1} 
% \beginsong{Venite fedeli}[by={Stefani, Wade}]	% <<< COPIA TITOLO E AUTORE
% \transpose{-3} 						% <<< TRASPOSIZIONE #TONI + - (0 nullo)
% \ifchorded
% 	\textnote{Tonalità più facile da suonare}	% <<< EV COMMENTI (tonalità originale/migliore)
% \fi


% %%%%%% INTRODUZIONE
% \ifchorded
% \vspace*{\versesep}
% \textnote{Intro: \qquad \qquad  }%(\eighthnote 116) % << MODIFICA IL TEMPO
% % Metronomo: \eighthnote (ottavo) \quarternote (quarto) \halfnote (due quarti)
% \vspace*{-\versesep}
% \beginverse*

% \nolyrics

% %---- Prima riga -----------------------------
% \vspace*{-\versesep}
% \[(F*)] \[B&] \[G-] \[C]  \[F*]  \[B&*] \[F] \[C] \[F]	 % \[*D] per indicare le pennate, \rep{2} le ripetizioni

% %---- Ogni riga successiva -------------------
% %\vspace*{-\versesep}
% %\[G] \[C]  \[D]	

% %---- Ev Indicazioni -------------------------			
% \textnote{\textit{(come la seconda riga del ritornello)} }	

% \endverse
% \fi



% %%%%% STROFA
% \beginverse
% \memorize
% Ve\[F]nite, fe\[C]deli, 
% \[F*]l'an\[C*]ge\[F*]lo \[B&*]ci in\[F]vi\[C]ta, 
% v\[D-]eni\[C*]te, \[G*]v\[C*]eni-\[(D-*)]i-\[C*]te 
% \[F*]a Be\[C]tle-\[G7]em-\[C]me.
% \endverse




% %%%%% RITORNELLO
% \beginchorus
% \textnote{\textbf{Rit.}}
% \[F]Na\[B&*]sce \[F*]per \[B&]no\[F]i 
% \[C*]Cri\[A-*]sto \[D-*]Sal\[(G*-)]va\[C*]to-\[G*]o-\[C]re.
% Ve\[F]nite, ado\[F]ria\[C]mo, \brk\[F]venite, \[F*]a\[B&*]dori\[F]a\[C]mo, 
% v\[B&]enite, a\[(G-)]do\[C]ria\[D-*]mo \brk \[B&*]il Si\[F]gno\[C7]re Ge\[F]sù!
% \endchorus




% %%%%% STROFA
% \beginverse
% La ^luce del ^mo-o-ndo 
% ^bril^la in ^u^na ^grot^ta: 
% la ^fe^de ^ci ^gui-^i-^da 
% ^a Be^tle-^em-^me.
% \endverse


% \beginverse
% \chordsoff
% La ^notte ri^splende, ^tut^to il ^mon^do at^ten^de; 
% se^guia^mo i ^pa^^sto^ri ^a Be^^tle^me.
% \endverse


% \beginverse
% \chordsoff
% Il ^Figlio di ^Dio, ^Re ^dell'^u^ni^ver^so, 
% si é ^fat^to ^bam^^bi^no ^a Be^^tlem^me.
% \endverse


% \beginverse
% \chordsoff
% «Sia ^gloria nei ^cieli, ^pa^ce ^sul^la ^ter^ra» 
% un ^an^ge^lo an^^nun^cia ^a Be^^tlem^me.
% \endverse



% \endsong


% \fi
% %++++++++++++++++++++++++++++++++++++++++++++++++++++++++++++
% %			FINE CANZONE TRASPOSTA
% %++++++++++++++++++++++++++++++++++++++++++++++++++++++++++++






%******* END SONGS ENVIRONMENT ******
\setcounter{GlobalSongCounter}{\thesongnum}
\end{songs}


\songchapter{Santo}
%...............................................................................
%     _______.     ___      .__   __. .___________.  ______   
%    /       |    /   \     |  \ |  | |           | /  __  \  
%   |   (----`   /  ^  \    |   \|  | `---|  |----`|  |  |  | 
%    \   \      /  /_\  \   |  . `  |     |  |     |  |  |  | 
%.----)   |    /  _____  \  |  |\   |     |  |     |  `--'  | 
%|_______/    /__/     \__\ |__| \__|     |__|      \______/  
%                                                           
%...............................................................................
\begin{songs}{}
\songcolumns{\canzsongcolumsnumber}
\setcounter{songnum}{\theGlobalSongCounter} %set songnum counter, otherwise would be reset

%set the default path inside current folder
\makeatletter
\def\input@path{{Songs/Santo/}}
\makeatother


%***** INSERT SONGS HERE ******


%-------------------------------------------------------------
%			INIZIO	CANZONE
%-------------------------------------------------------------


%titolo: 	Santo Milan
%autore: 	Gen Verde
%tonalita: 	Sol 



%%%%%% TITOLO E IMPOSTAZONI
\beginsong{Santo Milan}[by={Gen Verde}] 	% <<< MODIFICA TITOLO E AUTORE
\transpose{0} 						% <<< TRASPOSIZIONE #TONI (0 nullo)
\momenti{}							% <<< INSERISCI MOMENTI	
% momenti vanno separati da ; e vanno scelti tra:
% Ingresso; Atto penitenziale; Acclamazione al Vangelo; Dopo il Vangelo; Offertorio; Comunione; Ringraziamento; Fine; Santi; Pasqua; Avvento; Natale; Quaresima; Canti Mariani; Battesimo; Prima Comunione; Cresima; Matrimonio; Meditazione;
\ifchorded
	\textnote{Tonalità originale }	% <<< EV COMMENTI (tonalità originale/migliore)
\fi


%%%%%% INTRODUZIONE
\ifchorded
\vspace*{\versesep}
\textnote{Intro: \qquad \qquad  }%(\eighthnote 116) % << MODIFICA IL TEMPO
% Metronomo: \eighthnote (ottavo) \quarternote (quarto) \halfnote (due quarti)
\vspace*{-\versesep}
\beginverse*

\nolyrics

%---- Prima riga -----------------------------
\vspace*{-\versesep}
\[A] \[E]  \[C#-] \[B]	 % \[*D] per indicare le pennate, \rep{2} le ripetizioni

%---- Ogni riga successiva -------------------
\vspace*{-\versesep}
\[F#-] \[E]  \[A]  \[B]	

%---- Ev Indicazioni -------------------------			
\textnote{\textit{(come le prime due righe)} }	

\endverse
\fi








%%%%% RITORNELLO
\beginchorus

\[A]San\[E]to, \[C#-]San\[B]to,
\[F#-]Santo il Si\[E]gnore, \[A]Dio dell'uni\[B]verso.
\[A]San\[E]to, \[C#-]San\[B]to.
I \[F#-]cieli e la \[E]terra 
sono \[A]pieni della tua \[(F#-)]glo\[E]ria.

\endchorus



%%%%% STROFA
\beginverse*		%Oppure \beginverse* se non si vuole il numero di fianco
%\memorize 		% <<< DECOMMENTA se si vuole utilizzarne la funzione
%\chordsoff		% <<< DECOMMENTA se vuoi una strofa senza accordi

O\[A]sanna nel\[B]l'alto dei \[A]cie\[B]li.
O\[F#-]sanna nell'alto dei \[A]cieli.

\endverse



%%%%% RITORNELLO
\beginchorus

\[A]San\[E]to, \[C#-]San\[B]to,
\[F#-]Santo il Si\[E]gnore, \[A]Dio dell'uni\[B]verso.
\[A]San\[E]to, \[C#-]San\[B]to.
I \[F#-]cieli e la \[E]terra 
sono \[A]pieni della tua \[(F#-)]glo\[E]ria.

\endchorus




%%%%% STROFA
\beginverse*		%Oppure \beginverse* se non si vuole il numero di fianco
%\memorize 		% <<< DECOMMENTA se si vuole utilizzarne la funzione
%\chordsoff		& <<< DECOMMENTA se vuoi una strofa senza accordi

\[B]Benedetto co\[A]lui che viene
nel \[E]nome del Sig\[B]nore.
O\[A]sanna nel\[B]l'alto dei \[A]cie\[B]li.
O\[F#-]sanna nell'alto dei \[A]cieli.

\endverse


%%%%% RITORNELLO
\beginchorus

\[A]San\[E]to, \[C#-]San\[B]to,
\[F#-]Sa-\[A]a-n\[E]to.  \[*E] 

\endchorus











\endsong
%------------------------------------------------------------
%			FINE CANZONE
%------------------------------------------------------------




% %++++++++++++++++++++++++++++++++++++++++++++++++++++++++++++
% %			CANZONE TRASPOSTA
% %++++++++++++++++++++++++++++++++++++++++++++++++++++++++++++
% \ifchorded
% %decremento contatore per avere stesso numero
% \addtocounter{songnum}{-1} 
% \beginsong{Santo Milan}[by={Gen Verde}] 	% <<< COPIA TITOLO E AUTORE
% \transpose{-2} 						% <<< TRASPOSIZIONE #TONI + - (0 nullo)
% \ifchorded
% 	\textnote{Tonalità più facile per le chitarre}	% <<< EV COMMENTI (tonalità originale/migliore)
% \fi


% %%%%%% INTRODUZIONE
% \ifchorded
% \vspace*{\versesep}
% \textnote{Intro: \qquad \qquad  }%(\eighthnote 116) % << MODIFICA IL TEMPO
% % Metronomo: \eighthnote (ottavo) \quarternote (quarto) \halfnote (due quarti)
% \vspace*{-\versesep}
% \beginverse*

% \nolyrics

% %---- Prima riga -----------------------------
% \vspace*{-\versesep}
% \[A] \[E]  \[C#-] \[B]	 % \[*D] per indicare le pennate, \rep{2} le ripetizioni

% %---- Ogni riga successiva -------------------
% \vspace*{-\versesep}
% \[F#-] \[E]  \[A]  \[B]	

% %---- Ev Indicazioni -------------------------			
% \textnote{\textit{(come le prime due righe)} }	

% \endverse
% \fi








% %%%%% RITORNELLO
% \beginchorus

% \[A]San\[E]to, \[C#-]San\[B]to,
% \[F#-]Santo il Si\[E]gnore, \[A]Dio dell'uni\[B]verso.
% \[A]San\[E]to, \[C#-]San\[B]to.
% I \[F#-]cieli e la \[E]terra 
% sono \[A]pieni della tua \[(F#-)]glo\[E]ria.

% \endchorus



% %%%%% STROFA
% \beginverse*		%Oppure \beginverse* se non si vuole il numero di fianco
% %\memorize 		% <<< DECOMMENTA se si vuole utilizzarne la funzione
% %\chordsoff		% <<< DECOMMENTA se vuoi una strofa senza accordi

% O\[A]sanna nel\[B]l'alto dei \[A]cie\[B]li.
% O\[F#-]sanna nell'alto dei \[A]cieli.

% \endverse



% %%%%% RITORNELLO
% \beginchorus

% \[A]San\[E]to, \[C#-]San\[B]to,
% \[F#-]Santo il Si\[E]gnore, \[A]Dio dell'uni\[B]verso.
% \[A]San\[E]to, \[C#-]San\[B]to.
% I \[F#-]cieli e la \[E]terra 
% sono \[A]pieni della tua \[(F#-)]glo\[E]ria.

% \endchorus




% %%%%% STROFA
% \beginverse*		%Oppure \beginverse* se non si vuole il numero di fianco
% %\memorize 		% <<< DECOMMENTA se si vuole utilizzarne la funzione
% %\chordsoff		& <<< DECOMMENTA se vuoi una strofa senza accordi

% \[B]Benedetto co\[A]lui che viene
% nel \[E]nome del Sig\[B]nore.
% O\[A]sanna nel\[B]l'alto dei \[A]cie\[B]li.
% O\[F#-]sanna nell'alto dei \[A]cieli.

% \endverse


% %%%%% RITORNELLO
% \beginchorus

% \[A]San\[E]to, \[C#-]San\[B]to,
% \[F#-]Sa-\[A]a-n\[E]to.  \[*E] 

% \endchorus






% \endsong

% \fi
% %++++++++++++++++++++++++++++++++++++++++++++++++++++++++++++
% %			FINE CANZONE TRASPOSTA
% %++++++++++++++++++++++++++++++++++++++++++++++++++++++++++++





%******* END SONGS ENVIRONMENT ******

\end{songs}








% INDEXES AND CLOSURE
%-------------------------------------------------------------------------------

\ifcanzsingole
	\relax
\else
	\iftitleindex
		\ifxetex
		\printindex[alfabetico]
		\else
		\printindex
		\fi
	\else
	\fi
	\ifauthorsindex
	\printindex[autori]
	\else
	\fi
	\iftematicindex
	\printindex[tematico]
	\else
	\fi	
\fi

\cleartoleftpage

\colophon



\end{document}
